\begin{center}
\section*{بخش ۵۷ - حمله بردن سگ بر کور گدا}
\label{sec:sh057}
\addcontentsline{toc}{section}{\nameref{sec:sh057}}
\begin{longtable}{l p{0.5cm} r}
یک سگی در کوی بر کور گدا
&&
حمله می‌آورد چون شیر وغا
\\
سگ کند آهنگ درویشان بخشم
&&
در کشد مه خاک درویشان بچشم
\\
کور عاجز شد ز بانگ و بیم سگ
&&
اندر آمد کور در تعظیم سگ
\\
کای امیر صید و ای شیر شکار
&&
دست دست تست دست از من بدار
\\
کز ضرورت دم خر را آن حکیم
&&
کرد تعظیم و لقب دادش کریم
\\
گفت او هم از ضرورت کای اسد
&&
از چو من لاغر شکارت چه رسد
\\
گور می‌گیرند یارانت به دشت
&&
کور می‌گیری تو در کوچه بگشت
\\
گور می‌جویند یارانت بصید
&&
کور می‌جویی تو در کوچه بکید
\\
آن سگ عالم شکار گور کرد
&&
وین سگ بی‌مایه قصد کور کرد
\\
علم چون آموخت سگ رست از ضلال
&&
می‌کند در بیشه‌ها صید حلال
\\
سگ چو عالم گشت شد چالاک زحف
&&
سگ چو عارف گشت شد اصحاب کهف
\\
سگ شناسا شد که میر صید کیست
&&
ای خدا آن نور اشناسنده چیست
\\
کور نشناسد نه از بی چشمی است
&&
بلک این زانست کز جهلست مست
\\
نیست خود بی‌چشم‌تر کور از زمین
&&
این زمین از فضل حق شد خصم بین
\\
نور موسی دید و موسی را نواخت
&&
خسف قارون کرد و قارون را شناخت
\\
رجف کرد اندر هلاک هر دعی
&&
فهم کرد از حق که یاارض ابلعی
\\
خاک و آب و باد و نار با شرر
&&
بی‌خبر با ما و با حق با خبر
\\
ما بعکس آن ز غیر حق خبیر
&&
بی‌خبر از حق و از چندین نذیر
\\
لاجرم اشفقن منها جمله‌شان
&&
کند شد ز آمیز حیوان حمله‌شان
\\
گفت بیزاریم جمله زین حیات
&&
کو بود با خلق حی با حق موات
\\
چون بماند از خلق گردد او یتیم
&&
انس حق را قلب می‌باید سلیم
\\
چون ز کوری دزد دزدد کاله‌ای
&&
می‌کند آن کور عمیا ناله‌ای
\\
تا نگوید دزد او را کان منم
&&
کز تو دزدیدم که دزد پر فنم
\\
کی شناسد کور دزد خویش را
&&
چون ندارد نور چشم و آن ضیا
\\
چون بگوید هم بگیر او را تو سخت
&&
تا بگوید او علامتهای رخت
\\
پس جهاد اکبر آمد عصر دزد
&&
تا بگوید که چه برد آن زن بمزد
\\
اولا دزدید کحل دیده‌ات
&&
چون ستانی باز یابی تبصرت
\\
کالهٔ حکمت که گم کردهٔ دلست
&&
پیش اهل دل یقین آن حاصلست
\\
کوردل با جان و با سمع و بصر
&&
می‌نداند دزد شیطان را ز اثر
\\
ز اهل دل جو از جماد آن را مجو
&&
که جماد آمد خلایق پیش او
\\
مشورت جوینده آمد نزد او
&&
کای اب کودک شده رازی بگو
\\
گفت رو زین حلقه کین در باز نیست
&&
باز گرد امروز روز راز نیست
\\
گر مکان را ره بدی در لامکان
&&
همچو شیخان بودمی من بر دکان
\\
\end{longtable}
\end{center}
