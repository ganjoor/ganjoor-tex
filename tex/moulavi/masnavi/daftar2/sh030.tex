\begin{center}
\section*{بخش ۳۰ - امتحان کردن خواجهٔ لقمان زیرکی لقمان را}
\label{sec:sh030}
\addcontentsline{toc}{section}{\nameref{sec:sh030}}
\begin{longtable}{l p{0.5cm} r}
نی که لقمان را که بندهٔ پاک بود
&&
روز و شب در بندگی چالاک بود
\\
خواجه‌اش می‌داشتی در کار پیش
&&
بهترش دیدی ز فرزندان خویش
\\
زانک لقمان گرچه بنده‌زاد بود
&&
خواجه بود و از هوا آزاد بود
\\
گفت شاهی شیخ را اندر سخن
&&
چیزی از بخشش ز من درخواست کن
\\
گفت ای شه شرم ناید مر ترا
&&
که چنین گویی مرا زین برتر آ
\\
من دو بنده دارم و ایشان حقیر
&&
وآن دو بر تو حاکمانند و امیر
\\
گفت شه آن دو چه‌اند این زلتست
&&
گفت آن یک خشم و دیگر شهوتست
\\
شاه آن دان کو ز شاهی فارغست
&&
بی مه و خورشید نورش بازغست
\\
مخزن آن دارد که مخزن ذات اوست
&&
هستی او دارد که با هستی عدوست
\\
خواجهٔ لقمان بظاهر خواجه‌وش
&&
در حقیقت بنده لقمان خواجه‌اش
\\
در جهان بازگونه زین بسیست
&&
در نظرشان گوهری کم از خسیست
\\
مر بیابان را مفازه نام شد
&&
نام و رنگی عقلشان را دام شد
\\
یک گره را خود معرف جامه است
&&
در قبا گویند کو از عامه است
\\
یک گره را ظاهر سالوس زهد
&&
نور باید تا بود جاسوس زهد
\\
نور باید پاک از تقلید و غول
&&
تا شناسد مرد را بی فعل و قول
\\
در رود در قلب او از راه عقل
&&
نقد او بیند نباشد بند نقل
\\
بندگان خاص علام الغیوب
&&
در جهان جان جواسیس القلوب
\\
در درون دل در آید چون خیال
&&
پیش او مکشوف باشد سر حال
\\
در تن گنجشک چیست از برگ و ساز
&&
که شود پوشیده آن بر عقل باز
\\
آنک واقف گشت بر اسرار هو
&&
سر مخلوقات چه بود پیش او
\\
آنک بر افلاک رفتارش بود
&&
بر زمین رفتن چه دشوارش بود
\\
در کف داود کاهن گشت موم
&&
موم چه بود در کف او ای ظلوم
\\
بود لقمان بنده‌شکلی خواجه‌ای
&&
بندگی بر ظاهرش دیباجه‌ای
\\
چون رود خواجه به جای ناشناس
&&
در غلام خویش پوشاند لباس
\\
او بپوشد جامه‌های آن غلام
&&
مر غلام خویش را سازد امام
\\
در پیش چون بندگان در ره شود
&&
تا نباید زو کسی آگه شود
\\
گوید ای بنده تو رو بر صدر شین
&&
من بگیرم کفش چون بندهٔ کهین
\\
تو درشتی کن مرا دشنام ده
&&
مر مرا تو هیچ توقیری منه
\\
ترک خدمت خدمت تو داشتم
&&
تا به غربت تخم حیلت کاشتم
\\
خواجگان این بندگیها کرده‌اند
&&
تا گمان آید که ایشان بنده‌اند
\\
چشم‌پر بودند و سیر از خواجگی
&&
کارها را کرده‌اند آمادگی
\\
وین غلامان هوا بر عکس آن
&&
خویشتن بنموده خواجهٔ عقل و جان
\\
آید از خواجه ره افکندگی
&&
ناید از بنده به غیر بندگی
\\
پس از آن عالم بدین عالم چنان
&&
تعبیتها هست بر عکس این بدان
\\
خواجهٔ لقمان ازین حال نهان
&&
بود واقف دیده بود از وی نشان
\\
راز می‌دانست و خوش می‌راند خر
&&
از برای مصلحت آن راه‌بر
\\
مر ورا آزاد کردی از نخست
&&
لیک خشنودی لقمان را بجست
\\
زانک لقمان را مراد این بود تا
&&
کس نداند سر آن شیر و فتی
\\
چه عجب گر سر ز بد پنهان کنی
&&
این عجب که سر ز خود پنهان کنی
\\
کار پنهان کن تو از چشمان خود
&&
تا بود کارت سلیم از چشم بد
\\
خویش را تسلیم کن بر دام مزد
&&
وانگه از خود بی ز خود چیزی بدزد
\\
می‌دهند افیون به مرد زخم‌مند
&&
تا که پیکان از تنش بیرون کنند
\\
وقت مرگ از رنج او را می‌درند
&&
او بدان مشغول شد جان می‌برند
\\
چون به هر فکری که دل خواهی سپرد
&&
از تو چیزی در نهان خواهند برد
\\
پس بدان مشغول شو کان بهترست
&&
تا ز تو چیزی برد کان کهترست
\\
هرچه تحصیلی کنی ای معتنی
&&
می در آید دزد از آن سو کایمنی
\\
بار بازرگان چو در آب اوفتد
&&
دست اندر کالهٔ بهتر زند
\\
چونک چیزی فوت خواهد شد در آب
&&
ترک کمتر گوی و بهتر را بیاب
\\
\end{longtable}
\end{center}
