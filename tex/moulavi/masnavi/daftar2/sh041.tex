\begin{center}
\section*{بخش ۴۱ - گفتن نابینای سایل کی دو کوری دارم}
\label{sec:sh041}
\addcontentsline{toc}{section}{\nameref{sec:sh041}}
\begin{longtable}{l p{0.5cm} r}
بود کوری کو همی‌گفت الامان
&&
من دو کوری دارم ای اهل زمان
\\
پس دوباره رحمتم آرید هان
&&
چون دو کوری دارم و من در میان
\\
گفت یک کوریت می‌بینیم ما
&&
آن دگر کوری چه باشد وا نما
\\
گفت زشت‌آوازم و ناخوش نوا
&&
زشت‌آوازی و کوری شد دوتا
\\
بانگ زشتم مایهٔ غم می‌شود
&&
مهر خلق از بانگ من کم می‌شود
\\
زشت آوازم بهر جا که رود
&&
مایهٔ خشم و غم و کین می‌شود
\\
بر دو کوری رحم را دوتا کنید
&&
این چنین ناگنج را گنجا کنید
\\
زشتی آواز کم شد زین گله
&&
خلق شد بر وی برحمت یک‌دله
\\
کرد نیکو چون بگفت او راز را
&&
لطف آواز دلش آواز را
\\
وانک آواز دلش هم بد بود
&&
آن سه کوری دوری سرمد بود
\\
لیک وهابان که بی علت دهند
&&
بوک دستی بر سر زشتش نهند
\\
چونک آوازش خوش و مظلوم شد
&&
زو دل سنگین‌دلان چون موم شد
\\
نالهٔ کافر چو زشتست و شهیق
&&
زان نمی‌گردد اجابت را رفیق
\\
اخسؤا بر زشت آواز آمدست
&&
کو ز خون خلق چون سگ بود مست
\\
چونک نالهٔ خرس رحمت‌کش بود
&&
ناله‌ات نبود چنین ناخوش بود
\\
دان که با یوسف تو گرگی کرده‌ای
&&
یا ز خون بی گناهی خورده‌ای
\\
توبه کن وز خورده استفراغ کن
&&
ور جراحت کهنه شد رو داغ کن
\\
\end{longtable}
\end{center}
