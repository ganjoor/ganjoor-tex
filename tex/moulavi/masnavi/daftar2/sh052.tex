\begin{center}
\section*{بخش ۵۲ - گفتن شیخ ابویزید را کی کعبه منم  گرد من طوافی می‌کن}
\label{sec:sh052}
\addcontentsline{toc}{section}{\nameref{sec:sh052}}
\begin{longtable}{l p{0.5cm} r}
سوی مکه شیخ امت بایزید
&&
از برای حج و عمره می‌دوید
\\
او به هر شهری که رفتی از نخست
&&
مر عزیزان را بکردی بازجست
\\
گرد می‌گشتی که اندر شهر کیست
&&
کو بر ارکان بصیرت متکیست
\\
گفت حق اندر سفر هر جا روی
&&
باید اول طالب مردی شوی
\\
قصد گنجی کن که این سود و زیان
&&
در تبع آید تو آن را فرع دان
\\
هر که کارد قصد گندم باشدش
&&
کاه خود اندر تبع می‌آیدش
\\
که بکاری بر نیاید گندمی
&&
مردمی جو مردمی جو مردمی
\\
قصد کعبه کن چو وقت حج بود
&&
چونک رفتی مکه هم دیده شود
\\
قصد در معراج دید دوست بود
&&
درتبع عرش و ملایک هم نمود
\\
\end{longtable}
\end{center}
