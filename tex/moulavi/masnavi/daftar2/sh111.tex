\begin{center}
\section*{بخش ۱۱۱ - شرح کردن شیخ سر آن درخت با آن طالب مقلد}
\label{sec:sh111}
\addcontentsline{toc}{section}{\nameref{sec:sh111}}
\begin{longtable}{l p{0.5cm} r}
بود شیخی عالمی قطبی کریم
&&
اندر آن منزل که آیس شد ندیم
\\
گفت من نومید پیش او روم
&&
ز آستان او براه اندر شوم
\\
تا دعای او بود همراه من
&&
چونک نومیدم من از دلخواه من
\\
رفت پیش شیخ با چشم پر آب
&&
اشک می‌بارید مانند سحاب
\\
گفت شیخا وقت رحم و رقتست
&&
ناامیدم وقت لطف این ساعتست
\\
گفت واگو کز چه نومیدیستت
&&
چیست مطلوب تو رو با چیستت
\\
گفت شاهنشاه کردم اختیار
&&
از برای جستن یک شاخسار
\\
که درختی هست نادر در جهات
&&
میوهٔ او مایهٔ آب حیات
\\
سالها جستم ندیدم یک نشان
&&
جز که طنز و تسخر این سرخوشان
\\
شیخ خندید و بگفتش ای سلیم
&&
این درخت علم باشد در علیم
\\
بس بلند و بس شگرف و بس بسیط
&&
آب حیوانی ز دریای محیط
\\
تو بصورت رفته‌ای ای بی‌خبر
&&
زان ز شاخ معنیی بی بار و بر
\\
گه درختش نام شد گه آفتاب
&&
گاه بحرش نام گشت و گه سحاب
\\
آن یکی کش صد هزار آثار خاست
&&
کمترین آثار او عمر بقاست
\\
گرچه فردست او اثر دارد هزار
&&
آن یکی را نام شاید بی‌شمار
\\
آن یکی شخصی ترا باشد پدر
&&
در حق شخصی دگر باشد پسر
\\
در حق دیگر بود قهر و عدو
&&
در حق دیگر بود لطف و نکو
\\
صد هزاران نام و او یک آدمی
&&
صاحب هر وصفش از وصفی عمی
\\
هر که جوید نام گر صاحب ثقه‌ست
&&
همچو تو نومید و اندر تفرقه‌ست
\\
تو چه بر چفسی برین نام درخت
&&
تا بمانی تلخ‌کام و شوربخت
\\
در گذر از نام و بنگر در صفات
&&
تا صفاتت ره نماید سوی ذات
\\
اختلاف خلق از نام اوفتاد
&&
چون بمعنی رفت آرام اوفتاد
\\
\end{longtable}
\end{center}
