\begin{center}
\section*{بخش ۳۹ - رنجانیدن امیری خفته‌ای را کی مار در دهانش رفته بود}
\label{sec:sh039}
\addcontentsline{toc}{section}{\nameref{sec:sh039}}
\begin{longtable}{l p{0.5cm} r}
عاقلی بر اسپ می‌آمد سوار
&&
در دهان خفته‌ای می‌رفت مار
\\
آن سوار آن را بدید و می‌شتافت
&&
تا رماند مار را فرصت نیافت
\\
چونک از عقلش فراوان بد مدد
&&
چند دبوسی قوی بر خفته زد
\\
برد او را زخم آن دبوس سخت
&&
زو گریزان تا بزیر یک درخت
\\
سیب پوسیده بسی بد ریخته
&&
گفت ازین خور ای بدرد آویخته
\\
سیب چندان مر ورا در خورد داد
&&
کز دهانش باز بیرون می‌فتاد
\\
بانگ می‌زد کای امیر آخر چرا
&&
قصد من کردی تو نادیده جفا
\\
گر تر از اصلست با جانم ستیز
&&
تیغ زن یکبارگی خونم بریز
\\
شوم ساعت که شدم بر تو پدید
&&
ای خنک آن را که روی تو ندید
\\
بی جنایت بی گنه بی بیش و کم
&&
ملحدان جایز ندارند این ستم
\\
می‌جهد خون از دهانم با سخن
&&
ای خدا آخر مکافاتش تو کن
\\
هر زمان می‌گفت او نفرین نو
&&
اوش می‌زد کاندرین صحرا بدو
\\
زخم دبوس و سوار همچو باد
&&
می‌دوید و باز در رو می‌فتاد
\\
ممتلی و خوابناک و سست بد
&&
پا و رویش صد هزاران زخم شد
\\
تا شبانگه می‌کشید و می‌گشاد
&&
تا ز صفرا قی شدن بر وی فتاد
\\
زو بر آمد خورده‌ها زشت و نکو
&&
مار با آن خورده بیرون جست ازو
\\
چون بدید از خود برون آن مار را
&&
سجده آورد آن نکوکردار را
\\
سهم آن مار سیاه زشت زفت
&&
چون بدید آن دردها از وی برفت
\\
گفت خود تو جبرئیل رحمتی
&&
یا خدایی که ولی نعمتی
\\
ای مبارک ساعتی که دیدیم
&&
مرده بودم جان نو بخشیدیم
\\
تو مرا جویان مثال مادران
&&
من گریزان از تو مانند خران
\\
خر گریزد از خداوند از خری
&&
صاحبش در پی ز نیکو گوهری
\\
نه از پی سود و زیان می‌جویدش
&&
لیک تا گرگش ندرد یا ددش
\\
ای خنک آن را که بیند روی تو
&&
یا در افتد ناگهان در کوی تو
\\
ای روان پاک بستوده ترا
&&
چند گفتم ژاژ و بیهوده ترا
\\
ای خداوند و شهنشاه و امیر
&&
من نگفتم جهل من گفت آن مگیر
\\
شمه‌ای زین حال اگر دانستمی
&&
گفتن بیهوده کی توانستمی
\\
بس ثنایت گفتمی ای خوش خصال
&&
گر مرا یک رمز می‌گفتی ز حال
\\
لیک خامش کرده می‌آشوفتی
&&
خامشانه بر سرم می‌کوفتی
\\
شد سرم کالیوه عقل از سر بجست
&&
خاصه این سر را که مغزش کمترست
\\
عفو کن ای خوب‌روی خوب‌کار
&&
آنچ گفتم از جنون اندر گذار
\\
گفت اگر من گفتمی رمزی از آن
&&
زهرهٔ تو آب گشتی آن زمان
\\
گر ترا من گفتمی اوصاف مار
&&
ترس از جانت بر آوردی دمار
\\
مصطفی فرمود اگر گویم براست
&&
شرح آن دشمن که در جان شماست
\\
زهره‌های پردلان هم بر درد
&&
نی رود ره نی غم کاری خورد
\\
نه دلش را تاب ماند در نیاز
&&
نه تنش را قوت روزه و نماز
\\
همچو موشی پیش گربه لا شود
&&
همچو بره پیش گرگ از جا رود
\\
اندرو نه حیله ماند نه روش
&&
پس کنم ناگفته‌تان من پرورش
\\
همچو بوبکر ربابی تن زنم
&&
دست چون داود در آهن زنم
\\
تا محال از دست من حالی شود
&&
مرغ پر بر کنده را بالی شود
\\
چون یدالله فوق ایدیهم بود
&&
دست ما را دست خود فرمود احد
\\
پس مرا دست دراز آمد یقین
&&
بر گذشته ز آسمان هفتمین
\\
دست من بنمود بر گردون هنر
&&
مقریا بر خوان که انشق القمر
\\
این صفت هم بهر ضعف عقلهاست
&&
با ضعیفان شرح قدرت کی رواست
\\
خود بدانی چون بر آری سر ز خواب
&&
ختم شد والله اعلم بالصواب
\\
مر ترا نه قوت خوردن بدی
&&
نه ره و پروای قی کردن بدی
\\
می‌شنیدم فحش و خر می‌راندم
&&
رب یسر زیر لب می‌خواندم
\\
از سبب گفتن مرا دستور نی
&&
ترک تو گفتن مرا مقدور نی
\\
هر زمان می‌گفتم از درد درون
&&
اهد قومی انهم لا یعلمون
\\
سجده‌ها می‌کرد آن رسته ز رنج
&&
کای سعادت ای مرا اقبال و گنج
\\
از خدا یابی جزاها ای شریف
&&
قوت شکرت ندارد این ضعیف
\\
شکر حق گوید ترا ای پیشوا
&&
آن لب و چانه ندارم و آن نوا
\\
دشمنی عاقلان زین سان بود
&&
زهر ایشان ابتهاج جان بود
\\
دوستی ابله بود رنج و ضلال
&&
این حکایت بشنو از بهر مثال
\\
\end{longtable}
\end{center}
