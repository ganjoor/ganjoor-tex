\begin{center}
\section*{بخش ۸۱ - متردد شدن در میان مذهبهای مخالف و بیرون‌شو و مخلص یافتن}
\label{sec:sh081}
\addcontentsline{toc}{section}{\nameref{sec:sh081}}
\begin{longtable}{l p{0.5cm} r}
همچنانک هر کسی در معرفت
&&
می‌کند موصوف غیبی را صفت
\\
فلسفی از نوع دیگر کرده شرح
&&
باحثی مر گفت او را کرده جرح
\\
وآن دگر در هر دو طعنه می‌زند
&&
وآن دگر از زرق جانی می‌کند
\\
هر یک از ره این نشانها زان دهند
&&
تا گمان آید که ایشان زان ده‌اند
\\
این حقیقت دان نه حق‌اند این همه
&&
نه به کلی گمرهانند این رمه
\\
زانک بی حق باطلی ناید پدید
&&
قلب را ابله به بوی زر خرید
\\
گر نبودی در جهان نقدی روان
&&
قلبها را خرج کردن کی توان
\\
تا نباشد راست کی باشد دروغ
&&
آن دروغ از راست می‌گیرد فروغ
\\
بر امید راست کژ را می‌خرند
&&
زهر در قندی رود آنگه خورند
\\
گر نباشد گندم محبوب‌نوش
&&
چه برد گندم‌نمای جو فروش
\\
پس مگو کین جمله دمها باطل‌اند
&&
باطلان بر بوی حق دام دل‌اند
\\
پس مگو جمله خیالست و ضلال
&&
بی‌حقیقت نیست در عالم خیال
\\
حق شب قدرست در شبها نهان
&&
تا کند جان هر شبی را امتحان
\\
نه همه شبها بود قدر ای جوان
&&
نه همه شبها بود خالی از آن
\\
در میان دلق‌پوشان یک فقیر
&&
امتحان کن وانک حقست آن بگیر
\\
مؤمن کیس ممیز کو که تا
&&
باز داند حیزکان را از فتی
\\
گرنه معیوبات باشد در جهان
&&
تاجران باشند جمله ابلهان
\\
پس بود کالاشناسی سخت سهل
&&
چونک عیبی نیست چه نااهل و اهل
\\
ور همه عیبست دانش سود نیست
&&
چون همه چوبست اینجا عود نیست
\\
آنک گوید جمله حق‌اند احمقیست
&&
وانک گوید جمله باطل او شقیست
\\
تاجران انبیا کردند سود
&&
تاجران رنگ و بو کور و کبود
\\
می‌نماید مار اندر چشم مال
&&
هر دو چشم خویش را نیکو بمال
\\
منگر اندر غبطهٔ این بیع و سود
&&
بنگر اندر خسر فرعون و ثمود
\\
اندرین گردون مکرر کن نظر
&&
زانک حق فرمود ثم ارجع بصر
\\
\end{longtable}
\end{center}
