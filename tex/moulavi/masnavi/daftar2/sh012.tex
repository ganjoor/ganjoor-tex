\begin{center}
\section*{بخش ۱۲ - ترسانیدن شخصی زاهدی را کی کم گری تا کور نشوی}
\label{sec:sh012}
\addcontentsline{toc}{section}{\nameref{sec:sh012}}
\begin{longtable}{l p{0.5cm} r}
زاهدی را گفت یاری در عمل
&&
کم گری تا چشم را ناید خلل
\\
گفت زاهد از دو بیرون نیست حال
&&
چشم بیند یا نبیند آن جمال
\\
گر ببیند نور حق خود چه غمست
&&
در وصال حق دو دیده چه کمست
\\
ور نخواهد دید حق را گو برو
&&
این چنین چشم شقی گو کور شو
\\
غم مخور از دیده کان عیسی تراست
&&
چپ مرو تا بخشدت دو چشم راست
\\
عیسی روح تو با تو حاضرست
&&
نصرت از وی خواه کو خوش ناصرست
\\
لیک بیگار تن پر استخوان
&&
بر دل عیسی منه تو هر زمان
\\
همچو آن ابله که اندر داستان
&&
ذکر او کردیم بهر راستان
\\
زندگی تن مجو از عیسی‌ات
&&
کام فرعونی مخواه از موسی‌ات
\\
بر دل خود کم نه اندیشهٔ معاش
&&
عیش کم ناید تو بر درگاه باش
\\
این بدن خرگاه آمد روح را
&&
یا مثال کشتیی مر نوح را
\\
ترک چون باشد بیابد خرگهی
&&
خاصه چون باشد عزیز درگهی
\\
\end{longtable}
\end{center}
