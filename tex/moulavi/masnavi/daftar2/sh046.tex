\begin{center}
\section*{بخش ۴۶ - سبب پریدن و چرخیدن مرغی با مرغی کی جنس او نبود}
\label{sec:sh046}
\addcontentsline{toc}{section}{\nameref{sec:sh046}}
\begin{longtable}{l p{0.5cm} r}
آن حکیمی گفت دیدم هم تکی
&&
در بیابان زاغ را با لکلکی
\\
در عجب ماندم بجستم حالشان
&&
تا چه قدر مشترک یابم نشان
\\
چون شدم نزدیک من حیران و دنگ
&&
خود بدیدم هر دوان بودند لنگ
\\
خاصه شه‌بازی که او عرشی بود
&&
با یکی جغدی که او فرشی بود
\\
آن یکی خورشید علیین بود
&&
وین دگر خفاش کز سجین بود
\\
آن یکی نوری ز هر عیبی بری
&&
وین یکی کوری گدای هر دری
\\
آن یکی ماهی که بر پروین زند
&&
وین یکی کرمی که در سرگین زید
\\
آن یکی یوسف‌رخی عیسی‌نفس
&&
وین یکی گرگی و یا خر با جرس
\\
آن یکی پران شده در لامکان
&&
وین یکی در کاهدان همچون سگان
\\
با زبان معنوی گل با جعل
&&
این همی‌گوید که ای گنده‌بغل
\\
گر گریزانی ز گلشن بی گمان
&&
هست آن نفرت کمال گلستان
\\
غیرت من بر سر تو دورباش
&&
می‌زند کای خس ازینجا دور باش
\\
ور بیامیزی تو با من ای دنی
&&
این گمان آید که از کان منی
\\
بلبلان را جای می‌زیبد چمن
&&
مر جعل را در چمین خوشتر وطن
\\
حق مرا چون از پلیدی پاک داشت
&&
چون سزد بر من پلیدی را گماشت
\\
یک رگم زیشان بد و آن را برید
&&
در من آن بدرگ کجا خواهد رسید
\\
یک نشان آدم آن بود از ازل
&&
که ملایک سر نهندش از محل
\\
یک نشان دیگر آنک آن بلیس
&&
ننهدش سر که منم شاه و رئیس
\\
پس اگر ابلیس هم ساجد شدی
&&
او نبودی آدم او غیری بدی
\\
هم سجود هر ملک میزان اوست
&&
هم جحود آن عدو برهان اوست
\\
هم گواه اوست اقرار ملک
&&
هم گواه اوست کفران سگک
\\
\end{longtable}
\end{center}
