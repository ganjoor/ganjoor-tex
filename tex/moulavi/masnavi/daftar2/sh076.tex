\begin{center}
\section*{بخش ۷۶ - فوت شدن دزد بواز دادن آن شخص صاحب‌خانه را کی نزدیک آمده بود کی دزد را دریابد و بگیرد}
\label{sec:sh076}
\addcontentsline{toc}{section}{\nameref{sec:sh076}}
\begin{longtable}{l p{0.5cm} r}
این بدان ماند که شخصی دزد دید
&&
در وثاق اندر پی او می‌دوید
\\
تا دو سه میدان دوید اندر پیش
&&
تا در افکند آن تعب اندر خویش
\\
اندر آن حمله که نزدیک آمدش
&&
تا بدو اندر جهد در یابدش
\\
دزد دیگر بانگ کردش که بیا
&&
تا ببینی این علامات بلا
\\
زود باش و باز گرد ای مرد کار
&&
تا ببینی حال اینجا زار زار
\\
گفت باشد کان طرف دزدی بود
&&
گر نگردم زود این بر من رود
\\
در زن و فرزند من دستی زند
&&
بستن این دزد سودم کی کند
\\
این مسلمان از کرم می‌خواندم
&&
گر نگردم زود پیش آید ندم
\\
بر امید شفقت آن نیکخواه
&&
دزد را بگذاشت باز آمد براه
\\
گفت ای یار نکو احوال چیست
&&
این فغان و بانگ تو از دست کیست
\\
گفت اینک بین نشان پای دزد
&&
این طرف رفتست دزد زن‌بمزد
\\
نک نشان پای دزد قلتبان
&&
در پی او رو بدین نقش و نشان
\\
گفت ای ابله چه می‌گویی مرا
&&
من گرفته بودم آخر مر ورا
\\
دزد را از بانگ تو بگذاشتم
&&
من تو خر را آدمی پنداشتم
\\
این چه ژاژست و چه هرزه ای فلان
&&
من حقیقت یافتم چه بود نشان
\\
گفت من از حق نشانت می‌دهم
&&
این نشانست از حقیقت آگهم
\\
گفت طراری تو یا خود ابلهی
&&
بلک تو دزدی و زین حال آگهی
\\
خصم خود را می‌کشیدم من کشان
&&
تو رهانیدی ورا کاینک نشان
\\
تو جهت‌گو من برونم از جهات
&&
در وصال آیات کو یا بینات
\\
صنع بیند مرد محجوب از صفات
&&
در صفات آنست کو گم کرد ذات
\\
واصلان چون غرق ذات‌اند ای پسر
&&
کی کنند اندر صفات او نظر
\\
چونک اندر قعر جو باشد سرت
&&
کی به رنگ آب افتد منظرت
\\
ور به رنگ آب باز آیی ز قعر
&&
پس پلاسی بستدی دادی تو شعر
\\
طاعت عامه گناه خاصگان
&&
وصلت عامه حجاب خاص دان
\\
مر وزیری را کند شه محتسب
&&
شه عدو او بود نبود محب
\\
هم گناهی کرده باشد آن وزیر
&&
بی سبب نبود تغیر ناگزیر
\\
آنک ز اول محتسب بد خود ورا
&&
بخت و روزی آن بدست از ابتدا
\\
لیک آنک اول وزیر شه بدست
&&
محتسب کردن سبب فعل بدست
\\
چون ترا شه ز آستانه پیش خواند
&&
باز سوی آستانه باز راند
\\
تو یقین می‌دان که جرمی کرده‌ای
&&
جبر را از جهل پیش آورده‌ای
\\
که مرا روزی و قسمت این بدست
&&
پس چرا دی بودت آن دولت به دست
\\
قسمت خود خود بریدی تو ز جهل
&&
قسمت خود را فزاید مرد اهل
\\
\end{longtable}
\end{center}
