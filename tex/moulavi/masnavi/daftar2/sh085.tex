\begin{center}
\section*{بخش ۸۵ - حکایت هندو کی با یار خود جنگ می‌کرد بر کاری و خبر نداشت کی او هم بدان مبتلاست}
\label{sec:sh085}
\addcontentsline{toc}{section}{\nameref{sec:sh085}}
\begin{longtable}{l p{0.5cm} r}
چار هندو در یکی مسجد شدند
&&
بهر طاعت راکع و ساجد شدند
\\
هر یکی بر نیتی تکبیر کرد
&&
در نماز آمد بمسکینی و درد
\\
مؤذن آمد از یکی لفظی بجست
&&
کای مؤذن بانگ کردی وقت هست
\\
گفت آن هندوی دیگر از نیاز
&&
هی سخن گفتی و باطل شد نماز
\\
آن سیم گفت آن دوم را ای عمو
&&
چه زنی طعنه برو خود را بگو
\\
آن چهارم گفت حمد الله که من
&&
در نیفتادم بچه چون آن سه تن
\\
پس نماز هر چهاران شد تباه
&&
عیب‌گویان بیشتر گم کرده راه
\\
ای خنک جانی که عیب خویش دید
&&
هر که عیبی گفت آن بر خود خرید
\\
زانک نیم او ز عیبستان بدست
&&
وآن دگر نیمش ز غیبستان بدست
\\
چونک بر سر مرا ترا ده ریش هست
&&
مرهمت بر خویش باید کار بست
\\
عیب کردن خویش را داروی اوست
&&
چون شکسته گشت جای ارحمواست
\\
گر همان عیبت نبود ایمن مباش
&&
بوک آن عیب از تو گردد نیز فاش
\\
لا تخافوا از خدا نشنیده‌ای
&&
پس چه خود را ایمن و خوش دیده‌ای
\\
سالها ابلیس نیکونام زیست
&&
گشت رسوا بین که او را نام چیست
\\
در جهان معروف بد علیای او
&&
گشت معروفی بعکس ای وای او
\\
تا نه‌ای ایمن تو معروفی مجو
&&
رو بشوی از خوف پس بنمای رو
\\
تا نروید ریش تو ای خوب من
&&
بر دگر ساده‌زنخ طعنه مزن
\\
این نگر که مبتلا شد جان او
&&
در چهی افتاد تا شد پند تو
\\
تو نیفتادی که باشی پند او
&&
زهر او نوشید تو خور قند او
\\
\end{longtable}
\end{center}
