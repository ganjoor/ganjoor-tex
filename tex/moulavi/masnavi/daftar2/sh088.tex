\begin{center}
\section*{بخش ۸۸ - شکایت گفتن پیرمردی به طبیب از رنجوریها و جواب گفتن طبیب او را}
\label{sec:sh088}
\addcontentsline{toc}{section}{\nameref{sec:sh088}}
\begin{longtable}{l p{0.5cm} r}
گفت پیری مر طبیبی را که من
&&
در زحیرم از دماغ خویشتن
\\
گفت از پیریست آن ضعف دماغ
&&
گفت بر چشمم ز ظلمت هست داغ
\\
گفت از پیریست ای شیخ قدیم
&&
گفت پشتم درد می‌آید عظیم
\\
گفت از پیریست ای شیخ نزار
&&
گفت هر چه می‌خورم نبود گوار
\\
گفت ضعف معده هم از پیریست
&&
گفت وقت دم مرا دمگیریست
\\
گفت آری انقطاع دم بود
&&
چون رسد پیری دو صد علت شود
\\
گفت ای احمق برین بر دوختی
&&
از طبیبی تو همین آموختی
\\
ای مدمغ عقلت این دانش نداد
&&
که خدا هر رنج را درمان نهاد
\\
تو خر احمق ز اندک‌مایگی
&&
بر زمین ماندی ز کوته‌پایگی
\\
پس طبیبش گفت ای عمر تو شصت
&&
این غضب وین خشم هم از پیریست
\\
چون همه اوصاف و اجزا شد نحیف
&&
خویشتن‌داری و صبرت شد ضعیف
\\
بر نتابد دو سخن زو هی کند
&&
تاب یک جرعه ندارد قی کند
\\
جز مگر پیری که از حقست مست
&&
در درون او حیات طیبه‌ست
\\
از برون پیرست و در باطن صبی
&&
خود چه چیزست آن ولی و آن نبی
\\
گر نه پیدااند پیش نیک و بد
&&
چیست با ایشان خان را این حسد
\\
ور نمی‌دانندشان علم الیقین
&&
چیست این بغض و حیل‌سازی و کین
\\
ور بدانندی جزای رستخیز
&&
چون زنندی خویش بر شمشیر تیز
\\
بر تو می‌خندد مبین او را چنان
&&
صد قیامت در درونستش نهان
\\
دوزخ و جنت همه اجزای اوست
&&
هرچه اندیشی تو او بالای اوست
\\
هرچه اندیشی پذیرای فناست
&&
آنک در اندیشه ناید آن خداست
\\
بر در این خانه گستاخی ز چیست
&&
گر همی‌دانند کاندر خانه کیست
\\
ابلهان تعظیم مسجد می‌کنند
&&
در جفای اهل دل جد می‌کنند
\\
آن مجازست این حقیقت ای خران
&&
نیست مسجد جز درون سروران
\\
مسجدی کان اندرون اولیاست
&&
سجده‌گاه جمله است آنجا خداست
\\
تا دل اهل دلی نامد به درد
&&
هیچ قرنی را خدا رسوا نکرد
\\
قصد جنگ انبیا می‌داشتند
&&
جسم دیدند آدمی پنداشتند
\\
در تو هست اخلاق آن پیشینیان
&&
چون نمی‌ترسی که تو باشی همان
\\
آن نشانیها همه چون در تو هست
&&
چون تو زیشانی کجا خواهی برست
\\
\end{longtable}
\end{center}
