\begin{center}
\section*{بخش ۷۸ - فریفتن منافقان پیغامبر را علیه السلام تا به مسجد ضرارش برند}
\label{sec:sh078}
\addcontentsline{toc}{section}{\nameref{sec:sh078}}
\begin{longtable}{l p{0.5cm} r}
بر رسول حق فسونها خواندند
&&
رخش دستان و حیل می‌راندند
\\
آن رسول مهربان رحم‌کیش
&&
جز تبسم جز بلی ناورد پیش
\\
شکرهای آن جماعت یاد کرد
&&
در اجابت قاصدان را شاد کرد
\\
می‌نمود آن مکر ایشان پیش او
&&
یک به یک زان سان که اندر شیر مو
\\
موی را نادیده می‌کرد آن لطیف
&&
شیر را شاباش می‌گفت آن ظریف
\\
صد هزاران موی مکر و دمدمه
&&
چشم خوابانید آن دم زان همه
\\
راست می‌فرمود آن بحر کرم
&&
بر شما من از شما مشفق‌ترم
\\
من نشسته بر کنار آتشی
&&
با فروغ و شعلهٔ بس ناخوشی
\\
همچو پروانه شما آن سو دوان
&&
هر دو دست من شده پروانه‌ران
\\
چون بر آن شد تا روان گردد رسول
&&
غیرت حق بانگ زد مشنو ز غول
\\
کین خبیثان مکر و حیلت کرده‌اند
&&
جمله مقلوبست آنچ آورده‌اند
\\
قصد ایشان جز سیه‌رویی نبود
&&
خیر دین کی جست ترسا و جهود
\\
مسجدی بر جسر دوزخ ساختند
&&
با خدا نرد دغاها باختند
\\
قصدشان تفریق اصحاب رسول
&&
فضل حق را کی شناسد هر فضول
\\
تا جهودی را ز شام اینجا کشند
&&
که بوعظ او جهودان سرخوشند
\\
گفت پیغامبر که آری لیک ما
&&
بر سر راهیم و بر عزم غزا
\\
زین سفر چون باز گردم آنگهان
&&
سوی آن مسجد روان گردم روان
\\
دفعشان گفت و به سوی غزو تاخت
&&
با دغایان از دغا نردی بباخت
\\
چون بیامد از غزا باز آمدند
&&
چنگ اندر وعدهٔ ماضی زدند
\\
گفت حقش ای پیمبر فاش گو
&&
عذر را ور جنگ باشد باش گو
\\
گفتشان بس بد درون و دشمنید
&&
تا نگویم رازهاتان تن زنید
\\
چون نشانی چند از اسرارشان
&&
در بیان آورد بد شد کارشان
\\
قاصدان زو باز گشتند آن زمان
&&
حاش لله حاش لله دم‌زنان
\\
هر منافق مصحفی زیر بغل
&&
سوی پیغامبر بیاورد از دغل
\\
بهر سوگندان که ایمان جنتیست
&&
زانک سوگند آن کژان را سنتیست
\\
چون ندارد مرد کژ در دین وفا
&&
هر زمانی بشکند سوگند را
\\
راستان را حاجت سوگند نیست
&&
زانک ایشان را دو چشم روشنیست
\\
نقض میثاق و عهود از احمقیست
&&
حفظ ایمان و وفاکار تقیست
\\
گفت پیغامبر که سوگند شما
&&
راست گیرم یا که سوگند خدا
\\
باز سوگندی دگر خوردند قوم
&&
مصحف اندر دست و بر لب مهر صوم
\\
که بحق این کلام پاک راست
&&
کان بنای مسجد از بهر خداست
\\
اندر آنجا هیچ حیله و مکر نیست
&&
اندر آنجا ذکر و صدق و یاربیست
\\
گفت پیغامبر که آواز خدا
&&
می‌رسد در گوش من همچون صدا
\\
مهر بر گوش شما بنهاد حق
&&
تا به آواز خدا نارد سبق
\\
نک صریح آواز حق می‌آیدم
&&
همچو صاف از درد می‌پالایدم
\\
همچنانک موسی از سوی درخت
&&
بانگ حق بشنید کای مسعودبخت
\\
از درخت انی انا الله می‌شنید
&&
با کلام انوار می‌آمد پدید
\\
چون ز نور وحی در می‌ماندند
&&
باز نو سوگندها می‌خواندند
\\
چون خدا سوگند را خواند سپر
&&
کی نهد اسپر ز کف پیگارگر
\\
باز پیغامبر به تکذیب صریح
&&
قد کذبتم گفت با ایشان فصیح
\\
\end{longtable}
\end{center}
