\begin{center}
\section*{بخش ۲۴ - حسد کردن حشم بر غلام خاص}
\label{sec:sh024}
\addcontentsline{toc}{section}{\nameref{sec:sh024}}
\begin{longtable}{l p{0.5cm} r}
پادشاهی بنده‌ای را از کرم
&&
بر گزیده بود بر جملهٔ حشم
\\
جامگی او وظیفهٔ چل امیر
&&
ده یک قدرش ندیدی صد وزیر
\\
از کمال طالع و اقبال و بخت
&&
او ایازی بود و شه محمود وقت
\\
روح او با روح شه در اصل خویش
&&
پیش ازین تن بوده هم پیوند و خویش
\\
کار آن دارد که پیش از تن بدست
&&
بگذر از اینها که نو حادث شدست
\\
کار عارف‌راست کو نه احولست
&&
چشم او بر کشتهای اولست
\\
آنچ گندم کاشتندش و آنچ جو
&&
چشم او آنجاست روز و شب گرو
\\
آنچ آبستست شب جز آن نزاد
&&
حیله‌ها و مکرها بادست باد
\\
کی کند دل خوش به حیلتهای گش
&&
آنک بیند حیلهٔ حق بر سرش
\\
او درون دام و دامی می‌نهد
&&
جان تو نی آن جهد نی این جهد
\\
گر بروید ور بریزد صد گیاه
&&
عاقبت بر روید آن کشتهٔ اله
\\
کشت نو کارند بر کشت نخست
&&
این دوم فانیست و آن اول درست
\\
تخم اول کامل و بگزیده است
&&
تخم ثانی فاسد و پوسیده است
\\
افکن این تدبیر خود را پیش دوست
&&
گرچه تدبیرت هم از تدبیر اوست
\\
کار آن دارد که حق افراشتست
&&
آخر آن روید که اول کاشتست
\\
هرچه کاری از برای او بکار
&&
چون اسیر دوستی ای دوستدار
\\
گرد نفس دزد و کار او مپیچ
&&
هرچه آن نه کار حق هیچست هیچ
\\
پیش از آنک روز دین پیدا شود
&&
نزد مالک دزد شب رسوا شود
\\
رخت دزدیده بتدبیر و فنش
&&
مانده روز داوری بر گردنش
\\
صد هزاران عقل با هم بر جهند
&&
تا بغیر دام او دامی نهند
\\
دام خود را سخت‌تر یابند و بس
&&
کی نماید قوتی با باد خس
\\
گر تو گویی فایدهٔ هستی چه بود
&&
در سؤالت فایده هست ای عنود
\\
گر ندارد این سؤالت فایده
&&
چه شنویم این را عبث بی عایده
\\
ور سالت را بسی فایده‌هاست
&&
پس جهان بی فایده آخر چراست
\\
ور جهان از یک جهت بی فایده‌ست
&&
از جهتهای دگر پر عایده‌ست
\\
فایدهٔ تو گر مرا فایده نیست
&&
مر ترا چون فایده‌ست از وی مه‌ایست
\\
حسن یوسف عالمی را فایده
&&
گرچه بر اخوان عبث بد زایده
\\
لحن داوودی چنان محبوب بود
&&
لیک بر محروم بانگ چوب بود
\\
آب نیل از آب حیوان بد فزون
&&
لیک بر محروم و منکر بود خون
\\
هست بر مؤمن شهیدی زندگی
&&
بر منافق مردنست و ژندگی
\\
چیست در عالم بگو یک نعمتی
&&
که نه محرومند از وی امتی
\\
گاو و خر را فایده چه در شکر
&&
هست هر جان را یکی قوتی دگر
\\
لیک گر آن قوت بر وی عارضیست
&&
پس نصیحت کردن او را رایضیست
\\
چون کسی کو از مرض گل داشت دوست
&&
گرچه پندارد که آن خود قوت اوست
\\
قوت اصلی را فرامش کرده است
&&
روی در قوت مرض آورده است
\\
نوش را بگذاشته سم خورده است
&&
قوت علت را چو چربش کرده است
\\
قوت اصلی بشر نور خداست
&&
قوت حیوانی مرورا ناسزاست
\\
لیک از علت درین افتاد دل
&&
که خورد او روز و شب زین آب و گل
\\
روی زرد و پای سست و دل سبک
&&
کو غذای والسما ذات الحبک
\\
آن غذای خاصگان دولتست
&&
خوردن آن بی گلو و آلتست
\\
شد غذای آفتاب از نور عرش
&&
مر حسود و دیو را از دود فرش
\\
در شهیدان یرزقون فرمود حق
&&
آن غذا را نی دهان بد نی طبق
\\
دل ز هر یاری غذایی می‌خورد
&&
دل ز هر علمی صفایی می‌برد
\\
صورت هر آدمی چون کاسه ایست
&&
چشم از معنی او حساسه ایست
\\
از لقای هر کسی چیزی خوری
&&
وز قران هر قرین چیزی بری
\\
چون ستاره با ستاره شد قرین
&&
لایق هر دو اثر زاید یقین
\\
چون قران مرد و زن زاید بشر
&&
وز قران سنگ و آهن شد شرر
\\
وز قران خاک با بارانها
&&
میوه‌ها و سبزه و ریحانها
\\
وز قران سبزه‌ها با آدمی
&&
دلخوشی و بی‌غمی و خرمی
\\
وز قران خرمی با جان ما
&&
می‌بزاید خوبی و احسان ما
\\
قابل خوردن شود اجسام ما
&&
چون بر آید از تفرج کام ما
\\
سرخ رویی از قران خون بود
&&
خون ز خورشید خوش گلگون بود
\\
بهترین رنگها سرخی بود
&&
وان ز خورشیدست و از وی می‌رسد
\\
هر زمینی کان قرین شد با زحل
&&
شوره گشت و کشت را نبود محل
\\
قوت اندر فعل آید ز اتفاق
&&
چون قران دیو با اهل نفاق
\\
این معانی راست از چرخ نهم
&&
بی همه طاق و طرم طاق و طرم
\\
خلق را طاق و طرم عاریتست
&&
امر را طاق و طرم ماهیتست
\\
از پی طاق و طرم خواری کشند
&&
بر امید عز در خواری خوشند
\\
بر امید عز ده‌روزهٔ خدوک
&&
گردن خود کرده‌اند از غم چو دوک
\\
چون نمی‌آیند اینجا که منم
&&
کاندرین عز آفتاب روشنم
\\
مشرق خورشید برج قیرگون
&&
آفتاب ما ز مشرقها برون
\\
مشرق او نسبت ذرات او
&&
نه بر آمد نه فرو شد ذات او
\\
ما که واپس ماند ذرات وییم
&&
در دو عالم آفتاب بی فییم
\\
باز گرد شمس می‌گردم عجب
&&
هم ز فر شمس باشد این سبب
\\
شمس باشد بر سببها مطلع
&&
هم ازو حبل سببها منقطع
\\
صد هزاران بار ببریدم امید
&&
از کی از شمس این شما باور کنید
\\
تو مرا باور مکن کز آفتاب
&&
صبر دارم من و یا ماهی ز آب
\\
ور شوم نومید نومیدی من
&&
عین صنع آفتابست ای حسن
\\
عین صنع از نفس صانع چون برد
&&
هیچ هست از غیر هستی چون چرد
\\
جمله هستیها ازین روضه چرند
&&
گر براق و تازیان ور خود خرند
\\
وانک گردشها از آن دریا ندید
&&
هر دم آرد رو به محرابی جدید
\\
او ز بحر عذب آب شور خورد
&&
تا که آب شور او را کور کرد
\\
بحر می‌گوید به دست راست خور
&&
ز آب من ای کور تا یابی بصر
\\
هست دست راست اینجا ظن راست
&&
کو بداند نیک و بد را کز کجاست
\\
نیزه‌گردانیست ای نیزه که تو
&&
راست می‌گردی گهی گاهی دوتو
\\
ما ز عشق شمس دین بی ناخنیم
&&
ورنه ما آن کور را بینا کنیم
\\
هان ضیاء الحق حسام الدین تو زود
&&
داروش کن کوری چشم حسود
\\
توتیای کبریای تیزفعل
&&
داروی ظلمت‌کش استیزفعل
\\
آنک گر بر چشم اعمی بر زند
&&
ظلمت صد ساله را زو بر کند
\\
جمله کوران را دواکن جز حسود
&&
کز حسودی بر تو می‌آرد جحود
\\
مر حسودت را اگر چه آن منم
&&
جان مده تا همچنین جان می‌کنم
\\
آنک او باشد حسود آفتاب
&&
وانک می‌رنجد ز بود آفتاب
\\
اینت درد بی‌دوا کوراست آه
&&
اینت افتاده ابد در قعر چاه
\\
نفی خورشید ازل بایست او
&&
کی برآید این مراد او بگو
\\
باز آن باشد که باز آید به شاه
&&
باز کورست آنک شد گم‌کرده راه
\\
راه را گم کرد و در ویران فتاد
&&
باز در ویران بر جغدان فتاد
\\
او همه نورست از نور رضا
&&
لیک کورش کرد سرهنگ قضا
\\
خاک در چشمش زد و از راه برد
&&
در میان جغد و ویرانش سپرد
\\
بر سری جغدانش بر سر می‌زنند
&&
پر و بال نازنینش می‌کنند
\\
ولوله افتاد در جغدان که ها
&&
باز آمد تا بگیرد جای ما
\\
چون سگان کوی پر خشم و مهیب
&&
اندر افتادند در دلق غریب
\\
باز گوید من چه در خوردم به جغد
&&
صد چنین ویران فدا کردم به جغد
\\
من نخواهم بود اینجا می‌روم
&&
سوی شاهنشاه راجع می‌شوم
\\
خویشتن مکشید ای جغدان که من
&&
نه مقیمم می‌روم سوی وطن
\\
این خراب آباد در چشم شماست
&&
ورنه ما را ساعد شه ناز جاست
\\
جغد گفتا باز حیلت می‌کند
&&
تا ز خان و مان شما را بر کند
\\
خانه‌های ما بگیرد او بمکر
&&
برکند ما را به سالوسی ز وکر
\\
می‌نماید سیری این حیلت‌پرست
&&
والله از جمله حریصان بترست
\\
او خورد از حرص طین را همچو دبس
&&
دنبه مسپارید ای یاران به خرس
\\
لاف از شه می‌زند وز دست شه
&&
تا برد او ما سلیمان را ز ره
\\
خود چه جنس شاه باشد مرغکی
&&
مشنوش گر عقل داری اندکی
\\
جنس شاهست او و یا جنس وزیر
&&
هیچ باشد لایق گوزینه سیر
\\
آنچ می‌گوید ز مکر و فعل و فن
&&
هست سلطان با حشم جویای من
\\
اینت مالیخولیای ناپذیر
&&
اینت لاف خام و دام گول‌گیر
\\
هر که این باور کند از ابلهیست
&&
مرغک لاغر چه درخورد شهیست
\\
کمترین جغد ار زند بر مغز او
&&
مر ورا یاری‌گری از شاه کو
\\
گفت باز ار یک پر من بشکند
&&
بیخ جغدستان شهنشه بر کند
\\
جغد چه بود خود اگر بازی مرا
&&
دل برنجاند کند با من جفا
\\
شه کند توده به هر شیب و فراز
&&
صد هزاران خرمن از سرهای باز
\\
پاسبان من عنایات ویست
&&
هر کجا که من روم شه در پیست
\\
در دل سلطان خیال من مقیم
&&
بی خیال من دل سلطان سقیم
\\
چون بپراند مرا شه در روش
&&
می‌پرم بر اوج دل چون پرتوش
\\
همچو ماه و آفتابی می‌پرم
&&
پرده‌های آسمانها می‌درم
\\
روشنی عقلها از فکرتم
&&
انفطار آسمان از فطرتم
\\
بازم و حیران شود در من هما
&&
جغد کی بود تا بداند سر ما
\\
شه برای من ز زندان یاد کرد
&&
صد هزاران بسته را آزاد کرد
\\
یک دمم با جغدها دمساز کرد
&&
از دم من جغدها را باز کرد
\\
ای خنک جغدی که در پرواز من
&&
فهم کرد از نیکبختی راز من
\\
در من آویزید تا نازان شوید
&&
گرچه جغدانید شهبازان شوید
\\
آنک باشد با چنان شاهی حبیب
&&
هر کجا افتد چرا باشد غریب
\\
هر که باشد شاه دردش را دوا
&&
گر چو نی نالد نباشد بی نوا
\\
مالک ملک نیم من طبل‌خوار
&&
طبل بازم می‌زند شه از کنار
\\
طبل باز من ندای ارجعی
&&
حق گواه من به رغم مدعی
\\
من نیم جنس شهنشه دور ازو
&&
لیک دارم در تجلی نور ازو
\\
نیست جنسیت ز روی شکل و ذات
&&
آب جنس خاک آمد در نبات
\\
باد جنس آتش آمد در قوام
&&
طبع را جنس آمدست آخر مدام
\\
جنس ما چون نیست جنس شاه ما
&&
مای ما شد بهر مای او فنا
\\
چون فنا شد مای ما او ماند فرد
&&
پیش پای اسپ او گردم چو گرد
\\
خاک شد جان و نشانیهای او
&&
هست بر خاکش نشان پای او
\\
خاک پایش شو برای این نشان
&&
تا شوی تاج سر گردن‌کشان
\\
تا که نفریبد شما را شکل من
&&
نقل من نوشید پیش از نقل من
\\
ای بسا کس را که صورت راه زد
&&
قصد صورت کرد و بر الله زد
\\
آخر این جان با بدن پیوسته است
&&
هیچ این جان با بدن مانند هست
\\
تاب نور چشم با پیهست جفت
&&
نور دل در قطرهٔ خونی نهفت
\\
شادی اندر گرده و غم در جگر
&&
عقل چون شمعی درون مغز سر
\\
این تعلقها نه بی کیفست و چون
&&
عقلها در دانش چونی زبون
\\
جان کل با جان جزو آسیب کرد
&&
جان ازو دری ستد در جیب کرد
\\
همچو مریم جان از آن آسیب جیب
&&
حامله شد از مسیح دلفریب
\\
آن مسیحی نه که بر خشک و ترست
&&
آن مسیحی کز مساحت برترست
\\
پس ز جان جان چو حامل گشت جان
&&
از چنین جانی شود حامل جهان
\\
پس جهان زاید جهانی دیگری
&&
این حشر را وا نماید محشری
\\
تا قیامت گر بگویم بشمرم
&&
من ز شرح این قیامت قاصرم
\\
این سخنها خود بمعنی یا ربیست
&&
حرفها دام دم شیرین‌لبیست
\\
چون کند تقصیر پس چون تن زند
&&
چونک لبیکش به یارب می‌رسد
\\
هست لبیکی که نتوانی شنید
&&
لیک سر تا پای بتوانی چشید
\\
\end{longtable}
\end{center}
