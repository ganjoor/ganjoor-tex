\begin{center}
\section*{بخش ۶۰ - تتمهٔ نصیحت رسول علیه السلام بیمار را}
\label{sec:sh060}
\addcontentsline{toc}{section}{\nameref{sec:sh060}}
\begin{longtable}{l p{0.5cm} r}
گفت پیغامبر مر آن بیمار را
&&
چون عیادت کرد یار زار را
\\
که مگر نوعی دعایی کرده‌ای
&&
از جهالت زهربایی خورده‌ای
\\
یاد آور چه دعا می‌گفته‌ای
&&
چون ز مکر نفس می‌آشفته‌ای
\\
گفت یادم نیست الا همتی
&&
دار با من یادم آید ساعتی
\\
از حضور نوربخش مصطفی
&&
پیش خاطر آمد او را آن دعا
\\
تافت زان روزن که از دل تا دلست
&&
روشنی که فرق حق و باطلست
\\
گفت اینک یادم آمد ای رسول
&&
آن دعا که گفته‌ام من بوالفضول
\\
چون گرفتار گنه می‌آمدم
&&
غرقه دست اندر حشایش می‌زدم
\\
از تو تهدید و وعیدی می‌رسید
&&
مجرمان را از عذاب بس شدید
\\
مضطرب می‌گشتم و چاره نبود
&&
بند محکم بود و قفل ناگشود
\\
نی مقام صبر و نی راه گریز
&&
نی امید توبه نی جای ستیز
\\
من چو هاروت و چو ماروت از حزن
&&
آه می‌کردم که ای خلاق من
\\
از خطر هاروت و ماروت آشکار
&&
چاه بابل را بکردند اختیار
\\
تا عذاب آخرت اینجا کشند
&&
گربزند و عاقل و ساحروشند
\\
نیک کردند و بجای خویش بود
&&
سهل‌تر باشد ز آتش رنج دود
\\
حد ندارد وصف رنج آن جهان
&&
سهل باشد رنج دنیا پیش آن
\\
ای خنک آن کو جهادی می‌کند
&&
بر بدن زجری و دادی می‌کند
\\
تا ز رنج آن جهانی وا رهد
&&
بر خود این رنج عبادت می‌نهد
\\
من همی‌گفتم که یا رب آن عذاب
&&
هم درین عالم بران بر من شتاب
\\
تا در آن عالم فراغت باشدم
&&
در چنین درخواست حلقه می‌زدم
\\
این چنین رنجوریی پیدام شد
&&
جان من از رنج بی آرام شد
\\
مانده‌ام از ذکر و از اوراد خود
&&
بی‌خبر گشتم ز خویش و نیک و بد
\\
گر نمی‌دیدم کنون من روی تو
&&
ای خجسته وی مبارک بوی تو
\\
می‌شدم از بند من یکبارگی
&&
کردیم شاهانه این غمخوارگی
\\
گفت هی هی این دعا دیگر مکن
&&
بر مکن تو خویش را از بیخ و بن
\\
تو چه طاقت داری ای مور نژند
&&
که نهد بر تو چنان کوه بلند
\\
گفت توبه کردم ای سلطان که من
&&
از سر جلدی نلافم هیچ فن
\\
این جهان تیهست و تو موسی و ما
&&
از گنه در تیه مانده مبتلا
\\
قوم موسی راه می‌پیموده‌اند
&&
آخر اندر گام اول بوده‌اند
\\
سالها ره می‌رویم و در اخیر
&&
همچنان در منزل اول اسیر
\\
گر دل موسی ز ما راضی بدی
&&
تیه را راه و کران پیدا شدی
\\
ور بکل بیزار بودی او ز ما
&&
کی رسیدی خوانمان هیچ از سما
\\
کی ز سنگی چشمه‌ها جوشان شدی
&&
در بیابان‌مان امان جان شدی
\\
بل به جای خوان خود آتش آمدی
&&
اندرین منزل لهب بر ما زدی
\\
چون دو دل شد موسی اندر کار ما
&&
گاه خصم ماست و گاهی یار ما
\\
خشمش آتش می‌زند در رخت ما
&&
حلم او رد می‌کند تیر بلا
\\
کی بود که حلم گردد خشم نیز
&&
نیست این نادر ز لطفت ای عزیز
\\
مدح حاضر وحشتست از بهر این
&&
نام موسی می‌برم قاصد چنین
\\
ورنه موسی کی روا دارد که من
&&
پیش تو یاد آورم از هیچ تن
\\
عهد ما بشکست صد بار و هزار
&&
عهد تو چون کوه ثابت بر قرار
\\
عهد ما کاه و به هر بادی زبون
&&
عهد تو کوه و ز صد که هم فزون
\\
حق آن قوت که بر تلوین ما
&&
رحمتی کن ای امیر لونها
\\
خویش را دیدیم و رسوایی خویش
&&
امتحان ما مکن ای شاه بیش
\\
تا فضیحتهای دیگر را نهان
&&
کرده باشی ای کریم مستعان
\\
بی‌حدی تو در جمال و در کمال
&&
در کژی ما بی‌حدیم و در ضلال
\\
بی حدی خویش بگمار ای کریم
&&
بر کژی بی حد مشتی لئیم
\\
هین که از تقطیع ما یک تار ماند
&&
مصر بودیم و یکی دیوار ماند
\\
البقیه البقیه ای خدیو
&&
تا نگردد شاد کلی جان دیو
\\
بهر ما نی بهر آن لطف نخست
&&
که تو کردی گمرهان را باز جست
\\
چون نمودی قدرتت بنمای رحم
&&
ای نهاده رحمها در لحم و شحم
\\
این دعا گر خشم افزاید ترا
&&
تو دعا تعلیم فرما مهترا
\\
آنچنان کادم بیفتاد از بهشت
&&
رجعتش دادی که رست از دیو زشت
\\
دیو کی بود کو ز آدم بگذرد
&&
بر چنین نطعی ازو بازی برد
\\
در حقیقت نفع آدم شد همه
&&
لعنت حاسد شده آن دمدمه
\\
بازیی دید و دو صد بازی ندید
&&
پس ستون خانهٔ خود را برید
\\
آنشی زد شب بکشت دیگران
&&
باد آتش را بکشت او بران
\\
چشم‌بندی بود لعنت دیو را
&&
تا زیان خصم دید آن ریو را
\\
خود زیان جان او شد ریو او
&&
گویی آدم بود دیو دیو او
\\
لعنت این باشد که کژبینش کند
&&
حاسد و خودبین و پر کینش کند
\\
تا نداند که هر آنک کرد بد
&&
عاقبت باز آید و بر وی زند
\\
جمله فرزین‌بندها بیند بعکس
&&
مات بر وی گردد و نقصان و وکس
\\
زانک گر او هیچ بیند خویش را
&&
مهلک و ناسور بیند ریش را
\\
درد خیزد زین چنین دیدن درون
&&
درد او را از حجاب آرد برون
\\
تا نگیرد مادران را درد زه
&&
طفل در زادن نیابد هیچ ره
\\
این امانت در دل و دل حامله‌ست
&&
این نصیحتها مثال قابله‌ست
\\
قابله گوید که زن را درد نیست
&&
درد باید درد کودک را رهیست
\\
آنک او بی‌درد باشد ره‌زنست
&&
زانک بی‌دردی انا الحق گفتنست
\\
آن انا بی وقت گفتن لعنتست
&&
آن انا در وقت گفتن رحمتست
\\
آن انا منصور رحمت شد یقین
&&
آن انا فرعون لعنت شد ببین
\\
لاجرم هر مرغ بی‌هنگام را
&&
سر بریدن واجبست اعلام را
\\
سر بریدن چیست کشتن نفس را
&&
در جهاد و ترک گفتن تفس را
\\
آنچنانک نیش کزدم بر کنی
&&
تا که یابد او ز کشتن ایمنی
\\
بر کنی دندان پر زهری ز مار
&&
تا رهد مار از بلای سنگسار
\\
هیچ نکشد نفس را جز ظل پیر
&&
دامن آن نفس‌کش را سخت گیر
\\
چون بگیری سخت آن توفیق هوست
&&
در تو هر قوت که آید جذب اوست
\\
ما رمیت اذ رمیت راست دان
&&
هر چه کارد جان بود از جان جان
\\
دست گیرنده ویست و بردبار
&&
دم بدم آن دم ازو اومید دار
\\
نیست غم گر دیر بی او مانده‌ای
&&
دیرگیر و سخت‌گیرش خوانده‌ای
\\
دیر گیرد سخت گیرد رحمتش
&&
یک دمت غایب ندارد حضرتش
\\
ور تو خواهی شرح این وصل و ولا
&&
از سر اندیشه می‌خوان والضحی
\\
ور تو گویی هم بدیها از ویست
&&
لیک آن نقصان فضل او کیست
\\
آن بدی دادن کمال اوست هم
&&
من مثالی گویمت ای محتشم
\\
کرد نقاشی دو گونه نقشها
&&
نقشهای صاف و نقشی بی صفا
\\
نقش یوسف کرد و حور خوش‌سرشت
&&
نقش عفریتان و ابلیسان زشت
\\
هر دو گونه نقش استادی اوست
&&
زشتی او نیست آن رادی اوست
\\
زشت را در غایت زشتی کند
&&
جمله زشتیها به گردش بر تند
\\
تا کمال دانشش پیدا شود
&&
منکر استادیش رسوا شود
\\
ور نداند زشت کردن ناقص است
&&
زین سبب خلاق گبر و مخلص است
\\
پس ازین رو کفر و ایمان شاهدند
&&
بر خداوندیش و هر دو ساجدند
\\
لیک مؤمن دان که طوعا ساجدست
&&
زانک جویای رضا و قاصدست
\\
هست کرها گبر هم یزدان‌پرست
&&
لیک قصد او مرادی دیگرست
\\
قلعهٔ سلطان عمارت می‌کند
&&
لیک دعوی امارت می‌کند
\\
گشته یاغی تا که ملک او بود
&&
عاقبت خود قلعه سلطانی شود
\\
مؤمن آن قلعه برای پادشاه
&&
می‌کند معمور نه از بهر جاه
\\
زشت گوید ای شه زشت‌آفرین
&&
قادری بر خوب و بر زشت مهین
\\
خوب گوید ای شه حسن و بها
&&
پاک گردانیدیم از عیبها
\\
\end{longtable}
\end{center}
