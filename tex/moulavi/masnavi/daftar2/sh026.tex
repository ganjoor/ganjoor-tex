\begin{center}
\section*{بخش ۲۶ - فرمودن والی آن مرد را کی این خاربن را کی نشانده‌ای بر سر راه بر کن}
\label{sec:sh026}
\addcontentsline{toc}{section}{\nameref{sec:sh026}}
\begin{longtable}{l p{0.5cm} r}
همچو آن شخص درشت خوش‌سخن
&&
در میان ره نشاند او خاربن
\\
ره گذریانش ملامت‌گر شدند
&&
پس بگفتندش بکن این را نکند
\\
هر دمی آن خاربن افزون شدی
&&
پای خلق از زخم آن پر خون شدی
\\
جامه‌های خلق بدریدی ز خار
&&
پای درویشان بخستی زار زار
\\
چون بجد حاکم بدو گفت این بکن
&&
گفت آری بر کنم روزیش من
\\
مدتی فردا و فردا وعده داد
&&
شد درخت خار او محکم نهاد
\\
گفت روزی حاکمش ای وعده کژ
&&
پیش آ در کار ما واپس مغژ
\\
گفت الایام یا عم بیننا
&&
گفت عجل لا تماطل دیننا
\\
تو که می‌گویی که فردا این بدان
&&
که بهر روزی که می‌آید زمان
\\
آن درخت بد جوان‌تر می‌شود
&&
وین کننده پیر و مضطر می‌شود
\\
خاربن در قوت و برخاستن
&&
خارکن در پیری و در کاستن
\\
خاربن هر روز و هر دم سبز و تر
&&
خارکن هر روز زار و خشک تر
\\
او جوان‌تر می‌شود تو پیرتر
&&
زود باش و روزگار خود مبر
\\
خاربن دان هر یکی خوی بدت
&&
بارها در پای خار آخر زدت
\\
بارها از خوی خود خسته شدی
&&
حس نداری سخت بی‌حس آمدی
\\
گر ز خسته گشتن دیگر کسان
&&
که ز خلق زشت تو هست آن رسان
\\
غافلی باری ز زخم خود نه‌ای
&&
تو عذاب خویش و هر بیگانه‌ای
\\
یا تبر بر گیر و مردانه بزن
&&
تو علی‌وار این در خیبر بکن
\\
یا به گلبن وصل کن این خار را
&&
وصل کن با نار نور یار را
\\
تا که نور او کشد نار ترا
&&
وصل او گلشن کند خار ترا
\\
تو مثال دوزخی او مؤمنست
&&
کشتن آتش به مؤمن ممکنست
\\
مصطفی فرمود از گفت جحیم
&&
کو بممن لابه‌گر گردد ز بیم
\\
گویدش بگذر ز من ای شاه زود
&&
هین که نورت سوز نارم را ربود
\\
پس هلاک نار نور مؤمنست
&&
زانک بی ضد دفع ضد لا یمکنست
\\
نار ضد نور باشد روز عدل
&&
کان ز قهر انگیخته شد این ز فضل
\\
گر همی خواهی تو دفع شر نار
&&
آب رحمت بر دل آتش گمار
\\
چشمهٔ آن آب رحمتمؤمنست
&&
آب حیوان روح پاک محسنست
\\
بس گریزانست نفس تو ازو
&&
زانک تو از آتشی او آب خو
\\
ز آب آتش زان گریزان می‌شود
&&
کآتشش از آب ویران می‌شود
\\
حس و فکر تو همه از آتشست
&&
حس شیخ و فکر او نور خوشست
\\
آب نور او چو بر آتش چکد
&&
چک چک از آتش بر آید برجهد
\\
چون کند چک‌چک تو گویش مرگ و درد
&&
تا شود این دوزخ نفس تو سرد
\\
تا نسوزد او گلستان ترا
&&
تا نسوزد عدل و احسان ترا
\\
بعد از آن چیزی که کاری بر دهد
&&
لاله و نسرین و سیسنبر دهد
\\
باز پهنا می‌رویم از راه راست
&&
باز گرد ای خواجه راه ما کجاست
\\
اندر آن تقریر بودیم ای حسود
&&
که خرت لنگست و منزل دور زود
\\
سال بیگه گشت وقت کشت نی
&&
جز سیه‌رویی و فعل زشت نی
\\
کرم در بیخ درخت تن فتاد
&&
بایدش بر کند و در آتش نهاد
\\
هین و هین ای راه‌رو بیگاه شد
&&
آفتاب عمر سوی چاه شد
\\
این دو روزک را که زورت هست زود
&&
پیر افشانی بکن از راه جود
\\
این قدر تخمی که ماندستت بباز
&&
تا بروید زین دو دم عمر دراز
\\
تا نمردست این چراغ با گهر
&&
هین فتیلش ساز و روغن زودتر
\\
هین مگو فردا که فرداها گذشت
&&
تا بکلی نگذرد ایام کشت
\\
پند من بشنو که تن بند قویست
&&
کهنه بیرون کن گرت میل نویست
\\
لب ببند و کف پر زر بر گشا
&&
بخل تن بگذار و پیش آور سخا
\\
ترک شهوتها و لذتها سخاست
&&
هر که در شهوت فرو شد برنخاست
\\
این سخا شاخیست از سرو بهشت
&&
وای او کز کف چنین شاخی بهشت
\\
عروة الوثقاست این ترک هوا
&&
برکشد این شاخ جان را بر سما
\\
تا برد شاخ سخا ای خوب‌کیش
&&
مر ترا بالاکشان تا اصل خویش
\\
یوسف حسنی و این عالم چو چاه
&&
وین رسن صبرست بر امر اله
\\
یوسفا آمد رسن در زن دو دست
&&
از رسن غافل مشو بیگه شدست
\\
حمد لله کین رسن آویختند
&&
فضل و رحمت را بهم آمیختند
\\
تا ببینی عالم جان جدید
&&
عالم بس آشکار ناپدید
\\
این جهان نیست چون هستان شده
&&
وان جهان هست بس پنهان شده
\\
خاک بر بادست و بازی می‌کند
&&
کژنمایی پرده‌سازی می‌کند
\\
اینک بر کارست بی‌کارست و پوست
&&
وانک پنهانست مغز و اصل اوست
\\
خاک همچون آلتی در دست باد
&&
باد را دان عالی و عالی‌نژاد
\\
چشم خاکی را به خاک افتد نظر
&&
بادبین چشمی بود نوعی دگر
\\
اسپ داند اسپ را کو هست یار
&&
هم سواری داند احوال سوار
\\
چشم حس اسپست و نور حق سوار
&&
بی‌سواره اسپ خود ناید به کار
\\
پس ادب کن اسپ را از خوی بد
&&
ورنه پیش شاه باشد اسپ رد
\\
چشم اسپ از چشم شه رهبر بود
&&
چشم او بی‌چشم شه مضطر بود
\\
چشم اسپان جز گیاه و جز چرا
&&
هر کجا خوانی بگوید نی چرا
\\
نور حق بر نور حس راکب شود
&&
آنگهی جان سوی حق راغب شود
\\
اسپ بی راکب چه داند رسم راه
&&
شاه باید تا بداند شاه‌راه
\\
سوی حسی رو که نورش راکبست
&&
حس را آن نور نیکو صاحبست
\\
نور حس را نور حق تزیین بود
&&
معنی نور علی نور این بود
\\
نور حسی می‌کشد سوی ثری
&&
نور حقش می‌برد سوی علی
\\
زانک محسوسات دونتر عالمیست
&&
نور حق دریا و حس چون شب‌نمیست
\\
لیک پیدا نیست آن راکب برو
&&
جز به آثار و به گفتار نکو
\\
نور حسی کو غلیظست و گران
&&
هست پنهان در سواد دیدگان
\\
چونک نور حس نمی‌بینی ز چشم
&&
چون ببینی نور آن دینی ز چشم
\\
نور حس با این غلیظی مختفیست
&&
چون خفی نبود ضیائی کان صفیست
\\
این جهان چون خس به دست باد غیب
&&
عاجزی پیش گرفت و داد غیب
\\
گه بلندش می‌کند گاهیش پست
&&
گه درستش می‌کند گاهی شکست
\\
گه یمینش می‌برد گاهی یسار
&&
گه گلستانش کند گاهیش خار
\\
دست پنهان و قلم بین خط‌گزار
&&
اسپ در جولان و ناپیدا سوار
\\
تیر پران بین و ناپیدا کمان
&&
جانها پیدا و پنهان جان جان
\\
تیر را مشکن که این تیر شهیست
&&
نیست پرتاوی ز شصت آگهیست
\\
ما رمیت اذ رمیت گفت حق
&&
کار حق بر کارها دارد سبق
\\
خشم خود بشکن تو مشکن تیر را
&&
چشم خشمت خون شمارد شیر را
\\
بوسه ده بر تیر و پیش شاه بر
&&
تیر خون‌آلود از خون تو تر
\\
آنچ پیدا عاجز و بسته و زبون
&&
وآنچ ناپیدا چنان تند و حرون
\\
ما شکاریم این چنین دامی کراست
&&
گوی چوگانیم چوگانی کجاست
\\
می‌درد می‌دوزد این خیاط کو
&&
می‌دمد می‌سوزد این نفاط کو
\\
ساعتی کافر کند صدیق را
&&
ساعتی زاهد کند زندیق را
\\
زانک مخلص در خطر باشد ز دام
&&
تا ز خود خالص نگردد او تمام
\\
زانک در راهست و ره‌زن بی‌حدست
&&
آن رهد کو در امان ایزدست
\\
آینه خالص نگشت او مخلص است
&&
مرغ را نگرفته است او مقنص است
\\
چونک مخلص گشت مخلص باز رست
&&
در مقام امن رفت و برد دست
\\
هیچ آیینه دگر آهن نشد
&&
هیچ نانی گندم خرمن نشد
\\
هیچ انگوری دگر غوره نشد
&&
هیچ میوهٔ پخته با کوره نشد
\\
پخته گرد و از تغیر دور شو
&&
رو چو برهان محقق نور شو
\\
چون ز خود رستی همه برهان شدی
&&
چونک بنده نیست شد سلطان شدی
\\
ور عیان خواهی صلاح الدین نمود
&&
دیده‌ها را کرد بینا و گشود
\\
فقر را از چشم و از سیمای او
&&
دید هر چشمی که دارد نور هو
\\
شیخ فعالست بی‌آلت چو حق
&&
با مریدان داده بی گفتی سبق
\\
دل به دست او چو موم نرم رام
&&
مهر او گه ننگ سازد گاه نام
\\
مهر مومش حاکی انگشتریست
&&
باز آن نقش نگین حاکی کیست
\\
حاکی اندیشهٔ آن زرگرست
&&
سلسلهٔ هر حلقه اندر دیگرست
\\
این صدا در کوه دلها بانگ کیست
&&
گه پرست از بانگ این که گه تهیست
\\
هر کجا هست او حکیمست اوستاد
&&
بانگ او زین کوه دل خالی مباد
\\
هست که کوا مثنا می‌کند
&&
هست که کآواز صدتا می‌کند
\\
می‌زهاند کوه از آن آواز و قال
&&
صد هزاران چشمهٔ آب زلال
\\
چون ز که آن لطف بیرون می‌شود
&&
آبها در چشمه‌ها خون می‌شود
\\
زان شهنشاه همایون‌نعل بود
&&
که سراسر طور سینا لعل بود
\\
جان پذیرفت و خرد اجزای کوه
&&
ما کم از سنگیم آخر ای گروه
\\
نه ز جان یک چشمه جوشان می‌شود
&&
نه بدن از سبزپوشان می‌شود
\\
نی صدای بانگ مشتاقی درو
&&
نی صفای جرعهٔ ساقی درو
\\
کو حمیت تا ز تیشه وز کلند
&&
این چنین که را بکلی بر کنند
\\
بوک بر اجزای او تابد مهی
&&
بوک در وی تاب مه یابد رهی
\\
چون قیامت کوهها را برکند
&&
بر سر ما سایه کی می‌افکند
\\
این قیامت زان قیامت کی کمست
&&
آن قیامت زخم و این چون مرهمست
\\
هر که دید این مرهم از زخم ایمنست
&&
هر بدی کین حسن دید او محسنست
\\
ای خنک زشتی که خوبش شد حریف
&&
وای گل‌رویی که جفتش شد خریف
\\
نان مرده چون حریف جان شود
&&
زنده گردد نان و عین آن شود
\\
هیزم تیره حریف نار شد
&&
تیرگی رفت و همه انوار شد
\\
در نمکلان چون خر مرده فتاد
&&
آن خری و مردگی یکسو نهاد
\\
صبغة الله هست خم رنگ هو
&&
پیسها یک رنگ گردد اندرو
\\
چون در آن خم افتد و گوییش قم
&&
از طرب گوید منم خم لا تلم
\\
آن منم خم خود انا الحق گفتنست
&&
رنگ آتش دارد الا آهنست
\\
رنگ آهن محو رنگ آتشست
&&
ز آتشی می‌لافد و خامش وشست
\\
چون بسرخی گشت همچون زر کان
&&
پس انا النارست لافش بی زبان
\\
شد ز رنگ و طبع آتش محتشم
&&
گوید او من آتشم من آتشم
\\
آتشم من گر ترا شکیست و ظن
&&
آزمون کن دست را بر من بزن
\\
آتشم من بر تو گر شد مشتبه
&&
روی خود بر روی من یک‌دم بنه
\\
آدمی چون نور گیرد از خدا
&&
هست مسجود ملایک ز اجتبا
\\
نیز مسجود کسی کو چون ملک
&&
رسته باشد جانش از طغیان و شک
\\
آتش چه آهن چه لب ببند
&&
ریش تشبیه مشبه را مخند
\\
پار در دریا منه کم‌گوی از آن
&&
بر لب دریا خمش کن لب گزان
\\
گرچه صد چون من ندارد تاب بحر
&&
لیک می‌نشکیبم از غرقاب بحر
\\
جان و عقل من فدای بحر باد
&&
خونبهای عقل و جان این بحر داد
\\
تا که پایم می‌رود رانم درو
&&
چون نماند پا چو بطانم درو
\\
بی‌ادب حاضر ز غایب خوشترست
&&
حلقه گرچه کژ بود نی بر درست
\\
ای تن‌آلوده بگرد حوض گرد
&&
پاک کی گردد برون حوض مرد
\\
پاک کو از حوض مهجور اوفتاد
&&
او ز پاکی خویش هم دور اوفتاد
\\
پاکی این حوض بی‌پایان بود
&&
پاکی اجسام کم میزان بود
\\
زانک دل حوضست لیکن در کمین
&&
سوی دریا راه پنهان دارد این
\\
پاکی محدود تو خواهد مدد
&&
ورنه اندر خرج کم گردد عدد
\\
آب گفت آلوده را در من شتاب
&&
گفت آلوده که دارم شرم از آب
\\
گفت آب این شرم بی من کی رود
&&
بی من این آلوده زایل کی شود
\\
ز آب هر آلوده کو پنهان شود
&&
الحیاء یمنع الایمان بود
\\
دل ز پایهٔ حوض تن گلناک شد
&&
تن ز آب حوض دلها پاک شد
\\
گرد پایهٔ حوض دل گرد ای پسر
&&
هان ز پایهٔ حوض تن می‌کن حذر
\\
بحر تن بر بحر دل بر هم زنان
&&
در میانشان برزخ لا یبغیان
\\
گر تو باشی راست ور باشی تو کژ
&&
پیشتر می‌غژ بدو واپس مغژ
\\
پیش شاهان گر خطر باشد بجان
&&
لیک نشکیبند ازو با همتان
\\
شاه چون شیرین‌تر از شکر بود
&&
جان به شیرینی رود خوشتر بود
\\
ای ملامت‌گر سلامت مر ترا
&&
ای سلامت‌جو رها کن تو مرا
\\
جان من کوره‌ست با آتش خوشست
&&
کوره را این بس که خانهٔ آتشست
\\
همچو کوره عشق را سوزیدنیست
&&
هر که او زین کور باشد کوره نیست
\\
برگ بی برگی ترا چون برگ شد
&&
جان باقی یافتی و مرگ شد
\\
چون ترا غم شادی افزودن گرفت
&&
روضهٔ جانت گل و سوسن گرفت
\\
آنچ خوف دیگران آن امن تست
&&
بط قوی از بحر و مرغ خانه سست
\\
باز دیوانه شدم من ای طبیب
&&
باز سودایی شدم من ای حبیب
\\
حلقه‌های سلسلهٔ تو ذو فنون
&&
هر یکی حلقه دهد دیگر جنون
\\
داد هر حلقه فنونی دیگرست
&&
پس مرا هر دم جنونی دیگرست
\\
پس فنون باشد جنون این شد مثل
&&
خاصه در زنجیر این میر اجل
\\
آنچنان دیوانگی بگسست بند
&&
که همه دیوانگان پندم دهند
\\
\end{longtable}
\end{center}
