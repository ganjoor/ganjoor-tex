\begin{center}
\section*{بخش ۸۰ - قصهٔ آن شخص کی اشتر ضالهٔ خود می‌جست و می‌پرسید}
\label{sec:sh080}
\addcontentsline{toc}{section}{\nameref{sec:sh080}}
\begin{longtable}{l p{0.5cm} r}
اشتری گم کردی و جستیش چست
&&
چون بیابی چون ندانی کان تست
\\
ضاله چه بود ناقهٔ گم کرده‌ای
&&
از کفت بگریخته در پرده‌ای
\\
آمده در بار کردن کاروان
&&
اشتر تو زان میان گشته نهان
\\
می‌دوی این سو و آن سو خشک‌لب
&&
کاروان شد دور و نزدیکست شب
\\
رخت مانده بر زمین در راه خوف
&&
تو پی اشتر دوان گشته بطوف
\\
کای مسلمانان که دیدست اشتری
&&
جسته بیرون بامداد از آخری
\\
هر که بر گوید نشان از اشترم
&&
مژدگانی می‌دهم چندین درم
\\
باز می‌جویی نشان از هر کسی
&&
ریش خندت می‌کند زین هر خسی
\\
که اشتری دیدیم می‌رفت این طرف
&&
اشتری سرخی به سوی آن علف
\\
آن یکی گوید بریده گوش بود
&&
وآن دگر گوید جلش منقوش بود
\\
آن یکی گوید شتر یک چشم بود
&&
وآن دگر گوید ز گر بی پشم بود
\\
از برای مژدگانی صد نشان
&&
از گزافه هر خسی کرده بیان
\\
\end{longtable}
\end{center}
