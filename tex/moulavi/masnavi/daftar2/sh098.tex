\begin{center}
\section*{بخش ۹۸ - بقیهٔ قصهٔ طعنه زدن آن مرد بیگانه در شیخ}
\label{sec:sh098}
\addcontentsline{toc}{section}{\nameref{sec:sh098}}
\begin{longtable}{l p{0.5cm} r}
آن خبیث از شیخ می‌لایید ژاژ
&&
کژنگر باشد همیشه عقل کاژ
\\
که منش دیدم میان مجلسی
&&
او ز تقوی عاریست و مفلسی
\\
ورکه باور نیستت خیز امشبان
&&
تا ببینی فسق شیخت را عیان
\\
شب ببردش بر سر یک روزنی
&&
گفت بنگر فسق و عشرت کردنی
\\
بنگر آن سالوس روز و فسق شب
&&
روز همچون مصطفی شب بولهب
\\
روز عبدالله او را گشته نام
&&
شب نعوذ بالله و در دست جام
\\
دید شیشه در کف آن پیر پر
&&
گفت شیخا مر ترا هم هست غر
\\
تو نمی‌گفتی که در جام شراب
&&
دیو می‌میزد شتابان نا شتاب
\\
گفت جامم را چنان پر کرده‌اند
&&
کاندرو اندر نگنجد یک سپند
\\
بنگر اینجا هیچ گنجد ذره‌ای
&&
این سخن را کژ شنیده غره‌ای
\\
جام ظاهر خمر ظاهر نیست این
&&
دور دار این را ز شیخ غیب‌بین
\\
جام می هستی شیخست ای فلیو
&&
کاندرو اندر نگنجد بول دیو
\\
پر و مالامال از نور حقست
&&
جام تن بشکست نور مطلقست
\\
نور خورشید ار بیفتد بر حدث
&&
او همان نورست نپذیرد خبث
\\
شیخ گفت این خود نه جامست و نه می
&&
هین بزیر آ منکرا بنگر بوی
\\
آمد و دید انگبین خاص بود
&&
کور شد آن دشمن کور و کبود
\\
گفت پیر آن دم مرید خویش را
&&
رو برای من بجو می ای کیا
\\
که مرا رنجیست مضطر گشته‌ام
&&
من ز رنج از مخمصه بگذشته‌ام
\\
در ضرورت هست هر مردار پاک
&&
بر سر منکر ز لعنت باد خاک
\\
گرد خمخانه بر آمد آن مرید
&&
بهر شیخ از هر خمی او می‌چشید
\\
در همه خمخانه‌ها او می ندید
&&
گشته بد پر از عسل خم نبید
\\
گفت ای رندان چه حالست این چه کار
&&
هیچ خمی در نمی‌بینم عقار
\\
جمله رندان نزد آن شیخ آمدند
&&
چشم گریان دست بر سر می‌زدند
\\
در خرابات آمدی شیخ اجل
&&
جمله میها از قدومت شد عسل
\\
کرده‌ای مبدل تو می را از حدث
&&
جان ما را هم بدل کن از خبث
\\
گر شود عالم پر از خون مال‌مال
&&
کی خورد بندهٔ خدا الا حلال
\\
\end{longtable}
\end{center}
