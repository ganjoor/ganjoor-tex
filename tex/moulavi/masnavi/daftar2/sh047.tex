\begin{center}
\section*{بخش ۴۷ - تتمهٔ اعتماد آن مغرور بر تملق خرس}
\label{sec:sh047}
\addcontentsline{toc}{section}{\nameref{sec:sh047}}
\begin{longtable}{l p{0.5cm} r}
شخص خفت و خرس می‌راندش مگس
&&
وز ستیز آمد مگس زو باز پس
\\
چند بارش راند از روی جوان
&&
آن مگس زو باز می‌آمد دوان
\\
خشمگین شد با مگس خرس و برفت
&&
بر گرفت از کوه سنگی سخت زفت
\\
سنگ آورد و مگس را دید باز
&&
بر رخ خفته گرفته جای و ساز
\\
بر گرفت آن آسیا سنگ و بزد
&&
بر مگس تا آن مگس وا پس خزد
\\
سنگ روی خفته را خشخاش کرد
&&
این مثل بر جمله عالم فاش کرد
\\
مهر ابله مهر خرس آمد یقین
&&
کین او مهرست و مهر اوست کین
\\
عهد او سستست و ویران و ضعیف
&&
گفت او زفت و وفای او نحیف
\\
گر خورد سوگند هم باور مکن
&&
بشکند سوگند مرد کژسخن
\\
چونک بی‌سوگند گفتش بد دروغ
&&
تو میفت از مکر و سوگندش بدوغ
\\
نفس او میرست و عقل او اسیر
&&
صد هزاران مصحفش خود خورده گیر
\\
چونک بی سوگند پیمان بشکند
&&
گر خورد سوگند هم آن بشکند
\\
زانک نفس آشفته‌تر گردد از آن
&&
که کنی بندش به سوگند گران
\\
چون اسیری بند بر حاکم نهد
&&
حاکم آن را بر درد بیرون جهد
\\
بر سرش کوبد ز خشم آن بند را
&&
می‌زند بر روی او سوگند را
\\
تو ز اوفوا بالعقودش دست شو
&&
احفظوا ایمانکم با او مگو
\\
وانک حق را ساخت در پیمان سند
&&
تن کند چون تار و گرد او تند
\\
\end{longtable}
\end{center}
