\begin{center}
\section*{بخش ۹۲ - قصهٔ اعرابی و ریگ در جوال کردن و ملامت کردن آن فیلسوف او را}
\label{sec:sh092}
\addcontentsline{toc}{section}{\nameref{sec:sh092}}
\begin{longtable}{l p{0.5cm} r}
یک عرابی بار کرده اشتری
&&
دو جوال زفت از دانه پری
\\
او نشسته بر سر هر دو جوال
&&
یک حدیث‌انداز کرد او را سال
\\
از وطن پرسید و آوردش بگفت
&&
واندر آن پرسش بسی درها بسفت
\\
بعد از آن گفتش که این هر دو جوال
&&
چیست آکنده بگو مصدوق حال
\\
گفت اندر یک جوالم گندمست
&&
در دگر ریگی نه قوت مردمست
\\
گفت تو چون بار کردی این رمال
&&
گفت تا تنها نماند آن جوال
\\
گفت نیم گندم آن تنگ را
&&
در دگر ریز از پی فرهنگ را
\\
تا سبک گردد جوال و هم شتر
&&
گفت شاباش ای حکیم اهل و حر
\\
این چنین فکر دقیق و رای خوب
&&
تو چنین عریان پیاده در لغوب
\\
رحمش آمد بر حکیم و عزم کرد
&&
کش بر اشتر بر نشاند نیک‌مرد
\\
باز گفتش ای حکیم خوش‌سخن
&&
شمه‌ای از حال خود هم شرح کن
\\
این چنین عقل و کفایت که تراست
&&
تو وزیری یا شهی بر گوی راست
\\
گفت این هر دو نیم از عامه‌ام
&&
بنگر اندر حال و اندر جامه‌ام
\\
گفت اشتر چند داری چند گاو
&&
گفت نه این و نه آن ما را مکاو
\\
گفت رختت چیست باری در دکان
&&
گفت ما را کودکان و کو مکان
\\
گفت پس از نقد پرسم نقد چند
&&
که توی تنهارو و محبوب‌پند
\\
کیمیای مس عالم با توست
&&
عقل و دانش را گوهر تو بر توست
\\
گفت والله نیست یا وجه العرب
&&
در همه ملکم وجوه قوت شب
\\
پا برهنه تن برهنه می‌دوم
&&
هر که نانی می‌دهد آنجا روم
\\
مر مرا زین حکمت و فضل و هنر
&&
نیست حاصل جز خیال و درد سر
\\
پس عرب گفتش که رو دور از برم
&&
تا نبارد شومی تو بر سرم
\\
دور بر آن حکمت شومت ز من
&&
نطق تو شومست بر اهل زمن
\\
یا تو آن سو رو من این سو می‌دوم
&&
ور ترا ره پیش من وا پس روم
\\
یک جوالم گندم و دیگر ز ریگ
&&
به بود زین حیله‌های مردریگ
\\
احمقی‌ام پس مبارک احمقیست
&&
که دلم با برگ و جانم متقیست
\\
گر تو خواهی کت شقاوت کم شود
&&
جهد کن تا از تو حکمت کم شود
\\
حکمتی کز طبع زاید وز خیال
&&
حکمتی نی فیض نور ذوالجلال
\\
حکمت دنیا فزاید ظن و شک
&&
حکمت دینی برد فوق فلک
\\
زوبعان زیرک آخر زمان
&&
بر فزوده خویش بر پیشینیان
\\
حیله‌آموزان جگرها سوخته
&&
فعلها و مکرها آموخته
\\
صبر و ایثار و سخای نفس و جود
&&
باد داده کان بود اکسیر سود
\\
فکر آن باشد که بگشاید رهی
&&
راه آن باشد که پیش آید شهی
\\
شاه آن باشد که پیش شه رود
&&
نه بمخزنها و لشکر شه شود
\\
تا بماند شاهی او سرمدی
&&
همچو عز ملک دین احمدی
\\
\end{longtable}
\end{center}
