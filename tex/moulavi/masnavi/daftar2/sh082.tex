\begin{center}
\section*{بخش ۸۲ - امتحان هر چیزی تا ظاهر شود خیر و شری کی در ویست}
\label{sec:sh082}
\addcontentsline{toc}{section}{\nameref{sec:sh082}}
\begin{longtable}{l p{0.5cm} r}
یک نظر قانع مشو زین سقف نور
&&
بارها بنگر ببین هل من فطور
\\
چونک گفتت کاندرین سقف نکو
&&
بارها بنگر چو مرد عیب‌جو
\\
پس زمین تیره را دانی که چند
&&
دیدن و تمییز باید در پسند
\\
تا بپالاییم صافان را ز درد
&&
چند باید عقل ما را رنج برد
\\
امتحانهای زمستان و خزان
&&
تاب تابستان بهار همچو جان
\\
بادها و ابرها و برقها
&&
تا پدید آرد عوارض فرقها
\\
تا برون آرد زمین خاک‌رنگ
&&
هرچه اندر جیب دارد لعل و سنگ
\\
هرچه دزدیدست این خاک دژم
&&
از خزانهٔ حق و دریای کرم
\\
شحنهٔ تقدیر گوید راست گو
&&
آنچ بردی شرح وا ده مو بمو
\\
دزد یعنی خاک گوید هیچ هیچ
&&
شحنه او را در کشد در پیچ پیچ
\\
شحنه گاهش لطف گوید چون شکر
&&
گه بر آویزد کند هر چه بتر
\\
تا میان قهر و لطف آن خفیه‌ها
&&
ظاهر آید ز آتش خوف و رجا
\\
آن بهاران لطف شحنهٔ کبریاست
&&
و آن خزان تهدید و تخویف خداست
\\
و آن زمستان چارمیخ معنوی
&&
تا تو ای دزد خفی ظاهر شوی
\\
پس مجاهد را زمانی بسط دل
&&
یک زمانی قبض و درد و غش و غل
\\
زانک این آب و گلی کابدان ماست
&&
منکر و دزد ضیای جانهاست
\\
حق تعالی گرم و سرد و رنج و درد
&&
بر تن ما می‌نهد ای شیرمرد
\\
خوف و جوع و نقص اموال و بدن
&&
جمله بهر نقد جان ظاهر شدن
\\
این وعید و وعده‌ها انگیختست
&&
بهر این نیک و بدی کآمیختست
\\
چونک حق و باطلی آمیختند
&&
نقد و قلب اندر حرمدان ریختند
\\
پس محک می‌بایدش بگزیده‌ای
&&
در حقایق امتحانها دیده‌ای
\\
تا شود فاروق این تزویرها
&&
تا بود دستور این تدبیرها
\\
شیر ده ای مادر موسی ورا
&&
واندر آب افکن میندیش از بلا
\\
هر که در روز الست آن شیر خورد
&&
همچو موسی شیر را تمییز کرد
\\
گر تو بر تمییز طفلت مولعی
&&
این زمان یا ام موسی ارضعی
\\
تا ببیند طعم شیر مادرش
&&
تا فرو ناید بدایهٔ بد سرش
\\
\end{longtable}
\end{center}
