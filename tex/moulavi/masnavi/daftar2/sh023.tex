\begin{center}
\section*{بخش ۲۳ - قسم غلام در صدق و وفای یار خود از طهارت ظن خود}
\label{sec:sh023}
\addcontentsline{toc}{section}{\nameref{sec:sh023}}
\begin{longtable}{l p{0.5cm} r}
گفت نه والله بالله العظیم
&&
مالک الملک و به رحمان و رحیم
\\
آن خدایی که فرستاد انبیا
&&
نه بحاجت بل بفضل و کبریا
\\
آن خداوندی که از خاک ذلیل
&&
آفرید او شهسواران جلیل
\\
پاکشان کرد از مزاج خاکیان
&&
بگذرانید از تک افلاکیان
\\
بر گرفت از نار و نور صاف ساخت
&&
وانگه او بر جملهٔ انوار تاخت
\\
آن سنابرقی که بر ارواح تافت
&&
تا که آدم معرفت زان نور یافت
\\
آن کز آدم رست و دست شیث چید
&&
پس خلیفه‌ش کرد آدم کان بدید
\\
نوح از آن گوهر که برخوردار بود
&&
در هوای بحر جان دربار بود
\\
جان ابراهیم از آن انوار زفت
&&
بی حذر در شعله‌های نار رفت
\\
چونک اسمعیل در جویش فتاد
&&
پیش دشنهٔ آبدارش سر نهاد
\\
جان داوود از شعاعش گرم شد
&&
آهن اندر دست‌بافش نرم شد
\\
چون سلیمان بد وصالش را رضیع
&&
دیو گشتش بنده فرمان و مطیع
\\
در قضا یعقوب چون بنهاد سر
&&
چشم روشن کرد از بوی پسر
\\
یوسف مه‌رو چو دید آن آفتاب
&&
شد چنان بیدار در تعبیر خواب
\\
چون عصا از دست موسی آب خورد
&&
ملکت فرعون را یک لقمه کرد
\\
نردبانش عیسی مریم چو یافت
&&
بر فراز گنبد چارم شتافت
\\
چون محمد یافت آن ملک و نعیم
&&
قرص مه را کرد او در دم دو نیم
\\
چون ابوبکر آیت توفیق شد
&&
با چنان شه صاحب و صدیق شد
\\
چون عمر شیدای آن معشوق شد
&&
حق و باطل را چو دل فاروق شد
\\
چونک عثمان آن عیان را عین گشت
&&
نور فایض بود و ذی النورین گشت
\\
چون ز رویش مرتضی شد درفشان
&&
گشت او شیر خدا در مرج جان
\\
چون جنید از جند او دید آن مدد
&&
خود مقاماتش فزون شد از عدد
\\
بایزید اندر مزیدش راه دید
&&
نام قطب العارفین از حق شنید
\\
چونک کرخی کرخ او را شد حرس
&&
شد خلیفهٔ عشق و ربانی نفس
\\
پور ادهم مرکب آن سو راند شاد
&&
گشت او سلطان سلطانان داد
\\
وان شقیق از شق آن راه شگرف
&&
گشت او خورشید رای و تیز طرف
\\
صد هزاران پادشاهان نهان
&&
سر فرازانند زان سوی جهان
\\
نامشان از رشک حق پنهان بماند
&&
هر گدایی نامشان را بر نخواند
\\
حق آن نور و حق نورانیان
&&
کاندر آن بحرند همچون ماهیان
\\
بحر جان و جان بحر ار گویمش
&&
نیست لایق نام نو می‌جویمش
\\
حق آن آنی که این و آن ازوست
&&
مغزها نسبت بدو باشند پوست
\\
که صفات خواجه‌تاش و یار من
&&
هست صد چندان که این گفتار من
\\
آنچ می‌دانم ز وصف آن ندیم
&&
باورت ناید چه گویم ای کریم
\\
شاه گفت اکنون از آن خود بگو
&&
چند گویی آن این و آن او
\\
تو چه داری و چه حاصل کرده‌ای
&&
از تک دریا چه در آورده‌ای
\\
روز مرگ این حس تو باطل شود
&&
نور جان داری که یار دل شود
\\
در لحد کین چشم را خاک آگند
&&
هست آنچ گور را روشن کند
\\
آن زمان که دست و پایت بر درد
&&
پر و بالت هست تا جان بر پرد
\\
آن زمان کین جان حیوانی نماند
&&
جان باقی بایدت بر جا نشاند
\\
شرط من جا بالحسن نه کردنست
&&
این حسن را سوی حضرت بردنست
\\
جوهری داری ز انسان یا خری
&&
این عرضها که فنا شد چون بری
\\
این عرضهای نماز و روزه را
&&
چونک لایبقی زمانین انتفی
\\
نقل نتوان کرد مر اعراض را
&&
لیک از جوهر برند امراض را
\\
تا مبدل گشت جوهر زین عرض
&&
چون ز پرهیزی که زایل شد مرض
\\
گشت پرهیز عرض جوهر بجهد
&&
شد دهان تلخ از پرهیز شهد
\\
از زراعت خاکها شد سنبله
&&
داروی مو کرد مو را سلسله
\\
آن نکاح زن عرض بد شد فنا
&&
جوهر فرزند حاصل شد ز ما
\\
جفت کردن اسپ و اشتر را عرض
&&
جوهر کره بزاییدن غرض
\\
هست آن بستان نشاندن هم عرض
&&
کشت جوهر گشت بستان نک غرض
\\
هم عرض دان کیمیا بردن به کار
&&
جوهری زان کیمیا گر شد بیار
\\
صیقلی کردن عرض باشد شها
&&
زین عرض جوهر همی‌زاید صفا
\\
پس مگو که من عملها کرده‌ام
&&
دخل آن اعراض را بنما مرم
\\
این صفت کردن عرض باشد خمش
&&
سایهٔ بز را پی قربان مکش
\\
گفت شاها بی قنوط عقل نیست
&&
گر تو فرمایی عرض را نقل نیست
\\
پادشاها جز که یاس بنده نیست
&&
گر عرض کان رفت باز آینده نیست
\\
گر نبودی مر عرض را نقل و حشر
&&
فعل بودی باطل و اقوال فشر
\\
این عرضها نقل شد لونی دگر
&&
حشر هر فانی بود کونی دگر
\\
نقل هر چیزی بود هم لایقش
&&
لایق گله بود هم سایقش
\\
وقت محشر هر عرض را صورتیست
&&
صورت هر یک عرض را نوبتیست
\\
بنگر اندر خود نه تو بودی عرض
&&
جنبش جفتی و جفتی با غرض
\\
بنگر اندر خانه و کاشانه‌ها
&&
در مهندس بود چون افسانه‌ها
\\
آن فلان خانه که ما دیدیم خوش
&&
بود موزون صفه و سقف و درش
\\
از مهندس آن عرض و اندیشه‌ها
&&
آلت آورد و ستون از بیشه‌ها
\\
چیست اصل و مایهٔ هر پیشه‌ای
&&
جز خیال و جز عرض و اندیشه‌ای
\\
جمله اجزای جهان را بی غرض
&&
در نگر حاصل نشد جز از عرض
\\
اول فکر آخر آمد در عمل
&&
بنیت عالم چنان دان در ازل
\\
میوه‌ها در فکر دل اول بود
&&
در عمل ظاهر بخر می‌شود
\\
چون عمل کردی شجر بنشاندی
&&
اندر آخر حرف اول خواندی
\\
گرچه شاخ و برگ و بیخش اولست
&&
آن همه از بهر میوه مرسلست
\\
پس سری که مغز آن افلاک بود
&&
اندر آخر خواجهٔ لولاک بود
\\
نقل اعراضست این بحث و مقال
&&
نقل اعراضست این شیر و شگال
\\
جمله عالم خود عرض بودند تا
&&
اندرین معنی بیامد هل اتی
\\
این عرضها از چه زاید از صور
&&
وین صور هم از چه زاید از فکر
\\
این جهان یک فکرتست از عقل کل
&&
عقل چون شاهست و صورتها رسل
\\
عالم اول جهان امتحان
&&
عالم ثانی جزای این و آن
\\
چاکرت شاها جنایت می‌کند
&&
آن عرض زنجیر و زندان می‌شود
\\
بنده‌ات چون خدمت شایسته کرد
&&
آن عرض نی خلعتی شد در نبرد
\\
این عرض با جوهر آن بیضست و طیر
&&
این از آن و آن ازین زاید بسیر
\\
گفت شاهنشه چنین گیر المراد
&&
این عرضهای تو یک جوهر نزاد
\\
گفت مخفی داشتست آن را خرد
&&
تا بود غیب این جهان نیک و بد
\\
زانک گر پیدا شدی اشکال فکر
&&
کافر و مؤمن نگفتی جز که ذکر
\\
پس عیان بودی نه غیب ای شاه این
&&
نقش دین و کفر بودی بر جبین
\\
کی درین عالم بت و بتگر بدی
&&
چون کسی را زهره تسخر بدی
\\
پس قیامت بودی این دنیای ما
&&
در قیامت کی کند جرم و خطا
\\
گفت شه پوشید حق پاداش بد
&&
لیک از عامه نه از خاصان خود
\\
گر به دامی افکنم من یک امیر
&&
از امیران خفیه دارم نه از وزیر
\\
حق به من بنمود پس پاداش کار
&&
وز صورهای عملها صد هزار
\\
تو نشانی ده که من دانم تمام
&&
ماه را بر من نمی‌پوشد غمام
\\
گفت پس از گفت من مقصود چیست
&&
چون تو می‌دانی که آنچ بود چیست
\\
گفت شه حکمت در اظهار جهان
&&
آنک دانسته برون آید عیان
\\
آنچ می‌دانست تا پیدا نکرد
&&
بر جهان ننهاد رنج طلق و درد
\\
یک زمان بی کار نتوانی نشست
&&
تا بدی یا نیکیی از تو نجست
\\
این تقاضاهای کار از بهر آن
&&
شد موکل تا شود سرت عیان
\\
پس کلابهٔ تن کجا ساکن شود
&&
چون سر رشتهٔ ضمیرش می‌کشد
\\
تاسهٔ تو شد نشان آن کشش
&&
بر تو بی کاری بود چون جان‌کنش
\\
این جهان و آن جهان زاید ابد
&&
هر سبب مادر اثر از وی ولد
\\
چون اثر زایید آن هم شد سبب
&&
تا بزاید او اثرهای عجب
\\
این سببها نسل بر نسلست لیک
&&
دیده‌ای باید منور نیک نیک
\\
شاه با او در سخن اینجا رسید
&&
یا بدید از وی نشانی یا ندید
\\
گر بدید آن شاه جویا دور نیست
&&
لیک ما را ذکر آن دستور نیست
\\
چون ز گرمابه بیامد آن غلام
&&
سوی خویشش خواند آن شاه و همام
\\
گفت صحا لک نعیم دائم
&&
بس لطیفی و ظریف و خوب‌رو
\\
ای دریغا گر نبودی در تو آن
&&
که همی‌گوید برای تو فلان
\\
شاد گشتی هر که رویت دیدیی
&&
دیدنت ملک جهان ارزیدیی
\\
گفت رمزی زان بگو ای پادشاه
&&
کز برای من بگفت آن دین‌تباه
\\
گفت اول وصف دوروییت کرد
&&
کاشکارا تو دوایی خفیه درد
\\
خبث یارش را چو از شه گوش کرد
&&
در زمان دریای خشمش جوش کرد
\\
کف برآورد آن غلام و سرخ گشت
&&
تا که موج هجو او از حد گذشت
\\
کو ز اول دم که با من یار بود
&&
همچو سگ در قحط بس گه‌خوار بود
\\
چون دمادم کرد هجوش چون جرس
&&
دست بر لب زد شهنشاهش که بس
\\
گفت دانستم ترا از وی بدان
&&
از تو جان گنده‌ست و از یارت دهان
\\
پس نشین ای گنده‌جان از دور تو
&&
تا امیر او باشد و مامور تو
\\
در حدیث آمد که تسبیح از ریا
&&
همچو سبزهٔ گولخن دان ای کیا
\\
پس بدان که صورت خوب و نکو
&&
با خصال بد نیرزد یک تسو
\\
ور بود صورت حقیر و ناپذیر
&&
چون بود خلقش نکو در پاش میر
\\
صورت ظاهر فنا گردد بدان
&&
عالم معنی بماند جاودان
\\
چند بازی عشق با نقش سبو
&&
بگذر از نقش سبو رو آب جو
\\
صورتش دیدی ز معنی غافلی
&&
از صدف دری گزین گر عاقلی
\\
این صدفهای قوالب در جهان
&&
گرچه جمله زنده‌اند از بحر جان
\\
لیک اندر هر صدف نبود گهر
&&
چشم بگشا در دل هر یک نگر
\\
کان چه دارد وین چه دارد می‌گزین
&&
زانک کم‌یابست آن در ثمین
\\
گر به صورت می‌روی کوهی به شکل
&&
در بزرگی هست صد چندان که لعل
\\
هم به صورت دست و پا و پشم تو
&&
هست صد چندان که نقش چشم تو
\\
لیک پوشیده نباشد بر تو این
&&
کز همه اعضا دو چشم آمد گزین
\\
از یک اندیشه که آید در درون
&&
صد جهان گردد به یک دم سرنگون
\\
جسم سلطان گر به صورت یک بود
&&
صد هزاران لشکرش در پی دود
\\
باز شکل و صورت شاه صفی
&&
هست محکوم یکی فکر خفی
\\
خلق بی‌پایان ز یک اندیشه بین
&&
گشته چون سیلی روانه بر زمین
\\
هست آن اندیشه پیش خلق خرد
&&
لیک چون سیلی جهان را خورد و برد
\\
پس چو می‌بینی که از اندیشه‌ای
&&
قایمست اندر جهان هر پیشه‌ای
\\
خانه‌ها و قصرها و شهرها
&&
کوهها و دشتها و نهرها
\\
هم زمین و بحر و هم مهر و فلک
&&
زنده از وی همچو کز دریا سمک
\\
پس چرا از ابلهی پیش تو کور
&&
تن سلیمانست و اندیشه چو مور
\\
می‌نماید پیش چشمت که بزرگ
&&
هست اندیشه چو موش و کوه گرگ
\\
عالم اندر چشم تو هول و عظیم
&&
ز ابر و رعد و چرخ داری لرز و بیم
\\
وز جهان فکرتی ای کم ز خر
&&
ایمن و غافل چو سنگ بی‌خبر
\\
زانک نقشی وز خرد بی‌بهره‌ای
&&
آدمی خو نیستی خرکره‌ای
\\
سایه را تو شخص می‌بینی ز جهل
&&
شخص از آن شد نزد تو بازی و سهل
\\
باش تا روزی که آن فکر و خیال
&&
بر گشاید بی‌حجابی پر و بال
\\
کوهها بینی شده چون پشم نرم
&&
نیست گشته این زمین سرد و گرم
\\
نه سما بینی نه اختر نه وجود
&&
جز خدای واحد حی ودود
\\
یک فسانه راست آمد یا دروغ
&&
تا دهد مر راستیها را فروغ
\\
\end{longtable}
\end{center}
