\begin{center}
\section*{بخش ۹۰ - ترسیدن کودک از آن شخص صاحب جثه و گفتن آن شخص کی ای کودک مترس کی من نامردم}
\label{sec:sh090}
\addcontentsline{toc}{section}{\nameref{sec:sh090}}
\begin{longtable}{l p{0.5cm} r}
کنک زفتی کودکی را یافت فرد
&&
زرد شد کودک ز بیم قصد مرد
\\
گفت ایمن باش ای زیبای من
&&
که تو خواهی بود بر بالای من
\\
من اگر هولم مخنث دان مرا
&&
همچو اشتر بر نشین می‌ران مرا
\\
صورت مردان و معنی این چنین
&&
از برون آدم درون دیو لعین
\\
آن دهل را مانی ای زفت چو عاد
&&
که برو آن شاخ را می‌کوفت باد
\\
روبهی اشکار خود را باد داد
&&
بهر طبلی همچو خیک پر ز باد
\\
چون ندید اندر دهل او فربهی
&&
گفت خوکی به ازین خیک تهی
\\
روبهان ترسند ز آواز دهل
&&
عاقلش چندان زند که لا تقل
\\
\end{longtable}
\end{center}
