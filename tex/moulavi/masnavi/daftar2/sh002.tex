\begin{center}
\section*{بخش ۲ - هلال پنداشتن آن شخص خیال را در عهد عمر رضی الله عنه}
\label{sec:sh002}
\addcontentsline{toc}{section}{\nameref{sec:sh002}}
\begin{longtable}{l p{0.5cm} r}
ماه روزه گشت در عهد عمر
&&
بر سر کوهی دویدند آن نفر
\\
تا هلال روزه را گیرند فال
&&
آن یکی گفت ای عمر اینک هلال
\\
چون عمر بر آسمان مه را ندید
&&
گفت کین مه از خیال تو دمید
\\
ورنه من بیناترم افلاک را
&&
چون نمی‌بینم هلال پاک را
\\
گفت تر کن دست و بر ابرو بمال
&&
آنگهان تو در نگر سوی هلال
\\
چونک او تر کرد ابرو مه ندید
&&
گفت ای شه نیست مه شد ناپدید
\\
گفت آری موی ابرو شد کمان
&&
سوی تو افکند تیری از گمان
\\
چون یکی مو کژ شد او را راه زد
&&
تا به دعوی لاف دید ماه زد
\\
موی کژ چون پردهٔ گردون بود
&&
چون همه اجزات کژ شد چون بود
\\
راست کن اجزات را از راستان
&&
سر مکش ای راست‌رو ز آن آستان
\\
هم ترازو را ترازو راست کرد
&&
هم ترازو را ترازو کاست کرد
\\
هر که با ناراستان هم‌سنگ شد
&&
در کمی افتاد و عقلش دنگ شد
\\
رو اشداء علی‌الکفار باش
&&
خاک بر دلداری اغیار پاش
\\
بر سر اغیار چون شمشیر باش
&&
هین مکن روباه‌بازی شیر باش
\\
تا ز غیرت از تو یاران نسکلند
&&
زانک آن خاران عدو این گلند
\\
آتش اندر زن به گرگان چون سپند
&&
زانک آن گرگان عدو یوسفند
\\
جان بابا گویدت ابلیس هین
&&
تا بدم بفریبدت دیو لعین
\\
این چنین تلبیس با بابات کرد
&&
آدمی را این سیه‌رخ مات کرد
\\
بر سر شطرنج چستست این غراب
&&
تو مبین بازی به چشم نیم‌خواب
\\
زانک فرزین‌بندها داند بسی
&&
که بگیرد در گلویت چون خسی
\\
در گلو ماند خس او سالها
&&
چیست آن خس مهر جاه و مالها
\\
مال خس باشد چو هست ای بی‌ثبات
&&
در گلویت مانع آب حیات
\\
گر برد مالت عدوی پر فنی
&&
ره‌زنی را برده باشد ره‌زنی
\\
\end{longtable}
\end{center}
