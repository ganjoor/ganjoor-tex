\begin{center}
\section*{بخش ۴۳ - گفتن موسی علیه السلام گوساله‌پرست را کی آن خیال‌اندیشی و حزم تو کجاست}
\label{sec:sh043}
\addcontentsline{toc}{section}{\nameref{sec:sh043}}
\begin{longtable}{l p{0.5cm} r}
گفت موسی با یکی مست خیال
&&
کای بداندیش از شقاوت وز ضلال
\\
صد گمانت بود در پیغامبریم
&&
با چنین برهان و این خلق کریم
\\
صد هزاران معجزه دیدی ز من
&&
صد خیالت می‌فزود و شک و ظن
\\
از خیال و وسوسه تنگ آمدی
&&
طعن بر پیغامبری‌ام می‌زدی
\\
گرد از دریا بر آوردم عیان
&&
تا رهیدیت از شر فرعونیان
\\
ز آسمان چل سال کاسه و خوان رسید
&&
وز دعاام جوی از سنگی دوید
\\
این و صد چندین و چندین گرم و سرد
&&
از تو ای سرد آن توهم کم نکرد
\\
بانگ زد گوساله‌ای از جادوی
&&
سجده کردی که خدای من توی
\\
آن توهمهات را سیلاب برد
&&
زیرکی باردت را خواب برد
\\
چون نبودی بد گمان در حق او
&&
چون نهادی سر چنان ای زشت‌خو
\\
چون خیالت نامد از تزویر او
&&
وز فساد سحر احمق‌گیر او
\\
سامریی خود که باشد ای سگان
&&
که خدایی بر تراشد در جهان
\\
چون درین تزویر او یک‌دل شدی
&&
وز همه اشکالها عاطل شدی
\\
گاو می‌شاید خدایی را بلاف
&&
در رسولی‌ام تو چون کردی خلاف
\\
پیش گاوی سجده کردی از خری
&&
گشت عقلت صید سحر سامری
\\
چشم دزدیدی ز نور ذوالجلال
&&
اینت جهل وافر و عین ضلال
\\
شه بر آن عقل و گزینش که تراست
&&
چون تو کان جهل را کشتن سزاست
\\
گاو زرین بانگ کرد آخر چه گفت
&&
کاحمقان را این همه رغبت شکفت
\\
زان عجب‌تر دیده‌ایت از من بسی
&&
لیک حق را کی پذیرد هر خسی
\\
باطلان را چه رباید باطلی
&&
عاطلان را چه خوش آید عاطلی
\\
زانک هر جنسی رباید جنس خود
&&
گاو سوی شیر نر کی رو نهد
\\
گرگ بر یوسف کجا عشق آورد
&&
جز مگر از مکر تا او را خورد
\\
چون ز گرگی وا رهد محرم شود
&&
چون سگ کهف از بنی آدم شود
\\
چون ابوبکر از محمد برد بو
&&
گفت هذا لیس وجه کاذب
\\
چون نبد بوجهل از اصحاب درد
&&
دید صد شق قمر باور نکرد
\\
دردمندی کش ز بام افتاد طشت
&&
زو نهان کردیم حق پنهان نگشت
\\
وانک او جاهل بد از دردش بعید
&&
چند بنمودند و او آن را ندید
\\
آینهٔ دل صاف باید تا درو
&&
وا شناسی صورت زشت از نکو
\\
\end{longtable}
\end{center}
