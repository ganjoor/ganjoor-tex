\begin{center}
\section*{بخش ۷۰ - باز الحاح کردن معاویه ابلیس را}
\label{sec:sh070}
\addcontentsline{toc}{section}{\nameref{sec:sh070}}
\begin{longtable}{l p{0.5cm} r}
گفت غیر راستی نرهاندت
&&
داد سوی راستی می‌خواندت
\\
راست گو تا وا رهی از چنگ من
&&
مکر ننشاند غبار جنگ من
\\
گفت چون دانی دروغ و راست را
&&
ای خیال اندیش پر اندیشه‌ها
\\
گفت پیغامبر نشانی داده است
&&
قلب و نیکو را محک بنهاده است
\\
گفته است الکذب ریب فی القلوب
&&
گفت الصدق طمانین طروب
\\
دل نیارامد ز گفتار دروغ
&&
آب و روغن هیچ نفروزد فروغ
\\
در حدیث راست آرام دلست
&&
راستیها دانهٔ دام دلست
\\
دل مگر رنجور باشد بد دهان
&&
که نداند چاشنی این و آن
\\
چون شود از رنج و علت دل سلیم
&&
طعم کذب و راست را باشد علیم
\\
حرص آدم چون سوی گندم فزود
&&
از دل آدم سلیمی را ربود
\\
پس دروغ و عشوه‌ات را گوش کرد
&&
غره گشت و زهر قاتل نوش کرد
\\
کزدم از گندم ندانست آن نفس
&&
می‌پرد تمییز از مست هوس
\\
خلق مست آرزواند و هوا
&&
زان پذیرااند دستان ترا
\\
هر که خود را از هوا خو باز کرد
&&
چشم خود را آشنای راز کرد
\\
\end{longtable}
\end{center}
