\begin{center}
\section*{بخش ۱۵ - فروختن صوفیان بهیمهٔ مسافر را جهت سماع}
\label{sec:sh015}
\addcontentsline{toc}{section}{\nameref{sec:sh015}}
\begin{longtable}{l p{0.5cm} r}
صوفیی در خانقاه از ره رسید
&&
مرکب خود برد و در آخر کشید
\\
آبکش داد و علف از دست خویش
&&
نه چنان صوفی که ما گفتیم پیش
\\
احتیاطش کرد از سهو و خباط
&&
چون قضا آید چه سودست احتیاط
\\
صوفیان تقصیر بودند و فقیر
&&
کاد فقراً ان یکن کفراً یبیر
\\
ای توانگر که تو سیری هین مخند
&&
بر کژی آن فقیر دردمند
\\
از سر تقصیر آن صوفی رمه
&&
خرفروشی در گرفتند آن همه
\\
کز ضرورت هست مرداری مباح
&&
بس فسادی کز ضرورت شد صلاح
\\
هم در آن دم آن خرک بفروختند
&&
لوت آوردند و شمع افروختند
\\
ولوله افتاد اندر خانقه
&&
کامشبان لوت و سماعست و وله
\\
چند ازین صبر و ازین سه روزه چند
&&
چند ازین زنبیل و این دریوزه چند
\\
ما هم از خلقیم و جان داریم ما
&&
دولت امشب میهمان داریم ما
\\
تخم باطل را از آن می‌کاشتند
&&
کانک آن جان نیست جان پنداشتند
\\
وان مسافر نیز از راه دراز
&&
خسته بود و دید آن اقبال و ناز
\\
صوفیانش یک بیک بنواختند
&&
نرد خدمتهای خوش می‌باختند
\\
گفت چون می‌دید میلانش بوی
&&
گر طرب امشب نخواهم کرد کی
\\
لوت خوردند و سماع آغاز کرد
&&
خانقه تا سقف شد پر دود و گرد
\\
دود مطبخ گرد آن پا کوفتن
&&
ز اشتیاق و وجد جان آشوفتن
\\
گاه دست‌افشان قدم می‌کوفتند
&&
گه به سجده صفه را می‌روفتند
\\
دیر یابد صوفی آز از روزگار
&&
زان سبب صوفی بود بسیارخوار
\\
جز مگر آن صوفیی کز نور حق
&&
سیر خورد او فارغست از ننگ دق
\\
از هزاران اندکی زین صوفیند
&&
باقیان در دولت او می‌زیند
\\
چون سماع آمد ز اول تا کران
&&
مطرب آغازید یک ضرب گران
\\
خر برفت و خر برفت آغاز کرد
&&
زین حرارت جمله را انباز کرد
\\
زین حرارت پای‌کوبان تا سحر
&&
کف‌زنان خر رفت و خر رفت ای پسر
\\
از ره تقلید آن صوفی همین
&&
خر برفت آغاز کرد اندر حنین
\\
چون گذشت آن نوش و جوش و آن سماع
&&
روز گشت و جمله گفتند الوداع
\\
خانقه خالی شد و صوفی بماند
&&
گرد از رخت آن مسافر می‌فشاند
\\
رخت از حجره برون آورد او
&&
تا بخر بر بندد آن همراه‌جو
\\
تا رسد در همرهان او می‌شتافت
&&
رفت در آخر خر خود را نیافت
\\
گفت آن خادم به آبش برده است
&&
زانک خر دوش آب کمتر خورده است
\\
خادم آمد گفت صوفی خر کجاست
&&
گفت خادم ریش بین جنگی بخاست
\\
گفت من خر را به تو بسپرده‌ام
&&
من ترا بر خر موکل کرده‌ام
\\
از تو خواهم آنچ من دادم به تو
&&
باز ده آنچ فرستادم به تو
\\
بحث با توجیه کن حجت میار
&&
آنچ من بسپردمت وا پس سپار
\\
گفت پیغمبر که دستت هر چه برد
&&
بایدش در عاقبت وا پس سپرد
\\
ور نه‌ای از سرکشی راضی بدین
&&
نک من و تو خانهٔ قاضی دین
\\
گفت من مغلوب بودم صوفیان
&&
حمله آوردند و بودم بیم جان
\\
تو جگربندی میان گربگان
&&
اندر اندازی و جویی زان نشان
\\
در میان صد گرسنه گرده‌ای
&&
پیش صد سگ گربهٔ پژمرده‌ای
\\
گفت گیرم کز تو ظلما بستدند
&&
قاصد خون من مسکین شدند
\\
تو نیایی و نگویی مر مرا
&&
که خرت را می‌برند ای بی‌نوا
\\
تا خر از هر که بود من وا خرم
&&
ورنه توزیعی کنند ایشان زرم
\\
صد تدارک بود چون حاضر بدند
&&
این زمان هر یک به اقلیمی شدند
\\
من که را گیرم که را قاضی برم
&&
این قضا خود از تو آمد بر سرم
\\
چون نیایی و نگویی ای غریب
&&
پیش آمد این چنین ظلمی مهیب
\\
گفت والله آمدم من بارها
&&
تا ترا واقف کنم زین کارها
\\
تو همی‌گفتی که خر رفت ای پسر
&&
از همه گویندگان با ذوق‌تر
\\
باز می‌گشتم که او خود واقفست
&&
زین قضا راضیست مردی عارفست
\\
گفت آن را جمله می‌گفتند خوش
&&
مر مرا هم ذوق آمد گفتنش
\\
مر مرا تقلیدشان بر باد داد
&&
که دو صد لعنت بر آن تقلید باد
\\
خاصه تقلید چنین بی‌حاصلان
&&
خشم ابراهیم با بر آفلان
\\
عکس ذوق آن جماعت می‌زدی
&&
وین دلم زان عکس ذوقی می‌شدی
\\
عکس چندان باید از یاران خوش
&&
که شوی از بحر بی‌عکس آب‌کش
\\
عکس کاول زد تو آن تقلید دان
&&
چون پیاپی شد شود تحقیق آن
\\
تا نشد تحقیق از یاران مبر
&&
از صدف مگسل نگشت آن قطره در
\\
صاف خواهی چشم و عقل و سمع را
&&
بر دران تو پرده‌های طمع را
\\
زانک آن تقلید صوفی از طمع
&&
عقل او بر بست از نور و لمع
\\
طمع لوت و طمع آن ذوق و سماع
&&
مانع آمد عقل او را ز اطلاع
\\
گر طمع در آینه بر خاستی
&&
در نفاق آن آینه چون ماستی
\\
گر ترازو را طمع بودی به مال
&&
راست کی گفتی ترازو وصف حال
\\
هر نبیی گفت با قوم از صفا
&&
من نخواهم مزد پیغام از شما
\\
من دلیلم حق شما را مشتری
&&
داد حق دلالیم هر دو سری
\\
چیست مزد کار من دیدار یار
&&
گرچه خود بوبکر بخشد چل هزار
\\
چل هزار او نباشد مزد من
&&
کی بود شبه شبه در عدن
\\
یک حکایت گویمت بشنو بهوش
&&
تا بدانی که طمع شد بند گوش
\\
هر که را باشد طمع الکن شود
&&
با طمع کی چشم و دل روشن شود
\\
پیش چشم او خیال جاه و زر
&&
همچنان باشد که موی اندر بصر
\\
جز مگر مستی که از حق پر بود
&&
گرچه بدهی گنجها او حر بود
\\
هر که از دیدار برخوردار شد
&&
این جهان در چشم او مردار شد
\\
لیک آن صوفی ز مستی دور بود
&&
لاجرم در حرص او شبکور بود
\\
صد حکایت بشنود مدهوش حرص
&&
در نیاید نکته‌ای در گوش حرص
\\
\end{longtable}
\end{center}
