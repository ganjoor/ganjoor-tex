\begin{center}
\section*{بخش ۱۰۲ - تشنیع صوفیان بر آن صوفی کی پیش شیخ بسیار می‌گوید}
\label{sec:sh102}
\addcontentsline{toc}{section}{\nameref{sec:sh102}}
\begin{longtable}{l p{0.5cm} r}
صوفیان بر صوفیی شنعه زدند
&&
پیش شیخ خانقاهی آمدند
\\
شیخ را گفتند داد جان ما
&&
تو ازین صوفی بجو ای پیشوا
\\
گفت آخر چه گله‌ست ای صوفیان
&&
گفت این صوفی سه خو دارد گران
\\
در سخن بسیارگو همچون جرس
&&
در خورش افزون خورد از بیست کس
\\
ور بخسپد هست چون اصحاب کهف
&&
صوفیان کردند پیش شیخ زحف
\\
شیخ رو آورد سوی آن فقیر
&&
که ز هر حالی که هست اوساط گیر
\\
در خبر خیر الامور اوساطها
&&
نافع آمد ز اعتدال اخلاطها
\\
گر یکی خلطی فزون شد از عرض
&&
در تن مردم پدید آید مرض
\\
بر قرین خویش مفزا در صفت
&&
کان فراق آرد یقین در عاقبت
\\
نطق موسی بد بر اندازه ولیک
&&
هم فزون آمد ز گفت یار نیک
\\
آن فزونی با خضر آمد شقاق
&&
گفت رو تو مکثری هذا فراق
\\
موسیا بسیارگویی دور شو
&&
ور نه با من گنگ باش و کور شو
\\
ور نرفتی وز ستیزه شسته‌ای
&&
تو بمعنی رفته‌ای بگسسته‌ای
\\
چون حدث کردی تو ناگه در نماز
&&
گویدت سوی طهارت رو بتاز
\\
ور نرفتی خشک خنبان می‌شوی
&&
خود نمازت رفت پیشین ای غوی
\\
رو بر آنها که هم‌جفت توند
&&
عاشقان و تشنهٔ گفت توند
\\
پاسبان بر خوابناکان بر فزود
&&
ماهیان را پاسبان حاجت نبود
\\
جامه‌پوشان را نظر بر گازرست
&&
جان عریان را تجلی زیورست
\\
یا ز عریانان به یکسو باز رو
&&
یا چو ایشان فارغ از تنجامه شو
\\
ور نمی‌توانی که کل عریان شوی
&&
جامه کم کن تا ره اوسط روی
\\
\end{longtable}
\end{center}
