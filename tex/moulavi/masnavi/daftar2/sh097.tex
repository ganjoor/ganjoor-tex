\begin{center}
\section*{بخش ۹۷ - دعوی کردن آن شخص کی خدای تعالی مرا نمی‌گیرد به گناه و جواب گفتن شعیب علیه السلام مرورا}
\label{sec:sh097}
\addcontentsline{toc}{section}{\nameref{sec:sh097}}
\begin{longtable}{l p{0.5cm} r}
آن یکی می‌گفت در عهد شعیب
&&
که خدا از من بسی دیدست عیب
\\
چند دید از من گناه و جرمها
&&
وز کرم یزدان نمی‌گیرد مرا
\\
حق تعالی گفت در گوش شعیب
&&
در جواب او فصیح از راه غیب
\\
که بگفتی چند کردم من گناه
&&
وز کرم نگرفت در جرمم اله
\\
عکس می‌گویی و مقلوب ای سفیه
&&
ای رها کرده ره و بگرفته تیه
\\
چند چندت گیرم و تو بی‌خبر
&&
در سلاسل مانده‌ای پا تا بسر
\\
زنگ تو بر توت ای دیگ سیاه
&&
کرد سیمای درونت را تباه
\\
بر دلت زنگار بر زنگارها
&&
جمع شد تا کور شد ز اسرارها
\\
گر زند آن دود بر دیگ نوی
&&
آن اثر بنماید ار باشد جوی
\\
زانک هر چیزی بضد پیدا شود
&&
بر سپیدی آن سیه رسوا شود
\\
چون سیه شد دیگ پس تاثیر دود
&&
بعد ازین بر وی که بیند زود زود
\\
مرد آهنگر که او زنگی بود
&&
دود را با روش هم‌رنگی بود
\\
مرد رومی کو کند آهنگری
&&
رویش ابلق گردد از دودآوری
\\
پس بداند زود تاثیر گناه
&&
تا بنالد زود گوید ای اله
\\
چون کند اصرار و بد پیشه کند
&&
خاک اندر چشم اندیشه کند
\\
توبه نندیشد دگر شیرین شود
&&
بر دلش آن جرم تا بی‌دین شود
\\
آن پشیمانی و یا رب رفت ازو
&&
شست بر آیینه زنگ پنج تو
\\
آهنش را زنگها خوردن گرفت
&&
گوهرش را زنگ کم کردن گرفت
\\
چون نویسی کاغد اسپید بر
&&
آن نبشته خوانده آید در نظر
\\
چون نویسی بر سر بنوشته خط
&&
فهم ناید خواندنش گردد غلط
\\
کان سیاهی بر سیاهی اوفتاد
&&
هر دو خط شد کور و معنیی نداد
\\
ور سیم باره نویسی بر سرش
&&
پس سیه کردی چو جان پر شرش
\\
پس چه چاره جز پناه چاره‌گر
&&
ناامیدی مس و اکسیرش نظر
\\
ناامیدیها بپیش او نهید
&&
تا ز درد بی‌دوا بیرون جهید
\\
چون شعیب این نکته‌ها با وی بگفت
&&
زان دم جان در دل او گل شکفت
\\
جان او بشنید وحی آسمان
&&
گفت اگر بگرفت ما را کو نشان
\\
گفت یا رب دفع من می‌گوید او
&&
آن گرفتن را نشان می‌جوید او
\\
گفت ستارم نگویم رازهاش
&&
جز یکی رمز از برای ابتلاش
\\
یک نشان آنک می‌گیرم ورا
&&
آنک طاعت دارد و صوم و دعا
\\
وز نماز و از زکات و غیر آن
&&
لیک یک ذره ندارد ذوق جان
\\
می‌کند طاعات و افعال سنی
&&
لیک یک ذره ندارد چاشنی
\\
طاعتش نغزست و معنی نغز نی
&&
جوزها بسیار و در وی مغز نی
\\
ذوق باید تا دهد طاعات بر
&&
مغز باید تا دهد دانه شجر
\\
دانهٔ بی‌مغز کی گردد نهال
&&
صورت بی‌جان نباشد جز خیال
\\
\end{longtable}
\end{center}
