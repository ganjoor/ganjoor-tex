\begin{center}
\section*{بخش ۶ - نواختن مصطفی علیه‌السلام آن عرب  مهمان را و تسکین دادن او را از اضطراب و گریه و نوحه کی بر خود می‌کرد در خجالت و ندامت و آتش نومیدی}
\label{sec:sh006}
\addcontentsline{toc}{section}{\nameref{sec:sh006}}
\begin{longtable}{l p{0.5cm} r}
این سخن پایان ندارد آن عرب
&&
ماند از الطاف آن شه در عجب
\\
خواست دیوانه شدن عقلش رمید
&&
دست عقل مصطفی بازش کشید
\\
گفت این سو آ بیامد آنچنان
&&
که کسی برخیزد از خواب گران
\\
گفت این سو آ مکن هین با خود آ
&&
که ازین سو هست با تو کارها
\\
آب بر رو زد در آمد در سخن
&&
کای شهید حق شهادت عرضه کن
\\
تا گواهی بدهم و بیرون شوم
&&
سیرم از هستی در آن هامون شوم
\\
ما درین دهلیز قاضی قضا
&&
بهر دعوی الستیم و بلی
\\
که بلی گفتیم و آن را ز امتحان
&&
فعل و قول ما شهودست و بیان
\\
از چه در دهلیز قاضی ای گواه
&&
حبس باشی ده شهادت از پگاه
\\
زان بخواندندت بدین‌جا تا که تو
&&
آن گواهی بدهی و ناری عتو
\\
از لجاج خویشتن بنشسته‌ای
&&
اندرین تنگی کف و لب بسته‌ای
\\
تا بندهی آن گواهی ای شهید
&&
تو ازین دهلیز کی خواهی رهید
\\
یک زمان کارست بگزار و بتاز
&&
کار کوته را مکن بر خود دراز
\\
خواه در صد سال خواهی یک زمان
&&
این امانت واگزار و وا رهان
\\
\end{longtable}
\end{center}
