\begin{center}
\section*{بخش ۷۳ - فیما یرجی من رحمة الله تعالی معطی النعم قبل استحقاقها و  هو الذی ینزل الغیث من بعد ما قنطوا و رب بعد یورث قربا و رب  معصیة میمونة و رب سعادة تاتی من حیث یرجی النقم لیعلم ان الله یبدل سیاتهم حسنات}
\label{sec:sh073}
\addcontentsline{toc}{section}{\nameref{sec:sh073}}
\begin{longtable}{l p{0.5cm} r}
در حدیث آمد که روز رستخیز
&&
امر آید هر یکی تن را که خیز
\\
نفخ صور امرست از یزدان پاک
&&
که بر آرید ای ذرایر سر ز خاک
\\
باز آید جان هر یک در بدن
&&
هم‌چو وقت صبح هوش آید به تن
\\
جان تن خود را شناسد وقت روز
&&
در خراب خود در آید چون کنوز
\\
جسم خود بشناسد و در وی رود
&&
جان زرگر سوی درزی کی رود
\\
جان عالم سوی عالم می‌دود
&&
روح ظالم سوی ظالم می‌دود
\\
که شناسا کردشان علم اله
&&
چونک بره و میش وقت صبحگاه
\\
پای کفش خود شناسد در ظلم
&&
چون نداند جان تن خود ای صنم
\\
صبح حشر کوچکست ای مستجیر
&&
حشر اکبر را قیاس از وی بگیر
\\
آنچنان که جان بپرد سوی طین
&&
نامه پرد تا یسار و تا یمین
\\
در کفش بنهند نامهٔ بخل و جود
&&
فسق و تقوی آنچ دی خو کرده بود
\\
چون شود بیدار از خواب او سحر
&&
باز آید سوی او آن خیر و شر
\\
گر ریاضت داده باشد خوی خویش
&&
وقت بیداری همان آید به پیش
\\
ور بد او دی خام و زشت و در ضلال
&&
چون عزا نامه سیه یابد شمال
\\
ور بد او دی پاک و با تقوی و دین
&&
وقت بیداری برد در ثمین
\\
هست ما را خواب و بیداری ما
&&
بر نشان مرگ و محشر دو گوا
\\
حشر اصغر حشر اکبر را نمود
&&
مرگ اصغر مرگ اکبر را زدود
\\
لیک این نامه خیالست و نهان
&&
وآن شود در حشر اکبر بس عیان
\\
این خیال اینجا نهان پیدا اثر
&&
زین خیال آنجا برویاند صور
\\
در مهندس بین خیال خانه‌ای
&&
در دلش چون در زمینی دانه‌ای
\\
آن خیال از اندرون آید برون
&&
چون زمین که زاید از تخم درون
\\
هر خیالی کو کند در دل وطن
&&
روز محشر صورتی خواهد شدن
\\
چون خیال آن مهندس در ضمیر
&&
چون نبات اندر زمین دانه‌گیر
\\
مخلصم زین هر دو محشر قصه‌ایست
&&
مؤمنان را در بیانش حصه‌ایست
\\
چون بر آید آفتاب رستخیز
&&
بر جهند از خاک زشت و خوب تیز
\\
سوی دیوان قضا پویان شوند
&&
نقد نیک و بد به کوره می‌روند
\\
نقد نیکو شادمان و ناز ناز
&&
نقد قلب اندر زحیر و در گداز
\\
لحظه لحظه امتحانها می‌رسد
&&
سر دلها می‌نماید در جسد
\\
چون ز قندیل آب و روغن گشته فاش
&&
یا چو خاکی که بروید سرهاش
\\
از پیاز و گندنا و کوکنار
&&
سر دی پیدا کند دست بهار
\\
آن یکی سرسبز نحن المتقون
&&
وآن دگر هم‌چون بنفشه سرنگون
\\
چشمها بیرون جهید از خطر
&&
گشته ده چشمه ز بیم مستقر
\\
باز مانده دیده‌ها در انتظار
&&
تا که نامه ناید از سوی یسار
\\
چشم گردان سوی راست و سوی چپ
&&
زانک نبود بخت نامهٔ راست زپ
\\
نامه‌ای آید به دست بنده‌ای
&&
سر سیه از جرم و فسق آگنده‌ای
\\
اندرو یک خیر و یک توفیق نه
&&
جز که آزار دل صدیق نه
\\
پر ز سر تا پای زشتی و گناه
&&
تسخر و خنبک زدن بر اهل راه
\\
آن دغل‌کاری و دزدیهای او
&&
و آن چو فرعونان انا و انای او
\\
چون بخواند نامهٔ خود آن ثقیل
&&
داند او که سوی زندان شد رحیل
\\
پس روان گردد چو دزدان سوی دار
&&
جرم پیدا بسته راه اعتذار
\\
آن هزاران حجت و گفتار بد
&&
بر دهانش گشته چون مسمار بد
\\
رخت دزدی بر تن و در خانه‌اش
&&
گشته پیدا گم شده افسانه‌اش
\\
پس روان گردد به زندان سعیر
&&
که نباشد خار را ز آتش گزیر
\\
چون موکل آن ملایک پیش و پس
&&
بوده پنهان گشته پیدا چون عسس
\\
می‌برندش می‌سپوزندش به نیش
&&
که برو ای سگ به کهدانهای خویش
\\
می‌کشد پا بر سر هر راه او
&&
تا بود که بر جهد زان چاه او
\\
منتظر می‌ایستد تن می‌زند
&&
در امیدی روی وا پس می‌کند
\\
اشک می‌بارد چون باران خزان
&&
خشک اومیدی چه دارد او جز آن
\\
هر زمانی روی وا پس می‌کند
&&
رو به درگاه مقدس می‌کند
\\
پس ز حق امر آید از اقلیم نور
&&
که بگوییدش کای بطال عور
\\
انتظار چیستی ای کان شر
&&
رو چه وا پس می‌کنی ای خیره‌سر
\\
نامه‌ات آنست کت آمد به دست
&&
ای خدا آزار و ای شیطان‌پرست
\\
چون بدیدی نامهٔ کردار خویش
&&
چه نگری پس بین جزای کار خویش
\\
بیهده چه مول مولی می‌زنی
&&
در چنین چه کو امید روشنی
\\
نه ترا از روی ظاهر طاعتی
&&
نه ترا در سر و باطن نیتی
\\
نه ترا شبها مناجات و قیام
&&
نه ترا در روز پرهیز و صیام
\\
نه ترا حفظ زبان ز آزار کس
&&
نه نظر کردن به عبرت پیش و پس
\\
پیش چه بود یاد مرگ و نزع خویش
&&
پس چه باشد مردن یاران ز پیش
\\
نه ترا بر ظلم توبهٔ پر خروش
&&
ای دغا گندم‌نمای جوفروش
\\
چون ترازوی تو کژ بود و دغا
&&
راست چون جویی ترازوی جزا
\\
چونک پای چپ بدی در غدر و کاست
&&
نامه چون آید ترا در دست راست
\\
چون جزا سایه‌ست ای قد تو خم
&&
سایهٔ تو کژ فتد در پیش هم
\\
زین قبل آید خطابات درشت
&&
که شود که را از آن هم کوز پشت
\\
بنده گوید آنچ فرمودی بیان
&&
صد چنانم صد چنانم صد چنان
\\
خود تو پوشیدی بترها را به حلم
&&
ورنه می‌دانی فضیحتها به علم
\\
لیک بیرون از جهاد و فعل خویش
&&
از ورای خیر و شر و کفر و کیش
\\
وز نیاز عاجزانهٔ خویشتن
&&
وز خیال و وهم من یا صد چو من
\\
بودم اومیدی به محض لطف تو
&&
از ورای راست باشی یا عتو
\\
بخشش محضی ز لطف بی‌عوض
&&
بودم اومید ای کریم بی‌عوض
\\
رو سپس کردم بدان محض کرم
&&
سوی فعل خویشتن می‌ننگرم
\\
سوی آن اومید کردم روی خویش
&&
که وجودم داده‌ای از پیش بیش
\\
خلعت هستی بدادی رایگان
&&
من همیشه معتمد بودم بر آن
\\
چون شمارد جرم خود را و خطا
&&
محض بخشایش در آید در عطا
\\
کای ملایک باز آریدش به ما
&&
که بدستش چشم دل سوی رجا
\\
لاابالی وار آزادش کنیم
&&
وآن خطاها را همه خط بر زنیم
\\
لا ابالی مر کسی را شد مباح
&&
کش زیان نبود ز غدر و از صلاح
\\
آتشی خوش بر فروزیم از کرم
&&
تا نماند جرم و زلت بیش و کم
\\
آتشی کز شعله‌اش کمتر شرار
&&
می‌بسوزد جرم و جبر و اختیار
\\
شعله در بنگاه انسانی زنیم
&&
خار را گلزار روحانی کنیم
\\
ما فرستادیم از چرخ نهم
&&
کیمیا یصلح لکم اعمالکم
\\
خود چه باشد پیش نور مستقر
&&
کر و فر اختیار بوالبشر
\\
گوشت‌پاره آلت گویای او
&&
پیه‌پاره منظر بینای او
\\
مسمع او آن دو پاره استخوان
&&
مدرکش دو قطره خون یعنی جنان
\\
کرمکی و از قذر آکنده‌ای
&&
طمطراقی در جهان افکنده‌ای
\\
از منی بودی منی را واگذار
&&
ای ایاز آن پوستین را یاد دار
\\
\end{longtable}
\end{center}
