\begin{center}
\section*{بخش ۱۴۱ - فرمودن شاه به ایاز بار دگر کی شرح چارق و پوستین آشکارا بگو تا خواجه تاشانت از آن اشارت پند گیرد کی الدین النصیحة و موعظه یابند}
\label{sec:sh141}
\addcontentsline{toc}{section}{\nameref{sec:sh141}}
\begin{longtable}{l p{0.5cm} r}
سر چارق را بیان کن ای ایاز
&&
پیش چارق چیستت چندین نیاز
\\
تا بنوشد سنقر و بک یا رقت
&&
سر سر پوستین و چارقت
\\
ای ایاز از تو غلامی نور یافت
&&
نورت از پستی سوی گردون شتافت
\\
حسرت آزادگان شد بندگی
&&
بندگی را چون تو دادی زندگی
\\
مؤمن آن باشد که اندر جزر و مد
&&
کافر از ایمان او حسرت خورد
\\
\end{longtable}
\end{center}
