\begin{center}
\section*{بخش ۲۶ - قصهٔ آن حکیم کی دید طاوسی را کی پر زیبای خود را می‌کند به منقار و می‌انداخت و تن خود را کل و زشت می‌کرد از تعجب پرسید کی دریغت نمی‌آید گفت می‌آید اما پیش من جان از پر عزیزتر است و این پر عدوی جان منست}
\label{sec:sh026}
\addcontentsline{toc}{section}{\nameref{sec:sh026}}
\begin{longtable}{l p{0.5cm} r}
پر خود می‌کند طاوسی به دشت
&&
یک حکیمی رفته بود آنجا بگشت
\\
گفت طاوسا چنین پر سنی
&&
بی‌دریغ از بیخ چون برمی‌کنی
\\
خود دلت چون می‌دهد تا این حلل
&&
بر کنی اندازیش اندر وحل
\\
هر پرت را از عزیزی و پسند
&&
حافظان در طی مصحف می‌نهند
\\
بهر تحریک هوای سودمند
&&
از پر تو بادبیزن می‌کنند
\\
این چه ناشکری و چه بی‌باکیست
&&
تو نمی‌دانی که نقاشش کیست
\\
یا همی‌دانی و نازی می‌کنی
&&
قاصدا قلع طرازی می‌کنی
\\
ای بسا نازا که گردد آن گناه
&&
افکند مر بنده را از چشم شاه
\\
ناز کردن خوشتر آید از شکر
&&
لیک کم خایش که دارد صد خطر
\\
ایمن آبادست آن راه نیاز
&&
ترک نازش گیر و با آن ره بساز
\\
ای بسا نازآوری زد پر و بال
&&
آخر الامر آن بر آن کس شد وبال
\\
خوشی ناز ار دمی بفرازدت
&&
بیم و ترس مضمرش بگدازدت
\\
وین نیاز ار چه که لاغر می‌کند
&&
صدر را چون بدر انور می‌کند
\\
چون ز مرده زنده بیرون می‌کشد
&&
هر که مرده گشت او دارد رشد
\\
چون ز زنده مرده بیرون می‌کند
&&
نفس زنده سوی مرگی می‌تند
\\
مرده شو تا مخرج الحی الصمد
&&
زنده‌ای زین مرده بیرون آورد
\\
دی شوی بینی تو اخراج بهار
&&
لیل گردی بینی ایلاج نهار
\\
بر مکن آن پر که نپذیرد رفو
&&
روی مخراش از عزا ای خوب‌رو
\\
آنچنان رویی که چون شمس ضحاست
&&
آنچنان رخ را خراشیدن خطاست
\\
زخم ناخن بر چنان رخ کافریست
&&
که رخ مه در فراق او گریست
\\
یا نمی‌بینی تو روی خویش را
&&
ترک کن خوی لجاج اندیش را
\\
\end{longtable}
\end{center}
