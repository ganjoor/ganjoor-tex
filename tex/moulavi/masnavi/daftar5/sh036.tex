\begin{center}
\section*{بخش ۳۶ - صفت کشتن خلیل علیه‌السلام زاغ را کی آن اشارت به قمع کدام صفت بود از صفات مذمومهٔ مهلکه در مرید}
\label{sec:sh036}
\addcontentsline{toc}{section}{\nameref{sec:sh036}}
\begin{longtable}{l p{0.5cm} r}
این سخن را نیست پایان و فراغ
&&
ای خلیل حق چرا کشتی تو زاغ
\\
بهر فرمان حکمت فرمان چه بود
&&
اندکی ز اسرار آن باید نمود
\\
کاغ کاغ و نعرهٔ زاغ سیاه
&&
دایما باشد به دنیا عمرخواه
\\
هم‌چو ابلیس از خدای پاک فرد
&&
تا قیامت عمر تن درخواست کرد
\\
گفت انظرنی الی یوم الجزا
&&
کاشکی گفتی که تبنا ربنا
\\
عمر بی توبه همه جان کندنست
&&
مرگ حاضر غایب از حق بودنست
\\
عمر و مرگ این هر دو با حق خوش بود
&&
بی‌خدا آب حیات آتش بود
\\
آن هم از تاثیر لعنت بود کو
&&
در چنان حضرت همی‌شد عمرجو
\\
از خدا غیر خدا را خواستن
&&
ظن افزونیست و کلی کاستن
\\
خاصه عمری غرق در بیگانگی
&&
در حضور شیر روبه‌شانگی
\\
عمر بیشم ده که تا پس‌تر روم
&&
مهلم افزون کن که تا کمتر شوم
\\
تا که لعنت را نشانه او بود
&&
بد کسی باشد که لعنت‌جو بود
\\
عمر خوش در قرب جان پروردنست
&&
عمر زاغ از بهر سرگین خوردنست
\\
عمر بیشم ده که تا گه می‌خورم
&&
دایم اینم ده که بس بدگوهرم
\\
گرنه گه خوارست آن گنده‌دهان
&&
گویدی کز خوی زاغم وا رهان
\\
\end{longtable}
\end{center}
