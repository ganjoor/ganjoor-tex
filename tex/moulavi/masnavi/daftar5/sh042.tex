\begin{center}
\section*{بخش ۴۲ - تفسیر انی اری سبع بقرات سمان یاکلهن سبع عجاف آن گاوان لاغر را خدا به صفت شیران گرسنه آفریده  بود تا آن هفت گاو فربه را  به اشتها می‌خوردند اگر چه  آن خیالات صور گاوان در آینهٔ خواب نمودند تو معنی بگیر}
\label{sec:sh042}
\addcontentsline{toc}{section}{\nameref{sec:sh042}}
\begin{longtable}{l p{0.5cm} r}
آن عزیز مصر می‌دیدی به خواب
&&
چونک چشم غیب را شد فتح باب
\\
هفت گاو فربه بس پروری
&&
خوردشان آن هفت گاو لاغری
\\
در درون شیران بدند آن لاغران
&&
ورنه گاوان را نبودندی خوران
\\
پس بشر آمد به صورت مرد کار
&&
لیک در وی شیر پنهان مردخوار
\\
مرد را خوش وا خورد فردش کند
&&
صاف گردد دردش ار دردش کند
\\
زان یکی درد او ز جمله دردها
&&
وا رهد پا بر نهد او بر سها
\\
چند گویی هم‌چو زاغ پر نحوس
&&
ای خلیل از بهر چه کشتی خروس
\\
گفت فرمان حکمت فرمان بگو
&&
تا مسبح گردم آن را مو به مو
\\
\end{longtable}
\end{center}
