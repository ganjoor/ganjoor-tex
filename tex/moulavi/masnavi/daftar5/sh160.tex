\begin{center}
\section*{بخش ۱۶۰ - وصف ضعیف دلی و سستی صوفی سایه پرورد مجاهده ناکرده درد و داغ عشق ناچشیده به سجده و دست‌بوس عام و به حرمت نظر کردن و بانگشت نمودن ایشان کی امروز در زمانه صوفی اوست غره شده و بوهم بیمار شده هم‌چون آن معلم کی کودکان گفتند کی رنجوری و با این وهم کی من مجاهدم مرا درین ره پهلوان می‌دانند با غازیان به غزا رفته کی به ظاهر نیز هنر بنمایم در جهاد اکبر مستثناام جهاد اصغر خود پیش من چه محل دارد خیال شیر دیده و دلیریها کرده و مست این دلیری شده و روی به بیشه نهاده به قصد شیر و شیر به زبان حال گفته کی کلا سوف تعلمون ثم کلا سوف تعلمون}
\label{sec:sh160}
\addcontentsline{toc}{section}{\nameref{sec:sh160}}
\begin{longtable}{l p{0.5cm} r}
رفت یک صوفی به لشکر در غزا
&&
ناگهان آمد قطاریق و وغا
\\
ماند صوفی با بنه و خیمه و ضعاف
&&
فارسان راندند تا صف مصاف
\\
مثقلان خاک بر جا ماندند
&&
سابقون السابقون در راندند
\\
جنگها کرده مظفر آمدند
&&
باز گشته با غنایم سودمند
\\
ارمغان دادند کای صوفی تو نیز
&&
او برون انداخت نستد هیچ چیز
\\
پس بگفتندش که خشمینی چرا
&&
گفت من محروم ماندم از غزا
\\
زان تلطف هیچ صوفی خوش نشد
&&
که میان غزو خنجر کش نشد
\\
پس بگفتندش که آوردیم اسیر
&&
آن یکی را بهر کشتن تو بگیر
\\
سر ببرش تا تو هم غازی شوی
&&
اندکی خوش گشت صوفی دل‌قوی
\\
که آب را گر در وضو صد روشنیست
&&
چونک آن نبود تیمم کردنیست
\\
برد صوفی آن اسیر بسته را
&&
در پس خرگه که آرد او غزا
\\
دیر ماند آن صوفی آنجا با اسیر
&&
قوم گفتا دیر ماند آنجا فقیر
\\
کافر بسته دو دست او کشتنیست
&&
بسملش را موجب تاخیر چیست
\\
آمد آن یک در تفحص در پیش
&&
دید کافر را به بالای ویش
\\
هم‌چو نر بالای ماده وآن اسیر
&&
هم‌چو شیری خفته بالای فقیر
\\
دستها بسته همی‌خایید او
&&
از سر استیز صوفی را گلو
\\
گبر می‌خایید با دندان گلوش
&&
صوفی افتاده به زیر و رفته هوش
\\
دست‌بسته گبر و هم‌چون گربه‌ای
&&
خسته کرده حلق او بی‌حربه‌ای
\\
نیم کشتش کرده با دندان اسیر
&&
ریش او پر خون ز حلق آن فقیر
\\
هم‌چو تو کز دست نفس بسته دست
&&
هم‌چو آن صوفی شدی بی‌خویش و پست
\\
ای شده عاجز ز تلی کیش تو
&&
صد هزاران کوهها در پیش تو
\\
زین قدر خرپشته مردی از شکوه
&&
چون روی بر عقبه‌های هم‌چو کوه
\\
غازیان کشتند کافر را بتیغ
&&
هم در آن ساعت ز حمیت بی‌دریغ
\\
بر رخ صوفی زدند آب و گلاب
&&
تا به هوش آید ز بی‌خویشی و خواب
\\
چون به خویش آمد بدید آن قوم را
&&
پس بپرسیدند چون بد ماجرا
\\
الله الله این چه حالست ای عزیز
&&
این چنین بی‌هوش گشتی از چه چیز
\\
از اسیر نیم‌کشت بسته‌دست
&&
این چنین بی‌هوش افتادی و پست
\\
گفت چون قصد سرش کردم به خشم
&&
طرفه در من بنگرید آن شوخ‌چشم
\\
چشم را وا کرد پهن او سوی من
&&
چشم گردانید و شد هوشم ز تن
\\
گردش چشمش مرا لشکر نمود
&&
من ندانم گفت چون پر هول بود
\\
قصه کوته کن کزان چشم این چنین
&&
رفتم از خود اوفتادم بر زمین
\\
\end{longtable}
\end{center}
