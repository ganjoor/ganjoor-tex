\begin{center}
\section*{بخش ۱۲۹ - مثل شیطان بر در رحمان}
\label{sec:sh129}
\addcontentsline{toc}{section}{\nameref{sec:sh129}}
\begin{longtable}{l p{0.5cm} r}
حاش لله ایش شاء الله کان
&&
حاکم آمد در مکان و لامکان
\\
هیچ کس در ملک او بی‌امر او
&&
در نیفزاید سر یک تای مو
\\
ملک ملک اوست فرمان آن او
&&
کمترین سگ بر در آن شیطان او
\\
ترکمان را گر سگی باشد به در
&&
بر درش بنهاده باشد رو و سر
\\
کودکان خانه دمش می‌کشند
&&
باشد اندر دست طفلان خوارمند
\\
باز اگر بیگانه‌ای معبر کند
&&
حمله بر وی هم‌چو شیر نر کند
\\
که اشداء علی الکفار شد
&&
با ولی گل با عدو چون خار شد
\\
ز آب تتماجی که دادش ترکمان
&&
آنچنان وافی شدست و پاسبان
\\
پس سگ شیطان که حق هستش کند
&&
اندرو صد فکرت و حیلت تند
\\
آب روها را غذای او کند
&&
تا برد او آب روی نیک و بد
\\
این تتماجست آب روی عام
&&
که سگ شیطان از آن یابد طعام
\\
بر در خرگاه قدرت جان او
&&
چون نباشد حکم را قربان بگو
\\
گله گله از مرید و از مرید
&&
چون سگ باسط ذراعی بالوصید
\\
بر در کهف الوهیت چو سگ
&&
ذره ذره امرجو بر جسته رگ
\\
ای سگ دیو امتحان می‌کن که تا
&&
چون درین ره می‌نهند این خلق پا
\\
حمله می‌کن منع می‌کن می‌نگر
&&
تا که باشد ماده اندر صدق و نر
\\
پس اعوذ از بهر چه باشد چو سگ
&&
گشته باشد از ترفع تیزتگ
\\
این اعوذ آنست کای ترک خطا
&&
بانگ بر زن بر سگت ره بر گشا
\\
تا بیایم بر در خرگاه تو
&&
حاجتی خواهم ز جود و جاه تو
\\
چونک ترک از سطوت سگ عاجزست
&&
این اعوذ و این فغان ناجایزست
\\
ترک هم گوید اعوذ از سگ که من
&&
هم ز سگ در مانده‌ام اندر وطن
\\
تو نمی‌یاری برین در آمدن
&&
من نمی‌آرم ز در بیرون شدن
\\
خاک اکنون بر سر ترک و قنق
&&
که یکی سگ هر دو را بندد عنق
\\
حاش لله ترک بانگی بر زند
&&
سگ چه باشد شیر نر خون قی کند
\\
ای که خود را شیر یزدان خوانده‌ای
&&
سالها شد با سگی در مانده‌ای
\\
چون کند این سگ برای تو شکار
&&
چون شکار سگ شدستی آشکار
\\
\end{longtable}
\end{center}
