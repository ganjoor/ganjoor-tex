\begin{center}
\section*{بخش ۸۲ - بازگشتن نمامان از حجرهٔ ایاز به سوی شاه توبره تهی و خجل هم‌چون بدگمانان در حق انبیا علیهم‌السلام بر وقت ظهور برائت و پاکی ایشان کی یوم تبیض وجوه و تسود وجوه و قوله تری الذین کذبوا علی الله وجوههم مسودة}
\label{sec:sh082}
\addcontentsline{toc}{section}{\nameref{sec:sh082}}
\begin{longtable}{l p{0.5cm} r}
شاه قاصد گفت هین احوال چیست
&&
که بغلتان از زر و همیان تهیست
\\
ور نهان کردید دینار و تسو
&&
فر شادی در رخ و رخسار کو
\\
گرچه پنهان بیخ هر بیخ آورست
&&
برگ سیماهم وجوهم اخضرست
\\
آنچ خورد آن بیخ از زهر و ز قند
&&
نک منادی می‌کند شاخ بلند
\\
بیخ اگر بی‌برگ و از مایه تهیست
&&
برگهای سبز اندر شاخ چیست
\\
بر زبان بیخ گل مهری نهد
&&
شاخ دست و پا گواهی می‌دهد
\\
آن امینان جمله در عذر آمدند
&&
هم‌چو سایه پیش مه ساجد شدند
\\
عذر آن گرمی و لاف و ما و من
&&
پیش شه رفتند با تیغ و کفن
\\
از خجالت جمله انگشتان گزان
&&
هر یکی می‌گفت کای شاه جهان
\\
گر بریزی خون حلالستت حلال
&&
ور ببخشی هست انعام و نوال
\\
کرده‌ایم آنها که از ما می‌سزید
&&
تا چه فرمایی تو ای شاه مجید
\\
گر ببخشی جرم ما ای دل‌فروز
&&
شب شبیها کرده باشد روز روز
\\
گر ببخشی یافت نومیدی گشاد
&&
ورنه صد چون ما فدای شاه باد
\\
گفت شه نه این نواز و این گداز
&&
من نخواهم کرد هست آن ایاز
\\
\end{longtable}
\end{center}
