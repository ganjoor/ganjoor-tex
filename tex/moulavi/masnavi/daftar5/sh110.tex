\begin{center}
\section*{بخش ۱۱۰ - دوم بار آمدن روبه بر این خر گریخته تا باز بفریبدش}
\label{sec:sh110}
\addcontentsline{toc}{section}{\nameref{sec:sh110}}
\begin{longtable}{l p{0.5cm} r}
پس بیامد زود روبه سوی خر
&&
گفت خر از چون تو یاری الحذر
\\
ناجوامردا چه کردم من ترا
&&
که به پیش اژدها بردی مرا
\\
موجب کین تو با جانم چه بود
&&
غیر خبث جوهر تو ای عنود
\\
هم‌چو کزدم کو گزد پای فتی
&&
نارسیده از وی او را زحمتی
\\
یا چو دیوی کو عدوی جان ماست
&&
نارسیده زحمتش از ما و کاست
\\
بلک طبعا خصم جان آدمیست
&&
از هلاک آدمی در خرمیست
\\
از پی هر آدمی او نسکلد
&&
خو و طبع زشت خود او کی هلد
\\
زانک خبث ذات او بی‌موجبی
&&
هست سوی ظلم و عدوان جاذبی
\\
هر زمان خواند ترا تا خرگهی
&&
که در اندازد ترا اندر چهی
\\
که فلان جا حوض آبست و عیون
&&
تا در اندازد به حوضت سرنگون
\\
آدمی را با همه وحی و نظر
&&
اندر افکند آن لعین در شور و شر
\\
بی‌گناهی بی‌گزند سابقی
&&
که رسد او را ز آدم ناحقی
\\
گفت روبه آن طلسم سحر بود
&&
که ترا در چشم آن شیری نمود
\\
ورنه من از تو به تن مسکین‌ترم
&&
که شب و روز اندر آنجا می‌چرم
\\
گرنه زان گونه طلسمی ساختی
&&
هر شکم‌خواری بدانجا تاختی
\\
یک جهان بی‌نوا پر پیل و ارج
&&
بی‌طلسمی کی بماندی سبز مرج
\\
من ترا خود خواستم گفتن به درس
&&
که چنان هولی اگر بینی مترس
\\
لیک رفت از یاد علم آموزیت
&&
که بدم مستغرق دلسوزیت
\\
دیدمت در جوع کلب و بی‌نوا
&&
می‌شتابیدم که آیی تا دوا
\\
ورنه با تو گفتمی شرح طلسم
&&
که آن خیالی می‌نماید نیست جسم
\\
\end{longtable}
\end{center}
