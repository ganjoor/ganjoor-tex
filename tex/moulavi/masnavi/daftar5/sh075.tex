\begin{center}
\section*{بخش ۷۵ - بیان آنک آنچ بیان کرده می‌شود صورت قصه است وانگه آن صورتیست کی در خورد این صورت گیرانست و درخورد آینهٔ تصویر ایشان و از قدوسیتی کی حقیقت این قصه راست نطق را ازین تنزیل شرم می‌آید و از خجالت سر و ریش و قلم گم می‌کند و العاقل یکفیه الاشاره}
\label{sec:sh075}
\addcontentsline{toc}{section}{\nameref{sec:sh075}}
\begin{longtable}{l p{0.5cm} r}
زانک پیلم دید هندستان به خواب
&&
از خراج اومید بر ده شد خراب
\\
کیف یاتی النظم لی والقافیه
&&
بعد ما ضاعت اصول العافیه
\\
ما جنون واحد لی فی الشجون
&&
بل جنون فی جنون فی جنون
\\
ذاب جسمی من اشارات الکنی
&&
منذ عاینت البقاء فی الفنا
\\
ای ایاز از عشق تو گشتم چو موی
&&
ماندم از قصه تو قصهٔ من بگوی
\\
بس فسانهٔ عشق تو خواندم به جان
&&
تو مرا که افسانه گشتستم بخوان
\\
خود تو می‌خوانی نه من ای مقتدی
&&
من که طورم تو موسی وین صدا
\\
کوه بیچاره چه داند گفت چیست
&&
زانک موسی می‌بداند که تهیست
\\
کوه می‌داند به قدر خویشتن
&&
اندکی دارد ز لطف روح تن
\\
تن چو اصطرلاب باشد ز احتساب
&&
آیتی از روح هم‌چون آفتاب
\\
آن منجم چون نباشد چشم‌تیز
&&
شرط باشد مرد اصطرلاب‌ریز
\\
تا صطرلابی کند از بهر او
&&
تا برد از حالت خورشید بو
\\
جان کز اصطرلاب جوید او صواب
&&
چه قدر داند ز چرخ و آفتاب
\\
تو که ز اصطرب دیده بنگری
&&
درجهان دیدن یقین بس قاصری
\\
تو جهان را قدر دیده دیده‌ای
&&
کو جهان سبلت چرا مالیده‌ای
\\
عارفان را سرمه‌ای هست آن بجوی
&&
تا که دریا گردد این چشم چو جوی
\\
ذره‌ای از عقل و هوش ار با منست
&&
این چه سودا و پریشان گفتنست
\\
چونک مغز من ز عقل و هش تهیست
&&
پس گناه من درین تخلیط چیست
\\
نه گناه اوراست که عقلم ببرد
&&
عقل جملهٔ عاقلان پیشش بمرد
\\
یا مجیر العقل فتان الحجی
&&
ما سواک للعقول مرتجی
\\
ما اشتهیت العقل مذ جننتنی
&&
ما حسدت الحسن مذ زینتنی
\\
هل جنونی فی هواک مستطاب
&&
قل بلی والله یجزیک الثواب
\\
گر بتازی گوید او ور پارسی
&&
گوش و هوشی کو که در فهمش رسی
\\
بادهٔ او درخور هر هوش نیست
&&
حلقهٔ او سخرهٔ هر گوش نیست
\\
باز دیگر آمدم دیوانه‌وار
&&
رو رو ای جان زود زنجیری بیار
\\
غیر آن زنجیر زلف دلبرم
&&
گر دو صد زنجیر آری بردرم
\\
\end{longtable}
\end{center}
