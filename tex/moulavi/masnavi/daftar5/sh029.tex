\begin{center}
\section*{بخش ۲۹ - در بیان آنک ثواب عمل عاشق از حق هم حق است}
\label{sec:sh029}
\addcontentsline{toc}{section}{\nameref{sec:sh029}}
\begin{longtable}{l p{0.5cm} r}
عاشقان را شادمانی و غم اوست
&&
دست‌مزد و اجرت خدمت هم اوست
\\
غیر معشوق ار تماشایی بود
&&
عشق نبود هرزه سودایی بود
\\
عشق آن شعله‌ست کو چون بر فروخت
&&
هرچه جز معشوق باقی جمله سوخت
\\
تیغ لا در قتل غیر حق براند
&&
در نگر زان پس که بعد لا چه ماند
\\
ماند الا الله باقی جمله رفت
&&
شاد باش ای عشق شرکت‌سوز زفت
\\
خود همو بود آخرین و اولین
&&
شرک جز از دیدهٔ احول مبین
\\
ای عجب حسنی بود جز عکس آن
&&
نیست تن را جنبشی از غیر جان
\\
آن تنی را که بود در جان خلل
&&
خوش نگردد گر بگیری در عسل
\\
این کسی داند که روزی زنده بود
&&
از کف این جان جان جامی ربود
\\
وانک چشم او ندیدست آن رخان
&&
پیش او جانست این تف دخان
\\
چون ندید او عمر عبدالعزیز
&&
پیش او عادل بود حجاج نیز
\\
چون ندید او مار موسی را ثبات
&&
در حبال سحر پندارد حیات
\\
مرغ کو ناخورده است آب زلال
&&
اندر آب شور دارد پر و بال
\\
جز به ضد ضد را همی نتوان شناخت
&&
چون ببیند زخم بشناسد نواخت
\\
لاجرم دنیا مقدم آمدست
&&
تا بدانی قدر اقلیم الست
\\
چون ازینجا وا رهی آنجا روی
&&
در شکرخانهٔ ابد شاکر شوی
\\
گویی آنجا خاک را می‌بیختم
&&
زین جهان پاک می‌بگریختم
\\
ای دریغا پیش ازین بودیم اجل
&&
تا عذابم کم بدی اندر وجل
\\
\end{longtable}
\end{center}
