\begin{center}
\section*{بخش ۱۳۵ - و هم‌چنین قد جف القلم یعنی جف القلم و کتب لا یستوی الطاعة والمعصیة لا یستوی الامانة و السرقة جف القلم ان لا یستوی الشکر و الکفران جف القلم ان الله لا یضیع اجر المحسنین}
\label{sec:sh135}
\addcontentsline{toc}{section}{\nameref{sec:sh135}}
\begin{longtable}{l p{0.5cm} r}
هم‌چنین تاویل قد جف القلم
&&
بهر تحریضست بر شغل اهم
\\
پس قلم بنوشت که هر کار را
&&
لایق آن هست تاثیر و جزا
\\
کژ روی جف القلم کژ آیدت
&&
راستی آری سعادت زایدت
\\
ظلم آری مدبری جف القلم
&&
عدل آری بر خوری جف القلم
\\
چون بدزدد دست شد جف القلم
&&
خورد باده مست شد جف القلم
\\
تو روا داری روا باشد که حق
&&
هم‌چو معزول آید از حکم سبق
\\
که ز دست من برون رفتست کار
&&
پیش من چندین میا چندین مزار
\\
بلک معنی آن بود جف القلم
&&
نیست یکسان پیش من عدل و ستم
\\
فرق بنهادم میان خیر و شر
&&
فرق بنهادم ز بد هم از بتر
\\
ذره‌ای گر در تو افزونی ادب
&&
باشد از یارت بداند فضل رب
\\
قدر آن ذره ترا افزون دهد
&&
ذره چون کوهی قدم بیرون نهد
\\
پادشاهی که به پیش تخت او
&&
فرق نبود از امین و ظلم‌جو
\\
آنک می‌لرزد ز بیم رد او
&&
وانک طعنه می‌زند در جد او
\\
فرق نبود هر دو یک باشد برش
&&
شاه نبود خاک تیره بر سرش
\\
ذره‌ای گر جهد تو افزون بود
&&
در ترازوی خدا موزون بود
\\
پیش این شاهان هماره جان کنی
&&
بی‌خبر ایشان ز غدر و روشنی
\\
گفت غمازی که بد گوید ترا
&&
ضایع آرد خدمتت را سالها
\\
پیش شاهی که سمیعست و بصیر
&&
گفت غمازان نباشد جای‌گیر
\\
جمله غمازان ازو آیس شوند
&&
سوی ما آیند و افزایند پند
\\
بس جفا گویند شه را پیش ما
&&
که برو جف القلم کم کن وفا
\\
معنی جف القلم کی آن بود
&&
که جفاها با وفا یکسان بود
\\
بل جفا را هم جفا جف القلم
&&
وآن وفا را هم وفا جف القلم
\\
عفو باشد لیک کو فر امید
&&
که بود بنده ز تقوی روسپید
\\
دزد را گر عفو باشد جان برد
&&
کی وزیر و خازن مخزن شود
\\
ای امین الدین ربانی بیا
&&
کز امانت رست هر تاج و لوا
\\
پور سلطان گر برو خاین شود
&&
آن سرش از تن بدان باین شود
\\
وز غلامی هندوی آرد وفا
&&
دولت او را می‌زند طال بقا
\\
چه غلام ار بر دری سگ باوفاست
&&
در دل سالار او را صد رضاست
\\
زین چو سگ را بوسه بر پوزش دهد
&&
گر بود شیری چه پیروزش کند
\\
جز مگر دزدی که خدمتها کند
&&
صدق او بیخ جفا را بر کند
\\
چون فضیل ره‌زنی کو راست باخت
&&
زانک ده مرده به سوی توبه تاخت
\\
وآنچنان که ساحران فرعون را
&&
رو سیه کردند از صبر و وفا
\\
دست و پا دادند در جرم قود
&&
آن به صد ساله عبادت کی شود
\\
تو که پنجه سال خدمت کرده‌ای
&&
کی چنین صدقی به دست آورده‌ای
\\
\end{longtable}
\end{center}
