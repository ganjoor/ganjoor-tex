\begin{center}
\section*{بخش ۱۴۶ - حکایت ضیاء دلق کی سخت دراز بود و برادرش شیخ اسلام تاج بلخ به غایت کوتاه بالا بود و این شیخ اسلام از برادرش ضیا ننگ داشتی ضیا در آمد به درس او و همه صدور بلخ حاضر به درس او ضیا خدمتی کرد و بگذشت شیخ اسلام او را نیم قیامی کرد سرسری گفت آری سخت درازی پاره‌ای در دزد}
\label{sec:sh146}
\addcontentsline{toc}{section}{\nameref{sec:sh146}}
\begin{longtable}{l p{0.5cm} r}
آن ضیاء دلق خوش الهام بود
&&
دادر آن تاج شیخ اسلام بود
\\
تاج شیخ اسلام دار الملک بلخ
&&
بود کوته‌قد و کوچک هم‌چو فرخ
\\
گرچه فاضل بود و فحل و ذو فنون
&&
این ضیا اندر ظرافت بد فزون
\\
او بسی کوته ضیا بی‌حد دراز
&&
بود شیخ اسلام را صد کبر و ناز
\\
زین برادر عار و ننگش آمدی
&&
آن ضیا هم واعظی بد با هدی
\\
روز محفل اندر آمد آن ضیا
&&
بارگه پر قاضیان و اصفیا
\\
کرد شیخ اسلام از کبر تمام
&&
این برادر را چنین نصف القیام
\\
گفت او را بس درازی بهر مزد
&&
اندکی زان قد سروت هم بدزد
\\
پس ترا خود هوش کو یا عقل کو
&&
تا خوری می ای تو دانش را عدو
\\
روت بس زیباست نیلی هم بکش
&&
ضحکه باشد نیل بر روی حبش
\\
در تو نوری کی درآمد ای غوی
&&
تا تو بیهوشی و ظلمت‌جو شوی
\\
سایه در روزست جستن قاعده
&&
در شب ابری تو سایه‌جو شده
\\
گر حلال آمد پی قوت عوام
&&
طالبان دوست را آمد حرام
\\
عاشقان را باده خون دل بود
&&
چشمشان بر راه و بر منزل بود
\\
در چنین راه بیابان مخوف
&&
این قلاوز خرد با صد کسوف
\\
خاک در چشم قلاوزان زنی
&&
کاروان را هالک و گمره کنی
\\
نان جو حقا حرامست و فسوس
&&
نفس را در پیش نه نان سبوس
\\
دشمن راه خدا را خوار دار
&&
دزد را منبر منه بر دار دار
\\
دزد را تو دست ببریدن پسند
&&
از بریدن عاجزی دستش ببند
\\
گر نبندی دست او دست تو بست
&&
گر تو پایش نشکنی پایت شکست
\\
تو عدو را می دهی و نی‌شکر
&&
بهر چه گو زهر خند و خاک خور
\\
زد ز غیرت بر سبو سنگ و شکست
&&
او سبو انداخت و از زاهد بجست
\\
رفت پیش میر و گفتش باده کو
&&
ماجرا را گفت یک یک پیش او
\\
\end{longtable}
\end{center}
