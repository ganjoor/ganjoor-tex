\begin{center}
\section*{بخش ۱۱۶ - رفتن این شیخ در خانهٔ امیری بهر کدیه روزی چهار بار به زنبیل به اشارت غیب و عتاب کردن امیر او را بدان وقاحت و عذر  گفتن او امیر را}
\label{sec:sh116}
\addcontentsline{toc}{section}{\nameref{sec:sh116}}
\begin{longtable}{l p{0.5cm} r}
شیخ روزی چار کرت چون فقیر
&&
بهر کدیه رفت در قصر امیر
\\
در کفش زنبیل و شی لله زنان
&&
خالق جان می‌بجوید تای نان
\\
نعلهای بازگونه‌ست ای پسر
&&
عقل کلی را کند هم خیره‌سر
\\
چون امیرش دید گفتش ای وقیح
&&
گویمت چیزی منه نامم شحیح
\\
این چه سغری و چه رویست و چه کار
&&
که به روزی اندر آیی چار بار
\\
کیست اینجا شیخ اندر بند تو
&&
من ندیدم نر گدا مانند تو
\\
حرمت و آب گدایان برده‌ای
&&
این چه عباسی زشت آورده‌ای
\\
غاشیه بر دوش تو عباس دبس
&&
هیچ ملحد را مباد این نفس نحس
\\
گفت امیرا بنده فرمانم خموش
&&
ز آتشم آگه نه‌ای چندین مجوش
\\
بهر نان در خویش حرصی دیدمی
&&
اشکم نان‌خواه را بدریدمی
\\
هفت سال از سوز عشق جسم‌پز
&&
در بیابان خورده‌ام من برگ رز
\\
تا ز برگ خشک و تازه خوردنم
&&
سبز گشته بود این رنگ تنم
\\
تا تو باشی در حجاب بوالبشر
&&
سرسری در عاشقان کمتر نگر
\\
زیرکان که مویها بشکافتند
&&
علم هیات را به جان دریافتند
\\
علم نارنجات و سحر و فلسفه
&&
گرچه نشناسند حق المعرفه
\\
لیک کوشیدند تا امکان خود
&&
بر گذشتند از همه اقران خود
\\
عشق غیرت کرد و زیشان در کشید
&&
شد چنین خورشید زیشان ناپدید
\\
نور چشمی کو به روز استاره دید
&&
آفتابی چون ازو رو در کشید
\\
زین گذر کن پند من بپذیر هین
&&
عاشقان را تو به چشم عشق بین
\\
وقت نازک باشد و جان در رصد
&&
با تو نتوان گفت آن دم عذر خود
\\
فهم کن موقوف آن گفتن مباش
&&
سینه‌های عاشقان را کم خراش
\\
نه گمانی برده‌ای تو زین نشاط
&&
حزم را مگذار می‌کن احتیاط
\\
واجبست و جایزست و مستحیل
&&
این وسط را گیر در حزم ای دخیل
\\
\end{longtable}
\end{center}
