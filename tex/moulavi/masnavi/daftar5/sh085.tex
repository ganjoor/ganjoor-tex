\begin{center}
\section*{بخش ۸۵ - تعجیل فرمودن پادشاه ایاز را کی زود این حکم را به فیصل رسان و منتظر مدار و ایام بیننا مگو کی الانتظار موت الاحمر و جواب گفتن ایاز شاه را}
\label{sec:sh085}
\addcontentsline{toc}{section}{\nameref{sec:sh085}}
\begin{longtable}{l p{0.5cm} r}
گفت ای شه جملگی فرمان تراست
&&
با وجود آفتاب اختر فناست
\\
زهره کی بود یا عطارد یا شهاب
&&
کو برون آید به پیش آفتاب
\\
گر ز دلق و پوستین بگذشتمی
&&
کی چنین تخم ملامت کشتمی
\\
قفل کردن بر در حجره چه بود
&&
در میان صد خیالیی حسود
\\
دست در کرده درون آب جو
&&
هر یکی زیشان کلوخ خشک‌جو
\\
پس کلوخ خشک در جو کی بود
&&
ماهیی با آب عاصی کی شود
\\
بر من مسکین جفا دارند ظن
&&
که وفا را شرم می‌آید ز من
\\
گر نبودی زحمت نامحرمی
&&
چند حرفی از وفا واگفتمی
\\
چون جهانی شبهت و اشکال‌جوست
&&
حرف می‌رانیم ما بیرون پوست
\\
گر تو خود را بشکنی مغزی شوی
&&
داستان مغز نغزی بشنوی
\\
جوز را در پوستها آوازهاست
&&
مغز و روغن را خود آوازی کجاست
\\
دارد آوازی نه اندر خورد گوش
&&
هست آوازش نهان در گوش نوش
\\
گرنه خوش‌آوازی مغزی بود
&&
ژغژغ آواز قشری کی شنود
\\
ژغژغ آن زان تحمل می‌کنی
&&
تا که خاموشانه بر مغزی زنی
\\
چند گاهی بی‌لب و بی‌گوش شو
&&
وانگهان چون لب حریف نوش شو
\\
چند گفتی نظم و نثر و راز فاش
&&
خواجه یک روز امتحان کن گنگ باش
\\
\end{longtable}
\end{center}
