\begin{center}
\section*{بخش ۱۷ - تمثیل روشهای مختلف و همتهای گوناگون به اختلاف تحری  متحریان در وقت نماز قبله را در وقت تاریکی و تحری غواصان در  قعر بحر}
\label{sec:sh017}
\addcontentsline{toc}{section}{\nameref{sec:sh017}}
\begin{longtable}{l p{0.5cm} r}
هم‌چو قومی که تحری می‌کنند
&&
بر خیال قبله سویی می‌تنند
\\
چونک کعبه رو نماید صبحگاه
&&
کشف گردد که کی گم کردست راه
\\
یا چو غواصان به زیر قعر آب
&&
هر کسی چیزی همی‌چیند شتاب
\\
بر امید گوهر و در ثمین
&&
توبره پر می‌کنند از آن و این
\\
چون بر آیند از تگ دریای ژرف
&&
کشف گردد صاحب در شگرف
\\
وآن دگر که برد مروارید خرد
&&
وآن دگر که سنگ‌ریزه و شبه برد
\\
هکذی یبلوهم بالساهره
&&
فتنة ذات افتضاح قاهره
\\
هم‌چنین هر قوم چون پروانگان
&&
گرد شمعی پرزنان اندر جهان
\\
خویشتن بر آتشی برمی‌زنند
&&
گرد شمع خود طوافی می‌کنند
\\
بر امید آتش موسی بخت
&&
کز لهیبش سبزتر گردد درخت
\\
فضل آن آتش شنیده هر رمه
&&
هر شرر را آن گمان برده همه
\\
چون برآید صبحدم نور خلود
&&
وا نماید هر یکی چه شمع بود
\\
هر کرا پر سوخت زان شمع ظفر
&&
بدهدش آن شمع خوش هشتاد پر
\\
جوق پروانهٔ دو دیده دوخته
&&
مانده زیر شمع بد پر سوخته
\\
می‌تپد اندر پشیمانی و سوز
&&
می‌کند آه از هوای چشم‌دوز
\\
شمع او گوید که چون من سوختم
&&
کی ترا برهانم از سوز و ستم
\\
شمع او گریان که من سرسوخته
&&
چون کنم مر غیر را افروخته
\\
\end{longtable}
\end{center}
