\begin{center}
\section*{بخش ۱۳۱ - درک وجدانی چون اختیار و اضطرار و خشم و اصطبار و سیری و ناهار به جای حس است کی زرد از سرخ بداند و فرق کند و خرد از بزرگ و طلخ از شیرین و مشک از سرگین و درشت از نرم به حس مس و گرم از سرد و سوزان از شیر گرم و تر از خشک و مس دیوار از مس درخت پس منکر وجدانی منکر حس باشد و زیاده که وجدانی از حس ظاهرترست زیرا حس را توان بستن و منع کردن از احساس و بستن راه و مدخل وجدانیات را ممکن نیست و العاقل تکفیه الاشارة}
\label{sec:sh131}
\addcontentsline{toc}{section}{\nameref{sec:sh131}}
\begin{longtable}{l p{0.5cm} r}
درک وجدانی به جای حس بود
&&
هر دو در یک جدول ای عم می‌رود
\\
نغز می‌آید برو کن یا مکن
&&
امر و نهی و ماجراها و سخن
\\
این که فردا این کنم یا آن کنم
&&
این دلیل اختیارست ای صنم
\\
وان پشیمانی که خوردی زان بدی
&&
ز اختیار خویش گشتی مهتدی
\\
جمله قران امر و نهیست و وعید
&&
امر کردن سنگ مرمر را کی دید
\\
هیچ دانا هیچ عاقل این کند
&&
با کلوخ و سنگ خشم و کین کند
\\
که بگفتم کین چنین کن یا چنان
&&
چون نکردید ای موات و عاجزان
\\
عقل کی حکمی کند بر چوب و سنگ
&&
عقل کی چنگی زند بر نقش چنگ
\\
کای غلام بسته دست اشکسته‌پا
&&
نیزه برگیر و بیا سوی وغا
\\
خالقی که اختر و گردون کند
&&
امر و نهی جاهلانه چون کند
\\
احتمال عجز از حق راندی
&&
جاهل و گیج و سفیهش خواندی
\\
عجز نبود از قدر ور گر بود
&&
جاهلی از عاجزی بدتر بود
\\
ترک می‌گوید قنق را از کرم
&&
بی‌سگ و بی‌دلق آ سوی درم
\\
وز فلان سوی اندر آ هین با ادب
&&
تا سگم بندد ز تو دندان و لب
\\
تو به عکس آن کنی بر در روی
&&
لاجرم از زخم سگ خسته شوی
\\
آن‌چنان رو که غلامان رفته‌اند
&&
تا سگش گردد حلیم و مهرمند
\\
تو سگی با خود بری یا روبهی
&&
سگ بشورد از بن هر خرگهی
\\
غیر حق را گر نباشد اختیار
&&
خشم چون می‌آیدت بر جرم‌دار
\\
چون همی‌خایی تو دندان بر عدو
&&
چون همی بینی گناه و جرم ازو
\\
گر ز سقف خانه چوبی بشکند
&&
بر تو افتد سخت مجروحت کند
\\
هیچ خشمی آیدت بر چوب سقف
&&
هیچ اندر کین او باشی تو وقف
\\
که چرا بر من زد و دستم شکست
&&
او عدو و خصم جان من بدست
\\
کودکان خرد را چون می‌زنی
&&
چون بزرگان را منزه می‌کنی
\\
آنک دزدد مال تو گویی بگیر
&&
دست و پایش را ببر سازش اسیر
\\
وآنک قصد عورت تو می‌کند
&&
صد هزاران خشم از تو می‌دمد
\\
گر بیاید سیل و رخت تو برد
&&
هیچ با سیل آورد کینی خرد
\\
ور بیامد باد و دستارت ربود
&&
کی ترا با باد دل خشمی نمود
\\
خشم در تو شد بیان اختیار
&&
تا نگویی جبریانه اعتذار
\\
گر شتربان اشتری را می‌زند
&&
آن شتر قصد زننده می‌کند
\\
خشم اشتر نیست با آن چوب او
&&
پس ز مختاری شتر بردست بو
\\
هم‌چنین سگ گر برو سنگی زنی
&&
بر تو آرد حمله گردد منثنی
\\
سنگ را گر گیرد از خشم توست
&&
که تو دوری و ندارد بر تو دست
\\
عقل حیوانی چو دانست اختیار
&&
این مگو ای عقل انسان شرم دار
\\
روشنست این لیکن از طمع سحور
&&
آن خورنده چشم می‌بندد ز نور
\\
چونک کلی میل او نان خوردنیست
&&
رو به تاریکی نهد که روز نیست
\\
حرص چون خورشید را پنهان کند
&&
چه عجب گر پشت بر برهان کند
\\
\end{longtable}
\end{center}
