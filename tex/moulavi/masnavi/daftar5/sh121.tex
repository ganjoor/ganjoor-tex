\begin{center}
\section*{بخش ۱۲۱ - غالب شدن مکر روبه بر استعصام خر}
\label{sec:sh121}
\addcontentsline{toc}{section}{\nameref{sec:sh121}}
\begin{longtable}{l p{0.5cm} r}
خر بسی کوشید و او را دفع گفت
&&
لیک جوع الکلب با خر بود جفت
\\
غالب آمد حرص و صبرش بد ضعیف
&&
بس گلوها که برد عشق رغیف
\\
زان رسولی کش حقایق داد دست
&&
کاد فقر ان یکن کفر آمدست
\\
گشته بود آن خر مجاعت را اسیر
&&
گفت اگر مکرست یک ره مرده گیر
\\
زین عذاب جوع باری وا رهم
&&
گر حیات اینست من مرده بهم
\\
گر خر اول توبه و سوگند خورد
&&
عاقبت هم از خری خبطی بکرد
\\
حرص کور و احمق و نادان کند
&&
مرگ را بر احمقان آسان کند
\\
نیست آسان مرگ بر جان خران
&&
که ندارند آب جان جاودان
\\
چون ندارد جان جاوید او شقیست
&&
جرات او بر اجل از احمقیست
\\
جهد کن تا جان مخلد گردد
&&
تا به روز مرگ برگی باشدت
\\
اعتمادش نیز بر رازق نبود
&&
که بر افشاند برو از غیب جود
\\
تاکنونش فضل بی‌روزی نداشت
&&
گرچه گه‌گه بر تنش جوعی گماشت
\\
گر نباشد جوع صد رنج دگر
&&
از پی هیضه بر آرد از تو سر
\\
رنج جوع اولی بود خود زان علل
&&
هم به لطف و هم به خفت هم عمل
\\
رنج جوع از رنجها پاکیزه‌تر
&&
خاصه در جوعست صد نفع و هنر
\\
\end{longtable}
\end{center}
