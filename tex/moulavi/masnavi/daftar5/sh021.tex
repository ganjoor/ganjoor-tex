\begin{center}
\section*{بخش ۲۱ - در بیان آنک لطف حق را همه کس داند و قهر حق را همه کس داند و همه از قهر حق گریزانند و به لطف حق در آویزان اما حق تعالی قهرها را در لطف پنهان کرد و لطفها را در قهر پنهان کرد نعل بازگونه و تلبیس و مکرالله بود تا اهل تمیز و ینظر به نور الله از حالی‌بینان و ظاهربینان جدا شوند کی لیبلوکم ایکم احسن عملا}
\label{sec:sh021}
\addcontentsline{toc}{section}{\nameref{sec:sh021}}
\begin{longtable}{l p{0.5cm} r}
گفت درویشی به درویشی که تو
&&
چون بدیدی حضرت حق را بگو
\\
گفت بی‌چون دیدم اما بهر قال
&&
بازگویم مختصر آن را مثال
\\
دیدمش سوی چپ او آذری
&&
سوی دست راست جوی کوثری
\\
سوی چپش بس جهان‌سوز آتشی
&&
سوی دست راستش جوی خوشی
\\
سوی آن آتش گروهی برده دست
&&
بهر آن کوثر گروهی شاد و مست
\\
لیک لعب بازگونه بود سخت
&&
پیش پای هر شقی و نیکبخت
\\
هر که در آتش همی رفت و شرر
&&
از میان آب بر می‌کرد سر
\\
هر که سوی آب می‌رفت از میان
&&
او در آتش یافت می‌شد در زمان
\\
هر که سوی راست شد و آب زلال
&&
سر ز آتش بر زد از سوی شمال
\\
وانک شد سوی شمال آتشین
&&
سر برون می‌کرد از سوی یمین
\\
کم کسی بر سر این مضمر زدی
&&
لاجرم کم کس در آن آتش شدی
\\
جز کسی که بر سرش اقبال ریخت
&&
کو رها کرد آب و در آتش گریخت
\\
کرده ذوق نقد را معبود خلق
&&
لاجرم زین لعب مغبون بود خلق
\\
جوق‌جوق وصف صف از حرص و شتاب
&&
محترز ز آتش گریزان سوی آب
\\
لاجرم ز آتش برآوردند سر
&&
اعتبارالاعتبار ای بی‌خبر
\\
بانگ می‌زد آتش ای گیجان گول
&&
من نیم آتش منم چشمهٔ قبول
\\
چشم‌بندی کرده‌اند ای بی‌نظر
&&
در من آی و هیچ مگریز از شرر
\\
ای خلیل اینجا شرار و دود نیست
&&
جز که سحر و خدعهٔ نمرود نیست
\\
چون خلیل حق اگر فرزانه‌ای
&&
آتش آب تست و تو پروانه‌ای
\\
جان پروانه همی‌دارد ندا
&&
کای دریغا صد هزارم پر بدی
\\
تا همی سوزید ز آتش بی‌امان
&&
کوری چشم و دل نامحرمان
\\
بر من آرد رحم جاهل از خری
&&
من برو رحم آرم از بینش‌وری
\\
خاصه این آتش که جان آبهاست
&&
کار پروانه به عکس کار ماست
\\
او ببینند نور و در ناری رود
&&
دل ببیند نار و در نوری شود
\\
این چنین لعب آمد از رب جلیل
&&
تا ببینی کیست از آل خلیل
\\
آتشی را شکل آبی داده‌اند
&&
واندر آتش چشمه‌ای بگشاده‌اند
\\
ساحری صحن برنجی را به فن
&&
صحن پر کرمی کند در انجمن
\\
خانه را او پر ز کزدمها نمود
&&
از دم سحر و خود آن کزدم نبود
\\
چونک جادو می‌نماید صد چنین
&&
چون بود دستان جادوآفرین
\\
لاجرم از سحر یزدان قرن قرن
&&
اندر افتادند چون زن زیر پهن
\\
ساحرانشان بنده بودند و غلام
&&
اندر افتادند چون صعوه به دام
\\
هین بخوان قرآن ببین سحر حلال
&&
سرنگونی مکرهای کالجبال
\\
من نیم فرعون کایم سوی نیل
&&
سوی آتش می‌روم من چون خلیل
\\
نیست آتش هست آن ماء معین
&&
وآن دگر از مکر آب آتشین
\\
پس نکو گفت آن رسول خوش‌جواز
&&
ذره‌ای عقلت به از صوم و نماز
\\
زانک عقلت جوهرست این دو عرض
&&
این دو در تکمیل آن شد مفترض
\\
تا جلا باشد مر آن آیینه را
&&
که صفا آید ز طاعت سینه را
\\
لیک گر آیینه از بن فاسدست
&&
صیقل او را دیر باز آرد به دست
\\
وان گزین آیینه که خوش مغرس است
&&
اندکی صیقل گری آن را بس است
\\
\end{longtable}
\end{center}
