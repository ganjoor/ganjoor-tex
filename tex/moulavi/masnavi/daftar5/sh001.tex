\begin{center}
\section*{بخش ۱ - سر آغاز}
\label{sec:sh001}
\addcontentsline{toc}{section}{\nameref{sec:sh001}}
\begin{longtable}{l p{0.5cm} r}
شه حسام‌الدین که نور انجمست
&&
طالب آغاز سفر پنجمست
\\
این ضیاء الحق حسام الدین راد
&&
اوستادان صفا را اوستاد
\\
گر نبودی خلق محجوب و کثیف
&&
ور نبودی حلقها تنگ و ضعیف
\\
در مدیحت داد معنی دادمی
&&
غیر این منطق لبی بگشادمی
\\
لیک لقمهٔ باز آن صعوه نیست
&&
چاره اکنون آب و روغن کردنیست
\\
مدح تو حیفست با زندانیان
&&
گویم اندر مجمع روحانیان
\\
شرح تو غبنست با اهل جهان
&&
هم‌چو راز عشق دارم در نهان
\\
مدح تعریفست در تخریق حجاب
&&
فارغست از شرح و تعریف آفتاب
\\
مادح خورشید مداح خودست
&&
که دو چشمم روشن و نامرمدست
\\
ذم خورشید جهان ذم خودست
&&
که دو چشمم کور و تاریک به دست
\\
تو ببخشا بر کسی کاندر جهان
&&
شد حسود آفتاب کامران
\\
تو اندش پوشید هیچ از دیده‌ها
&&
وز طراوت دادن پوسیده‌ها
\\
یا ز نور بی‌حدش توانند کاست
&&
یا به دفع جاه او توانند خاست
\\
هر کسی کو حاسد کیهان بود
&&
آن حسد خود مرگ جاویدان بود
\\
قدر تو بگذشت از درک عقول
&&
عقل اندر شرح تو شد بوالفضول
\\
گر چه عاجز آمد این عقل از بیان
&&
عاجزانه جنبشی باید در آن
\\
ان شیئا کله لا یدرک
&&
اعلموا ان کله لا یترک
\\
گر نتانی خورد طوفان سحاب
&&
کی توان کردن بترک خورد آب
\\
راز را گر می‌نیاری در میان
&&
درکها را تازه کن از قشر آن
\\
نطقها نسبت به تو قشرست لیک
&&
پیش دیگر فهمها مغزست نیک
\\
آسمان نسبت به عرش آمد فرود
&&
ورنه بس عالیست سوی خاک‌تود
\\
من بگویم وصف تو تا ره برند
&&
پیش از آن کز فوت آن حسرت خورند
\\
نور حقی و به حق جذاب جان
&&
خلق در ظلمات وهم‌اند و گمان
\\
شرط تعظیمست تا این نور خوش
&&
گردد این بی‌دیدگان را سرمه‌کش
\\
نور یابد مستعد تیزگوش
&&
کو نباشد عاشق ظلمت چو موش
\\
سست‌چشمانی که شب جولان کنند
&&
کی طواف مشعلهٔ ایمان کنند
\\
نکته‌های مشکل باریک شد
&&
بند طبعی که ز دین تاریک شد
\\
تا بر آراید هنر را تار و پود
&&
چشم در خورشید نتواند گشود
\\
هم‌چو نخلی برنیارد شاخها
&&
کرده موشانه زمین سوراخها
\\
چار وصفست این بشر را دل‌فشار
&&
چارمیخ عقل گشته این چهار
\\
\end{longtable}
\end{center}
