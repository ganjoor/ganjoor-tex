\begin{center}
\section*{بخش ۱۰۹ - در بیان آنک نقض عهد و توبه موجب بلا بود بلک موجب  مسخ است چنانک در حق اصحاب سبت و در حق اصحاب مایدهٔ  عیسی و جعل منهم القردة و الخنازیر و اندرین امت مسخ دل باشد و به قیامت تن را صورت دل دهند نعوذ بالله}
\label{sec:sh109}
\addcontentsline{toc}{section}{\nameref{sec:sh109}}
\begin{longtable}{l p{0.5cm} r}
نقض میثاق و شکست توبه‌ها
&&
موجب لعنت شود در انتها
\\
نقض توبه و عهد آن اصحاب سبت
&&
موجب مسخ آمد و اهلاک و مقت
\\
پس خدا آن قوم را بوزینه کرد
&&
چونک عهد حق شکستند از نبرد
\\
اندرین امت نبد مسخ بدن
&&
لیک مسخ دل بود ای بوالفطن
\\
چون دل بوزینه گردد آن دلش
&&
از دل بوزینه شد خوار آن گلش
\\
گر هنر بودی دلش را ز اختبار
&&
خوار کی بودی ز صورت آن حمار
\\
آن سگ اصحاب خوش بد سیرتش
&&
هیچ بودش منقصت زان صورتش
\\
مسخ ظاهر بود اهل سبت را
&&
تا ببیند خلق ظاهر کبت را
\\
از ره سر صد هزاران دگر
&&
گشته از توبه شکستن خوک و خر
\\
\end{longtable}
\end{center}
