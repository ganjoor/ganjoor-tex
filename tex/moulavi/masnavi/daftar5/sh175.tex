\begin{center}
\section*{بخش ۱۷۵ - تشنیع زدن امرا بر ایاز کی چرا شکستش و جواب دادن ایاز ایشان را}
\label{sec:sh175}
\addcontentsline{toc}{section}{\nameref{sec:sh175}}
\begin{longtable}{l p{0.5cm} r}
گفت ایاز ای مهتران نامور
&&
امر شه بهتر به قیمت یا گهر
\\
امر سلطان به بود پیش شما
&&
یا که این نیکو گهر بهر خدا
\\
ای نظرتان بر گهر بر شاه نه
&&
قبله‌تان غولست و جادهٔ راه نه
\\
من ز شه بر می‌نگردانم بصر
&&
من چو مشرک روی نارم با حجر
\\
بی‌گهر جانی که رنگین سنگ را
&&
برگزیند پس نهد شاه مرا
\\
پشت سوی لعبت گل‌رنگ کن
&&
عقل در رنگ‌آورنده دنگ کن
\\
اندر آ در جو سبو بر سنگ زن
&&
آتش اندر بو و اندر رنگ زن
\\
گر نه‌ای در راه دین از ره‌زنان
&&
رنگ و بو مپرست مانند زنان
\\
سر فرود انداختند آن مهتران
&&
عذرجویان گشه زان نسیان به جان
\\
از دل هر یک دو صد آه آن زمان
&&
هم‌چو دودی می‌شدی تا آسمان
\\
کرد اشارت شه به جلاد کهن
&&
که ز صدرم این خسان را دور کن
\\
این خسان چه لایق صدر من‌اند
&&
کز پی سنگ امر ما را بشکنند
\\
امر ما پیش چنین اهل فساد
&&
بهر رنگین سنگ شد خوار و کساد
\\
\end{longtable}
\end{center}
