\begin{center}
\section*{بخش ۱۷۳ - دادن شاه گوهر را میان دیوان و مجمع به دست وزیر کی این چند ارزد و مبالغه کردن وزیر در قیمت او و فرمودن شاه او را کی اکنون این را بشکن و گفت وزیر کی این را چون بشکنم الی آخر القصه}
\label{sec:sh173}
\addcontentsline{toc}{section}{\nameref{sec:sh173}}
\begin{longtable}{l p{0.5cm} r}
شاه روزی جانب دیوان شتافت
&&
جمله ارکان را در آن دیوان بیافت
\\
گوهری بیرون کشید او مستنیر
&&
پس نهادش زود در کف وزیر
\\
گفت چونست و چه ارزد این گهر
&&
گفت به ارزد ز صد خروار زر
\\
گفت بشکن گفت چونش بشکنم
&&
نیک‌خواه مخزن و مالت منم
\\
چون روا دارم که مثل این گهر
&&
که نیاید در بها گردد هدر
\\
گفت شاباش و بدادش خلعتی
&&
گوهر از وی بستد آن شاه و فتی
\\
کرد ایثار وزیر آن شاه جود
&&
هر لباس و حله کو پوشیده بود
\\
ساعتیشان کرد مشغول سخن
&&
از قضیه تازه و راز کهن
\\
بعد از آن دادش به دست حاجبی
&&
که چه ارزد این به پیش طالبی
\\
گفت ارزد این به نیمهٔ مملکت
&&
کش نگهدارا خدا از مهلکت
\\
گفت بشکن گفت ای خورشیدتیغ
&&
بس دریغست این شکستن را دریغ
\\
قیمتش بگذار بین تاب و لمع
&&
که شدست این نور روز او را تبع
\\
دست کی جنبد مرا در کسر او
&&
که خزینهٔ شاه را باشم عدو
\\
شاه خلعت داد ادرارش فزود
&&
پس دهان در مدح عقل او گشود
\\
بعد یک ساعت به دست میر داد
&&
در را آن امتحان کن باز داد
\\
او همین گفت و همه میران همین
&&
هر یکی را خلعتی داد او ثمین
\\
جامگیهاشان همی‌افزود شاه
&&
آن خسیسان را ببرد از ره به جاه
\\
این چنین گفتند پنجه شصت امیر
&&
جمله یک یک هم به تقلید وزیر
\\
گرچه تقلدست استون جهان
&&
هست رسوا هر مقلد ز امتحان
\\
\end{longtable}
\end{center}
