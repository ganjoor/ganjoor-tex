\begin{center}
\section*{بخش ۱۵۷ - تمثیل فکر هر روزینه کی اندر دل آید به مهمان نو کی از اول روز در خانه فرود آید و فضیلت مهمان‌نوازی و ناز مهمان کشیدن و تحکم و بدخویی کند به خداوند خانه}
\label{sec:sh157}
\addcontentsline{toc}{section}{\nameref{sec:sh157}}
\begin{longtable}{l p{0.5cm} r}
هر دمی فکری چو مهمان عزیز
&&
آید اندر سینه‌ات هر روز نیز
\\
فکر را ای جان به جای شخص دان
&&
زانک شخص از فکر دارد قدر و جان
\\
فکر غم گر راه شادی می‌زند
&&
کارسازیهای شادی می‌کند
\\
خانه می‌روبد به تندی او ز غیر
&&
تا در آید شادی نو ز اصل خیر
\\
می‌فشاند برگ زرد از شاخ دل
&&
تا بروید برگ سبز متصل
\\
می‌کند بیخ سرور کهنه را
&&
تا خرامد ذوق نو از ما ورا
\\
غم کند بیخ کژ پوسیده را
&&
تا نماید بیخ رو پوشیده را
\\
غم ز دل هر چه بریزد یا برد
&&
در عوض حقا که بهتر آورد
\\
خاصه آن را که یقینش باشد این
&&
که بود غم بندهٔ اهل یقین
\\
گر ترش‌رویی نیارد ابر و برق
&&
رز بسوزد از تبسمهای شرق
\\
سعد و نحس اندر دلت مهمان شود
&&
چون ستاره خانه خانه می‌رود
\\
آن زمان که او مقیم برج تست
&&
باش هم‌چون طالعش شیرین و چست
\\
تا که با مه چون شود او متصل
&&
شکر گوید از تو با سلطان دل
\\
هفت سال ایوب با صبر و رضا
&&
در بلا خوش بود با ضیف خدا
\\
تا چو وا گردد بلای سخت‌رو
&&
پیش حق گوید به صدگون شکر او
\\
کز محبت با من محبوب کش
&&
رو نکرد ایوب یک لحظه ترش
\\
از وفا و خجلت علم خدا
&&
بود چون شیر و عسل او با بلا
\\
فکر در سینه در آید نو به نو
&&
خند خندان پیش او تو باز رو
\\
که اعذنی خالقی من شره
&&
لا تحرمنی انل من بره
\\
رب اوزعنی لشکر ما اری
&&
لا تعقب حسرة لی ان مضی
\\
آن ضمیر رو ترش را پاس‌دار
&&
آن ترش را چون شکر شیرین شمار
\\
ابر را گر هست ظاهر رو ترش
&&
گلشن آرنده‌ست ابر و شوره‌کش
\\
فکر غم را تو مثال ابر دان
&&
با ترش تو رو ترش کم کن چنان
\\
بوک آن گوهر به دست او بود
&&
جهد کن تا از تو او راضی رود
\\
ور نباشد گوهر و نبود غنی
&&
عادت شیرین خود افزون کنی
\\
جای دیگر سود دارد عادتت
&&
ناگهان روزی بر آید حاجتت
\\
فکرتی کز شادیت مانع شود
&&
آن به امر و حکمت صانع شود
\\
تو مخوان دو چار دانگش ای جوان
&&
بوک نجمی باشد و صاحب‌قران
\\
تو مگو فرعیست او را اصل گیر
&&
تا بوی پیوسته بر مقصود چیر
\\
ور تو آن را فرع گیری و مضر
&&
چشم تو در اصل باشد منتظر
\\
زهر آمد انتظارش اندر چشش
&&
دایما در مرگ باشی زان روش
\\
اصل دان آن را بگیرش در کنار
&&
بازره دایم ز مرگ انتظار
\\
\end{longtable}
\end{center}
