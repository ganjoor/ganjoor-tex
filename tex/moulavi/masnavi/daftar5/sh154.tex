\begin{center}
\section*{بخش ۱۵۴ - دگربار استدعاء شاه از ایاز کی تاویل کار خود بگو و مشکل منکران را و طاعنان را حل کن کی ایشان را در آن التباس رها کردن مروت نیست}
\label{sec:sh154}
\addcontentsline{toc}{section}{\nameref{sec:sh154}}
\begin{longtable}{l p{0.5cm} r}
این سخن از حد و اندازه‌ست بیش
&&
ای ایاز اکنون بگو احوال خویش
\\
هست احوال تو از کان نوی
&&
تو بدین احوال کی راضی شوی
\\
هین حکایت کن از آن احوال خوش
&&
خاک بر احوال و درس پنج و شش
\\
حال باطن گر نمی‌آید بگفت
&&
حال ظاهر گویمت در طاق وجفت
\\
که ز لطف یار تلخیهای مات
&&
گشت بر جان خوشتر از شکرنبات
\\
زان نبات ار گرد در دریا رود
&&
تلخی دریا همه شیرین شود
\\
صدهزار احوال آمد هم‌چنین
&&
باز سوی غیب رفتند ای امین
\\
حال هر روزی بدی مانند نی
&&
هم‌چو جو اندر روش کش بند نی
\\
شادی هر روز از نوعی دگر
&&
فکرت هر روز را دیگر اثر
\\
\end{longtable}
\end{center}
