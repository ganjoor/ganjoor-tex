\begin{center}
\section*{بخش ۳۹ - قصهٔ محبوس شدن آن آهوبچه در آخر خران و طعنهٔ آن خران ببر آن غریب گاه به جنگ و گاه به تسخر و مبتلی گشتن او به کاه خشک کی غذای او نیست و این صفت بندهٔ خاص خداست میان اهل دنیا و اهل هوا  و شهوت کی الاسلام بدا غریبا و سیعود غریبا فطوبی للغرباء  صدق رسول الله}
\label{sec:sh039}
\addcontentsline{toc}{section}{\nameref{sec:sh039}}
\begin{longtable}{l p{0.5cm} r}
آهوی را کرد صیادی شکار
&&
اندر آخر کردش آن بی‌زینهار
\\
آخری را پر ز گاوان و خران
&&
حبس آهو کرد چون استمگران
\\
آهو از وحشت به هر سو می‌گریخت
&&
او به پیش آن خران شب کاه ریخت
\\
از مجاعت و اشتها هر گاو و خر
&&
کاه را می‌خورد خوشتر از شکر
\\
گاه آهو می‌رمید از سو به سو
&&
گه ز دود و گرد که می‌تافت رو
\\
هرکرا با ضد خود بگذاشتند
&&
آن عقوبت را چو مرگ انگاشتند
\\
تا سلیمان گفت که آن هدهد اگر
&&
هجر را عذری نگوید معتبر
\\
بکشمش یا خود دهم او را عذاب
&&
یک عذاب سخت بیرون از حساب
\\
هان کدامست آن عذاب این معتمد
&&
در قفس بودن به غیر جنس خود
\\
زین بدن اندر عذابی ای بشر
&&
مرغ روحت بسته با جنسی دگر
\\
روح بازست و طبایع زاغها
&&
دارد از زاغان و چغدان داغها
\\
او بمانده در میانشان زارزار
&&
هم‌چو بوبکری به شهر سبزوار
\\
\end{longtable}
\end{center}
