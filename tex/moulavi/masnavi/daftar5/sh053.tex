\begin{center}
\section*{بخش ۵۳ - در بیان آنک مرد بدکار چون متمکن شود در بدکاری و اثر دولت نیکوکاران ببیند شیطان شود و مانع خیر گردد از حسد هم‌چون شیطان کی خرمن سوخته همه را خرمن سوخته خواهد ارایت الذی  ینهی عبدا اذا صلی}
\label{sec:sh053}
\addcontentsline{toc}{section}{\nameref{sec:sh053}}
\begin{longtable}{l p{0.5cm} r}
وافیان را چون ببینی کرده سود
&&
تو چو شیطانی شوی آنجا حسود
\\
هرکرا باشد مزاج و طبع سست
&&
او نخواهد هیچ کس را تن‌درست
\\
گر نخواهی رشک ابلیسی بیا
&&
از در دعوی به درگاه وفا
\\
چون وفاات نیست باری دم مزن
&&
که سخن دعویست اغلب ما و من
\\
این سخن در سینه دخل مغزهاست
&&
در خموشی مغز جان را صد نماست
\\
چون بیامد در زبان شد خرج مغز
&&
خرج کم کن تا بماند مغز نغز
\\
مرد کم گوینده را فکرست زفت
&&
قشر گفتن چون فزون شد مغز رفت
\\
پوست افزون بود لاغر بود مغز
&&
پوست لاغر شد چو کامل گشت و نغز
\\
بنگر این هر سه ز خامی رسته را
&&
جوز را و لوز را و پسته را
\\
هر که او عصیان کند شیطان شود
&&
که حسود دولت نیکان شود
\\
چونک در عهد خدا کردی وفا
&&
از کرم عهدت نگه دارد خدا
\\
از وفای حق تو بسته دیده‌ای
&&
اذکروا اذکرکم نشنیده‌ای
\\
گوش نه اوفوا به عهدی گوش‌دار
&&
تا که اوفی عهدکم آید ز یار
\\
عهد و قرض ما چه باشد ای حزین
&&
هم‌چو دانهٔ خشک کشتن در زمین
\\
نه زمین را زان فروغ و لمتری
&&
نه خداوند زمین را توانگری
\\
جز اشارت که ازین می‌بایدم
&&
که تو دادی اصل این را از عدم
\\
خوردم و دانه بیاوردم نشان
&&
که ازین نعمت به سوی ما کشان
\\
پس دعای خشک هل ای نیک‌بخت
&&
که فشاند دانه می‌خواهد درخت
\\
گر نداری دانه ایزد زان دعا
&&
بخشدت نخلی که نعم ما سعی
\\
هم‌چو مریم درد بودش دانه نی
&&
سبز کرد آن نخل را صاحب‌فنی
\\
زانک وافی بود آن خاتون راد
&&
بی‌مرادش داد یزدان صد مراد
\\
آن جماعت را که وافی بوده‌اند
&&
بر همه اصنافشان افزوده‌اند
\\
گشت دریاها مسخرشان و کوه
&&
چار عنصر نیز بندهٔ آن گروه
\\
این خود اکرامیست از بهر نشان
&&
تا ببینند اهل انکار آن عیان
\\
آن کرامتهای پنهانشان که آن
&&
در نیاید در حواس و در بیان
\\
کار آن دارد خود آن باشد ابد
&&
دایما نه منقطع نه مسترد
\\
\end{longtable}
\end{center}
