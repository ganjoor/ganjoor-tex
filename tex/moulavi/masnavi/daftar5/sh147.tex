\begin{center}
\section*{بخش ۱۴۷ - رفتن امیر خشم‌آلود برای گوشمال زاهد}
\label{sec:sh147}
\addcontentsline{toc}{section}{\nameref{sec:sh147}}
\begin{longtable}{l p{0.5cm} r}
میر چون آتش شد و برجست راست
&&
گفت بنما خانهٔ زاهد کجاست
\\
تا بدین گرز گران کوبم سرش
&&
آن سر بی‌دانش مادرغرش
\\
او چه داند امر معروف از سگی
&&
طالب معروفی است و شهرگی
\\
تا بدین سالوس خود را جا کند
&&
تا به چیزی خویشتن پیدا کند
\\
کو ندارد خود هنر الا همان
&&
که تسلس می‌کند با این و آن
\\
او اگر دیوانه است و فتنه‌کاو
&&
داروی دیوانه باشد کیر گاو
\\
تا که شیطان از سرش بیرون رود
&&
بی‌لت خربندگان خر چون رود
\\
میر بیرون جست دبوسی بدست
&&
نیم شب آمد به زاهد نیم‌مست
\\
خواست کشتن مرد زاهد را ز خشم
&&
مرد زاهد گشت پنهان زیر پشم
\\
مرد زاهد می‌شنید از میر آن
&&
زیر پشم آن رسن‌تابان نهان
\\
گفت در رو گفتن زشتی مرد
&&
آینه تاند که رو را سخت کرد
\\
روی باید آینه‌وار آهنین
&&
تات گوید روی زشت خود ببین
\\
\end{longtable}
\end{center}
