\begin{center}
\section*{بخش ۱۷۷ - تفسیر گفتن ساحران فرعون را در وقت سیاست با او کی لا ضیر انا الی ربنا منقلبون}
\label{sec:sh177}
\addcontentsline{toc}{section}{\nameref{sec:sh177}}
\begin{longtable}{l p{0.5cm} r}
نعرهٔ لا ضیر بشنید آسمان
&&
چرخ گویی شد پی آن صولجان
\\
ضربت فرعون ما را نیست ضیر
&&
لطف حق غالب بود بر قهر غیر
\\
گر بدانی سر ما را ای مضل
&&
می‌رهانیمان ز رنج ای کوردل
\\
هین بیا زین سو ببین کین ارغنون
&&
می‌زند یا لیت قومی یعلمون
\\
داد ما را داد حق فرعونیی
&&
نه چو فرعونیت و ملکت فانیی
\\
سر بر آر و ملک بین زنده و جلیل
&&
ای شده غره به مصر و رود نیل
\\
گر تو ترک این نجس خرقه کنی
&&
نیل را در نیل جان غرقه کنی
\\
هین بدار از مصر ای فرعون دست
&&
در میان مصر جان صد مصر هست
\\
تو انا رب همی‌گویی به عام
&&
غافل از ماهیت این هر دو نام
\\
رب بر مربوب کی لرزان بود
&&
کی انادان بند جسم و جان بود
\\
نک انا ماییم رسته از انا
&&
از انای پر بلای پر عنا
\\
آن انایی بر تو ای سگ شوم بود
&&
در حق ما دولت محتوم بود
\\
گر نبودیت این انایی کینه‌کش
&&
کی زدی بر ما چنین اقبال خوش
\\
شکر آنک از دار فانی می‌رهیم
&&
بر سر این دار پندت می‌دهیم
\\
دار قتل ما براق رحلتست
&&
دار ملک تو غرور و غفلتست
\\
این حیاتی خفیه در نقش ممات
&&
وان مماتی خفیه در قشر حیات
\\
می‌نماید نور نار و نار نور
&&
ورنه دنیا کی بدی دارالغرور
\\
هین مکن تعجیل اول نیست شو
&&
چون غروب آری بر آ از شرق ضو
\\
از انایی ازل دل دنگ شد
&&
این انایی سرد گشت و ننگ شد
\\
زان انای بی‌انا خوش گشت جان
&&
شد جهان او از انایی جهان
\\
از انا چون رست اکنون شد انا
&&
آفرینها بر انای بی عنا
\\
کو گریزان و انایی در پیش
&&
می‌دود چون دید وی را بی ویش
\\
طالب اویی نگردد طالبت
&&
چون بمردی طالبت شد مطلبت
\\
زنده‌ای کی مرده‌شو شوید ترا
&&
طالبی کی مطلبت جوید ترا
\\
اندرین بحث ار خرده ره‌بین بدی
&&
فخر رازی رازدان دین بدی
\\
لیک چون من لمن یذق لم یدر بود
&&
عقل و تخییلات او حیرت فزود
\\
کی شود کشف از تفکر این انا
&&
آن انا مکشوف شد بعد از فنا
\\
می‌فتد این عقلها در افتقاد
&&
در مغا کی حلول و اتحاد
\\
ای ایاز گشته فانی ز اقتراب
&&
هم‌چو اختر در شعاع آفتاب
\\
بلک چون نطفه مبدل تو به تن
&&
نه از حلول و اتحادی مفتتن
\\
عفو کن ای عفو در صندوق تو
&&
سابق لطفی همه مسبوق تو
\\
من کی باشم که بگویم عفو کن
&&
ای تو سلطان و خلاصهٔ امر کن
\\
من کی باشم که بوم من با منت
&&
ای گرفته جمله منها دامنت
\\
\end{longtable}
\end{center}
