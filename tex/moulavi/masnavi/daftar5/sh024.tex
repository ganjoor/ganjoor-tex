\begin{center}
\section*{بخش ۲۴ - در بیان آنک هیچ چشم بدی آدمی را چنان مهلک نیست کی چشم پسند  خویشتن مگر کی چشم او مبدل شده باشد به نور حق که بی یسمع و بی یبصر و خویشتن او بی‌خویشتن شده}
\label{sec:sh024}
\addcontentsline{toc}{section}{\nameref{sec:sh024}}
\begin{longtable}{l p{0.5cm} r}
پر طاوست مبین و پای بین
&&
تا که سؤ العین نگشاید کمین
\\
که بلغزد کوه از چشم بدان
&&
یزلقونک از نبی بر خوان بدان
\\
احمد چون کوه لغزید از نظر
&&
در میان راه بی‌گل بی‌مطر
\\
در عجب درماند کین لغزش ز چیست
&&
من نپندارم که این حالت تهیست
\\
تا بیامد آیت و آگاه کرد
&&
کان ز چشم بد رسیدت وز نبرد
\\
گر بدی غیر تو در دم لا شدی
&&
صید چشم و سخرهٔ افنا شدی
\\
لیک آمد عصمتی دامن‌کشان
&&
وین که لغزیدی بد از بهر نشان
\\
عبرتی گیر اندر آن که کن نگاه
&&
برگ خود عرضه مکن ای کم ز کاه
\\
\end{longtable}
\end{center}
