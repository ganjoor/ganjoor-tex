\begin{center}
\section*{بخش ۵۰ - در معنی این بیت «گر راه روی راه برت بگشایند   ور نیست شوی بهستیت بگرایند»}
\label{sec:sh050}
\addcontentsline{toc}{section}{\nameref{sec:sh050}}
\begin{longtable}{l p{0.5cm} r}
گر زلیخا بست درها هر طرف
&&
یافت یوسف هم ز جنبش منصرف
\\
باز شد قفل و در و شد ره پدید
&&
چون توکل کرد یوسف برجهید
\\
گر چه رخنه نیست عالم را پدید
&&
خیره یوسف‌وار می‌باید دوید
\\
تا گشاید قفل و در پیدا شود
&&
سوی بی‌جایی شما را جا شود
\\
آمدی اندر جهان ای ممتحن
&&
هیچ می‌بینی طریق آمدن
\\
تو ز جایی آمدی وز موطنی
&&
آمدن را راه دانی هیچ نی
\\
گر ندانی تا نگویی راه نیست
&&
زین ره بی‌راهه ما را رفتنیست
\\
می‌روی در خواب شادان چپ و راست
&&
هیچ دانی راه آن میدان کجاست
\\
تو ببند آن چشم و خود تسلیم کن
&&
خویش را بینی در آن شهر کهن
\\
چشم چون بندی که صد چشم خمار
&&
بند چشم تست این سو از غرار
\\
چارچشمی تو ز عشق مشتری
&&
بر امید مهتری و سروری
\\
ور بخسپی مشتری بینی به خواب
&&
چغد بد کی خواب بیند جز خراب
\\
مشتری خواهی بهر دم پیچ پیچ
&&
تو چه داری که فروشی هیچ هیچ
\\
گر دلت را نان بدی یا چاشتی
&&
از خریداران فراغت داشتی
\\
\end{longtable}
\end{center}
