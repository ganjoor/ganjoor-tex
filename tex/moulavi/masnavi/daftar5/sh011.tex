\begin{center}
\section*{بخش ۱۱ - در بیان آنک نور خود از اندرون شخص منور بی‌آنک فعلی و قولی بیان کند گواهی دهد بر نور وی در بیان آنک آن‌نور خود را از اندرون سر عارف ظاهر کند بر خلقان بی‌فعل عارف و بی‌قول عارف افزون از آنک به قول و فعل او ظاهر شود چنانک آفتاب بلند شود بانگ خروس و اعلام مذن و علامات دیگر حاجت نیاید}
\label{sec:sh011}
\addcontentsline{toc}{section}{\nameref{sec:sh011}}
\begin{longtable}{l p{0.5cm} r}
لیک نور سالکی کز حد گذشت
&&
نور او پر شد بیابانها و دشت
\\
شاهدی‌اش فارغ آمد از شهود
&&
وز تکلفها و جانبازی و جود
\\
نور آن گوهر چو بیرون تافتست
&&
زین تسلسها فراغت یافتست
\\
پس مجو از وی گواه فعل و گفت
&&
که ازو هر دو جهان چون گل شکفت
\\
این گواهی چیست اظهار نهان
&&
خواه قول و خواه فعل و غیر آن
\\
که عرض اظهار سر جوهرست
&&
وصف باقی وین عرض بر معبرست
\\
این نشان زر نماند بر محک
&&
زر بماند نیک نام و بی ز شک
\\
این صلات و این جهاد و این صیام
&&
هم نماند جان بماند نیک‌نام
\\
جان چنین افعال و اقوالی نمود
&&
بر محک امر جوهر را بسود
\\
که اعتقادم راستست اینک گواه
&&
لیک هست اندر گواهان اشتباه
\\
تزکیه باید گواهان را بدان
&&
تزکیش صدقی که موقوفی بدان
\\
حفظ لفظ اندر گواه قولیست
&&
حفظ عهد اندر گواه فعلیست
\\
گر گواه قول کژ گوید ردست
&&
ور گواه فعل کژ پوید ردست
\\
قول و فعل بی‌تناقض بایدت
&&
تا قبول اندر زمان بیش آیدت
\\
سعیکم شتی تناقض اندرید
&&
روز می‌دوزید شب بر می‌درید
\\
پس گواهی با تناقض کی شنود
&&
یا مگر حلمی کند از لطف خود
\\
فعل و قول اظهار سرست و ضمیر
&&
هر دو پیدا می‌کند سر ستیر
\\
چون گواهت تزکیه شد شد قبول
&&
ورنه محبوس است اندر مول مول
\\
تا تو بستیزی ستیزند ای حرون
&&
فانتظرهم انهم منتظرون
\\
\end{longtable}
\end{center}
