\begin{center}
\section*{بخش ۱۱۸ - اشارت آمدن از غیب به شیخ کی این دو سال به فرمان ما بستدی و بدادی بعد ازین بده و مستان دست در زیر حصیر می‌کن کی آن را چون  انبان بوهریره کردیم در حق تو هر چه خواهی بیابی تا یقین شود عالمیان  را کی ورای این عالمیست کی خاک به کف گیری زر شود مرده درو آید  زنده شود نحس اکبر در وی آید سعد اکبر شود کفر درو آید ایمان گردد  زهر درو آید تریاق شود نه داخل این  عالمست و نه خارج این عالم نه  تحت و نه فوق نه متصل نه منفصل بی‌چون و بی چگونه هر دم ازو  هزاران اثر و نمونه ظاهر می‌شود چنانک صنعت دست با صورت دست و  غمزهٔ چشم با صورت چشم و فصاحت زبان با صورت زبان نه داخلست  و نه خارج او نه متصل و نه منفصل والعاقل تکفیه الاشارة}
\label{sec:sh118}
\addcontentsline{toc}{section}{\nameref{sec:sh118}}
\begin{longtable}{l p{0.5cm} r}
تا دو سال این کار کرد آن مرد کار
&&
بعد از آن امر آمدش از کردگار
\\
بعد ازین می‌ده ولی از کس مخواه
&&
ما بدادیمت ز غیب این دستگاه
\\
هر که خواهد از تو از یک تا هزار
&&
دست در زیر حصیری کن بر آر
\\
هین ز گنج رحمت بی‌مر بده
&&
در کف تو خاک گردد زر بده
\\
هر چه خواهندت بده مندیش از آن
&&
داد یزدان را تو بیش از بیش دان
\\
دست زیر بوریا کن ای سند
&&
از برای روی‌پوش چشم بد
\\
پس ز زیر بوریا پر کن تو مشت
&&
ده به دست سایل بشکسته پشت
\\
بعد ازین از اجر ناممنون بده
&&
هر که خواهد گوهر مکنون بده
\\
رو ید الله فوق ایدیهم تو باش
&&
هم‌چو دست حق گزافی رزق پاش
\\
وام داران را ز عهده وا رهان
&&
هم‌چو باران سبز کن فرش جهان
\\
بود یک سال دگر کارش همین
&&
که بدادی زر ز کیسهٔ رب دین
\\
زر شدی خاک سیه اندر کفش
&&
حاتم طایی گدایی در صفش
\\
\end{longtable}
\end{center}
