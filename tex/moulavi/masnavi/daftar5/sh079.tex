\begin{center}
\section*{بخش ۷۹ - بیان اتحاد عاشق و معشوق از روی حقیقت اگر چه متضادند  از روی آنک نیاز ضد بی‌نیازیست چنان که آینه بی‌صورتست و  ساده است و بی‌صورتی ضد صورتست ولکن میان ایشان اتحادیست  در حقیقت کی شرح آن درازست و العاقل یکفیه الاشاره}
\label{sec:sh079}
\addcontentsline{toc}{section}{\nameref{sec:sh079}}
\begin{longtable}{l p{0.5cm} r}
جسم مجنون را ز رنج و دوریی
&&
اندر آمد ناگهان رنجوریی
\\
خون بجوش آمد ز شعلهٔ اشتیاق
&&
تا پدید آمد بر آن مجنون خناق
\\
پس طبیب آمد بدار و کردنش
&&
گفت چاره نیست هیچ از رگ‌زنش
\\
رگ زدن باید برای دفع خون
&&
رگ‌زنی آمد بدانجا ذو فنون
\\
بازوش بست و گرفت آن نیش او
&&
بانک بر زد در زمان آن عشق‌خو
\\
مزد خود بستان و ترک فصد کن
&&
گر بمیرم گو برو جسم کهن
\\
گفت آخر از چه می‌ترسی ازین
&&
چون نمی‌ترسی تو از شیر عرین
\\
شیر و گرگ و خرس و هر گور و دده
&&
گرد بر گرد تو شب گرد آمده
\\
می نه آیدشان ز تو بوی بشر
&&
ز انبهی عشق و وجد اندر جگر
\\
گرگ و خرس و شیر داند عشق چیست
&&
کم ز سگ باشد که از عشق او عمیست
\\
گر رگ عشقی نبودی کلب را
&&
کی بجستی کلب کهفی قلب را
\\
هم ز جنس او به صورت چون سگان
&&
گر نشد مشهور هست اندر جهان
\\
بو نبردی تو دل اندر جنس خویش
&&
کی بری تو بوی دل از گرگ و میش
\\
گر نبودی عشق هستی کی بدی
&&
کی زدی نان بر تو و کی تو شدی
\\
نان تو شد از چه ز عشق و اشتها
&&
ورنه نان را کی بدی تا جان رهی
\\
عشق نان مرده را می جان کند
&&
جان که فانی بود جاویدان کند
\\
گفت مجنون من نمی‌ترسم ز نیش
&&
صبر من از کوه سنگین هست بیش
\\
منبلم بی‌زخم ناساید تنم
&&
عاشقم بر زخمها بر می‌تنم
\\
لیک از لیلی وجود من پرست
&&
این صدف پر از صفات آن درست
\\
ترسم ای فصاد گر فصدم کنی
&&
نیش را ناگاه بر لیلی زنی
\\
داند آن عقلی که او دل‌روشنیست
&&
در میان لیلی و من فرق نیست
\\
\end{longtable}
\end{center}
