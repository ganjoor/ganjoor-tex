\begin{center}
\section*{بخش ۹۴ - تشبیه کردن قطب کی عارف واصلست در اجری دادن خلق از قوت مغفرت و رحمت بر مراتبی کی حقش الهام دهد و تمثیل بشیر که دد اجری خوار و باقی خوار ویند بر مراتب قرب ایشان بشیر نه قرب مکانی بلک قرب صفتی و تفاصیل این بسیارست  والله الهادی}
\label{sec:sh094}
\addcontentsline{toc}{section}{\nameref{sec:sh094}}
\begin{longtable}{l p{0.5cm} r}
قطب شیر و صید کردن کار او
&&
باقیان این خلق باقی‌خوار او
\\
تا توانی در رضای قطب کوش
&&
تا قوی گردد کند صید وحوش
\\
چو برنجد بی‌نوا مانند خلق
&&
کز کف عقلست جمله رزق حلق
\\
زانک وجد حلق باقی خورد اوست
&&
این نگه دار ار دل تو صیدجوست
\\
او چو عقل و خلق چون اعضا و تن
&&
بستهٔ عقلست تدبیر بدن
\\
ضعف قطب از تن بود از روح نی
&&
ضعف در کشتی بود در نوح نی
\\
قطب آن باشد که گرد خود تند
&&
گردش افلاک گرد او بود
\\
یاریی ده در مرمهٔ کشتی‌اش
&&
گر غلام خاص و بنده گشتی‌اش
\\
یاریت در تو فزاید نه اندرو
&&
گفت حق ان تنصروا الله تنصروا
\\
هم‌چو روبه صید گیر و کن فداش
&&
تا عوض گیری هزاران صید بیش
\\
روبهانه باشد آن صید مرید
&&
مرده گیرد صید کفتار مرید
\\
مرده پیش او کشی زنده شود
&&
چرک در پالیز روینده شود
\\
گفت روبه شیر را خدمت کنم
&&
حیله‌ها سازم ز عقلش بر کنم
\\
حیله و افسونگری کار منست
&&
کار من دستان و از ره بردنست
\\
از سر که جانب جو می‌شتافت
&&
آن خر مسکین لاغر را بیافت
\\
پس سلام گرم کرد و پیش رفت
&&
پیش آن ساده دل درویش رفت
\\
گفت چونی اندرین صحرای خشک
&&
در میان سنگ لاخ و جای خشک
\\
گفت خر گر در غمم گر در ارم
&&
قسمتم حق کرد من زان شاکرم
\\
شکر گویم دوست را در خیر و شر
&&
زانک هست اندر قضا از بد بتر
\\
چونک قسام اوست کفر آمد گله
&&
صبر باید صبر مفتاح الصله
\\
غیر حق جمله عدواند اوست دوست
&&
با عدو از دوست شکوت کی نکوست
\\
تا دهد دوغم نخواهم انگبین
&&
زانک هر نعمت غمی دارد قرین
\\
\end{longtable}
\end{center}
