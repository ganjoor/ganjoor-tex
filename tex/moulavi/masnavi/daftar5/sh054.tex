\begin{center}
\section*{بخش ۵۴ - مناجات}
\label{sec:sh054}
\addcontentsline{toc}{section}{\nameref{sec:sh054}}
\begin{longtable}{l p{0.5cm} r}
ای دهندهٔ قوت و تمکین و ثبات
&&
خلق را زین بی‌ثباتی ده نجات
\\
اندر آن کاری که ثابت بودنیست
&&
قایمی ده نفس را که منثنیست
\\
صبرشان بخش و کفهٔ میزان گران
&&
وا رهانشان از فن صورتگران
\\
وز حسودی بازشان خر ای کریم
&&
تا نباشند از حسد دیو رجیم
\\
در نعیم فانی مال و جسد
&&
چون همی‌سوزند عامه از حسد
\\
پادشاهان بین که لشکر می‌کشند
&&
از حسد خویشان خود را می‌کشند
\\
عاشقان لعبتان پر قذر
&&
کرده قصد خون و جان همدگر
\\
ویس و رامین خسرو و شیرین بخوان
&&
که چه کردند از حسد آن ابلهان
\\
که فنا شد عاشق و معشوق نیز
&&
هم نه چیزند و هواشان هم نه چیز
\\
پاک الهی که عدم بر هم زند
&&
مر عدم را بر عدم عاشق کند
\\
در دل نه‌دل حسدها سر کند
&&
نیست را هست این چنین مضطر کند
\\
این زنانی کز همه مشفق‌تراند
&&
از حسد دو ضره خود را می‌خورند
\\
تا که مردانی که خود سنگین‌دلند
&&
از حسد تا در کدامین منزلند
\\
گر نکردی شرع افسونی لطیف
&&
بر دریدی هر کسی جسم حریف
\\
شرع بهر دفع شر رایی زند
&&
دیو را در شیشهٔ حجت کند
\\
از گواه و از یمین و از نکول
&&
تا به شیشه در رود دیو فضول
\\
مثل میزانی که خشنودی دو ضد
&&
جمع می‌آید یقین در هزل و جد
\\
شرع چون کیله و ترازو دان یقین
&&
که بدو خصمان رهند از جنگ و کین
\\
گر ترازو نبود آن خصم از جدال
&&
کی رهد از وهم حیف و احتیال
\\
پس درین مردار زشت بی‌وفا
&&
این همه رشکست و خصمست و جفا
\\
پس در اقبال و دولت چون بود
&&
چون شود جنی و انسی در حسد
\\
آن شیاطین خود حسود کهنه‌اند
&&
یک زمان از ره‌زنی خالی نه‌اند
\\
وآن بنی آدم که عصیان کشته‌اند
&&
از حسودی نیز شیطان گشته‌اند
\\
از نبی برخوان که شیطانان انس
&&
گشته‌اند از مسخ حق با دیو جنس
\\
دیو چون عاجز شود در افتتان
&&
استعانت جوید او زین انسیان
\\
که شما یارید با ما یاریی
&&
جانب مایید جانب داریی
\\
گر کسی را ره زنند اندر جهان
&&
هر دو گون شیطان بر آید شادمان
\\
ور کسی جان برد و شد در دین بلند
&&
نوحه می‌دارند آن دو رشک‌مند
\\
هر دو می‌خایند دندان حسد
&&
بر کسی که داد ادیب او را خرد
\\
\end{longtable}
\end{center}
