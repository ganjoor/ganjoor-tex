\begin{center}
\section*{بخش ۱۵۰ - جواب گفتن امیر مر آن شفیعان را و همسایگان زاهد را کی گستاخی چرا کرد و سبوی ما را چرا شکست من درین باب شفاعت قبول نخواهم کرد کی سوگند خورده‌ام کی سزای او را بدهم}
\label{sec:sh150}
\addcontentsline{toc}{section}{\nameref{sec:sh150}}
\begin{longtable}{l p{0.5cm} r}
میر گفت او کیست کو سنگی زند
&&
بر سبوی ما سبو را بشکند
\\
چون گذر سازد ز کویم شیر نر
&&
ترس ترسان بگذرد با صد حذر
\\
بندهٔ ما را چرا آزرد دل
&&
کرد ما را پیش مهمانان خجل
\\
شربتی که به ز خون اوست ریخت
&&
این زمان هم‌چون زنان از ما گریخت
\\
لیک جان از دست من او کی برد
&&
گیر هم‌چون مرغ بالا بر پرد
\\
تیر قهر خویش بر پرش زنم
&&
پر و بال مردریگش بر کنم
\\
گر رود در سنگ سخت از کوششم
&&
از دل سنگش کنون بیرون کشم
\\
من برانم بر تن او ضربتی
&&
که بود قوادکان را عبرتی
\\
با همه سالوس با ما نیز هم
&&
داد او و صد چو او این دم دهم
\\
خشم خون‌خوارش شده بد سرکشی
&&
از دهانش می بر آمد آتشی
\\
\end{longtable}
\end{center}
