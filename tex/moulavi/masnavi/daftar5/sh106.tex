\begin{center}
\section*{بخش ۱۰۶ - غالب شدن حیلهٔ روباه بر استعصام و تعفف خر و کشیدن روبه خر را سوی شیر به بیشه}
\label{sec:sh106}
\addcontentsline{toc}{section}{\nameref{sec:sh106}}
\begin{longtable}{l p{0.5cm} r}
روبه اندر حیله پای خود فشرد
&&
ریش خر بگرفت و آن خر را ببرد
\\
مطرب آن خانقه کو تا که تفت
&&
دف زند که خر برفت و خر برفت
\\
چونک خرگوشی برد شیری به چاه
&&
چون نیارد روبهی خر تا گیاه
\\
گوش را بر بند و افسونها مخور
&&
جز فسون آن ولی دادگر
\\
آن فسون خوشتر از حلوای او
&&
آنک صد حلواست خاک پای او
\\
خنبهای خسروانی پر ز می
&&
مایه برده از می لبهای وی
\\
عاشق می باشد آن جان بعید
&&
کو می لبهای لعلش را ندید
\\
آب شیرین چون نبیند مرغ کور
&&
چون نگردد گرد چشمهٔ آب شور
\\
موسی جان سینه را سینا کند
&&
طوطیان کور را بینا کند
\\
خسرو شیرین جان نوبت زدست
&&
لاجرم در شهر قند ارزان شدست
\\
یوسفان غیب لشکر می‌کشند
&&
تنگهای قند و شکر می‌کشند
\\
اشتران مصر را رو سوی ما
&&
بشنوید ای طوطیان بانگ درا
\\
شهر ما فردا پر از شکر شود
&&
شکر ارزانست ارزان‌تر شود
\\
در شکر غلطید ای حلواییان
&&
هم‌چو طوطی کوری صفراییان
\\
نیشکر کوبید کار اینست و بس
&&
جان بر افشانید یار اینست و بس
\\
نقل بر نقلست و می بر می هلا
&&
بر مناره رو بزن بانگ صلا
\\
سرکهٔ نه ساله شیرین می‌شود
&&
سنگ و مرمر لعل و زرین می‌شود
\\
آفتاب اندر فلک دستک‌زنان
&&
ذره‌ها چون عاشقان بازی‌کنان
\\
چشمها مخمور شد از سبزه‌زار
&&
گل شکوفه می‌کند بر شاخسار
\\
چشم دولت سحر مطلق می‌کند
&&
روح شد منصور انا الحق می‌زند
\\
گر خری را می‌برد روبه ز سر
&&
گو ببر تو خر مباش و غم مخور
\\
\end{longtable}
\end{center}
