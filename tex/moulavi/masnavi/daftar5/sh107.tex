\begin{center}
\section*{بخش ۱۰۷ - حکایت آن شخص کی از ترس خویشتن را در خانه‌ای انداخت رخها زرد چون زعفران لبها کبود چون نیل دست لرزان چون برگ درخت خداوند خانه پرسید کی خیرست چه واقعه است گفت بیرون خر می‌گیرند به سخره گفت مبارک خر می‌گیرند تو خر نیستی چه می‌ترسی گفت خر به جد می‌گیرند تمییز برخاسته است امروز ترسم کی مرا خر گیرند}
\label{sec:sh107}
\addcontentsline{toc}{section}{\nameref{sec:sh107}}
\begin{longtable}{l p{0.5cm} r}
آن یکی در خانه‌ای در می‌گریخت
&&
زرد رو و لب کبود و رنگ ریخت
\\
صاحب خانه بگفتش خیر هست
&&
که همی لرزد ترا چون پیر دست
\\
واقعه چونست چون بگریختی
&&
رنگ رخساره چنین چون ریختی
\\
گفت بهر سخرهٔ شاه حرون
&&
خر همی‌گیرند امروز از برون
\\
گفت می‌گیرند کو خر جان عم
&&
چون نه‌ای خر رو ترا زین چیست غم
\\
گفت بس جدند و گرم اندر گرفت
&&
گر خرم گیرند هم نبود شگفت
\\
بهر خرگیری بر آوردند دست
&&
جدجد تمییز هم برخاستست
\\
چونک بی‌تمییزیان‌مان سرورند
&&
صاحب خر را به جای خر برند
\\
نیست شاه شهر ما بیهوده گیر
&&
هست تمییزش سمیعست و بصیر
\\
آدمی باش و ز خرگیران مترس
&&
خر نه‌ای ای عیسی دوران مترس
\\
چرخ چارم هم ز نور تو پرست
&&
حاش لله که مقامت آخرست
\\
تو ز چرخ و اختران هم برتری
&&
گرچه بهر مصلحت در آخری
\\
میر آخر دیگر و خر دیگرست
&&
نه هر آنک اندر آخر شد خرست
\\
چه در افتادیم در دنبال خر
&&
از گلستان گوی و از گلهای تر
\\
از انار و از ترنج و شاخ سیب
&&
وز شراب و شاهدان بی‌حساب
\\
یا از آن دریا که موجش گوهرست
&&
گوهرش گوینده و بیناورست
\\
یا از آن مرغان که گل‌چین می‌کنند
&&
بیضه‌ها زرین و سیمین می‌کنند
\\
یا از آن بازان که کبکان پرورند
&&
هم نگون اشکم هم استان می‌پرند
\\
نردبانهاییست پنهان در جهان
&&
پایه پایه تا عنان آسمان
\\
هر گره را نردبانی دیگرست
&&
هر روش را آسمانی دیگرست
\\
هر یکی از حال دیگر بی‌خبر
&&
ملک با پهنا و بی‌پایان و سر
\\
این در آن حیران که او از چیست خوش
&&
وآن درین خیره که حیرت چیستش
\\
صحن ارض الله واسع آمده
&&
هر درختی از زمینی سر زده
\\
بر درختان شکر گویان برگ و شاخ
&&
که زهی ملک و زهی عرصهٔ فراخ
\\
بلبلان گرد شکوفهٔ پر گره
&&
که از آنچ می‌خوری ما را بده
\\
این سخن پایان ندارد کن رجوع
&&
سوی آن روباه و شیر و سقم و جوع
\\
\end{longtable}
\end{center}
