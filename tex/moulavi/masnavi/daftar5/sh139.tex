\begin{center}
\section*{بخش ۱۳۹ - گفتن خویشاوندان مجنون را کی حسن لیلی باندازه‌ایست چندان نیست ازو نغزتر در شهر ما بسیارست یکی و دو و ده بر تو عرضه کنیم اختیار کن ما را و خود را وا رهان و جواب گفتن مجنون ایشان را}
\label{sec:sh139}
\addcontentsline{toc}{section}{\nameref{sec:sh139}}
\begin{longtable}{l p{0.5cm} r}
ابلهان گفتند مجنون را ز جهل
&&
حسن لیلی نیست چندان هست سهل
\\
بهتر از وی صد هزاران دلربا
&&
هست هم‌چون ماه اندر شهر ما
\\
گفت صورت کوزه است و حسن می
&&
می خدایم می‌دهد از نقش وی
\\
مر شما را سرکه داد از کوزه‌اش
&&
تا نباشد عشق اوتان گوش کش
\\
از یکی کوزه دهد زهر و عسل
&&
هر یکی را دست حق عز و جل
\\
کوزه می‌بینی ولیکن آب شراب
&&
روی ننماید به چشم ناصواب
\\
قاصرات الطرف باشد ذوق جان
&&
جز به خصم خود بنماید نشان
\\
قاصرات الطرف آمد آن مدام
&&
وین حجاب ظرفها هم‌چون خیام
\\
هست دریا خیمه‌ای در وی حیات
&&
بط را لیکن کلاغان را ممات
\\
زهر باشد مار را هم قوت و برگ
&&
غیر او را زهر او دردست و مرگ
\\
صورت هر نعمتی و محنتی
&&
هست این را دوزخ آن را جنتی
\\
پس همه اجسام و اشیا تبصرون
&&
واندرو قوتست و سم لاتبصرون
\\
هست هر جسمی چو کاسه و کوزه‌ای
&&
اندرو هم قوت و هم دلسوزه‌ای
\\
کاسه پیدا اندرو پنهان رغد
&&
طاعمش داند کزان چه می‌خورد
\\
صورت یوسف چو جامی بود خوب
&&
زان پدر می‌خورد صد بادهٔ طروب
\\
باز اخوان را از آن زهراب بود
&&
کان دریشان خشم و کینه می‌فزود
\\
باز از وی مر زلیخا را سکر
&&
می‌کشید از عشق افیونی دگر
\\
غیر آنچ بود مر یعقوب را
&&
بود از یوسف غذا آن خوب را
\\
گونه‌گونه شربت و کوزه یکی
&&
تا نماند در می غیبت شکی
\\
باده از غیبست و کوزه زین جهان
&&
کوزه پیدا باده در وی بس نهان
\\
بس نهان از دیدهٔ نامحرمان
&&
لیک بر محرم هویدا و عیان
\\
یا الهی سکرت ابصارنا
&&
فاعف عنا اثقلت اوزارنا
\\
یا خفیا قد ملات الخافقین
&&
قد علوت فوق نور المشرقین
\\
انت سر کاشف اسرارنا
&&
انت فجر مفجر انهارنا
\\
یا خفی الذات محسوس العطا
&&
انت کالماء و نحن کالرحا
\\
انت کالریح و نحن کالغبار
&&
تختفی الریح و غبراها جهار
\\
تو بهاری ما چو باغ سبز خوش
&&
او نهان و آشکارا بخششش
\\
تو چو جانی ما مثال دست و پا
&&
قبض و بسط دست از جان شد روا
\\
تو چو عقلی ما مثال این زبان
&&
این زبان از عقل دارد این بیان
\\
تو مثال شادی و ما خنده‌ایم
&&
که نتیجهٔ شادی فرخنده‌ایم
\\
جنبش ما هر دمی خود اشهدست
&&
که گواه ذوالجلال سرمدست
\\
گردش سنگ آسیا در اضطراب
&&
اشهد آمد بر وجود جوی آب
\\
ای برون از وهم و قال و قیل من
&&
خاک بر فرق من و تمثیل من
\\
بنده نشکیبد ز تصویر خوشت
&&
هر دمت گوید که جانم مفرشت
\\
هم‌چو آن چوپان که می‌گفت ای خدا
&&
پیش چوپان و محب خود بیا
\\
تا شپش جویم من از پیراهنت
&&
چارقت دوزم ببوسم دامنت
\\
کس نبودش در هوا و عشق جفت
&&
لیک قاصر بود از تسبیح و گفت
\\
عشق او خرگاه بر گردون زده
&&
جان سگ خرگاه آن چوپان شده
\\
چونک بحر عشق یزدان جوش زد
&&
بر دل او زد ترا بر گوش زد
\\
\end{longtable}
\end{center}
