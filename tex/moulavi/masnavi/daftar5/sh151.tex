\begin{center}
\section*{بخش ۱۵۱ - دو بار دست و پای امیر را بوسیدن و لابه کردن شفیعان و همسایگان زاهد}
\label{sec:sh151}
\addcontentsline{toc}{section}{\nameref{sec:sh151}}
\begin{longtable}{l p{0.5cm} r}
آن شفیعان از دم هیهای او
&&
چند بوسیدند دست و پای او
\\
کای امیر از تو نشاید کین کشی
&&
گر بشد باده تو بی‌باده خوشی
\\
باده سرمایه ز لطف تو برد
&&
لطف آب از لطف تو حسرت خورد
\\
پادشاهی کن ببخشش ای رحیم
&&
ای کریم ابن الکریم ابن الکریم
\\
هر شرابی بندهٔ این قد و خد
&&
جمله مستان را بود بر تو حسد
\\
هیچ محتاج می گلگون نه‌ای
&&
ترک کن گلگونه تو گلگونه‌ای
\\
ای رخ چون زهره‌ات شمس الضحی
&&
ای گدای رنگ تو گلگونه‌ها
\\
باده کاندر خنب می‌جوشد نهان
&&
ز اشتیاق روی تو جوشد چنان
\\
ای همه دریا چه خواهی کرد نم
&&
وی همه هستی چه می‌جویی عدم
\\
ای مه تابان چه خواهی کرد گرد
&&
ای که مه در پیش رویت روی‌زرد
\\
تاج کرمناست بر فرق سرت
&&
طوق اعطیناک آویز برت
\\
تو خوش و خوبی و کان هر خوشی
&&
تو چرا خود منت باده کشی
\\
جوهرست انسان و چرخ او را عرض
&&
جمله فرع و پایه‌اند و او غرض
\\
ای غلامت عقل و تدبیرات و هوش
&&
چون چنینی خویش را ارزان فروش
\\
خدمتت بر جمله هستی مفترض
&&
جوهری چون نجده خواهد از عرض
\\
علم جویی از کتبها ای فسوس
&&
ذوق جویی تو ز حلوا ای فسوس
\\
بحر علمی در نمی پنهان شده
&&
در سه گز تن عالمی پنهان شده
\\
می چه باشد یا سماع و یا جماع
&&
تا بجویی زو نشاط و انتفاع
\\
آفتاب از ذره‌ای شد وام خواه
&&
زهره‌ای از خمره‌ای شد جام‌خواه
\\
جان بی‌کیفی شده محبوس کیف
&&
آفتابی حبس عقده اینت حیف
\\
\end{longtable}
\end{center}
