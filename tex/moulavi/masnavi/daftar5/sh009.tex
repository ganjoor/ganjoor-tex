\begin{center}
\section*{بخش ۹ - استعانت آب از حق جل جلاله بعد از تیره شدن}
\label{sec:sh009}
\addcontentsline{toc}{section}{\nameref{sec:sh009}}
\begin{longtable}{l p{0.5cm} r}
ناله از باطن برآرد کای خدا
&&
آنچ دادی دادم و ماندم گدا
\\
ریختم سرمایه بر پاک و پلید
&&
ای شه سرمایه‌ده هل من مزید
\\
ابر را گوید ببر جای خوشش
&&
هم تو خورشیدا به بالا بر کشش
\\
راههای مختلف می‌راندش
&&
تا رساند سوی بحر بی‌حدش
\\
خود غرض زین آب جان اولیاست
&&
کو غسول تیرگیهای شماست
\\
چون شود تیره ز غدر اهل فرش
&&
باز گردد سوی پاکی بخش عرش
\\
باز آرد زان طرف دامن کشان
&&
از طهارات محیط او درسشان
\\
از تیمم وا رماند جمله را
&&
وز تحری طالبان قبله را
\\
ز اختلاط خلق یابد اعتلال
&&
آن سفر جوید که ارحنا یا بلال
\\
ای بلال خوش‌نوای خوش‌صهیل
&&
میذنه بر رو بزن طبل رحیل
\\
جان سفر رفت و بدن اندر قیام
&&
وقت رجعت زین سبب گوید سلام
\\
این مثل چون واسطه‌ست اندر کلام
&&
واسطه شرطست بهر فهم عام
\\
اندر آتش کی رود بی‌واسطه
&&
جز سمندر کو رهید از رابطه
\\
واسطهٔ حمام باید مر ترا
&&
تا ز آتش خوش کنی تو طبع را
\\
چون نتانی شد در آتش چون خلیل
&&
گشت حمامت رسول آبت دلیل
\\
سیری از حقست لیک اهل طبع
&&
کی رسد بی‌واسطهٔ نان در شبع
\\
لطف از حقست لیکن اهل تن
&&
درنیابد لطف بی‌پردهٔ چمن
\\
چون نماند واسطهٔ تن بی‌حجاب
&&
هم‌چو موسی نور مه یابد ز جیب
\\
این هنرها آب را هم شاهدست
&&
که اندرونش پر ز لطف ایزدست
\\
\end{longtable}
\end{center}
