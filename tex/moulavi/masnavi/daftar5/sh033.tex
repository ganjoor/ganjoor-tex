\begin{center}
\section*{بخش ۳۳ - بیان آنک هنرها و زیرکیها و مال دنیا هم‌چون پرهای طاوس عدو جانست}
\label{sec:sh033}
\addcontentsline{toc}{section}{\nameref{sec:sh033}}
\begin{longtable}{l p{0.5cm} r}
پس هنر آمد هلاکت خام را
&&
کز پی دانه نبیند دام را
\\
اختیار آن را نکو باشد که او
&&
مالک خود باشد اندر اتقوا
\\
چون نباشد حفظ و تقوی زینهار
&&
دور کن آلت بینداز اختیار
\\
جلوه‌گاه و اختیارم آن پرست
&&
بر کنم پر را که در قصد سرست
\\
نیست انگارد پر خود را صبور
&&
تا پرش در نفکند در شر و شور
\\
پس زیانش نیست پر گو بر مکن
&&
گر رسد تیری به پیش آرد مجن
\\
لیک بر من پر زیبا دشمنیست
&&
چونک از جلوه‌گری صبریم نیست
\\
گر بدی صبر و حفاظم راه‌بر
&&
بر فزودی ز اختیارم کر و فر
\\
هم‌چو طفلم یا چو مست اندر فتن
&&
نیست لایق تیغ اندر دست من
\\
گر مرا عقلی بدی و منزجر
&&
تیغ اندر دست من بودی ظفر
\\
عقل باید نورده چون آفتاب
&&
تا زند تیغی که نبود جز صواب
\\
چون ندارم عقل تابان و صلاح
&&
پس چرا در چاه نندازم سلاح
\\
در چه اندازم کنون تیغ و مجن
&&
کین سلاح خصم من خواهد شدن
\\
چون ندارم زور و یاری و سند
&&
تیغم او بستاند و بر من زند
\\
رغم این نفس وقیحه‌خوی را
&&
که نپوشد رو خراشم روی را
\\
تا شود کم این جمال و این کمال
&&
چون نماند رو کم افتم در وبال
\\
چون بدین نیت خراشم بزه نیست
&&
که به زخم این روی را پوشیدنیست
\\
گر دلم خوی ستیری داشتی
&&
روی خوبم جز صفا نفراشتی
\\
چون ندیدم زور و فرهنگ و صلاح
&&
خصم دیدم زود بشکستم سلاح
\\
تا نگردد تیغ من او را کمال
&&
تا نگردد خنجرم بر من وبال
\\
می‌گریزم تا رگم جنبان بود
&&
کی فرار از خویشتن آسان بود
\\
آنک از غیری بود او را فرار
&&
چون ازو ببرید گیرد او قرار
\\
من که خصمم هم منم اندر گریز
&&
تا ابد کار من آمد خیزخیز
\\
نه به هندست آمن و نه در ختن
&&
آنک خصم اوست سایهٔ خویشتن
\\
\end{longtable}
\end{center}
