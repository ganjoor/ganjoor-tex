\begin{center}
\section*{بخش ۱۶۶ - پشیمان شدن آن سرلشکر از آن خیانت کی کرد و سوگند دادن او آن کنیزک را کی به خلیفه باز نگوید از آنچ رفت}
\label{sec:sh166}
\addcontentsline{toc}{section}{\nameref{sec:sh166}}
\begin{longtable}{l p{0.5cm} r}
چند روزی هم بر آن بد بعد از آن
&&
شد پشیمان او از آن جرم گران
\\
داد سوگندش کای خورشیدرو
&&
با خلیفه زینچ شد رمزی مگو
\\
چون ندید او را خلیفه مست گشت
&&
پس ز بام افتاد او را نیز طشت
\\
دید صد چندان که وصفش کرده بود
&&
کی بود خود دیده مانند شنود
\\
وصف تصویرست بهر چشم هوش
&&
صورت آن چشم دان نه زان گوش
\\
کرد مردی از سخن‌دانی سال
&&
حق و باطل چیست ای نیکو مقال
\\
گوش را بگرفت و گفت این باطلست
&&
چشم حقست و یقینش حاصلست
\\
آن به نسبت باطل آمد پیش این
&&
نسبتست اغلب سخنها ای امین
\\
ز آفتاب ار کرد خفاش احتجاب
&&
نیست محجوب از خیال آفتاب
\\
خوف او را خود خیالش می‌دهد
&&
آن خیالش سوی ظلمت می‌کشد
\\
آن خیال نور می‌ترساندش
&&
بر شب ظلمات می‌چفساندش
\\
از خیال دشمن و تصویر اوست
&&
که تو بر چفسیده‌ای بر یار و دوست
\\
موسیا کشفت لمع بر که فراشت
&&
آن مخیل تاب تحقیقت نداشت
\\
هین مشو غره بدانک قابلی
&&
مر خیالش را و زین ره واصلی
\\
از خیال حرب نهراسید کس
&&
لا شجاعه قبل حرب این دان و بس
\\
بر خیال حرب خیز اندر فکر
&&
می‌کند چون رستمان صد کر و فر
\\
نقش رستم که آن به حمامی بود
&&
قرن حمله فکر هر خامی بود
\\
این خیال سمع چون مبصر شود
&&
حیز چه بود رستمی مضطر شود
\\
جهد کن کز گوش در چشمت رود
&&
آنچ که آن باطل بدست آن حق شود
\\
زان سپس گوشت شود هم طبع چشم
&&
گوهری گردد دو گوش هم‌چو یشم
\\
بلک جمله تن چو آیینه شود
&&
جمله چشم و گوهر سینه شود
\\
گوش انگیزد خیال و آن خیال
&&
هست دلالهٔ وصال آن جمال
\\
جهد کن تا این خیال افزون شود
&&
تا دلاله رهبر مجنون شود
\\
آن خلیفه گول هم یک چند نیز
&&
ریش گاوی کرد خوش با آن کنیز
\\
ملک را تو ملک غرب و شرق گیر
&&
چون نمی‌ماند تو آن را برق گیر
\\
مملکت کان می‌نماند جاودان
&&
ای دلت خفته تو آن را خواب دان
\\
تا چه خواهی کرد آن باد و بروت
&&
که بگیرد هم‌چو جلادی گلوت
\\
هم درین عالم بدان که مامنیست
&&
از منافق کم شنو کو گفت نیست
\\
\end{longtable}
\end{center}
