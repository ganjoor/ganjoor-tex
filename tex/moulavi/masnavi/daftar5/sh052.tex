\begin{center}
\section*{بخش ۵۲ - سبب عداوت عام و بیگانه زیستن ایشان به اولیاء خدا کی بحقشان می‌خوانند و با آب حیات ابدی}
\label{sec:sh052}
\addcontentsline{toc}{section}{\nameref{sec:sh052}}
\begin{longtable}{l p{0.5cm} r}
بلک از چفسیدگی در خان و مان
&&
تلخشان آید شنیدن این بیان
\\
خرقه‌ای بر ریش خر چفسید سخت
&&
چونک خواهی بر کنی زو لخت لخت
\\
جفته اندازد یقین آن خر ز درد
&&
حبذا آن کس کزو پرهیز کرد
\\
خاصه پنجه ریش و هر جا خرقه‌ای
&&
بر سرش چفسیده در نم غرقه‌ای
\\
خان و مان چون خرقه و این حرص‌ریش
&&
حرص هر که بیش باشد ریش بیش
\\
خان و مان چغد ویرانست و بس
&&
نشنود اوصاف بغداد و طبس
\\
گر بیاید باز سلطانی ز راه
&&
صد خبر آرد بدین چغدان ز شاه
\\
شرح دارالملک و باغستان و جو
&&
پس برو افسوس دارد صد عدو
\\
که چه باز آورد افسانهٔ کهن
&&
کز گزاف و لاف می‌بافد سخن
\\
کهنه ایشانند و پوسیدهٔ ابد
&&
ورنه آن دم کهنه را نو می‌کند
\\
مردگان کهنه را جان می‌دهد
&&
تاج عقل و نور ایمان می‌دهد
\\
دل مدزد از دلربای روح‌بخش
&&
که سوارت می‌کند بر پشت رخش
\\
سر مدزد از سر فراز تاج‌ده
&&
کو ز پای دل گشاید صد گره
\\
با کی گویم در همه ده زنده کو
&&
سوی آب زندگی پوینده کو
\\
تو به یک خواری گریزانی ز عشق
&&
تو به جز نامی چه می‌دانی ز عشق
\\
عشق را صد ناز و استکبار هست
&&
عشق با صد ناز می‌آید به دست
\\
عشق چون وافیست وافی می‌خرد
&&
در حریف بی‌وفا می‌ننگرد
\\
چون درختست آدمی و بیخ عهد
&&
بیخ را تیمار می‌باید به جهد
\\
عهد فاسد بیخ پوسیده بود
&&
وز ثمار و لطف ببریده بود
\\
شاخ و برگ نخل گر چه سبز بود
&&
با فساد بیخ سبزی نیست سود
\\
ور ندارد برگ سبز و بیخ هست
&&
عاقبت بیرون کند صد برگ دست
\\
تو مشو غره به علمش عهد جو
&&
علم چون قشرست و عهدش مغز او
\\
\end{longtable}
\end{center}
