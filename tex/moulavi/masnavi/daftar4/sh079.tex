\begin{center}
\section*{بخش ۷۹ - قصهٔ سبحانی ما اعظم شانی گفتن ابویزید قدس الله سره و اعتراض مریدان و جواب این مر ایشان را نه به طریق گفت زبان بلک از راه عیان}
\label{sec:sh079}
\addcontentsline{toc}{section}{\nameref{sec:sh079}}
\begin{longtable}{l p{0.5cm} r}
با مریدان آن فقیر محتشم
&&
بایزید آمد که نک یزدان منم
\\
گفت مستانه عیان آن ذوفنون
&&
لا اله الا انا ها فاعبدون
\\
چون گذشت آن حال گفتندش صباح
&&
تو چنین گفتی و این نبود صلاح
\\
گفت این بار ار کنم من مشغله
&&
کاردها بر من زنید آن دم هله
\\
حق منزه از تن و من با تنم
&&
چون چنین گویم بباید کشتنم
\\
چون وصیت کرد آن آزادمرد
&&
هر مریدی کاردی آماده کرد
\\
مست گشت او باز از آن سغراق زفت
&&
آن وصیتهاش از خاطر برفت
\\
نقل آمد عقل او آواره شد
&&
صبح آمد شمع او بیچاره شد
\\
عقل چون شحنه‌ست چون سلطان رسید
&&
شحنهٔ بیچاره در کنجی خزید
\\
عقل سایهٔ حق بود حق آفتاب
&&
سایه را با آفتاب او چه تاب
\\
چون پری غالب شود بر آدمی
&&
گم شود از مرد وصف مردمی
\\
هر چه گوید آن پری گفته بود
&&
زین سری زان آن سری گفته بود
\\
چون پری را این دم و قانون بود
&&
کردگار آن پری خود چون بود
\\
اوی او رفته پری خود او شده
&&
ترک بی‌الهام تازی‌گو شده
\\
چون به خود آید نداند یک لغت
&&
چون پری را هست این ذات و صفت
\\
پس خداوند پری و آدمی
&&
از پری کی باشدش آخر کمی
\\
شیرگیر ار خون نره شیر خورد
&&
تو بگویی او نکرد آن باده کرد
\\
ور سخن پردازد از زر کهن
&&
تو بگویی باده گفتست آن سخن
\\
باده‌ای را می‌بود این شر و شور
&&
نور حق را نیست آن فرهنگ و زور
\\
که ترا از تو به کل خالی کند
&&
تو شوی پست او سخن عالی کند
\\
گر چه قرآن از لب پیغامبرست
&&
هر که گوید حق نگفت او کافرست
\\
چون همای بی‌خودی پرواز کرد
&&
آن سخن را بایزید آغاز کرد
\\
عقل را سیل تحیر در ربود
&&
زان قوی‌تر گفت که اول گفته بود
\\
نیست اندر جبه‌ام الا خدا
&&
چند جویی بر زمین و بر سما
\\
آن مریدان جمله دیوانه شدند
&&
کاردها در جسم پاکش می‌زدند
\\
هر یکی چون ملحدان گرده کوه
&&
کارد می‌زد پیر خود را بی ستوه
\\
هر که اندر شیخ تیغی می‌خلید
&&
بازگونه از تن خود می‌درید
\\
یک اثر نه بر تن آن ذوفنون
&&
وان مریدان خسته و غرقاب خون
\\
هر که او سویی گلویش زخم برد
&&
حلق خود ببریده دید و زار مرد
\\
وآنک او را زخم اندر سینه زد
&&
سینه‌اش بشکافت و شد مردهٔ ابد
\\
وآنک آگه بود از آن صاحب‌قران
&&
دل ندادش که زند زخم گران
\\
نیم‌دانش دست او را بسته کرد
&&
جان ببرد الا که خود را خسته کرد
\\
روز گشت و آن مریدان کاسته
&&
نوحه‌ها از خانه‌شان برخاسته
\\
پیش او آمد هزاران مرد و زن
&&
کای دو عالم درج در یک پیرهن
\\
این تن تو گر تن مردم بدی
&&
چون تن مردم ز خنجر گم شدی
\\
با خودی با بی‌خودی دوچار زد
&&
با خود اندر دیدهٔ خود خار زد
\\
ای زده بر بی‌خودان تو ذوالفقار
&&
بر تن خود می‌زنی آن هوش دار
\\
زانک بی‌خود فانی است و آمنست
&&
تا ابد در آمنی او ساکنست
\\
نقش او فانی و او شد آینه
&&
غیر نقش روی غیر آن جای نه
\\
گر کنی تف سوی روی خود کنی
&&
ور زنی بر آینه بر خود زنی
\\
ور ببینی روی زشت آن هم توی
&&
ور ببینی عیسی و مریم توی
\\
او نه اینست و نه آن او ساده است
&&
نقش تو در پیش تو بنهاده است
\\
چون رسید اینجا سخن لب در ببست
&&
چون رسید اینجا قلم درهم شکست
\\
لب ببند ار چه فصاحت دست داد
&&
دم مزن والله اعلم بالرشاد
\\
برکنار بامی ای مست مدام
&&
پست بنشین یا فرود آ والسلام
\\
هر زمانی که شدی تو کامران
&&
آن دم خوش را کنار بام دان
\\
بر زمان خوش هراسان باش تو
&&
هم‌چو گنجش خفیه کن نه فاش تو
\\
تا نیاید بر ولا ناگه بلا
&&
ترس ترسان رو در آن مکمن هلا
\\
ترس جان در وقت شادی از زوال
&&
زان کنار بام غیبست ارتحال
\\
گر نمی‌بینی کنار بام راز
&&
روح می‌بیند که هستش اهتزاز
\\
هر نکالی ناگهان کان آمدست
&&
بر کنار کنگرهٔ شادی بدست
\\
جز کنار بام خود نبود سقوط
&&
اعتبار از قوم نوح و قوم لوط
\\
\end{longtable}
\end{center}
