\begin{center}
\section*{بخش ۱۲۵ - قصهٔ فرزندان عزیر علیه‌السلام کی از پدر احوال پدر می‌پرسیدند می‌گفت آری دیدمش می‌آید بعضی شناختندش بیهوش شدند بعضی نشناختند می‌گفتند خود مژده‌ای داد این بیهوش شدن  چیست}
\label{sec:sh125}
\addcontentsline{toc}{section}{\nameref{sec:sh125}}
\begin{longtable}{l p{0.5cm} r}
هم‌چو پوران عزیز اندر گذر
&&
آمده پرسان ز احوال پدر
\\
گشته ایشان پیر و باباشان جوان
&&
پس پدرشان پیش آمد ناگهان
\\
پس بپرسیدند ازو کای ره‌گذر
&&
از عزیر ما عجب داری خبر
\\
که کسی‌مان گفت که امروز آن سند
&&
بعد نومیدی ز بیرون می‌رسد
\\
گفت آری بعد من خواهد رسید
&&
آن یکی خوش شد چو این مژده شنید
\\
بانگ می‌زد کای مبشر باش شاد
&&
وان دگر بشناخت بیهوش اوفتاد
\\
که چه جای مژده است ای خیره‌سر
&&
که در افتادیم در کان شکر
\\
وهم را مژده‌ست و پیش عقل نقد
&&
ز انک چشم وهم شد محجوب فقد
\\
کافران را درد و مؤمن را بشیر
&&
لیک نقد حال در چشم بصیر
\\
زانک عاشق در دم نقدست مست
&&
لاجرم از کفر و ایمان برترست
\\
کفر و ایمان هر دو خود دربان اوست
&&
کوست مغز و کفر و دین او را دو پوست
\\
کفر قشر خشک رو بر تافته
&&
باز ایمان قشر لذت یافته
\\
قشرهای خشک را جا آتش است
&&
قشر پیوسته به مغز جان خوش است
\\
مغز خود از مرتبهٔ خوش برترست
&&
برترست از خوش که لذت گسترست
\\
این سخن پایان ندارد باز گرد
&&
تا برآرد موسیم از بحر گرد
\\
درخور عقل عوام این گفته شد
&&
از سخن باقی آن بنهفته شد
\\
زر عقلت ریزه است ای متهم
&&
بر قراضه مهر سکه چون نهم
\\
عقل تو قسمت شده بر صد مهم
&&
بر هزاران آرزو و طم و رم
\\
جمع باید کرد اجزا را به عشق
&&
تا شوی خوش چون سمرقند و دمشق
\\
جو جوی چون جمع گردی ز اشتباه
&&
پس توان زد بر تو سکهٔ پادشاه
\\
ور ز مثقالی شوی افزون تو خام
&&
از تو سازد شه یکی زرینه جام
\\
پس برو هم نام و هم القاب شاه
&&
باشد و هم صورتش ای وصل خواه
\\
تا که معشوقت بود هم نان هم آب
&&
هم چراغ و شاهد و نقل شراب
\\
جمع کن خود را جماعت رحمتست
&&
تا توانم با تو گفتن آنچ هست
\\
زانک گفتن از برای باوریست
&&
جان شرک از باوری حق بریست
\\
جان قسمت گشته بر حشو فلک
&&
در میان شصت سودا مشترک
\\
پس خموشی به دهد او را ثبوت
&&
پس جواب احمقان آمد سکوت
\\
این همی‌دانم ولی مستی تن
&&
می‌گشاید بی‌مراد من دهن
\\
آنچنان که از عطسه و از خامیاز
&&
این دهان گردد بناخواه تو باز
\\
\end{longtable}
\end{center}
