\begin{center}
\section*{بخش ۶۳ - تفسیر اوجس فی نفسه خیفة موسی قلنا لا تخف انک انت الا علی}
\label{sec:sh063}
\addcontentsline{toc}{section}{\nameref{sec:sh063}}
\begin{longtable}{l p{0.5cm} r}
گفت موسی سحر هم حیران‌کنیست
&&
چون کنم کین خلق را تمییز نیست
\\
گفت حق تمییز را پیدا کنم
&&
عقل بی‌تمییز را بینا کنم
\\
گرچه چون دریا برآوردند کف
&&
موسیا تو غالب آیی لا تخف
\\
بود اندر عهده خود سحر افتخار
&&
چون عصا شد مار آنها گشت عار
\\
هر کسی را دعوی حسن و نمک
&&
سنگ مرگ آمد نمکها را محک
\\
سحر رفت و معجزهٔ موسی گذشت
&&
هر دو را از بام بود افتاد طشت
\\
بانگ طشت سحر جز لعنت چه ماند
&&
بانگ طشت دین به جز رفعت چه ماند
\\
چون محک پنهان شدست از مرد و زن
&&
در صف آ ای قلب و اکنون لاف زن
\\
وقت لافستت محک چون غایبست
&&
می‌برندت از عزیزی دست دست
\\
قلب می‌گوید ز نخوت هر دمم
&&
ای زر خالص من از تو کی کمم
\\
زر همی‌گوید بلی ای خواجه‌تاش
&&
لیک می‌آید محک آماده باش
\\
مرگ تن هدیه‌ست بر اصحاب راز
&&
زر خالص را چه نقصانست گاز
\\
قلب اگر در خویش آخربین بدی
&&
آن سیه که آخر شد او اول شدی
\\
چون شدی اول سیه اندر لقا
&&
دور بودی از نفاق و از شقا
\\
کیمیای فضل را طالب بدی
&&
عقل او بر زرق او غالب بدی
\\
چون شکسته‌دل شدی از حال خویش
&&
جابر اشکستگان دیدی به پیش
\\
عاقبت را دید و او اشکسته شد
&&
از شکسته‌بند در دم بسته شد
\\
فضل مسها را سوی اکسیر راند
&&
آن زراندود از کرم محروم ماند
\\
ای زراندوده مکن دعوی ببین
&&
که نماند مشتریت اعمی چنین
\\
نور محشر چشمشان بینا کند
&&
چشم بندی ترا رسوا کند
\\
بنگر آنها را که آخر دیده‌اند
&&
حسرت جانها و رشک دیده‌اند
\\
بنگر آنها را که حالی دیده‌اند
&&
سر فاسد ز اصل سر ببریده‌اند
\\
پیش حالی‌بین که در جهلست و شک
&&
صبح صادق صبح کاذب هر دو یک
\\
صبح کاذب صد هزاران کاروان
&&
داد بر باد هلاکت ای جوان
\\
نیست نقدی کش غلط‌انداز نیست
&&
وای آن جان کش محک و گاز نیست
\\
\end{longtable}
\end{center}
