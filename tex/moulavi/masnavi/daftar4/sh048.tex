\begin{center}
\section*{بخش ۴۸ - نشستن دیو بر مقام سلیمان علیه‌السلام و تشبه کردن او به کارهای سلیمان  علیه‌السلام و فرق ظاهر میان هر دو سلیمان و دیو خویشتن را سلیمان بن داود نام کردن}
\label{sec:sh048}
\addcontentsline{toc}{section}{\nameref{sec:sh048}}
\begin{longtable}{l p{0.5cm} r}
ورچه عقلت هست با عقل دگر
&&
یار باش و مشورت کن ای پدر
\\
با دو عقل از بس بلاها وا رهی
&&
پای خود بر اوج گردونها نهی
\\
دیو گر خود را سلیمان نام کرد
&&
ملک برد و مملکت را رام کرد
\\
صورت کار سلیمان دیده بود
&&
صورت اندر سر دیوی می‌نمود
\\
خلق گفتند این سلیمان بی‌صفاست
&&
از سلیمان تا سلیمان فرقهاست
\\
او چو بیداریست این هم‌چون وسن
&&
هم‌چنانک آن حسن با این حسن
\\
دیو می‌گفتی که حق بر شکل من
&&
صورتی کردست خوش بر اهرمن
\\
دیو را حق صورت من داده است
&&
تا نیندازد شما را او بشست
\\
گر پدید آید به دعوی زینهار
&&
صورت او را مدارید اعتبار
\\
دیوشان از مکر این می‌گفت لیک
&&
می‌نمود این عکس در دلهای نیک
\\
نیست بازی با ممیز خاصه او
&&
که بود تمییز و عقلش غیب‌گو
\\
هیچ سحر و هیچ تلبیس و دغل
&&
می‌نبندد پرده بر اهل دول
\\
پس همی گفتند با خود در جواب
&&
بازگونه می‌روی ای کژ خطاب
\\
بازگونه رفت خواهی همچنین
&&
سوی دوزخ اسفل اندر سافلین
\\
او اگر معزول گشتست و فقیر
&&
هست در پیشانیش بدر منیر
\\
تو اگر انگشتری را برده‌ای
&&
دوزخی چون زمهریر افسرده‌ای
\\
ما ببوش و عارض و طاق و طرنب
&&
سر کجا که خود همی ننهیم سنب
\\
ور به غفلت ما نهیم او را جبین
&&
پنجهٔ مانع برآید از زمین
\\
که منه آن سر مرین سر زیر را
&&
هین مکن سجده مرین ادبار را
\\
کردمی من شرح این بس جان‌فزا
&&
گر نبودی غیرت و رشک خدا
\\
هم قناعت کن تو بپذیر این قدر
&&
تا بگویم شرح این وقتی دگر
\\
نام خود کرده سلیمان نبی
&&
روی‌پوشی می‌کند بر هر صبی
\\
در گذر از صورت و از نام خیز
&&
از لقب وز نام در معنی گریز
\\
پس بپرس از حد او وز فعل او
&&
در میان حد و فعل او را بجو
\\
\end{longtable}
\end{center}
