\begin{center}
\section*{بخش ۶۱ - نصیحت دنیا اهل دنیا را به زبان حال و بی‌وفایی خود را نمودن  به وفا طمع دارندگان ازو}
\label{sec:sh061}
\addcontentsline{toc}{section}{\nameref{sec:sh061}}
\begin{longtable}{l p{0.5cm} r}
گفت بنمودم دغل لیکن ترا
&&
از نصیحت باز گفتم ماجرا
\\
هم‌چنین دنیا اگر چه خوش شکفت
&&
بانگ زد هم بی‌وفایی خویش گفت
\\
اندرین کون و فساد ای اوستاد
&&
آن دغل کون و نصیحت آن فساد
\\
کون می‌گوید بیا من خوش‌پیم
&&
وآن فسادش گفته رو من لا شی‌ام
\\
ای ز خوبی بهاران لب گزان
&&
بنگر آن سردی و زردی خزان
\\
روز دیدی طلعت خورشید خوب
&&
مرگ او را یاد کن وقت غروب
\\
بدر را دیدی برین خوش چار طاق
&&
حسرتش را هم ببین اندر محاق
\\
کودکی از حسن شد مولای خلق
&&
بعد فردا شد خرف رسوای خلق
\\
گر تن سیمین‌تنان کردت شکار
&&
بعد پیری بین تنی چون پنبه‌زار
\\
ای بدیده لوتهای چرب خیز
&&
فضلهٔ آن را ببین در آب‌ریز
\\
مر خبث را گو که آن خوبیت کو
&&
بر طبق آن ذوق و آن نغزی و بو
\\
گوید او آن دانه بد من دام آن
&&
چون شدی تو صید شد دانه نهان
\\
بس انامل رشک استادان شده
&&
در صناعت عاقبت لرزان شده
\\
نرگس چشم خمار هم‌چو جان
&&
آخر اعمش بین و آب از وی چکان
\\
حیدری کاندر صف شیران رود
&&
آخر او مغلوب موشی می‌شود
\\
طبع تیز دوربین محترف
&&
چون خر پیرش ببین آخر خرف
\\
زلف جعد مشکبار عقل‌بر
&&
آخرا چون دم زشت خنگ خر
\\
خوش ببین کونش ز اول باگشاد
&&
وآخر آن رسواییش بین و فساد
\\
زانک او بنمود پیدا دام را
&&
پیش تو بر کند سبلت خام را
\\
پس مگو دنیا به تزویرم فریفت
&&
ورنه عقل من ز دامش می‌گریخت
\\
طوق زرین و حمایل بین هله
&&
غل و زنجیری شدست و سلسله
\\
همچنین هر جزو عالم می‌شمر
&&
اول و آخر در آرش در نظر
\\
هر که آخربین‌تر او مسعودتر
&&
هر که آخربین‌تر او مطرودتر
\\
روی هر یک چون مه فاخر ببین
&&
چونک اول دیده شد آخر ببین
\\
تا نباشی هم‌چو ابلیس اعوری
&&
نیم بیند نیم نی چون ابتری
\\
دید طین آدم و دینش ندید
&&
این جهان دید آن جهان‌بینش ندید
\\
فضل مردان بر زنان ای بو شجاع
&&
نیست بهر قوت و کسب و ضیاع
\\
ورنه شیر و پیل را بر آدمی
&&
فضل بودی بهر قوت ای عمی
\\
فضل مردان بر زن ای حالی‌پرست
&&
زان بود که مرد پایان بین‌ترست
\\
مرد کاندر عاقبت‌بینی خمست
&&
او ز اهل عاقبت چون زن کمست
\\
از جهان دو بانگ می‌آید به ضد
&&
تا کدامین را تو باشی مستعد
\\
آن یکی بانگش نشور اتقیا
&&
وان یکی بانگش فریب اشقیا
\\
من شکوفهٔ خارم ای خوش گرمدار
&&
گل بریزد من بمانم شاخ خار
\\
بانگ اشکوفه‌ش که اینک گل‌فروش
&&
بانگ خار او که سوی ما مکوش
\\
این پذیرفتی بماندی زان دگر
&&
که محب از ضد محبوبست کر
\\
آن یکی بانگ این که اینک حاضرم
&&
بانگ دیگر بنگر اندر آخرم
\\
حاضری‌ام هست چون مکر و کمین
&&
نقش آخر ز آینهٔ اول ببین
\\
چون یکی زین دو جوال اندر شدی
&&
آن دگر را ضد و نا درخور شدی
\\
ای خنک آنکو ز اول آن شنید
&&
کش عقول و مسمع مردان شنید
\\
خانه خالی یافت و جا را او گرفت
&&
غیر آنش کژ نماید یا شگفت
\\
کوزهٔ نو کو به خود بولی کشید
&&
آن خبث را آب نتواند برید
\\
در جهان هر چیز چیزی می‌کشد
&&
کفر کافر را و مرشد را رشد
\\
کهربا هم هست و مغناطیس هست
&&
تا تو آهن یا کهی آیی بشست
\\
برد مغناطیست ار تو آهنی
&&
ور کهی بر کهربا بر می‌تنی
\\
آن یکی چون نیست با اخیار یار
&&
لاجرم شد پهلوی فجار جار
\\
هست موسی پیش قبطی بس ذمیم
&&
هست هامان پیش سبطی بس رجیم
\\
جان هامان جاذب قبطی شده
&&
جان موسی طالب سبطی شده
\\
معدهٔ خر که کشد در اجتذاب
&&
معدهٔ آدم جذوب گندم آب
\\
گر تو نشناسی کسی را از ظلام
&&
بنگر او را کوش سازیدست امام
\\
\end{longtable}
\end{center}
