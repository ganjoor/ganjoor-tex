\begin{center}
\section*{بخش ۲۱ - تفسیر این حدیث کی مثل امتی کمثل سفینة نوح من تمسک بها  نجا و من تخلف عنها غرق}
\label{sec:sh021}
\addcontentsline{toc}{section}{\nameref{sec:sh021}}
\begin{longtable}{l p{0.5cm} r}
بهر این فرمود پیغامبر که من
&&
هم‌چو کشتی‌ام به طوفان زمن
\\
ما و اصحابم چو آن کشتی نوح
&&
هر که دست اندر زند یابد فتوح
\\
چونک با شیخی تو دور از زشتیی
&&
روز و شب سیاری و در کشتیی
\\
در پناه جان جان‌بخشی توی
&&
کشتی اندر خفته‌ای ره می‌روی
\\
مسکل از پیغامبر ایام خویش
&&
تکیه کم کن بر فن و بر کام خویش
\\
گرچه شیری چون روی ره بی‌دلیل
&&
خویش‌بین و در ضلالی و ذلیل
\\
هین مپر الا که با پرهای شیخ
&&
تا ببینی عون و لشکرهای شیخ
\\
یک زمانی موج لطفش بال تست
&&
آتش قهرش دمی حمال تست
\\
قهر او را ضد لطفش کم شمر
&&
اتحاد هر دو بین اندر اثر
\\
یک زمان چون خاک سبزت می‌کند
&&
یک زمان پر باد و گبزت می‌کند
\\
جسم عارف را دهد وصف جماد
&&
تا برو روید گل و نسرین شاد
\\
لیک او بیند نبیند غیر او
&&
جز به مغز پاک ندهد خلد بو
\\
مغز را خالی کن از انکار یار
&&
تا که ریحان یابد از گلزار یار
\\
تا بیابی بوی خلد از یار من
&&
چون محمد بوی رحمن از یمن
\\
در صف معراجیان گر بیستی
&&
چون براقت بر کشاند نیستی
\\
نه چو معراج زمینی تا قمر
&&
بلک چون معراج کلکی تا شکر
\\
نه چو معراج بخاری تا سما
&&
بل چو معراج جنینی تا نهی
\\
خوش براقی گشت خنگ نیستی
&&
سوی هستی آردت گر نیستی
\\
کوه و دریاها سمش مس می‌کند
&&
تا جهان حس را پس می‌کند
\\
پا بکش در کشتی و می‌رو روان
&&
چون سوی معشوق جان جان روان
\\
دست نه و پای نه رو تا قدم
&&
آن چنانک تاخت جانها از عدم
\\
بردریدی در سخن پردهٔ قیاس
&&
گر نبودی سمع سامع را نعاس
\\
ای فلک بر گفت او گوهر ببار
&&
از جهان او جهانا شرم دار
\\
گر بباری گوهرت صد تا شود
&&
جامدت بیننده و گویا شود
\\
پس نثاری کرده باشی بهر خود
&&
چونک هر سرمایهٔ تو صد شود
\\
\end{longtable}
\end{center}
