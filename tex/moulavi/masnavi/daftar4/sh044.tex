\begin{center}
\section*{بخش ۴۴ - بقیهٔ عمارت کردن سلیمان علیه‌السلام مسجد اقصی را به تعلیم و وحی خدا جهت حکمتهایی کی او داند و معاونت ملایکه و دیو  و پری و آدمی آشکارا}
\label{sec:sh044}
\addcontentsline{toc}{section}{\nameref{sec:sh044}}
\begin{longtable}{l p{0.5cm} r}
ای سلیمان مسجد اقصی بساز
&&
لشکر بلقیس آمد در نماز
\\
چونک او بنیاد آن مسجد نهاد
&&
جن و انس آمد بدن در کار داد
\\
یک گروه از عشق و قومی بی‌مراد
&&
هم‌چنانک در ره طاعت عباد
\\
خلق دیوانند و شهوت سلسله
&&
می‌کشدشان سوی دکان و غله
\\
هست این زنجیر از خوف و وله
&&
تو مبین این خلق را بی‌سلسله
\\
می‌کشاندشان سوی کسب و شکار
&&
می‌کشاندشان سوی کان و بحار
\\
می‌کشدشان سوی نیک و سوی بد
&&
گفت حق فی جیدها حبل المسد
\\
قد جعلنا الحبل فی اعناقهم
&&
واتخذنا الحبل من اخلاقهم
\\
لیس من مستقذر مستنقه
&&
قط الا طایره فی عنقه
\\
حرص تو در کار بد چون آتشست
&&
اخگر از رنگ خوش آتش خوشست
\\
آن سیاهی فحم در آتش نهان
&&
چونک آتش شد سیاهی شد عیان
\\
اخگر از حرص تو شد فحم سیاه
&&
حرص چون شد ماند آن فحم تباه
\\
آن زمان آن فحم اخگر می‌نمود
&&
آن نه حسن کار نار حرص بود
\\
حرص کارت را بیاراییده بود
&&
حرص رفت و ماند کار تو کبود
\\
غوله‌ای را که بر آرایید غول
&&
پخته پندارد کسی که هست گول
\\
آزمایش چون نماید جان او
&&
کند گردد ز آزمون دندان او
\\
از هوس آن دام دانه می‌نمود
&&
عکس غول حرص و آن خود خام بود
\\
حرص اندر کار دین و خیر جو
&&
چون نماند حرص باشد نغزرو
\\
خیرها نغزند نه از عکس غیر
&&
تاب حرص ار رفت ماند تاب خیر
\\
تاب حرص از کار دنیا چون برفت
&&
فحم باشد مانده از اخگر بتفت
\\
کودکان را حرص می‌آرد غرار
&&
تا شوند از ذوق دل دامن‌سوار
\\
چون ز کودک رفت آن حرص بدش
&&
بر دگر اطفال خنده آیدش
\\
که چه می‌کردم چه می‌دیدم درین
&&
خل ز عکس حرص بنمود انگبین
\\
آن بنای انبیا بی حرص بود
&&
زان چنان پیوسته رونقها فزود
\\
ای بسا مسجد بر آورده کرام
&&
لیک نبود مسجد اقصاش نام
\\
کعبه را که هر دمی عزی فزود
&&
آن ز اخلاصات ابراهیم بود
\\
فضل آن مسجد خاک و سنگ نیست
&&
لیک در بناش حرص و جنگ نیست
\\
نه کتبشان مثل کتب دیگران
&&
نی مساجدشان نی کسب وخان و مان
\\
نه ادبشان نه غضبشان نه نکال
&&
نه نعاس و نه قیاس و نه مقال
\\
هر یکیشان را یکی فری دگر
&&
مرغ جانشان طایر از پری دگر
\\
دل همی لرزد ز ذکر حالشان
&&
قبلهٔ افعال ما افعالشان
\\
مرغشان را بیضه‌ها زرین بدست
&&
نیم‌شب جانشان سحرگه بین شدست
\\
هر چه گویم من به جان نیکوی قوم
&&
نقص گفتم گشته ناقص‌گوی قوم
\\
مسجد اقصی بسازید ای کرام
&&
که سلیمان باز آمد والسلام
\\
ور ازین دیوان و پریان سر کشند
&&
جمله را املاک در چنبر کشند
\\
دیو یک دم کژ رود از مکر و زرق
&&
تازیانه آیدش بر سر چو برق
\\
چون سلیمان شو که تا دیوان تو
&&
سنگ برند از پی ایوان تو
\\
چون سلیمان باش بی‌وسواس و ریو
&&
تا ترا فرمان برد جنی و دیو
\\
خاتم تو این دلست و هوش دار
&&
تا نگردد دیو را خاتم شکار
\\
پس سلیمانی کند بر تو مدام
&&
دیو با خاتم حذر کن والسلام
\\
آن سلیمانی دلا منسوخ نیست
&&
در سر و سرت سلیمانی کنیست
\\
دیو هم وقتی سلیمانی کند
&&
لیک هر جولاهه اطلس کی تند
\\
دست جنباند چو دست او ولیک
&&
در میان هر دوشان فرقیست نیک
\\
\end{longtable}
\end{center}
