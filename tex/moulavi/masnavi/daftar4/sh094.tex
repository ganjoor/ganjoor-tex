\begin{center}
\section*{بخش ۹۴ - باز گفتن موسی علیه‌السلام اسرار فرعون را و واقعات او را ظهر الغیب تابخبیری حق ایمان آورد یا گمان برد}
\label{sec:sh094}
\addcontentsline{toc}{section}{\nameref{sec:sh094}}
\begin{longtable}{l p{0.5cm} r}
ز آهن تیره بقدرت می‌نمود
&&
واقعاتی که در آخر خواست بود
\\
تا کنی کمتر تو آن ظلم و بدی
&&
آن همی‌دیدی و بتر می‌شدی
\\
نقشهای زشت خوابت می‌نمود
&&
می‌رمیدی زان و آن نقش تو بود
\\
هم‌چو آن زنگی که در آیینه دید
&&
روی خود را زشت و بر آیینه رید
\\
که چه زشتی لایق اینی و بس
&&
زشتیم آن تواست ای کور خس
\\
این حدث بر روی زشتت می‌کنی
&&
نیست بر من زانک هستم روشنی
\\
گاه می‌دیدی لباست سوخته
&&
گه دهان و چشم تو بر دوخته
\\
گاه حیوان قاصد خونت شده
&&
گه سر خود را به دندان دده
\\
گه نگون اندر میان آبریز
&&
گه غریق سیل خون‌آمیز تیز
\\
گه ندات آمد ازین چرخ نقی
&&
که شقیی و شقیی و شقی
\\
گه ندات آمد صریحا از جبال
&&
که برو هستی ز اصحاب الشمال
\\
گه ندا می‌آمدت از هر جماد
&&
تا ابد فرعون در دوزخ فتاد
\\
زین بترها که نمی‌گویم ز شرم
&&
تا نگردد طبع معکوس تو گرم
\\
اندکی گفتم به تو ای ناپذیر
&&
ز اندکی دانی که هستم من خبیر
\\
خویشتن را کور می‌کردی و مات
&&
تا نیندیشی ز خواب و واقعات
\\
چند بگریزی نک آمد پیش تو
&&
کوری ادراک مکراندیش تو
\\
\end{longtable}
\end{center}
