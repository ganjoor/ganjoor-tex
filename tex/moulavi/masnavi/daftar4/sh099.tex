\begin{center}
\section*{بخش ۹۹ - غره شدن آدمی به ذکاوت و تصویرات طبع خویشتن و طلب ناکردن علم  غیب کی علم انبیاست}
\label{sec:sh099}
\addcontentsline{toc}{section}{\nameref{sec:sh099}}
\begin{longtable}{l p{0.5cm} r}
دیدم اندر خانه من نقش و نگار
&&
بودم اندر عشق خانه بی‌قرار
\\
بودم از گنج نهانی بی‌خبر
&&
ورنه دستنبوی من بودی تبر
\\
آه گر داد تبر را دادمی
&&
این زمان غم را تبرا دادمی
\\
چشم را بر نقش می‌انداختم
&&
هم‌چو طفلان عشقها می‌باختم
\\
پس نکو گفت آن حکیم کامیار
&&
که تو طفلی خانه پر نقش و نگار
\\
در الهی‌نامه بس اندرز کرد
&&
که بر آر دودمان خویش گرد
\\
بس کن ای موسی بگو وعدهٔ سوم
&&
که دل من ز اضطرابش گشت گم
\\
گفت موسی آن سوم ملک دوتو
&&
دو جهانی خالص از خصم و عدو
\\
بیشتر زان ملک که اکنون داشتی
&&
کان بد اندر جنگ و این در آشتی
\\
آنک در جنگت چنان ملکی دهد
&&
بنگر اندر صلح خوانت چون نهد
\\
آن کرم که اندر جفا آنهات داد
&&
در وفا بنگر چه باشد افتقاد
\\
گفت ای موسی چهارم چیست زود
&&
بازگو صبرم شد و حرصم فزود
\\
گفت چارم آنک مانی تو جوان
&&
موی هم‌چون قیر و رخ چون ارغوان
\\
رنگ و بو در پیش ما بس کاسدست
&&
لیک تو پستی سخن کردیم پست
\\
افتخار از رنگ و بو و از مکان
&&
هست شادی و فریب کودکان
\\
\end{longtable}
\end{center}
