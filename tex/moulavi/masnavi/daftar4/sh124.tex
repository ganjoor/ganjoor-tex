\begin{center}
\section*{بخش ۱۲۴ - بیان آنک مجموع عالم صورت عقل کلست چون با عقل کل بکژروی جفا کردی صورت عالم ترا غم فزاید اغلب احوال چنانک دل با پدر بد کردی صورت پدر غم فزاید ترا و نتوانی رویش را دیدن اگر چه پیش از آن نور دیده بوده باشد و راحت جان}
\label{sec:sh124}
\addcontentsline{toc}{section}{\nameref{sec:sh124}}
\begin{longtable}{l p{0.5cm} r}
کل عالم صورت عقل کلست
&&
کوست بابای هر آنک اهل قل است
\\
چون کسی با عقل کل کفران فزود
&&
صورت کل پیش او هم سگ نمود
\\
صلح کن با این پدر عاقی بهل
&&
تا که فرش زر نماید آب و گل
\\
پس قیامت نقد حال تو بود
&&
پیش تو چرخ و زمین مبدل شود
\\
من که صلحم دایما با این پدر
&&
این جهان چون جنتستم در نظر
\\
هر زمان نو صورتی و نو جمال
&&
تا ز نو دیدن فرو میرد ملال
\\
من همی‌بینم جهان را پر نعیم
&&
آبها از چشمه‌ها جوشان مقیم
\\
بانگ آبش می‌رسد در گوش من
&&
مست می‌گردد ضمیر و هوش من
\\
شاخه‌ها رقصان شده چون تایبان
&&
برگها کف‌زن مثال مطربان
\\
برق آیینه‌ست لامع از نمد
&&
گر نماید آینه تا چون بود
\\
از هزاران می‌نگویم من یکی
&&
ز آنک آکندست هر گوش از شکی
\\
پیش وهم این گفت مژده دادنست
&&
عقل گوید مژده چه نقد منست
\\
\end{longtable}
\end{center}
