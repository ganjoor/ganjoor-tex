\begin{center}
\section*{بخش ۱۰۳ - قصهٔ باز پادشاه و کمپیر زن}
\label{sec:sh103}
\addcontentsline{toc}{section}{\nameref{sec:sh103}}
\begin{longtable}{l p{0.5cm} r}
باز اسپیدی به کمپیری دهی
&&
او ببرد ناخنش بهر بهی
\\
ناخنی که اصل کارست و شکار
&&
کور کمپیری ببرد کوروار
\\
که کجا بودست مادر که ترا
&&
ناخنان زین سان درازست ای کیا
\\
ناخن و منقار و پرش را برید
&&
وقت مهر این می‌کند زال پلید
\\
چونک تتماجش دهد او کم خورد
&&
خشم گیرد مهرها را بر درد
\\
که چنین تتماج پختم بهر تو
&&
تو تکبر می‌نمایی و عتو
\\
تو سزایی در همان رنج و بلا
&&
نعمت و اقبال کی سازد ترا
\\
آن تتماجش دهد کین را بگیر
&&
گر نمی‌خواهی که نوشی زان فطیر
\\
آب تتماجش نگیرد طبع باز
&&
زال بترنجد شود خشمش دراز
\\
از غضب شربای سوزان بر سرش
&&
زن فرو ریزد شود کل مغفرش
\\
اشک از آن چشمش فرو ریزد ز سوز
&&
یاد آرد لطف شاه دل‌فروز
\\
زان دو چشم نازنین با دلال
&&
که ز چهرهٔ شاد دارد صد کمال
\\
چشم مازاغش شده پر زخم زاغ
&&
چشم نیک از چشم بد با درد و داغ
\\
چشم دریا بسطتی کز بسط او
&&
هر دو عالم می‌نماید تار مو
\\
گر هزاران چرخ در چشمش رود
&&
هم‌چو چشمه پیش قلزم گم شود
\\
چشم بگذشته ازین محسوسها
&&
یافته از غیب‌بینی بوسها
\\
خود نمی‌یابم یکی گوشی که من
&&
نکته‌ای گویم از آن چشم حسن
\\
می‌چکید آن آب محمود جلیل
&&
می‌ربودی قطره‌اش را جبرئیل
\\
تا بمالد در پر و منقال خویش
&&
گر دهد دستوریش آن خوب کیش
\\
باز گوید خشم کمپیر ار فروخت
&&
فر و نور و علم و صبرم را نسوخت
\\
باز جانم باز صد صورت تند
&&
زخم بر ناقه نه بر صالح زند
\\
صالح از یک‌دم که آرد با شکوه
&&
صد چنان ناقه بزاید متن کوه
\\
دل همی گوید خموش و هوش دار
&&
ورنه درانید غیرت پود و تار
\\
غیرتش را هست صد حلم نهان
&&
ورنه سوزیدی به یک دم صد جهان
\\
نخوت شاهی گرفتش جای پند
&&
تا دل خود را ز بند پند کند
\\
که کنم بار رای هامان مشورت
&&
کوست پشت ملک و قطب مقدرت
\\
مصطفی را رای‌زن صدیق رب
&&
رای‌زن بوجهل را شد بولهب
\\
عرق جنسیت چنانش جذب کرد
&&
کان نصیحتها به پیشش گشت سرد
\\
جنس سوی جنس صد پره پرد
&&
بر خیالش بندها را بر درد
\\
\end{longtable}
\end{center}
