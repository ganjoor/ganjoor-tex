\begin{center}
\section*{بخش ۱۳۲ - در خواستن قبطی دعای خیر و هدایت از سبطی و دعا کردن سبطی  قبطی را به خیر و مستجاب شدن از اکرم الاکرمین وارحم الراحمین}
\label{sec:sh132}
\addcontentsline{toc}{section}{\nameref{sec:sh132}}
\begin{longtable}{l p{0.5cm} r}
گفت قبطی تو دعایی کن که من
&&
از سیاهی دل ندارم آن دهن
\\
که بود که قفل این دل وا شود
&&
زشت را در بزم خوبان جا شود
\\
مسخی از تو صاحب خوبی شود
&&
یا بلیسی باز کروبی شود
\\
یا بفر دست مریم بوی مشک
&&
یابد و تری و میوه شاخ خشک
\\
سبطی آن دم در سجود افتاد و گفت
&&
کای خدای عالم جهر و نهفت
\\
جز تو پیش کی بر آرد بنده دست
&&
هم دعا و هم اجابت از توست
\\
هم ز اول تو دهی میل دعا
&&
تو دهی آخر دعاها را جزا
\\
اول و آخر توی ما در میان
&&
هیچ هیچی که نیاید در بیان
\\
این چنین می‌گفت تا افتاد طشت
&&
از سر بام و دلش بیهوش گشت
\\
باز آمد او به هوش اندر دعا
&&
لیس للانسان الا ما سعی
\\
در دعا بود او که ناگه نعره‌ای
&&
از دل قبطی بجست و غره‌ای
\\
که هلا بشتاب و ایمان عرضه کن
&&
تا ببرم زود زنار کهن
\\
آتشی در جان من انداختند
&&
مر بلیسی را به جان بنواختند
\\
دوستی تو و از تو ناشکفت
&&
حمدلله عاقبت دستم گرفت
\\
کیمیایی بود صحبتهای تو
&&
کم مباد از خانهٔ دل پای تو
\\
تو یکی شاخی بدی از نخل خلد
&&
چون گرفتم او مرا تا خلد برد
\\
سیل بود آنک تنم را در ربود
&&
برد سیلم تا لب دریای جود
\\
من به بوی آب رفتم سوی سیل
&&
بحر دیدم در گرفتم کیل کیل
\\
طاس آوردش که اکنون آب‌گیر
&&
گفت رو شد آبها پیشم حقیر
\\
شربتی خوردم ز الله اشتری
&&
تا به محشر تشنگی ناید مرا
\\
آنک جوی و چشمه‌ها را آب داد
&&
چشمه‌ای در اندرون من گشاد
\\
این جگر که بود گرم و آب‌خوار
&&
گشت پیش همت او آب خوار
\\
کاف کافی آمد او بهر عباد
&&
صدق وعدهٔ کهیعص
\\
کافیم بدهم ترا من جمله خیر
&&
بی‌سبب بی‌واسطهٔ یاری غیر
\\
کافیم بی‌نان ترا سیری دهم
&&
بی‌سپاه و لشکرت میری دهم
\\
بی‌بهارت نرگس و نسرین دهم
&&
بی‌کتاب و اوستا تلقین دهم
\\
کافیم بی داروت درمان کنم
&&
گور را و چاه را میدان کنم
\\
موسیی را دل دهم با یک عصا
&&
تا زند بر عالمی شمشیرها
\\
دست موسی را دهم یک نور و تاب
&&
که طپانچه می‌زند بر آفتاب
\\
چوب را ماری کنم من هفت سر
&&
که نزاید ماده مار او را ز نر
\\
خون نیامیزم در آب نیل من
&&
خود کنم خون عین آبش را به فن
\\
شادیت را غم کنم چون آب نیل
&&
که نیابی سوی شادیها سبیل
\\
باز چون تجدید ایمان بر تنی
&&
باز از فرعون بیزاری کنی
\\
موسی رحمت ببینی آمده
&&
نیل خون بینی ازو آبی شده
\\
چون سر رشته نگه داری درون
&&
نیل ذوق تو نگردد هیچ خون
\\
من گمان بردم که ایمان آورم
&&
تا ازین طوفان خون آبی خورم
\\
من چه دانستم که تبدیلی کند
&&
در نهاد من مرا نیلی کند
\\
سوی چشم خود یکی نیلم روان
&&
برقرارم پیش چشم دیگران
\\
هم‌چنانک این جهان پیش نبی
&&
غرق تسبیحست و پیش ما غبی
\\
پیش چشمش این جهان پر عشق و داد
&&
پیش چشم دیگران مرده و جماد
\\
پست و بالا پیش چشمش تیزرو
&&
از کلوخ و خشت او نکته شنو
\\
با عوام این جمله بسته و مرده‌ای
&&
زین عجب‌تر من ندیدم پرده‌ای
\\
گورها یکسان به پیش چشم ما
&&
روضه و حفره به چشم اولیا
\\
عامه گفتندی که پیغامبر ترش
&&
از چه گشتست و شدست او ذوق‌کش
\\
خاص گفتندی که سوی چشمتان
&&
می‌نماید او ترش ای امتان
\\
یک زمان درچشم ما آیید تا
&&
خنده‌ها بینید اندر هل اتی
\\
از سر امرود بن بنماید آن
&&
منعکس صورت بزیر آ ای جوان
\\
آن درخت هستی است امرودبن
&&
تا بر آنجایی نماید نو کهن
\\
تا بر آنجایی ببینی خارزار
&&
پر ز کزدمهای خشم و پر ز مار
\\
چون فرود آیی ببینی رایگان
&&
یک جهان پر گل‌رخان و دایگان
\\
\end{longtable}
\end{center}
