\begin{center}
\section*{بخش ۶۶ - حکایت آن مداح کی از جهت ناموس شکر ممدوح می‌کرد و بوی اندوه و غم اندرون او و خلاقت دلق ظاهر او می‌نمود کی آن  شکرها لافست و دروغ}
\label{sec:sh066}
\addcontentsline{toc}{section}{\nameref{sec:sh066}}
\begin{longtable}{l p{0.5cm} r}
آن یکی با دلق آمد از عراق
&&
باز پرسیدند یاران از فراق
\\
گفت آری بد فراق الا سفر
&&
بود بر من بس مبارک مژده‌ور
\\
که خلیفه داد ده خلعت مرا
&&
که قرینش باد صد مدح و ثنا
\\
شکرها و حمدها بر می‌شمرد
&&
تا که شکر از حد و اندازه ببرد
\\
پس بگفتندش که احوال نژند
&&
بر دروغ تو گواهی می‌دهند
\\
تن برهنه سر برهنه سوخته
&&
شکر را دزدیده یا آموخته
\\
کو نشان شکر و حمد میر تو
&&
بر سر و بر پای بی توفیر تو
\\
گر زبانت مدح آن شه می‌تند
&&
هفت اندامت شکایت می‌کند
\\
در سخای آن شه و سلطان جود
&&
مر ترا کفشی و شلواری نبود
\\
گفت من ایثار کردم آنچ داد
&&
میر تقصیری نکرد از افتقاد
\\
بستدم جمله عطاها از امیر
&&
بخش کردم بر یتیم و بر فقیر
\\
مال دادم بستدم عمر دراز
&&
در جزا زیرا که بودم پاک‌باز
\\
پس بگفتندش مبارک مال رفت
&&
چیست اندر باطنت این دود نفت
\\
صد کراهت در درون تو چو خار
&&
کی بود انده نشان ابتشار
\\
کو نشان عشق و ایثار و رضا
&&
گر درستست آنچ گفتی ما مضی
\\
خود گرفتم مال گم شد میل کو
&&
سیل اگر بگذشت جای سیل کو
\\
چشم تو گر بد سیاه و جان‌فزا
&&
گر نماند او جان‌فزا ازرق چرا
\\
کو نشان پاک‌بازی ای ترش
&&
بوی لاف کژ همی‌آید خمش
\\
صد نشان باشد درون ایثار را
&&
صد علامت هست نیکوکار را
\\
مال در ایثار اگر گردد تلف
&&
در درون صد زندگی آید خلف
\\
در زمین حق زراعت کردنی
&&
تخمهای پاک آنگه دخل نی
\\
گر نروید خوشه از روضات هو
&&
پس چه واسع باشد ارض الله بگو
\\
چونک این ارض فنا بی‌ریع نیست
&&
چون بود ارض الله آن مستوسعیست
\\
این زمین را ریع او خود بی‌حدست
&&
دانه‌ای را کمترین خود هفصدست
\\
حمد گفتی کو نشان حامدون
&&
نه برونت هست اثر نه اندرون
\\
حمد عارف مر خدا را راستست
&&
که گواه حمد او شد پا و دست
\\
از چه تاریک جسمش بر کشید
&&
وز تک زندان دنیااش خرید
\\
اطلس تقوی و نور مؤتلف
&&
آیت حمدست او را بر کتف
\\
وا رهیده از جهان عاریه
&&
ساکن گلزار و عین جاریه
\\
بر سریر سر عالی‌همتش
&&
مجلس و جا و مقام و رتبتش
\\
مقعد صدقی که صدیقان درو
&&
جمله سر سبزند و شاد و تازه‌رو
\\
حمدشان چون حمد گلشن از بهار
&&
صد نشانی دارد و صد گیر و دار
\\
بر بهارش چشمه و نخل و گیاه
&&
وآن گلستان و نگارستان گواه
\\
شاهد شاهد هزاران هر طرف
&&
در گواهی هم‌چو گوهر بر صدف
\\
بوی سر بد بیاید از دمت
&&
وز سر و رو تابد ای لافی غمت
\\
بوشناسانند حاذق در مصاف
&&
تو به جلدی های هو کم کن گزاف
\\
تو ملاف از مشک کان بوی پیاز
&&
از دم تو می‌کند مکشوف راز
\\
گل‌شکر خوردم همی‌گویی و بوی
&&
می‌زند از سیر که یافه مگوی
\\
هست دل مانندهٔ خانهٔ کلان
&&
خانهٔ دل را نهان همسایگان
\\
از شکاف روزن و دیوارها
&&
مطلع گردند بر اسرار ما
\\
از شکافی که ندارد هیچ وهم
&&
صاحب خانه و ندارد هیچ سهم
\\
از نبی بر خوان که دیو و قوم او
&&
می‌برند از حال انسی خفیه بو
\\
از رهی که انس از آن آگاه نیست
&&
زانک زین محسوس و زین اشباه نیست
\\
در میان ناقدان زرقی متن
&&
با محک ای قلب دون لافی مزن
\\
مر محک را ره بود در نقد و قلب
&&
که خدایش کرد امیر جسم و قلب
\\
چون شیاطین با غلیظیهای خویش
&&
واقف‌اند از سر ما و فکر و کیش
\\
مسلکی دارند دزدیده درون
&&
ما ز دزدیهای ایشان سرنگون
\\
دم به دم خبط و زیانی می‌کنند
&&
صاحب نقب و شکاف روزنند
\\
پس چرا جان‌های روشن در جهان
&&
بی‌خبر باشند از حال نهان
\\
در سرایت کمتر از دیوان شدند
&&
روحها که خیمه بر گردون زدند
\\
دیو دزدانه سوی گردون رود
&&
از شهاب محرق او مطعون شود
\\
سرنگون از چرخ زیر افتد چنان
&&
که شقی در جنگ از زخم سنان
\\
آن ز رشک روحهای دل‌پسند
&&
از فلکشان سرنگون می‌افکنند
\\
تو اگر شلی و لنگ و کور و کر
&&
این گمان بر روحهای مه مبر
\\
شرم دار و لاف کم زن جان مکن
&&
که بسی جاسوس هست آن سوی تن
\\
\end{longtable}
\end{center}
