\begin{center}
\section*{بخش ۱۱۰ - جواب دهری کی منکر الوهیت است و عالم را قدیم می‌گوید}
\label{sec:sh110}
\addcontentsline{toc}{section}{\nameref{sec:sh110}}
\begin{longtable}{l p{0.5cm} r}
دی یکی می‌گفت عالم حادثست
&&
فانیست این چرخ و حقش وارثست
\\
فلسفیی گفت چون دانی حدوث
&&
حادثی ابر چون داند غیوث
\\
ذره‌ای خود نیستی از انقلاب
&&
تو چه می‌دانی حدوث آفتاب
\\
کرمکی کاندر حدث باشد دفین
&&
کی بداند آخر و بدو زمین
\\
این به تقلید از پدر بشنیده‌ای
&&
از حماقت اندرین پیچیده‌ای
\\
چیست برهان بر حدوث این بگو
&&
ورنه خامش کن فزون گویی مجو
\\
گفت دیدم اندرین بحث عمیق
&&
بحث می‌کردند روزی دو فریق
\\
در جدال و در خصام و در ستوه
&&
گشت هنگامه بر آن دو کس گروه
\\
من به سوی جمع هنگامه شدم
&&
اطلاع از حال ایشان بستدم
\\
آن یکی می‌گفت گردون فانیست
&&
بی‌گمانی این بنا را بانیست
\\
وان دگر گفت این قدیم و بی کیست
&&
نیستش بانی و یا بانی ویست
\\
گفت منکر گشته‌ای خلاق را
&&
روز و شب آرنده و رزاق را
\\
گفت بی برهان نخواهم من شنید
&&
آنچ گولی آن به تقلیدی گزید
\\
هین بیاور حجت و برهان که من
&&
نشنوم بی حجت این را در زمن
\\
گفت حجت در درون جانمست
&&
در درون جان نهان برهانمست
\\
تو نمی‌بینی هلال از ضعف چشم
&&
من همی بینم مکن بر من تو خشم
\\
گفت و گو بسیار گشت و خلق گیج
&&
در سر و پایان این چرخ پسیج
\\
گفت یارا در درونم حجتیست
&&
بر حدوث آسمانم آیتیست
\\
من یقین دارم نشانش آن بود
&&
مر یقین‌دان را که در آتش رود
\\
در زبان می‌ناید آن حجت بدان
&&
هم‌چو حال سر عشق عاشقان
\\
نیست پیدا سر گفت و گوی من
&&
جز که زردی و نزاری روی من
\\
اشک و خون بر رخ روانه می‌دود
&&
حجت حسن و جمالش می‌شود
\\
گفت من اینها ندانم حجتی
&&
که بود در پیش عامه آیتی
\\
گفت چون قلبی و نقدی دم زنند
&&
که تو قلبی من تکویم ارجمند
\\
هست آتش امتحان آخرین
&&
کاندر آتش در فتند این دو قرین
\\
عام و خاص از حالشان عالم شوند
&&
از گمان و شک سوی ایقان روند
\\
آب و آتش آمد ای جان امتحان
&&
نقد و قلبی را که آن باشد نهان
\\
تا من و تو هر دو در آتش رویم
&&
حجت باقی حیرانان شویم
\\
تا من و تو هر دو در بحر اوفتیم
&&
که من و تو این کره را آیتیم
\\
هم‌چنان کردند و در آتش شدند
&&
هر دو خود را بر تف آتش زدند
\\
از خدا گوینده مرد مدعی
&&
رست و سوزید اندر آتش آن دعی
\\
از مؤذن بشنو این اعلام را
&&
کوری افزون‌روان خام را
\\
که نسوزیدست این نام از اجل
&&
کش مسمی صدر بودست و اجل
\\
صد هزاران زین رهان اندر قران
&&
بر دریده پرده‌های منکران
\\
چون گرو بستند غالب شد صواب
&&
در دوام و معجزات و در جواب
\\
فهم کردم کانک دم زد از سبق
&&
وز حدوث چرخ پیروزست و حق
\\
حجت منکر هماره زردرو
&&
یک نشان بر صدق آن انکار کو
\\
یک مناره در ثنای منکران
&&
کو درین عالم که تا باشد نشان
\\
منبری کو که بر آنجا مخبری
&&
یاد آرد روزگار منکری
\\
روی دینار و درم از نامشان
&&
تا قیامت می‌دهد زین حق نشان
\\
سکهٔ شاهان همی گردد دگر
&&
سکهٔ احمد ببین تا مستقر
\\
بر رخ نقره و یا روی زری
&&
وا نما بر سکه نام منکری
\\
خود مگیر این معجز چون آفتاب
&&
صد زبان بین نام او ام‌الکتاب
\\
زهره نی کس را که یک حرفی از آن
&&
یا بدزدد یا فزاید در بیان
\\
یار غالب شو که تا غالب شوی
&&
یار مغلوبان مشو هین ای غوی
\\
حجت منکر همین آمد که من
&&
غیر این ظاهر نمی‌بینم وطن
\\
هیچ نندیشد که هر جا ظاهریست
&&
آن ز حکمتهای پنهان مخبریست
\\
فایدهٔ هر ظاهری خود باطنیست
&&
هم‌چو نفع اندر دواها کامنست
\\
\end{longtable}
\end{center}
