\begin{center}
\section*{بخش ۲۰ - در بیان آنک حکما گویند آدمی عالم صغریست و حکمای اللهی گویند آدمی عالم کبریست زیرا آن علم حکما بر صورت آدمی مقصور بود و علم این حکما در حقیقت حقیقت آدمی موصول بود}
\label{sec:sh020}
\addcontentsline{toc}{section}{\nameref{sec:sh020}}
\begin{longtable}{l p{0.5cm} r}
پس به صورت عالم اصغر توی
&&
پس به معنی عالم اکبر توی
\\
ظاهر آن شاخ اصل میوه است
&&
باطنا بهر ثمر شد شاخ هست
\\
گر نبودی میل و اومید ثمر
&&
کی نشاندی باغبان بیخ شجر
\\
پس به معنی آن شجر از میوه زاد
&&
گر به صورت از شجر بودش ولاد
\\
مصطفی زین گفت که آدم و انبیا
&&
خلف من باشند در زیر لوا
\\
بهر این فرموده است آن ذو فنون
&&
رمز نحن اخرون السابقون
\\
گر بصورت من ز آدم زاده‌ام
&&
من به معنی جد جد افتاده‌ام
\\
کز برای من بدش سجدهٔ ملک
&&
وز پی من رفت بر هفتم فلک
\\
پس ز من زایید در معنی پدر
&&
پس ز میوه زاد در معنی شجر
\\
اول فکر آخر آمد در عمل
&&
خاصه فکری کو بود وصف ازل
\\
حاصل اندر یک زمان از آسمان
&&
می‌رود می‌آید ایدر کاروان
\\
نیست بر این کاروان این ره دراز
&&
کی مفازه زفت آید با مفاز
\\
دل به کعبه می‌رود در هر زمان
&&
جسم طبع دل بگیرد ز امتنان
\\
این دراز و کوتهی مر جسم راست
&&
چه دراز و کوته آنجا که خداست
\\
چون خدا مر جسم را تبدیل کرد
&&
رفتنش بی‌فرسخ و بی‌میل کرد
\\
صد امیدست این زمان بردار گام
&&
عاشقانه ای فتی خل الکلام
\\
گرچه پلهٔ چشم بر هم می‌زنی
&&
در سفینه خفته‌ای ره می‌کنی
\\
\end{longtable}
\end{center}
