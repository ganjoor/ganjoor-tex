\begin{center}
\section*{بخش ۱۱۳ - خشم کردن پادشاه بر ندیم و شفاعت کردن شفیع آن مغضوب علیه را و از پادشاه درخواستن و پادشاه شفاعت او قبول کردن و رنجیدن ندیم از این شفیع کی چرا شفاعت کردی}
\label{sec:sh113}
\addcontentsline{toc}{section}{\nameref{sec:sh113}}
\begin{longtable}{l p{0.5cm} r}
پادشاهی بر ندیمی خشم کرد
&&
خواست تا از وی برآرد دود و گرد
\\
کرد شه شمشیر بیرون از غلاف
&&
تا زند بر وی جزای آن خلاف
\\
هیچ کس را زهره نه تا دم زند
&&
یا شفیعی بر شفاعت بر تند
\\
جز عمادالملک نامی در خواص
&&
در شفاعت مصطفی‌وارانه خاص
\\
بر جهید و زود در سجده فتاد
&&
در زمان شه تیغ قهر از کف نهاد
\\
گفت اگر دیوست من بخشیدمش
&&
ور بلیسی کرد من پوشیدمش
\\
چونک آمد پای تو اندر میان
&&
راضیم گر کرد مجرم صد زیان
\\
صد هزاران خشم را توانم شکست
&&
که ترا آن فضل و آن مقدار هست
\\
لابه‌ات را هیچ نتوانم شکست
&&
زآنک لابهٔ تو یقین لابهٔ منست
\\
گر زمین و آسمان بر هم زدی
&&
ز انتقام این مرد بیرون نامدی
\\
ور شدی ذره به ذره لابه‌گر
&&
او نبردی این زمان از تیغ سر
\\
بر تو می‌ننهیم منت ای کریم
&&
لیک شرح عزت تست ای ندیم
\\
این نکردی تو که من کردم یقین
&&
ایی صفاتت در صفات ما دفین
\\
تو درین مستعملی نی عاملی
&&
زانک محمول منی نی حاملی
\\
ما رمیت اذ رمیت گشته‌ای
&&
خویشتن در موج چون کف هشته‌ای
\\
لا شدی پهلوی الا خانه‌گیر
&&
این عجب که هم اسیری هم امیر
\\
آنچ دادی تو ندای شاه داد
&&
اوست بس الله اعلم بالرشاد
\\
وآن ندیم رسته از زخم و بلا
&&
زین شفیع آزرد و برگشت از ولا
\\
دوستی ببرید زان مخلص تمام
&&
رو به حایط کرد تا نارد سلام
\\
زین شفیع خویشتن بیگانه شد
&&
زین تعجب خلق در افسانه شد
\\
که نه مجنونست یاری چون برید
&&
از کسی که جان او را وا خرید
\\
وا خریدش آن دم از گردن زدن
&&
خاک نعل پاش بایستی شدن
\\
بازگونه رفت و بیزاری گرفت
&&
با چنین دلدار کین‌داری گرفت
\\
پس ملامت کرد او را مصلحی
&&
کیین جفا چون می‌کنی با ناصحی
\\
جان تو بخرید آن دلدار خاص
&&
آن دم از گردن زدن کردت خلاص
\\
گر بدی کردی نبایستی رمید
&&
خاصه نیکی کرد آن یار حمید
\\
گفت بهر شاه مبذولست جان
&&
او چرا آید شفیع اندر میان
\\
لی مع‌الله وقت بود آن دم مرا
&&
لا یسع فیه نبی مجتبی
\\
من نخواهم رحمتی جز زخم شاه
&&
من نخواهم غیر آن شه را پناه
\\
غیر شه را بهر آن لا کرده‌ام
&&
که به سوی شه تولا کرده‌ام
\\
گر ببرد او به قهر خود سرم
&&
شاه بخشد شصت جان دیگرم
\\
کار من سربازی و بی‌خویشی است
&&
کار شاهنشاه من سربخشی است
\\
فخر آن سر که کف شاهش برد
&&
ننگ آن سر کو به غیری سر برد
\\
شب که شاه از قهر در قیرش کشید
&&
ننگ دارد از هزاران روز عید
\\
خود طواف آنک او شه‌بین بود
&&
فوق قهر و لطف و کفر و دین بود
\\
زان نیامد یک عبارت در جهان
&&
که نهانست و نهانست و نهان
\\
زانک این اسما و الفاظ حمید
&&
از گلابهٔ آدمی آمد پدید
\\
علم الاسما بد آدم را امام
&&
لیک نه اندر لباس عین و لام
\\
چون نهاد از آب و گل بر سر کلاه
&&
گشت آن اسمای جانی روسیاه
\\
که نقاب حرف و دم در خود کشید
&&
تا شود بر آب و گل معنی پدید
\\
گرچه از یک وجه منطق کاشف است
&&
لیک از ده وجه پرده و مکنف است
\\
\end{longtable}
\end{center}
