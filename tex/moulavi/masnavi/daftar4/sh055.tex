\begin{center}
\section*{بخش ۵۵ - در بیان آنک ترک الجواب جواب مقرر این سخن کی جواب الاحمق سکوت شرح این هر دو درین قصه است کی  گفته می‌آید}
\label{sec:sh055}
\addcontentsline{toc}{section}{\nameref{sec:sh055}}
\begin{longtable}{l p{0.5cm} r}
بود شاهی بود او را بنده‌ای
&&
مرده عقلی بود و شهوت‌زنده‌ای
\\
خرده‌های خدمتش بگذاشتی
&&
بد سگالیدی نکو پنداشتی
\\
گفت شاهنشه جرااش کم کنید
&&
ور بجنگد نامش از خط بر زنید
\\
عقل او کم بود و حرص او فزون
&&
چون جرا کم دید شد تند و حرون
\\
عقل بودی گرد خود کردی طواف
&&
تا بدیدی جرم خود گشتی معاف
\\
چون خری پابسته تندد از خری
&&
هر دو پایش بسته گردد بر سری
\\
پس بگوید خر که یک بندم بست
&&
خود مدان کان دو ز فعل آن خسست
\\
\end{longtable}
\end{center}
