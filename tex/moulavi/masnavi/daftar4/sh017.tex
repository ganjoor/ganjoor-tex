\begin{center}
\section*{بخش ۱۷ - شرح انما الممنون اخوة والعلماء کنفس واحدة خاصه اتحاد داود و سلیمان و سایر انبیا علیهم‌السلام کی اگر یکی ازیشان را منکر شوی ایمان به هیچ نبی درست نباشد و این علامت اتحادست کی یک خانه از هزاران خانه ویران کنی آن همه ویران شود و یک دیوار قایم نماند کی لانفرق بین احد منهم و العاقل یکفیه الاشارة این خود از اشارت گذشت}
\label{sec:sh017}
\addcontentsline{toc}{section}{\nameref{sec:sh017}}
\begin{longtable}{l p{0.5cm} r}
گرچه بر ناید به جهد و زور تو
&&
لیک مسجد را برآرد پور تو
\\
کردهٔ او کردهٔ تست ای حکیم
&&
مؤمنان را اتصالی دان قدیم
\\
مؤمنان معدود لیک ایمان یکی
&&
جسمشان معدود لیکن جان یکی
\\
غیرفهم و جان که در گاو و خرست
&&
آدمی را عقل و جانی دیگرست
\\
باز غیرجان و عقل آدمی
&&
هست جانی در ولی آن دمی
\\
جان حیوانی ندارد اتحاد
&&
تو مجو این اتحاد از روح باد
\\
گر خورد این نان نگردد سیر آن
&&
ور کشد بار این نگردد او گران
\\
بلک این شادی کند از مرگ او
&&
از حسد میرد چو بیند برگ او
\\
جان گرگان و سگان هر یک جداست
&&
متحد جانهای شیران خداست
\\
جمع گفتم جانهاشان من به اسم
&&
کان یکی جان صد بود نسبت به جسم
\\
هم‌چو آن یک نور خورشید سما
&&
صد بود نسبت بصحن خانه‌ها
\\
لیک یک باشد همه انوارشان
&&
چونک برگیری تو دیوار از میان
\\
چون نماند خانه‌ها را قاعده
&&
مؤمنان مانند نفس واحده
\\
فرق و اشکالات آید زین مقال
&&
زانک نبود مثل این باشد مثال
\\
فرقها بی‌حد بود از شخص شیر
&&
تا به شخص آدمی‌زاد دلیر
\\
لیک در وقت مثال ای خوش‌نظر
&&
اتحاد از روی جانبازی نگر
\\
کان دلیر آخر مثال شیر بود
&&
نیست مثل شیر در جملهٔ حدود
\\
متحد نقشی ندارد این سرا
&&
تا که مثلی وا نمایم من ترا
\\
هم مثال ناقصی دست آورم
&&
تا ز حیرانی خرد را وا خرم
\\
شب بهر خانه چراغی می‌نهند
&&
تا به نور آن ز ظلمت می‌رهند
\\
آن چراغ این تن بود نورش چو جان
&&
هست محتاج فتیل و این و آن
\\
آن چراغ شش فتیلهٔ این حواس
&&
جملگی بر خواب و خور دارد اساس
\\
بی‌خور و بی‌خواب نزید نیم دم
&&
با خور و با خواب نزید نیز هم
\\
بی‌فتیل و روغنش نبود بقا
&&
با فتیل و روغن او هم بی‌وفا
\\
زانک نور علتی‌اش مرگ‌جوست
&&
چون زید که روز روشن مرگ اوست
\\
جمله حسهای بشر هم بی‌بقاست
&&
زانک پیش نور روز حشر لاست
\\
نور حس و جان بابایان ما
&&
نیست کلی فانی و لا چون گیا
\\
لیک مانند ستاره و ماهتاب
&&
جمله محوند از شعاع آفتاب
\\
آنچنان که سوز و درد زخم کیک
&&
محو گردد چون در آید مار الیک
\\
آنچنان که عور اندر آب جست
&&
تا در آب از زخم زنبوران برست
\\
می‌کند زنبور بر بالا طواف
&&
چون بر آرد سر ندارندش معاف
\\
آب ذکر حق و زنبور این زمان
&&
هست یاد آن فلانه وان فلان
\\
دم بخور در آب ذکر و صبر کن
&&
تا رهی از فکر و وسواس کهن
\\
بعد از آن تو طبع آن آب صفا
&&
خود بگیری جملگی سر تا به پا
\\
آنچنان که از آب آن زنبور شر
&&
می‌گریزد از تو هم گیرد حذر
\\
بعد از آن خواهی تو دور از آب باش
&&
که بسر هم‌طبع آبی خواجه‌تاش
\\
بس کسانی کز جهان بگذشته‌اند
&&
لا نیند و در صفات آغشته‌اند
\\
در صفات حق صفات جمله‌شان
&&
هم‌چو اختر پیش آن خور بی‌نشان
\\
گر ز قرآن نقل خواهی ای حرون
&&
خوان جمیع هم لدینا محضرون
\\
محضرون معدوم نبود نیک بین
&&
تا بقای روحها دانی یقین
\\
روح محجوب از بقا بس در عذاب
&&
روح واصل در بقا پاک از حجاب
\\
زین چراغ حس حیوان المراد
&&
گفتمت هان تا نجویی اتحاد
\\
روح خود را متصل کن ای فلان
&&
زود با ارواح قدس سالکان
\\
صد چراغت ار مرند ار بیستند
&&
پس جدا اند و یگانه نیستند
\\
زان همه جنگند این اصحاب ما
&&
جنگ کس نشنید اندر انبیا
\\
زانک نور انبیا خورشید بود
&&
نور حس ما چراغ و شمع و دود
\\
یک بمیرد یک بماند تا به روز
&&
یک بود پژمرده دیگر با فروز
\\
جان حیوانی بود حی از غذا
&&
هم بمیرد او بهر نیک و بذی
\\
گر بمیرد این چراغ و طی شود
&&
خانهٔ همسایه مظلم کی شود
\\
نور آن خانه چو بی این هم به پاست
&&
پس چراغ حس هر خانه جداست
\\
این مثال جان حیوانی بود
&&
نه مثال جان ربانی بود
\\
باز از هندوی شب چون ماه زاد
&&
در سر هر روزنی نوری فتاد
\\
نور آن صد خانه را تو یک شمر
&&
که نماند نور این بی آن دگر
\\
تا بود خورشید تابان بر افق
&&
هست در هر خانه نور او قنق
\\
باز چون خورشید جان آفل شود
&&
نور جمله خانه‌ها زایل شود
\\
این مثال نور آمد مثل نی
&&
مر ترا هادی عدو را ره‌زنی
\\
بر مثال عنکبوت آن زشت‌خو
&&
پرده‌های گنده را بر بافد او
\\
از لعاب خویش پردهٔ نور کرد
&&
دیدهٔ ادراک خود را کور کرد
\\
گردن اسپ ار بگیرد بر خورد
&&
ور بگیرد پاش بستاند لگد
\\
کم نشین بر اسپ توسن بی‌لگام
&&
عقل و دین را پیشوا کن والسلام
\\
اندرین آهنگ منگر سست و پست
&&
کاندرین ره صبر و شق انفسست
\\
\end{longtable}
\end{center}
