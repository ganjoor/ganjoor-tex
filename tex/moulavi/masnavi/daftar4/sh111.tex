\begin{center}
\section*{بخش ۱۱۱ - تفسیر این آیت کی و ما خلقنا السموات والارض و ما بینهما الا بالحق نیافریدمشان بهر همین کی شما می‌بینید  بلک بهر معنی و حکمت باقیه کی شما نمی‌بینید آن را}
\label{sec:sh111}
\addcontentsline{toc}{section}{\nameref{sec:sh111}}
\begin{longtable}{l p{0.5cm} r}
هیچ نقاشی نگارد زین نقش
&&
بی امید نفع بهر عین نقش
\\
بلک بهر میهمانان و کهان
&&
که به فرجه وارهند از اندهان
\\
شادی بچگان و یاد دوستان
&&
دوستان رفته را از نقش آن
\\
هیچ کوزه‌گر کند کوزه شتاب
&&
بهر عین کوزه نه بر بوی آب
\\
هیچ کاسه گر کند کاسه تمام
&&
بهر عین کاسه نه بهر طعام
\\
هیچ خطاطی نویسد خط به فن
&&
بهر عین خط نه بهر خواندن
\\
نقش ظاهر بهر نقش غایبست
&&
وان برای غایب دیگر ببست
\\
تا سوم چارم دهم بر می‌شمر
&&
این فواید را به مقدار نظر
\\
هم‌چو بازیهای شطرنج ای پسر
&&
فایدهٔ هر لعب در تالی نگر
\\
این نهادند بهر آن لعب نهان
&&
وان برای آن و آن بهر فلان
\\
هم‌چنین دیده جهات اندر جهات
&&
در پی هم تا رسی در برد و مات
\\
اول از بهر دوم باشد چنان
&&
که شدن بر پایه‌های نردبان
\\
و آن دوم بهر سوم می‌دان تمام
&&
تا رسی تو پایه پایه تا به بام
\\
شهوت خوردن ز بهر آن منی
&&
آن منی از بهر نسل و روشنی
\\
کندبینش می‌نبیند غیر این
&&
عقل او بی‌سیر چون نبت زمین
\\
نبت را چه خوانده چه ناخوانده
&&
هست پای او به گل در مانده
\\
گر سرش جنبد پیر باد رو
&&
تو به سر جنبانیش غره مشو
\\
آن سرش گوید سمعنا ای صبا
&&
پای او گوید عصینا خلنا
\\
چون ندارد سیر می‌راند چون عام
&&
بر توکل می‌نهد چون کور گام
\\
بر توکل تا چه آید در نبرد
&&
چون توکل کردن اصحاب نرد
\\
وآن نظرهایی که آن افسرده نیست
&&
جز رونده و جز درندهٔ پرده نیست
\\
آنچ در ده سال خواهد آمدن
&&
این زمان بیند به چشم خویشتن
\\
هم‌چنین هر کس به اندازهٔ نظر
&&
غیب و مستقبل ببیند خیر وشر
\\
چونک سد پیش و سد پس نماند
&&
شد گذاره چشم و لوح غیب خواند
\\
چون نظر پس کرد تا بدو وجود
&&
ماجرا و آغاز هستی رو نمود
\\
بحث املاک زمین با کبریا
&&
در خلیفه کردن بابای ما
\\
چون نظر در پیش افکند او بدید
&&
آنچ خواهد بود تا محشر پدید
\\
پس ز پس می‌بیند او تا اصل اصل
&&
پیش می‌بیند عیان تا روز فصل
\\
هر کسی اندازهٔ روشن‌دلی
&&
غیب را بیند به قدر صیقلی
\\
هر که صیقل بیش کرد او بیش دید
&&
بیشتر آمد برو صورت پدید
\\
گر تو گویی کان صفا فضل خداست
&&
نیز این توفیق صیقل زان عطاست
\\
قدر همت باشد آن جهد و دعا
&&
لیس للانسان الا ما سعی
\\
واهب همت خداوندست و بس
&&
همت شاهی ندارد هیچ خس
\\
نیست تخصیص خدا کس را به کار
&&
مانع طوع و مراد و اختیار
\\
لیک چون رنجی دهد بدبخت را
&&
او گریزاند به کفران رخت را
\\
نیکبختی را چو حق رنجی دهد
&&
رخت را نزدیکتر وا می‌نهد
\\
بددلان از بیم جان در کارزار
&&
کرده اسباب هزیمت اختیار
\\
پردلان در جنگ هم از بیم جان
&&
حمله کرده سوی صف دشمنان
\\
رستمان را ترس و غم وا پیش برد
&&
هم ز ترس آن بددل اندر خویش مرد
\\
چون محک آمد بلا و بیم جان
&&
زان پدید آید شجاع از هر جبان
\\
\end{longtable}
\end{center}
