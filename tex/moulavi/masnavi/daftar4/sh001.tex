\begin{center}
\section*{بخش ۱ - سر آغاز}
\label{sec:sh001}
\addcontentsline{toc}{section}{\nameref{sec:sh001}}
\begin{longtable}{l p{0.5cm} r}
ای ضیاء الحق حسام الدین توی
&&
که گذشت از مه به نورت مثنوی
\\
همت عالی تو ای مرتجا
&&
می‌کشد این را خدا داند کجا
\\
گردن این مثنوی را بسته‌ای
&&
می‌کشی آن سوی که دانسته‌ای
\\
مثنوی پویان کشنده ناپدید
&&
ناپدید از جاهلی کش نیست دید
\\
مثنوی را چون تو مبدا بوده‌ای
&&
گر فزون گردد توش افزوده‌ای
\\
چون چنین خواهی خدا خواهد چنین
&&
می‌دهد حق آرزوی متقین
\\
کان لله بوده‌ای در ما مضی
&&
تا که کان الله پیش آمد جزا
\\
مثنوی از تو هزاران شکر داشت
&&
در دعا و شکر کفها بر فراشت
\\
در لب و کفش خدا شکر تو دید
&&
فضل کرد و لطف فرمود و مزید
\\
زانک شاکر را زیادت وعده است
&&
آنچنانک قرب مزد سجده است
\\
گفت واسجد واقترب یزدان ما
&&
قرب جان شد سجده ابدان ما
\\
گر زیادت می‌شود زین رو بود
&&
نه از برای بوش و های و هو بود
\\
با تو ما چون رز به تابستان خوشیم
&&
حکم داری هین بکش تا می‌کشیم
\\
خوش بکش این کاروان را تا به حج
&&
ای امیر صبر مفتاح الفرج
\\
حج زیارت کردن خانه بود
&&
حج رب البیت مردانه بود
\\
زان ضیا گفتم حسام‌الدین ترا
&&
که تو خورشیدی و این دو وصفها
\\
کین حسام و این ضیا یکیست هین
&&
تیغ خورشید از ضیا باشد یقین
\\
نور از آن ماه باشد وین ضیا
&&
آن خورشید این فرو خوان از نبا
\\
شمس را قرآن ضیا خواند ای پدر
&&
و آن قمر را نور خواند این را نگر
\\
شمس چون عالی‌تر آمد خود ز ماه
&&
پس ضیا از نور افزون دان به جاه
\\
بس کس اندر نور مه منهج ندید
&&
چون برآمد آفتاب آن شد پدید
\\
آفتاب اعواض را کامل نمود
&&
لاجرم بازارها در روز بود
\\
تا که قلب و نقد نیک آید پدید
&&
تا بود از غبن و از حیله بعید
\\
تا که نورش کامل آمد در زمین
&&
تاجران را رحمة للعالمین
\\
لیک بر قلاب مبغوضست و سخت
&&
زانک ازو شد کاسد او را نقد و رخت
\\
پس عدو جان صرافست قلب
&&
دشمن درویش کی بود غیر کلب
\\
انبیا با دشمنان بر می‌تنند
&&
پس ملایک رب سلم می‌زنند
\\
کین چراغی را که هست او نور کار
&&
از پف و دمهای دزدان دور دار
\\
دزد و قلابست خصم نور بس
&&
زین دو ای فریادرس فریاد رس
\\
روشنی بر دفتر چارم بریز
&&
کآفتاب از چرخ چارم کرد خیز
\\
هین ز چارم نور ده خورشیدوار
&&
تا بتابد بر بلاد و بر دیار
\\
هر کش افسانه بخواند افسانه است
&&
وآنک دیدش نقد خود مردانه است
\\
آب نیلست و به قبطی خون نمود
&&
قوم موسی را نه خون بد آب بود
\\
دشمن این حرف این دم در نظر
&&
شد ممثل سرنگون اندر سقر
\\
ای ضیاء الحق تو دیدی حال او
&&
حق نمودت پاسخ افعال او
\\
دیدهٔ غیبت چو غیبست اوستاد
&&
کم مبادا زین جهان این دید و داد
\\
این حکایت را که نقد وقت ماست
&&
گر تمامش می‌کنی اینجا رواست
\\
ناکسان را ترک کن بهر کسان
&&
قصه را پایان بر و مخلص رسان
\\
این حکایت گر نشد آنجا تمام
&&
چارمین جلدست آرش در نظام
\\
\end{longtable}
\end{center}
