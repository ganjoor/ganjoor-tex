\begin{center}
\section*{بخش ۱۳۸ - موری بر کاغذ می‌رفت نبشتن قلم دید  قلم را ستودن گرفت موری دیگر کی  چشم تیزتر بود گفت ستایش انگشتان  را کن کی آن هنر ازیشان می‌بینم  موری دگر کی از هر دو چشم روشن‌تر بود گفت من بازو را ستایم کی انگشتان فرع بازواند الی آخره}
\label{sec:sh138}
\addcontentsline{toc}{section}{\nameref{sec:sh138}}
\begin{longtable}{l p{0.5cm} r}
مورکی بر کاغذی دید او قلم
&&
گفت با مور دگر این راز هم
\\
که عجایب نقشها آن کلک کرد
&&
هم‌چو ریحان و چو سوسن‌زار و ورد
\\
گفت آن مور اصبعست آن پیشه‌ور
&&
وین قلم در فعل فرعست و اثر
\\
گفت آن مور سوم کز بازوست
&&
که اصبع لاغر ز زورش نقش بست
\\
هم‌چنین می‌رفت بالا تا یکی
&&
مهتر موران فطن بود اندکی
\\
گفت کز صورت مبینید این هنر
&&
که به خواب و مرگ گردد بی‌خبر
\\
صورت آمد چون لباس و چون عصا
&&
جز به عقل و جان نجنبد نقشها
\\
بی‌خبر بود او که آن عقل و فاد
&&
بی ز تقلیب خدا باشد جماد
\\
یک زمان از وی عنایت بر کند
&&
عقل زیرک ابلهیها می‌کند
\\
چونش گویا یافت ذوالقرنین گفت
&&
چونک کوه قاف در نطق سفت
\\
کای سخن‌گوی خبیر رازدان
&&
از صفات حق بکن با من بیان
\\
گفت رو کان وصف از آن هایل‌ترست
&&
که بیان بر وی تواند برد دست
\\
یا قلم را زهره باشد که به سر
&&
بر نویسد بر صحایف زان خبر
\\
گفت کمتر داستانی باز گو
&&
از عجبهای حق ای حبر نکو
\\
گفت اینک دشت سیصدساله راه
&&
کوههای برف پر کردست شاه
\\
کوه بر که بی‌شمار و بی‌عدد
&&
می‌رسد در هر زمان برفش مدد
\\
کوه برفی می‌زند بر دیگری
&&
می‌رساند برف سردی تا ثری
\\
کوه برفی می‌زند بر کوه برف
&&
دم به دم ز انبار بی‌حد و شگرف
\\
گر نبودی این چنین وادی شها
&&
تف دوزخ محو کردی مر مرا
\\
غافلان را کوههای برف دان
&&
تا نسوزد پرده‌های عاقلان
\\
گر نبودی عکس جهل برف‌باف
&&
سوختی از نار شوق آن کوه قاف
\\
آتش از قهر خدا خود ذره‌ایست
&&
بهر تهدید لئیمان دره‌ایست
\\
با چنین قهری که زفت و فایق است
&&
برد لطفش بین که بر وی سابق است
\\
سبق بی‌چون و چگونهٔ معنوی
&&
سابق و مسبوق دیدی بی‌دوی
\\
گر ندیدی آن بود از فهم پست
&&
که عقول خلق زان کان یک جوست
\\
عیب بر خود نه نه بر آیات دین
&&
کی رسد بر چرخ دین مرغ گلین
\\
مرغ را جولانگه عالی هواست
&&
زانک نشو او ز شهوت وز هواست
\\
پس تو حیران باش بی‌لا و بلی
&&
تا ز رحمت پیشت آید محملی
\\
چون ز فهم این عجایب کودنی
&&
گر بلی گویی تکلف می‌کنی
\\
ور بگویی نی زند نی گردنت
&&
قهر بر بندد بدان نی روزنت
\\
پس همین حیران و واله باش و بس
&&
تا درآید نصر حق از پیش و پس
\\
چونک حیران گشتی و گیج و فنا
&&
با زبان حال گفتی اهدنا
\\
زفت زفتست و چو لرزان می‌شوی
&&
می‌شود آن زفت نرم و مستوی
\\
زانک شکل زفت بهر منکرست
&&
چونک عاجز آمدی لطف و برست
\\
\end{longtable}
\end{center}
