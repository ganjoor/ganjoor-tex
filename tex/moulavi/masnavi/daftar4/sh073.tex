\begin{center}
\section*{بخش ۷۳ - شنیدن شیخ ابوالحسن رضی الله عنه خبر دادن ابویزید را و  بود او و احوال او}
\label{sec:sh073}
\addcontentsline{toc}{section}{\nameref{sec:sh073}}
\begin{longtable}{l p{0.5cm} r}
هم‌چنان آمد که او فرموده بود
&&
بوالحسن از مردمان آن را شنود
\\
که حسن باشد مرید و امتم
&&
درس گیرد هر صباح از تربتم
\\
گفت من هم نیز خوابش دیده‌ام
&&
وز روان شیخ این بشنیده‌ام
\\
هر صباحی رو نهادی سوی گور
&&
ایستادی تا ضحی اندر حضور
\\
یا مثال شیخ پیشش آمدی
&&
یا که بی‌گفتی شکالش حل شدی
\\
تا یکی روزی بیامد با سعود
&&
گورها را برف نو پوشیده بود
\\
توی بر تو برفها هم‌چون علم
&&
قبه قبه دیده و شد جانش به غم
\\
بانگش آمد از حظیرهٔ شیخ حی
&&
ها انا ادعوک کی تسعی الی
\\
هین بیا این سو بر آوازم شتاب
&&
عالم ار برفست روی از من متاب
\\
حال او زان روز شد خوب و بدید
&&
آن عجایب را که اول می‌شنید
\\
\end{longtable}
\end{center}
