\begin{center}
\section*{بخش ۲۷ - دیدن درویش جماعت مشایخ را در  خواب و درخواست کردن روزی حلال بی‌مشغول شدن به کسب و از عبادت ماندن و ارشاد ایشان او را و میوه‌های تلخ و ترش کوهی بر وی شیرین شدن به داد آن مشایخ}
\label{sec:sh027}
\addcontentsline{toc}{section}{\nameref{sec:sh027}}
\begin{longtable}{l p{0.5cm} r}
آن یکی درویش گفت اندر سمر
&&
خضریان را من بدیدم خواب در
\\
گفتم ایشان را که روزی حلال
&&
از کجا نوشم که نبود آن وبال
\\
مر مرا سوی کهستان راندند
&&
میوه‌ها زان بیشه می‌افشاندند
\\
که خدا شیرین بکرد آن میوه را
&&
در دهان تو به همتهای ما
\\
هین بخور پاک و حلال و بی‌حساب
&&
بی صداع و نقل و بالا و نشیب
\\
پس مرا زان رزق نطقی رو نمود
&&
ذوق گفت من خردها می‌ربود
\\
گفتم این فتنه‌ست ای رب جهان
&&
بخششی ده از همه خلقان نهان
\\
شد سخن از من دل خوش یافتم
&&
چون انار از ذوق می‌بشکافتم
\\
گفتم ار چیزی نباشد در بهشت
&&
غیر این شادی که دارم در سرشت
\\
هیچ نعمت آرزو ناید دگر
&&
زین نپردازم به حور و نیشکر
\\
مانده بود از کسب یک دو حبه‌ام
&&
دوخته در آستین جبه‌ام
\\
\end{longtable}
\end{center}
