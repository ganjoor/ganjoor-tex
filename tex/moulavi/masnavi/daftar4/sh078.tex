\begin{center}
\section*{بخش ۷۸ - جواب گفتن مصطفی علیه‌السلام اعتراض کننده را}
\label{sec:sh078}
\addcontentsline{toc}{section}{\nameref{sec:sh078}}
\begin{longtable}{l p{0.5cm} r}
در حضور مصطفای قندخو
&&
چون ز حد برد آن عرب از گفت و گو
\\
آن شه والنجم و سلطان عبس
&&
لب گزید آن سرد دم را گفت بس
\\
دست می‌زند بهر منعش بر دهان
&&
چند گویی پیش دانای نهان
\\
پیش بینا برده‌ای سرگین خشک
&&
که بخر این را به جای ناف مشک
\\
بعر را ای گنده‌مغز گنده‌مخ
&&
زیر بینی بنهی و گویی که اخ
\\
اخ اخی برداشتی ای گیج گاج
&&
تا که کالای بدت یابد رواج
\\
تا فریبی آن مشام پاک را
&&
آن چریدهٔ گلشن افلاک را
\\
حلم او خود را اگر چه گول ساخت
&&
خویشتن را اندکی باید شناخت
\\
دیگ را گر باز ماند امشب دهن
&&
گربه را هم شرم باید داشتن
\\
خویشتن گر خفته کرد آن خوب فر
&&
سخت بیدارست دستارش مبر
\\
چند گویی ای لجوج بی‌صفا
&&
این فسون دیو پیش مصطفی
\\
صد هزاران حلم دارند این گروه
&&
هر یکی حلمی از آنها صد چو کوه
\\
حلمشان بیدار را ابله کند
&&
زیرک صد چشم را گمره کند
\\
حلمشان هم‌چون شراب خوب نغز
&&
نغز نغزک بر رود بالای مغز
\\
مست را بین زان شراب پرشگفت
&&
هم‌چو فرزین مست کژ رفتن گرفت
\\
مرد برنا زان شراب زودگیر
&&
در میان راه می‌افتد چو پیر
\\
خاصه این باده که از خم بلی است
&&
نه میی که مستی او یکشبیست
\\
آنک آن اصحاب کهف از نقل و نقل
&&
سیصد و نه سال گم کردند عقل
\\
زان زنان مصر جامی خورده‌اند
&&
دستها را شرحه شرحه کرده‌اند
\\
ساحران هم سکر موسی داشتند
&&
دار را دلدار می‌انگاشتند
\\
جعفر طیار زان می بود مست
&&
زان گرو می‌کرد بی‌خود پا و دست
\\
\end{longtable}
\end{center}
