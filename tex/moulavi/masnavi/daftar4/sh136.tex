\begin{center}
\section*{بخش ۱۳۶ - بیان آنک خلق دوزخ گرسنگانند و نالانند به حق کی روزیهای ما را فربه گردان و زود زاد به ما رسان کی ما را صبر نماند}
\label{sec:sh136}
\addcontentsline{toc}{section}{\nameref{sec:sh136}}
\begin{longtable}{l p{0.5cm} r}
این سخن پایان ندارد موسیا
&&
هین رها کن آن خران را در گیا
\\
تا همه زان خوش علف فربه شوند
&&
هین که گرگانند ما را خشم‌مند
\\
نالهٔ گرگان خود را موقنیم
&&
این خران را طعمهٔ ایشان کنیم
\\
این خران را کیمیای خوش دمی
&&
از لب تو خواست کردن آدمی
\\
تو بسی کردی به دعوت لطف و جود
&&
آن خران را طالع و روزی نبود
\\
پس فرو پوشان لحاف نعمتی
&&
تا بردشان زود خواب غفلتی
\\
تا چو بجهند از چنین خواب این رده
&&
شمع مرده باشد و ساقی شده
\\
داشت طغیانشان ترا در حیرتی
&&
پس بنوشند از جزا هم حسرتی
\\
تا که عدل ما قدم بیرون نهد
&&
در جزا هر زشت را درخور دهد
\\
که آن شهی که می‌ندیدندیش فاش
&&
بود با ایشان نهان اندر معاش
\\
چون خرد با تست مشرف بر تنت
&&
گر چه زو قاصر بود این دیدنت
\\
نیست قاصر دیدن او ای فلان
&&
از سکون و جنبشت در امتحان
\\
چه عجب گر خالق آن عقل نیز
&&
با تو باشد چون نه‌ای تو مستجیز
\\
از خرد غافل شود بر بد تند
&&
بعد آن عقلش ملامت می‌کند
\\
تو شدی غافل ز عقلت عقل نی
&&
کز حضورستش ملامت کردنی
\\
گر نبودی حاضر و غافل بدی
&&
در ملامت کی ترا سیلی زدی
\\
ور ازو غافل نبودی نفس تو
&&
کی چنان کردی جنون و تفس تو
\\
پس تو و عقلت چو اصطرلاب بود
&&
زین بدانی قرب خورشید وجود
\\
قرب بی‌چونست عقلت را به تو
&&
نیست چپ و راست و پس یا پیش رو
\\
قرب بی‌چون چون نباشد شاه را
&&
که نیابد بحث عقل آن راه را
\\
نیست آن جنبش که در اصبع تراست
&&
پیش اصبع یا پسش یا چپ و راست
\\
وقت خواب و مرگ از وی می‌رود
&&
وقت بیداری قرینش می‌شود
\\
از چه ره می‌آید اندر اصبعت
&&
که اصبعت بی او ندارد منفعت
\\
نور چشم و مردمک در دیده‌ات
&&
از چه ره آمد به غیر شش جهت
\\
عالم خلقست با سوی و جهات
&&
بی‌جهت دان عالم امر و صفات
\\
بی‌جهت دان عالم امر ای صنم
&&
بی‌جهت‌تر باشد آمر لاجرم
\\
بی‌جهت بد عقل و علام البیان
&&
عقل‌تر از عقل و جان‌تر هم ز جان
\\
بی‌تعلق نیست مخلوقی بدو
&&
آن تعلق هست بی‌چون ای عمو
\\
زانک فصل و وصل نبود در روان
&&
غیر فصل و وصل نندیشد گمان
\\
غیر فصل و وصل پی بر از دلیل
&&
لیک پی بردن بننشاند غلیل
\\
پی پیاپی می‌بر ار دوری ز اصل
&&
تا رگ مردیت آرد سوی وصل
\\
این تعلق را خرد چون ره برد
&&
بستهٔ فصلست و وصلست این خرد
\\
زین وصیت کرد ما را مصطفی
&&
بحث کم جویید در ذات خدا
\\
آنک در ذاتش تفکر کردنیست
&&
در حقیقت آن نظر در ذات نیست
\\
هست آن پندار او زیرا به راه
&&
صد هزاران پرده آمد تا اله
\\
هر یکی در پرده‌ای موصول خوست
&&
وهم او آنست که آن خود عین هوست
\\
پس پیمبر دفع کرد این وهم از او
&&
تا نباشد در غلط سوداپز او
\\
وانکه اندر وهم او ترک ادب
&&
بی‌ادب را سرنگونی داد رب
\\
سرنگونی آن بود کو سوی زیر
&&
می‌رود پندارد او کو هست چیر
\\
زانک حد مست باشد این چنین
&&
کو نداند آسمان را از زمین
\\
در عجبهااش به فکر اندر روید
&&
از عظیمی وز مهابت گم شوید
\\
چون ز صنعش ریش و سبلت گم کند
&&
حد خود داند ز صانع تن زند
\\
جز که لا احصی نگوید او ز جان
&&
کز شمار و حد برونست آن بیان
\\
\end{longtable}
\end{center}
