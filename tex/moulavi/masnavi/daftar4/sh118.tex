\begin{center}
\section*{بخش ۱۱۸ - حکایت آن پادشاه‌زاده کی پادشاهی حقیقی  بوی روی نمود یوم یفرالمرء من اخیه و  امه و ابیه نقد وقت او شد پادشاهی  این خاک تودهٔ کودک طبعان کی قلعه گیری نام کنند آن کودک کی چیره آید بر سر  خاک توده برآید و لاف زندگی قلعه مراست کودکان دیگر بر وی  رشک برند کی التراب ربیع الصبیان آن پادشاه‌زاده چو از قید رنگها  برست گفت من این خاکهای رنگین را همان خاک دون می‌گویم زر  و اطلس و اکسون نمی‌گویم من ازین اکسون رستم یکسون رفتم و  آتیناه الحکم صبیا ارشاد حق را مرور سالها حاجت نیست در قدرت کن فیکون هیچ کس سخن قابلیت نگوید}
\label{sec:sh118}
\addcontentsline{toc}{section}{\nameref{sec:sh118}}
\begin{longtable}{l p{0.5cm} r}
پادشاهی داشت یک برنا پسر
&&
باطن و ظاهر مزین از هنر
\\
خواب دید او کان پسر ناگه بمرد
&&
صافی عالم بر آن شه گشت درد
\\
خشک شد از تاب آتش مشک او
&&
که نماند از تف آتش اشک او
\\
آنچنان پر شد ز دود و درد شاه
&&
که نمی‌یابید در وی راه آه
\\
خواست مردن قالبش بی‌کار شد
&&
عمر مانده بود شه بیدار شد
\\
شادیی آمد ز بیداریش پیش
&&
که ندیده بود اندر عمر خویش
\\
که ز شادی خواست هم فانی شدن
&&
بس مطوق آمد این جان و بدن
\\
از دم غم می‌بمیرد این چراغ
&&
وز دم شادی بمیرد اینت لاغ
\\
در میان این دو مرگ او زنده است
&&
این مطوق شکل جای خنده است
\\
شاه با خود گفت شادی را سبب
&&
آنچنان غم بود از تسبیب رب
\\
ای عجب یک چیز از یک روی مرگ
&&
وان ز یک روی دگر احیا و برگ
\\
آن یکی نسبت بدان حالت هلاک
&&
باز هم آن سوی دیگر امتساک
\\
شادی تن سوی دنیاوی کمال
&&
سوی روز عاقبت نقص و زوال
\\
خنده را در خواب هم تعبیر خوان
&&
گریه گوید با دریغ و اندهان
\\
گریه را در خواب شادی و فرح
&&
هست در تعبیر ای صاحب مرح
\\
شاه اندیشید کین غم خود گذشت
&&
لیک جان از جنس این بدظن گشت
\\
ور رسد خاری چنین اندر قدم
&&
که رود گل یادگاری بایدم
\\
چون فنا را شد سبب بی‌منتهی
&&
پس کدامین راه را بندیم ما
\\
صد دریچه و در سوی مرگ لدیغ
&&
می‌کند اندر گشادن ژیغ ژیغ
\\
ژیغ‌ژیغ تلخ آن درهای مرگ
&&
نشنود گوش حریص از حرص برگ
\\
از سوی تن دردها بانگ درست
&&
وز سوی خصمان جفا بانگ درست
\\
جان سر بر خوان دمی فهرست طب
&&
نار علتها نظر کن ملتهب
\\
زان همه غرها درین خانه رهست
&&
هر دو گامی پر ز کزدمها چهست
\\
باد تندست و چراغم ابتری
&&
زو بگیرانم چراغ دیگری
\\
تا بود کز هر دو یک وافی شود
&&
گر به باد آن یک چراغ از جا رود
\\
هم‌چو عارف کن تن ناقص چراغ
&&
شمع دل افروخت از بهر فراغ
\\
تا که روزی کین بمیرد ناگهان
&&
پیش چشم خود نهد او شمع جان
\\
او نکرد این فهم پس داد از غرر
&&
شمع فانی را بفانیی دگر
\\
\end{longtable}
\end{center}
