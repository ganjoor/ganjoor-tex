\begin{center}
\section*{بخش ۶۵ - بقیهٔ نوشتن آن غلام رقعه به طلب اجری}
\label{sec:sh065}
\addcontentsline{toc}{section}{\nameref{sec:sh065}}
\begin{longtable}{l p{0.5cm} r}
رفت پیش از نامه پیش مطبخی
&&
کای بخیل از مطبخ شاه سخی
\\
دور ازو وز همت او کین قدر
&&
از جری‌ام آیدش اندر نظر
\\
گفت بهر مصلحت فرموده است
&&
نه برای بخل و نه تنگی دست
\\
گفت دهلیزیست والله این سخن
&&
پیش شه خاکست هم زر کهن
\\
مطبخی ده گونه حجت بر فراشت
&&
او همه رد کرد از حرصی که داشت
\\
چون جری کم آمدش در وقت چاشت
&&
زد بسی تشنیع او سودی نداشت
\\
گفت قاصد می‌کنید اینها شما
&&
گفت نه که بنده فرمانیم ما
\\
این مگیر از فرع این از اصل گیر
&&
بر کمان کم زن که از بازوست تیر
\\
ما رمیت اذ رمیت ابتلاست
&&
بر نبی کم نه گنه کان از خداست
\\
آب از سر تیره است ای خیره‌خشم
&&
پیشتر بنگر یکی بگشای چشم
\\
شد ز خشم و غم درون بقعه‌ای
&&
سوی شه بنوشت خشمین رقعه‌ای
\\
اندر آن رقعه ثنای شاه گفت
&&
گوهر جود و سخای شاه سفت
\\
کای ز بحر و ابر افزون کف تو
&&
در قضای حاجت حاجات‌جو
\\
زانک ابر آنچ دهد گریان دهد
&&
کف تو خندان پیاپی خوان نهد
\\
ظاهر رقعه اگر چه مدح بود
&&
بوی خشم از مدح اثرها می‌نمود
\\
زان همه کار تو بی‌نورست و زشت
&&
که تو دوری دور از نور سرشت
\\
رونق کار خسان کاسد شود
&&
هم‌چو میوهٔ تازه زو فاسد شود
\\
رونق دنیا برآرد زو کساد
&&
زانک هست از عالم کون و فساد
\\
خوش نگردد از مدیحی سینه‌ها
&&
چونک در مداح باشد کینه‌ها
\\
ای دل از کین و کراهت پاک شو
&&
وانگهان الحمد خوان چالاک شو
\\
بر زبان الحمد و اکراه درون
&&
از زبان تلبیس باشد یا فسون
\\
وانگهان گفته خدا که ننگرم
&&
من به ظاهر من به باطن ناظرم
\\
\end{longtable}
\end{center}
