\begin{center}
\section*{بخش ۱۰۱ - قوله علیه السلام من بشرنی بخروج صفر بشرته بالجنة}
\label{sec:sh101}
\addcontentsline{toc}{section}{\nameref{sec:sh101}}
\begin{longtable}{l p{0.5cm} r}
احمد آخر زمان را انتقال
&&
در ربیع اول آید بی جدال
\\
چون خبر یابد دلش زین وقت نقل
&&
عاشق آن وقت گردد او به عقل
\\
چون صفر آید شود شاد از صفر
&&
که پس این ماه می‌سازم سفر
\\
هر شبی تا روز زین شوق هدی
&&
ای رفیق راه اعلی می‌زدی
\\
گفت هر کس که مرا مژده دهد
&&
چون صفر پای از جهان بیرون نهد
\\
که صفر بگذشت و شد ماه ربیع
&&
مژده‌ور باشم مر او را و شفیع
\\
گفت عکاشه صفر بگذشت و رفت
&&
گفت که جنت ترا ای شیر زفت
\\
دیگری آمد که بگذشت آن صفر
&&
گفت عکاشه ببرد از مژده بر
\\
پس رجال از نقل عالم شادمان
&&
وز بقااش شادمان این کودکان
\\
چونک آب خوش ندید آن مرغ کور
&&
پیش او کوثر نیامد آب شور
\\
هم‌چنین موسی کرامت می‌شمرد
&&
که نگردد صاف اقبال تو درد
\\
گفت احسنت و نکو گفت ولیک
&&
تا کنم من مشورت با یار نیک
\\
\end{longtable}
\end{center}
