\begin{center}
\section*{بخش ۱۶ - قصهٔ مسجد اقصی و خروب و عزم کردن داود علیه‌السلام پیش از سلیمان علیه‌السلام بر بنای آن مسجد}
\label{sec:sh016}
\addcontentsline{toc}{section}{\nameref{sec:sh016}}
\begin{longtable}{l p{0.5cm} r}
چون درآمد عزم داودی به تنگ
&&
که بسازد مسجد اقصی به سنگ
\\
وحی کردش حق که ترک این بخوان
&&
که ز دستت برنیاید این مکان
\\
نیست در تقدیر ما آنک تو این
&&
مسجد اقصی بر آری ای گزین
\\
گفت جرمم چیست ای دانای راز
&&
که مرا گویی که مسجد را مساز
\\
گفت بی‌جرمی تو خونها کرده‌ای
&&
خون مظلومان بگردن برده‌ای
\\
که ز آواز تو خلقی بی‌شمار
&&
جان بدادند و شدند آن را شکار
\\
خون بسی رفتست بر آواز تو
&&
بر صدای خوب جان‌پرداز تو
\\
گفت مغلوب تو بودم مست تو
&&
دست من بر بسته بود از دست تو
\\
نه که هر مغلوب شه مرحوم بود
&&
نه که المغلوب کالمعدوم بود
\\
گفت این مغلوب معدومیست کو
&&
جز به نسبت نیست معدوم ایقنوا
\\
این چنین معدوم کو از خویش رفت
&&
بهترین هستها افتاد و زفت
\\
او به نسبت با صفات حق فناست
&&
در حقیقت در فنا او را بقاست
\\
جملهٔ ارواح در تدبیر اوست
&&
جملهٔ اشباح هم در تیر اوست
\\
آنک او مغلوب اندر لطف ماست
&&
نیست مضطر بلک مختار ولاست
\\
منتهای اختیار آنست خود
&&
که اختیارش گردد اینجا مفتقد
\\
اختیاری را نبودی چاشنی
&&
گر نگشتی آخر او محو از منی
\\
در جهان گر لقمه و گر شربتست
&&
لذت او فرع محو لذتست
\\
گرچه از لذات بی‌تاثیر شد
&&
لذتی بود او و لذت‌گیر شد
\\
\end{longtable}
\end{center}
