\begin{center}
\section*{بخش ۳۹ - حکایت آن پیر عرب کی دلالت کرد حلیمه را به استعانت به بتان}
\label{sec:sh039}
\addcontentsline{toc}{section}{\nameref{sec:sh039}}
\begin{longtable}{l p{0.5cm} r}
پیرمردی پیشش آمد با عصا
&&
کای حلیمه چه فتاد آخر ترا
\\
که چنین آتش ز دل افروختی
&&
این جگرها را ز ماتم سوختی
\\
گفت احمد را رضیعم معتمد
&&
پس بیاوردم که بسپارم به جد
\\
چون رسیدم در حطیم آوازها
&&
می‌رسید و می‌شنیدم از هوا
\\
من چو آن الحان شنیدم از هوا
&&
طفل را بنهادم آنجا زان صدا
\\
تا ببینم این ندا آواز کیست
&&
که ندایی بس لطیف و بس شهیست
\\
نه از کسی دیدم بگرد خود نشان
&&
نه ندا می منقطع شد یک زمان
\\
چونک واگشتم ز حیرتهای دل
&&
طفل را آنجا ندیدم وای دل
\\
گفتش ای فرزند تو انده مدار
&&
که نمایم مر ترا یک شهریار
\\
که بگوید گر بخواهد حال طفل
&&
او بداند منزل و ترحال طفل
\\
پس حلیمه گفت ای جانم فدا
&&
مر ترا ای شیخ خوب خوش‌ندا
\\
هین مرا بنمای آن شاه نظر
&&
کش بود از حال طفل من خبر
\\
برد او را پیش عزی کین صنم
&&
هست در اخبار غیبی مغتنم
\\
ما هزاران گم شده زو یافتیم
&&
چون به خدمت سوی او بشتافتیم
\\
پیر کرد او را سجود و گفت زود
&&
ای خداوند عرب ای بحر جود
\\
گفت ای عزی تو بس اکرامها
&&
کرده‌ای تا رسته‌ایم از دامها
\\
بر عرب حقست از اکرام تو
&&
فرض گشته تا عرب شد رام تو
\\
این حلیمهٔ سعدی از اومید تو
&&
آمد اندر ظل شاخ بید تو
\\
که ازو فرزند طفلی گم شدست
&&
نام آن کودک محمد آمدست
\\
چون محمد گفت آن جمله بتان
&&
سرنگون گشت و ساجد آن زمان
\\
که برو ای پیر این چه جست و جوست
&&
آن محمد را که عزل ما ازوست
\\
ما نگون و سنگسار آییم ازو
&&
ما کساد و بی‌عیار آییم ازو
\\
آن خیالاتی که دیدندی ز ما
&&
وقت فترت گاه گاه اهل هوا
\\
گم شود چون بارگاه او رسید
&&
آب آمد مر تیمم را درید
\\
دور شو ای پیر فتنه کم فروز
&&
هین ز رشک احمدی ما را مسوز
\\
دور شو بهر خدا ای پیر تو
&&
تا نسوزی ز آتش تقدیر تو
\\
این چه دم اژدها افشردنست
&&
هیچ دانی چه خبر آوردنست
\\
زین خبر جوشد دل دریا و کان
&&
زین خبر لرزان شود هفت آسمان
\\
چون شنید از سنگها پیر این سخن
&&
پس عصا انداخت آن پیر کهن
\\
پس ز لرزه و خوف و بیم آن ندا
&&
پیر دندانها به هم بر می‌زدی
\\
آنچنان که اندر زمستان مرد عور
&&
او همی لرزید و می‌گفت ای ثبور
\\
چون در آن حالت بدید او پیر را
&&
زان عجب گم کرد زن تدبیر را
\\
گفت پیر اگر چه من در محنتم
&&
حیرت اندر حیرت اندر حیرتم
\\
ساعتی بادم خطیبی می‌کند
&&
ساعتی سنگم ادیبی می‌کند
\\
باد با حرفم سخنها می‌دهد
&&
سنگ و کوهم فهم اشیا می‌دهد
\\
گاه طفلم را ربوده غیبیان
&&
غیبیان سبز پر آسمان
\\
از کی نالم با کی گویم این گله
&&
من شدم سودایی اکنون صد دله
\\
غیرتش از شرح غیبم لب ببست
&&
این قدر گویم که طفلم گم شدست
\\
گر بگویم چیز دیگر من کنون
&&
خلق بندندم به زنجیر جنون
\\
گفت پیرش کای حلیمه شاد باش
&&
سجدهٔ شکر آر و رو را کم خراش
\\
غم مخور یاوه نگردد او ز تو
&&
بلک عالم یاوه گردد اندرو
\\
هر زمان از رشک غیرت پیش و پس
&&
صد هزاران پاسبانست و حرس
\\
آن ندیدی کان بتان ذو فنون
&&
چون شدند از نام طفلت سرنگون
\\
این عجب قرنیست بر روی زمین
&&
پیر گشتم من ندیدم جنس این
\\
زین رسالت سنگها چون ناله داشت
&&
تا چه خواهد بر گنه کاران گماشت
\\
سنگ بی‌جرمست در معبودیش
&&
تو نه‌ای مضطر که بنده بودیش
\\
او که مضطر این چنین ترسان شدست
&&
تا که بر مجرم چه‌ها خواهند بست
\\
\end{longtable}
\end{center}
