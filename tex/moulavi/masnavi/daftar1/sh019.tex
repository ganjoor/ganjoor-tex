\begin{center}
\section*{بخش ۱۹ - فهم کردن حاذقان نصاری مکر وزیر را}
\label{sec:sh019}
\addcontentsline{toc}{section}{\nameref{sec:sh019}}
\begin{longtable}{l p{0.5cm} r}
هر که صاحب ذوق بود از گفت او
&&
لذتی می‌دید و تلخی جفت او
\\
نکته‌ها می‌گفت او آمیخته
&&
در جلاب قند زهری ریخته
\\
ظاهرش می‌گفت در ره چست شو
&&
وز اثر می‌گفت جان را سست شو
\\
ظاهر نقره گر اسپیدست و نو
&&
دست و جامه می سیه گردد ازو
\\
آتش ار چه سرخ رویست از شرر
&&
تو ز فعل او سیه کاری نگر
\\
برق اگر نوری نماید در نظر
&&
لیک هست از خاصیت دزد بصر
\\
هر که جز آگاه و صاحب ذوق بود
&&
گفت او در گردن او طوق بود
\\
مدتی شش سال در هجران شاه
&&
شد وزیر اتباع عیسی را پناه
\\
دین و دل را کل بدو بسپرد خلق
&&
پیش امر و حکم او می‌مرد خلق
\\
\end{longtable}
\end{center}
