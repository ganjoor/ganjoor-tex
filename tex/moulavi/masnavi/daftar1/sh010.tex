\begin{center}
\section*{بخش ۱۰ - بیان آنک کشتن و زهر دادن مرد زرگر به اشارت الهی بود نه به هوای نفس و تامل فاسد}
\label{sec:sh010}
\addcontentsline{toc}{section}{\nameref{sec:sh010}}
\begin{longtable}{l p{0.5cm} r}
کشتن آن مرد بر دست حکیم
&&
نه پی اومید بود و نه ز بیم
\\
او نکشتش از برای طبع شاه
&&
تا نیامد امر و الهام اله
\\
آن پسر را کش خضر ببرید حلق
&&
سر آن را در نیابد عام خلق
\\
آنک از حق یابد او وحی و جواب
&&
هرچه فرماید بود عین صواب
\\
آنک جان بخشد اگر بکشد رواست
&&
نایبست و دست او دست خداست
\\
همچو اسمعیل پیشش سر بنه
&&
شاد و خندان پیش تیغش جان بده
\\
تا بماند جانت خندان تا ابد
&&
همچو جان پاک احمد با احد
\\
عاشقان آنگه شراب جان کشند
&&
که به دست خویش خوبانشان کشند
\\
شاه آن خون از پی شهوت نکرد
&&
تو رها کن بدگمانی و نبرد
\\
تو گمان بردی که کرد آلودگی
&&
در صفا غش کی هلد پالودگی
\\
بهر آنست این ریاضت وین جفا
&&
تا بر آرد کوره از نقره جفا
\\
بهر آنست امتحان نیک و بد
&&
تا بجوشد بر سر آرد زر زبد
\\
گر نبودی کارش الهام اله
&&
او سگی بودی دراننده نه شاه
\\
پاک بود از شهوت و حرص و هوا
&&
نیک کرد او لیک نیک بد نما
\\
گر خضر در بحر کشتی را شکست
&&
صد درستی در شکست خضر هست
\\
وهم موسی با همه نور و هنر
&&
شد از آن محجوب تو بی پر مپر
\\
آن گل سرخست تو خونش مخوان
&&
مست عقلست او تو مجنونش مخوان
\\
گر بدی خون مسلمان کام او
&&
کافرم گر بردمی من نام او
\\
می‌بلرزد عرش از مدح شقی
&&
بدگمان گردد ز مدحش متقی
\\
شاه بود و شاه بس آگاه بود
&&
خاص بود و خاصهٔ الله بود
\\
آن کسی را کش چنین شاهی کشد
&&
سوی بخت و بهترین جاهی کشد
\\
گر ندیدی سود او در قهر او
&&
کی شدی آن لطف مطلق قهرجو
\\
بچه می‌لرزد از آن نیش حجام
&&
مادر مشفق در آن دم شادکام
\\
نیم جان بستاند و صد جان دهد
&&
آنچ در وهمت نیاید آن دهد
\\
تو قیاس از خویش می‌گیری ولیک
&&
دور دور افتاده‌ای بنگر تو نیک
\\
\end{longtable}
\end{center}
