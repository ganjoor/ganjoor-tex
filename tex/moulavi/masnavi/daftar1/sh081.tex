\begin{center}
\section*{بخش ۸۱ - تفسیر و هو معکم اینما کنتم}
\label{sec:sh081}
\addcontentsline{toc}{section}{\nameref{sec:sh081}}
\begin{longtable}{l p{0.5cm} r}
بار دیگر ما به قصه آمدیم
&&
ما از آن قصه برون خود کی شدیم
\\
گر به جهل آییم آن زندان اوست
&&
ور به علم آییم آن ایوان اوست
\\
ور به خواب آییم مستان وییم
&&
ور به بیداری به دستان وییم
\\
ور بگرییم ابر پر زرق وییم
&&
ور بخندیم آن زمان برق وییم
\\
ور بخشم و جنگ عکس قهر اوست
&&
ور بصلح و عذر عکس مهر اوست
\\
ما کییم اندر جهان پیچ پیچ
&&
چون الف او خود چه دارد هیچ‌هیچ
\\
\end{longtable}
\end{center}
