\begin{center}
\section*{بخش ۸۹ - باز گفتن بازرگان با طوطی آنچ دید از طوطیان هندوستان}
\label{sec:sh089}
\addcontentsline{toc}{section}{\nameref{sec:sh089}}
\begin{longtable}{l p{0.5cm} r}
کرد بازرگان تجارت را تمام
&&
باز آمد سوی منزل دوستکام
\\
هر غلامی را بیاورد ارمغان
&&
هر کنیزک را ببخشید او نشان
\\
گفت طوطی ارمغان بنده کو
&&
آنچ دیدی و آنچ گفتی بازگو
\\
گفت نه من خود پشیمانم از آن
&&
دست خود خایان و انگشتان گزان
\\
من چرا پیغام خامی از گزاف
&&
بردم از بی‌دانشی و از نشاف
\\
گفت ای خواجه پشیمانی ز چیست
&&
چیست آن کین خشم و غم را مقتضیست
\\
گفت گفتم آن شکایتهای تو
&&
با گروهی طوطیان همتای تو
\\
آن یکی طوطی ز دردت بوی برد
&&
زهره‌اش بدرید و لرزید و بمرد
\\
من پشیمان گشتم این گفتن چه بود
&&
لیک چون گفتم پشیمانی چه سود
\\
نکته‌ای کان جست ناگه از زبان
&&
همچو تیری دان که آن جست از کمان
\\
وا نگردد از ره آن تیر ای پسر
&&
بند باید کرد سیلی را ز سر
\\
چون گذشت از سر جهانی را گرفت
&&
گر جهان ویران کند نبود شگفت
\\
فعل را در غیب اثرها زادنیست
&&
و آن موالیدش بحکم خلق نیست
\\
بی‌شریکی جمله مخلوق خداست
&&
آن موالید ار چه نسبتشان به ماست
\\
زید پرانید تیری سوی عمرو
&&
عمرو را بگرفت تیرش همچو نمر
\\
مدت سالی همی زایید درد
&&
دردها را آفریند حق نه مرد
\\
زید رامی آن دم ار مرد از وجل
&&
دردها می‌زاید آنجا تا اجل
\\
زان موالید وجع چون مرد او
&&
زید را ز اول سبب قتال گو
\\
آن وجعها را بدو منسوب دار
&&
گرچه هست آن جمله صنع کردگار
\\
همچنین کشت و دم و دام و جماع
&&
آن موالیدست حق را مستطاع
\\
اولیا را هست قدرت از اله
&&
تیر جسته باز آرندش ز راه
\\
بسته درهای موالید از سبب
&&
چون پشیمان شد ولی زان دست رب
\\
گفته ناگفته کند از فتح باب
&&
تا از آن نه سیخ سوزد نه کباب
\\
از همه دلها که آن نکته شنید
&&
آن سخن را کرد محو و ناپدید
\\
گرت برهان باید و حجت مها
&&
بازخوان من آیة او ننسها
\\
آیت انسوکم ذکری بخوان
&&
قدرت نسیان نهادنشان بدان
\\
چون به تذکیر و به نسیان قادرند
&&
بر همه دلهای خلقان قاهرند
\\
چون بنسیان بست او راه نظر
&&
کار نتوان کرد ور باشد هنر
\\
خلتم سخریة اهل السمو
&&
از نبی خوانید تا انسوکم
\\
صاحب ده پادشاه جسمهاست
&&
صاحب دل شاه دلهای شماست
\\
فرع دید آمد عمل بی‌هیچ شک
&&
پس نباشد مردم الا مردمک
\\
من تمام این نیارم گفت از آن
&&
منع می‌آید ز صاحب مرکزان
\\
چون فراموشی خلق و یادشان
&&
با ویست و او رسد فریادشان
\\
صد هزاران نیک و بد را آن بهی
&&
می‌کند هر شب ز دلهاشان تهی
\\
روز دلها را از آن پر می‌کند
&&
آن صدفها را پر از در می‌کند
\\
آن همه اندیشهٔ پیشانها
&&
می‌شناسند از هدایت خانها
\\
پیشه و فرهنگ تو آید به تو
&&
تا در اسباب بگشاید به تو
\\
پیشهٔ زرگر بهنگر نشد
&&
خوی این خوش‌خو با آن منکر نشد
\\
پیشه‌ها و خلقها همچون جهاز
&&
سوی خصم آیند روز رستخیز
\\
پیشه‌ها و خلقها از بعد خواب
&&
واپس آید هم به خصم خود شتاب
\\
پیشه‌ها و اندیشه‌ها در وقت صبح
&&
هم بدانجا شد که بود آن حسن و قبح
\\
چون کبوترهای پیک از شهرها
&&
سوی شهر خویش آرد بهرها
\\
\end{longtable}
\end{center}
