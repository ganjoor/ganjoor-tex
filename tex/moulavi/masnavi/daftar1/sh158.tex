\begin{center}
\section*{بخش ۱۵۸ - پرسیدن پیغمبر صلی الله علیه و سلم مر زید  را که امروز چونی و چون برخاستی و جواب  گفتن او که اصبحت ممنا یا رسول الله}
\label{sec:sh158}
\addcontentsline{toc}{section}{\nameref{sec:sh158}}
\begin{longtable}{l p{0.5cm} r}
گفت پیغامبر صباحی زید را
&&
کیف اصبحت ای رفیق با صفا
\\
گفت عبدا مؤمنا باز اوش گفت
&&
کو نشان از باغ ایمان گر شکفت
\\
گفت تشنه بوده‌ام من روزها
&&
شب نخفتستم ز عشق و سوزها
\\
تا ز روز و شب گذر کردم چنان
&&
که ز اسپر بگذرد نوک سنان
\\
که از آن سو جملهٔ ملت یکیست
&&
صد هزاران سال و یک ساعت یکیست
\\
هست ازل را و ابد را اتحاد
&&
عقل را ره نیست آن سو ز افتقاد
\\
گفت ازین ره کو ره‌آوردی بیار
&&
در خور فهم و عقول این دیار
\\
گفت خلقان چون ببینند آسمان
&&
من ببینم عرش را با عرشیان
\\
هشت جنت هفت دوزخ پیش من
&&
هست پیدا همچو بت پیش شمن
\\
یک بیک وا می‌شناسم خلق را
&&
همچو گندم من ز جو در آسیا
\\
که بهشتی کیست و بیگانه کیست
&&
پیش من پیدا چو مار و ماهیست
\\
این زمان پیدا شده بر این گروه
&&
یوم تبیض و تسود وجوه
\\
پیش ازین هرچند جان پر عیب بود
&&
در رحم بود و ز خلقان غیب بود
\\
الشقی من شقی فی بطن الام
&&
من سمات الجسم یعرف حالهم
\\
تن چو مادر طفل جان را حامله
&&
مرگ درد زادنست و زلزله
\\
جمله جانهای گذشته منتظر
&&
تا چگونه زاید آن جان بطر
\\
زنگیان گویند خود از ماست او
&&
رومیان گویند بس زیباست او
\\
چون بزاید در جهان جان و جود
&&
پس نماند اختلاف بیض و سود
\\
گر بود زنگی برندش زنگیان
&&
روم را رومی برد هم از میان
\\
تا نزاد او مشکلات عالمست
&&
آنک نازاده شناسد او کمست
\\
او مگر ینظر بنور الله بود
&&
کاندرون پوست او را ره بود
\\
اصل آب نطفه اسپیدست و خوش
&&
لیک عکس جان رومی و حبش
\\
می‌دهد رنگ احسن التقویم را
&&
تا به اسفل می‌برد این نیم را
\\
این سخن پایان ندارد باز ران
&&
تا نمانیم از قطار کاروان
\\
یوم تبیض و تسود وجوه
&&
ترک و هندو شهره گردد زان گروه
\\
در رحم پیدا نباشد هند و ترک
&&
چونک زاید بیندش زار و سترگ
\\
جمله را چون روز رستاخیز من
&&
فاش می‌بینم عیان از مرد و زن
\\
هین بگویم یا فرو بندم نفس
&&
لب گزیدش مصطفی یعنی که بس
\\
یا رسول الله بگویم سر حشر
&&
در جهان پیدا کنم امروز نشر
\\
هل مرا تا پرده‌ها را بر درم
&&
تا چو خورشیدی بتابد گوهرم
\\
تا کسوف آید ز من خورشید را
&&
تا نمایم نخل را و بید را
\\
وا نمایم راز رستاخیز را
&&
نقد را و نقد قلب‌آمیز را
\\
دستها ببریده اصحاب شمال
&&
وا نمایم رنگ کفر و رنگ آل
\\
وا گشایم هفت سوراخ نفاق
&&
در ضیای ماه بی خسف و محاق
\\
وا نمایم من پلاس اشقیا
&&
بشنوانم طبل و کوس انبیا
\\
دوزخ و جنات و برزخ در میان
&&
پیش چشم کافران آرم عیان
\\
وا نمایم حوض کوثر را به جوش
&&
کاب بر روشان زند بانگش به گوش
\\
وان کسان که تشنه بر گردش دوان
&&
گشته‌اند این دم نمایم من عیان
\\
می‌بساید دوششان بر دوش من
&&
نعره‌هاشان می‌رسد در گوش من
\\
اهل جنت پیش چشمم ز اختیار
&&
در کشیده یک‌دگر را در کنار
\\
دست همدیگر زیارت می‌کنند
&&
از لبان هم بوسه غارت می‌کنند
\\
کر شد این گوشم ز بانگ آه آه
&&
از خسان و نعرهٔ واحسرتاه
\\
این اشارتهاست گویم از نغول
&&
لیک می‌ترسم ز آزار رسول
\\
همچنین می‌گفت سرمست و خراب
&&
داد پیغامبر گریبانش بتاب
\\
گفت هین در کش که اسبت گرم شد
&&
عکس حق لا یستحی زد شرم شد
\\
آینهٔ تو جست بیرون از غلاف
&&
آینه و میزان کجا گوید خلاف
\\
آینه و میزان کجا بندد نفس
&&
بهر آزار و حیاء هیچ‌کس
\\
آینه و میزان محکهای سنی
&&
گر دو صد سالش تو خدمتها کنی
\\
کز برای من بپوشان راستی
&&
بر فزون بنما و منما کاستی
\\
اوت گوید ریش و سبلت بر مخند
&&
آینه و میزان و آنگه ریو و پند
\\
چون خدا ما را برای آن فراخت
&&
که بما بتوان حقیقت را شناخت
\\
این نباشد ما چه آرزیم ای جوان
&&
کی شویم آیین روی نیکوان
\\
لیک در کش در نمد آیینه را
&&
کز تجلی کرد سینا سینه را
\\
گفت آخر هیچ گنجد در بغل
&&
آفتاب حق و خورشید ازل
\\
هم دغل را هم بغل را بر درد
&&
نه جنون ماند به پیشش نه خرد
\\
گفت یک اصبع چو بر چشمی نهی
&&
بیند از خورشید عالم را تهی
\\
یک سر انگشت پردهٔ ماه شد
&&
وین نشان ساتری شاه شد
\\
تا بپوشاند جهان را نقطه‌ای
&&
مهر گردد منکسف از سقطه‌ای
\\
لب ببند و غور دریایی نگر
&&
بحر را حق کرد محکوم بشر
\\
همچو چشمهٔ سلسبیل و زنجبیل
&&
هست در حکم بهشتی جلیل
\\
چار جوی جنت اندر حکم ماست
&&
این نه زور ما ز فرمان خداست
\\
هر کجا خواهیم داریمش روان
&&
همچو سحر اندر مراد ساحران
\\
همچو این دو چشمهٔ چشم روان
&&
هست در حکم دل و فرمان جان
\\
گر بخواهد رفت سوی زهر و مار
&&
ور بخواهد رفت سوی اعتبار
\\
گر بخواهد سوی محسوسات رفت
&&
ور بخواهد سوی ملبوسات رفت
\\
گر بخواهد سوی کلیات راند
&&
ور بخواهد حبس جزویات ماند
\\
همچنین هر پنج حس چون نایزه
&&
بر مراد و امر دل شد جایزه
\\
هر طرف که دل اشارت کردشان
&&
می‌رود هر پنج حس دامن‌کشان
\\
دست و پا در امر دل اندر ملا
&&
همچو اندر دست موسی آن عصا
\\
دل بخواهد پا در آید زو به رقص
&&
یا گریزد سوی افزونی ز نقص
\\
دل بخواهد دست آید در حساب
&&
با اصابع تا نویسد او کتاب
\\
دست در دست نهانی مانده است
&&
او درون تن را برون بنشانده است
\\
گر بخواهد بر عدو ماری شود
&&
ور بخواهد بر ولی یاری شود
\\
ور بخواهد کفچه‌ای در خوردنی
&&
ور بخواهد همچو گرز ده‌منی
\\
دل چه می‌گوید بدیشان ای عجب
&&
طرفه وصلت طرفه پنهانی سبب
\\
دل مگر مهر سلیمان یافتست
&&
که مهار پنج حس بر تافتست
\\
پنج حسی از برون میسور او
&&
پنج حسی از درون مامور او
\\
ده حس است و هفت اندام و دگر
&&
آنچ اندر گفت ناید می‌شمر
\\
چون سلیمانی دلا در مهتری
&&
بر پری و دیو زن انگشتری
\\
گر درین ملکت بری باشی ز ریو
&&
خاتم از دست تو نستاند سه دیو
\\
بعد از آن عالم بگیرد اسم تو
&&
دو جهان محکوم تو چون جسم تو
\\
ور ز دستت دیو خاتم را ببرد
&&
پادشاهی فوت شد بختت بمرد
\\
بعد از آن یا حسرتا شد یا عباد
&&
بر شما محتوم تا یوم التناد
\\
مکر خود را گر تو انکار آوری
&&
از ترازو و آینه کی جان بری
\\
\end{longtable}
\end{center}
