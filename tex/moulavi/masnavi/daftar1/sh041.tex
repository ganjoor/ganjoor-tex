\begin{center}
\section*{بخش ۴۱ - طنز و انکار کردن پادشاه جهود و قبول ناکردن نصیحت خاصان خویش}
\label{sec:sh041}
\addcontentsline{toc}{section}{\nameref{sec:sh041}}
\begin{longtable}{l p{0.5cm} r}
این عجایب دید آن شاه جهود
&&
جز که طنز و جز که انکارش نبود
\\
ناصحان گفتند از حد مگذران
&&
مرکب استیزه را چندین مران
\\
ناصحان را دست بست و بند کرد
&&
ظلم را پیوند در پیوند کرد
\\
بانگ آمد کار چون اینجا رسید
&&
پای دار ای سگ که قهر ما رسید
\\
بعد از آن آتش چهل گز بر فروخت
&&
حلقه گشت و آن جهودان را بسوخت
\\
اصل ایشان بود آتش ز ابتدا
&&
سوی اصل خویش رفتند انتها
\\
هم ز آتش زاده بودند آن فریق
&&
جزوها را سوی کل باشد طریق
\\
آتشی بودندمؤمن‌سوز و بس
&&
سوخت خود را آتش ایشان چو خس
\\
آنک بودست امه الهاویه
&&
هاویه آمد مرورا زاویه
\\
مادر فرزند جویان ویست
&&
اصلها مر فرعها را در پیست
\\
آبها در حوض اگر زندانیست
&&
باد نشفش می‌کند کار کانیست
\\
می‌رهاند می‌برد تا معدنش
&&
اندک اندک تا نبینی بردنش
\\
وین نفس جانهای ما را همچنان
&&
اندک اندک دزدد از حبس جهان
\\
تا الیه یصعد اطیاب الکلم
&&
صاعدا منا الی حیث علم
\\
ترتقی انفاسنا بالمنتقی
&&
متحفا منا الی دار البقا
\\
ثم تاتینا مکافات المقال
&&
ضعف ذاک رحمة من ذی الجلال
\\
ثم یلجینا الی امثالها
&&
کی ینال العبد مما نالها
\\
هکذی تعرج و تنزل دائما
&&
ذا فلا زلت علیه قائما
\\
پارسی گوییم یعنی این کشش
&&
زان طرف آید که آمد آن چشش
\\
چشم هر قومی به سویی مانده‌ست
&&
کان طرف یک روز ذوقی رانده‌ست
\\
ذوق جنس از جنس خود باشد یقین
&&
ذوق جزو از کل خود باشد ببین
\\
یا مگر آن قابل جنسی بود
&&
چون بدو پیوست جنس او شود
\\
همچو آب و نان که جنس ما نبود
&&
گشت جنس ما و اندر ما فزود
\\
نقش جنسیت ندارد آب و نان
&&
ز اعتبار آخر آن را جنس دان
\\
ور ز غیر جنس باشد ذوق ما
&&
آن مگر مانند باشد جنس را
\\
آنک مانندست باشد عاریت
&&
عاریت باقی نماند عاقبت
\\
مرغ را گر ذوق آید از صفیر
&&
چونک جنس خود نیابد شد نفیر
\\
تشنه را گر ذوق آید از سراب
&&
چون رسد در وی گریزد جوید آب
\\
مفلسان هم خوش شوند از زر قلب
&&
لیک آن رسوا شود در دار ضرب
\\
تا زر اندودیت از ره نفکند
&&
تا خیال کژ ترا چه نفکند
\\
از کلیله باز جو آن قصه را
&&
واندر آن قصه طلب کن حصه را
\\
\end{longtable}
\end{center}
