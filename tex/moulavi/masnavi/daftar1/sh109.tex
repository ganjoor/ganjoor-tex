\begin{center}
\section*{بخش ۱۰۹ - تفسیر دعای آن دو فرشته کی هر روز بر سر هر بازاری منادی می‌کنند کی اللهم اعط کل منفق خلفا اللهم اعط کل ممسک تلفا و بیان کردن کی آن منفق مجاهد راه حقست نی مسرف راه هوا}
\label{sec:sh109}
\addcontentsline{toc}{section}{\nameref{sec:sh109}}
\begin{longtable}{l p{0.5cm} r}
گفت پیغامبر که دایم بهر پند
&&
دو فرشته خوش منادی می‌کنند
\\
کای خدایا منفقان را سیر دار
&&
هر درمشان را عوض ده صد هزار
\\
ای خدایا ممسکان را در جهان
&&
تو مده الا زیان اندر زیان
\\
ای بسا امساک کز انفاق به
&&
مال حق را جز به امر حق مده
\\
تا عوض یابی تو گنج بی‌کران
&&
تا نباشی از عداد کافران
\\
کاشتران قربان همی‌کردند تا
&&
چیره گردد تیغشان بر مصطفی
\\
امر حق را باز جو از واصلی
&&
امر حق را در نیابد هر دلی
\\
چون غلام یاغیی کو عدل کرد
&&
مال شه بر یاغیان او بذل کرد
\\
در نبی انذار اهل غفلتست
&&
کان همه انفاقهاشان حسرتست
\\
عدل این یاغی و داداش نزد شاه
&&
چه فزاید دوری و روی سیاه
\\
سروران مکه در حرب رسول
&&
بودشان قربان به اومید قبول
\\
بهر اینمؤمنهمی‌گوید ز بیم
&&
در نماز اهد الصراط المستقیم
\\
آن درم دادن سخی را لایقست
&&
جان سپردن خود سخای عاشقست
\\
نان دهی از بهر حق نانت دهند
&&
جان دهی از بهر حق جانت دهند
\\
گر بریزد برگهای این چنار
&&
برگ بی‌برگیش بخشد کردگار
\\
گر نماند از جود در دست تو مال
&&
کی کند فضل الهت پای‌مال
\\
هر که کارد گردد انبارش تهی
&&
لیکش اندر مزرعه باشد بهی
\\
وانک در انبار ماند و صرفه کرد
&&
اشپش و موش حوادث پاک خورد
\\
این جهان نفیست در اثبات جو
&&
صورتت صفرست در معنیت جو
\\
جان شور تلخ پیش تیغ بر
&&
جان چون دریای شیرین را بخر
\\
ور نمی‌دانی شدن زین آستان
&&
باری از من گوش کن این داستان
\\
\end{longtable}
\end{center}
