\begin{center}
\section*{بخش ۶۶ - قصهٔ هدهد و سلیمان در بیان آنک چون قضا آید چشمهای روشن بسته شود}
\label{sec:sh066}
\addcontentsline{toc}{section}{\nameref{sec:sh066}}
\begin{longtable}{l p{0.5cm} r}
چون سلیمان را سراپرده زدند
&&
جمله مرغانش به خدمت آمدند
\\
هم‌زبان و محرم خود یافتند
&&
پیش او یک یک بجان بشتافتند
\\
جمله مرغان ترک کرده چیک چیک
&&
با سلیمان گشته افصح من اخیک
\\
همزبانی خویشی و پیوندی است
&&
مرد با نامحرمان چون بندی است
\\
ای بسا هندو و ترک همزبان
&&
ای بسا دو ترک چون بیگانگان
\\
پس زبان محرمی خود دیگرست
&&
همدلی از همزبانی بهترست
\\
غیرنطق و غیر ایما و سجل
&&
صد هزاران ترجمان خیزد ز دل
\\
جمله مرغان هر یکی اسرار خود
&&
از هنر وز دانش و از کار خود
\\
با سلیمان یک بیک وا می‌نمود
&&
از برای عرضه خود را می‌ستود
\\
از تکبر نه و از هستی خویش
&&
بهر آن تا ره دهد او را به پیش
\\
چون بباید برده را از خواجه‌ای
&&
عرضه دارد از هنر دیباجه‌ای
\\
چونک دارد از خریداریش ننگ
&&
خود کند بیمار و کر و شل و لنگ
\\
نوبت هدهد رسید و پیشه‌اش
&&
و آن بیان صنعت و اندیشه‌اش
\\
گفت ای شه یک هنر کان کهترست
&&
باز گویم گفت کوته بهترست
\\
گفت بر گو تا کدامست آن هنر
&&
گفت من آنگه که باشم اوج بر
\\
بنگرم از اوج با چشم یقین
&&
من ببینم آب در قعر زمین
\\
تا کجایست و چه عمقستش چه رنگ
&&
از چه می‌جوشد ز خاکی یا ز سنگ
\\
ای سلیمان بهر لشگرگاه را
&&
در سفر می‌دار این آگاه را
\\
پس سلیمان گفت ای نیکو رفیق
&&
در بیابانهای بی آب عمیق
\\
\end{longtable}
\end{center}
