\begin{center}
\section*{بخش ۱۷۰ - افتادن رکابدار هر باری پیش امیر الممنین علی کرم الله وجهه کی ای امیر الممنین مرا بکش و ازین قضا برهان}
\label{sec:sh170}
\addcontentsline{toc}{section}{\nameref{sec:sh170}}
\begin{longtable}{l p{0.5cm} r}
باز آمد کای علی زودم بکش
&&
تا نبینم آن دم و وقت ترش
\\
من حلالت می‌کنم خونم بریز
&&
تا نبیند چشم من آن رستخیز
\\
گفتم ار هر ذره‌ای خونی شود
&&
خنجر اندر کف به قصد تو رود
\\
یک سر مو از تو نتواند برید
&&
چون قلم بر تو چنان خطی کشید
\\
لیک بی غم شو شفیع تو منم
&&
خواجهٔ روحم نه مملوک تنم
\\
پیش من این تن ندارد قیمتی
&&
بی تن خویشم فتی ابن الفتی
\\
خنجر و شمشیر شد ریحان من
&&
مرگ من شد بزم و نرگسدان من
\\
آنک او تن را بدین سان پی کند
&&
حرص میری و خلافت کی کند
\\
زان به ظاهر کو شد اندر جاه و حکم
&&
تا امیران را نماید راه و حکم
\\
تا امیری را دهد جانی دگر
&&
تا دهد نخل خلافت را ثمر
\\
\end{longtable}
\end{center}
