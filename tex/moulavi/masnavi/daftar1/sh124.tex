\begin{center}
\section*{بخش ۱۲۴ - در معنی آنک مرج البحرین یلتقیان بینهما برزخ لا یبغیان}
\label{sec:sh124}
\addcontentsline{toc}{section}{\nameref{sec:sh124}}
\begin{longtable}{l p{0.5cm} r}
اهل نار و خلد را بین همدکان
&&
در میانشان برزخ لایبغیان
\\
اهل نار و اهل نور آمیخته
&&
در میانشان کوه قاف انگیخته
\\
همچو در کان خاک و زر کرد اختلاط
&&
در میانشان صد بیابان و رباط
\\
همچنانک عقد در در و شبه
&&
مختلط چون میهمان یک‌شبه
\\
بحر را نیمیش شیرین چون شکر
&&
طعم شیرین رنگ روشن چون قمر
\\
نیم دیگر تلخ همچون زهر مار
&&
طعم تلخ و رنگ مظلم همچو قار
\\
هر دو بر هم می‌زنند از تحت و اوج
&&
بر مثال آب دریا موج موج
\\
صورت بر هم زدن از جسم تنگ
&&
اختلاط جانها در صلح و جنگ
\\
موجهای صلح بر هم می‌زند
&&
کینه‌ها از سینه‌ها بر می‌کند
\\
موجهای جنگ بر شکل دگر
&&
مهرها را می‌کند زیر و زبر
\\
مهر تلخان را به شیرین می‌کشد
&&
زانک اصل مهرها باشد رشد
\\
قهر شیرین را به تلخی می‌برد
&&
تلخ با شیرین کجا اندر خورد
\\
تلخ و شیرین زین نظر ناید پدید
&&
از دریچهٔ عاقبت دانند دید
\\
چشم آخربین تواند دید راست
&&
چشم آخربین غرورست و خطاست
\\
ای بسا شیرین که چون شکر بود
&&
لیک زهر اندر شکر مضمر بود
\\
آنک زیرکتر ببو بشناسدش
&&
و آن دگر چون بر لب و دندان زدش
\\
پس لبش ردش کند پیش از گلو
&&
گرچه نعره می‌زند شیطان کلوا
\\
و آن دگر را در گلو پیدا کند
&&
و آن دگر را در بدن رسوا کند
\\
وان دگر را در حدث سوزش دهد
&&
ذوق آن زخم جگردوزش دهد
\\
وان دگر را بعد ایام و شهور
&&
وان دگر را بعد مرگ از قعر گور
\\
ور دهندش مهلت اندر قعر گور
&&
لابد آن پیدا شود یوم النشور
\\
هر نبات و شکری را در جهان
&&
مهلتی پیداست از دور زمان
\\
سالها باید که اندر آفتاب
&&
لعل یابد رنگ و رخشانی و تاب
\\
باز تره در دو ماه اندر رسد
&&
باز تا سالی گل احمر رسد
\\
بهر این فرمود حق عز و جل
&&
سورة الانعام در ذکر اجل
\\
این شنیدی مو بمویت گوش باد
&&
آب حیوانست خوردی نوش باد
\\
آب حیوان خوان مخوان این را سخن
&&
روح نو بین در تن حرف کهن
\\
نکتهٔ دیگر تو بشنو ای رفیق
&&
همچو جان او سخت پیدا و دقیق
\\
در مقامی هست هم این زهر مار
&&
از تصاریف خدایی خوش‌گوار
\\
در مقامی زهر و در جایی دوا
&&
در مقامی کفر و در جایی روا
\\
گرچه آنجا او گزند جان بود
&&
چون بدینجا در رسد درمان بود
\\
آب در غوره ترش باشد ولیک
&&
چون به انگوری رسد شیرین و نیک
\\
باز در خم او شود تلخ و حرام
&&
در مقام سرکگی نعم الادام
\\
\end{longtable}
\end{center}
