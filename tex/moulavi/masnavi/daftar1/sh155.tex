\begin{center}
\section*{بخش ۱۵۵ - اول کسی کی در مقابلهٔ نص قیاس آورد ابلیس بود}
\label{sec:sh155}
\addcontentsline{toc}{section}{\nameref{sec:sh155}}
\begin{longtable}{l p{0.5cm} r}
اول آن کس کین قیاسکها نمود
&&
پیش انوار خدا ابلیس بود
\\
گفت نار از خاک بی شک بهترست
&&
من ز نار و او ز خاک اکدرست
\\
پس قیاس فرع بر اصلش کنیم
&&
او ز ظلمت ما ز نور روشنیم
\\
گفت حق نه بلک لا انساب شد
&&
زهد و تقوی فضل را محراب شد
\\
این نه میراث جهان فانی است
&&
که به انسابش بیابی جانی است
\\
بلک این میراثهای انبیاست
&&
وارث این جانهای اتقیاست
\\
پور آن بوجهل شد مؤمن عیان
&&
پور آن نوح نبی از گمرهان
\\
زادهٔ خاکی منور شد چو ماه
&&
زادهٔ آتش توی رو روسیاه
\\
این قیاسات و تحری روز ابر
&&
یا بشب مر قبله را کردست حبر
\\
لیک با خورشید و کعبه پیش رو
&&
این قیاس و این تحری را مجو
\\
کعبه نادیده مکن رو زو متاب
&&
از قیاس الله اعلم بالصواب
\\
چون صفیری بشنوی از مرغ حق
&&
ظاهرش را یاد گیری چون سبق
\\
وانگهی از خود قیاساتی کنی
&&
مر خیال محض را ذاتی کنی
\\
اصطلاحاتیست مر ابدال را
&&
که نباشد زان خبر اقوال را
\\
منطق الطیری به صوت آموختی
&&
صد قیاس و صد هوس افروختی
\\
همچو آن رنجور دلها از تو خست
&&
کر بپندار اصابت گشته مست
\\
کاتب آن وحی زان آواز مرغ
&&
برده ظنی کو بود همباز مرغ
\\
مرغ پری زد مرورا کور کرد
&&
نک فرو بردش به قعر مرگ و درد
\\
هین به عکسی یا به ظنی هم شما
&&
در میفتید از مقامات سما
\\
گرچه هاروتید و ماروت و فزون
&&
از همه بر بام نحن الصافون
\\
بر بدیهای بدان رحمت کنید
&&
بر منی و خویش‌بین لعنت کنید
\\
هین مبادا غیرت آید از کمین
&&
سرنگون افتید در قعر زمین
\\
هر دو گفتند ای خدا فرمان تراست
&&
بی امان تو امانی خود کجاست
\\
این همی گفتند و دلشان می‌طپید
&&
بد کجا آید ز ما نعم العبید
\\
خار خار دو فرشته هم نهشت
&&
تا که تخم خویش‌بینی را نکشت
\\
پس همی گفتند کای ارکانیان
&&
بی خبر از پاکی روحانیان
\\
ما برین گردون تتقها می‌تنیم
&&
بر زمین آییم و شادروان زنیم
\\
عدل توزیم و عبادت آوریم
&&
باز هر شب سوی گردون بر پریم
\\
تا شویم اعجوبهٔ دور زمان
&&
تا نهیم اندر زمین امن و امان
\\
آن قیاس حال گردون بر زمین
&&
راست ناید فرق دارد در کمین
\\
\end{longtable}
\end{center}
