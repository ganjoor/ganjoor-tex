\begin{center}
\section*{بخش ۱۱۴ - صبر فرمودن اعرابی زن خود را و فضیلت صبر و فقر بیان کردن با زن}
\label{sec:sh114}
\addcontentsline{toc}{section}{\nameref{sec:sh114}}
\begin{longtable}{l p{0.5cm} r}
شوی گفتش چند جویی دخل و کشت
&&
خود چه ماند از عمر افزون‌تر گذشت
\\
عاقل اندر بیش و نقصان ننگرد
&&
زانک هر دو همچو سیلی بگذرد
\\
خواه صاف و خواه سیل تیره‌رو
&&
چون نمی‌پاید دمی از وی مگو
\\
اندرین عالم هزاران جانور
&&
می‌زید خوش‌عیش بی زیر و زبر
\\
شکر می‌گوید خدا را فاخته
&&
بر درخت و برگ شب نا ساخته
\\
حمد می‌گوید خدا را عندلیب
&&
کاعتماد رزق بر تست ای مجیب
\\
باز دست شاه را کرده نوید
&&
از همه مردار ببریده امید
\\
همچنین از پشه‌گیری تا به پیل
&&
شد عیال الله و حق نعم المعیل
\\
این همه غمها که اندر سینه‌هاست
&&
از بخار و گرد باد و بود ماست
\\
این غمان بیخ‌کن چون داس ماست
&&
این چنین شد و آنچنان وسواس ماست
\\
دان که هر رنجی ز مردن پاره‌ایست
&&
جزو مرگ از خود بران گر چاره‌ایست
\\
چون ز جزو مرگ نتوانی گریخت
&&
دان که کلش بر سرت خواهند ریخت
\\
جزو مرگ ار گشت شیرین مر ترا
&&
دان که شیرین می‌کند کل را خدا
\\
دردها از مرگ می‌آید رسول
&&
از رسولش رو مگردان ای فضول
\\
هر که شیرین می‌زید او تلخ مرد
&&
هر که او تن را پرستد جان نبرد
\\
گوسفندان را ز صحرا می‌کشند
&&
آنک فربه‌تر مر آن را می‌کشند
\\
شب گذشت و صبح آمد ای تمر
&&
چند گیری این فسانهٔ زر ز سر
\\
تو جوان بودی و قانع‌تر بدی
&&
زر طلب گشتی خود اول زر بدی
\\
رز بدی پر میوه چون کاسد شدی
&&
وقت میوه پختنت فاسد شدی
\\
میوه‌ات باید که شیرین‌تر شود
&&
چون رسن تابان نه واپس‌تر رود
\\
جفت مایی جفت باید هم‌صفت
&&
تا برآید کارها با مصلحت
\\
جفت باید بر مثال همدگر
&&
در دو جفت کفش و موزه در نگر
\\
گر یکی کفش از دو تنگ آید به پا
&&
هر دو جفتش کار ناید مر ترا
\\
جفت در یک خرد وان دیگر بزرگ
&&
جفت شیر بیشه دیدی هیچ گرگ
\\
راست ناید بر شتر جفت جوال
&&
آن یکی خالی و این پر مال مال
\\
من روم سوی قناعت دل‌قوی
&&
تو چرا سوی شناعت می‌روی
\\
مرد قانع از سر اخلاص و سوز
&&
زین نسق می‌گفت با زن تا بروز
\\
\end{longtable}
\end{center}
