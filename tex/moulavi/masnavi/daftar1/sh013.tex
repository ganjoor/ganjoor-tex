\begin{center}
\section*{بخش ۱۳ - آموختن وزیر مکر پادشاه را}
\label{sec:sh013}
\addcontentsline{toc}{section}{\nameref{sec:sh013}}
\begin{longtable}{l p{0.5cm} r}
او وزیری داشت گبر و عشوه ده
&&
کو بر آب از مکر بر بستی گره
\\
گفت ترسایان پناه جان کنند
&&
دین خود را از ملک پنهان کنند
\\
کم کش ایشان را که کشتن سود نیست
&&
دین ندارد بوی مشک و عود نیست
\\
سر پنهانست اندر صد غلاف
&&
ظاهرش با تست و باطن بر خلاف
\\
شاه گفتش پس بگو تدبیر چیست
&&
چارهٔ آن مکر و آن تزویر چیست
\\
تا نماند در جهان نصرانیی
&&
نی هویدا دین و نه پنهانیی
\\
گفت ای شه گوش و دستم را ببر
&&
بینی‌ام بشکاف و لب در حکم مر
\\
بعد از آن در زیردار آور مرا
&&
تا بخواهد یک شفاعت گر مرا
\\
بر منادی‌گاه کن این کار تو
&&
بر سر راهی که باشد چارسو
\\
آنگهم از خود بران تا شهر دور
&&
تا در اندازم دریشان شر و شور
\\
\end{longtable}
\end{center}
