\begin{center}
\section*{بخش ۱۶۸ - تعجب کردن آدم علیه‌السلام از ضلالت  ابلیس لعین و عجب آوردن}
\label{sec:sh168}
\addcontentsline{toc}{section}{\nameref{sec:sh168}}
\begin{longtable}{l p{0.5cm} r}
چشم آدم بر بلیسی کو شقی‌ست
&&
از حقارت وز زیافت بنگریست
\\
خویش‌بینی کرد و آمد خودگزین
&&
خنده زد بر کار ابلیس لعین
\\
بانگ بر زد غیرت حق کای صفی
&&
تو نمی‌دانی ز اسرار خفی
\\
پوستین را بازگونه گر کند
&&
کوه را از بیخ و از بن برکند
\\
پردهٔ صد آدم آن دم بر درد
&&
صد بلیس نو مسلمان آورد
\\
گفت آدم توبه کردم زین نظر
&&
این چنین گستاخ نندیشم دگر
\\
یا غیاث المستغیثین اهدنا
&&
لا افتخار بالعلوم و الغنی
\\
لا تزغ قلبا هدیت بالکرم
&&
واصرف السؤ الذی خط القلم
\\
بگذران از جان ما سؤ القضا
&&
وامبر ما را ز اخوان صفا
\\
تلخ‌تر از فرقت تو هیچ نیست
&&
بی پناهت غیر پیچاپیچ نیست
\\
رخت ما هم رخت ما را راه‌زن
&&
جسم ما مر جان ما را جامه کن
\\
دست ما چون پای ما را می‌خورد
&&
بی امان تو کسی جان چون برد
\\
ور برد جان زین خطرهای عظیم
&&
برده باشد مایهٔ ادبار و بیم
\\
زانک جان چون واصل جانان نبود
&&
تا ابد با خویش کورست و کبود
\\
چون تو ندهی راه جان خود برده گیر
&&
جان که بی تو زنده باشد مرده گیر
\\
گر تو طعنه می‌زنی بر بندگان
&&
مر ترا آن می‌رسد ای کامران
\\
ور تو ماه و مهر را گویی جفا
&&
ور تو قد سرو را گویی دوتا
\\
ور تو چرخ و عرش را خوانی حقیر
&&
ور تو کان و بحر را گویی فقیر
\\
آن بنسبت با کمال تو رواست
&&
ملک اکمال فناها مر تراست
\\
که تو پاکی از خطر وز نیستی
&&
نیستان را موجد و مغنیستی
\\
آنک رویانید داند سوختن
&&
زانک چون بدرید داند دوختن
\\
می‌بسوزد هر خزان مر باغ را
&&
باز رویاند گل صباغ را
\\
کای بسوزیده برون آ تازه شو
&&
بار دیگر خوب و خوب‌آوازه شو
\\
چشم نرگس کور شد بازش بساخت
&&
حلق نی ببرید و بازش خود نواخت
\\
ما چو مصنوعیم و صانع نیستیم
&&
جز زبون و جز که قانع نیستیم
\\
ما همه نفسی و نفسی می‌زنیم
&&
گر نخواهی ما همه آهرمنیم
\\
زان ز آهرمن رهیدستیم ما
&&
که خریدی جان ما را از عمی
\\
تو عصاکش هر کرا که زندگیست
&&
بی عصا و بی عصاکش کور چیست
\\
غیر تو هر چه خوشست و ناخوشست
&&
آدمی سوزست و عین آتشست
\\
هر که را آتش پناه و پشت شد
&&
هم مجوسی گشت و هم زردشت شد
\\
کل شیء ما خلا الله باطل
&&
ان فضل الله غیم هاطل
\\
\end{longtable}
\end{center}
