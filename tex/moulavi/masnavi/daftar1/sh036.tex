\begin{center}
\section*{بخش ۳۶ - حکایت پادشاه جهود دیگر کی در هلاک دین عیسی سعی نمود}
\label{sec:sh036}
\addcontentsline{toc}{section}{\nameref{sec:sh036}}
\begin{longtable}{l p{0.5cm} r}
یک شه دیگر ز نسل آن جهود
&&
در هلاک قوم عیسی رو نمود
\\
گر خبر خواهی ازین دیگر خروج
&&
سوره بر خوان واسما ذات البروج
\\
سنت بد کز شه اول بزاد
&&
این شه دیگر قدم بر وی نهاد
\\
هر که او بنهاد ناخوش سنتی
&&
سوی او نفرین رود هر ساعتی
\\
نیکوان رفتند و سنتها بماند
&&
وز لئیمان ظلم و لعنتها بماند
\\
تا قیامت هرکه جنس آن بدان
&&
در وجود آید بود رویش بدان
\\
رگ رگست این آب شیرین و آب شور
&&
در خلایق می‌رود تا نفخ صور
\\
نیکوان را هست میراث از خوشاب
&&
آن چه میراثست اورثنا الکتاب
\\
شد نیاز طالبان ار بنگری
&&
شعله‌ها از گوهر پیغامبری
\\
شعله‌ها با گوهران گردان بود
&&
شعله آن جانب رود هم کان بود
\\
نور روزن گرد خانه می‌دود
&&
زانک خور برجی به برجی می‌رود
\\
هر که را با اختری پیوستگیست
&&
مر ورا با اختر خود هم‌تگیست
\\
طالعش گر زهره باشد در طرب
&&
میل کلی دارد و عشق و طلب
\\
ور بود مریخی خون‌ریزخو
&&
جنگ و بهتان و خصومت جوید او
\\
اخترانند از ورای اختران
&&
که احتراق و نحس نبود اندر آن
\\
سایران در آسمانهای دگر
&&
غیر این هفت آسمان معتبر
\\
راسخان در تاب انوار خدا
&&
نه به هم پیوسته نه از هم جدا
\\
هر که باشد طالع او زان نجوم
&&
نفس او کفار سوزد در رجوم
\\
خشم مریخی نباشد خشم او
&&
منقلب رو غالب و مغلوب خو
\\
نور غالب ایمن از نقص و غسق
&&
درمیان اصبعین نور حق
\\
حق فشاند آن نور را بر جانها
&&
مقبلان بر داشته دامانها
\\
و آن نثار نور را وا یافته
&&
روی از غیر خدا برتافته
\\
هر که را دامان عشقی نابده
&&
زان نثار نور بی بهره شده
\\
جزوها را رویها سوی کلست
&&
بلبلان را عشق با روی گلست
\\
گاو را رنگ از برون و مرد را
&&
از درون جو رنگ سرخ و زرد را
\\
رنگهای نیک از خم صفاست
&&
رنگ زشتان از سیاهابهٔ جفاست
\\
صبغة الله نام آن رنگ لطیف
&&
لعنة الله بوی این رنگ کثیف
\\
آنچ از دریا به دریا می‌رود
&&
از همانجا کامد آنجا می‌رود
\\
از سر که سیلهای تیزرو
&&
وز تن ما جان عشق آمیز رو
\\
\end{longtable}
\end{center}
