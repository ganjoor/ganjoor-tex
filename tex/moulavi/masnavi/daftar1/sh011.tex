\begin{center}
\section*{بخش ۱۱ - حکایت بقال و طوطی و روغن ریختن طوطی در دکان}
\label{sec:sh011}
\addcontentsline{toc}{section}{\nameref{sec:sh011}}
\begin{longtable}{l p{0.5cm} r}
بود بقالی و وی را طوطیی
&&
خوش‌نوایی سبز و گویا طوطیی
\\
بر دکان بودی نگهبان دکان
&&
نکته گفتی با همه سوداگران
\\
در خطاب آدمی ناطق بدی
&&
در نوای طوطیان حاذق بدی
\\
خواجه روزی سوی خانه رفته بود
&&
بر دکان طوطی نگهبانی نمود
\\
گربه‌ای برجست ناگه بر دکان
&&
بهر موشی طوطیک از بیم جان
\\
جست از سوی دکان سویی گریخت
&&
شیشه‌های روغن گل را بریخت
\\
از سوی خانه بیامد خواجه‌اش
&&
بر دکان بنشست فارغ خواجه‌وش
\\
دید پر روغن دکان و جامه چرب
&&
بر سرش زد گشت طوطی کل ز ضرب
\\
روزکی چندی سخن کوتاه کرد
&&
مرد بقال از ندامت آه کرد
\\
ریش بر می‌کند و می‌گفت ای دریغ
&&
کافتاب نعمتم شد زیر میغ
\\
دست من بشکسته بودی آن زمان
&&
که زدم من بر سر آن خوش زبان
\\
هدیه‌ها می‌داد هر درویش را
&&
تا بیابد نطق مرغ خویش را
\\
بعد سه روز و سه شب حیران و زار
&&
بر دکان بنشسته بد نومیدوار
\\
می‌نمود آن مرغ را هر گون نهفت
&&
تا که باشد اندر آید او بگفت
\\
جولقیی سر برهنه می‌گذشت
&&
با سر بی مو چو پشت طاس و طشت
\\
آمد اندر گفت طوطی آن زمان
&&
بانگ بر درویش زد چون عاقلان
\\
کز چه ای کل با کلان آمیختی
&&
تو مگر از شیشه روغن ریختی
\\
از قیاسش خنده آمد خلق را
&&
کو چو خود پنداشت صاحب دلق را
\\
کار پاکان را قیاس از خود مگیر
&&
گر چه ماند در نبشتن شیر و شیر
\\
جمله عالم زین سبب گمراه شد
&&
کم کسی ز ابدال حق آگاه شد
\\
همسری با انبیا برداشتند
&&
اولیا را همچو خود پنداشتند
\\
گفته اینک ما بشر ایشان بشر
&&
ما و ایشان بستهٔ خوابیم و خور
\\
این ندانستند ایشان از عمی
&&
هست فرقی درمیان بی‌منتهی
\\
هر دو گون زنبور خوردند از محل
&&
لیک شد زان نیش و زین دیگر عسل
\\
هر دو گون آهو گیا خوردند و آب
&&
زین یکی سرگین شد و زان مشک ناب
\\
هر دو نی خوردند از یک آب‌خور
&&
این یکی خالی و آن پر از شکر
\\
صد هزاران این چنین اشباه بین
&&
فرقشان هفتاد ساله راه بین
\\
این خورد گردد پلیدی زو جدا
&&
آن خورد گردد همه نور خدا
\\
این خورد زاید همه بخل و حسد
&&
وآن خورد زاید همه نور احد
\\
این زمین پاک و آن شوره‌ست و بد
&&
این فرشتهٔ پاک و آن دیوست و دد
\\
هر دو صورت گر به هم ماند رواست
&&
آب تلخ و آب شیرین را صفاست
\\
جز که صاحب ذوق کی شناسد بیاب
&&
او شناسد آب خوش از شوره آب
\\
سحر را با معجزه کرده قیاس
&&
هر دو را بر مکر پندارد اساس
\\
ساحران موسی از استیزه را
&&
برگرفته چون عصای او عصا
\\
زین عصا تا آن عصا فرقیست ژرف
&&
زین عمل تا آن عمل راهی شگرف
\\
لعنة الله این عمل را در قفا
&&
رحمة الله آن عمل را در وفا
\\
کافران اندر مری بوزینه طبع
&&
آفتی آمد درون سینه طبع
\\
هرچه مردم می‌کند بوزینه هم
&&
آن کند کز مرد بیند دم بدم
\\
او گمان برده که من کردم چو او
&&
فرق را کی داند آن استیزه‌رو
\\
این کند از امر و او بهر ستیز
&&
بر سر استیزه‌رویان خاک ریز
\\
آن منافق با موافق در نماز
&&
از پی استیزه آید نه نیاز
\\
در نماز و روزه و حج و زکات
&&
با منافق مؤمنان در برد و مات
\\
مؤمنان را برد باشد عاقبت
&&
بر منافق مات اندر آخرت
\\
گرچه هر دو بر سر یک بازی‌اند
&&
هر دو با هم مروزی و رازی‌اند
\\
هر یکی سوی مقام خود رود
&&
هر یکی بر وفق نام خود رود
\\
مؤمنش خوانند جانش خوش شود
&&
ور منافق تیز و پر آتش شود
\\
نام او محبوب از ذات وی است
&&
نام این مبغوض از آفات وی است
\\
میم و واو و میم و نون تشریف نیست
&&
لطف مؤمن جز پی تعریف نیست
\\
گر منافق خوانیش این نام دون
&&
همچو کزدم می‌خلد در اندرون
\\
گرنه این نام اشتقاق دوزخست
&&
پس چرا در وی مذاق دوزخست
\\
زشتی آن نام بد از حرف نیست
&&
تلخی آن آب بحر از ظرف نیست
\\
حرف ظرف آمد درو معنی چون آب
&&
بحر معنی عنده ام الکتاب
\\
بحر تلخ و بحر شیرین در جهان
&&
در میانشان برزخ لا یبغیان
\\
وانگه این هر دو ز یک اصلی روان
&&
بر گذر زین هر دو رو تا اصل آن
\\
زر قلب و زر نیکو در عیار
&&
بی محک هرگز ندانی ز اعتبار
\\
هر که را در جان خدا بنهد محک
&&
هر یقین را باز داند او ز شک
\\
در دهان زنده خاشاکی جهد
&&
آنگه آرامد که بیرونش نهد
\\
در هزاران لقمه یک خاشاک خرد
&&
چون در آمد حس زنده پی ببرد
\\
حس دنیا نردبان این جهان
&&
حس دینی نردبان آسمان
\\
صحت این حس بجویید از طبیب
&&
صحت آن حس بجویید از حبیب
\\
صحت این حس ز معموری تن
&&
صحت آن حس ز تخریب بدن
\\
راه جان مر جسم را ویران کند
&&
بعد از آن ویرانی آبادان کند
\\
کرد ویران خانه بهر گنج زر
&&
وز همان گنجش کند معمورتر
\\
آب را ببرید و جو را پاک کرد
&&
بعد از آن در جو روان کرد آب خورد
\\
پوست را بشکافت و پیکان را کشید
&&
پوست تازه بعد از آنش بر دمید
\\
قلعه ویران کرد و از کافر ستد
&&
بعد از آن بر ساختش صد برج و سد
\\
کار بی‌چون را که کیفیت نهد
&&
اینک گفتم این ضرورت می‌دهد
\\
گه چنین بنماید و گه ضد این
&&
جز که حیرانی نباشد کار دین
\\
نه چنان حیران که پشتش سوی اوست
&&
بل چنان حیران و غرق و مست دوست
\\
آن یکی را روی او شد سوی دوست
&&
وان یکی را روی او خود روی اوست
\\
روی هر یک می‌نگر می‌دار پاس
&&
بوک گردی تو ز خدمت روشناس
\\
چون بسی ابلیس آدم‌روی هست
&&
پس بهر دستی نشاید داد دست
\\
زانک صیاد آورد بانگ صفیر
&&
تا فریبد مرغ را آن مرغ‌گیر
\\
بشنود آن مرغ بانگ جنس خویش
&&
از هوا آید بیاید دام و نیش
\\
حرف درویشان بدزدد مرد دون
&&
تا بخواند بر سلیمی زان فسون
\\
کار مردان روشنی و گرمیست
&&
کار دونان حیله و بی‌شرمیست
\\
شیر پشمین از برای کد کنند
&&
بومسیلم را لقب احمد کنند
\\
بومسیلم را لقب کذاب ماند
&&
مر محمد را اولوا الالباب ماند
\\
آن شراب حق ختامش مشک ناب
&&
باده را ختمش بود گند و عذاب
\\
\end{longtable}
\end{center}
