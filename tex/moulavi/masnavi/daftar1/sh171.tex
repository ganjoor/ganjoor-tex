\begin{center}
\section*{بخش ۱۷۱ - بیان آنک فتح طلبیدن مصطفی صلی الله علیه و سلم مکه را و غیر مکه را جهت دوستی ملک دنیا نبود چون فرموده است الدنیا جیفة بلک بامر بود}
\label{sec:sh171}
\addcontentsline{toc}{section}{\nameref{sec:sh171}}
\begin{longtable}{l p{0.5cm} r}
جهد پیغامبر بفتح مکه هم
&&
کی بود در حب دنیا متهم
\\
آنک او از مخزن هفت آسمان
&&
چشم و دل بر بست روز امتحان
\\
از پی نظارهٔ او حور و جان
&&
پر شده آفاق هر هفت آسمان
\\
خویشتن آراسته از بهر او
&&
خود ورا پروای غیر دوست کو
\\
آنچنان پر گشته از اجلال حق
&&
که درو هم ره نیابد آل حق
\\
لا یسع فینا نبی مرسل
&&
والملک و الروح ایضا فاعقلوا
\\
گفت ما زاغیم همچون زاغ نه
&&
مست صباغیم مست باغ نه
\\
چونک مخزنهای افلاک و عقول
&&
چون خسی آمد بر چشم رسول
\\
پس چه باشد مکه و شام و عراق
&&
که نماید او نبرد و اشتیاق
\\
آن گمان بر وی ضمیر بد کند
&&
کو قیاس از جهل و حرص خود کند
\\
آبگینهٔ زرد چون سازی نقاب
&&
زرد بینی جمله نور آفتاب
\\
بشکن آن شیشهٔ کبود و زرد را
&&
تا شناسی گرد را و مرد را
\\
گرد فارس گرد سر افراشته
&&
گرد را تو مرد حق پنداشته
\\
گرد دید ابلیس و گفت این فرع طین
&&
چون فزاید بر من آتش‌جبین
\\
تا تو می‌بینی عزیزان را بشر
&&
دانک میراث بلیسست آن نظر
\\
گر نه فرزندی بلیسی ای عنید
&&
پس به تو میراث آن سگ چون رسید
\\
من نیم سگ شیر حقم حق‌پرست
&&
شیر حق آنست کز صورت برست
\\
شیر دنیا جوید اشکاری و برگ
&&
شیر مولی جوید آزادی و مرگ
\\
چونک اندر مرگ بیند صد وجود
&&
همچو پروانه بسوزاند وجود
\\
شد هوای مرگ طوق صادقان
&&
که جهودان را بد این دم امتحان
\\
در نبی فرمود کای قوم یهود
&&
صادقان را مرگ باشد گنج و سود
\\
همچنانک آرزوی سود هست
&&
آرزوی مرگ بردن زان بهست
\\
ای جهودان بهر ناموس کسان
&&
بگذرانید این تمنا بر زبان
\\
یک جهودی این قدر زهره نداشت
&&
چون محمد این علم را بر فراشت
\\
گفت اگر رانید این را بر زبان
&&
یک یهودی خود نماند در جهان
\\
پس یهودان مال بردند و خراج
&&
که مکن رسوا تو ما را ای سراج
\\
این سخن را نیست پایانی پدید
&&
دست با من ده چو چشمت دوست دید
\\
\end{longtable}
\end{center}
