\begin{center}
\section*{بخش ۱۳۰ - در نمد دوختن زن عرب سبوی آب باران را و مهر نهادن بر وی از غایت اعتقاد عرب}
\label{sec:sh130}
\addcontentsline{toc}{section}{\nameref{sec:sh130}}
\begin{longtable}{l p{0.5cm} r}
مرد گفت آری سبو را سر ببند
&&
هین که این هدیه‌ست ما را سودمند
\\
در نمد در دوز تو این کوزه را
&&
تا گشاید شه بهدیه روزه را
\\
کین چنین اندر همه آفاق نیست
&&
جز رحیق و مایهٔ اذواق نیست
\\
زانک ایشان ز آبهای تلخ و شور
&&
دایما پر علت‌اند و نیم‌کور
\\
مرغ کاب شور باشد مسکنش
&&
او چه داند جای آب روشنش
\\
ای که اندر چشمهٔ شورست جاث
&&
تو چه دانی شط و جیحون و فرات
\\
ای تو نارسته ازین فانی رباط
&&
تو چه دانی محو و سکر و انبساط
\\
ور بدانی نقلت از اب و جدست
&&
پیش تو این نامها چون ابجدست
\\
ابجد و هوز چه فاش است و پدید
&&
بر همه طفلان و معنی بس بعید
\\
پس سبو برداشت آن مرد عرب
&&
در سفر شد می‌کشیدش روز و شب
\\
بر سبو لرزان بد از آفات دهر
&&
هم کشیدش از بیابان تا به شهر
\\
زن مصلا باز کرده از نیاز
&&
رب سلم ورد کرده در نماز
\\
که نگه‌دار آب ما را از خسان
&&
یا رب آن گوهر بدان دریا رسان
\\
گرچه شویم آگهست و پر فنست
&&
لیک گوهر را هزاران دشمنست
\\
خود چه باشد گوهر آب کوثرست
&&
قطره‌ای زینست کاصل گوهرست
\\
از دعاهای زن و زاری او
&&
وز غم مرد و گران‌باری او
\\
سالم از دزدان و از آسیب سنگ
&&
برد تا دار الخلافه بی‌درنگ
\\
دید درگاهی پر از انعامها
&&
اهل حاجت گستریده دامها
\\
دم بدم هر سوی صاحب‌حاجتی
&&
یافته زان در عطا و خلعتی
\\
بهر گبر و مؤمن و زیبا و زشت
&&
همچو خورشید و مطر نی چون بهشت
\\
دید قومی درنظر آراسته
&&
قوم دیگر منتظر بر خاسته
\\
خاص و عامه از سلیمان تا بمور
&&
زنده گشته چون جهان از نفخ صور
\\
اهل صورت در جواهر بافته
&&
اهل معنی بحر معنی یافته
\\
آنک بی همت چه با همت شده
&&
وانک با همت چه با نعمت شده
\\
\end{longtable}
\end{center}
