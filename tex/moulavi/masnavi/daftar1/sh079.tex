\begin{center}
\section*{بخش ۷۹ - سوال کردن رسول روم از امیرالممنین عمر رضی‌الله عنه}
\label{sec:sh079}
\addcontentsline{toc}{section}{\nameref{sec:sh079}}
\begin{longtable}{l p{0.5cm} r}
مرد گفتش کای امیرالمؤمنین
&&
جان ز بالا چون در آمد در زمین
\\
مرغ بی‌اندازه چون شد در قفس
&&
گفت حق بر جان فسون خواند و قصص
\\
بر عدمها کان ندارد چشم و گوش
&&
چون فسون خواند همی آید به جوش
\\
از فسون او عدمها زود زود
&&
خوش معلق می‌زند سوی وجود
\\
باز بر موجود افسونی چو خواند
&&
زو دو اسپه در عدم موجود راند
\\
گفت در گوش گل و خندانش کرد
&&
گفت با سنگ و عقیق کانش کرد
\\
گفت با جسم آیتی تا جان شد او
&&
گفت با خورشید تا رخشان شد او
\\
باز در گوشش دمد نکتهٔ مخوف
&&
در رخ خورشید افتد صد کسوف
\\
تا به گوش ابر آن گویا چه خواند
&&
کو چو مشک از دیدهٔ خود اشک راند
\\
تا به گوش خاک حق چه خوانده است
&&
کو مراقب گشت و خامش مانده است
\\
در تردد هر که او آشفته است
&&
حق به گوش او معما گفته است
\\
تا کند محبوسش اندر دو گمان
&&
آن کنم آن گفت یا خود ضد آن
\\
هم ز حق ترجیح یابد یک طرف
&&
زان دو یک را برگزیند زان کنف
\\
گر نخواهی در تردد هوش جان
&&
کم فشار این پنبه اندر گوش جان
\\
تا کنی فهم آن معماهاش را
&&
تا کنی ادراک رمز و فاش را
\\
پس محل وحی گردد گوش جان
&&
وحی چه بود گفتنی از حس نهان
\\
گوش جان و چشم جان جز این حس است
&&
گوش عقل و گوش ظن زین مفلس است
\\
لفظ جبرم عشق را بی‌صبر کرد
&&
وانک عاشق نیست حبس جبر کرد
\\
این معیت با حقست و جبر نیست
&&
این تجلی مه است این ابر نیست
\\
ور بود این جبر جبر عامه نیست
&&
جبر آن امارهٔ خودکامه نیست
\\
جبر را ایشان شناسند ای پسر
&&
که خدا بگشادشان در دل بصر
\\
غیب و آینده بریشان گشت فاش
&&
ذکر ماضی پیش ایشان گشت لاش
\\
اختیار و جبر ایشان دیگرست
&&
قطره‌ها اندر صدفها گوهرست
\\
هست بیرون قطرهٔ خرد و بزرگ
&&
در صدف آن در خردست و سترگ
\\
طبع ناف آهوست آن قوم را
&&
از برون خون و درونشان مشکها
\\
تو مگو کین مایه بیرون خون بود
&&
چون رود در ناف مشکی چون شود
\\
تو مگو کین مس برون بد محتقر
&&
در دل اکسیر چون گیرد گهر
\\
اختیار و جبر در تو بد خیال
&&
چون دریشان رفت شد نور جلال
\\
نان چو در سفره‌ست باشد آن جماد
&&
در تن مردم شود او روح شاد
\\
در دل سفره نگردد مستحیل
&&
مستحیلش جان کند از سلسبیل
\\
قوت جانست این ای راست‌خوان
&&
تا چه باشد قوت آن جان جان
\\
گوشت پارهٔ آدمی با عقل و جان
&&
می‌شکافد کوه را با بحر و کان
\\
زور جان کوه کن شق حجر
&&
زور جان جان در انشق القمر
\\
گر گشاید دل سر انبان راز
&&
جان به سوی عرش سازد ترک‌تاز
\\
\end{longtable}
\end{center}
