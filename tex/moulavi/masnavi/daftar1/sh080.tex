\begin{center}
\section*{بخش ۸۰ - اضافت کردن آدم علیه‌السلام آن زلت را به خویشتن کی ربنا ظلمنا و اضافت کردن ابلیس گناه خود را به خدای تعالی کی بما اغویتنی}
\label{sec:sh080}
\addcontentsline{toc}{section}{\nameref{sec:sh080}}
\begin{longtable}{l p{0.5cm} r}
کرد حق و کرد ما هر دو ببین
&&
کرد ما را هست دان پیداست این
\\
گر نباشد فعل خلق اندر میان
&&
پس مگو کس را چرا کردی چنان
\\
خلق حق افعال ما را موجدست
&&
فعل ما آثار خلق ایزدست
\\
ناطقی یا حرف بیند یا غرض
&&
کی شود یک دم محیط دو عرض
\\
گر به معنی رفت شد غافل ز حرف
&&
پیش و پس یک دم نبیند هیچ طرف
\\
آن زمان که پیش‌بینی آن زمان
&&
تو پس خود کی ببینی این بدان
\\
چون محیط حرف و معنی نیست جان
&&
چون بود جان خالق این هر دوان
\\
حق محیط جمله آمد ای پسر
&&
وا ندارد کارش از کار دگر
\\
گفت شیطان که بما اغویتنی
&&
کرد فعل خود نهان دیو دنی
\\
گفت آدم که ظلمنا نفسنا
&&
او ز فعل حق نبد غافل چو ما
\\
در گنه او از ادب پنهانش کرد
&&
زان گنه بر خود زدن او بر بخورد
\\
بعد توبه گفتش ای آدم نه من
&&
آفریدم در تو آن جرم و محن
\\
نه که تقدیر و قضای من بد آن
&&
چون به وقت عذر کردی آن نهان
\\
گفت ترسیدم ادب نگذاشتم
&&
گفت هم من پاس آنت داشتم
\\
هر که آرد حرمت او حرمت برد
&&
هر که آرد قند لوزینه خورد
\\
طیبات از بهر کی للطیبین
&&
یار را خوش کن برنجان و ببین
\\
یک مثال ای دل پی فرقی بیار
&&
تا بدانی جبر را از اختیار
\\
دست کان لرزان بود از ارتعاش
&&
وانک دستی تو بلرزانی ز جاش
\\
هر دو جنبش آفریدهٔ حق شناس
&&
لیک نتوان کرد این با آن قیاس
\\
زان پشیمانی که لرزانیدیش
&&
مرتعش را کی پشیمان دیدیش
\\
بحث عقلست این چه عقل آن حیله‌گر
&&
تا ضعیفی ره برد آنجا مگر
\\
بحث عقلی گر در و مرجان بود
&&
آن دگر باشد که بحث جان بود
\\
بحث جان اندر مقامی دیگرست
&&
بادهٔ جان را قوامی دیگرست
\\
آن زمان که بحث عقلی ساز بود
&&
این عمر با بوالحکم همراز بود
\\
چون عمر از عقل آمد سوی جان
&&
بوالحکم بوجهل شد در حکم آن
\\
سوی حس و سوی عقل او کاملست
&&
گرچه خود نسبت به جان او جاهلست
\\
بحث عقل و حس اثر دان یا سبب
&&
بحث جانی یا عجب یا بوالعجب
\\
ضؤ جان آمد نماند ای مستضی
&&
لازم و ملزوم و نافی مقتضی
\\
زانک بینایی که نورش بازغست
&&
از دلیل چون عصا بس فارغست
\\
\end{longtable}
\end{center}
