\begin{center}
\section*{بخش ۱۰۱ - در معنی این حدیث کی اغتنموا برد الربیع الی آخره}
\label{sec:sh101}
\addcontentsline{toc}{section}{\nameref{sec:sh101}}
\begin{longtable}{l p{0.5cm} r}
گفت پیغامبر ز سرمای بهار
&&
تن مپوشانید یاران زینهار
\\
زانک با جان شما آن می‌کند
&&
کان بهاران با درختان می‌کند
\\
لیک بگریزید از سرد خزان
&&
کان کند کو کرد با باغ و رزان
\\
راویان این را به ظاهر برده‌اند
&&
هم بر آن صورت قناعت کرده‌اند
\\
بی‌خبر بودند از جان آن گروه
&&
کوه را دیده ندیده کان بکوه
\\
آن خزان نزد خدا نفس و هواست
&&
عقل و جان عین بهارست و بقاست
\\
مر ترا عقلیست جزوی در نهان
&&
کامل العقلی بجو اندر جهان
\\
جزو تو از کل او کلی شود
&&
عقل کل بر نفس چون غلی شود
\\
پس بتاویل این بود کانفاس پاک
&&
چون بهارست و حیات برگ و تاک
\\
از حدیث اولیا نرم و درشت
&&
تن مپوشان زانک دینت راست پشت
\\
گرم گوید سرد گوید خوش بگیر
&&
تا ز گرم و سرد بجهی وز سعیر
\\
گرم و سردش نوبهار زندگیست
&&
مایهٔ صدق و یقین و بندگیست
\\
زان کزو بستان جانها زنده است
&&
زین جواهر بحر دل آگنده است
\\
بر دل عاقل هزاران غم بود
&&
گر ز باغ دل خلالی کم شود
\\
\end{longtable}
\end{center}
