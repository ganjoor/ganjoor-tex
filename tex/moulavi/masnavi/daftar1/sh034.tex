\begin{center}
\section*{بخش ۳۴ - منازعت امرا در ولی عهدی}
\label{sec:sh034}
\addcontentsline{toc}{section}{\nameref{sec:sh034}}
\begin{longtable}{l p{0.5cm} r}
یک امیری زان امیران پیش رفت
&&
پیش آن قوم وفا اندیش رفت
\\
گفت اینک نایب آن مرد من
&&
نایب عیسی منم اندر زمن
\\
اینک این طومار برهان منست
&&
کین نیابت بعد ازو آن منست
\\
آن امیر دیگر آمد از کمین
&&
دعوی او در خلافت بد همین
\\
از بغل او نیز طوماری نمود
&&
تا برآمد هر دو را خشم جهود
\\
آن امیران دگر یک‌یک قطار
&&
برکشیده تیغهای آبدار
\\
هر یکی را تیغ و طوماری به دست
&&
درهم افتادند چون پیلان مست
\\
صد هزاران مرد ترسا کشته شد
&&
تا ز سرهای بریده پشته شد
\\
خون روان شد همچو سیل از چپ و راست
&&
کوه کوه اندر هوا زین گرد خاست
\\
تخمهای فتنه‌ها کو کشته بود
&&
آفت سرهای ایشان گشته بود
\\
جوزها بشکست و آن کان مغز داشت
&&
بعد کشتن روح پاک نغز داشت
\\
کشتن و مردن که بر نقش تنست
&&
چون انار و سیب را بشکستنست
\\
آنچ شیرینست او شد ناردانگ
&&
وانک پوسیده‌ست نبود غیر بانگ
\\
آنچ با معنیست خود پیدا شود
&&
وانچ پوسیده‌ست او رسوا شود
\\
رو بمعنی کوش ای صورت‌پرست
&&
زانک معنی بر تن صورت‌پرست
\\
همنشین اهل معنی باش تا
&&
هم عطا یابی و هم باشی فتی
\\
جان بی‌معنی درین تن بی‌خلاف
&&
هست همچون تیغ چوبین در غلاف
\\
تا غلاف اندر بود باقیمتست
&&
چون برون شد سوختن را آلتست
\\
تیغ چوبین را مبر در کارزار
&&
بنگر اول تا نگردد کار زار
\\
گر بود چوبین برو دیگر طلب
&&
ور بود الماس پیش آ با طرب
\\
تیغ در زرادخانهٔ اولیاست
&&
دیدن ایشان شما را کیمیاست
\\
جمله دانایان همین گفته همین
&&
هست دانا رحمة للعالمین
\\
گر اناری می‌خری خندان بخر
&&
تا دهد خنده ز دانهٔ او خبر
\\
ای مبارک خنده‌اش کو از دهان
&&
می‌نماید دل چو در از درج جان
\\
نامبارک خندهٔ آن لاله بود
&&
کز دهان او سیاهی دل نمود
\\
نار خندان باغ را خندان کند
&&
صحبت مردانت از مردان کند
\\
گر تو سنگ صخره و مرمر شوی
&&
چون به صاحب دل رسی گوهر شوی
\\
مهر پاکان درمیان جان نشان
&&
دل مده الا به مهر دلخوشان
\\
کوی نومیدی مرو اومیدهاست
&&
سوی تاریکی مرو خورشیدهاست
\\
دل ترا در کوی اهل دل کشد
&&
تن ترا در حبس آب و گل کشد
\\
هین غذای دل بده از همدلی
&&
رو بجو اقبال را از مقبلی
\\
\end{longtable}
\end{center}
