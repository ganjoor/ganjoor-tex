\begin{center}
\section*{بخش ۳۳ - طلب کردن امت عیسی علیه‌السلام از امراکی ولی عهد از شما کدامست}
\label{sec:sh033}
\addcontentsline{toc}{section}{\nameref{sec:sh033}}
\begin{longtable}{l p{0.5cm} r}
بعد ماهی خلق گفتند ای مهان
&&
از امیران کیست بر جایش نشان
\\
تا به جای او شناسیمش امام
&&
دست و دامن را به دست او دهیم
\\
چونک شد خورشید و ما را کرد داغ
&&
چاره نبود بر مقامش از چراغ
\\
چونک شد از پیش دیده وصل یار
&&
نایبی باید ازومان یادگار
\\
چونک گل بگذشت و گلشن شد خراب
&&
بوی گل را از که یابیم از گلاب
\\
چون خدا اندر نیاید در عیان
&&
نایب حق‌اند این پیغامبران
\\
نه غلط گفتم که نایب با منوب
&&
گر دو پنداری قبیح آید نه خوب
\\
نه دو باشد تا توی صورت‌پرست
&&
پیش او یک گشت کز صورت برست
\\
چون به صورت بنگری چشم تو دوست
&&
تو به نورش در نگر کز چشم رست
\\
نور هر دو چشم نتوان فرق کرد
&&
چونک در نورش نظر انداخت مرد
\\
ده چراغ ار حاضر آید در مکان
&&
هر یکی باشد بصورت غیر آن
\\
فرق نتوان کرد نور هر یکی
&&
چون به نورش روی آری بی‌شکی
\\
گر تو صد سیب و صد آبی بشمری
&&
صد نماند یک شود چون بفشری
\\
در معانی قسمت و اعداد نیست
&&
در معانی تجزیه و افراد نیست
\\
اتحاد یار با یاران خوشست
&&
پای معنی‌گیر صورت سرکشست
\\
صورت سرکش گدازان کن برنج
&&
تا ببینی زیر او وحدت چو گنج
\\
ور تو نگدازی عنایتهای او
&&
خود گدازد ای دلم مولای او
\\
او نماید هم به دلها خویش را
&&
او بدوزد خرقهٔ درویش را
\\
منبسط بودیم یک جوهر همه
&&
بی‌سر و بی پا بدیم آن سر همه
\\
یک گهر بودیم همچون آفتاب
&&
بی گره بودیم و صافی همچو آب
\\
چون بصورت آمد آن نور سره
&&
شد عدد چون سایه‌های کنگره
\\
گنگره ویران کنید از منجنیق
&&
تا رود فرق از میان این فریق
\\
شرح این را گفتمی من از مری
&&
لیک ترسم تا نلغزد خاطری
\\
نکته‌ها چون تیغ پولادست تیز
&&
گر نداری تو سپر وا پس گریز
\\
پیش این الماس بی اسپر میا
&&
کز بریدن تیغ را نبود حیا
\\
زین سبب من تیغ کردم در غلاف
&&
تا که کژخوانی نخواند برخلاف
\\
آمدیم اندر تمامی داستان
&&
وز وفاداری جمع راستان
\\
کز پس این پیشوا بر خاستند
&&
بر مقامش نایبی می‌خواستند
\\
\end{longtable}
\end{center}
