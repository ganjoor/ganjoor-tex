\begin{center}
\section*{بخش ۵۸ - منع کردن خرگوش از راز ایشان را}
\label{sec:sh058}
\addcontentsline{toc}{section}{\nameref{sec:sh058}}
\begin{longtable}{l p{0.5cm} r}
گفت هر رازی نشاید باز گفت
&&
جفت طاق آید گهی گه طاق جفت
\\
از صفا گر دم زنی با آینه
&&
تیره گردد زود با ما آینه
\\
در بیان این سه کم جنبان لبت
&&
از ذهاب و از ذهب وز مذهبت
\\
کین سه را خصمست بسیار و عدو
&&
در کمینت ایستد چون داند او
\\
ور بگویی با یکی دو الوداع
&&
کل سر جاوز الاثنین شاع
\\
گر دو سه پرنده را بندی بهم
&&
بر زمین مانند محبوس از الم
\\
مشورت دارند سرپوشیده خوب
&&
در کنایت با غلط‌افکن مشوب
\\
مشورت کردی پیمبر بسته‌سر
&&
گفته ایشانش جواب و بی‌خبر
\\
در مثالی بسته گفتی رای را
&&
تا ندانند خصم از سر پای را
\\
او جواب خویش بگرفتی ازو
&&
وز سؤالش می‌نبردی غیر بو
\\
\end{longtable}
\end{center}
