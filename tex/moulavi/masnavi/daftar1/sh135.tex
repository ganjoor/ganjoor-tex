\begin{center}
\section*{بخش ۱۳۵ - مثل عرب اذا زنیت فازن بالحرة و اذا سرقت فاسرق الدرة}
\label{sec:sh135}
\addcontentsline{toc}{section}{\nameref{sec:sh135}}
\begin{longtable}{l p{0.5cm} r}
فازن بالحرة پی این شد مثل
&&
فاسرق الدرة بدین شد منتقل
\\
بنده سوی خواجه شد او ماند زار
&&
بوی گل شد سوی گل او ماند خار
\\
او بمانده دور از مطلوب خویش
&&
سعی ضایع رنج باطل پای ریش
\\
همچو صیادی که گیرد سایه‌ای
&&
سایه کی گردد ورا سرمایه‌ای
\\
سایهٔ مرغی گرفته مرد سخت
&&
مرغ حیران گشته بر شاخ درخت
\\
کین مدمغ بر کی می‌خندد عجب
&&
اینت باطل اینت پوسیده سبب
\\
ور تو گویی جزو پیوستهٔ کلست
&&
خار می‌خور خار مقرون گلست
\\
جز ز یک رو نیست پیوسته به کل
&&
ورنه خود باطل بدی بعث رسل
\\
چون رسولان از پی پیوستنند
&&
پس چه پیوندندشان چون یک تنند
\\
این سخن پایان ندارد ای غلام
&&
روز بیگه شد حکایت کن تمام
\\
\end{longtable}
\end{center}
