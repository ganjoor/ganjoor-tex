\begin{center}
\section*{بخش ۱۵۷ - قصهٔ مری کردن رومیان و چینیان در علم نقاشی و صورت‌گری}
\label{sec:sh157}
\addcontentsline{toc}{section}{\nameref{sec:sh157}}
\begin{longtable}{l p{0.5cm} r}
چینیان گفتند ما نقاش‌تر
&&
رومیان گفتند ما را کر و فر
\\
گفت سلطان امتحان خواهم درین
&&
کز شماها کیست در دعوی گزین
\\
اهل چین و روم چون حاضر شدند
&&
رومیان در علم واقف‌تر بدند
\\
چینیان گفتند یک خانه به ما
&&
خاص بسپارید و یک آن شما
\\
بود دو خانه مقابل در بدر
&&
زان یکی چینی ستد رومی دگر
\\
چینیان صد رنگ از شه خواستند
&&
پس خزینه باز کرد آن ارجمند
\\
هر صباحی از خزینه رنگها
&&
چینیان را راتبه بود از عطا
\\
رومیان گفتند نه نقش و نه رنگ
&&
در خور آید کار را جز دفع زنگ
\\
در فرو بستند و صیقل می‌زدند
&&
همچو گردون ساده و صافی شدند
\\
از دو صد رنگی به بی‌رنگی رهیست
&&
رنگ چون ابرست و بی‌رنگی مهیست
\\
هرچه اندر ابر ضو بینی و تاب
&&
آن ز اختر دان و ماه و آفتاب
\\
چینیان چون از عمل فارغ شدند
&&
از پی شادی دهلها می‌زدند
\\
شه در آمد دید آنجا نقشها
&&
می‌ربود آن عقل را و فهم را
\\
بعد از آن آمد به سوی رومیان
&&
پرده را بالا کشیدند از میان
\\
عکس آن تصویر و آن کردارها
&&
زد برین صافی شده دیوارها
\\
هر چه آنجا دید اینجا به نمود
&&
دیده را از دیده‌خانه می‌ربود
\\
رومیان آن صوفیانند ای پدر
&&
بی ز تکرار و کتاب و بی هنر
\\
لیک صیقل کرده‌اند آن سینه‌ها
&&
پاک از آز و حرص و بخل و کینه‌ها
\\
آن صفای آینه وصف دلست
&&
صورت بی منتها را قابلست
\\
صورت بی‌صورت بی حد غیب
&&
ز آینهٔ دل تافت بر موسی ز جیب
\\
گرچه آن صورت نگنجد در فلک
&&
نه بعرش و فرش و دریا و سمک
\\
زانک محدودست و معدودست آن
&&
آینهٔ دل را نباشد حد بدان
\\
عقل اینجا ساکت آمد یا مضل
&&
زانک دل یا اوست یا خود اوست دل
\\
عکس هر نقشی نتابد تا ابد
&&
جز ز دل هم با عدد هم بی عدد
\\
تا ابد هر نقش نو کاید برو
&&
می‌نماید بی حجابی اندرو
\\
اهل صیقل رسته‌اند از بوی و رنگ
&&
هر دمی بینند خوبی بی درنگ
\\
نقش و قشر علم را بگذاشتند
&&
رایت عین الیقین افراشتند
\\
رفت فکر و روشنایی یافتند
&&
نحر و بحر آشنایی یافتند
\\
مرگ کین جمله ازو در وحشتند
&&
می‌کنند این قوم بر وی ریش‌خند
\\
کس نیابد بر دل ایشان ظفر
&&
بر صدف آید ضرر نه بر گهر
\\
گرچه نحو و فقه را بگذاشتند
&&
لیک محو فقر را بر داشتند
\\
تا نقوش هشت جنت تافتست
&&
لوح دلشان را پذیرا یافتست
\\
برترند از عرش و کرسی و خلا
&&
ساکنان مقعد صدق خدا
\\
\end{longtable}
\end{center}
