\begin{center}
\section*{بخش ۹۷ - داستان پیر چنگی کی در عهد عمر رضی الله عنه از بهر خدا روز بی‌نوایی چنگ زد میان گورستان}
\label{sec:sh097}
\addcontentsline{toc}{section}{\nameref{sec:sh097}}
\begin{longtable}{l p{0.5cm} r}
آن شنیدستی که در عهد عمر
&&
بود چنگی مطربی با کر و فر
\\
بلبل از آواز او بی‌خود شدی
&&
یک طرب ز آواز خوبش صد شدی
\\
مجلس و مجمع دمش آراستی
&&
وز نوای او قیامت خاستی
\\
همچو اسرافیل کآوازش بفن
&&
مردگان را جان در آرد در بدن
\\
یار سایل بود اسرافیل را
&&
کز سماعش پر برستی فیل را
\\
سازد اسرافیل روزی ناله را
&&
جان دهد پوسیدهٔ صدساله را
\\
انبیا را در درون هم نغمه‌هاست
&&
طالبان را زان حیات بی‌بهاست
\\
نشنود آن نغمه‌ها را گوش حس
&&
کز ستمها گوش حس باشد نجس
\\
نشنود نغمهٔ پری را آدمی
&&
کو بود ز اسرار پریان اعجمی
\\
گر چه هم نغمهٔ پری زین عالمست
&&
نغمهٔ دل برتر از هر دو دمست
\\
که پری و آدمی زندانیند
&&
هر دو در زندان این نادانیند
\\
معشر الجن سورهٔ رحمان بخوان
&&
تستطیعوا تنفذوا را باز دان
\\
نغمه‌های اندرون اولیا
&&
اولا گوید که ای اجزای لا
\\
هین ز لای نفی سرها بر زنید
&&
این خیال و وهم یکسو افکنید
\\
ای همه پوسیده در کون و فساد
&&
جان باقیتان نرویید و نزاد
\\
گر بگویم شمه‌ای زان نغمه‌ها
&&
جانها سر بر زنند از دخمه‌ها
\\
گوش را نزدیک کن کان دور نیست
&&
لیک نقل آن به تو دستور نیست
\\
هین که اسرافیل وقتند اولیا
&&
مرده را زیشان حیاتست و نما
\\
جان هر یک مرده‌ای از گور تن
&&
بر جهد ز آوازشان اندر کفن
\\
گوید این آواز ز آواها جداست
&&
زنده کردن کار آواز خداست
\\
ما بمردیم و بکلی کاستیم
&&
بانگ حق آمد همه بر خاستیم
\\
بانگ حق اندر حجاب و بی حجاب
&&
آن دهد کو داد مریم را ز جیب
\\
ای فناتان نیست کرده زیر پوست
&&
باز گردید از عدم ز آواز دوست
\\
مطلق آن آواز خود از شه بود
&&
گرچه از حلقوم عبدالله بود
\\
گفته او را من زبان و چشم تو
&&
من حواس و من رضا و خشم تو
\\
رو که بی یسمع و بی یبصر توی
&&
سر توی چه جای صاحب‌سر توی
\\
چون شدی من کان لله از وله
&&
من ترا باشم که کان الله له
\\
گه توی گویم ترا گاهی منم
&&
هر چه گویم آفتاب روشنم
\\
هر کجا تابم ز مشکات دمی
&&
حل شد آنجا مشکلات عالمی
\\
ظلمتی را کآفتابش بر نداشت
&&
از دم ما گردد آن ظلمت چو چاشت
\\
آدمی را او بخویش اسما نمود
&&
دیگران را ز آدم اسما می‌گشود
\\
خواه ز آدم گیر نورش خواه ازو
&&
خواه از خم گیر می خواه از کدو
\\
کین کدو با خنب پیوستست سخت
&&
نی چو تو شاد آن کدوی نیکبخت
\\
گفت طوبی من رآنی مصطفی
&&
والذی یبصر لمن وجهی رای
\\
چون چراغی نور شمعی را کشید
&&
هر که دید آن را یقین آن شمع دید
\\
همچنین تا صد چراغ ار نقل شد
&&
دیدن آخر لقای اصل شد
\\
خواه از نور پسین بستان تو آن
&&
هیچ فرقی نیست خواه از شمع جان
\\
خواه بین نور از چراغ آخرین
&&
خواه بین نورش ز شمع غابرین
\\
\end{longtable}
\end{center}
