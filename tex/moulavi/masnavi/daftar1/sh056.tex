\begin{center}
\section*{بخش ۵۶ - ذکر دانش خرگوش و بیان فضیلت و منافع دانستن}
\label{sec:sh056}
\addcontentsline{toc}{section}{\nameref{sec:sh056}}
\begin{longtable}{l p{0.5cm} r}
این سخن پایان ندارد هوش‌دار
&&
هوش سوی قصهٔ خرگوش دار
\\
گوش خر بفروش و دیگر گوش خر
&&
کین سخن را در نیابد گوش خر
\\
رو تو روبه‌بازی خرگوش بین
&&
مکر و شیراندازی خرگوش بین
\\
خاتم ملک سلیمانست علم
&&
جمله عالم صورت و جانست علم
\\
آدمی را زین هنر بیچاره گشت
&&
خلق دریاها و خلق کوه و دشت
\\
زو پلنگ و شیر ترسان همچو موش
&&
زو نهنگ و بحر در صفرا و جوش
\\
زو پری و دیو ساحلها گرفت
&&
هر یکی در جای پنهان جا گرفت
\\
آدمی را دشمن پنهان بسیست
&&
آدمی با حذر عاقل کسیست
\\
خلق پنهان زشتشان و خوبشان
&&
می‌زند در دل بهر دم کوبشان
\\
بهر غسل ار در روی در جویبار
&&
بر تو آسیبی زند در آب خار
\\
گر چه پنهان خار در آبست پست
&&
چونک در تو می‌خلد دانی که هست
\\
خارخار وحیها و وسوسه
&&
از هزاران کس بود نه یک کسه
\\
باش تا حسهای تو مبدل شود
&&
تا ببینیشان و مشکل حل شود
\\
تا سخنهای کیان رد کرده‌ای
&&
تا کیان را سرور خود کرده‌ای
\\
\end{longtable}
\end{center}
