\begin{center}
\section*{بخش ۱۱۳ - در بیان آنک نادر افتد کی مریدی در مدعی مزور اعتقاد بصدق ببندد کی او کسی است و بدین اعتقاد به مقامی برسد کی شیخش در خواب ندیده باشد و آب و آتش او را گزند نکند و شیخش را گزند کند ولیکن بنادر نادر}
\label{sec:sh113}
\addcontentsline{toc}{section}{\nameref{sec:sh113}}
\begin{longtable}{l p{0.5cm} r}
لیک نادر طالب آید کز فروغ
&&
در حق او نافع آید آن دروغ
\\
او به قصد نیک خود جایی رسد
&&
گرچه جان پنداشت و آن آمد جسد
\\
چون تحری در دل شب قبله را
&&
قبله نی و آن نماز او روا
\\
مدعی را قحط جان اندر سرست
&&
لیک ما را قحط نان بر ظاهرست
\\
ما چرا چون مدعی پنهان کنیم
&&
بهر ناموس مزور جان کنیم
\\
\end{longtable}
\end{center}
