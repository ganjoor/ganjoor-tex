\begin{center}
\section*{بخش ۳۴ - مثل}
\label{sec:sh034}
\addcontentsline{toc}{section}{\nameref{sec:sh034}}
\begin{longtable}{l p{0.5cm} r}
آن‌چنان که کاروانی می‌رسید
&&
در دهی آمد دری را باز دید
\\
آن یکی گفت اندرین برد العجوز
&&
تا بیندازیم اینجا چند روز
\\
بانگ آمد نه بینداز از برون
&&
وانگهانی اندر آ تو اندرون
\\
هم برون افکن هر آنچ افکندنیست
&&
در میا با آن کای ن مجلس سنیست
\\
بد هلال استاددل جان‌روشنی
&&
سایس و بندهٔ امیریمؤمنی
\\
سایسی کردی در آخر آن غلام
&&
لیک سلطان سلاطین بنده نام
\\
آن امیر از حال بنده بی‌خبر
&&
که نبودش جز بلیسانه نظر
\\
آب و گل می‌دید و در وی گنج نه
&&
پنج و شش می‌دید و اصل پنج نه
\\
رنگ طین پیدا و نور دین نهان
&&
هر پیمبر این چنین بد در جهان
\\
آن مناره دید و در وی مرغ نی
&&
بر مناره شاه‌بازی پر فنی
\\
وان دوم می‌دید مرغی پرزنی
&&
لیک موی اندر دهان مرغ نی
\\
وانک او ینظر به نور الله بود
&&
هم ز مرغ و هم ز مو آگاه بود
\\
گفت آخر چشم سوی موی نه
&&
تا نبینی مو بنگشاید گره
\\
آن یکی گل دید نقشین دو وحل
&&
وآن دگر گل دید پر علم و عمل
\\
تن مناره علم و طاعت هم‌چو مرغ
&&
خواه سیصد مرغ‌گیر و یا دو مرغ
\\
مرد اوسط مرغ‌بینست او و بس
&&
غیر مرغی می‌نبیند پیش و پس
\\
موی آن نور نیست پنهان آن مرغ
&&
هیچ عاریت نباشد کار او
\\
علم او از جان او جوشد مدام
&&
پیش او نه مستعار آمد نه وام
\\
\end{longtable}
\end{center}
