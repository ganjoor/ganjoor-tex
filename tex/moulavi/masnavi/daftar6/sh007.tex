\begin{center}
\section*{بخش ۷ - در بیان آنک این غرور تنها آن هندو را نبود بلک هر آدمیی به چنین غرور مبتلاست در هر مرحله‌ای الا من عصم الله}
\label{sec:sh007}
\addcontentsline{toc}{section}{\nameref{sec:sh007}}
\begin{longtable}{l p{0.5cm} r}
چون بپیوستی بدان ای زینهار
&&
چند نالی در ندامت زار زار
\\
نام میری و وزیری و شهی
&&
در نهانش مرگ و درد و جان‌دهی
\\
بنده باش و بر زمین رو چون سمند
&&
چون جنازه نه که بر گردن برند
\\
جمله را حمال خود خواهد کفور
&&
چون سوار مرده آرندش به گور
\\
بر جنازه هر که را بینی به خواب
&&
فارس منصب شود عالی رکاب
\\
زانک آن تابوت بر خلقست بار
&&
بار بر خلقان فکندند این کبار
\\
بار خود بر کس منه بر خویش نه
&&
سروری را کم طلب درویش به
\\
مرکب اعناق مردم را مپا
&&
تا نیاید نقرست اندر دو پا
\\
مرکبی را که آخرش تو ده دهی
&&
که به شهری مانی و ویران‌دهی
\\
ده دهش اکنون که چون شهرت نمود
&&
تا نباید رخت در ویران گشود
\\
ده دهش اکنون که صد بستانت هست
&&
تا نگردی عاجز و ویران‌پرست
\\
گفت پیغامبر که جنت از اله
&&
گر همی‌خواهی ز کس چیزی مخواه
\\
چون نخواهی من کفیلم مر ترا
&&
جنت الماوی و دیدار خدا
\\
آن صحابی زین کفالت شد عیار
&&
تا یکی روزی که گشته بد سوار
\\
تازیانه از کفش افتاد راست
&&
خود فرو آمد ز کس آنرا نخواست
\\
آنک از دادش نیاید هیچ بد
&&
داند و بی‌خواهشی خود می‌دهد
\\
ور به امر حق بخواهی آن رواست
&&
آنچنان خواهش طریق انبیاست
\\
بد نماند چون اشارت کرد دوست
&&
کفر ایمان شد چون کفر از بهر اوست
\\
هر بدی که امر او پیش آورد
&&
آن ز نیکوهای عالم بگذرد
\\
زان صدف گر خسته گردد نیز پوست
&&
ده مده که صد هزاران در دروست
\\
این سخن پایان ندارد بازگرد
&&
سوی شاه و هم‌مزاج بازگرد
\\
باز رو در کان چو زر ده‌دهی
&&
تا رهد دستان تو از ده‌دهی
\\
صورتی را چون بدل ره می‌دهند
&&
از ندامت آخرش ده می‌دهند
\\
توبه می‌آرند هم پروانه‌وار
&&
باز نسیان می‌کشدشان سوی کار
\\
هم‌چو پروانه ز دور آن نار را
&&
نور دید و بست آن سو بار را
\\
چون بیامد سوخت پرش را گریخت
&&
باز چون طفلان فتاد و ملح ریخت
\\
بار دیگر بر گمان طمع سود
&&
خویش زد بر آتش آن شمع زود
\\
بار دیگر سوخت هم واپس بجست
&&
باز کردش حرص دل ناسی و مست
\\
آن زمان کز سوختن وا می‌جهد
&&
هم‌چو هندو شمع را ده می‌دهد
\\
که ای رخت تابان چون ماه شب‌فروز
&&
وی به صحبت کاذب و مغرورسوز
\\
باز از یادش رود توبه و انین
&&
کاوهن الرحمن کید الکاذبین
\\
\end{longtable}
\end{center}
