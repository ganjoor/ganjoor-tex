\begin{center}
\section*{بخش ۱۶ - حواله کردن مرغ گرفتاری خود را در دام به فعل و مکر و زرق  زاهد و جواب زاهد مرغ را}
\label{sec:sh016}
\addcontentsline{toc}{section}{\nameref{sec:sh016}}
\begin{longtable}{l p{0.5cm} r}
گفت آن مرغ این سزای او بود
&&
که فسون زاهدان را بشنود
\\
گفت زاهد نه سزای آن نشاف
&&
کو خورد مال یتیمان از گزاف
\\
بعد از آن نوحه‌گری آغاز کرد
&&
که فخ و صیاد لرزان شد ز درد
\\
کز تناقضهای دل پشتم شکست
&&
بر سرم جانا بیا می‌مال دست
\\
زیر دست تو سرم را راحتیست
&&
دست تو در شکربخشی آیتیست
\\
سایهٔ خود از سر من برمدار
&&
بی‌قرارم بی‌قرارم بی‌قرار
\\
خوابها بیزار شد از چشم من
&&
در غمت ای رشک سرو و یاسمن
\\
گر نیم لایق چه باشد گر دمی
&&
ناسزایی را بپرسی در غمی
\\
مر عدم را خود چه استحقاق بود
&&
که برو لطفت چنین درها گشود
\\
خاک گرگین را کرم آسیب کرد
&&
ده گهر از نور حس در جیب کرد
\\
پنج حس ظاهر و پنج نهان
&&
که بشر شد نطفهٔ مرده از آن
\\
توبه بی توفیقت ای نور بلند
&&
چیست جز بر ریش توبه ریش‌خند
\\
سبلتان توبه یک یک بر کنی
&&
توبه سایه‌ست و تو ماه روشنی
\\
ای ز تو ویران دکان و منزلم
&&
چون ننالم چون بیفشاری دلم
\\
چون گریزم زانک بی تو زنده نیست
&&
بی خداوندیت بود بنده نیست
\\
جان من بستان تو ای جان را اصول
&&
زانک بی‌تو گشته‌ام از جان ملول
\\
عاشقم من بر فن دیوانگی
&&
سیرم از فرهنگی و فرزانگی
\\
چون بدرد شرم گویم راز فاش
&&
چند ازین صبر و زحیر و ارتعاش
\\
در حیا پنهان شدم هم‌چون سجاف
&&
ناگهان بجهم ازین زیر لحاف
\\
ای رفیقان راهها را بست یار
&&
آهوی لنگیم و او شیر شکار
\\
جز که تسلیم و رضا کو چاره‌ای
&&
در کف شیر نری خون‌خواره‌ای
\\
او ندارد خواب و خور چون آفتاب
&&
روحها را می‌کند بی‌خورد و خواب
\\
که بیا من باش یا هم‌خوی من
&&
تا ببینی در تجلی روی من
\\
ور ندیدی چون چنین شیدا شدی
&&
خاک بودی طالب احیا شدی
\\
گر ز بی‌سویت ندادست او علف
&&
چشم جانت چون بماندست آن طرف
\\
گربه بر سوراخ زان شد معتکف
&&
که از آن سوراخ او شد معتلف
\\
گربهٔ دیگر همی‌گردد به بام
&&
کز شکار مرغ یابید او طعام
\\
آن یکی را قبله شد جولاهگی
&&
وآن یکی حارس برای جامگی
\\
وان یکی بی‌کار و رو در لامکان
&&
که از آن سو دادیش تو قوت جان
\\
کار او دارد که حق را شد مرید
&&
بهر کار او ز هر کاری برید
\\
دیگران چون کودکان این روز چند
&&
تا شب ترحال بازی می‌کنند
\\
خوابناکی کو ز یقظت می‌جهد
&&
دایهٔ وسواس عشوه‌ش می‌دهد
\\
رو بخسپ ای جان که نگذاریم ما
&&
که کسی از خواب بجهاند ترا
\\
هم تو خود را بر کنی از بیخ خواب
&&
هم‌چو تشنه که شنود او بانک آب
\\
بانگ آبم من به گوش تشنگان
&&
هم‌چو باران می‌رسم از آسمان
\\
بر جه ای عاشق برآور اضطراب
&&
بانگ آب و تشنه و آنگاه خواب
\\
\end{longtable}
\end{center}
