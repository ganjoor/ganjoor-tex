\begin{center}
\section*{بخش ۷۰ - حکایت مرید شیخ حسن خرقانی قدس الله سره}
\label{sec:sh070}
\addcontentsline{toc}{section}{\nameref{sec:sh070}}
\begin{longtable}{l p{0.5cm} r}
رفت درویشی ز شهر طالقان
&&
بهر صیت بوالحسین خارقان
\\
کوهها ببرید و وادی دراز
&&
بهر دید شیخ با صدق و نیاز
\\
آنچ در ره دید از رنج و ستم
&&
گرچه در خوردست کوته می‌کنم
\\
چون به مقصد آمد از ره آن جوان
&&
خانهٔ آن شاه را جست او نشان
\\
چون به صد حرمت بزد حلقهٔ درش
&&
زن برون کرد از در خانه سرش
\\
که چه می‌خواهی بگو ای ذوالکرم
&&
ژگفت بر قصد زیارت آمدم
\\
خنده‌ای زد زن که خه‌خه ریش بین
&&
این سفرگیری و این تشویش بین
\\
خود ترا کاری نبود آن جایگاه
&&
که به بیهوده کنی این عزم راه
\\
اشتهای گول‌گردی آمدت
&&
یا ملولی وطن غالب شدت
\\
یا مگر دیوت دو شاخه بر نهاد
&&
بر تو وسواس سفر را در گشاد
\\
گفت نافرجام و فحش و دمدمه
&&
من نتوانم باز گفتن آن همه
\\
از مثل وز ریش‌خند بی‌حساب
&&
آن مرید افتاد از غم در نشیب
\\
\end{longtable}
\end{center}
