\begin{center}
\section*{بخش ۳۷ - در بیان آنک مصطفی علیه‌السلام شنید کی عیسی علیه‌السلام بر روی آب رفت  فرمود لو ازداد یقینه لمشی علی الهواء}
\label{sec:sh037}
\addcontentsline{toc}{section}{\nameref{sec:sh037}}
\begin{longtable}{l p{0.5cm} r}
هم‌چو عیسی بر سرش گیرد فرات
&&
که ایمنی از غرقه در آب حیات
\\
گوید احمد گر یقینش افزون بدی
&&
خود هوایش مرکب و مامون بدی
\\
هم‌چو من که بر هوا راکب شدم
&&
در شب معراج مستصحب شدم
\\
گفت چون باشد سگی کوری پلید
&&
جست او از خواب خود را شیر دید
\\
نه چنان شیری که کس تیرش زند
&&
بل ز بیمش تیغ و پیکان بشکند
\\
کور بر اشکم رونده هم‌چو مار
&&
چشمها بگشاد در باغ و بهار
\\
چون بود آن چون که از چونی رهید
&&
در حیاتستان بی‌چونی رسید
\\
گشت چونی‌بخش اندر لامکان
&&
گرد خوانش جمله چونها چون سگان
\\
او ز بی‌چونی دهدشان استخوان
&&
در جنابت تن زن این سوره مخوان
\\
تا ز چونی غسل ناری تو تمام
&&
تو برین مصحف منه کف ای غلام
\\
گر پلیدم ور نظیفم ای شهان
&&
این نخوانم پس چه خوانم در جهان
\\
تو مرا گویی که از بهر ثواب
&&
غسل ناکرده مرو در حوض آب
\\
از برون حوض غیر خاک نیست
&&
هر که او در حوض ناید پاک نیست
\\
گر نباشد آبها را این کرم
&&
کو پذیرد مر خبث را دم به دم
\\
وای بر مشتاق و بر اومید او
&&
حسرتا بر حسرت جاوید او
\\
آب دارد صد کرم صد احتشام
&&
که پلیدان را پذیرد والسلام
\\
ای ضیاء الحق حسام‌الدین که نور
&&
پاسبان تست از شر الطیور
\\
پاسبان تست نور و ارتقاش
&&
ای تو خورشید مستر از خفاش
\\
چیست پرده پیش روی آفتاب
&&
جز فزونی شعشعه و تیزی تاب
\\
پردهٔ خورشید هم نور ربست
&&
بی‌نصیب از وی خفاشست و شبست
\\
هر دو چون در بعد و پرده مانده‌اند
&&
یا سیه‌رو یا فسرده مانده‌اند
\\
چون نبشتی بعضی از قصهٔ هلال
&&
داستان بدر آر اندر مقال
\\
آن هلال و بدر دارند اتحاد
&&
از دوی دورند و از نقص و فساد
\\
آن هلال از نقص در باطن بریست
&&
آن به ظاهر نقص تدریج آوریست
\\
درس گوید شب به شب تدریج را
&&
در تانی بر دهد تفریج را
\\
در تانی گوید ای عجول خام
&&
پایه‌پایه بر توان رفتن به بام
\\
دیگ را تدریج و استادانه جوش
&&
کار ناید قلیهٔ دیوانه جوش
\\
حق نه قادر بود بر خلق فلک
&&
در یکی لحظه به کن بی‌هیچ شک
\\
پس چرا شش روز آن را درکشید
&&
کل یوم الف عام ای مستفید
\\
خلقت طفل از چه اندر نه مه‌است
&&
زانک تدریج از شعار آن شه‌است
\\
خلقت آدم چرا چل صبح بود
&&
اندر آن گل اندک‌اندک می‌فزود
\\
نه چو تو ای خام که اکنون تاختی
&&
طفلی و خود را تو شیخی ساختی
\\
بر دویدی چون کدو فوق همه
&&
کو ترا پای جهاد و ملحمه
\\
تکیه کردی بر درختان و جدار
&&
بر شدی ای اقرعک هم قرع‌وار
\\
اول ار شد مرکبت سرو سهی
&&
لیک آخر خشک و بی‌مغزی تهی
\\
رنگ سبزت زرد شد ای قرع زود
&&
زانک از گلگونه بود اصلی نبود
\\
\end{longtable}
\end{center}
