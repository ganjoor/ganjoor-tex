\begin{center}
\section*{بخش ۱۲۳ - مثل}
\label{sec:sh123}
\addcontentsline{toc}{section}{\nameref{sec:sh123}}
\begin{longtable}{l p{0.5cm} r}
گفت با درویش روزی یک خسی
&&
که ترا این‌جا نمی‌داند کسی
\\
گفت او گر می‌نداند عامیم
&&
خویش را من نیک می‌دانم کیم
\\
وای اگر بر عکس بودی درد و ریش
&&
او بدی بینای من من کور خویش
\\
احمقم گیر احمقم من نیک‌بخت
&&
بخت بهتر از لجاج و روی سخت
\\
این سخن بر وفق ظنت می‌جهد
&&
ورنه بختم داد عقلم هم دهد
\\
\end{longtable}
\end{center}
