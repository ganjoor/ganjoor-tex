\begin{center}
\section*{بخش ۱۲ - حکایت آن صیادی کی خویشتن در گیاه پیچیده بود و دستهٔ گل و لاله را کله‌وار به سر فرو کشیده تا مرغان او را گیاه پندارند و آن مرغ زیرک بوی برد اندکی کی این آدمیست کی برین شکل گیاه ندیدم اما هم تمام بوی نبرد به افسون او مغرور شد زیرا در ادراک اول قاطعی نداشت در ادراک مکر دوم قاطعی داشت و هو الحرص و الطمع لا سیما عند فرط الحاجة و الفقر قال النبی صلی الله علیه و سلم کاد الفقر ان یکون کفرا}
\label{sec:sh012}
\addcontentsline{toc}{section}{\nameref{sec:sh012}}
\begin{longtable}{l p{0.5cm} r}
رفت مرغی در میان مرغزار
&&
بود آنجا دام از بهر شکار
\\
دانهٔ چندی نهاده بر زمین
&&
وآن صیاد آنجا نشسته در کمین
\\
خویشتن پیچیده در برگ و گیاه
&&
تا در افتد صید بیچاره ز راه
\\
مرغک آمد سوی او از ناشناخت
&&
پس طوافی کرد و پیش مرد تاخت
\\
گفت او را کیستی تو سبزپوش
&&
در بیابان در میان این وحوش
\\
گفت مرد زاهدم من منقطع
&&
با گیاهی گشتم اینجا مقتنع
\\
زهد و تقوی را گزیدم دین و کیش
&&
زانک می‌دیدم اجل را پیش خویش
\\
مرگ همسایه مرا واعظ شده
&&
کسب و دکان مرا برهم زده
\\
چون به آخر فرد خواهم ماندن
&&
خو نباید کرد با هر مرد و زن
\\
رو بخواهم کرد آخر در لحد
&&
آن به آید که کنم خو با احد
\\
چو زنخ را بست خواهند ای صنم
&&
آن به آید که زنخ کمتر زنم
\\
ای بزربفت و کمر آموخته
&&
آخرستت جامهٔ نادوخته
\\
رو به خاک آریم کز وی رسته‌ایم
&&
دل چرا در بی‌وفایان بسته‌ایم
\\
جد و خویشانمان قدیمی چار طبع
&&
ما به خویشی عاریت بستیم طمع
\\
سالها هم‌صحبتی و هم‌دمی
&&
با عناصر داشت جسم آدمی
\\
روح او خود از نفوس و از عقول
&&
روح اصول خویش را کرده نکول
\\
از عقول و از نفوس پر صفا
&&
نامه می‌آید به جان کای بی‌وفا
\\
یارکان پنج روزه یافتی
&&
رو ز یاران کهن بر تافتی
\\
کودکان گرچه که در بازی خوشند
&&
شب کشانشان سوی خانه می‌کشند
\\
شد برهنه وقت بازی طفل خرد
&&
دزد از ناگه قبا و کفش برد
\\
آن چنان گرم او به بازی در فتاد
&&
کان کلاه و پیرهن رفتش ز یاد
\\
شد شب و بازی او شد بی‌مدد
&&
رو ندارد کو سوی خانه رود
\\
نی شنیدی انما الدنیا لعب
&&
باد دادی رخت و گشتی مرتعب
\\
پیش از آنک شب شود جامه بجو
&&
روز را ضایع مکن در گفت و گو
\\
من به صحرا خلوتی بگزیده‌ام
&&
خلق را من دزد جامه دیده‌ام
\\
نیم عمر از آرزوی دلستان
&&
نیم عمر از غصه‌های دشمنان
\\
جبه را برد آن کله را این ببرد
&&
غرق بازی گشته ما چون طفل خرد
\\
نک شبانگاه اجل نزدیک شد
&&
خل هذا اللعب به سبک لاتعد
\\
هین سوار توبه شود در دزد رس
&&
جامه‌ها از دزد بستان باز پس
\\
مرکب توبه عجاب مرکبست
&&
بر فلک تازد به یک لحظه ز پست
\\
لیک مرکب را نگه می‌دار از آن
&&
کو بدزدید آن قبایت را نهان
\\
تا ندزدد مرکبت را نیز هم
&&
پاس دار این مرکبت را دم به دم
\\
\end{longtable}
\end{center}
