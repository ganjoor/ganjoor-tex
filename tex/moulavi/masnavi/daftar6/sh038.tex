\begin{center}
\section*{بخش ۳۸ - داستان آن عجوزه کی روی زشت خویشتن را جندره و گلگونه می‌ساخت و ساخته نمی‌شد و پذیرا نمی‌آمد}
\label{sec:sh038}
\addcontentsline{toc}{section}{\nameref{sec:sh038}}
\begin{longtable}{l p{0.5cm} r}
بود کمپیری نودساله کلان
&&
پر تشنج روی و رنگش زعفران
\\
چون سر سفره رخ او توی توی
&&
لیک در وی بود مانده عشق شوی
\\
ریخت دندانهاش و مو چون شیر شد
&&
قد کمان و هر حسش تغییر شد
\\
عشق شوی و شهوت و حرصش تمام
&&
عشق صید و پاره‌پاره گشته دام
\\
مرغ بی‌هنگام و راه بی‌رهی
&&
آتشی پر در بن دیگ تهی
\\
عاشق میدان و اسپ و پای نی
&&
عاشق زمر و لب و سرنای نی
\\
حرص در پیری جهودان را مباد
&&
ای شقیی که خداش این حرص داد
\\
ریخت دندانهای سگ چون پیر شد
&&
ترک مردم کرد و سرگین‌گیر شد
\\
این سگان شصت ساله را نگر
&&
هر دمی دندان سگشان تیزتر
\\
پیر سگ را ریخت پشم از پوستین
&&
این سگان پیر اطلس‌پوش بین
\\
عشقشان و حرصشان در فرج و زر
&&
دم به دم چون نسل سگ بین بیشتر
\\
این چنین عمری که مایهٔ دوزخ است
&&
مر قصابان غضب را مسلخ است
\\
چون بگویندش که عمر تو دراز
&&
می‌شود دلخوش دهانش از خنده باز
\\
این چنین نفرین دعا پندارد او
&&
چشم نگشاید سری بر نارد او
\\
گر بدیدی یک سر موی از معاد
&&
اوش گفتی این چنین عمر تو باد
\\
\end{longtable}
\end{center}
