\begin{center}
\section*{بخش ۵۴ - قال النبی علیه السلام ان الله تعالی یلقن الحکمة علی لسان الواعظین بقدر همم المستمعین}
\label{sec:sh054}
\addcontentsline{toc}{section}{\nameref{sec:sh054}}
\begin{longtable}{l p{0.5cm} r}
جذب سمعست ار کسی را خوش لبیست
&&
گرمی و جد معلم از صبیست
\\
چنگیی را کو نوازد بیست و چار
&&
چون نیابد گوش گردد چنگ بار
\\
نه حراره یادش آید نه غزل
&&
نه ده انگشتش بجنبد در عمل
\\
گر نبودی گوشهای غیب‌گیر
&&
وحی ناوردی ز گردون یک بشیر
\\
ور نبودی دیده‌های صنع‌بین
&&
نه فلک گشتی نه خندیدی زمین
\\
آن دم لولاک این باشد که کار
&&
از برای چشم تیزست و نظار
\\
عامه را از عشق هم‌خوابه و طبق
&&
کی بود پروای عشق صنع حق
\\
آب تتماجی نریزی در تغار
&&
تا سگی چندی نباشد طعمه‌خوار
\\
رو سگ کهف خداوندیش باش
&&
تا رهاند زین تغارت اصطفاش
\\
چونک دزدیهای بی‌رحمانه گفت
&&
کی کنند آن درزیان اندر نهفت
\\
اندر آن هنگامه ترکی از خطا
&&
سخت طیره شد ز کشف آن غطا
\\
شب چو روز رستخیز آن رازها
&&
کشف می‌کرد از پی اهل نهی
\\
هر کجا آیی تو در جنگی فراز
&&
بینی آنجا دو عدو در کشف راز
\\
آن زمان را محشر مذکور دان
&&
وان گلوی رازگو را صور دان
\\
که خدا اسباب خشمی ساختست
&&
وآن فضایح را بکوی انداختست
\\
بس که غدر درزیان را ذکر کرد
&&
حیف آمد ترک را و خشم و درد
\\
گفت ای قصاص در شهر شما
&&
کیست استاتر درین مکر و دغا
\\
\end{longtable}
\end{center}
