\begin{center}
\section*{بخش ۷۸ - انابت آن طالب گنج به حق تعالی بعد از طلب بسیار و عجز و اضطرار کی ای ولی الاظهار تو کن این پنهان را آشکار}
\label{sec:sh078}
\addcontentsline{toc}{section}{\nameref{sec:sh078}}
\begin{longtable}{l p{0.5cm} r}
گفت آن درویش ای دانای راز
&&
از پی این گنج کردم یاوه‌تاز
\\
دیو حرص و آز و مستعجل تگی
&&
نی تانی جست و نی آهستگی
\\
من ز دیگی لقمه‌ای نندوختم
&&
کف سیه کردم دهان را سوختم
\\
خود نگفتم چون درین ناموقنم
&&
زان گره‌زن این گره را حل کنم
\\
قول حق را هم ز حق تفسیر جو
&&
هین مگو ژاژ از گمان ای سخت‌رو
\\
آن گره کو زد همو بگشایدش
&&
مهره کو انداخت او بربایدش
\\
گرچه آسانت نمود آن سان سخن
&&
کی بود آسان رموز من لدن
\\
گفت یا رب توبه کردم زین شتاب
&&
چون تو در بستی تو کن هم فتح باب
\\
بر سر خرقه شدن بار دگر
&&
در دعا کردن بدم هم بی‌هنر
\\
کو هنر کو من کجا دل مستوی
&&
این همه عکس توست و خود توی
\\
هر شبی تدبیر و فرهنگم به خواب
&&
هم‌چو کشتی غرقه می‌گردد ز آب
\\
خود نه من می‌مانم و نه آن هنر
&&
تن چو مرداری فتاده بی‌خبر
\\
تا سحر جمله شب آن شاه علی
&&
خود همی‌گوید الستی و بلی
\\
کو بلی‌گو جمله را سیلاب برد
&&
یا نهنگی خورد کل را کرد و مرد
\\
صبح‌دم چون تیغ گوهردار خود
&&
از نیام ظلمت شب بر کند
\\
آفتاب شرق شب را طی کند
&&
از نهنگ آن خورده‌ها را قی کند
\\
رسته چون یونس ز معدهٔ آن نهنگ
&&
منتشر گردیم اندر بو و رنگ
\\
خلق چون یونس مسبح آمدند
&&
کاندر آن ظلمات پر راحت شدند
\\
هر یکی گوید به هنگام سحر
&&
چون ز بطن حوت شب آید به در
\\
کای کریمی که در آن لیل وحش
&&
گنج رحمت بنهی و چندین چشش
\\
چشم تیز و گوش تازه تن سبک
&&
از شب هم‌چون نهنگ ذوالحبک
\\
از مقامات وحش‌رو زین سپس
&&
هیچ نگریزیم ما با چون تو کس
\\
موسی آن را نار دید و نور بود
&&
زنگیی دیدیم شب را حور بود
\\
بعد ازین ما دیده خواهیم از تو بس
&&
تا نپوشد بحر را خاشاک و خس
\\
ساحران را چشم چون رست از عمی
&&
کف‌زنان بودند بی‌این دست و پا
\\
چشم‌بند خلق جز اسباب نیست
&&
هر که لرزد بر سبب ز اصحاب نیست
\\
لیک حق اصحابنا اصحاب را
&&
در گشاد و برد تا صدر سرا
\\
با کفش نامستحق و مستحق
&&
معتقان رحمت‌اند از بند رق
\\
در عدم ما مستحقان کی بدیم
&&
که برین جان و برین دانش زدیم
\\
ای بکرده یار هر اغیار را
&&
وی بداده خلعت گل خار را
\\
خاک ما را ثانیا پالیز کن
&&
هیچ نی را بار دیگر چیز کن
\\
این دعا تو امر کردی ز ابتدا
&&
ورنه خاکی را چه زهرهٔ این بدی
\\
چون دعامان امر کردی ای عجاب
&&
این دعای خویش را کن مستجاب
\\
شب شکسته کشتی فهم و حواس
&&
نه امیدی مانده نه خوف و نه یاس
\\
برده در دریای رحمت ایزدم
&&
تا ز چه فن پر کند بفرستدم
\\
آن یکی را کرده پر نور جلال
&&
وآن دگر را کرده پر وهم و خیال
\\
گر بخویشم هیچ رای و فن بدی
&&
رای و تدبیرم به حکم من بدی
\\
شب نرفتی هوش بی‌فرمان من
&&
زیر دام من بدی مرغان من
\\
بودمی آگه ز منزلهای جان
&&
وقت خواب و بیهشی و امتحان
\\
چون کفم زین حل و عقد او تهیست
&&
ای عجب این معجبی من ز کیست
\\
دیده را نادیده خود انگاشتم
&&
باز زنبیل دعا برداشتم
\\
چون الف چیزی ندارم ای کریم
&&
جز دلی دلتنگ‌تر از چشم میم
\\
این الف وین میم ام بود ماست
&&
میم ام تنگست الف زو نر گداست
\\
آن الف چیزی ندارد غافلیست
&&
میم دلتنگ آن زمان عاقلیست
\\
در زمان بیهشی خود هیچ من
&&
در زمان هوش اندر پیچ من
\\
هیچ دیگر بر چنین هیچی منه
&&
نام دولت بر چنین پیچی منه
\\
خود ندارم هیچ به سازد مرا
&&
که ز وهم دارم است این صد عنا
\\
در ندارم هم تو داراییم کن
&&
رنج دیدم راحت‌افزاییم کن
\\
هم در آب دیده عریان بیستم
&&
بر در تو چونک دیده نیستم
\\
آب دیدهٔ بندهٔ بی‌دیده را
&&
سبزه‌ای بخش و نباتی زین چرا
\\
ور نمانم آب آبم ده ز عین
&&
هم‌چو عینین نبی هطالتین
\\
او چو آب دیده جست از جود حق
&&
با چنان اقبال و اجلال و سبق
\\
چون نباشم ز اشک خون باریک‌ریس
&&
من تهی‌دست قصور کاسه‌لیس
\\
چون چنان چشم اشک را مفتون بود
&&
اشک من باید که صد جیحون بود
\\
قطره‌ای زان زین دو صد جیحون به است
&&
که بدان یک قطره انس و جن برست
\\
چونک باران جست آن روضهٔ بهشت
&&
چون نجوید آب شوره‌خاک زشت
\\
ای اخی دست از دعا کردن مدار
&&
با اجابت یا رد اویت چه کار
\\
نان که سد و مانع این آب بود
&&
دست از آن نان می‌بباید شست زود
\\
خویش را موزون و چست و سخته کن
&&
ز آب دیده نان خود را پخته کن
\\
\end{longtable}
\end{center}
