\begin{center}
\section*{بخش ۵۰ - سال کردن آن صوفی قاضی را}
\label{sec:sh050}
\addcontentsline{toc}{section}{\nameref{sec:sh050}}
\begin{longtable}{l p{0.5cm} r}
گفت صوفی چون ز یک کانست زر
&&
این چرا نفعست و آن دیگر ضرر
\\
چونک جمله از یکی دست آمدست
&&
این چرا هوشیار و آن مست آمدست
\\
چون ز یک دریاست این جوها روان
&&
این چرا نوش است و آن زهر دهان
\\
چون همه انوار از شمس بقاست
&&
صبح صادق صبح کاذب از چه خاست
\\
چون ز یک سرمه‌ست ناظر را کحل
&&
از چه آمد راست‌بینی و حول
\\
چونک دار الضرب را سلطان خداست
&&
نقد را چون ضرب خوب و نارواست
\\
چون خدا فرمود ره را راه من
&&
این خفیر از چیست و آن یک راه‌زن
\\
از یک اشکم چون رسد حر و سفیه
&&
چون یقین شد الولد سر ابیه
\\
وحدتی که دید با چندین هزار
&&
صد هزاران جنبش از عین قرار
\\
\end{longtable}
\end{center}
