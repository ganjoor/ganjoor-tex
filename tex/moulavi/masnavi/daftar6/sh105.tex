\begin{center}
\section*{بخش ۱۰۵ - روان شدن شه‌زادگان در ممالک پدر بعد از وداع کردن ایشان شاه را و اعادت کردن شاه وقت وداع وصیت را الی آخره}
\label{sec:sh105}
\addcontentsline{toc}{section}{\nameref{sec:sh105}}
\begin{longtable}{l p{0.5cm} r}
عزم ره کردند آن هر سه پسر
&&
سوی املاک پدر رسم سفر
\\
در طواف شهرها و قلعه‌هاش
&&
از پی تدبیر دیوان و معاش
\\
دست‌بوس شاه کردند و وداع
&&
پس بدیشان گفت آن شاه مطاع
\\
هر کجاتان دل کشد عازم شوید
&&
فی امان الله دست افشان روید
\\
غیر آن یک قلعه نامش هش‌ربا
&&
تنگ آرد بر کله‌داران قبا
\\
الله الله زان دز ذات الصور
&&
دور باشید و بترسید از خطر
\\
رو و پشت برجهاش و سقف و پست
&&
جمله تمثال و نگار و صورتست
\\
هم‌چو آن حجرهٔ زلیخا پر صور
&&
تا کند یوسف بناکامش نظر
\\
چونک یوسف سوی او می‌ننگرید
&&
خانه را پر نقش خود کرد آن مکید
\\
تا به هر سو که نگرد آن خوش‌عذار
&&
روی او را بیند او بی‌اختیار
\\
بهر دیده‌روشنان یزدان فرد
&&
شش جهت را مظهر آیات کرد
\\
تا بهر حیوان و نامی که نگزند
&&
از ریاض حسن ربانی چرند
\\
بهر این فرمود با آن اسپه او
&&
حیث ولیتم فثم وجهه
\\
از قدح‌گر در عطش آبی خورید
&&
در درون آب حق را ناظرید
\\
آنک عاشق نیست او در آب در
&&
صورت صورت خود بیند ای صاحب‌بصر
\\
صورت عاشق چو فانی شد درو
&&
پس در آب اکنون کرا بیند بگو
\\
حسن حق بینند اندر روی حور
&&
هم‌چو مه در آب از صنع غیور
\\
غیرتش بر عاشقی و صادقیست
&&
غیرتش بر دیو و بر استور نیست
\\
دیو اگر عاشق شود هم گوی برد
&&
جبرئیلی گشت و آن دیوی بمرد
\\
اسلم الشیطان آنجا شد پدید
&&
که یزیدی شد ز فضلش بایزید
\\
این سخن پایان ندارد ای گروه
&&
هین نگه دارید زان قلعه وجوه
\\
هین مبادا که هوستان ره زند
&&
که فتید اندر شقاوت تا ابد
\\
از خطر پرهیز آمد مفترض
&&
بشنوید از من حدیث بی‌غرض
\\
در فرج جویی خرد سر تیز به
&&
از کمین‌گاه بلا پرهیز به
\\
گر نمی‌گفت این سخن را آن پدر
&&
ور نمی‌فرمود زان قلعه حذر
\\
خود بدان قلعه نمی‌شد خیلشان
&&
خود نمی‌افتاد آن سو میلشان
\\
کان نبد معروف بس مهجور بود
&&
از قلاع و از مناهج دور بود
\\
چون بکرد آن منع دلشان زان مقال
&&
در هوس افتاد و در کوی خیال
\\
رغبتی زین منع در دلشان برست
&&
که بباید سر آن را باز جست
\\
کیست کز ممنوع گردد ممتنع
&&
چونک الانسان حریص ما منع
\\
نهی بر اهل تقی تبغیض شد
&&
نهی بر اهل هوا تحریض شد
\\
پس ازین یغوی به قوما کثیر
&&
هم ازین یهدی به قلبا خبیر
\\
کی رمد از نی حمام آشنا
&&
بل رمد زان نی حمامات هوا
\\
پس بگفتندش که خدمتها کنیم
&&
بر سمعنا و اطعناها تنیم
\\
رو نگردانیم از فرمان تو
&&
کفر باشد غفلت از احسان تو
\\
لیک استثنا و تسبیح خدا
&&
ز اعتماد خود بد از ایشان جدا
\\
ذکر استثنا و حزم ملتوی
&&
گفته شد در ابتدای مثنوی
\\
صد کتاب ار هست جز یک باب نیست
&&
صد جهت را قصد جز محراب نیست
\\
این طرق را مخلصی یک خانه است
&&
این هزاران سنبل از یک دانه است
\\
گونه‌گونه خوردنیها صد هزار
&&
جمله یک چیزست اندر اعتبار
\\
از یکی چون سیر گشتی تو تمام
&&
سرد شد اندر دلت پنجه طعام
\\
در مجاعت پس تو احول دیده‌ای
&&
که یکی را صد هزاران دیده‌ای
\\
گفته بودیم از سقام آن کنیز
&&
وز طبیبان و قصور فهم نیز
\\
کان طبیبان هم‌چو اسپ بی‌عذار
&&
غافل و بی‌بهره بودند از سوار
\\
کامشان پر زخم از قرع لگام
&&
سمشان مجروح از تحویل گام
\\
ناشده واقف که نک بر پشت ما
&&
رایض و چستیست استادی‌نما
\\
نیست سرگردانی ما زین لگام
&&
جز ز تصریف سوار دوست‌کام
\\
ما پی گل سوی بستان‌ها شده
&&
گل نموده آن و آن خاری بده
\\
هیچ‌شان این نی که گویند از خرد
&&
بر گلوی ما کی می‌کوبد لگد
\\
آن طبیبان آن‌چنان بندهٔ سبب
&&
گشته‌اند از مکر یزدان محتجب
\\
گر ببندی در صطبلی گاو نر
&&
باز یابی در مقام گاو خر
\\
از خری باشد تغافل خفته‌وار
&&
که نجویی تا کیست آن خفیه کار
\\
خود نگفته این مبدل تا کیست
&&
نیست پیدا او مگر افلاکیست
\\
تیر سوی راست پرانیده‌ای
&&
سوی چپ رفتست تیرت دیده‌ای
\\
سوی آهویی به صیدی تاختی
&&
خویش را تو صید خوکی ساختی
\\
در پی سودی دویده بهر کبس
&&
نارسیده سود افتاده به حبس
\\
چاهها کنده برای دیگران
&&
خویش را دیده فتاده اندر آن
\\
در سبب چون بی‌مرادت کرد رب
&&
پس چرا بدظن نگردی در سبب
\\
بس کسی از مکسبی خاقان شده
&&
دیگری زان مکسبه عریان شده
\\
بس کس از عقد زنان قارون شده
&&
بس کس از عقد زنان مدیون شده
\\
پس سبب گردان چو دم خر بود
&&
تکیه بر وی کم کنی بهتر بود
\\
ور سبب گیری نگیری هم دلیر
&&
که بس آفت‌هاست پنهانش به زیر
\\
سر استثناست این حزم و حذر
&&
زانک خر را بز نماید این قدر
\\
آنک چشمش بست گرچه گربزست
&&
ز احولی اندر دو چشمش خربزست
\\
چون مقلب حق بود ابصار را
&&
که بگرداند دل و افکار را
\\
چاه را تو خانه‌ای بینی لطیف
&&
دام را تو دانه‌ای بینی ظریف
\\
این تفسطط نیست تقلیب خداست
&&
می‌نماید که حقیقتها کجاست
\\
آنک انکار حقایق می‌کند
&&
جملگی او بر خیالی می‌تند
\\
او نمی‌گوید که حسبان خیال
&&
هم خیالی باشدت چشمی به مال
\\
\end{longtable}
\end{center}
