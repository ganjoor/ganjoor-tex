\begin{center}
\section*{بخش ۴۳ - حکایت آن رنجور کی طبیب درو اومید صحت ندید}
\label{sec:sh043}
\addcontentsline{toc}{section}{\nameref{sec:sh043}}
\begin{longtable}{l p{0.5cm} r}
آن یکی رنجور شد سوی طبیب
&&
گفت نبضم را فرو بین ای لبیب
\\
که ز نبض آگه شوی بر حال دل
&&
که رگ دستست با دل متصل
\\
چونک دل غیبست خواهی زو مثال
&&
زو بجو که با دلستش اتصال
\\
باد پنهانست از چشم ای امین
&&
در غبار و جنبش برگش ببین
\\
کز یمینست او وزان یا از شمال
&&
جنبش برگت بگوید وصف حال
\\
مستی دل را نمی‌دانی که کو
&&
وصف او از نرگس مخمور جو
\\
چون ز ذات حق بعیدی وصف ذات
&&
باز دانی از رسول و معجزات
\\
معجزاتی و کراماتی خفی
&&
بر زند بر دل ز پیران صفی
\\
که درونشان صد قیامت نقد هست
&&
کمترین آنک شود همسایه مست
\\
پس جلیس الله گشت آن نیک‌بخت
&&
کو به پهلوی سعیدی برد رخت
\\
معجزه کان بر جمادی زد اثر
&&
یا عصا با بحر یا شق‌القمر
\\
گر ترا بر جان زند بی‌واسطه
&&
متصل گردد به پنهان رابطه
\\
بر جمادات آن اثرها عاریه‌ست
&&
از پی روح خوش متواریه‌ست
\\
تا از آن جامد اثر گیرد ضمیر
&&
حبذا نان بی‌هیولای خمیر
\\
حبذا خوان مسیحی بی‌کمی
&&
حبذا بی‌باغ میوهٔ مریمی
\\
بر زند از جان کامل معجزات
&&
بر ضمیر جان طالب چون حیات
\\
معجزه بحرست و ناقص مرغ خاک
&&
مرغ آبی در وی آمن از هلاک
\\
عجزبخش جان هر نامحرمی
&&
لیک قدرت‌بخش جان هم‌دمی
\\
چون نیابی این سعادت در ضمیر
&&
پس ز ظاهر هر دم استدلال گیر
\\
که اثرها بر مشاعر ظاهرست
&&
وین اثرها از مؤثر مخبرست
\\
هست پنهان معنی هر داروی
&&
هم‌چو سحر و صنعت هر جادوی
\\
چون نظر در فعل و آثارش کنی
&&
گرچه پنهانست اظهارش کنی
\\
قوتی کان اندرونش مضمرست
&&
چون به فعل آید عیان و مظهرست
\\
چون به آثار این همه پیدا شدت
&&
چون نشد پیدا ز تاثیر ایزدت
\\
نه سببها و اثرها مغز و پوست
&&
چون بجویی جملگی آثار اوست
\\
دوست گیری چیزها را از اثر
&&
پس چرا ز آثاربخشی بی‌خبر
\\
از خیالی دوست گیری خلق را
&&
چون نگیری شاه غرب و شرق را
\\
این سخن پایان ندارد ای قباد
&&
حرص ما را اندرین پایان مباد
\\
\end{longtable}
\end{center}
