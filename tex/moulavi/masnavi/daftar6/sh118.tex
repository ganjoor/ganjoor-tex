\begin{center}
\section*{بخش ۱۱۸ - حکایت آن شخص کی خواب دید کی آنچ می‌طلبی از یسار  به مصر وفا شود آنجا گنجیست در فلان محله در فلان خانه چون به مصر  آمد کسی گفت من خواب دیده‌ایم کی  گنجیست به بغداد در فلان محله در  فلان خانه نام محله و خانهٔ این شخص  بگفت آن شخص فهم کرد کی  آن گنج در مصر گفتن جهت آن بود  کی مرا یقین کنند کی در غیر  خانهٔ خود نمی‌باید جستن ولیکن این گنج یقین و محقق جز در مصر حاصل نشود}
\label{sec:sh118}
\addcontentsline{toc}{section}{\nameref{sec:sh118}}
\begin{longtable}{l p{0.5cm} r}
بود یک میراثی مال و عقار
&&
جمله را خورد و بماند او عور و زار
\\
مال میراثی ندارد خود وفا
&&
چون بناکام از گذشته شد جدا
\\
او نداند قدر هم کاسان بیافت
&&
کو بکد و رنج و کسبش کم شتاف
\\
قدر جان زان می‌ندانی ای فلان
&&
که بدادت حق به بخشش رایگان
\\
نقد رفت و کاله رفته و خانه‌ها
&&
ماند چون چغدان در آن ویرانه‌ها
\\
گفت یا رب برگ دادی رفت برگ
&&
یا بده برگی و یا بفرست مرگ
\\
چون تهی شد یاد حق آغاز کرد
&&
یا رب و یا رب اجرنی ساز کرد
\\
چون پیمبر گفته مؤمن مزهرست
&&
در زمان خالیی ناله گرست
\\
چون شود پر مطربش بنهد ز دست
&&
پر مشو که آسیب دست او خوشست
\\
تی شو و خوش باش بین اصبعین
&&
کز می لا این سرمستست این
\\
رفت طغیان آب از چشمش گشاد
&&
آب چشمش زرع دین را آب داد
\\
\end{longtable}
\end{center}
