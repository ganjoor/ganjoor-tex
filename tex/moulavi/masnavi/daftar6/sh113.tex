\begin{center}
\section*{بخش ۱۱۳ - ذکر آن پادشاه که آن دانشمند را به اکراه در مجلس آورد و بنشاند ساقی شراب بر دانشمند عرضه کرد ساغر پیش او داشت رو بگردانید و ترشی و تندی آغاز کرد شاه ساقی را گفت کی هین در طبعش آر ساقی چندی بر سرش کوفت و شرابش در خورد داد الی آخره}
\label{sec:sh113}
\addcontentsline{toc}{section}{\nameref{sec:sh113}}
\begin{longtable}{l p{0.5cm} r}
پادشاهی مست اندر بزم خوش
&&
می‌گذشت آن یک فقیهی بر درش
\\
کرد اشارت کش درین مجلس کشید
&&
وان شراب لعل را با او چشید
\\
پس کشیدندش به شه بی‌اختیار
&&
شست در مجلس ترش چون زهر و مار
\\
عرضه کردش می نپذرفت او به خشم
&&
از شه و ساقی بگردانید چشم
\\
که به عمر خود نخوردستم شراب
&&
خوشتر آید از شرابم زهر ناب
\\
هین به جای می به من زهری دهید
&&
تا من از خویش و شما زین وا رهید
\\
می نخورده عربده آغاز کرد
&&
گشته در مجلس گران چون مرگ و درد
\\
هم‌چو اهل نفس و اهل آب و گل
&&
در جهان بنشسته با اصحاب دل
\\
حق ندارد خاصگان را در کمون
&&
از می احرار جز در یشربون
\\
عرضه می‌دارند بر محجوب جام
&&
حس نمی‌یابد از آن غیر کلام
\\
رو همی گرداند از ارشادشان
&&
که نمی‌بیند به دیده دادشان
\\
گر ز گوشش تا به حلقش ره بدی
&&
سر نصح اندر درونشان در شدی
\\
چون همه نارست جانش نیست نور
&&
که افکند در نار سوزان جز قشور
\\
مغز بیرون ماند و قشر گفت رفت
&&
کی شود از قشر معده گرم و زفت
\\
نار دوزخ جز که قشر افشار نیست
&&
نار را با هیچ مغزی کار نیست
\\
ور بود بر مغز ناری شعله‌زن
&&
بهر پختن دان نه بهر سوختن
\\
تا که باشد حق حکیم این قاعده
&&
مستمر دان در گذشته و نامده
\\
مغز نغز و قشرها مغفور ازو
&&
مغز را پس چون بسوزد دور ازو
\\
از عنایت گر بکوبد بر سرش
&&
اشتها آید شراب احمرش
\\
ور نکوبد ماند او بسته‌دهان
&&
چون فقیه از شرب و بزم این شهان
\\
گفت شه با ساقیش ای نیک‌پی
&&
چه خموشی ده به طبعش آر هی
\\
هست پنهان حاکمی بر هر خرد
&&
هرکه را خواهد به فن از سر برد
\\
آفتاب مشرق و تنویر او
&&
چون اسیران بسته در زنجیر او
\\
چرخ را چرخ اندر آرد در زمن
&&
چون بخواند در دماغش نیم فن
\\
عقل کو عقل دگر را سخره کرد
&&
مهره زو دارد ویست استاد نرد
\\
چند سیلی بر سرش زد گفت گیر
&&
در کشید از بیم سیلی آن زحیر
\\
مست گشت و شاد و خندان شد چو باغ
&&
در ندیمی و مضاحک رفت و لاغ
\\
شیرگیر و خوش شد انگشتک بزد
&&
سوی مبرز رفت تا میزک کند
\\
یک کنیزک بود در مبرز چو ماه
&&
سخت زیبا و ز قرناقان شاه
\\
چون بدید او را دهانش باز ماند
&&
عقل رفت و تن ستم‌پرداز ماند
\\
عمرها بوده عزب مشتاق و مست
&&
بر کنیزک در زمان در زد دو دست
\\
بس طپید آن دختر و نعره فراشت
&&
بر نیامد با وی و سودی نداشت
\\
زن به دست مرد در وقت لقا
&&
چون خمیر آمد به دست نانبا
\\
بسرشد گاهیش نرم و گه درشت
&&
زو بر آرد چاق چاقی زیر مشت
\\
گاه پهنش واکشد بر تخته‌ای
&&
درهمش آرد گهی یک لخته‌ای
\\
گاه در وی ریزد آب و گه نمک
&&
از تنور و آتشش سازد محک
\\
این چنین پیچند مطلوب و طلوب
&&
اندرین لعبند مغلوب و غلوب
\\
این لعب تنها نه شو را با زنست
&&
هر عشیق و عاشقی را این فنست
\\
از قدیم و حادث و عین و عرض
&&
پیچشی چون ویس و رامین مفترض
\\
لیک لعب هر یکی رنگی دگر
&&
پیچش هر یک ز فرهنگی دگر
\\
شوی و زن را گفته شد بهر مثال
&&
که مکن ای شوی زن را بد گسیل
\\
آن شب گردک نه ینگا دست او
&&
خوش امانت داد اندر دست تو
\\
کانچ با او تو کنی ای معتمد
&&
از بد و نیکی خدا با تو کند
\\
حاصل این‌جا این فقیه از بی‌خودی
&&
نه عفیفی ماندش و نه زاهدی
\\
آن فقیه افتاد بر آن حورزاد
&&
آتش او اندر آن پنبه فتاد
\\
جان به جان پیوست و قالب‌ها چخید
&&
چون دو مرغ سربریده می‌طپید
\\
چه سقایه چه ملک چه ارسلان
&&
چه حیا چه دین چه بیم و خوف جان
\\
چشمشان افتاده اندر عین و غین
&&
نه حسن پیداست این‌جا نه حسین
\\
شد دراز و کو طریق بازگشت
&&
انتظار شاه هم از حد گذشت
\\
شاه آمد تا ببیند واقعه
&&
دید آن‌جا زلزلهٔ القارعه
\\
آن فقیه از بیم برجست و برفت
&&
سوی مجلس جام را بربود تفت
\\
شه چون دوزخ پر شرار و پر نکال
&&
تشنهٔ خون دو جفت بدفعال
\\
چون فقیهش دید رخ پر خشم و قهر
&&
تلخ و خونی گشته هم‌چون جام زهر
\\
بانگ زد بر ساقیش که ای گرم‌دار
&&
چه نشستی خیره ده در طبعش آر
\\
خنده آمد شاه را گفت ای کیا
&&
آمدم با طبع آن دختر ترا
\\
پادشاهم کار من عدلست و داد
&&
زان خورم که یار را جودم بداد
\\
آنچ آن را من ننوشم هم‌چو نوش
&&
کی دهم در خورد یار و خویش و توش
\\
زان خورانم من غلامان را که من
&&
می‌خورم بر خوان خاص خویشتن
\\
زان خورانم بندگان را از طعام
&&
که خورم من خود ز پخته یا ز خام
\\
من چو پوشم از خز و اطلس لباس
&&
زان بپوشانم حشم را نه پلاس
\\
شرم دارم از نبی ذو فنون
&&
البسوهم گفت مما تلبسون
\\
مصطفی کرد این وصیت با بنون
&&
اطعموا الاذناب مما تاکلون
\\
دیگران را بس به طبع آورده‌ای
&&
در صبوری چست و راغب کرده‌ای
\\
هم به طبع‌آور بمردی خویش را
&&
پیشوا کن عقل صبراندیش را
\\
چون قلاووزی صبرت پر شود
&&
جان به اوج عرش و کرسی بر شود
\\
مصطفی بین که چو صبرش شد براق
&&
بر کشانیدش به بالای طباق
\\
\end{longtable}
\end{center}
