\begin{center}
\section*{بخش ۱۳۸ - رجوع کردن بدان قصه کی شاه‌زاده بدان طغیان زخم خورد از خاطر شاه پیش از استکمال فضایل دیگر از دنیا برفت}
\label{sec:sh138}
\addcontentsline{toc}{section}{\nameref{sec:sh138}}
\begin{longtable}{l p{0.5cm} r}
قصه کوته کن که رای نفس کور
&&
برد او را بعد سالی سوی گور
\\
شاه چون از محو شد سوی وجود
&&
چشم مریخیش آن خون کرده بود
\\
چون به ترکش بنگرید آن بی‌نظیر
&&
دید کم از ترکشش یک چوبه تیر
\\
گفت کو آن تیر و از حق باز جست
&&
گفت که اندر حلق او کز تیر تست
\\
عفو کرد آن شاه دریادل ولی
&&
آمده بد تیر اه بر مقتلی
\\
کشته شد در نوحهٔ او می‌گریست
&&
اوست جمله هم کشنده و هم ولیست
\\
ور نباشد هر دو او پس کل نیست
&&
هم کشندهٔ خلق و هم ماتم‌کنیست
\\
شکر می‌کرد آن شهید زردخد
&&
کان بزد بر جسم و بر معنی نزد
\\
جسم ظاهر عاقبت خود رفتنیست
&&
تا ابد معنی بخواهد شاد زیست
\\
آن عتاب ار رفت هم بر پوست رفت
&&
دوست بی‌آزار سوی دوست رفت
\\
گرچه او فتراک شاهنشه گرفت
&&
آخر از عین الکمال او ره گرفت
\\
و آن سوم کاهل‌ترین هر سه بود
&&
صورت و معنی به کلی او ربود
\\
\end{longtable}
\end{center}
