\begin{center}
\section*{بخش ۱۲۴ - بازگشتن آن شخص شادمان و مراد یافته و خدای را شکر گویان و سجده کنان و حیران در غرایب اشارات حق و ظهور تاویلات آن در  وجهی کی هیچ عقلی و فهمی بدانجا نرسد}
\label{sec:sh124}
\addcontentsline{toc}{section}{\nameref{sec:sh124}}
\begin{longtable}{l p{0.5cm} r}
باز گشت از مصر تا بغداد او
&&
ساجد و راکع ثناگر شکرگو
\\
جمله ره حیران و مست او زین عجب
&&
ز انعکاس روزی و راه طلب
\\
کر کجا اومیدوارم کرده بود
&&
وز کجا افشاند بر من سیم و سود
\\
این چه حکمت بود که قبلهٔ مراد
&&
کردم از خانه برون گمراه و شاد
\\
تا شتابان در ضلالت می‌شدم
&&
هر دم از مطلب جداتر می‌بدم
\\
باز آن عین ضلالت را به جود
&&
حق وسیلت کرد اندر رشد و سود
\\
گمرهی را منهج ایمان کند
&&
کژروی را محصد احسان کند
\\
تا نباشد هیچ محسن بی‌وجا
&&
تا نباشد هیچ خاین بی‌رجا
\\
اندرون زهر تریاق آن حفی
&&
کرد تا گویند ذواللطف الخفی
\\
نیست مخفی در نماز آن مکرمت
&&
در گنه خلعت نهد آن مغفرت
\\
منکران را قصد اذلال ثقات
&&
ذل شده عز و ظهور معجزات
\\
قصدشان ز انکار ذل دین بده
&&
عین ذل عز رسولان آمده
\\
گر نه انکار آمدی از هر بدی
&&
معجزه و برهان چرا نازل شدی
\\
خصم منکر تا نشد مصداق‌خواه
&&
کی کند قاضی تقاضای گواه
\\
معجزه هم‌چون گواه آمد زکی
&&
بهر صدق مدعی در بی‌شکی
\\
طعن چون می‌آمد از هر ناشناخت
&&
معجزه می‌داد حق و می‌نواخت
\\
مکر آن فرعون سیصد تو بده
&&
جمله ذل او و قمع او شده
\\
ساحران آورده حاضر نیک و بد
&&
تا که جرح معجزهٔ موسی کند
\\
تا عصا را باطل و رسوا کند
&&
اعتبارش را ز دلها بر کند
\\
عین آن مکر آیت موسی شود
&&
اعتبار آن عصا بالا رود
\\
لشکر آرد او پگه تا حول نیل
&&
تا زند بر موسی و قومش سبیل
\\
آمنی امت موسی شود
&&
او به تحت‌الارض و هامون در رود
\\
گر به مصر اندر بدی او نامدی
&&
وهم از سبطی کجا زایل شدی
\\
آمد و در سبط افکند او گداز
&&
که بدانک امن در خوفست راز
\\
آن بود لطف خفی کو را صمد
&&
نار بنماید خود آن نوری بود
\\
نیست مخفی مزد دادن در تقی
&&
ساحران را اجر بین بعد از خطا
\\
نیست مخفی وصل اندر پرورش
&&
ساحران را وصل داد او در برش
\\
نیست مخفی سیر با پای روا
&&
ساحران را سیر بین در قطع پا
\\
عارفان زانند دایم آمنون
&&
که گذر کردند از دریای خون
\\
امنشان از عین خوف آمد پدید
&&
لاجرم باشند هر دم در مزید
\\
امن دیدی گشته در خوفی خفی
&&
خوف بین هم در امیدی ای حفی
\\
آن امیر از مکر بر عیسی تند
&&
عیسی اندر خانه رو پنهان کند
\\
اندر آید تا شود او تاجدار
&&
خود ز شبه عیسی آید تاج‌دار
\\
هی می‌آویزید من عیسی نیم
&&
من امیرم بر جهودان خوش‌پیم
\\
زوترش بردار آویزید کو
&&
عیسی است از دست ما تخلیط‌جو
\\
چند لشکر می‌رود تا بر خورد
&&
برگ او فی گردد و بر سر خورد
\\
چند در عالم بود برعکس این
&&
زهر پندارد بود آن انگبین
\\
بس سپه بنهاده دل بر مرگ خویش
&&
روشنیها و ظفر آید به پیش
\\
ابرهه با پیل بهر ذل بیت
&&
آمده تا افکند حی را چو میت
\\
تا حریم کعبه را ویران کند
&&
جمله را زان جای سرگردان کند
\\
تا همه زوار گرد او تنند
&&
کعبهٔ او را همه قبله کنند
\\
وز عرب کینه کشد اندر گزند
&&
که چرا در کعبه‌ام آتش زنند
\\
عین سعیش عزت کعبه شده
&&
موجب اعزاز آن بیت آمده
\\
مکیان را عز یکی بد صد شده
&&
تا قیامت عزشان ممتد شده
\\
او و کعبهٔ او شده مخسوف‌تر
&&
از چیست این از عنایات قدر
\\
از جهاز ابرهه هم‌چون دده
&&
آن فقیران عرب توانگر شده
\\
او گمان برده که لشکر می‌کشید
&&
بهر اهل بیت او زر می‌کشید
\\
اندرین فسخ عزایم وین همم
&&
در تماشا بود در ره هر قدم
\\
خانه آمد گنج را او باز یافت
&&
کارش از لطف خدایی ساز یافت
\\
\end{longtable}
\end{center}
