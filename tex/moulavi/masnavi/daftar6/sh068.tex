\begin{center}
\section*{بخش ۶۸ - نومید شدن آن پادشاه از یافتن آن گنج و ملول شدن او از طلب آن}
\label{sec:sh068}
\addcontentsline{toc}{section}{\nameref{sec:sh068}}
\begin{longtable}{l p{0.5cm} r}
چونک تعویق آمد اندر عرض و طول
&&
شاه شد زان گنج دل سیر و ملول
\\
دشتها را گز گز آن شه چاه کند
&&
رقعه را از خشم پیش او فکند
\\
گفت گیر این رقعه کش آثار نیست
&&
تو بدین اولیتری کت کار نیست
\\
نیست این کار کسی کش هست کار
&&
که بسوزد گل بگردد گرد خار
\\
نادر افتد اهل این ماخولیا
&&
منتظر که روید از آهن گیا
\\
سخت جانی باید این فن را چو تو
&&
تو که داری جان سخت این را بجو
\\
گر نیابی نبودت هرگز ملال
&&
ور بیابی آن به تو کردم حلال
\\
عقل راه ناامیدی کی رود
&&
عشق باشد کان طرف بر سر دود
\\
لاابالی عشق باشد نی خرد
&&
عقل آن جوید کز آن سودی برد
\\
ترک‌تاز و تن‌گداز و بی‌حیا
&&
در بلا چون سنگ زیر آسیا
\\
سخت‌رویی که ندارد هیچ پشت
&&
بهره‌جویی را درون خویش کشت
\\
پاک می‌بازد نباشد مزدجو
&&
آنچنان که پاک می‌گیرد ز هو
\\
می‌دهد حق هستیش بی‌علتی
&&
می‌سپارد باز بی‌علت فتی
\\
که فتوت دادن بی علتست
&&
پاک‌بازی خارج هر ملتست
\\
زانک ملت فضل جوید یا خلاص
&&
پاک بازانند قربانان خاص
\\
نی خدا را امتحانی می‌کنند
&&
نی در سود و زیانی می‌زنند
\\
\end{longtable}
\end{center}
