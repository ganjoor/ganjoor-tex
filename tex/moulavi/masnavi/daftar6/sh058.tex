\begin{center}
\section*{بخش ۵۸ - بیان آنک بی‌کاران و افسانه‌جویان مثل آن ترک‌اند و عالم غرار غدار هم‌چو آن درزی و شهوات و زبان مضاحک گفتن این دنیاست و عمر هم‌چون آن اطلس پیش این درزی جهت قبای بقا و لباس تقوی ساختن}
\label{sec:sh058}
\addcontentsline{toc}{section}{\nameref{sec:sh058}}
\begin{longtable}{l p{0.5cm} r}
اطلس عمرت به مقراض شهور
&&
برد پاره‌پاره خیاط غرور
\\
تو تمنا می‌بری که اختر مدام
&&
لاغ کردی سعد بودی بر دوام
\\
سخت می‌تولی ز تربیعات او
&&
وز دلال و کینه و آفات او
\\
سخت می‌رنجی ز خاموشی او
&&
وز نحوس و قبض و کین‌کوشی او
\\
که چرا زهرهٔ طرب در رقص نیست
&&
بر سعود و رقص سعد او مه‌ایست
\\
اخترت گوید که گر افزون کنم
&&
لاغ را پس کلیت مغبون کنم
\\
تو مبین قلابی این اختران
&&
عشق خود بر قلب‌زن بین ای مهان
\\
\end{longtable}
\end{center}
