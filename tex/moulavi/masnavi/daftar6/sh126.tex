\begin{center}
\section*{بخش ۱۲۶ - مفتون شدن قاضی بر زن جوحی و در صندوق ماندن و نایب قاضی صندوق را خریدن باز سال دوم آمدن زن جوحی بر امید بازی پارینه و گفتن قاضی کی مرا آزاد کن و کسی دیگر را بجوی الی آخر القصه}
\label{sec:sh126}
\addcontentsline{toc}{section}{\nameref{sec:sh126}}
\begin{longtable}{l p{0.5cm} r}
جوحی هر سالی ز درویشی به فن
&&
رو بزن کردی کای دلخواه زن
\\
چون سلاحت هست رو صیدی بگیر
&&
تا بدوشانیم از صید تو شیر
\\
قوس ابرو تیر غمزه دام کید
&&
بهر چه دادت خدا از بهر صید
\\
رو پی مرغی شگرفی دام نه
&&
دانه بنما لیک در خوردش مده
\\
کام بنما و کن او را تلخ‌کام
&&
کی خورد دانه چو شد در حبس دام
\\
شد زن او نزد قاضی در گله
&&
که مرا افغان ز شوی ده‌دله
\\
قصه کوته کن که قاضی شد شکار
&&
از مقال و از جمال آن نگار
\\
گفت اندر محکمه‌ست این غلغله
&&
من نتوانم فهم کردن این گله
\\
گر به خلوت آیی ای سرو سهی
&&
از ستم‌کاری شو شرحم دهی
\\
گفت خانهٔ تو ز هر نیک و بدی
&&
باشد از بهر گله آمد شدی
\\
خانهٔ سر جمله پر سودا بود
&&
صدر پر وسواس و پر غوغا بود
\\
باقی اعضا ز فکر آسوده‌اند
&&
وآن صدور از صادران فرسوده‌اند
\\
در خزان و باد خوف حق گریز
&&
آن شقایق‌های پارین را بریز
\\
این شقایق منع نو اشکوفه‌هاست
&&
که درخت دل برای آن نماست
\\
خویش را در خواب کن زین افتکار
&&
سر ز زیر خواب در یقظت بر آر
\\
هم‌چو آن اصحاب کهف ای خواجه زود
&&
رو به ایقاظا که تحسبهم رقود
\\
گفت قاضی ای صنم معمول چیست
&&
گفت خانهٔ این کنیزک بس تهیست
\\
خصم در ده رفت و حارس نیز نیست
&&
بهر خلوت سخت نیکو مسکنیست
\\
امشب ار امکان بود آنجا بیا
&&
کار شب بی سمعه است و بی‌ریا
\\
جمله جاسوسان ز خمر خواب مست
&&
زنگی شب جمله را گردن زدست
\\
خواند بر قاضی فسون‌های عجب
&&
آن شکرلب وانگهانی از چه لب
\\
چند با آدم بلیس افسانه کرد
&&
چون حوا گفتش بخور آنگاه خورد
\\
اولین خون در جهان ظلم و داد
&&
از کف قابیل بهر زن فتاد
\\
نوح چون بر تابه بریان ساختی
&&
واهله بر تابه سنگ انداختی
\\
مکر زن بر کار او چیره شدی
&&
آب صاف وعظ او تیره شدی
\\
قوم را پیغام کردی از نهان
&&
که نگه دارید دین زین گمرهان
\\
\end{longtable}
\end{center}
