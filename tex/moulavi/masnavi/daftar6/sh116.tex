\begin{center}
\section*{بخش ۱۱۶ - بعد مکث ایشان متواری در بلاد چین در شهر تختگاه و بعد دراز شدن صبر بی‌صبر شدن آن بزرگین کی من رفتم الوداع خود را بر شاه عرضه کنم اما قدمی تنیلنی مقصودی او القی راسی کفادی ثم یا پای رساندم به مقصود و مراد یا سر بنهم هم‌چو دل از دست آن‌جا و نصیحت برادران او را سود ناداشتن یا عاذل العاشقین دع فة اضلها الله کیف ترشدها الی آخره}
\label{sec:sh116}
\addcontentsline{toc}{section}{\nameref{sec:sh116}}
\begin{longtable}{l p{0.5cm} r}
آن بزرگین گفت ای اخوان من
&&
ز انتظار آمد به لب این جان من
\\
لا ابالی گشته‌ام صبرم نماند
&&
مر مرا این صبر در آتش نشاند
\\
طاقت من زین صبوری طاق شد
&&
راقعهٔ من عبرت عشاق شد
\\
من ز جان سیر آمدم اندر فراق
&&
زنده بودن در فراق آمد نفاق
\\
چند درد فرقتش بکشد مرا
&&
سر ببر تا عشق سر بخشد مرا
\\
دین من از عشق زنده بودنست
&&
زندگی زین جان و سر ننگ منست
\\
تیغ هست از جان عاشق گردروب
&&
زانک سیف افتاد محاء الذنوب
\\
چون غبار تن بشد ماهم بتافت
&&
ماه جان من هوای صاف یافت
\\
عمرها بر طبل عشقت ای صنم
&&
ان فی متی حیاتی می‌زنم
\\
دعوی مرغابئی کردست جان
&&
کی ز طوفان بلا دارد فغان
\\
بط را ز اشکستن کشتی چه غم
&&
کشتی‌اش بر آب بس باشد قدم
\\
زنده زین دعوی بود جان و تنم
&&
من ازین دعوی چگونه تن زنم
\\
خواب می‌بینم ولی در خواب نه
&&
مدعی هستم ولی کذاب نه
\\
گر مرا صد بار تو گردن زنی
&&
هم‌چو شمعم بر فروزم روشنی
\\
آتش ار خرمن بگیرد پیش و پس
&&
شب‌روان را خرمن آن ماه بس
\\
کرده یوسف را نهان و مختبی
&&
حیلت اخوان ز یعقوب نبی
\\
خفیه کردندش به حیلت‌سازیی
&&
کرد آخر پیرهن غمازیی
\\
آن دو گفتندش نصیحت در سمر
&&
که مکن ز اخطار خود را بی‌خبر
\\
هین منه بر ریش‌های ما نمک
&&
هین مخور این زهر بر جلدی و شک
\\
جز به تدبیر یکی شیخی خبیر
&&
چون روی چون نبودت قلبی بصیر
\\
وای آن مرغی که ناروییده پر
&&
بر پرد بر اوج و افتد در خطر
\\
عقل باشد مرد را بال و پری
&&
چون ندارد عقل عقل رهبری
\\
یا مظفر یا مظفرجوی باش
&&
یا نظرور یا نظرورجوی باش
\\
بی ز مفتاح خرد این قرع باب
&&
از هوا باشد نه از روی صواب
\\
عالمی در دام می‌بین از هوا
&&
وز جراحت‌های هم‌رنگ دوا
\\
مار استادست بر سینه چو مرگ
&&
در دهانش بهر صید اشگرف برگ
\\
در حشایش چون حشیشی او بپاست
&&
مرغ پندارد که او شاخ گیاست
\\
چون نشیند بهر خور بر روی برگ
&&
در فتد اندر دهان مار و مرگ
\\
کرده تمساحی دهان خویش باز
&&
گرد دندانهاش کرمان دراز
\\
از بقیهٔ خور که در دندانش ماند
&&
کرم‌ها رویید و بر دندان نشاند
\\
مرغکان بینند کرم و قوت را
&&
مرج پندارند آن تابوت را
\\
چون دهان پر شد ز مرغ او ناگهان
&&
در کشدشان و فرو بندد دهان
\\
این جهان پر ز نقل و پر ز نان
&&
چون دهان باز آن تمساح دان
\\
بهر کرم و طعمه ای روزی‌تراش
&&
از فن تمساح دهر آمن مباش
\\
روبه افتد پهن اندر زیر خاک
&&
بر سر خاکش حبوب مکرناک
\\
تا بیاید زاغ غافل سوی آن
&&
پای او گیرد به مکر آن مکردان
\\
صدهزاران مکر در حیوان چو هست
&&
چون بود مکر بشر کو مهترست
\\
مصحفی در کف چو زین‌العابدین
&&
خنجری پر قهر اندر آستین
\\
گویدت خندان کای مولای من
&&
در دل او بابلی پر سحر و فن
\\
زهر قاتل صورتش شهدست و شیر
&&
هین مرو بی‌صحبت پیر خبیر
\\
جمله لذات هوا مکرست و زرق
&&
سوز و تاریکیست گرد نور برق
\\
برق نور کوته و کذب و مجاز
&&
گرد او ظلمات و راه تو دراز
\\
نه به نورش نامه توانی خواندن
&&
نه به منزل اسپ دانی راندن
\\
لیک جرم آنک باشی رهن برق
&&
از تو رو اندر کشد انوار شرق
\\
می‌کشاند مکر برقت بی‌دلیل
&&
در مفازهٔ مظلمی شب میل میل
\\
بر که افتی گاه و در جوی اوفتی
&&
گه بدین سو گه بدان سوی اوفتی
\\
خود نبینی تو دلیل ای جاه‌جو
&&
ور ببینی رو بگردانی ازو
\\
که سفر کردم درین ره شصت میل
&&
مر مرا گمراه گوید این دلیل
\\
گر نهم من گوش سوی این شگفت
&&
ز امر او راهم ز سر باید گرفت
\\
من درین ره عمر خود کردم گرو
&&
هرچه بادا باد ای خواجه برو
\\
راه کردی لیک در ظن چو برق
&&
عشر آن ره کن پی وحی چو شرق
\\
ظن لایغنی من الحق خوانده‌ای
&&
وز چنان برقی ز شرقی مانده‌ای
\\
هی در آ در کشتی ما ای نژند
&&
یا تو آن کشتی برین کشتی ببند
\\
گوید او چون ترک گیرم گیر و دار
&&
چون روم من در طفیلت کوروار
\\
کور با رهبر به از تنها یقین
&&
زان یکی ننگست و صد ننگست ازین
\\
می‌گریزی از پشه در کزدمی
&&
می‌گریزی در یمی تو از نمی
\\
می‌گریزی از جفاهای پدر
&&
در میان لوطیان و شور و شر
\\
می‌گریزی هم‌چو یوسف ز اندهی
&&
تا ز نرتع نلعب افتی در چهی
\\
در چه افتی زین تفرج هم‌چو او
&&
مر ترا لیک آن عنایت یار کو
\\
گر نبودی آن به دستوری پدر
&&
برنیاوردی ز چه تا حشر سر
\\
آن پدر بهر دل او اذن داد
&&
گفت چون اینست میلت خیر باد
\\
هر ضریری کز مسیحی سر کشد
&&
او جهودانه بماند از رشد
\\
قابل ضو بود اگر چه کور بود
&&
شد ازین اعراض او کور و کبود
\\
گویدش عیسی بزن در من دو دست
&&
ای عمی کحل عزیزی با منست
\\
از من ار کوری بیابی روشنی
&&
بر قمیص یوسف جان بر زنی
\\
کار و باری کت رسد بعد شکست
&&
اندر آن اقبال و منهاج رهست
\\
کار و باری که ندارد پا و سر
&&
ترک کن هی پیر خر ای پیر خر
\\
غیر پیر استاد و سرلشکر مباد
&&
پیر گردون نی ولی پیر رشاد
\\
در زمان چون پیر را شد زیردست
&&
روشنایی دید آن ظلمت‌پرست
\\
شرط تسلیم است نه کار دراز
&&
سود نبود در ضلالت ترک‌تاز
\\
من نجویم زین سپس راه اثیر
&&
پیر جویم پیر جویم پیر پیر
\\
پیر باشد نردبان آسمان
&&
تیر پران از که گردد از کمان
\\
نه ز ابراهیم نمرود گران
&&
کرد با کرکس سفر بر آسمان
\\
از هوا شد سوی بالا او بسی
&&
لیک بر گردون نپرد کرکسی
\\
گفتش ابراهیم ای مرد سفر
&&
کرکست من باشم اینت خوب‌تر
\\
چون ز من سازی به بالا نردبان
&&
بی پریدن بر روی بر آسمان
\\
آنچنان که می‌رود تا غرب و شرق
&&
بی ز زاد و راحله دل هم‌چو برق
\\
آنچنان که می‌رود شب ز اغتراب
&&
حس مردم شهرها در وقت خواب
\\
آنچنان که عارف از راه نهان
&&
خوش نشسته می‌رود در صد جهان
\\
گر ندادستش چنین رفتار دست
&&
این خبرها زان ولایت از کیست
\\
این خبرها وین روایات محق
&&
صد هزاران پیر بر وی متفق
\\
یک خلافی نی میان این عیون
&&
آنچنان که هست در علم ظنون
\\
آن تحری آمد اندر لیل تار
&&
وین حضور کعبه و وسط نهار
\\
خیز ای نمرود پر جوی از کسان
&&
نردبانی نایدت زین کرکسان
\\
عقل جزوی کرکس آمد ای مقل
&&
پر او با جیفه‌خواری متصل
\\
عقل ابدالان چو پر جبرئیل
&&
می‌پرد تا ظل سدره میل میل
\\
باز سلطانم گشم نیکوپیم
&&
فارغ از مردارم و کرکس نیم
\\
ترک کرکس کن که من باشم کست
&&
یک پر من بهتر از صد کرکست
\\
چند بر عمیا دوانی اسپ را
&&
باید استا پیشه را و کسپ را
\\
خویشتن رسوا مکن در شهر چین
&&
عاقلی جو خویش از وی در مچین
\\
آن چه گوید آن فلاطون زمان
&&
هین هوا بگار و رو بر وفق آن
\\
جمله می‌گویند اندر چین به جد
&&
بهر شاه خویشتن که لم یلد
\\
شاه ما خود هیچ فرزندی نزاد
&&
بلک سوی خویش زن را ره نداد
\\
هر که از شاهان ازین نوعش بگفت
&&
گردنش با تیغ بران کرد جفت
\\
شاه گوید چونک گفتی این مقال
&&
یا بکن ثابت که دارم من عیال
\\
مر مرا دختر اگر ثابت کنی
&&
یافتی از تیغ تیزم آمنی
\\
ورنه بی‌شک من ببرم حلق تو
&&
ای بگفته لاف کذب آمیغ تو
\\
بنگر ای از جهل گفته ناحقی
&&
پر ز سرهای بریده خندقی
\\
خندقی از قعر خندق تا گلو
&&
پر ز سرهای بریده زین غلو
\\
جمله اندر کار این دعوی شدند
&&
گردن خود را بدین دعوی زدند
\\
هان ببین این را به چشم اعتبار
&&
این چنین دعوی میندیش و میار
\\
تلخ خواهی کرد بر ما عمر ما
&&
کی برین می‌دارد ای دادر ترا
\\
گر رود صد سال آنک آگاه نیست
&&
بر عما آن از حساب راه نیست
\\
بی‌سلاحی در مرو در معرکه
&&
هم‌چو بی‌باکان مرو در تهلکه
\\
این همه گفتند و گفت آن ناصبور
&&
که مرا زین گفته‌ها آید نفور
\\
سینه پر آتش مرا چون منقل است
&&
کشت کامل گشت وقت منجل است
\\
صدر را صبری بد اکنون آن نماد
&&
بر مقام صبر عشق آتش نشاند
\\
صبر من مرد آن شبی که عشق زاد
&&
درگذشت او حاضران را عمر باد
\\
ای محدث از خطاب و از خطوب
&&
زان گذشتم آهن سردی مکوب
\\
سرنگونم هی رها کن پای من
&&
فهم کو در جملهٔ اجزای من
\\
اشترم من تا توانم می‌کشم
&&
چون فتادم زار با کشتن خوشم
\\
پر سر مقطوع اگر صد خندق است
&&
پیش درد من مزاج مطلق است
\\
من نخواهم زد دگر از خوف و بیم
&&
این چنین طبل هوا زیر گلیم
\\
من علم اکنون به صحرا می‌زنم
&&
یا سراندازی و یا روی صنم
\\
حلق کو نبود سزای آن شراب
&&
آن بریده به به شمشیر و ضراب
\\
دیده کو نبود ز وصلش در فره
&&
آن چنان دیده سپید کور به
\\
گوش کان نبود سزای راز او
&&
بر کنش که نبود آن بر سر نکو
\\
اندر آن دستی که نبود آن نصاب
&&
آن شکسته به به ساطور قصاب
\\
آنچنان پایی که از رفتار او
&&
جان نپیوندد به نرگس زار او
\\
آنچنان پا در حدید اولیترست
&&
که آنچنان پا عاقبت درد سرست
\\
\end{longtable}
\end{center}
