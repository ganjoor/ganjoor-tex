\begin{center}
\section*{بخش ۲۲ - تفسیر قوله علیه‌السلام موتوا قبل ان تموتوا بمیر ای دوست پیش از مرگ اگر می زندگی خواهی کی ادریس از چنین مردن بهشتی گشت پیش از ما}
\label{sec:sh022}
\addcontentsline{toc}{section}{\nameref{sec:sh022}}
\begin{longtable}{l p{0.5cm} r}
جان بسی کندی و اندر پرده‌ای
&&
زانک مردن اصل بد ناورده‌ای
\\
تا نمیری نیست جان کندن تمام
&&
بی‌کمال نردبان نایی به بام
\\
چون ز صد پایه دو پایه کم بود
&&
بام را کوشنده نامحرم بود
\\
چون رسن یک گز ز صد گز کم بود
&&
آب اندر دلو از چه کی رود
\\
غرق این کشتی نیابی ای امیر
&&
تا بننهی اندرو من الاخیر
\\
من آخر اصل دان کو طارقست
&&
کشتی وسواس و غی را غارقست
\\
آفتاب گنبد ازرق شود
&&
کشتی هش چونک مستغرق شود
\\
چون نمردی گشت جان کندن دراز
&&
مات شو در صبح ای شمع طراز
\\
تا نگشتند اختران ما نهان
&&
دانک پنهانست خورشید جهان
\\
گرز بر خود زن منی در هم شکن
&&
زانک پنبهٔ گوش آمد چشم تن
\\
گرز بر خود می‌زنی خود ای دنی
&&
عکس تست اندر فعالم این منی
\\
عکس خود در صورت من دیده‌ای
&&
در قتال خویش بر جوشیده‌ای
\\
هم‌چو آن شیری که در چه شد فرو
&&
عکس خود را خصم خود پنداشت او
\\
نفی ضد هست باشد بی‌شکی
&&
تا ز ضد ضد را بدانی اندکی
\\
این زمان جز نفی ضد اعلام نیست
&&
اندرین نشات دمی بی‌دام نیست
\\
بی‌حجابت باید آن ای ذو لباب
&&
مرگ را بگزین و بر دران حجاب
\\
نه چنان مرگی که در گوری روی
&&
مرگ تبدیلی که در نوری روی
\\
مرد بالغ گشت آن بچگی بمرد
&&
رومیی شد صبغت زنگی سترد
\\
خاک زر شد هیات خاکی نماند
&&
غم فرج شد خار غمناکی نماند
\\
مصطفی زین گفت کای اسرارجو
&&
مرده را خواهی که بینی زنده تو
\\
می‌رود چون زندگان بر خاکدان
&&
مرده و جانش شده بر آسمان
\\
جانش را این دم به بالا مسکنیست
&&
گر بمیرد روح او را نقل نیست
\\
زانک پیش از مرگ او کردست نقل
&&
این بمردن فهم آید نه به عقل
\\
نقل باشد نه چو نقل جان عام
&&
هم‌چو نقلی از مقامی تا مقام
\\
هرکه خواهد که ببیند بر زمین
&&
مرده‌ای را می‌رود ظاهر چنین
\\
مر ابوبکر تقی را گو ببین
&&
شد ز صدیقی امیرالمحشرین
\\
اندرین نشات نگر صدیق را
&&
تا به حشر افزون کنی تصدیق را
\\
پس محمد صد قیامت بود نقد
&&
زانک حل شد در فنای حل و عقد
\\
زادهٔ ثانیست احمد در جهان
&&
صد قیامت بود او اندر عیان
\\
زو قیامت را همی‌پرسیده‌اند
&&
ای قیامت تا قیامت راه چند
\\
با زبان حال می‌گفتی بسی
&&
که ز محشر حشر را پرسید کسی
\\
بهر این گفت آن رسول خوش‌پیام
&&
رمز موتوا قبل موت یا کرام
\\
هم‌چنانک مرده‌ام من قبل موت
&&
زان طرف آورده‌ام این صیت و صوت
\\
پس قیامت شو قیامت را ببین
&&
دیدن هر چیز را شرطست این
\\
تا نگردی او ندانی‌اش تمام
&&
خواه آن انوار باشد یا ظلام
\\
عقل گردی عقل را دانی کمال
&&
عشق گردی عشق را دانی ذبال
\\
گفتمی برهان این دعوی مبین
&&
گر بدی ادراک اندر خورد این
\\
هست انجیر این طرف بسیار و خوار
&&
گر رسد مرغی قنق انجیرخوار
\\
در همه عالم اگر مرد و زنند
&&
دم به دم در نزع و اندر مردنند
\\
آن سخنشان را وصیتها شمر
&&
که پدر گوید در آن دم با پسر
\\
تا بروید عبرت و رحمت بدین
&&
تا ببرد بیخ بغض و رشک و کین
\\
تو بدان نیت نگر در اقربا
&&
تا ز نزع او بسوزد دل ترا
\\
کل آت آت آن را نقد دان
&&
دوست را در نزع و اندر فقد دان
\\
وز غرضها زین نظر گردد حجاب
&&
این غرضها را برون افکن ز جیب
\\
ور نیاری خشک بر عجزی مه‌ایست
&&
دانک با عاجز گزیده معجزیست
\\
عجز زنجیریست زنجیرت نهاد
&&
چشم در زنجیرنه باید گشاد
\\
پس تضرع کن کای هادی زیست
&&
باز بودم بسته گشتم این ز چیست
\\
سخت‌تر افشرده‌ام در شر قدم
&&
که لفی خسرم ز قهرت دم به دم
\\
از نصیحتهای تو کر بوده‌ام
&&
بت‌شکن دعوی و بت‌گر بوده‌ام
\\
یاد صنعت فرض‌تر یا یاد مرگ
&&
مرگ مانند خزان تو اصل برگ
\\
سالها این مرگ طبلک می‌زند
&&
گوش تو بیگاه جنبش می‌کند
\\
گوید اندر نزع از جان آه مرگ
&&
این زمان کردت ز خود آگاه مرگ
\\
این گلوی مرگ از نعره گرفت
&&
طبل او بشکافت از ضرب شگفت
\\
در دقایق خویش را در بافتی
&&
رمز مردن این زمان در یافتی
\\
\end{longtable}
\end{center}
