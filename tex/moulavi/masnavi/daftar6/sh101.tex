\begin{center}
\section*{بخش ۱۰۱ - رجوع کردن به قصهٔ آن پای‌مرد و آن غریب وام‌دار و بازگشتن ایشان از سر گور خواجه و خواب دیدن پای‌مرد خواجه را الی آخره}
\label{sec:sh101}
\addcontentsline{toc}{section}{\nameref{sec:sh101}}
\begin{longtable}{l p{0.5cm} r}
بی‌نهایت آمد این خوش سرگذشت
&&
چون غریب از گور خواجه باز گشت
\\
پای مردش سوی خانهٔ خویش برد
&&
مهر صد دینار را فا او سپرد
\\
لوتش آورد و حکایت‌هاش گفت
&&
کز امید اندر دلش صد گل شکفت
\\
آنچ بعد العسر یسر او دیده بود
&&
با غریب از قصهٔ آن لب گشود
\\
نیم‌شب بگذشت و افسانه کنان
&&
خوابشان انداخت تا مرعای جان
\\
دید پامرد آن همایون خواجه را
&&
اندر آن شب خواب بر صدر سرا
\\
خواجه گفت ای پای‌مرد با نمک
&&
آنچ گفتی من شنیدم یک به یک
\\
لیک پاسخ دادنم فرمان نبود
&&
بی‌اشارت لب نیارستم گشود
\\
ما چو واقف گشته‌ایم از چون و چند
&&
مهر با لب‌های ما بنهاده‌اند
\\
تا نگردد رازهای غیب فاش
&&
تا نگردد منهدم عیش و معاش
\\
تا ندرد پردهٔ غفلت تمام
&&
تا نماند دیگ محنت نیم‌خام
\\
ما همه گوشیم کر شد نقش گوش
&&
ما همه نطقیم لیکن لب خموش
\\
هر چه ما دادیم دیدیم این زمان
&&
این جهان پرده‌ست و عینست آن جهان
\\
روز کشتن روز پنهان کردنست
&&
تخم در خاکی پریشان کردنست
\\
وقت بدرودن گه منجل زدن
&&
روز پاداش آمد و پیدا شدن
\\
\end{longtable}
\end{center}
