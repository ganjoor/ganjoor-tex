\begin{center}
\section*{بخش ۳۲ - قصهٔ هلال کی بندهٔ مخلص بود خدای را صاحب بصیرت بی‌تقلید پنهان شده در بندگی مخلوقان جهت مصلحت نه از عجز چنانک لقمان و یوسف از روی ظاهر و غیر ایشان بندهٔ سایس بود امیری را و آن امیر مسلمان بود اما چشم بسته داند اعمی که مادری دارد لیک چونی بوهم در نارد اگر با این دانش تعظیم این مادر کند ممکن بود کی از عمی خلاص یابد کی اذا اراد الله به عبد خیرا فتح عینی قلبه لیبصره بهما الغیب این راه ز زندگی دل حاصل کن کین زندگی تن صفت حیوانست}
\label{sec:sh032}
\addcontentsline{toc}{section}{\nameref{sec:sh032}}
\begin{longtable}{l p{0.5cm} r}
چون شنیدی بعضی اوصاف بلال
&&
بشنو اکنون قصهٔ ضعف هلال
\\
از بلال او بیش بود اندر روش
&&
خوی بد را بیش کرده بد کشش
\\
نه چو تو پس‌رو که هر دم پس‌تری
&&
سوی سنگی می‌روی از گوهری
\\
آن‌چنان کان خواجه را مهمان رسید
&&
خواجه از ایام و سالش بر رسید
\\
گفت عمرت چند سالست ای پسر
&&
بازگو و در مدزد و بر شمر
\\
گفت هجده هفده یا خود شانزده
&&
یا که پانزده ای برادرخوانده
\\
گفت واپس واپس ای خیره سرت
&&
باز می‌رو تا بکس مادرت
\\
\end{longtable}
\end{center}
