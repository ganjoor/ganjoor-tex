\begin{center}
\section*{بخش ۴۷ - بار دیگر رجوع کردن به قصهٔ صوفی و قاضی}
\label{sec:sh047}
\addcontentsline{toc}{section}{\nameref{sec:sh047}}
\begin{longtable}{l p{0.5cm} r}
گفت صوفی در قصاص یک قفا
&&
سر نشاید باد دادن از عمی
\\
خرقهٔ تسلیم اندر گردنم
&&
بر من آسان کرد سیلی خوردنم
\\
دید صوفی خصم خود را سخت زار
&&
گفت اگر مشتش زنم من خصم‌وار
\\
او به یک مشتم بریزد چون رصاص
&&
شاه فرماید مرا زجر و قصاص
\\
خیمه ویرانست و بشکسته وتد
&&
او بهانه می‌جود تا در فتد
\\
بهر این مرده دریغ آید دریغ
&&
که قصاصم افتد اندر زیر تیغ
\\
چون نمی‌توانست کف بر خصم زد
&&
عزمش آن شد کش سوی قاضی برد
\\
که ترازوی حق است و کیله‌اش
&&
مخلص است از مکر دیو و حیله‌اش
\\
هست او مقراض احقاد و جدال
&&
قاطع جن دو خصم و قیل و قال
\\
دیو در شیشه کند افسون او
&&
فتنه‌ها ساکن کند قانون او
\\
چون ترازو دید خصم پر طمع
&&
سرکشی بگذارد و گردد تبع
\\
ور ترازو نیست گر افزون دهیش
&&
از قسم راضی نگردد آگهیش
\\
هست قاضی رحمت و دفع ستیز
&&
قطره‌ای از بحر عدل رستخیز
\\
قطره گرچه خرد و کوته‌پا بود
&&
لطف آب بحر ازو پیدا بود
\\
از غبار ار پاک داری کله را
&&
تو ز یک قطره ببینی دجله را
\\
جزوها بر حال کلها شاهدست
&&
تا شفق غماز خورشید آمدست
\\
آن قسم بر جسم احمد راند حق
&&
آنچ فرمودست کلا والشفق
\\
مور بر دانه چرا لرزان بدی
&&
گر از آن یک دانه خرمن‌دان بدی
\\
بر سر حرف آ که صوفی بی‌دلست
&&
در مکافات جفا مستعجلست
\\
ای تو کرده ظلمها چون خوش‌دلی
&&
از تقاضای مکافی غافلی
\\
یا فراموشت شدست از کرده‌هات
&&
که فرو آویخت غفلت پرده‌هات
\\
گر نه خصمیهاستی اندر قفات
&&
جرم گردون رشک بردی بر صفات
\\
لیک محبوسی برای آن حقوق
&&
اندک اندک عذر می‌خواه از عقوق
\\
تا به یکبارت نگیرد محتسب
&&
آب خود روشن کن اکنون با محب
\\
رفت صوفی سوی آن سیلی‌زنش
&&
دست زد چون مدعی در دامنش
\\
اندر آوردش بر قاضی کشان
&&
کین خر ادبار را بر خر نشان
\\
یا به زخم دره او را ده جزا
&&
آنچنان که رای تو بیند سزا
\\
کانک از زجر تو میرد در دمار
&&
بر تو تاوان نیست آن باشد جبار
\\
در حد و تعزیر قاضی هر که مرد
&&
نیست بر قاضی ضمان کو نیست خرد
\\
نایب حقست و سایهٔ عدل حق
&&
آینهٔ هر مستحق و مستحق
\\
کو ادب از بهر مظلومی کند
&&
نه برای عرض و خشم و دخل خود
\\
چون برای حق و روز آجله‌ست
&&
گر خطایی شد دیت بر عاقله‌ست
\\
آنک بهر خود زند او ضامنست
&&
وآنک بهر حق زند او آمنست
\\
گر پدر زد مر پسر را و بمرد
&&
آن پدر را خون‌بها باید شمرد
\\
زانک او را بهر کار خویش زد
&&
خدمت او هست واجب بر ولد
\\
چون معلم زد صبی را شد تلف
&&
بر معلم نیست چیزی لا تخف
\\
کان معلم نایب افتاد و امین
&&
هر امین را هست حکمش همچنین
\\
نیست واجب خدمت استا برو
&&
پس نبود استا به زجرش کارجو
\\
ور پدر زد او برای خود زدست
&&
لاجرم از خونبها دادن نرست
\\
پس خودی را سر ببر ای ذوالفقار
&&
بی‌خودی شو فانیی درویش‌وار
\\
چون شدی بی‌خود هر آنچ تو کنی
&&
ما رمیت اذ رمیتی آمنی
\\
آن ضمان بر حق بود نه بر امین
&&
هست تفصیلش به فقه اندر مبین
\\
هر دکانی راست سودایی دگر
&&
مثنوی دکان فقرست ای پسر
\\
در دکان کفشگر چرمست خوب
&&
قالب کفش است اگر بینی تو چوب
\\
پیش بزازان قز و ادکن بود
&&
بهر گز باشد اگر آهن بود
\\
مثنوی ما دکان وحدتست
&&
غیر واحد هرچه بینی آن بتست
\\
بت ستودن بهر دام عامه را
&&
هم‌چنان دان کالغرانیق العلی
\\
خواندش در سورهٔ والنجم زود
&&
لیک آن فتنه بد از سوره نبود
\\
جمله کفار آن زمان ساجد شدند
&&
هم سری بود آنک سر بر در زدند
\\
بعد ازین حرفیست پیچاپیچ و دور
&&
با سلیمان باش و دیوان را مشور
\\
هین حدیث صوفی و قاضی بیار
&&
وان ستمکار ضعیف زار زار
\\
گفت قاضی ثبت العرش ای پسر
&&
تا برو نقشی کنم از خیر و شر
\\
کو زننده کو محل انتقام
&&
این خیالی گشته است اندر سقام
\\
شرع بهر زندگان و اغنیاست
&&
شرع بر اصحاب گورستان کجاست
\\
آن گروهی کز فقیری بی‌سرند
&&
صد جهت زان مردگان فانی‌تراند
\\
مرده از یک روست فانی در گزند
&&
صوفیان از صد جهت فانی شدند
\\
مرگ یک قتلست و این سیصد هزار
&&
هر یکی را خونبهایی بی‌شمار
\\
گرچه کشت این قوم را حق بارها
&&
ریخت بهر خونبها انبارها
\\
هم‌چو جرجیس‌اند هر یک در سرار
&&
کشته گشته زنده گشته شصت بار
\\
کشته از ذوق سنان دادگر
&&
می‌بسوزد که بزن زخمی دگر
\\
والله از عشق وجود جان‌پرست
&&
کشته بر قتل دوم عاشق‌ترست
\\
گفت قاضی من قضادار حیم
&&
حاکم اصحاب گورستان کیم
\\
این به صورت گر نه در گورست پست
&&
گورها در دودمانش آمدست
\\
بس بدیدی مرده اندر گور تو
&&
گور را در مرده بین ای کور تو
\\
گر ز گوری خشت بر تو اوفتاد
&&
عاقلان از گور کی خواهند داد
\\
گرد خشم و کینهٔ مرده مگرد
&&
هین مکن با نقش گرمابه نبرد
\\
شکر کن که زنده‌ای بر تو نزد
&&
کانک زنده رد کند حق کرد رد
\\
خشم احیا خشم حق و زخم اوست
&&
که به حق زنده‌ست آن پاکیزه‌پوست
\\
حق بکشت او را و در پاچه‌ش دمید
&&
زود قصابانه پوست از وی کشید
\\
نفخ در وی باقی آمد تا مب
&&
نفخ حق نبود چو نفخهٔ آن قصاب
\\
فرق بسیارست بین النفختین
&&
این همه زینست و آن سر جمله شین
\\
این حیات از وی برید و شد مضر
&&
وان حیات از نفخ حق شد مستمر
\\
این دم آن دم نیست کاید آن به شرح
&&
هین بر آ زین قعر چه بالای صرح
\\
نیستش بر خر نشاندن مجتهد
&&
نقش هیزم را کسی بر خر نهد
\\
بر نشست او نه پشت خر سزد
&&
پشت تابوتیش اولیتر سزد
\\
ظلم چه بود وضع غیر موضعش
&&
هین مکن در غیر موضع ضایعش
\\
گفت صوفی پس روا داری که او
&&
سیلیم زد بی‌قصاص و بی‌تسو
\\
این روا باشد که خر خرسی قلاش
&&
صوفیان را صفع اندازد بلاش
\\
گفت قاضی تو چه داری بیش و کم
&&
گفت دارم در جهان من شش درم
\\
گفت قاضی سه درم تو خرج کن
&&
آن سه دیگر را به او ده بی‌سخن
\\
زار و رنجورست و درویش و ضعیف
&&
سه درم در بایدش تره و رغیف
\\
بر قفای قاضی افتادش نظر
&&
از قفای صوفی آن بد خوب‌تر
\\
راست می‌کرد از پی سیلیش دست
&&
که قصاص سیلیم ارزان شدست
\\
سوی گوش قاضی آمد بهر راز
&&
سیلیی آورد قاضی را فراز
\\
گفت هر شش را بگیرید ای دو خصم
&&
من شوم آزاد بی خرخاش و وصم
\\
\end{longtable}
\end{center}
