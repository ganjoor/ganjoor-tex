\begin{center}
\section*{بخش ۶۵ - قصهٔ آن گنج‌نامه کی پهلوی قبه‌ای روی به قبله کن و تیر در کمان نه بینداز آنجا کی افتد گنجست}
\label{sec:sh065}
\addcontentsline{toc}{section}{\nameref{sec:sh065}}
\begin{longtable}{l p{0.5cm} r}
دید در خواب او شبی و خواب کو
&&
واقعهٔ بی‌خواب صوفی‌راست خو
\\
هاتفی گفتش کای دیده تعب
&&
رقعه‌ای در مشق وراقان طلب
\\
خفیه زان وراق کت همسایه است
&&
سوی کاغذپاره‌هاش آور تو دست
\\
رقعه‌ای شکلش چنین رنگش چنین
&&
بس بخوان آن را به خلوت ای حزین
\\
چون بدزدی آن ز وراق ای پسر
&&
پس برون رو ز انبهی و شور و شر
\\
تو بخوان آن را به خود در خلوتی
&&
هین مجو در خواندن آن شرکتی
\\
ور شود آن فاش هم غمگین مشو
&&
که نیابد غیر تو زان نیم جو
\\
ور کشد آن دیر هان زنهار تو
&&
ورد خود کن دم به دم لاتقنطوا
\\
این بگفت و دست خود آن مژده‌ور
&&
بر دل او زد که رو زحمت ببر
\\
چون به خویش آمد ز غیبت آن جوان
&&
می‌نگنجید از فرح اندر جهان
\\
زهرهٔ او بر دریدی از قلق
&&
گر نبودی رفق و حفظ و لطف حق
\\
یک فرح آن کز پس شصد حجاب
&&
گوش او بشنید از حضرت جواب
\\
از حجب چون حس سمعش در گذشت
&&
شد سرافراز و ز گردون بر گذشت
\\
که بود کان حس چشمش ز اعتبار
&&
زان حجاب غیب هم یابد گذار
\\
چون گذاره شد حواسش از حجاب
&&
پس پیاپی گرددش دید و خطاب
\\
جانب دکان وراق آمد او
&&
دست می‌برد او به مشقش سو به سو
\\
پیش چشمش آمد آن مکتوب زود
&&
با علاماتی که هاتف گفته بود
\\
در بغل زد گفت خواجه خیر باد
&&
این زمان وا می‌رسم ای اوستاد
\\
رفت کنج خلوتی و آن را بخواند
&&
وز تحیر واله و حیران بماند
\\
که بدین سان گنج‌نامهٔ بی‌بها
&&
چون فتاده ماند اندر مشقها
\\
باز اندر خاطرش این فکر جست
&&
کز پی هر چیز یزدان حافظست
\\
کی گذارد حافظ اندر اکتناف
&&
که کسی چیزی رباید از گزاف
\\
گر بیابان پر شود زر و نقود
&&
بی رضای حق جوی نتوان ربود
\\
ور بخوانی صد صحف بی سکته‌ای
&&
بی قدر یادت نماند نکته‌ای
\\
ور کنی خدمت نخوانی یک کتاب
&&
علمهای نادره یابی ز جیب
\\
شد ز جیب آن کف موسی ضو فشان
&&
کان فزون آمد ز ماه آسمان
\\
کانک می‌جستی ز چرخ با نهیب
&&
سر بر آوردستت ای موسی ز جیب
\\
تا بدانی که آسمانهای سمی
&&
هست عکس مدرکات آدمی
\\
نی که اول دست برد آن مجید
&&
از دو عالم پیشتر عقل آفرید
\\
این سخن پیدا و پنهانست بس
&&
که نباشد محرم عنقا مگس
\\
باز سوی قصه باز آ ای پسر
&&
قصهٔ گنج و فقیر آور به سر
\\
\end{longtable}
\end{center}
