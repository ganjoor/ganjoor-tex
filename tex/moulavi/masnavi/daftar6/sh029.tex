\begin{center}
\section*{بخش ۲۹ - وصیت کردن مصطفی علیه‌السلام صدیق را رضی الله عنه کی چون بلال را مشتری می‌شوی هر آینه ایشان از ستیز بر خواهند در بها فزود و بهای او را خواهند فزودن مرا درین فضیلت شریک خود کن وکیل من باش و نیم بها از من بستان}
\label{sec:sh029}
\addcontentsline{toc}{section}{\nameref{sec:sh029}}
\begin{longtable}{l p{0.5cm} r}
مصطفی گفتش کای اقبال‌جو
&&
اندرین من می‌شوم انباز تو
\\
تو وکیلم باش نیمی بهر من
&&
مشتری شو قبض کن از من ثمن
\\
گفت صد خدمت کنم رفت آن زمان
&&
سوی خانهٔ آن جهود بی‌امان
\\
گفت با خود کز کف طفلان گهر
&&
پس توان آسان خریدن ای پدر
\\
عقل و ایمان را ازین طفلان گول
&&
می‌خرد با ملک دنیا دیو غول
\\
آنچنان زینت دهد مردار را
&&
که خرد زیشان دو صد گلزار را
\\
آن‌چنان مهتاب پیماید به سحر
&&
کز خسان صد کیسه برباید به سحر
\\
انبیاشان تاجری آموختند
&&
پیش ایشان شمع دین افروختند
\\
دیو و غول ساحر از سحر و نبرد
&&
انبیا را در نظرشان زشت کرد
\\
زشت گرداند به جادویی عدو
&&
تا طلاق افتد میان جفت و شو
\\
دیده‌هاشان را به سحر می‌دوختند
&&
تا چنین جوهر به خس بفروختند
\\
این گهر از هر دو عالم برترست
&&
هین بخر زین طفل جاهل کو خرست
\\
پیش خر خرمهره و گوهر یکیست
&&
آن اشک را در در و دریا شکیست
\\
منکر بحرست و گوهرهای او
&&
کی بود حیوان در و پیرایه‌جو
\\
در سر حیوان خدا ننهاده است
&&
کو بود در بند لعل و درپرست
\\
مر خران را هیچ دیدی گوش‌وار
&&
گوش و هوش خر بود در سبزه‌زار
\\
احسن التقویم در والتین بخوان
&&
که گرامی گوهرست ای دوست جان
\\
احسن التقویم از عرش او فزون
&&
احسن التقویم از فکرت برون
\\
گر بگویم قیمت این ممتنع
&&
من بسوزم هم بسوزد مستمع
\\
لب ببند اینجا و خر این سو مران
&&
رفت این صدیق سوی آن خران
\\
حلقه در زد چو در را بر گشود
&&
رفت بی‌خود در سرای آن جهود
\\
بی‌خود و سرمست و پر آتش نشست
&&
از دهانش بس کلام تلخ جست
\\
کین ولی الله را چون می‌زنی
&&
این چه حقدست ای عدو روشنی
\\
گر ترا صدقیست اندر دین خود
&&
ظلم بر صادق دلت چون می‌دهد
\\
ای تو در دین جهودی ماده‌ای
&&
کین گمان داری تو بر شه‌زاده‌ای
\\
در همه ز آیینهٔ کژساز خود
&&
منگر ای مردود نفرین ابد
\\
آنچ آن دم از لب صدیق جست
&&
گر بگویم گم کنی تو پای و دست
\\
آن ینابیع الحکم هم‌چون فرات
&&
از دهان او دوان از بی‌جهات
\\
هم‌چو از سنگی که آبی شد روان
&&
نه ز پهلو مایه دارد نه از میان
\\
اسپر خود کرده حق آن سنگ را
&&
بر گشاده آب مینارنگ را
\\
هم‌چنانک از چشمهٔ چشم تو نور
&&
او روان کردست بی‌بخل و فتور
\\
نه ز پیه آن مایه دارد نه ز پوست
&&
روی‌پوشی کرد در ایجاد دوست
\\
در خلای گوش باد جاذبش
&&
مدرک صدق کلام و کاذبش
\\
آن چه بادست اندر آن خرد استخوان
&&
کو پذیرد حرف و صوت قصه‌خوان
\\
استخوان و باد روپوشست و بس
&&
در دو عالم غیر یزدان نیست کس
\\
مستمع او قایل او بی‌احتجاب
&&
زانک الاذنان من الراس ای مثاب
\\
گفت رحمت گر همی‌آید برو
&&
زر بده بستانش ای اکرام‌خو
\\
از منش وا خر چو می‌سوزد دلت
&&
بی‌منت حل نگردد مشکلت
\\
گفت صد خدمت کنم پانصد سجود
&&
بنده‌ای دارم تن اسپید و جهود
\\
تن سپید و دل سیاهستش بگیر
&&
در عوض ده تن سیاه و دل منیر
\\
پس فرستاد و بیاورد آن همام
&&
بود الحق سخت زیبا آن غلام
\\
آنچنان که ماند حیران آن جهود
&&
آن دل چون سنگش از جا رفت زود
\\
حالت صورت‌پرستان این بود
&&
سنگشان از صورتی مومین بود
\\
باز کرد استیزه و راضی نشد
&&
که برین افزون بده بی‌هیچ بد
\\
یک نصاب نقره هم بر وی فزود
&&
تا که راضی گشت حرص آن جهود
\\
\end{longtable}
\end{center}
