\begin{center}
\section*{بخش ۲۱ - حکایت آن مطرب کی در بزم امیر ترک این غزل آغاز کرد گلی یا سوسنی یا سرو یا ماهی نمی‌دانم ازین آشفتهٔ بی‌دل چه می‌خواهی نمی‌دانم  و بانگ بر زدن ترک کی آن بگو کی می‌دانی و جواب مطرب امیر را}
\label{sec:sh021}
\addcontentsline{toc}{section}{\nameref{sec:sh021}}
\begin{longtable}{l p{0.5cm} r}
مطرب آغازید پیش ترک مست
&&
در حجاب نغمه اسرار الست
\\
من ندانم که تو ماهی یا وثن
&&
من ندانم تا چه می‌خواهی ز من
\\
می‌ندانم که چه خدمت آرمت
&&
تن زنم یا در عبارت آرمت
\\
این عجب که نیستی از من جدا
&&
می‌ندانم من کجاام تو کجا
\\
می‌ندانم که مرا چون می‌کشی
&&
گاه در بر گاه در خون می‌کشی
\\
هم‌چنین لب در ندانم باز کرد
&&
می‌ندانم می‌ندانم ساز کرد
\\
چون ز حد شد می‌ندانم از شگفت
&&
ترک ما را زین حراره دل گرفت
\\
برجهید آن ترک و دبوسی کشید
&&
تا علیها بر سر مطرب رسید
\\
گرز را بگرفت سرهنگی بدست
&&
گفت نه مطرب کشی این دم بدست
\\
گفت این تکرار بی حد و مرش
&&
کوفت طبعم را بکوبم من سرش
\\
قلتبانا می‌ندانی گه مخور
&&
ور همی‌دانی بزن مقصود بر
\\
آن بگو ای گیج که می‌دانیش
&&
می‌ندانم می‌ندانم در مکش
\\
من بپرسم کز کجایی هی مری
&&
تو بگویی نه ز بلخ و نه از هری
\\
نه ز بغداد و نه موصل نه طراز
&&
در کشی در نی و نی راه دراز
\\
خود بگو من از کجاام باز ره
&&
هست تنقیح مناط اینجا بله
\\
یا بپرسیدم چه خوردی ناشتاب
&&
تو بگویی نه شراب و نه کباب
\\
نه قدید و نه ثرید و نه عدس
&&
آنچ خوردی آن بگو تنها و بس
\\
این سخن‌خایی دراز از بهر چیست
&&
گفت مطرب زانک مقصودم خفیست
\\
می‌رمد اثبات پیش از نفی تو
&&
نفی کردم تا بری ز اثبات بو
\\
در نوا آرم بنفی این ساز را
&&
چون بمیری مرگ گوید راز را
\\
\end{longtable}
\end{center}
