\begin{center}
\section*{بخش ۱۳۰ - باز آمدن زن جوحی به محکمهٔ قاضی سال دوم بر امید وظیفهٔ پارسال و شناختن قاضی او را الی اتمامه}
\label{sec:sh130}
\addcontentsline{toc}{section}{\nameref{sec:sh130}}
\begin{longtable}{l p{0.5cm} r}
بعد سالی باز جوحی از محن
&&
رو به زن کرد و بگفت ای چست زن
\\
آن وظیفهٔ پار را تجدید کن
&&
پیش قاضی از گلهٔ من گو سخن
\\
زن بر قاضی در آمد با زنان
&&
مر زنی را کرد آن زن ترجمان
\\
تا بنشناسد ز گفتن قاضیش
&&
یاد ناید از بلای ماضیش
\\
هست فتنه غمرهٔ غماز زن
&&
لیک آن صدتو شود ز آواز زن
\\
چون نمی‌توانست آوازی فراشت
&&
غمزهٔ تنهای زن سودی نداشت
\\
گفت قاضی رو تو خصمت را بیار
&&
تا دهم کار ترا با او قرار
\\
جوحی آمد قاضیش نشناخت زود
&&
کو به وقت لقیه در صندوق بود
\\
زو شنیده بود آواز از برون
&&
در شری و بیع و در نقص و فزون
\\
گفت نفقهٔ زن چرا ندهی تمام
&&
گفت از جان شرع را هستم غلام
\\
لیک اگر میرم ندارم من کفن
&&
مفلس این لعبم و شش پنج زن
\\
زین سخن قاضی مگر بشناختش
&&
یاد آورد آن دغل وان باختش
\\
گفت آن شش پنج با من باختی
&&
پار اندر شش درم انداختی
\\
نوبت من رفت امسال آن قمار
&&
با دگر کس باز دست از من بدار
\\
از شش و از پنج عارف گشت فرد
&&
محترز گشتست زین شش پنج نرد
\\
رست او از پنج حس و شش جهت
&&
از ورای آن همه کرد آگهت
\\
شد اشاراتش اشارات ازل
&&
جاوز الاوهام طرا و اعتزل
\\
زین چه شش گوشه گر نبود برون
&&
چون بر آرد یوسفی را از درون
\\
واردی بالای چرخ بی ستن
&&
جسم او چون دلو در چه چاره کن
\\
یوسفان چنگال در دلوش زده
&&
رسته از چاه و شه مصری شده
\\
دلوهای دیگر از چه آب‌جو
&&
دلو او فارغ ز آب اصحاب‌جو
\\
دلوها غواص آب از بهر قوت
&&
دلو او قوت و حیات جان حوت
\\
دلوها وابستهٔ چرخ بلند
&&
دلو او در اصبعین زورمند
\\
دلو چه و حبل چه و چرخ چی
&&
این مثال بس رکیکست ای اچی
\\
از کجا آرم مثالی بی‌شکست
&&
کفو آن نه آید و نه آمدست
\\
صد هزاران مرد پنهان در یکی
&&
صد کمان و تیر درج ناوکی
\\
ما رمیت اذ رمیتی فتنه‌ای
&&
صد هزاران خرمن اندر حفنه‌ای
\\
آفتابی در یکی ذره نهان
&&
ناگهان آن ذره بگشاید دهان
\\
ذره ذره گردد افلاک و زمین
&&
پیش آن خورشید چون جست از کمین
\\
این چنین جانی چه درخورد تنست
&&
هین بشو ای تن ازین جان هر دو دست
\\
ای تن گشته وثاق جان بسست
&&
چند تاند بحر درمشکی نشست
\\
ای هزاران جبرئیل اندر بشر
&&
ای مسیحان نهان در جوف خر
\\
ای هزاران کعبه پنهان در کنیس
&&
ای غلط‌انداز عفریت و بلیس
\\
سجده‌گاه لامکانی در مکان
&&
مر بلیسان را ز تو ویران دکان
\\
که چرا من خدمت این طین کنم
&&
صورتی را نم لقب چون دین کنم
\\
نیست صورت چشم را نیکو به مال
&&
تا ببینی شعشعهٔ نور جلال
\\
\end{longtable}
\end{center}
