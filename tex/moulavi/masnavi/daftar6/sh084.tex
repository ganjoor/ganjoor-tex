\begin{center}
\section*{بخش ۸۴ - منادی کردن سید ملک ترمد کی هر کی در سه یا چهار روز به سمرقند رود به فلان مهم خلعت و اسپ و غلام و کنیزک و چندین زر دهم و شنیدن دلقک خبر این منادی در ده و آمدن به اولاقی نزد شاه کی من باری نتوانم رفتن}
\label{sec:sh084}
\addcontentsline{toc}{section}{\nameref{sec:sh084}}
\begin{longtable}{l p{0.5cm} r}
سید ترمد که آنجا شاه بود
&&
مسخرهٔ او دلقک آگاه بود
\\
داشت کاری در سمرقند او مهم
&&
جست‌الاقی تا شود او مستتم
\\
زد منادی هر که اندر پنج روز
&&
آردم زانجا خبر بدهم کنوز
\\
دلقک اندر ده بد و آن را شنید
&&
بر نشست و تا بترمد می‌دوید
\\
مرکبی دو اندر آن ره شد سقط
&&
از دوانیدن فرس را زان نمط
\\
پس به دیوان در دوید از گرد راه
&&
وقت ناهنگام ره جست او به شاه
\\
فجفجی در جملهٔ دیوان فتاد
&&
شورشی در وهم آن سلطان فتاد
\\
خاص و عام شهر را دل شد ز دست
&&
تا چه تشویش و بلا حادث شدست
\\
یا عدوی قاهری در قصد ماست
&&
یا بلایی مهلکی از غیب خاست
\\
که ز ده دلقک به سیران درشت
&&
چند اسپی تازی اندر راه کشت
\\
جمع گشته بر سرای شاه خلق
&&
تا چرا آمد چنین اشتاب دلق
\\
از شتاب او و فحش اجتهاد
&&
غلغل و تشویش در ترمد فتاد
\\
آن یکی دو دست بر زانوزنان
&&
وآن دگر از وهم واویلی‌کنان
\\
از نفیر و فتنه و خوف نکال
&&
هر دلی رفته به صد کوی خیال
\\
هر کسی فالی همی‌زد از قیاس
&&
تا چه آتش اوفتاد اندر پلاس
\\
راه جست و راه دادش شاه زود
&&
چون زمین بوسید گفتش هی چه بود
\\
هرکه می‌پرسید حالی زان ترش
&&
دست بر لب می‌نهاد او که خمش
\\
وهم می‌افزود زین فرهنگ او
&&
جمله در تشویش گشته دنگ او
\\
کرد اشارت دلق که ای شاه کرم
&&
یک‌دمی بگذار تا من دم زنم
\\
تا که باز آید به من عقلم دمی
&&
که فتادم در عجایب عالمی
\\
بعد یک ساعت که شه از وهم و ظن
&&
تلخ گشتش هم گلو و هم دهن
\\
که ندیده بود دلقک را چنین
&&
که ازو خوشتر نبودش هم‌نشین
\\
دایما دستان و لاغ افراشتی
&&
شاه را او شاد و خندان داشتی
\\
آن چنان خندانش کردی در نشست
&&
که گرفتی شه شکم را با دو دست
\\
که ز زور خنده خوی کردی تنش
&&
رو در افتادی ز خنده کردنش
\\
باز امروز این چنین زرد و ترش
&&
دست بر لب می‌زند کای شه خمش
\\
وهم در وهم و خیال اندر خیال
&&
شاه را تا خود چه آید از نکال
\\
که دل شه با غم و پرهیز بود
&&
زانک خوارمشاه بس خون‌ریز بود
\\
بس شهان آن طرف را کشته بود
&&
یا به حیله یا به سطوت آن عنود
\\
این شه ترمد ازو در وهم بود
&&
وز فن دلقک خود آن وهمش فزود
\\
گفت زوتر بازگو تا حال چیست
&&
این چنین آشوب و شور تو ز کیست
\\
گفت من در ده شنیدم آنک شاه
&&
زد منادی بر سر هر شاه‌راه
\\
که کسی خواهم که تازد در سه روز
&&
تا سمرقند و دهم او را کنوز
\\
من شتابیدم بر تو بهر آن
&&
تا بگویم که ندارم آن توان
\\
این چنین چستی نیاید از چو من
&&
باری این اومید را بر من متن
\\
گفت شه لعنت برین زودیت باد
&&
که دو صد تشویش در شهر اوفتاد
\\
از برای این قدر خام‌ریش
&&
آتش افکندی درین مرج و حشیش
\\
هم‌چو این خامان با طبل و علم
&&
که الاقانیم در فقر و عدم
\\
لاف شیخی در جهان انداخته
&&
خویشتن را بایزیدی ساخته
\\
هم ز خود سالک شده واصل شده
&&
محفلی واکرده در دعوی‌کده
\\
خانهٔ داماد پرآشوب و شر
&&
قوم دختر را نبوده زین خبر
\\
ولوله که کار نیمی راست شد
&&
شرطهایی که ز سوی ماست شد
\\
خانه‌ها را روفتیم آراستیم
&&
زین هوس سرمست و خوش برخاستیم
\\
زان طرف آمد یکی پیغام نی
&&
مرغی آمد این طرف زان بام نی
\\
زین رسالات مزید اندر مزید
&&
یک جوابی زان حوالیتان رسید
\\
نی ولیکن یار ما زین آگهست
&&
زانک از دل سوی دل لا بد رهست
\\
پس از آن یاری که اومید شماست
&&
از جواب نامه ره خالی چراست
\\
صد نشانست از سرار و از جهار
&&
لیک بس کن پرده زین در بر مدار
\\
باز رو تا قصهٔ آن دلق گول
&&
که بلا بر خویش آورد از فضول
\\
پس وزیرش گفت ای حق را ستن
&&
بشنو از بندهٔ کمینه یک سخن
\\
دلقک از ده بهر کاری آمدست
&&
رای او گشت و پشیمانش شدست
\\
ز آب و روغن کهنه را نو می‌کند
&&
او به مسخرگی برون‌شو می‌کند
\\
غمد را بنمود و پنهان کرد تیغ
&&
باید افشردن مرورا بی‌دریغ
\\
پسته را یا جوز را تا نشکنی
&&
نی نماید دل نی بدهد روغنی
\\
مشنو این دفع وی و فرهنگ او
&&
در نگر در ارتعاش و رنگ او
\\
گفت حق سیماهم فی وجههم
&&
زانک غمازست سیما و منم
\\
این معاین هست ضد آن خبر
&&
که بشر به سرشته آمد این بشر
\\
گفت دلقک با فغان و با خروش
&&
صاحبا در خون این مسکین مکوش
\\
بس گمان و وهم آید در ضمیر
&&
کان نباشد حق و صادق ای امیر
\\
ان بعض الظن اثم است ای وزیر
&&
نیست استم راست خاصه بر فقیر
\\
شه نگیرد آنک می‌رنجاندش
&&
از چه گیرد آنک می‌خنداندش
\\
گفت صاحب پیش شه جاگیر شد
&&
کاشف این مکر و این تزویر شد
\\
گفت دلقک را سوی زندان برید
&&
چاپلوس و زرق او را کم خرید
\\
می‌زنیدش چون دهل اشکم‌تهی
&&
تا دهل‌وار او دهدمان آگهی
\\
تر و خشک و پر و تی باشد دهل
&&
بانگ او آگه کند ما را ز کل
\\
تا بگوید سر خود از اضطرار
&&
آنچنان که گیرد این دلها قرار
\\
چون طمانینست صدق و با فروغ
&&
دل نیارامد به گفتار دروغ
\\
کذب چون خس باشد و دل چون دهان
&&
خس نگردد در دهان هرگز نهان
\\
تا درو باشد زبانی می‌زند
&&
تا به دانش از دهان بیرون کند
\\
خاصه که در چشم افتد خس ز باد
&&
چشم افتد در نم و بند و گشاد
\\
ما پس این خس را زنیم اکنون لگد
&&
تا دهان و چشم ازین خس وا رهد
\\
گفت دلقک ای ملک آهسته باش
&&
روی حلم و مغفرت را کم‌خراش
\\
تا بدین حد چیست تعجیل نقم
&&
من نمی‌پرم به دست تو درم
\\
آن ادب که باشد از بهر خدا
&&
اندر آن مستعجلی نبود روا
\\
وآنچ باشد طبع و خشم و عارضی
&&
می‌شتابد تا نگردد مرتضی
\\
ترسد ار آید رضا خشمش رود
&&
انتقام و ذوق آن فایت شود
\\
شهوت کاذب شتابد در طعام
&&
خوف فوت ذوق هست آن خود سقام
\\
اشتها صادق بود تاخیر به
&&
تا گواریده شود آن بی‌گره
\\
تو پی دفع بلایم می‌زنی
&&
تا ببینی رخنه را بندش کنی
\\
تا از آن رخنه برون ناید بلا
&&
غیر آن رخنه بسی دارد قضا
\\
چارهٔ دفع بلا نبود ستم
&&
چاره احسان باشد و عفو و کرم
\\
گفت الصدقه مرد للبلا
&&
داو مرضاک به صدقه یا فتی
\\
صدقه نبود سوختن درویش را
&&
کور کردن چشم حلم‌اندیش را
\\
گفت شه نیکوست خیر و موقعش
&&
لیک چون خیری کنی در موضعش
\\
موضع رخ شه نهی ویرانیست
&&
موضع شه اسپ هم نادانیست
\\
در شریعت هم عطا هم زجر هست
&&
شاه را صدر و فرس را درگه است
\\
عدل چه بود وضع اندر موضعش
&&
ظلم چه بود وضع در ناموقعش
\\
نیست باطل هر چه یزدان آفرید
&&
از غضب وز حلم وز نصح و مکید
\\
خیر مطلق نیست زینها هیچ چیز
&&
شر مطلق نیست زینها هیچ نیز
\\
نفع و ضر هر یکی از موضعست
&&
علم ازین رو واجبست و نافعست
\\
ای بسا زجری که بر مسکین رود
&&
در ثواب از نان و حلوا به بود
\\
زانک حلوا بی‌اوان صفرا کند
&&
سیلیش از خبث مستنقا کند
\\
سیلیی در وقت بر مسکین بزن
&&
که رهاند آنش از گردن زدن
\\
زخم در معنی فتد از خوی بد
&&
چوب بر گرد اوفتد نه بر نمد
\\
بزم و زندن هست هر بهرام را
&&
بزم مخلص را و زندان خام را
\\
شق باید ریش را مرهم کنی
&&
چرک را در ریش مستحکم کنی
\\
تا خورد مر گوشت را در زیر آن
&&
نیم سودی باشد و پنجه زیان
\\
گفت دلقک من نمی‌گویم گذار
&&
من همی‌گویم تحریی بیار
\\
هین ره صبر و تانی در مبند
&&
صبر کن اندیشه می‌کن روز چند
\\
در تانی بر یقینی بر زنی
&&
گوش‌مال من بایقانی کنی
\\
در روش یمشی مکبا خود چرا
&&
چون همی‌شاید شدن در استوا
\\
مشورت کن با گروه صالحان
&&
بر پیمبر امر شاورهم بدان
\\
امرهم شوری برای این بود
&&
کز تشاور سهو و کژ کمتر رود
\\
این خردها چون مصابیح انورست
&&
بیست مصباح از یکی روشن‌ترست
\\
بوک مصباحی فتد اندر میان
&&
مشتعل گشته ز نور آسمان
\\
غیرت حق پرده‌ای انگیختست
&&
سفلی و علوی به هم آمیختست
\\
گفت سیروا می‌طلب اندر جهان
&&
بخت و روزی را همی‌کن امتحان
\\
در مجالس می‌طلب اندر عقول
&&
آن چنان عقلی که بود اندر رسول
\\
زانک میراث از رسول آنست و بس
&&
که ببیند غیبها از پیش و پس
\\
در بصرها می‌طلب هم آن بصر
&&
که نتابد شرح آن این مختصر
\\
بهر این کردست منع آن با شکوه
&&
از ترهب وز شدن خلوت به کوه
\\
تا نگردد فوت این نوع التقا
&&
کان نظر بختست و اکسیر بقا
\\
در میان صالحان یک اصلحیست
&&
بر سر توقیعش از سلطان صحیست
\\
کان دعا شد با اجابت مقترن
&&
کفو او نبود کبار انس و جن
\\
در مری‌اش آنک حلو و حامض است
&&
حجت ایشان بر حق داحض است
\\
که چو ما او را به خود افراشتیم
&&
عذر و حجت از میان بر داشتیم
\\
قبله را چون کرد دست حق عیان
&&
پس تحری بعد ازین مردود دان
\\
هین بگردان از تحری رو و سر
&&
که پدید آمد معاد و مستقر
\\
یک زمان زین قبله گر ذاهل شوی
&&
سخرهٔ هر قبلهٔ باطل شوی
\\
چون شوی تمییزده را ناسپاس
&&
بجهد از تو خطرت قبله‌شناس
\\
گر ازین انبار خواهی بر و بر
&&
نیم‌ساعت هم ز همدردان مبر
\\
که در آن دم که ببری زین معین
&&
مبتلی گردی تو با بئس القرین
\\
\end{longtable}
\end{center}
