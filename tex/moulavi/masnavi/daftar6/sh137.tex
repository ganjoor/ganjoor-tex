\begin{center}
\section*{بخش ۱۳۷ - رجوع کردن به قصهٔ پروردن حق تعالی نمرود را بی‌واسطهٔ مادر و دایه در طفلی}
\label{sec:sh137}
\addcontentsline{toc}{section}{\nameref{sec:sh137}}
\begin{longtable}{l p{0.5cm} r}
حاصل آن روضه چو باغ عارفان
&&
از سموم صرصر آمد در امان
\\
یک پلنگی طفلکان نو زاده بود
&&
گفتم او را شیر ده طاعت نمود
\\
پس بدادش شیر و خدمتهاش کرد
&&
تا که بالغ گشت و زفت و شیرمرد
\\
چون فطامش شد بگفتم با پری
&&
تا در آموزید نطق و داوری
\\
پرورش دادم مر او را زان چمن
&&
کی بگفت اندر بگنجد فن من
\\
داده من ایوب را مهر پدر
&&
بهر مهمانی کرمان بی‌ضرر
\\
داده کرمان را برو مهر ولد
&&
بر پدر من اینت قدرت اینت ید
\\
مادران را داب من آموختم
&&
چون بود لطفی که من افروختم
\\
صد عنایت کردم و صد رابطه
&&
تا ببیند لطف من بی‌واسطه
\\
تا نباشد از سبب در کش‌مکش
&&
تا بود هر استعانت از منش
\\
ورنه تا خود هیچ عذری نبودش
&&
شکوتی نبود ز هر یار بدش
\\
این حضانه دید با صد رابطه
&&
که بپروردم ورا بی‌واسطه
\\
شکر او آن بود ای بندهٔ جلیل
&&
که شد او نمرود و سوزندهٔ خلیل
\\
هم‌چنان کین شاه‌زاده شکر شاه
&&
کرد استکبار و استکثار جاه
\\
که چرا من تابع غیری شوم
&&
چونک صاحب ملک و اقبال نوم
\\
لطف‌های شه که ذکر آن گذشت
&&
از تجبر بر دلش پوشیده گشت
\\
هم‌چنان نمرود آن الطاف را
&&
زیر پا بنهاد از جهل و عمی
\\
این زمان کافر شد و ره می‌زند
&&
کبر و دعوی خدایی می‌کند
\\
رفته سوی آسمان با جلال
&&
با سه کرکس تا کند با من قتال
\\
صد هزاران طفل بی‌تلویم را
&&
کشته تا یابد وی ابراهیم را
\\
که منجم گفته کاندر حکم سال
&&
زاد خواهد دشمنی بهر قتال
\\
هین بکن در دفع آن خصم احتیاط
&&
هر که می‌زایید می‌کشت از خباط
\\
کوری او رست طفل وحی کش
&&
ماند خون‌های دگر در گردنش
\\
از پدر یابید آن ملک ای عجب
&&
تا غرورش داد ظلمات نسب
\\
دیگران را گر ام و اب شد حجاب
&&
او ز ما یابید گوهرها به جیب
\\
گرگ درنده‌ست نفس بد یقین
&&
چه بهانه می‌نهی بر هر قرین
\\
در ضلالت هست صد کل را کله
&&
نفس زشت کفرناک پر سفه
\\
زین سبب می‌گویم این بندهٔ فقیر
&&
سلسله از گردن سگ برمگیر
\\
گر معلم گشت این سگ هم سگست
&&
باش ذلت نفسه کو بدرگست
\\
فرض می‌آری به جا گر طایفی
&&
بر سهیلی چون ادیم طایفی
\\
تا سهیلت وا خرد از شر پوست
&&
تا شوی چون موزه‌ای هم‌پای دوست
\\
جمله قرآن شرح خبث نفس‌هاست
&&
بنگر اندر مصحف آن چشمت کجاست
\\
ذکر نفس عادیان کالت بیافت
&&
در قتال انبیا مو می‌شکافت
\\
قرن قرن از شوم نفس بی‌ادب
&&
ناگهان اندر جهان می‌زد لهب
\\
\end{longtable}
\end{center}
