\begin{center}
\section*{بخش ۱۳۲ - در بیان آنک دوزخ گوید کی قنطرهٔ صراط بر سر اوست ای ممن از صراط زودتر بگذر زود بشتاب تا عظمت نور تو آتش ما را نکشد جز یا ممن فان نورک اطفاء ناری}
\label{sec:sh132}
\addcontentsline{toc}{section}{\nameref{sec:sh132}}
\begin{longtable}{l p{0.5cm} r}
زآتش عاشق ازین رو ای صفی
&&
می‌شود دوزخ ضعیف و منطقی
\\
گویدش بگذر سبک ای محتشم
&&
ورنه ز آتش‌های تو مرد آتشم
\\
کفر که کبریت دوزخ اوست و بس
&&
بین که می‌پخساند او را این نفس
\\
زود کبریت بدین سودا سپار
&&
تا نه دوزخ بر تو تازد نه شرار
\\
گویدش جنت گذر کن هم‌چو باد
&&
ورنه گردد هر چه من دارم کساد
\\
که تو صاحب‌خرمنی من خوشه‌چین
&&
من بتی‌ام تو ولایت‌های چین
\\
هست لرزان زو جحیم و هم جنان
&&
نه مر این را نه مر آن را زو امان
\\
رفت عمرش چاره را فرصت نیافت
&&
صبر بس سوزان بدت وجان بر نتافت
\\
مدتی دندان‌کنان این می‌کشید
&&
نارسیده عمر او آخر رسید
\\
صورت معشوق زو شد در نهفت
&&
رفت و شد با معنی معشوق جفت
\\
گفت لبسش گر ز شعر و ششترست
&&
اعتناق بی‌حجابش خوشترست
\\
من شدم عریان ز تن او از خیال
&&
می‌خرامم در نهایات الوصال
\\
این مباحث تا بدین‌جا گفتنیست
&&
هرچه آید زین سپس بنهفتنیست
\\
ور بگویی ور بکوشی صد هزار
&&
هست بیگار و نگردد آشکار
\\
تا به دریا سیر اسپ و زین بود
&&
بعد ازینت مرکب چوبین بود
\\
مرکب چوبین به خشکی ابترست
&&
خاص آن دریاییان را رهبرست
\\
این خموشی مرکب چوبین بود
&&
بحریان را خامشی تلقین بود
\\
هر خموشی که ملولت می‌کند
&&
نعره‌های عشق آن سو می‌زند
\\
تو همی‌گویی عجب خامش چراست
&&
او همی‌گوید عجب گوشش کجاست
\\
من ز نعره کر شدم او بی‌خبر
&&
تیزگوشان زین سمر هستند کر
\\
آن یکی در خواب نعره می‌زند
&&
صد هزاران بحث و تلقین می‌کند
\\
این نشسته پهلوی او بی‌خبر
&&
خفته خود آنست و کر زان شور و شر
\\
وان کسی کش مرکب چوبین شکست
&&
غرقه شد در آب او خود ماهیست
\\
نه خموشست و نه گویا نادریست
&&
حال او را در عبارت نام نیست
\\
نیست زین دو هر دو هست آن بوالعجب
&&
شرح این گفتن برونست از ادب
\\
این مثال آمد رکیک و بی‌ورود
&&
لیک در محسوس ازین بهتر نبود
\\
\end{longtable}
\end{center}
