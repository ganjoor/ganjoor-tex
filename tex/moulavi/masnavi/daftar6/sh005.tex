\begin{center}
\section*{بخش ۵ - حکایت غلام هندو کی به خداوندزادهٔ خود پنهان هوای آورده بود چون دختر را با مهتر زاده‌ای عقد کردند غلام خبر یافت رنجور شد و می‌گداخت و هیچ طبیب علت او را در نمی‌یافت و او را زهرهٔ گفتن نه}
\label{sec:sh005}
\addcontentsline{toc}{section}{\nameref{sec:sh005}}
\begin{longtable}{l p{0.5cm} r}
خواجه‌ای را بود هندو بنده‌ای
&&
پروریده کرده او را زنده‌ای
\\
علم و آدابش تمام آموخته
&&
در دلش شمع هنر افروخته
\\
پروریدش از طفولیت به ناز
&&
در کنار لطف آن اکرام‌ساز
\\
بود هم این خواجه را خوش دختری
&&
سیم‌اندامی گشی خوش‌گوهری
\\
چون مراهق گشت دختر طالبان
&&
بذل می‌کردند کابین گران
\\
می‌رسیدش از سوی هر مهتری
&&
بهر دختر دم به دم خوزه‌گری
\\
گفت خواجه مال را نبود ثبات
&&
روز آید شب رود اندر جهات
\\
حسن صورت هم ندارد اعتبار
&&
که شود رخ زرد از یک زخم خار
\\
سهل باشد نیز مهترزادگی
&&
که بود غره به مال و بارگی
\\
ای بسا مهتربچه کز شور و شر
&&
شد ز فعل زشت خود ننگ پدر
\\
پر هنر را نیز اگر باشد نفیس
&&
کم پرست و عبرتی گیر از بلیس
\\
علم بودش چون نبودش عشق دین
&&
او ندید از آدم الا نقش طین
\\
گرچه دانی دقت علم ای امین
&&
زانت نگشاید دو دیدهٔ غیب‌بین
\\
او نبیند غیر دستاری و ریش
&&
از معرف پرسد از بیش و کمیش
\\
عارفا تو از معرف فارغی
&&
خود همی‌بینی که نور بازغی
\\
کار تقوی دارد و دین و صلاح
&&
که ازو باشد بدو عالم فلاح
\\
کرد یک داماد صالح اختیار
&&
که بد او فخر همه خیل و تبار
\\
پس زنان گفتند او را مال نیست
&&
مهتری و حسن و استقلال نیست
\\
گفت آنها تابع زهدند و دین
&&
بی‌زر او گنجیست بر روی زمین
\\
چون به جد تزویج دختر گشت فاش
&&
دست پیمان و نشانی و قماش
\\
پس غلام خرد که اندر خانه بود
&&
گشت بیمار و ضعیف و زار زود
\\
هم‌چو بیمار دقی او می‌گداخت
&&
علت او را طبیبی کم شناخت
\\
عقل می‌گفتی که رنجش از دلست
&&
داروی تن در غم دل باطلست
\\
آن غلامک دم نزد از حال خویش
&&
کز چه می‌آید برو در سینه نیش
\\
گفت خاتون را شبی شوهر که تو
&&
باز پرسش در خلا از حال او
\\
تو به جای مادری او را بود
&&
که غم خود پیش تو پیدا کند
\\
چونک خاتون در گوش این کلام
&&
روز دیگر رفت نزدیک غلام
\\
پس سرش را شانه می‌کرد آن ستی
&&
با دو صد مهر و دلال و آشتی
\\
آنچنان که مادران مهربان
&&
نرم کردش تا در آمد در بیان
\\
که مرا اومید از تو این نبود
&&
که دهی دختر به بیگانهٔ عنود
\\
خواجه‌زادهٔ ما و ما خسته‌جگر
&&
حیف نبود که رود جای دگر
\\
خواست آن خاتون ز خشمی که آمدش
&&
که زند وز بام زیر اندازدش
\\
کو که باشد هندوی مادرغری
&&
که طمع دارد به خواجه دختری
\\
گفت صبر اولی بود خود را گرفت
&&
گفت با خواجه که بشنو این شگفت
\\
این چنین گراء کی خاین بود
&&
ما گمان برده که هست او معتمد
\\
\end{longtable}
\end{center}
