\begin{center}
\section*{بخش ۱۰۳ - حکایت آن پادشاه و وصیت کردن او سه پسر خویش را کی درین سفر در ممالک من فلان جا چنین ترتیب نهید و فلان جا چنین نواب نصب کنید اما الله الله به فلان قلعه  مروید و گرد آن مگردید}
\label{sec:sh103}
\addcontentsline{toc}{section}{\nameref{sec:sh103}}
\begin{longtable}{l p{0.5cm} r}
بود شاهی شاه را بد سه پسر
&&
هر سه صاحب‌فطنت و صاحب‌نظر
\\
هر یکی از دیگری استوده‌تر
&&
در سخا و در وغا و کر و فر
\\
پیش شه شه‌زادگان استاده جمع
&&
قرة العینان شه هم‌چون سه شمع
\\
از ره پنهان ز عینین پسر
&&
می‌کشید آبی نخیل آن پدر
\\
تا ز فرزند آب این چشمه شتاب
&&
می‌رود سوی ریاض مام و باب
\\
تازه می‌باشد ریاض والدین
&&
گشته جاری عینشان زین هر دو عین
\\
چون شود چشمه ز بیماری علیل
&&
خشک گردد برگ و شاخ آن نخیل
\\
خشکی نخلش همی‌گوید پدید
&&
که ز فرزندان شجر نم می‌کشید
\\
ای بسا کاریز پنهان هم‌چنین
&&
متصل با جانتان یا غافلین
\\
ای کشیده ز آسمان و از زمین
&&
مایه‌ها تا گشته جسم تو سمین
\\
عاریه‌ست این کم همی‌باید فشارد
&&
کانچ بگرفتی همی‌باید گزارد
\\
جز نفخت کان ز وهاب آمدست
&&
روح را باش آن دگرها بیهدست
\\
بیهده نسبت به جان می‌گویمش
&&
نی بنسبت با صنیع محکمش
\\
\end{longtable}
\end{center}
