\begin{center}
\section*{بخش ۱۵ - روان شدن خواجه به سوی ده}
\label{sec:sh015}
\addcontentsline{toc}{section}{\nameref{sec:sh015}}
\begin{longtable}{l p{0.5cm} r}
خواجه در کار آمد و تجهیز ساخت
&&
مرغ عزمش سوی ده اشتاب تاخت
\\
اهل و فرزندان سفر را ساختند
&&
رخت را بر گاو عزم انداختند
\\
شادمانان و شتابان سوی ده
&&
که بری خوردیم از ده مژده ده
\\
مقصد ما را چراگاه خوشست
&&
یار ما آنجا کریم و دلکشست
\\
با هزاران آرزومان خوانده است
&&
بهر ما غرس کرم بنشانده است
\\
ما ذخیرهٔ ده زمستان دراز
&&
از بر او سوی شهر آریم باز
\\
بلک باغ ایثار راه ما کند
&&
در میان جان خودمان جا کند
\\
عجلوا اصحابنا کی تربحوا
&&
عقل می‌گفت از درون لا تفرحوا
\\
من رباح الله کونوا رابحین
&&
ان ربی لا یحب الفرحین
\\
افرحوا هونا بما آتاکم
&&
کل آت مشغل الهاکم
\\
شاد از وی شو مشو از غیر وی
&&
او بهارست و دگرها ماه دی
\\
هر چه غیر اوست استدراج تست
&&
گرچه تخت و ملکتست و تاج تست
\\
شاد از غم شو که غم دام لقاست
&&
اندرین ره سوی پستی ارتقاست
\\
غم یکی گنجیست و رنج تو چو کان
&&
لیک کی در گیرد این در کودکان
\\
کودکان چون نام بازی بشنوند
&&
جمله با خر گور هم تگ می‌دوند
\\
ای خران کور این سو دامهاست
&&
در کمین این سوی خون‌آشامهاست
\\
تیرها پران کمان پنهان ز غیب
&&
بر جوانی می‌رسد صد تیر شیب
\\
گام در صحرای دل باید نهاد
&&
زانک در صحرای گل نبود گشاد
\\
ایمن آبادست دل ای دوستان
&&
چشمه‌ها و گلستان در گلستان
\\
عج الی القلب و سر یا ساریه
&&
فیه اشجار و عین جاریه
\\
ده مرو ده مرد را احمق کند
&&
عقل را بی نور و بی رونق کند
\\
قول پیغامبر شنو ای مجتبی
&&
گور عقل آمد وطن در روستا
\\
هر که را در رستا بود روزی و شام
&&
تا بماهی عقل او نبود تمام
\\
تا بماهی احمقی با او بود
&&
از حشیش ده جز اینها چه درود
\\
وانک ماهی باشد اندر روستا
&&
روزگاری باشدش جهل و عمی
\\
ده چه باشد شیخ واصل ناشده
&&
دست در تقلید و حجت در زده
\\
پیش شهر عقل کلی این حواس
&&
چون خران چشم‌بسته در خراس
\\
این رها کن صورت افسانه گیر
&&
هل تو دردانه تو گندم‌دانه گیر
\\
گر بدر ره نیست هین بر می‌ستان
&&
گر بدان ره نیستت این سو بران
\\
ظاهرش گیر ار چه ظاهر کژ پرد
&&
عاقبت ظاهر سوی باطن برد
\\
اول هر آدمی خود صورتست
&&
بعد از آن جان کو جمال سیرتست
\\
اول هر میوه جز صورت کیست
&&
بعد از آن لذت که معنی ویست
\\
اولا خرگاه سازند و خرند
&&
ترک را زان پس به مهمان آورند
\\
صورتت خرگاه دان معنیت ترک
&&
معنیت ملاح دان صورت چو فلک
\\
بهر حق این را رها کن یک نفس
&&
تا خر خواجه بجنباند جرس
\\
\end{longtable}
\end{center}
