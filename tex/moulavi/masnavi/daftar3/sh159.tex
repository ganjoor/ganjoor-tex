\begin{center}
\section*{بخش ۱۵۹ - خجل گشتن خروس پیش سگ به سبب دروغ شدن در آن سه وعده}
\label{sec:sh159}
\addcontentsline{toc}{section}{\nameref{sec:sh159}}
\begin{longtable}{l p{0.5cm} r}
چند چند آخر دروغ و مکر تو
&&
خود نپرد جز دروغ از وکر تو
\\
گفت حاشا از من و از جنس من
&&
که بگردیم از دروغی ممتحن
\\
ما خروسان چون مؤذن راست‌گوی
&&
هم رقیب آفتاب و وقت‌جوی
\\
پاسبان آفتابیم از درون
&&
گر کنی بالای ما طشتی نگون
\\
پاسبان آفتابند اولیا
&&
در بشر واقف ز اسرار خدا
\\
اصل ما را حق پی بانگ نماز
&&
داد هدیه آدمی را در جهاز
\\
گر بناهنگام سهوی‌مان رود
&&
در اذان آن مقتل ما می‌شود
\\
گفت ناهنگام حی عل فلاح
&&
خون ما را می‌کند خوار و مباح
\\
آنک معصوم آمد و پاک از غلط
&&
آن خروس جان وحی آمد فقط
\\
آن غلامش مرد پیش مشتری
&&
شد زیان مشتری آن یکسری
\\
او گریزانید مالش را ولیک
&&
خون خود را ریخت اندر یاب نیک
\\
یک زیان دفع زیانها می‌شدی
&&
جسم و مال ماست جانها را فدا
\\
پیش شاهان در سیاست‌گستری
&&
می‌دهی تو مال و سر را می‌خری
\\
اعجمی چون گشته‌ای اندر قضا
&&
می‌گریزانی ز داور مال را
\\
\end{longtable}
\end{center}
