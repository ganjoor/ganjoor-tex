\begin{center}
\section*{بخش ۲۲۸ - یافتن عاشق معشوق را و بیان آنک جوینده یابنده بود کی و من یعمل مثقال ذرة خیرا یره}
\label{sec:sh228}
\addcontentsline{toc}{section}{\nameref{sec:sh228}}
\begin{longtable}{l p{0.5cm} r}
کان جوان در جست و جو بد هفت سال
&&
از خیال وصل گشته چون خیال
\\
سایهٔ حق بر سر بنده بود
&&
عاقبت جوینده یابنده بود
\\
گفت پیغامبر که چون کوبی دری
&&
عاقبت زان در برون آید سری
\\
چون نشینی بر سر کوی کسی
&&
عاقبت بینی تو هم روی کسی
\\
چون ز چاهی می‌کنی هر روز خاک
&&
عاقبت اندر رسی در آب پاک
\\
جمله دانند این اگر تو نگروی
&&
هر چه می‌کاریش روزی بدروی
\\
سنگ بر آهن زدی آتش نجست
&&
این نباشد ور بباشد نادرست
\\
آنک روزی نیستش بخت و نجات
&&
ننگرد عقلش مگر در نادرات
\\
کان فلان کس کشت کرد و بر نداشت
&&
و آن صدف برد و صدف گوهر نداشت
\\
بلعم باعور و ابلیس لعین
&&
سود نامدشان عبادتها و دین
\\
صد هزاران انبیا و ره‌روان
&&
ناید اندر خاطر آن بدگمان
\\
این دو را گیرد که تاریکی دهد
&&
در دلش ادبار جز این کی نهد
\\
بس کسا که نان خورد دلشاد او
&&
مرگ او گردد بگیرد در گلو
\\
پس تو ای ادبار رو هم نان مخور
&&
تا نیفتی همچو او در شور و شر
\\
صد هزاران خلق نانها می‌خورند
&&
زور می‌یابند و جان می‌پرورند
\\
تو بدان نادر کجا افتاده‌ای
&&
گر نه محرومی و ابله زاده‌ای
\\
این جهان پر آفتاب و نور ماه
&&
او بهشته سر فرو برده به چاه
\\
که اگر حقست پس کو روشنی
&&
سر ز چه بردار و بنگر ای دنی
\\
جمله عالم شرق و غرب آن نور یافت
&&
تا تو در چاهی نخواهد بر تو تافت
\\
چه رها کن رو به ایوان و کروم
&&
کم ستیز اینجا بدان کاللج شوم
\\
هین مگو کاینک فلانی کشت کرد
&&
در فلان سالی ملخ کشتش بخورد
\\
پس چرا کارم که اینجا خوف هست
&&
من چرا افشانم این گندم ز دست
\\
و آنک او نگذاشت کشت و کار را
&&
پر کند کوری تو انبار را
\\
چون دری می‌کوفت او از سلوتی
&&
عاقبت در یافت روزی خلوتی
\\
جست از بیم عسس شب او به باغ
&&
یار خود را یافت چون شمع و چراغ
\\
گفت سازندهٔ سبب را آن نفس
&&
ای خدا تو رحمتی کن بر عسس
\\
ناشناسا تو سببها کرده‌ای
&&
از در دوزخ بهشتم برده‌ای
\\
بهر آن کردی سبب این کار را
&&
تا ندارم خوار من یک خار را
\\
در شکست پای بخشد حق پری
&&
هم ز قعر چاه بگشاید دری
\\
تو مبین که بر درختی یا به چاه
&&
تو مرا بین که منم مفتاح راه
\\
گر تو خواهی باقی این گفت و گو
&&
ای اخی در دفتر چارم بجو
\\
\end{longtable}
\end{center}
