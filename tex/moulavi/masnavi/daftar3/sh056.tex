\begin{center}
\section*{بخش ۵۶ - عذر گفتن نظم کننده و مدد خواستن}
\label{sec:sh056}
\addcontentsline{toc}{section}{\nameref{sec:sh056}}
\begin{longtable}{l p{0.5cm} r}
ای تقاضاگر درون همچون جنین
&&
چون تقاضا می‌کنی اتمام این
\\
سهل گردان ره نما توفیق ده
&&
یا تقاضا را بهل بر ما منه
\\
چون ز مفلس زر تقاضا می‌کنی
&&
زر ببخشش در سر ای شاه غنی
\\
بی تو نظم و قافیه شام و سحر
&&
زهره کی دارد که آید در نظر
\\
نظم و تجنیس و قوافی ای علیم
&&
بندهٔ امر توند از ترس و بیم
\\
چون مسبح کرده‌ای هر چیز را
&&
ذات بی تمییز و با تمییز را
\\
هر یکی تسبیح بر نوعی دگر
&&
گوید و از حال آن این بی‌خبر
\\
آدمی منکر ز تسبیح جماد
&&
و آن جماد اندر عبادت اوستاد
\\
بلک هفتاد و دو ملت هر یکی
&&
بی‌خبر از یکدگر واندر شکی
\\
چون دو ناطق را ز حال همدگر
&&
نیست آگه چون بود دیوار و در
\\
چون من از تسبیح ناطق غافلم
&&
چون بداند سبحهٔ صامت دلم
\\
هست سنی را یکی تسبیح خاص
&&
هست جبری را ضد آن در مناص
\\
سنی از تسبیح جبری بی‌خبر
&&
جبری از تسبیح سنی بی اثر
\\
این همی‌گوید که آن ضالست و گم
&&
بی‌خبر از حال او وز امر قم
\\
و آن همی گوید که این را چه خبر
&&
جنگشان افکند یزدان از قدر
\\
گوهر هر یک هویدا می‌کند
&&
جنس از ناجنس پیدا می‌کند
\\
قهر را از لطف داند هر کسی
&&
خواه دانا خواه نادان یا خسی
\\
لیک لطفی قهر در پنهان شده
&&
یا که قهری در دل لطف آمده
\\
کم کسی داند مگر ربانیی
&&
کش بود در دل محک جانیی
\\
باقیان زین دو گمانی می‌برند
&&
سوی لانهٔ خود به یک پر می‌پرند
\\
\end{longtable}
\end{center}
