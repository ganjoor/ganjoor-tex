\begin{center}
\section*{بخش ۱۹۳ - عشق جالینوس برین حیات دنیا بود کی هنر او همینجا بکار می‌آید هنری نورزیده است کی در آن بازار بکار آید آنجا خود را به عوام یکسان می‌بیند}
\label{sec:sh193}
\addcontentsline{toc}{section}{\nameref{sec:sh193}}
\begin{longtable}{l p{0.5cm} r}
آنچنانک گفت جالینوس راد
&&
از هوای این جهان و از مراد
\\
راضیم کز من بماند نیم جان
&&
که ز کون استری بینم جهان
\\
گربه می‌بیند بگرد خود قطار
&&
مرغش آیس گشته بودست از مطار
\\
یا عدم دیدست غیر این جهان
&&
در عدم نادیده او حشری نهان
\\
چون جنین کش می‌کشد بیرون کرم
&&
می‌گریزد او سپس سوی شکم
\\
لطف رویش سوی مصدر می‌کند
&&
او مقر در پشت مادر می‌کند
\\
که اگر بیرون فتم زین شهر و کام
&&
ای عجب بینم بدیده این مقام
\\
یا دری بودی در آن شهر وخم
&&
که نظاره کردمی اندر رحم
\\
یا چو چشمهٔ سوزنی راهم بدی
&&
که ز بیرونم رحم دیده شدی
\\
آن جنین هم غافلست از عالمی
&&
همچو جالینوس او نامحرمی
\\
اونداند کن رطوباتی که هست
&&
آن مدد از عالم بیرونیست
\\
آنچنانک چار عنصر در جهان
&&
صد مدد آرد ز شهر لامکان
\\
آب و دانه در قفس گر یافتست
&&
آن ز باغ و عرصه‌ای درتافتست
\\
جانهای انبیا بینند باغ
&&
زین قفس در وقت نقلان و فراغ
\\
پس ز جالینوس و عالم فارغند
&&
همچو ماه اندر فلکها بازغند
\\
ور ز جالینوس این گفت افتراست
&&
پس جوابم بهر جالینوس نیست
\\
این جواب آنکس آمد کین بگفت
&&
که نبودستش دل پر نور جفت
\\
مرغ جانش موش شد سوراخ‌جو
&&
چون شنید از گربگان او عرجوا
\\
زان سبب جانش وطن دید و قرار
&&
اندرین سوراخ دنیا موش‌وار
\\
هم درین سوراخ بنایی گرفت
&&
درخور سوراخ دانایی گرفت
\\
پیشه‌هایی که مرورا در مزید
&&
کاندرین سوراخ کار آید گزید
\\
زانک دل بر کند از بیرون شدن
&&
بسته شد راه رهیدن از بدن
\\
عنکبوت ار طبع عنقا داشتی
&&
از لعابی خیمه کی افراشتی
\\
گربه کرده چنگ خود اندر قفس
&&
نام چنگش درد و سرسام و مغص
\\
گربه مرگست و مرض چنگال او
&&
می‌زند بر مرغ و پر و بال او
\\
گوشه گوشه می‌جهد سوی دوا
&&
مرگ چون قاضیست و رنجوری گوا
\\
چون پیادهٔ قاضی آمد این گواه
&&
که همی‌خواند ترا تا حکم گاه
\\
مهلتی می‌خواهی از وی در گریز
&&
گر پذیرد شد و گرنه گفت خیز
\\
جستن مهلت دوا و چاره‌ها
&&
که زنی بر خرقهٔ تن پاره‌ها
\\
عاقبت آید صباحی خشم‌وار
&&
چند باشد مهلت آخر شرم دار
\\
عذر خود از شه بخواه ای پرحسد
&&
پیش از آنک آنچنان روزی رسد
\\
وانک در ظلمت براند بارگی
&&
برکند زان نور دل یکبارگی
\\
می‌گریزد از گوا و مقصدش
&&
کان گوا سوی قضا می‌خواندش
\\
\end{longtable}
\end{center}
