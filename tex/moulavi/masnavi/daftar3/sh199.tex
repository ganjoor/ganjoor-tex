\begin{center}
\section*{بخش ۱۹۹ - تمثیل صابر شدن ممن چون بر شر و خیر بلا واقف شود}
\label{sec:sh199}
\addcontentsline{toc}{section}{\nameref{sec:sh199}}
\begin{longtable}{l p{0.5cm} r}
سگ شکاری نیست او را طوق نیست
&&
خام و ناجوشیده جز بی‌ذوق نیست
\\
گفت نخود چون چنینست ای ستی
&&
خوش بجوشم یاریم ده راستی
\\
تو درین جوشش چو معمار منی
&&
کفچلیزم زن که بس خوش می‌زنی
\\
همچو پیلم بر سرم زن زخم و داغ
&&
تا نبینم خواب هندستان و باغ
\\
تا که خود را در دهم در جوش من
&&
تا رهی یابم در آن آغوش من
\\
زانک انسان در غنا طاغی شود
&&
همچو پیل خواب‌بین یاغی شود
\\
پیل چون در خواب بیند هند را
&&
پیلبان را نشنود آرد دغا
\\
\end{longtable}
\end{center}
