\begin{center}
\section*{بخش ۱۹۲ - جواب گفتن عاشق عاذلان را}
\label{sec:sh192}
\addcontentsline{toc}{section}{\nameref{sec:sh192}}
\begin{longtable}{l p{0.5cm} r}
گفت او ای ناصحان من بی ندم
&&
از جهان زندگی سیر آمدم
\\
منبلی‌ام زخم جو و زخم‌خواه
&&
عافیت کم جوی از منبل براه
\\
منبلی نی کو بود خود برگ‌جو
&&
منبلی‌ام لاابالی مرگ‌جو
\\
منبلی نی کو به کف پول آورد
&&
منبلی چستی کزین پل بگذرد
\\
آن نه کو بر هر دکانی بر زند
&&
بل جهد از کون و کانی بر زند
\\
مرگ شیرین گشت و نقلم زین سرا
&&
چون قفس هشتن پریدن مرغ را
\\
آن قفس که هست عین باغ در
&&
مرغ می‌بیند گلستان و شجر
\\
جوق مرغان از برون گرد قفس
&&
خوش همی‌خوانند ز آزادی قصص
\\
مرغ را اندر قفس زان سبزه‌زار
&&
نه خورش ماندست و نه صبر و قرار
\\
سر ز هر سوراخ بیرون می‌کند
&&
تا بود کین بند از پا برکند
\\
چون دل و جانش چنین بیرون بود
&&
آن قفس را در گشایی چون بود
\\
نه چنان مرغ قفس در اندهان
&&
گرد بر گردش به حلقه گربگان
\\
کی بود او را درین خوف و حزن
&&
آرزوی از قفس بیرون شدن
\\
او همی‌خواهد کزین ناخوش حصص
&&
صد قفس باشد بگرد این قفس
\\
\end{longtable}
\end{center}
