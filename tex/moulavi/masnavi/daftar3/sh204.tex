\begin{center}
\section*{بخش ۲۰۴ - بیان آنک رفتن انبیا و اولیا به کوهها و غارها جهت پنهان کردن خویش نیست و جهت خوف تشویش خلق نیست بلک جهت ارشاد خلق است و تحریض بر انقطاع از دنیا  به قدر ممکن}
\label{sec:sh204}
\addcontentsline{toc}{section}{\nameref{sec:sh204}}
\begin{longtable}{l p{0.5cm} r}
آنک گویند اولیا در که بوند
&&
تا ز چشم مردمان پنهان شوند
\\
پیش خلق ایشان فراز صد که‌اند
&&
گام خود بر چرخ هفتم می‌نهند
\\
پس چرا پنهان شود که‌جو بود
&&
کو ز صد دریا و که زان سو بود
\\
حاجتش نبود به سوی که گریخت
&&
کز پیش کرهٔ فلک صد نعل ریخت
\\
چرخ گردید و ندید او گرد جان
&&
تعزیت‌جامه بپوشید آسمان
\\
گر به ظاهر آن پری پنهان بود
&&
آدمی پنهان‌تر از پریان بود
\\
نزد عاقل زان پری که مضمرست
&&
آدمی صد بار خود پنهان‌ترست
\\
آدمی نزدیک عاقل چون خفیست
&&
چون بود آدم که در غیب او صفیست
\\
\end{longtable}
\end{center}
