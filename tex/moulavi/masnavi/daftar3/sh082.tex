\begin{center}
\section*{بخش ۸۲ - بقیهٔ حکایت نابینا و مصحف}
\label{sec:sh082}
\addcontentsline{toc}{section}{\nameref{sec:sh082}}
\begin{longtable}{l p{0.5cm} r}
مرد مهمان صبرکرد و ناگهان
&&
کشف گشتش حال مشکل در زمان
\\
نیم‌شب آواز قرآن را شنید
&&
جست از خواب آن عجایب را بدید
\\
که ز مصحف کور می‌خواندی درست
&&
گشت بی‌صبر و ازو آن حال جست
\\
گفت آیا ای عجب با چشم کور
&&
چون همی‌خوانی همی‌بینی سطور
\\
آنچ می‌خوانی بر آن افتاده‌ای
&&
دست را بر حرف آن بنهاده‌ای
\\
اصبعت در سیر پیدا می‌کند
&&
که نظر بر حرف داری مستند
\\
گفت ای گشته ز جهل تن جدا
&&
این عجب می‌داری از صنع خدا
\\
من ز حق در خواستم کای مستعان
&&
بر قرائت من حریصم همچو جان
\\
نیستم حافظ مرا نوری بده
&&
در دو دیده وقت خواندن بی‌گره
\\
باز ده دو دیده‌ام را آن زمان
&&
که بگیرم مصحف و خوانم عیان
\\
آمد از حضرت ندا کای مرد کار
&&
ای بهر رنجی به ما اومیدوار
\\
حسن ظنست و امیدی خوش ترا
&&
که ترا گوید بهر دم برتر آ
\\
هر زمان که قصد خواندن باشدت
&&
یا ز مصحفها قرائت بایدت
\\
من در آن دم وا دهم چشم ترا
&&
تا فرو خوانی معظم جوهرا
\\
همچنان کرد و هر آنگاهی که من
&&
وا گشایم مصحف اندر خواندن
\\
آن خبیری که نشد غافل ز کار
&&
آن گرامی پادشاه و کردگار
\\
باز بخشد بینشم آن شاه فرد
&&
در زمان همچون چراغ شب‌نورد
\\
زین سبب نبود ولی را اعتراض
&&
هرچه بستاند فرستد اعتیاض
\\
گر بسوزد باغت انگورت دهد
&&
در میان ماتمی سورت دهد
\\
آن شل بی‌دست را دستی دهد
&&
کان غمها را دل مستی دهد
\\
لا نسلم و اعتراض از ما برفت
&&
چون عوض می‌آید از مفقود زفت
\\
چونک بی آتش مرا گرمی رسد
&&
راضیم گر آتشش ما را کشد
\\
بی چراغی چون دهد او روشنی
&&
گر چراغت شد چه افغان می‌کنی
\\
\end{longtable}
\end{center}
