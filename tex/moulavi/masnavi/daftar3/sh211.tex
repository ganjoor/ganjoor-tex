\begin{center}
\section*{بخش ۲۱۱ - رسیدن بانگ طلسمی نیم‌شب مهمان مسجد را}
\label{sec:sh211}
\addcontentsline{toc}{section}{\nameref{sec:sh211}}
\begin{longtable}{l p{0.5cm} r}
بشنو اکنون قصهٔ آن بانگ سخت
&&
که نرفت از جا بدان آن نیکبخت
\\
گفت چون ترسم چو هست این طبل عید
&&
تا دهل ترسد که زخم او را رسید
\\
ای دهلهای تهی بی قلوب
&&
قسمتان از عید جان شد زخم چوب
\\
شد قیامت عید و بی‌دینان دهل
&&
ما چو اهل عید خندان همچو گل
\\
بشنو اکنون این دهل چون بانگ زد
&&
دیگ دولتبا چگونه می‌پزد
\\
چونک بشنود آن دهل آن مرد دید
&&
گفت چون ترسد دلم از طبل عید
\\
گفت با خود هین ملرزان دل کزین
&&
مرد جان بددلان بی‌یقین
\\
وقت آن آمد که حیدروار من
&&
ملک گیرم یا بپردازم بدن
\\
بر جهید و بانگ بر زد کای کیا
&&
حاضرم اینک اگر مردی بیا
\\
در زمان بشکست ز آواز آن طلسم
&&
زر همی‌ریزید هر سو قسم قسم
\\
ریخت چند این زر که ترسید آن پسر
&&
تا نگیرد زر ز پری راه در
\\
بعد از آن برخاست آن شیر عتید
&&
تا سحرگه زر به بیرون می‌کشید
\\
دفن می‌کرد و همی آمد بزر
&&
با جوال و توبره بار دگر
\\
گنجها بنهاد آن جانباز از آن
&&
کوری ترسانی واپس خزان
\\
این زر ظاهر بخاطر آمدست
&&
در دل هر کور دور زرپرست
\\
کودکان اسفالها را بشکنند
&&
نام زر بنهند و در دامن کنند
\\
اندر آن بازی چو گویی نام زر
&&
آن کند در خاطر کودک گذر
\\
بل زر مضروب ضرب ایزدی
&&
کو نگردد کاسد آمد سرمدی
\\
آن زری کین زر از آن زر تاب یافت
&&
گوهر و تابندگی و آب یافت
\\
آن زری که دل ازو گردد غنی
&&
غالب آید بر قمر در روشنی
\\
شمع بود آن مسجد و پروانه او
&&
خویشتن در باخت آن پروانه‌خو
\\
پر بسوخت او را ولیکن ساختش
&&
بس مبارک آمد آن انداختش
\\
همچو موسی بود آن مسعودبخت
&&
کاتشی دید او به سوی آن درخت
\\
چون عنایتها برو موفور بود
&&
نار می‌پنداشت و خود آن نور بود
\\
مرد حق را چون ببینی ای پسر
&&
تو گمان داری برو نار بشر
\\
تو ز خود می‌آیی و آن در تو است
&&
نار و خار ظن باطل این سو است
\\
او درخت موسی است و پر ضیا
&&
نور خوان نارش مخوان باری بیا
\\
نه فطام این جهان ناری نمود
&&
سالکان رفتند و آن خود نور بود
\\
پس بدان که شمع دین بر می‌شود
&&
این نه همچون شمع آتشها بود
\\
این نماید نور و سوزد یار را
&&
و آن بصورت نار و گل زوار را
\\
این چو سازنده ولی سوزنده‌ای
&&
و آن گه وصلت دل افروزنده‌ای
\\
شکل شعلهٔ نور پاک سازوار
&&
حاضران را نور و دوران را چو نار
\\
\end{longtable}
\end{center}
