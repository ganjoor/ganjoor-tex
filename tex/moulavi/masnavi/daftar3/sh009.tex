\begin{center}
\section*{بخش ۹ - قصهٔ اهل سبا و طاغی کردن نعمت ایشان را و  در رسیدن شومی طغیان و کفران در ایشان و بیان فضیلت شکر و وفا}
\label{sec:sh009}
\addcontentsline{toc}{section}{\nameref{sec:sh009}}
\begin{longtable}{l p{0.5cm} r}
تو نخواندی قصهٔ اهل سبا
&&
یا بخواندی و ندیدی جز صدا
\\
از صدا آن کوه خود آگاه نیست
&&
سوی معنی هوش که را راه نیست
\\
او همی بانگی کند بی گوش و هوش
&&
چون خمش کردی تو او هم شد خموش
\\
داد حق اهل سبا را بس فراغ
&&
صد هزاران قصر و ایوانها و باغ
\\
شکر آن نگزاردند آن بد رگان
&&
در وفا بودند کمتر از سگان
\\
مر سگی را لقمهٔ نانی ز در
&&
چون رسد بر در همی‌بندد کمر
\\
پاسبان و حارس در می‌شود
&&
گرچه بر وی جور و سختی می‌رود
\\
هم بر آن در باشدش باش و قرار
&&
کفر دارد کرد غیری اختیار
\\
ور سگی آید غریبی روز و شب
&&
آن سگانش می‌کنند آن دم ادب
\\
که برو آنجا که اول منزلست
&&
حق آن نعمت گروگان دلست
\\
می‌گزندش که برو بر جای خویش
&&
حق آن نعمت فرو مگذار بیش
\\
از در دل و اهل دل آب حیات
&&
چند نوشیدی و وا شد چشمهات
\\
بس غذای سکر و وجد و بی‌خودی
&&
از در اهل دلان بر جان زدی
\\
باز این در را رها کردی ز حرص
&&
گرد هر دکان همی‌گردی ز حرص
\\
بر در آن منعمان چرب‌دیگ
&&
می‌دوی بهر ثرید مردریگ
\\
چربش اینجا دان که جان فربه شود
&&
کار نااومید اینجا به شود
\\
\end{longtable}
\end{center}
