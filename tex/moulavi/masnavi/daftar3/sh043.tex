\begin{center}
\section*{بخش ۴۳ - مهلت دادن موسی علیه‌السلام فرعون را  تا ساحران را جمع کند از مداین}
\label{sec:sh043}
\addcontentsline{toc}{section}{\nameref{sec:sh043}}
\begin{longtable}{l p{0.5cm} r}
گفت امر آمد برو مهلت ترا
&&
من بجای خود شدم رستی ز ما
\\
او همی‌شد و اژدها اندر عقب
&&
چون سگ صیاد دانا و محب
\\
چون سگ صیاد جنبان کرده دم
&&
سنگ را می‌کرد ریگ او زیر سم
\\
سنگ و آهن را بدم در می‌کشید
&&
خرد می‌خایید آهن را پدید
\\
در هوا می‌کرد خود بالای برج
&&
که هزیمت می‌شد از وی روم و گرج
\\
کفک می‌انداخت چون اشتر ز کام
&&
قطره‌ای بر هر که زد می‌شد جذام
\\
ژغژغ دندان او دل می‌شکست
&&
جان شیران سیه می‌شد ز دست
\\
چون به قوم خود رسید آن مجتبی
&&
شدق او بگرفت باز او شد عصا
\\
تکیه بر وی کرد و می‌گفت ای عجب
&&
پیش ما خورشید و پیش خصم شب
\\
ای عجب چون می‌نبیند این سپاه
&&
عالمی پر آفتاب چاشتگاه
\\
چشم باز و گوش باز و این ذکا
&&
خیره‌ام در چشم‌بندی خدا
\\
من ازیشان خیره ایشان هم ز من
&&
از بهاری خار ایشان من سمن
\\
پیششان بردم بسی جام رحیق
&&
سنگ شد آبش به پیش این فریق
\\
دسته گل بستم و بردم به پیش
&&
هر گلی چون خار گشت و نوش نیش
\\
آن نصیب جان بی‌خویشان بود
&&
چونک با خویش‌اند پیدا کی شود
\\
خفتهٔ بیدار باید پیش ما
&&
تا به بیداری ببیند خوابها
\\
دشمن این خواب خوش شد فکر خلق
&&
تا نخسپد فکرتش بستست حلق
\\
حیرتی باید که روبد فکر را
&&
خورده حیرت فکر را و ذکر را
\\
هر که کاملتر بود او در هنر
&&
او بمعنی پس بصورت پیشتر
\\
راجعون گفت و رجوع این سان بود
&&
که گله وا گردد و خانه رود
\\
چونک واگردید گله از ورود
&&
پس فتد آن بز که پیش آهنگ بود
\\
پیش افتد آن بز لنگ پسین
&&
اضحک الرجعی وجوه العابسین
\\
از گزافه کی شدند این قوم لنگ
&&
فخر را دادند و بخریدند ننگ
\\
پا شکسته می‌روند این قوم حج
&&
از حرج راهیست پنهان تا فرج
\\
دل ز دانشها بشستند این فریق
&&
زانک این دانش نداند آن طریق
\\
دانشی باید که اصلش زان سرست
&&
زانک هر فرعی به اصلش رهبرست
\\
هر پری بر عرض دریا کی پرد
&&
تا لدن علم لدنی می‌برد
\\
پس چرا علمی بیاموزی به مرد
&&
کش بباید سینه را زان پاک کرد
\\
پس مجو پیشی ازین سر لنگ باش
&&
وقت وا گشتن تو پیش آهنگ باش
\\
آخرون السابقون باش ای ظریف
&&
بر شجر سابق بود میوهٔ طریف
\\
گرچه میوه آخر آید در وجود
&&
اولست او زانک او مقصود بود
\\
چون ملایک گوی لا علم لنا
&&
تا بگیرد دست تو علمتنا
\\
گر درین مکتب ندانی تو هجا
&&
همچو احمد پری از نور حجی
\\
گر نباشی نامدار اندر بلاد
&&
گم نه‌ای الله اعلم بالعباد
\\
اندر آن ویران که آن معروف نیست
&&
از برای حفظ گنجینهٔ زریست
\\
موضع معروف کی بنهند گنج
&&
زین قبل آمد فرج در زیر رنج
\\
خاطر آرد بس شکال اینجا ولیک
&&
بسکلد اشکال را استور نیک
\\
هست عشقش آتشی اشکال‌سوز
&&
هر خیالی را بروبد نور روز
\\
هم از آن سو جو جواب ای مرتضا
&&
کین سؤال آمد از آن سو مر ترا
\\
گوشهٔ بی گوشهٔ دل شه‌رهیست
&&
تاب لا شرقی و لا غرب از مهیست
\\
تو ازین سو و از آن سو چون گدا
&&
ای که معنی چه می‌جویی صدا
\\
هم از آن سو جو که وقت درد تو
&&
می‌شوی در ذکر یا ربی دوتو
\\
وقت درد و مرگ آن سو می‌نمی
&&
چونک دردت رفت چونی اعجمی
\\
وقت محنت گشته‌ای الله گو
&&
چونک محنت رفت گویی راه کو
\\
این از آن آمد که حق را بی گمان
&&
هر که بشناسد بود دایم بر آن
\\
وانک در عقل و گمان هستش حجاب
&&
گاه پوشیدست و گه بدریده جیب
\\
عقل جزوی گاه چیره گه نگون
&&
عقل کلی آمن از ریب المنون
\\
عقل بفروش و هنر حیرت بخر
&&
رو به خواری نه بخارا ای پسر
\\
ما چه خود را در سخن آغشته‌ایم
&&
کز حکایت ما حکایت گشته‌ایم
\\
من عدم و افسانه گردم در حنین
&&
تا تقلب یابم اندر ساجدین
\\
این حکایت نیست پیش مرد کار
&&
وصف حالست و حضور یار غار
\\
آن اساطیر اولین که گفت عاق
&&
حرف قرآن را بد آثار نفاق
\\
لامکانی که درو نور خداست
&&
ماضی و مستقبل و حال از کجاست
\\
ماضی و مستقبلش نسبت به تست
&&
هر دو یک چیزند پنداری که دوست
\\
یک تنی او را پدر ما را پسر
&&
بام زیر زید و بر عمرو آن زبر
\\
نسبت زیر و زبر شد زان دو کس
&&
سقف سوی خویش یک چیزست بس
\\
نیست مثل آن مثالست این سخن
&&
قاصر از معنی نو حرف کهن
\\
چون لب جو نیست مشکا لب ببند
&&
بی لب و ساحل بدست این بحر قند
\\
\end{longtable}
\end{center}
