\begin{center}
\section*{بخش ۱۸۵ - رو نهادن آن بندهٔ عاشق سوی بخارا}
\label{sec:sh185}
\addcontentsline{toc}{section}{\nameref{sec:sh185}}
\begin{longtable}{l p{0.5cm} r}
رو نهاد آن عاشق خونابه‌ریز
&&
دل‌طپان سوی بخارا گرم و تیز
\\
ریگ آمون پیش او همچون حریر
&&
آب جیحون پیش او چون آبگیر
\\
آن بیابان پیش او چون گلستان
&&
می‌فتاد از خنده او چون گل‌ستان
\\
در سمرقندست قند اما لبش
&&
از بخارا یافت و آن شد مذهبش
\\
ای بخارا عقل‌افزا بوده‌ای
&&
لیکن ازمن عقل و دین بربوده‌ای
\\
بدر می‌جویم از آنم چون هلال
&&
صدر می‌جویم درین صف نعال
\\
چون سواد آن بخارا را بدید
&&
در سواد غم بیاضی شد پدید
\\
ساعتی افتاد بیهوش و دراز
&&
عقل او پرید در بستان راز
\\
بر سر و رویش گلابی می‌زدند
&&
از گلاب عشق او غافل بدند
\\
او گلستانی نهانی دیده بود
&&
غارت عشقش ز خود ببریده بود
\\
تو فسرده درخور این دم نه‌ای
&&
با شکر مقرون نه‌ای گرچه نیی
\\
رخت عقلت با توست و عاقلی
&&
کز جنودا لم تروها غافلی
\\
\end{longtable}
\end{center}
