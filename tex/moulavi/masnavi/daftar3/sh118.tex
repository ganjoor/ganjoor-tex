\begin{center}
\section*{بخش ۱۱۸ - قصهٔ اهل سبا و حماقت ایشان و اثر ناکردن  نصیحت انبیا در احمقان}
\label{sec:sh118}
\addcontentsline{toc}{section}{\nameref{sec:sh118}}
\begin{longtable}{l p{0.5cm} r}
یادم آمد قصهٔ اهل سبا
&&
کز دم احمق صباشان شد وبا
\\
آن سبا ماند به شهر بس کلان
&&
در فسانه بشنوی از کودکان
\\
کودکان افسانه‌ها می‌آورند
&&
درج در افسانه‌شان بس سر و پند
\\
هزلها گویند در افسانه‌ها
&&
گنج می‌جو در همه ویرانه‌ها
\\
بود شهری بس عظیم و مه ولی
&&
قدر او قدر سکره بیش نی
\\
بس عظیم و بس فراخ و بس دراز
&&
سخت زفت زفت اندازهٔ پیاز
\\
مردم ده شهر مجموع اندرو
&&
لیک جمله سه تن ناشسته‌رو
\\
اندرو خلق و خلایق بی‌شمار
&&
لیک آن جمله سه خام پخته‌خوار
\\
جان ناکرده به جانان تاختن
&&
گر هزارانست باشد نیم تن
\\
آن یکی بس دور بین و دیده‌کور
&&
از سلیمان کور و دیده پای مور
\\
و آن دگر بس تیزگوش و سخت کر
&&
گنج و در وی نیست یک جو سنگ زر
\\
وآن دگر عور و برهنه لاشه‌باز
&&
لیک دامنهای جامهٔ او دراز
\\
گفت کور اینک سپاهی می‌رسند
&&
من همی‌بینم که چه قومند و چند
\\
گفت کر آری شنودم بانگشان
&&
که چه می‌گویند پیدا و نهان
\\
آن برهنه گفت ترسان زین منم
&&
که ببرند از درازی دامنم
\\
کور گفت اینک به نزدیک آمدند
&&
خیز بگریزیم پیش از زخم و بند
\\
کر همی‌گوید که آری مشغله
&&
می‌شود نزدیکتر یاران هله
\\
آن برهنه گفت آوه دامنم
&&
از طمع برند و من ناآمنم
\\
شهر را هشتند و بیرون آمدند
&&
در هزیمت در دهی اندر شدند
\\
اندر آن ده مرغ فربه یافتند
&&
لیک ذرهٔ گوشت بر وی نه نژند
\\
مرغ مردهٔ خشک وز زخم کلاغ
&&
استخوانها زار گشته چون پناغ
\\
زان همی‌خوردند چون از صید شیر
&&
هر یکی از خوردنش چون پیل سیر
\\
هر سه زان خوردند و بس فربه شدند
&&
چون سه پیل بس بزرگ و مه شدند
\\
آنچنان کز فربهی هر یک جوان
&&
در نگنجیدی ز زفتی در جهان
\\
با چنین گبزی و هفت اندام زفت
&&
از شکاف در برون جستند و رفت
\\
راه مرگ خلق ناپیدا رهیست
&&
در نظر ناید که آن بی‌جا رهیست
\\
نک پیاپی کاروانها مقتفی
&&
زین شکاف در که هست آن مختفی
\\
بر در ار جویی نیابی آن شکاف
&&
سخت ناپیدا و زو چندین زفاف
\\
\end{longtable}
\end{center}
