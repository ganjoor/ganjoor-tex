\begin{center}
\section*{بخش ۱۴ - قصهٔ اهل ضروان و حیلت کردن ایشان تا بی زحمت درویشان باغها را قطاف کنند}
\label{sec:sh014}
\addcontentsline{toc}{section}{\nameref{sec:sh014}}
\begin{longtable}{l p{0.5cm} r}
قصهٔ اصحاب ضروان خوانده‌ای
&&
پس چرا در حیله‌جویی مانده‌ای
\\
حیله می‌کردند کزدم‌نیش چند
&&
که برند از روزی درویش چند
\\
شب همه شب می‌سگالیدند مکر
&&
روی در رو کرده چندین عمرو و بکر
\\
خفیه می‌گفتند سرها آن بدان
&&
تا نباید که خدا در یابد آن
\\
با گل انداینده اسگالید گل
&&
دست کاری می‌کند پنهان ز دل
\\
گفت الا یعلم هواک من خلق
&&
ان فی نجواک صدقا ام ملق
\\
گفت یغفل عن ظعین قد غدا
&&
من یعاین این مثواه غدا
\\
اینما قد هبطا او صعدا
&&
قد تولاه و احصی عددا
\\
گوش را اکنون ز غفلت پاک کن
&&
استماع هجر آن غمناک کن
\\
آن زکاتی دان که غمگین را دهی
&&
گوش را چون پیش دستانش نهی
\\
بشنوی غمهای رنجوران دل
&&
فاقهٔ جان شریف از آب و گل
\\
خانهٔ پر دود دارد پر فنی
&&
مر ورا بگشا ز اصغا روزنی
\\
گوش تو او را چو راه دم شود
&&
دود تلخ از خانهٔ او کم شود
\\
غمگساری کن تو با ما ای روی
&&
گر به سوی رب اعلی می‌روی
\\
این تردد حبس و زندانی بود
&&
که بنگذارد که جان سویی رود
\\
این بدین سو آن بدان سو می‌کشد
&&
هر یکی گویا منم راه رشد
\\
این تردد عقبهٔ راه حقست
&&
ای خنک آن را که پایش مطلقست
\\
بی‌تردد می‌رود در راه راست
&&
ره نمی‌دانی بجو گامش کجاست
\\
گام آهو را بگیر و رو معاف
&&
تا رسی از گام آهو تا بناف
\\
زین روش بر اوج انور می‌روی
&&
ای برادر گر بر آذر می‌روی
\\
نه ز دریا ترس نه از موج و کف
&&
چون شنیدی تو خطاب لا تخف
\\
لا تخف دان چونک خوفت داد حق
&&
نان فرستد چون فرستادت طبق
\\
خوف آن کس راست کو را خوف نیست
&&
غصهٔ آن کس را کش اینجا طوف نیست
\\
\end{longtable}
\end{center}
