\begin{center}
\section*{بخش ۳ - بقیهٔ قصهٔ متعرضان پیل‌بچگان}
\label{sec:sh003}
\addcontentsline{toc}{section}{\nameref{sec:sh003}}
\begin{longtable}{l p{0.5cm} r}
هر دهان را پیل بویی می‌کند
&&
گرد معدهٔ هر بشر بر می‌تند
\\
تا کجا یابد کباب پور خویش
&&
تا نماید انتقام و زور خویش
\\
گوشتهای بندگان حق خوری
&&
غیبت ایشان کنی کیفر بری
\\
هان که بویای دهانتان خالقست
&&
کی برد جان غیر آن کو صادقست
\\
وای آن افسوسیی کش بوی‌گیر
&&
باشد اندر گور منکر یا نکیر
\\
نه دهان دزدیدن امکان زان مهان
&&
نه دهان خوش کردن از دارودهان
\\
آب و روغن نیست مر روپوش را
&&
راه حیلت نیست عقل و هوش را
\\
چند کوبد زخمهای گرزشان
&&
بر سر هر ژاژخا و مرزشان
\\
گرز عزرائیل را بنگر اثر
&&
گر نبینی چوب و آهن در صور
\\
هم بصورت می‌نماید گه گهی
&&
زان همان رنجور باشد آگهی
\\
گوید آن رنجور ای یاران من
&&
چیست این شمشیر بر ساران من
\\
ما نمی‌بینیم باشد این خیال
&&
چه خیالست این که این هست ارتحال
\\
چه خیالست این که این چرخ نگون
&&
از نهیب این خیالی شد کنون
\\
گرزها و تیغها محسوس شد
&&
پیش بیمار و سرش منکوس شد
\\
او همی‌بیند که آن از بهر اوست
&&
چشم دشمن بسته زان و چشم دوست
\\
حرص دنیا رفت و چشمش تیز شد
&&
چشم او روشن گه خون‌ریز شد
\\
مرغ بی‌هنگام شد آن چشم او
&&
از نتیجهٔ کبر او و خشم او
\\
سر بریدن واجب آید مرغ را
&&
کو بغیر وقت جنباند درا
\\
هر زمان نزعیست جزو جانت را
&&
بنگر اندر نزع جان ایمانت را
\\
عمر تو مانند همیان زرست
&&
روز و شب مانند دینار اشمرست
\\
می‌شمارد می‌دهد زر بی وقوف
&&
تا که خالی گردد و آید خسوف
\\
گر ز که بستانی و ننهی بجای
&&
اندر آید کوه زان دادن ز پای
\\
پس بنه بر جای هر دم را عوض
&&
تا ز واسجد واقترب یابی غرض
\\
در تمامی کارها چندین مکوش
&&
جز به کاری که بود در دین مکوش
\\
عاقبت تو رفت خواهی ناتمام
&&
کارهاات ابتر و نان تو خام
\\
وان عمارت کردن گور و لحد
&&
نه به سنگست و به چوب و نه لبد
\\
بلک خود را در صفا گوری کنی
&&
در منی او کنی دفن منی
\\
خاک او گردی و مدفون غمش
&&
تا دمت یابد مددها از دمش
\\
گورخانه و قبه‌ها و کنگره
&&
نبود از اصحاب معنی آن سره
\\
بنگر اکنون زنده اطلس‌پوش را
&&
هیچ اطلس دست گیرد هوش را
\\
در عذاب منکرست آن جان او
&&
گزدم غم دل دل غمدان او
\\
از برون بر ظاهرش نقش و نگار
&&
وز درون ز اندیشه‌ها او زار زار
\\
و آن یکی بینی در آن دلق کهن
&&
چون نبات اندیشه و شکر سخن
\\
\end{longtable}
\end{center}
