\begin{center}
\section*{بخش ۱۷۲ - تشبیه نص با قیاس}
\label{sec:sh172}
\addcontentsline{toc}{section}{\nameref{sec:sh172}}
\begin{longtable}{l p{0.5cm} r}
نص وحی روح قدسی دان یقین
&&
وان قیاس عقل جزوی تحت این
\\
عقل از جان گشت با ادراک و فر
&&
روح او را کی شود زیر نظر
\\
لیک جان در عقل تاثیری کند
&&
زان اثر آن عقل تدبیری کند
\\
نوح‌وار ار صدقی زد در تو روح
&&
کو یم و کشتی و کو طوفان نوح
\\
عقل اثر را روح پندارد ولیک
&&
نور خور از قرص خور دورست نیک
\\
زان به قرصی سالکی خرسند شد
&&
تا ز نورش سوی قرص افکند شد
\\
زانک این نوری که اندر سافل است
&&
نیست دایم روز و شب او آفل است
\\
وانک اندر قرص دارد باش و جا
&&
غرقهٔ آن نور باشد دایما
\\
نه سحابش ره زند خود نه غروب
&&
وا رهید او از فراق سینه کوب
\\
این‌چنین کس اصلش از افلاک بود
&&
یا مبدل گشت گر از خاک بود
\\
زانک خاکی را نباشد تاب آن
&&
که زند بر وی شعاعش جاودان
\\
گر زند بر خاک دایم تاب خور
&&
آنچنان سوزد که ناید زو ثمر
\\
دایم اندر آب کار ماهی است
&&
مار را با او کجا همراهی است
\\
لیک در که مارهای پر فن‌اند
&&
اندرین یم ماهییها می‌کنند
\\
مکرشان گر خلق را شیدا کند
&&
هم ز دریا تاسه‌شان رسوا کند
\\
واندرین یم ماهیان پر فن‌اند
&&
مار را از سحر ماهی می‌کنند
\\
ماهیان قعر دریای جلال
&&
بحرشان آموخته سحر حلال
\\
بس محال از تاب ایشان حال شد
&&
نحس آنجا رفت و نیکوفال شد
\\
تا قیامت گر بگویم زین کلام
&&
صد قیامت بگذرد وین ناتمام
\\
\end{longtable}
\end{center}
