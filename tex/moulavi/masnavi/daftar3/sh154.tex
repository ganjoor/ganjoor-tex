\begin{center}
\section*{بخش ۱۵۴ - وجه عبرت گرفتن ازین حکایت و یقین دانستن کی ان مع العسر یسرا}
\label{sec:sh154}
\addcontentsline{toc}{section}{\nameref{sec:sh154}}
\begin{longtable}{l p{0.5cm} r}
عبرتست آن قصه ای جان مر ترا
&&
تا که راضی باشی در حکم خدا
\\
تا که زیرک باشی و نیکوگمان
&&
چون ببینی واقعهٔ بد ناگهان
\\
دیگران گردند زرد از بیم آن
&&
تو چو گل خندان گه سود و زیان
\\
زانک گل گر برگ برگش می‌کنی
&&
خنده نگذارد نگردد منثنی
\\
گوید از خاری چرا افتم بغم
&&
خنده را من خود ز خار آورده‌ام
\\
هرچه از تو یاوه گردد از قضا
&&
تو یقین دان که خریدت از بلا
\\
ما التصوف قال وجدان الفرح
&&
فی الفؤاد عند اتیان الترح
\\
آن عقابش را عقابی دان که او
&&
در ربود آن موزه را زان نیک‌خو
\\
تا رهاند پاش را از زخم مار
&&
ای خنک عقلی که باشد بی غبار
\\
گفت لا تاسوا علی ما فاتکم
&&
ان اتی السرحان واردی شاتکم
\\
کان بلا دفع بلاهای بزرگ
&&
و آن زیان منع زیانهای سترگ
\\
\end{longtable}
\end{center}
