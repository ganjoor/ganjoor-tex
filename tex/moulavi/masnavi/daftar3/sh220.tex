\begin{center}
\section*{بخش ۲۲۰ - آگاه شدن پیغامبر علیه السلام از طعن ایشان بر شماتت او}
\label{sec:sh220}
\addcontentsline{toc}{section}{\nameref{sec:sh220}}
\begin{longtable}{l p{0.5cm} r}
گرچه نشنید آن موکل آن سخن
&&
رفت در گوشی که آن بد من لدن
\\
بوی پیراهان یوسف را ندید
&&
آنک حافظ بود و یعقوبش کشید
\\
آن شیاطین بر عنان آسمان
&&
نشنوند آن سر لوح غیب‌دان
\\
آن محمد خفته و تکیه زده
&&
آمده سر گرد او گردان شده
\\
او خورد حلوا که روزیشست باز
&&
آن نه کانگشتان او باشد دراز
\\
نجم ثاقب گشته حارس دیوران
&&
که بهل دزدی ز احمد سر ستان
\\
ای دویده سوی دکان از پگاه
&&
هین به مسجد رو بجو رزق اله
\\
پس رسول آن گفتشان را فهم کرد
&&
گفت آن خنده نبودم از نبرد
\\
مرده‌اند ایشان و پوسیدهٔ فنا
&&
مرده کشتن نیست مردی پیش ما
\\
خود کیند ایشان که مه گردد شکاف
&&
چونک من پا بفشرم اندر مصاف
\\
آنگهی کآزاد بودیت و مکین
&&
مر شما را بسته می‌دیدم چنین
\\
ای بنازیده به ملک و خاندان
&&
نزد عاقل اشتری بر ناودان
\\
نقش تن را تا فتاد از بام طشت
&&
پیش چشمم کل آت آت گشت
\\
بنگرم در غوره می بینم عیان
&&
بنگرم در نیست شی بینم عیان
\\
بنگرم سر عالمی بینم نهان
&&
آدم و حوا نرسته از جهان
\\
مر شما را وقت ذرات الست
&&
دیده‌ام پا بسته و منکوس و پست
\\
از حدوث آسمان بی عمد
&&
آنچ دانسته بدم افزون نشد
\\
من شما را سرنگون می‌دیده‌ام
&&
پیش از آن کز آب و گل بالیده‌ام
\\
نو ندیدم تا کنم شادی بدان
&&
این همی‌دیدم در آن اقبالتان
\\
بستهٔ قهر خفی وانگه چه قهر
&&
قند می‌خوردید و در وی درج زهر
\\
این چنین قندی پر از زهر ار عدو
&&
خوش بنوشد چت حسد آید برو
\\
با نشاط آن زهر می‌کردید نوش
&&
مرگتان خفیه گرفته هر دو گوش
\\
من نمی‌کردم غزا از بهر آن
&&
تا ظفر یابم فرو گیرم جهان
\\
کین جهان جیفه‌ست و مردار و رخیص
&&
بر چنین مردار چون باشم حریص
\\
سگ نیم تا پرچم مرده کنم
&&
عیسی‌ام آیم که تا زنده‌ش کنم
\\
زان همی‌کردم صفوف جنگ چاک
&&
تا رهانم مر شما را از هلاک
\\
زان نمی‌برم گلوهای بشر
&&
تا مرا باشد کر و فر و حشر
\\
زان همی‌برم گلویی چند تا
&&
زان گلوها عالمی یابد رها
\\
که شما پروانه‌وار از جهل خویش
&&
پیش آتش می‌کنید این حمله کیش
\\
من همی‌رانم شما را همچو مست
&&
از در افتادن در آتش با دو دست
\\
آنک خود را فتحها پنداشتید
&&
تخم منحوسی خود می‌کاشتید
\\
یکدگر را جد جد می‌خواندید
&&
سوی اژدرها فرس می‌راندید
\\
قهر می‌کردید و اندر عین قهر
&&
خود شما مقهور قهر شیر دهر
\\
\end{longtable}
\end{center}
