\begin{center}
\section*{بخش ۱۴۴ - نومید شدن انبیا از قبول و پذیرای منکران قوله حتی اذا استیاس الرسل}
\label{sec:sh144}
\addcontentsline{toc}{section}{\nameref{sec:sh144}}
\begin{longtable}{l p{0.5cm} r}
انبیا گفتند با خاطر که چند
&&
می‌دهیم این را و آن را وعظ و پند
\\
چند کوبیم آهن سردی ز غی
&&
در دمیدن در قفض هین تا بکی
\\
جنبش خلق از قضا و وعده است
&&
تیزی دندان ز سوز معده است
\\
نفس اول راند بر نفس دوم
&&
ماهی از سر گنده باشد نه ز دم
\\
لیک هم می‌دان و خر می‌ران چو تیر
&&
چونک بلغ گفت حق شد ناگزیر
\\
تو نمی‌دانی کزین دو کیستی
&&
جهد کن چندانک بینی چیستی
\\
چون نهی بر پشت کشتی بار را
&&
بر توکل می‌کنی آن کار را
\\
تو نمی‌دانی که از هر دو کیی
&&
غرقه‌ای اندر سفر یا ناجیی
\\
گر بگویی تا ندانم من کیم
&&
بر نخواهم تاخت در کشتی و یم
\\
من درین ره ناجیم یا غرقه‌ام
&&
کشف گردان کز کدامین فرقه‌ام
\\
من نخواهم رفت این ره با گمان
&&
بر امید خشک همچون دیگران
\\
هیچ بازرگانیی ناید ز تو
&&
زانک در غیبست سر این دو رو
\\
تاجر ترسنده‌طبع شیشه‌جان
&&
در طلب نه سود دارد نه زیان
\\
بل زیان دارد که محرومست و خوار
&&
نور او یابد که باشد شعله‌خوار
\\
چونک بر بوکست جمله کارها
&&
کار دین اولی کزین یابی رها
\\
نیست دستوری بدینجا قرع باب
&&
جز امید الله اعلم بالصواب
\\
\end{longtable}
\end{center}
