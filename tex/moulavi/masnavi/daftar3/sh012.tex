\begin{center}
\section*{بخش ۱۲ - بقیهٔ داستان رفتن خواجه به دعوت روستایی سوی ده}
\label{sec:sh012}
\addcontentsline{toc}{section}{\nameref{sec:sh012}}
\begin{longtable}{l p{0.5cm} r}
شد ز حد هین باز گرد ای یار گرد
&&
روستایی خواجه را بین خانه برد
\\
قصهٔ اهل سبا یک گوشه نه
&&
آن بگو کان خواجه چون آمد به ده
\\
روستایی در تملق شیوه کرد
&&
تا که حزم خواجه را کالیوه کرد
\\
از پیام اندر پیام او خیره شد
&&
تا زلال حزم خواجه تیره شد
\\
هم ازینجا کودکانش در پسند
&&
نرتع و نلعب بشادی می‌زدند
\\
همچو یوسف کش ز تقدیر عجب
&&
نرتع و نلعب ببرد از ظل آب
\\
آن نه بازی بلک جانبازیست آن
&&
حیله و مکر و دغاسازیست آن
\\
هرچه از یارت جدا اندازد آن
&&
مشنو آن را کان زیان دارد زیان
\\
گر بود آن سود صد در صد مگیر
&&
بهر زر مگسل ز گنجور ای فقیر
\\
این شنو که چند یزدان زجر کرد
&&
گفت اصحاب نبی را گرم و سرد
\\
زانک بر بانگ دهل در سال تنگ
&&
جمعه را کردند باطل بی درنگ
\\
تا نباید دیگران ارزان خرند
&&
زان جلب صرفه ز ما ایشان برند
\\
ماند پیغامبر بخلوت در نماز
&&
با دو سه درویش ثابت پر نیاز
\\
گفت طبل و لهو و بازرگانیی
&&
چونتان ببرید از ربانیی
\\
قد فضضتم نحو قمح هائما
&&
ثم خلیتم نبیا قائما
\\
بهر گندم تخم باطل کاشتید
&&
و آن رسول حق را بگذاشتید
\\
صحبت او خیر من لهوست و مال
&&
بین کرا بگذاشتی چشمی بمال
\\
خود نشد حرص شما را این یقین
&&
که منم رزاق و خیر الرازقین
\\
آنک گندم را ز خود روزی دهد
&&
کی توکلهات را ضایع نهد
\\
از پی گندم جدا گشتی از آن
&&
که فرستادست گندم ز آسمان
\\
\end{longtable}
\end{center}
