\begin{center}
\section*{بخش ۲۲۱ - بیان آنک طاغی در عین قاهری مقهورست و در عین منصوری ماسور}
\label{sec:sh221}
\addcontentsline{toc}{section}{\nameref{sec:sh221}}
\begin{longtable}{l p{0.5cm} r}
دزد قهرخواجه کرد و زر کشید
&&
او بدان مشغول خود والی رسید
\\
گر ز خواجه آن زمان بگریختی
&&
کی برو والی حشر انگیختی
\\
قاهری دزد مقهوریش بود
&&
زانک قهر او سر او را ربود
\\
غالبی بر خواجه دام او شود
&&
تا رسد والی و بستاند قود
\\
ای که تو بر خلق چیره گشته‌ای
&&
در نبرد و غالبی آغشته‌ای
\\
آن به قاصد منهزم کردستشان
&&
تا ترا در حلقه می‌آرد کشان
\\
هین عنان در کش پی این منهزم
&&
در مران تا تو نگردی منخزم
\\
چون کشانیدت بدین شیوه به دام
&&
حمله بینی بعد از آن اندر زحام
\\
عقل ازین غالب شدن کی گشت شاد
&&
چون درین غالب شدن دید او فساد
\\
تیزچشم آمد خرد بینای پیش
&&
که خدایش سرمه کرد از کحل خویش
\\
گفت پیغامبر که هستند از فنون
&&
اهل جنت در خصومتها زبون
\\
از کمال حزم و سؤ الظن خویش
&&
نه ز نقص و بد دلی و ضعف کیش
\\
در فره دادن شنیده در کمون
&&
حکمت لولا رجال مومنون
\\
دست‌کوتاهی ز کفار لعین
&&
فرض شد بهر خلاص مؤمنین
\\
قصهٔ عهد حدیبیه بخوان
&&
کف ایدیکم تمامت زان بدان
\\
نیز اندر غالبی هم خویش را
&&
دید او مغلوب دام کبریا
\\
زان نمی‌خندم من از زنجیرتان
&&
که بکردم ناگهان شبگیرتان
\\
زان همی‌خندم که با زنجیر و غل
&&
می‌کشمتان سوی سروستان و گل
\\
ای عجب کز آتش بی‌زینهار
&&
بسته می‌آریمتان تا سبزه‌زار
\\
از سوی دوزخ به زنجیر گران
&&
می‌کشمتان تا بهشت جاودان
\\
هر مقلد را درین ره نیک و بد
&&
همچنان بسته به حضرت می‌کشد
\\
جمله در زنجیر بیم و ابتلا
&&
می‌روند این ره بغیر اولیا
\\
می‌کشند این راه را بیگاروار
&&
جز کسانی واقف از اسرار کار
\\
جهد کن تا نور تو رخشان شود
&&
تا سلوک و خدمتت آسان شود
\\
کودکان را می‌بری مکتب به زور
&&
زانک هستند از فواید چشم‌کور
\\
چون شود واقف به مکتب می‌دود
&&
جانش از رفتن شکفته می‌شود
\\
می‌رود کودک به مکتب پیچ پیچ
&&
چون ندید از مزد کار خویش هیچ
\\
چون کند در کیسه دانگی دست‌مزد
&&
آنگهان بی‌خواب گردد شب چو دزد
\\
جهد کن تا مزد طاعت در رسد
&&
بر مطیعان آنگهت آید حسد
\\
ائتیا کرها مقلد گشته را
&&
ائتیا طوعا صفا بسرشته را
\\
این محب حق ز بهر علتی
&&
و آن دگر را بی غرض خود خلتی
\\
این محب دایه لیک از بهر شیر
&&
و آن دگر دل داده بهر این ستیر
\\
طفل را از حسن او آگاه نه
&&
غیر شیر او را ازو دلخواه نه
\\
و آن دگر خود عاشق دایه بود
&&
بی غرض در عشق یک‌رایه بود
\\
پس محب حق باومید و بترس
&&
دفتر تقلید می‌خواند بدرس
\\
و آن محب حق ز بهر حق کجاست
&&
که ز اغراض و ز علتها جداست
\\
گر چنین و گر چنان چون طالبست
&&
جذب حق او را سوی حق جاذبست
\\
گر محب حق بود لغیره
&&
کی ینال دائما من خیره
\\
یا محب حق بود لعینه
&&
لاسواه خائفا من بینه
\\
هر دو را این جست و جوها زان سریست
&&
این گرفتاری دل زان دلبریست
\\
\end{longtable}
\end{center}
