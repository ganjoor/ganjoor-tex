\begin{center}
\section*{بخش ۱۹ - افتادن شغال در خم رنگ و رنگین شدن و دعوی طاوسی کردن میان شغالان}
\label{sec:sh019}
\addcontentsline{toc}{section}{\nameref{sec:sh019}}
\begin{longtable}{l p{0.5cm} r}
آن شغالی رفت اندر خم رنگ
&&
اندر آن خم کرد یک ساعت درنگ
\\
پس بر آمد پوستش رنگین شده
&&
که منم طاووس علیین شده
\\
پشم رنگین رونق خوش یافته
&&
آفتاب آن رنگها بر تافته
\\
دید خود را سبز و سرخ و فور و زرد
&&
خویشتن را بر شغالان عرضه کرد
\\
جمله گفتند ای شغالک حال چیست
&&
که ترا در سر نشاطی ملتویست
\\
از نشاط از ما کرانه کرده‌ای
&&
این تکبر از کجا آورده‌ای
\\
یک شغالی پیش او شد کای فلان
&&
شید کردی یا شدی از خوش‌دلان
\\
شید کردی تا به منبر بر جهی
&&
تا ز لاف این خلق را حسرت دهی
\\
بس بکوشیدی ندیدی گرمیی
&&
پس ز شید آورده‌ای بی‌شرمیی
\\
گرمی آن اولیا و انبیاست
&&
باز بی‌شرمی پناه هر دغاست
\\
که التفات خلق سوی خود کشند
&&
که خوشیم و از درون بس ناخوشند
\\
\end{longtable}
\end{center}
