\begin{center}
\section*{بخش ۱۵۵ - استدعاء آن مرد از موسی زبان بهایم با طیور}
\label{sec:sh155}
\addcontentsline{toc}{section}{\nameref{sec:sh155}}
\begin{longtable}{l p{0.5cm} r}
گفت موسی را یکی مرد جوان
&&
که بیاموزم زبان جانوران
\\
تا بود کز بانگ حیوانات و دد
&&
عبرتی حاصل کنم در دین خود
\\
چون زبانهای بنی آدم همه
&&
در پی آبست و نان و دمدمه
\\
بوک حیوانات را دردی دگر
&&
باشد از تدبیر هنگام گذر
\\
گفت موسی رو گذر کن زین هوس
&&
کین خطر دارد بسی در پیش و پس
\\
عبرت و بیداری از یزدان طلب
&&
نه از کتاب و از مقال و حرف و لب
\\
گرم‌تر شد مرد زان منعش که کرد
&&
گرم‌تر گردد همی از منع مرد
\\
گفت ای موسی چو نور تو بتافت
&&
هر چه چیزی بود چیزی از تو یافت
\\
مر مرا محروم کردن زین مراد
&&
لایق لطفت نباشد ای جواد
\\
این زمان قایم مقام حق توی
&&
یاس باشد گر مرا مانع شوی
\\
گفت موسی یا رب این مرد سلیم
&&
سخره کردستش مگر دیو رجیم
\\
گر بیاموزم زیان‌کارش بود
&&
ور نیاموزم دلش بد می‌شود
\\
گفت ای موسی بیاموزش که ما
&&
رد نکردیم از کرم هرگز دعا
\\
گفت یا رب او پشیمانی خورد
&&
دست خاید جامه‌ها را بر درد
\\
نیست قدرت هر کسی را سازوار
&&
عجز بهتر مایهٔ پرهیزکار
\\
فقر ازین رو فخر آمد جاودان
&&
که به تقوی ماند دست نارسان
\\
زان غنا و زان غنی مردود شد
&&
که ز قدرت صبرها بدرود شد
\\
آدمی را عجز و فقر آمد امان
&&
از بلای نفس پر حرص و غمان
\\
آن غم آمد ز آرزوهای فضول
&&
که بدان خو کرده است آن صید غول
\\
آرزوی گل بود گل‌خواره را
&&
گلشکر نگوارد آن بیچاره را
\\
\end{longtable}
\end{center}
