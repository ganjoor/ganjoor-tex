\begin{center}
\section*{بخش ۶۹ - دیدن زرگر عاقبت کار را و سخن بر وفق عاقبت گفتن با مستعیر ترازو}
\label{sec:sh069}
\addcontentsline{toc}{section}{\nameref{sec:sh069}}
\begin{longtable}{l p{0.5cm} r}
آن یکی آمد به پیش زرگری
&&
که ترازو ده که بر سنجم زری
\\
گفت خواجه رو مرا غربال نیست
&&
گفت میزان ده برین تسخر مه‌ایست
\\
گفت جاروبی ندارم در دکان
&&
گفت بس بس این مضاحک رابمان
\\
من ترازویی که می‌خواهم بده
&&
خویشتن را کر مکن هر سو مجه
\\
گفت بشنیدم سخن کر نیستم
&&
تا نپنداری که بی معنیستم
\\
این شنیدم لیک پیری مرتعش
&&
دست لرزان جسم تو نا منتعش
\\
وان زر تو هم قراضهٔ خرد مرد
&&
دست لرزد پس بریزد زر خرد
\\
پس بگویی خواجه جاروبی بیار
&&
تا بجویم زر خود را در غبار
\\
چون بروبی خاک را جمع آوری
&&
گوییم غلبیر خواهم ای جری
\\
من ز اول دیدم آخر را تمام
&&
جای دیگر رو ازینجا والسلام
\\
\end{longtable}
\end{center}
