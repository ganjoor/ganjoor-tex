\begin{center}
\section*{بخش ۹۳ - مخفی بودن آن درختان ازچشم خلق}
\label{sec:sh093}
\addcontentsline{toc}{section}{\nameref{sec:sh093}}
\begin{longtable}{l p{0.5cm} r}
این عجب‌تر که بریشان می‌گذشت
&&
صد هزاران خلق از صحرا و دشت
\\
ز آرزوی سایه جان می‌باختند
&&
از گلیمی سایه‌بان می‌ساختند
\\
سایهٔ آن را نمی‌دیدند هیچ
&&
صد تفو بر دیده‌های پیچ پیچ
\\
ختم کرده قهر حق بر دیده‌ها
&&
که نبیند ماه را بیند سها
\\
ذره‌ای را بیند و خورشید نه
&&
لیک از لطف و کرم نومید نه
\\
کاروانها بی نوا وین میوه‌ها
&&
پخته می‌ریزد چه سحرست ای خدا
\\
سیب پوسیده همی‌چیدند خلق
&&
درهم افتاده بیغما خشک‌حلق
\\
گفته هر برگ و شکوفه آن غصون
&&
دم بدم یا لیت قوم یعلمون
\\
بانگ می‌آمد ز سوی هر درخت
&&
سوی ما آیید خلق شوربخت
\\
بانگ می‌آمد ز غیرت بر شجر
&&
چشمشان بستیم کلا لا وزر
\\
گر کسی می‌گفتشان کین سو روید
&&
تا ازین اشجار مستسعد شوید
\\
جمله می‌گفتند کین مسکین مست
&&
از قضاء الله دیوانه شدست
\\
مغز این مسکین ز سودای دراز
&&
وز ریاضت گشت فاسد چون پیاز
\\
او عجب می‌ماند یا رب حال چیست
&&
خلق را این پرده و اضلال چیست
\\
خلق گوناگون با صد رای و عقل
&&
یک قدم آن سو نمی‌آرند نقل
\\
عاقلان و زیرکانشان ز اتفاق
&&
گشته منکر زین چنین باغی و عاق
\\
یا منم دیوانه و خیره شده
&&
دیو چیزی مرا مرا بر سر زده
\\
چشم می‌مالم بهر لحظه که من
&&
خواب می‌بینم خیال اندر زمن
\\
خواب چه بود بر درختان می‌روم
&&
میوه‌هاشان می‌خورم چون نگروم
\\
باز چون من بنگرم در منکران
&&
که همی‌گیرند زین بستان کران
\\
با کمال احتیاج و افتقار
&&
ز آرزوی نیم غوره جانسپار
\\
ز اشتیاق و حرص یک برگ درخت
&&
می‌زنند این بی‌نوایان آه سخت
\\
در هزیمت زین درخت و زین ثمار
&&
این خلایق صد هزار اندر هزار
\\
باز می‌گویم عجب من بی‌خودم
&&
دست در شاخ خیالی در زدم
\\
حتی اذا ما استیاس الرسل بگو
&&
تا بظنوا انهم قد کذبوا
\\
این قرائت خوان که تخفیف کذب
&&
این بود که خویش بیند محتجب
\\
در گمان افتاد جان انبیا
&&
ز اتفاق منکری اشقیا
\\
جائهم بعد التشکک نصرنا
&&
ترکشان گو بر درخت جان بر آ
\\
می‌خور و می‌ده بدان کش روزیست
&&
هر دم و هر لحظه سحرآموزیست
\\
خلق‌گویان ای عجب این بانگ چیست
&&
چونک صحرا از درخت و بر تهیست
\\
گیج گشتیم از دم سوداییان
&&
که به نزدیک شما باغست و خوان
\\
چشم می‌مالیم اینجا باغ نیست
&&
یا بیابانیست یا مشکل رهیست
\\
ای عجب چندین دراز این گفت و گو
&&
چون بود بیهوده ور خود هست کو
\\
من همی‌گویم چو ایشان ای عجب
&&
این چنین مهری چرا زد صنع رب
\\
زین تنازعها محمد در عجب
&&
در تعجب نیز مانده بولهب
\\
زین عجب تا آن عجب فرقیست ژرف
&&
تا چه خواهد کرد سلطان شگرف
\\
ای دقوقی تیزتر ران هین خموش
&&
چند گویی چند چون قحطست گوش
\\
\end{longtable}
\end{center}
