\begin{center}
\section*{بخش ۱۴۱ - قصه عشق صوفی بر سفرهٔ تهی}
\label{sec:sh141}
\addcontentsline{toc}{section}{\nameref{sec:sh141}}
\begin{longtable}{l p{0.5cm} r}
صوفیی بر میخ روزی سفره دید
&&
چرخ می‌زد جامه‌ها را می‌درید
\\
بانگ می‌زد نک نوای بی‌نوا
&&
قحطها و دردها را نک دوا
\\
چونک دود و شور او بسیار شد
&&
هر که صوفی بود با او یار شد
\\
کخ‌کخی و های و هویی می‌زدند
&&
تای چندی مست و بی‌خود می‌شدند
\\
بوالفضولی گفت صوفی را که چیست
&&
سفره‌ای آویخته وز نان تهیست
\\
گفت رو رو نقش بی‌معنیستی
&&
تو بجو هستی که عاشق نیستی
\\
عشق نان بی نان غذای عاشق است
&&
بند هستی نیست هر کو صادقست
\\
عاشقان را کار نبود با وجود
&&
عاشقان را هست بی سرمایه سود
\\
بال نه و گرد عالم می‌پرند
&&
دست نه و گو ز میدان می‌برند
\\
آن فقیری کو ز معنی بوی یافت
&&
دست ببریده همی زنبیل بافت
\\
عاشقان اندر عدم خیمه زدند
&&
چون عدم یک‌رنگ و نفس واحدند
\\
شیرخواره کی شناسد ذوق لوت
&&
مر پری را بوی باشد لوت و پوت
\\
آدمی کی بو برد از بوی او
&&
چونک خوی اوست ضد خوی او
\\
یابد از بو آن پری بوی‌کش
&&
تو نیابی آن ز صد من لوت خوش
\\
پیش قبطی خون بود آن آب نیل
&&
آب باشد پیش سبطی جمیل
\\
جاده باشد بحر ز اسرائیلیان
&&
غرقه گه باشد ز فرعون عوان
\\
\end{longtable}
\end{center}
