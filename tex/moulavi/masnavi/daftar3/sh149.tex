\begin{center}
\section*{بخش ۱۴۹ - مشک آن غلام ازغیب پر آب کردن بمعجزه و آن غلام سیاه را سپیدرو کردن باذن الله تعالی}
\label{sec:sh149}
\addcontentsline{toc}{section}{\nameref{sec:sh149}}
\begin{longtable}{l p{0.5cm} r}
ای غلام اکنون تو پر بین مشک خود
&&
تا نگویی درشکایت نیک و بد
\\
آن سیه حیران شد از برهان او
&&
می‌دمید از لامکان ایمان او
\\
چشمه‌ای دید از هوا ریزان شده
&&
مشک او روپوش فیض آن شده
\\
زان نظر روپوشها هم بر درید
&&
تا معین چشمهٔ غیبی بدید
\\
چشمها پر آب کرد آن دم غلام
&&
شد فراموشش ز خواجه وز مقام
\\
دست و پایش ماند از رفتن به راه
&&
زلزله افکند در جانش اله
\\
باز بهر مصلحت بازش کشید
&&
که به خویش آ باز رو ای مستفید
\\
وقت حیرت نیست حیرت پیش تست
&&
این زمان در ره در آ چالاک و چست
\\
دستهای مصطفی بر رو نهاد
&&
بوسه‌های عاشقانه بس بداد
\\
مصطفی دست مبارک بر رخش
&&
آن زمان مالید و کرد او فرخش
\\
شد سپید آن زنگی و زادهٔ حبش
&&
همچو بدر و روز روشن شد شبش
\\
یوسفی شد در جمال و در دلال
&&
گفتش اکنون رو بده وا گوی حال
\\
او همی‌شد بی سر و بی پای مست
&&
پای می‌نشناخت در رفتن ز دست
\\
پس بیامد با دو مشک پر روان
&&
سوی خواجه از نواحی کاروان
\\
\end{longtable}
\end{center}
