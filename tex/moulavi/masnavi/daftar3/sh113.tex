\begin{center}
\section*{بخش ۱۱۳ - گواهی دادن دست و پا و زبان بر سر  ظالم هم در دنیا}
\label{sec:sh113}
\addcontentsline{toc}{section}{\nameref{sec:sh113}}
\begin{longtable}{l p{0.5cm} r}
پس همینجا دست و پایت در گزند
&&
بر ضمیر تو گواهی می‌دهند
\\
چون موکل می‌شود برتو ضمیر
&&
که بگو تو اعتقادت وا مگیر
\\
خاصه در هنگام خشم و گفت و گو
&&
می‌کند ظاهر سرت را مو بمو
\\
چون موکل می‌شود ظلم و جفا
&&
که هویدا کن مرا ای دست و پا
\\
چون همی‌گیرد گواه سر لگام
&&
خاصه وقت جوش و خشم و انتقام
\\
پس همان کس کین موکل می‌کند
&&
تا لوای راز بر صحرا زند
\\
پس موکلهای دیگر روز حشر
&&
هم تواند آفرید از بهر نشر
\\
ای بده دست آمده در ظلم و کین
&&
گوهرت پیداست حاجت نیست این
\\
نیست حاجت شهره گشتن در گزند
&&
بر ضمیر آتشینت واقف‌اند
\\
نفس تو هر دم بر آرد صد شرار
&&
که ببینیدم منم ز اصحاب نار
\\
جزو نارم سوی کل خود روم
&&
من نه نورم که سوی حضرت شوم
\\
همچنان کین ظالم حق ناشناس
&&
بهر گاوی کرد چندین التباس
\\
او ازو صد گاو برد و صد شتر
&&
نفس اینست ای پدر از وی ببر
\\
نیز روزی با خدا زاری نکرد
&&
یا ربی نامد ازو روزی بدرد
\\
کای خدا خصم مرا خشنود کن
&&
گر منش کردم زیان تو سود کن
\\
گر خطا کشتم دیت بر عاقله‌ست
&&
عاقلهٔ جانم تو بودی از الست
\\
سنگ می‌ندهد به استغفار در
&&
این بود انصاف نفس ای جان حر
\\
\end{longtable}
\end{center}
