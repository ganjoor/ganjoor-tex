\begin{center}
\section*{بخش ۱۰۸ - تضرع آن شخص از داوری داود علیه السلام}
\label{sec:sh108}
\addcontentsline{toc}{section}{\nameref{sec:sh108}}
\begin{longtable}{l p{0.5cm} r}
سجده کرد و گفت کای دانای سوز
&&
در دل داود انداز آن فروز
\\
در دلش نه آنچ تو اندر دلم
&&
اندر افکندی براز ای مفضلم
\\
این بگفت و گریه در شد های های
&&
تا دل داود بیرون شد ز جای
\\
گفت هین امروز ای خواهان گاو
&&
مهلتم ده وین دعاوی را مکاو
\\
تا روم من سوی خلوت در نماز
&&
پرسم این احوال از دانای راز
\\
خوی دارم در نماز این التفات
&&
معنی قرة عینی فی الصلوة
\\
روزن جانم گشادست از صفا
&&
می‌رسد بی واسطه نامهٔ خدا
\\
نامه و باران و نور از روزنم
&&
می‌فتد در خانه‌ام از معدنم
\\
دوزخست آن خانه کان بی روزنست
&&
اصل دین ای بنده روزن کردنست
\\
تیشهٔ هر بیشه‌ای کم زن بیا
&&
تیشه زن در کندن روزن هلا
\\
یا نمی‌دانی که نور آفتاب
&&
عکس خورشید برونست از حجاب
\\
نور این دانی که حیوان دید هم
&&
پس چه کرمنا بود بر آدمم
\\
من چو خورشیدم درون نور غرق
&&
می‌ندانم کرد خویش از نور فرق
\\
رفتنم سوی نماز و آن خلا
&&
بهر تعلیمست ره مر خلق را
\\
کژ نهم تا راست گردد این جهان
&&
حرب خدعه این بود ای پهلوان
\\
نیست دستوری و گر نه ریختی
&&
گرد از دریای راز انگیختی
\\
همچنین داود می‌گفت این نسق
&&
خواست گشتن عقل خلقان محترق
\\
پس گریبانش کشید از پس یکی
&&
که ندارم در یکیی‌اش شکی
\\
با خود آمد گفت را کوتاه کرد
&&
لب ببست و عزم خلوتگاه کرد
\\
\end{longtable}
\end{center}
