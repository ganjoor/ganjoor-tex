\begin{center}
\section*{بخش ۲۱۲ - ملاقات آن عاشق با صدر جهان}
\label{sec:sh212}
\addcontentsline{toc}{section}{\nameref{sec:sh212}}
\begin{longtable}{l p{0.5cm} r}
آن بخاری نیز خود بر شمع زد
&&
گشته بود از عشقش آسان آن کبد
\\
آه سوزانش سوی گردون شده
&&
در دل صدر جهان مهر آمده
\\
گفته با خود در سحرگه کای احد
&&
حال آن آوارهٔ ما چون بود
\\
او گناهی کرد و ما دیدیم لیک
&&
رحمت ما را نمی‌دانست نیک
\\
خاطر مجرم ز ما ترسان شود
&&
لیک صد اومید در ترسش بود
\\
من بترسانم وقیح یاوه را
&&
آنک ترسد من چه ترسانم ورا
\\
بهر دیگ سرد آذر می‌رود
&&
نه بدان کز جوش از سر می‌رود
\\
آمنان را من بترسانم به علم
&&
خایفان را ترس بردارم به حلم
\\
پاره‌دوزم پاره در موضع نهم
&&
هر کسی را شربت اندر خور دهم
\\
هست سر مرد چون بیخ درخت
&&
زان بروید برگهاش از چوب سخت
\\
درخور آن بیخ رسته برگها
&&
در درخت و در نفوس و در نهی
\\
برفلک پرهاست ز اشجار وفا
&&
اصلها ثابت و فرعه فی السما
\\
چون برست از عشق پر بر آسمان
&&
چون نروید در دل صدر جهان
\\
موج می‌زد در دلش عفو گنه
&&
که ز هر دل تا دل آمد روزنه
\\
که ز دل تا دل یقین روزن بود
&&
نه جدا و دور چون دو تن بود
\\
متصل نبود سفال دو چراغ
&&
نورشان ممزوج باشد در مساغ
\\
هیچ عاشق خود نباشد وصل‌جو
&&
که نه معشوقش بود جویای او
\\
لیک عشق عاشقان تن زه کند
&&
عشق معشوقان خوش و فربه کند
\\
چون درین دل برق مهر دوست جست
&&
اندر آن دل دوستی می‌دان که هست
\\
در دل تو مهر حق چون شد دوتو
&&
هست حق را بی گمانی مهر تو
\\
هیچ بانگ کف زدن ناید بدر
&&
از یکی دست تو بی دستی دگر
\\
تشنه می‌نالد که ای آب گوار
&&
آب هم نالد که کو آن آب‌خوار
\\
جذب آبست این عطش در جان ما
&&
ما از آن او و او هم آن ما
\\
حکمت حق در قضا و در قدر
&&
کرد ما را عاشقان همدگر
\\
جمله اجزای جهان زان حکم پیش
&&
جفت جفت و عاشقان جفت خویش
\\
هست هر جزوی ز عالم جفت‌خواه
&&
راست همچون کهربا و برگ کاه
\\
آسمان گوید زمین را مرحبا
&&
با توم چون آهن و آهن‌ربا
\\
آسمان مرد و زمین زن در خرد
&&
هرچه آن انداخت این می‌پرورد
\\
چون نماند گرمیش بفرستد او
&&
چون نماند تری و نم بدهد او
\\
برج خاکی خاک ارضی را مدد
&&
برج آبی تریش اندر دمد
\\
برج بادی ابر سوی او برد
&&
تا بخارات وخم را بر کشد
\\
برج آتش گرمی خورشید ازو
&&
همچو تابهٔ سرخ ز آتش پشت و رو
\\
هست سرگردان فلک اندر زمن
&&
همچو مردان گرد مکسب بهر زن
\\
وین زمین کدبانویها می‌کند
&&
بر ولادات و رضاعش می‌تند
\\
پس زمین و چرخ را دان هوشمند
&&
چونک کار هوشمندان می‌کنند
\\
گر نه از هم این دو دلبر می‌مزند
&&
پس چرا چون جفت در هم می‌خزند
\\
بی زمین کی گل بروید و ارغوان
&&
پس چه زاید ز آب و تاب آسمان
\\
بهر آن میلست در ماده به نر
&&
تا بود تکمیل کار همدگر
\\
میل اندر مرد و زن حق زان نهاد
&&
تا بقا یابد جهان زین اتحاد
\\
میل هر جزوی به جزوی هم نهد
&&
ز اتحاد هر دو تولیدی زهد
\\
شب چنین با روز اندر اعتناق
&&
مختلف در صورت اما اتفاق
\\
روز و شب ظاهر دو ضد و دشمنند
&&
لیک هر دو یک حقیقت می‌تنند
\\
هر یکی خواهان دگر را همچو خویش
&&
از پی تکمیل فعل و کار خویش
\\
زانک بی شب دخل نبود طبع را
&&
پس چه اندر خرج آرد روزها
\\
\end{longtable}
\end{center}
