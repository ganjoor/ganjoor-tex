\begin{center}
\section*{بخش ۱۷۹ - پیدا شدن روح القدس بصورت آدمی بر مریم بوقت برهنگی و غسل کردن و پناه گرفتن بحق تعالی}
\label{sec:sh179}
\addcontentsline{toc}{section}{\nameref{sec:sh179}}
\begin{longtable}{l p{0.5cm} r}
همچو مریم گوی پیش از فوت ملک
&&
نقش را کالعوذ بالرحمن منک
\\
دید مریم صورتی بس جان‌فزا
&&
جان‌فزایی دلربایی در خلا
\\
پیش او بر رست از روی زمین
&&
چون مه وخورشید آن روح الامین
\\
از زمین بر رست خوبی بی‌نقاب
&&
آنچنان کز شرق روید آفتاب
\\
لرزه بر اعضای مریم اوفتاد
&&
کو برهنه بود و ترسید از فساد
\\
صورتی که یوسف ار دیدی عیان
&&
دست از حیرت بریدی چو زنان
\\
همچو گل پیشش برویید آن ز گل
&&
چون خیالی که بر آرد سر ز دل
\\
گشت بی‌خود مریم و در بی‌خودی
&&
گفت بجهم در پناه ایزدی
\\
زانک عادت کرده بود آن پاک‌جیب
&&
در هزیمت رخت بردن سوی غیب
\\
چون جهان را دید ملکی بی‌قرار
&&
حازمانه ساخت زان حضرت حصار
\\
تا به گاه مرگ حصنی باشدش
&&
که نیابد خصم راه مقصدش
\\
از پناه حق حصاری به ندید
&&
یورتگه نزدیک آن دز برگزید
\\
چون بدید آن غمزه‌های عقل‌سوز
&&
که ازو می‌شد جگرها تیردوز
\\
شاه و لشکر حلقه در گوشش شده
&&
خسروان هوش بیهوشش شده
\\
صد هزاران شاه مملوکش برق
&&
صد هزاران بدر را داده به دق
\\
زهره نی مر زهره را تا دم زند
&&
عقل کلش چون ببیند کم زند
\\
من چگویم که مرا در دوخته‌ست
&&
دمگهم را دمگه او سوخته‌ست
\\
دود آن نارم دلیلم من برو
&&
دور از آن شه باطل ما عبروا
\\
خود نباشد آفتابی را دلیل
&&
جز که نور آفتاب مستطیل
\\
سایه کی بود تا دلیل او بود
&&
این بستش کع ذلیل او بود
\\
این جلالت در دلالت صادقست
&&
جمله ادراکات پس او سابقست
\\
جمله ادراکات بر خرهای لنگ
&&
او سوار باد پران چون خدنگ
\\
گر گریزد کس نیابد گرد شه
&&
ور گریزند او بگیرد پیش ره
\\
جمله ادراکات را آرام نی
&&
وقت میدانست وقت جام نی
\\
آن یکی وهمی چو بازی می‌پرد
&&
وآن دگر چون تیر معبر می‌درد
\\
وان دگر چون کشتی با بادبان
&&
وآن دگر اندر تراجع هر زمان
\\
چون شکاری می‌نمایدشان ز دور
&&
جمله حمله می‌فزایند آن طیور
\\
چونک ناپیدا شود حیران شوند
&&
همچو جغدان سوی هر ویران شوند
\\
منتظر چشمی به هم یک چشم باز
&&
تا که پیدا گردد آن صید به ناز
\\
چون بماند دیر گویند از ملال
&&
صید بود آن خود عجب یا خود خیال
\\
مصلحت آنست تا یک ساعتی
&&
قوتی گیرند و زور از راحتی
\\
گر نبودی شب همه خلقان ز آز
&&
خویشتن را سوختندی ز اهتزاز
\\
از هوس وز حرص سود اندوختن
&&
هر کسی دادی بدن را سوختن
\\
شب پدید آید چو گنج رحمتی
&&
تا رهند ازحرص خود یکساعتی
\\
چونک قبضی آیدت ای راه‌رو
&&
آن صلاح تست آتش دل مشو
\\
زآنک در خرجی در آن بسط و گشاد
&&
خرج را دخلی بباید زاعتداد
\\
گر هماره فصل تابستان بدی
&&
سوزش خورشید در بستان شدی
\\
منبتش را سوختی از بیخ و بن
&&
که دگر تازه نگشتی آن کهن
\\
گر ترش‌رویست آن دی مشفق است
&&
صیف خندانست اما محرقست
\\
چونک قبض آید تو در وی بسط بین
&&
تازه باش و چین میفکن در جبین
\\
کودکان خندان و دانایان ترش
&&
غم جگر را باشد و شادی ز شش
\\
چشم کودک همچو خر در آخرست
&&
چشم عاقل در حساب آخرست
\\
او در آخر چرب می‌بیند علف
&&
وین ز قصاب آخرش بیند تلف
\\
آن علف تلخست کین قصاب داد
&&
بهر لحم ما ترازویی نهاد
\\
رو ز حکمت خور علف کان را خدا
&&
بی غرض دادست از محض عطا
\\
فهم نان کردی نه حکمت ای رهی
&&
زانچ حق گفتت کلوا من رزقه
\\
رزق حق حکمت بود در مرتبت
&&
کان گلوگیرت نباشد عاقبت
\\
این دهان بستی دهانی باز شد
&&
کو خورندهٔ لقمه‌های راز شد
\\
گر ز شیر دیو تن را وابری
&&
در فطام اوبسی نعمت خوردی
\\
ترک‌جوشش شرح کردم نیم‌خام
&&
از حکیم غزنوی بشنو تمام
\\
در الهی‌نامه گوید شرح این
&&
آن حکیم غیب و فخرالعارفین
\\
غم خور و نان غم‌افزایان مخور
&&
زانک عاقل غم خورد کودک شکر
\\
قند شادی میوهٔ باغ غمست
&&
این فرح زخمست وآن غم مرهمست
\\
غم چو بینی در کنارش کش به عشق
&&
از سر ربوه نظر کن در دمشق
\\
عاقل از انگور می بیند همی
&&
عاشق از معدوم شی بیند همی
\\
جنگ می‌کردند حمالان پریر
&&
تو مکش تا من کشم حملش چو شیر
\\
زانک زان رنجش همی‌دیدند سود
&&
حمل را هر یک ز دیگر می‌ربود
\\
مزد حق کو مزد آن بی‌مایه کو
&&
این دهد گنجیت مزد و آن تسو
\\
گنج زری که چو خسپی زیر ریگ
&&
با تو باشد ان نباشد مردریگ
\\
پیش پیش آن جنازه‌ت می‌دود
&&
مونس گور و غریبی می‌شود
\\
بهر روز مرگ این دم مرده باش
&&
تا شوی با عشق سرمد خواجه‌تاش
\\
صبر می‌بیند ز پردهٔ اجتهاد
&&
روی چون گلنار و زلفین مراد
\\
غم چو آیینه‌ست پیش مجتهد
&&
کاندرین ضد می‌نماید روی ضد
\\
بعد ضد رنج آن ضد دگر
&&
رو دهد یعنی گشاد و کر و فر
\\
این دو وصف از پنجهٔ دستت ببین
&&
بعد قبض مشت بسط آید یقین
\\
پنجه را گر قبض باشد دایما
&&
یا همه بسط او بود چون مبتلا
\\
زین دو وصفش کار و مکسب منتظم
&&
چون پر مرغ این دو حال او را مهم
\\
چونک مریم مضطرب شد یک زمان
&&
همچنانک بر زمین آن ماهیان
\\
\end{longtable}
\end{center}
