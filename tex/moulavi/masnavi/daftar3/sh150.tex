\begin{center}
\section*{بخش ۱۵۰ - دیدن خواجه غلام خود را سپید و ناشناختن کی اوست و گفتن کی غلام مرا تو کشته‌ای خونت گرفت و خدا ترا به دست من انداخت}
\label{sec:sh150}
\addcontentsline{toc}{section}{\nameref{sec:sh150}}
\begin{longtable}{l p{0.5cm} r}
خواجه از دورش بدید و خیره ماند
&&
از تحیر اهل آن ده را بخواند
\\
راویهٔ ما اشتر ما هست این
&&
پس کجا شد بندهٔ زنگی‌جبین
\\
این یکی بدریست می‌آید ز دور
&&
می‌زند بر نور روز از روش نور
\\
کو غلام ما مگر سرگشته شد
&&
یا بدو گرگی رسید و کشته شد
\\
چون بیامد پیش گفتش کیستی
&&
از یمن زادی و یا ترکیستی
\\
گو غلامم را چه کردی راست گو
&&
گر بکشتی وا نما حیلت مجو
\\
گفت اگر کشتم بتو چون آمدم
&&
چون به پای خود درین خون آمدم
\\
کو غلام من بگفت اینک منم
&&
کرد دست فضل یزدان روشنم
\\
هی چه می‌گویی غلام من کجاست
&&
هین نخواهی رست از من جز براست
\\
گفت اسرار ترا با آن غلام
&&
جمله وا گویم یکایک من تمام
\\
زان زمانی که خریدی تو مرا
&&
تا به اکنون باز گویم ماجرا
\\
تا بدانی که همانم در وجود
&&
گرچه از شبدیز من صبحی گشود
\\
رنگ دیگر شد ولیکن جان پاک
&&
فارغ از رنگست و از ارکان و خاک
\\
تن‌شناسان زود ما را گم کنند
&&
آب‌نوشان ترک مشک و خم کنند
\\
جان‌شناسان از عددها فارغ‌اند
&&
غرقهٔ دریای بی‌چونند و چند
\\
جان شو و از راه جان جان را شناس
&&
یار بینش شو نه فرزند قیاس
\\
چون ملک با عقل یک سررشته‌اند
&&
بهر حکمت را دو صورت گشته‌اند
\\
آن ملک چون مرغ بال و پر گرفت
&&
وین خرد بگذاشت پر و فر گرفت
\\
لاجرم هر دو مناصر آمدند
&&
هر دو خوش رو پشت همدیگر شدند
\\
هم ملک هم عقل حق را واجدی
&&
هر دو آدم را معین و ساجدی
\\
نفس و شیطان بوده ز اول واحدی
&&
بوده آدم را عدو و حاسدی
\\
آنک آدم را بدن دید او رمید
&&
و آنک نور مؤتمن دید او خمید
\\
آن دو دیده‌روشنان بودند ازین
&&
وین دو را دیده ندیده غیر طین
\\
این بیان اکنون چو خر بر یخ بماند
&&
چون نشاید بر جهود انجیل خواند
\\
کی توان با شیعه گفتن از عمر
&&
کی توان بربط زدن در پیش کر
\\
لیک گر در ده به گوشه یک کسست
&&
های هویی که برآوردم بسست
\\
مستحق شرح را سنگ و کلوخ
&&
ناطقی گردد مشرح با رسوخ
\\
\end{longtable}
\end{center}
