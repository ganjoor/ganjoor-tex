\begin{center}
\section*{بخش ۲۲۵ - نواختن معشوق عاشق بیهوش را تا به هوش باز آید}
\label{sec:sh225}
\addcontentsline{toc}{section}{\nameref{sec:sh225}}
\begin{longtable}{l p{0.5cm} r}
می‌کشید از بیهشی‌اش در بیان
&&
اندک اندک از کرم صدر جهان
\\
بانگ زد در گوش او شه کای گدا
&&
زر نثار آوردمت دامن گشا
\\
جان تو کاندر فراقم می‌طپید
&&
چونک زنهارش رسیدم چون رمید
\\
ای بدیده در فراقم گرم و سرد
&&
با خود آ از بی‌خودی و باز گرد
\\
مرغ خانه اشتری را بی خرد
&&
رسم مهمانش به خانه می‌برد
\\
چون به خانه مرغ اشتر پا نهاد
&&
خانه ویران گشت و سقف اندر فتاد
\\
خانهٔ مرغست هوش و عقل ما
&&
هوش صالح طالب ناقهٔ خدا
\\
ناقه چون سر کرد در آب و گلش
&&
نه گل آنجا ماند نه جان و دلش
\\
کرد فضل عشق انسان را فضول
&&
زین فزون‌جویی ظلومست و جهول
\\
جاهلست و اندرین مشکل شکار
&&
می‌کشد خرگوش شیری در کنار
\\
کی کنار اندر کشیدی شیر را
&&
گر بدانستی و دیدی شیر را
\\
ظالمست او بر خود و بر جان خود
&&
ظلم بین کز عدلها گو می‌برد
\\
جهل او مر علمها را اوستاد
&&
ظلم او مر عدلها را شد رشاد
\\
دست او بگرفت کین رفته دمش
&&
آنگهی آید که من دم بخشمش
\\
چون به من زنده شود این مرده‌تن
&&
جان من باشد که رو آرد به من
\\
من کنم او را ازین جان محتشم
&&
جان که من بخشم ببیند بخششم
\\
جان نامحرم نبیند روی دوست
&&
جز همان جان کاصل او از کوی اوست
\\
در دمم قصاب‌وار این دوست را
&&
تا هلد آن مغز نغزش پوست را
\\
گفت ای جان رمیده از بلا
&&
وصل ما را در گشادیم الصلا
\\
ای خود ما بی‌خودی و مستی‌ات
&&
ای ز هست ما هماره هستی‌ات
\\
با تو بی لب این زمان من نو بنو
&&
رازهای کهنه گویم می‌شنو
\\
زانک آن لبها ازین دم می‌رمد
&&
بر لب جوی نهان بر می‌دمد
\\
گوش بی‌گوشی درین دم بر گشا
&&
بهر راز یفعل الله ما یشا
\\
چون صلای وصل بشنیدن گرفت
&&
اندک اندک مرده جنبیدن گرفت
\\
نه کم از خاکست کز عشوهٔ صبا
&&
سبز پوشد سر بر آرد از فنا
\\
کم ز آب نطفه نبود کز خطاب
&&
یوسفان زایند رخ چون آفتاب
\\
کم ز بادی نیست شد از امر کن
&&
در رحم طاوس و مرغ خوش‌سخن
\\
کم ز کوه سنگ نبود کز ولاد
&&
ناقه‌ای کان ناقه ناقه زاد زاد
\\
زین همه بگذر نه آن مایهٔ عدم
&&
عالمی زاد و بزاید دم بدم
\\
بر جهید و بر طپید و شاد شاد
&&
یک دو چرخی زد سجود اندر فتاد
\\
\end{longtable}
\end{center}
