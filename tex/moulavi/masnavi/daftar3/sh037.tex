\begin{center}
\section*{بخش ۳۷ - حکایت مارگیر کی اژدهای فسرده را مرده پنداشت در ریسمانهاش پیچید و  آورد به بغداد}
\label{sec:sh037}
\addcontentsline{toc}{section}{\nameref{sec:sh037}}
\begin{longtable}{l p{0.5cm} r}
یک حکایت بشنو از تاریخ‌گوی
&&
تا بری زین راز سرپوشیده بوی
\\
مارگیری رفت سوی کوهسار
&&
تا بگیرد او به افسونهاش مار
\\
گر گران و گر شتابنده بود
&&
آنک جویندست یابنده بود
\\
در طلب زن دایما تو هر دو دست
&&
که طلب در راه نیکو رهبرست
\\
لنگ و لوک و خفته‌شکل و بی‌ادب
&&
سوی او می‌غیژ و او را می‌طلب
\\
گه بگفت و گه بخاموشی و گه
&&
بوی کردن گیر هر سو بوی شه
\\
گفت آن یعقوب با اولاد خویش
&&
جستن یوسف کنید از حد بیش
\\
هر حس خود را درین جستن بجد
&&
هر طرف رانید شکل مستعد
\\
گفت از روح خدا لا تیاسوا
&&
همچو گم کرده پسر رو سو بسو
\\
از ره حس دهان پرسان شوید
&&
گوش را بر چار راه آن نهید
\\
هر کجا بوی خوش آید بو برید
&&
سوی آن سر کاشنای آن سرید
\\
هر کجا لطفی ببینی از کسی
&&
سوی اصل لطف ره یابی عسی
\\
این همه خوشها ز دریاییست ژرف
&&
جزو را بگذار و بر کل دار طرف
\\
جنگهای خلق بهر خوبیست
&&
برگ بی برگی نشان طوبیست
\\
خشمهای خلق بهر آشتیست
&&
دام راحت دایما بی‌راحتیست
\\
هر زدن بهر نوازش را بود
&&
هر گله از شکر آگه می‌کند
\\
بوی بر از جزو تا کل ای کریم
&&
بوی بر از ضد تا ضد ای حکیم
\\
جنگها می آشتی آرد درست
&&
مارگیر از بهر یاری مار جست
\\
بهر یاری مار جوید آدمی
&&
غم خورد بهر حریف بی‌غمی
\\
او همی‌جستی یکی ماری شگرف
&&
گرد کوهستان و در ایام برف
\\
اژدهایی مرده دید آنجا عظیم
&&
که دلش از شکل او شد پر ز بیم
\\
مارگیر اندر زمستان شدید
&&
مار می‌جست اژدهایی مرده دید
\\
مارگیر از بهر حیرانی خلق
&&
مار گیرد اینت نادانی خلق
\\
آدمی کوهیست چون مفتون شود
&&
کوه اندر مار حیران چون شود
\\
خویشتن نشناخت مسکین آدمی
&&
از فزونی آمد و شد در کمی
\\
خویشتن را آدمی ارزان فروخت
&&
بود اطلس خویش بر دلقی بدوخت
\\
صد هزاران مار و که حیران اوست
&&
او چرا حیران شدست و ماردوست
\\
مارگیر آن اژدها را بر گرفت
&&
سوی بغداد آمد از بهر شگفت
\\
اژدهایی چون ستون خانه‌ای
&&
می‌کشیدش از پی دانگانه‌ای
\\
کاژدهای مرده‌ای آورده‌ام
&&
در شکارش من جگرها خورده‌ام
\\
او همی مرده گمان بردش ولیک
&&
زنده بود و او ندیدش نیک نیک
\\
او ز سرماها و برف افسرده بود
&&
زنده بود و شکل مرده می‌نمود
\\
عالم افسردست و نام او جماد
&&
جامد افسرده بود ای اوستاد
\\
باش تا خورشید حشر آید عیان
&&
تا ببینی جنبش جسم جهان
\\
چون عصای موسی اینجا مار شد
&&
عقل را از ساکنان اخبار شد
\\
پارهٔ خاک ترا چون مرد ساخت
&&
خاکها را جملگی شاید شناخت
\\
مرده زین سو اند و زان سو زنده‌اند
&&
خامش اینجا و آن طرف گوینده‌اند
\\
چون از آن سوشان فرستد سوی ما
&&
آن عصا گردد سوی ما اژدها
\\
کوهها هم لحن داودی کند
&&
جوهر آهن بکف مومی بود
\\
باد حمال سلیمانی شود
&&
بحر با موسی سخن‌دانی شود
\\
ماه با احمد اشارت‌بین شود
&&
نار ابراهیم را نسرین شود
\\
خاک قارون را چو ماری در کشد
&&
استن حنانه آید در رشد
\\
سنگ بر احمد سلامی می‌کند
&&
کوه یحیی را پیامی می‌کند
\\
ما سمیعیم و بصیریم و خوشیم
&&
با شما نامحرمان ما خامشیم
\\
چون شما سوی جمادی می‌روید
&&
محرم جان جمادان چون شوید
\\
از جمادی عالم جانها روید
&&
غلغل اجزای عالم بشنوید
\\
فاش تسبیح جمادات آیدت
&&
وسوسهٔ تاویلها نربایدت
\\
چون ندارد جان تو قندیلها
&&
بهر بینش کرده‌ای تاویلها
\\
که غرض تسبیح ظاهر کی بود
&&
دعوی دیدن خیال غی بود
\\
بلک مر بیننده را دیدار آن
&&
وقت عبرت می‌کند تسبیح‌خوان
\\
پس چو از تسبیح یادت می‌دهد
&&
آن دلالت همچو گفتن می‌بود
\\
این بود تاویل اهل اعتزال
&&
و آن آنکس کو ندارد نور حال
\\
چون ز حس بیرون نیامد آدمی
&&
باشد از تصویر غیبی اعجمی
\\
این سخن پایان ندارد مارگیر
&&
می‌کشید آن مار را با صد زحیر
\\
تا به بغداد آمد آن هنگامه‌جو
&&
تا نهد هنگامه‌ای بر چارسو
\\
بر لب شط مرد هنگامه نهاد
&&
غلغله در شهر بغداد اوفتاد
\\
مارگیری اژدها آورده است
&&
بوالعجب نادر شکاری کرده است
\\
جمع آمد صد هزاران خام‌ریش
&&
صید او گشته چو او از ابلهیش
\\
منتظر ایشان و هم او منتظر
&&
تا که جمع آیند خلق منتشر
\\
مردم هنگامه افزون‌تر شود
&&
کدیه و توزیع نیکوتر رود
\\
جمع آمد صد هزاران ژاژخا
&&
حلقه کرده پشت پا بر پشت پا
\\
مرد را از زن خبر نه ز ازدحام
&&
رفته درهم چون قیامت خاص و عام
\\
چون همی حراقه جنبانید او
&&
می‌کشیدند اهل هنگامه گلو
\\
و اژدها کز زمهریر افسرده بود
&&
زیر صد گونه پلاس و پرده بود
\\
بسته بودش با رسنهای غلیظ
&&
احتیاطی کرده بودش آن حفیظ
\\
در درنگ انتظار و اتفاق
&&
تافت بر آن مار خورشید عراق
\\
آفتاب گرم‌سیرش گرم کرد
&&
رفت از اعضای او اخلاط سرد
\\
مرده بود و زنده گشت او از شگفت
&&
اژدها بر خویش جنبیدن گرفت
\\
خلق را از جنبش آن مرده مار
&&
گشتشان آن یک تحیر صد هزار
\\
با تحیر نعره‌ها انگیختند
&&
جملگان از جنبشش بگریختند
\\
می‌سکست او بند و زان بانگ بلند
&&
هر طرف می‌رفت چاقاچاق بند
\\
بندها بسکست و بیرون شد ز زیر
&&
اژدهایی زشت غران همچو شیر
\\
در هزیمت بس خلایق کشته شد
&&
از فتاده و کشتگان صد پشته شد
\\
مارگیر از ترس بر جا خشک گشت
&&
که چه آوردم من از کهسار و دشت
\\
گرگ را بیدار کرد آن کور میش
&&
رفت نادان سوی عزرائیل خویش
\\
اژدها یک لقمه کرد آن گیج را
&&
سهل باشد خون‌خوری حجاج را
\\
خویش را بر استنی پیچید و بست
&&
استخوان خورده را در هم شکست
\\
نفست اژدرهاست او کی مرده است
&&
از غم و بی آلتی افسرده است
\\
گر بیابد آلت فرعون او
&&
که بامر او همی‌رفت آب جو
\\
آنگه او بنیاد فرعونی کند
&&
راه صد موسی و صد هارون زند
\\
کرمکست آن اژدها از دست فقر
&&
پشه‌ای گردد ز جاه و مال صقر
\\
اژدها را دار در برف فراق
&&
هین مکش او را به خورشید عراق
\\
تا فسرده می‌بود آن اژدهات
&&
لقمهٔ اویی چو او یابد نجات
\\
مات کن او را و آمن شو ز مات
&&
رحم کم کن نیست او ز اهل صلات
\\
کان تف خورشید شهوت بر زند
&&
آن خفاش مردریگت پر زند
\\
می‌کشانش در جهاد و در قتال
&&
مردوار الله یجزیک الوصال
\\
چونک آن مرد اژدها را آورید
&&
در هوای گرم خوش شد آن مرید
\\
لاجرم آن فتنه‌ها کرد ای عزیز
&&
بیست همچندان که ما گفتیم نیز
\\
تو طمع داری که او را بی جفا
&&
بسته داری در وقار و در وفا
\\
هر خسی را این تمنی کی رسد
&&
موسیی باید که اژدرها کشد
\\
صدهزاران خلق ز اژدرهای او
&&
در هزیمت کشته شد از رای او
\\
\end{longtable}
\end{center}
