\begin{center}
\section*{بخش ۲۶ - قصهٔ خواب دیدن فرعون آمدن موسی را  علیه السلام و تدارک اندیشیدن}
\label{sec:sh026}
\addcontentsline{toc}{section}{\nameref{sec:sh026}}
\begin{longtable}{l p{0.5cm} r}
جهد فرعونی چو بی توفیق بود
&&
هرچه او می‌دوخت آن تفتیق بود
\\
از منجم بود در حکمش هزار
&&
وز معبر نیز و ساحر بی‌شمار
\\
مقدم موسی نمودندش بخواب
&&
که کند فرعون و ملکش را خراب
\\
با معبر گفت و با اهل نجوم
&&
چون بود دفع خیال و خواب شوم
\\
جمله گفتندش که تدبیری کنیم
&&
راه زادن را چو ره‌زن می‌زنیم
\\
تا رسید آن شب که مولد بود آن
&&
رای این دیدند آن فرعونیان
\\
که برون آرند آن روز از پگاه
&&
سوی میدان بزم و تخت پادشاه
\\
الصلا ای جمله اسرائیلیان
&&
شاه می‌خواند شما را زان مکان
\\
تا شما را رو نماید بی نقاب
&&
بر شما احسان کند بهر ثواب
\\
کان اسیران را به جز دوری نبود
&&
دیدن فرعون دستوری نبود
\\
گر فتادندی به ره در پیش او
&&
بهر آن یاسه بخفتندی برو
\\
یاسه این بد که نبیند هیچ اسیر
&&
در گه و بیگه لقای آن امیر
\\
بانگ چاووشان چو در ره بشنود
&&
تا ببیند رو به دیواری کند
\\
ور ببیند روی او مجرم بود
&&
آنچ بتر بر سر او آن رود
\\
بودشان حرص لقای ممتنع
&&
چون حریصست آدمی فیما منع
\\
\end{longtable}
\end{center}
