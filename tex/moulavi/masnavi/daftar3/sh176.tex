\begin{center}
\section*{بخش ۱۷۶ - جمع و توفیق میان نفی و اثبات یک چیز از روی نسبت و اختلاف جهت}
\label{sec:sh176}
\addcontentsline{toc}{section}{\nameref{sec:sh176}}
\begin{longtable}{l p{0.5cm} r}
نفی آن یک چیز و اثباتش رواست
&&
چون جهت شد مختلف نسبت دوتاست
\\
ما رمیت اذ رمیت از نسبتست
&&
نفی و اثباتست و هر دو مثبتست
\\
آن تو افکندی چو بر دست تو بود
&&
تو نه افکندی که قوت حق نمود
\\
زور آدم‌زاد را حدی بود
&&
مشت خاک اشکست لشکر کی شود
\\
مشت مشت تست و افکندن ز ماست
&&
زین دو نسبت نفی و اثباتش رواست
\\
یعرفون الانبیا اضدادهم
&&
مثل ما لا یشتبه اولادهم
\\
همچو فرزندان خود دانندشان
&&
منکران با صد دلیل و صد نشان
\\
لیک از رشک و حسد پنهان کنند
&&
خویشتن را بر ندانم می‌زنند
\\
پس چو یعرف گفت چون جای دگر
&&
گفت لایعرفهم غیری فذر
\\
انهم تحت قبابی کامنون
&&
جز که یزدانشان نداند ز آزمون
\\
هم بنسبت گیر این مفتوح را
&&
که بدانی و ندانی نوح را
\\
\end{longtable}
\end{center}
