\begin{center}
\section*{بخش ۴۹ - اختلاف کردن در چگونگی و شکل پیل}
\label{sec:sh049}
\addcontentsline{toc}{section}{\nameref{sec:sh049}}
\begin{longtable}{l p{0.5cm} r}
پیل اندر خانهٔ تاریک بود
&&
عرضه را آورده بودندش هنود
\\
از برای دیدنش مردم بسی
&&
اندر آن ظلمت همی‌شد هر کسی
\\
دیدنش با چشم چون ممکن نبود
&&
اندر آن تاریکیش کف می‌بسود
\\
آن یکی را کف به خرطوم اوفتاد
&&
گفت همچون ناودانست این نهاد
\\
آن یکی را دست بر گوشش رسید
&&
آن برو چون بادبیزن شد پدید
\\
آن یکی را کف چو بر پایش بسود
&&
گفت شکل پیل دیدم چون عمود
\\
آن یکی بر پشت او بنهاد دست
&&
گفت خود این پیل چون تختی بدست
\\
همچنین هر یک به جزوی که رسید
&&
فهم آن می‌کرد هر جا می‌شنید
\\
از نظرگه گفتشان شد مختلف
&&
آن یکی دالش لقب داد این الف
\\
در کف هر کس اگر شمعی بدی
&&
اختلاف از گفتشان بیرون شدی
\\
چشم حس همچون کف دستست و بس
&&
نیست کف را بر همهٔ او دست‌رس
\\
چشم دریا دیگرست و کف دگر
&&
کف بهل وز دیدهٔ دریا نگر
\\
جنبش کفها ز دریا روز و شب
&&
کف همی‌بینی و دریا نه عجب
\\
ما چو کشتیها بهم بر می‌زنیم
&&
تیره‌چشمیم و در آب روشنیم
\\
ای تو در کشتی تن رفته به خواب
&&
آب را دیدی نگر در آب آب
\\
آب را آبیست کو می‌راندش
&&
روح را روحیست کو می‌خواندش
\\
موسی و عیسی کجا بد کآفتاب
&&
کشت موجودات را می‌داد آب
\\
آدم و حوا کجا بد آن زمان
&&
که خدا افکند این زه در کمان
\\
این سخن هم ناقص است و ابترست
&&
آن سخن که نیست ناقص آن سرست
\\
گر بگوید زان بلغزد پای تو
&&
ور نگوید هیچ از آن ای وای تو
\\
ور بگوید در مثال صورتی
&&
بر همان صورت بچفسی ای فتی
\\
بسته‌پایی چون گیا اندر زمین
&&
سر بجنبانی ببادی بی‌یقین
\\
لیک پایت نیست تا نقلی کنی
&&
یا مگر پا را ازین گل بر کنی
\\
چون کنی پا را حیاتت زین گلست
&&
این حیاتت را روش بس مشکلست
\\
چون حیات از حق بگیری ای روی
&&
پس شوی مستغنی از گل می‌روی
\\
شیر خواره چون ز دایه بسکلد
&&
لوت‌خواره شد مرورا می‌هلد
\\
بستهٔ شیر زمینی چون حبوب
&&
جو فطام خویش از قوت القلوب
\\
حرف حکمت خور که شد نور ستیر
&&
ای تو نور بی‌حجب را ناپذیر
\\
تا پذیرا گردی ای جان نور را
&&
تا ببینی بی‌حجب مستور را
\\
چون ستاره سیر بر گردون کنی
&&
بلک بی گردون سفر بی‌چون کنی
\\
آنچنان کز نیست در هست آمدی
&&
هین بگو چون آمدی مست آمدی
\\
راههای آمدن یادت نماند
&&
لیک رمزی بر تو بر خواهیم خواند
\\
هوش را بگذار وانگه هوش‌دار
&&
گوش را بر بند وانگه گوش دار
\\
نه نگویم زانک خامی تو هنوز
&&
در بهاری تو ندیدستی تموز
\\
این جهان همچون درختست ای کرام
&&
ما برو چون میوه‌های نیم‌خام
\\
سخت گیرد خامها مر شاخ را
&&
زانک در خامی نشاید کاخ را
\\
چون بپخت و گشت شیرین لب‌گزان
&&
سست گیرد شاخها را بعد از آن
\\
چون از آن اقبال شیرین شد دهان
&&
سرد شد بر آدمی ملک جهان
\\
سخت‌گیری و تعصب خامی است
&&
تا جنینی کار خون‌آشامی است
\\
چیز دیگر ماند اما گفتنش
&&
با تو روح القدس گوید بی منش
\\
نه تو گویی هم بگوش خویشتن
&&
نه من ونه غیرمن ای هم تو من
\\
همچو آن وقتی که خواب اندر روی
&&
تو ز پیش خود به پیش خود شوی
\\
بشنوی از خویش و پنداری فلان
&&
با تو اندر خواب گفتست آن نهان
\\
تو یکی تو نیستی ای خوش رفیق
&&
بلک گردونی ودریای عمیق
\\
آن تو زفتت که آن نهصدتوست
&&
قلزمست وغرقه گاه صد توست
\\
خود چه جای حد بیداریست و خواب
&&
دم مزن والله اعلم بالصواب
\\
دم مزن تا بشنوی از دم ز نان
&&
آنچ نامد در زبان و در بیان
\\
دم مزن تا بشنوی زان آفتاب
&&
آنچ نامد درکتاب و در خطاب
\\
دم مزن تا دم زند بهر تو روح
&&
آشنا بگذار در کشتی نوح
\\
همچو کنعان کشنا می‌کرد او
&&
که نخواهم کشتی نوح عدو
\\
هی بیا در کشتی بابا نشین
&&
تا نگردی غرق طوفان ای مهین
\\
گفت نه من آشنا آموختم
&&
من به جز شمع تو شمع افروختم
\\
هین مکن کین موج طوفان بلاست
&&
دست و پا و آشنا امروز لاست
\\
باد قهرست و بلای شمع کش
&&
جز که شمع حق نمی‌پاید خمش
\\
گفت نه رفتم برآن کوه بلند
&&
عاصمست آن که مرا از هر گزند
\\
هین مکن که کوه کاهست این زمان
&&
جز حبیب خویش را ندهد امان
\\
گفت من کی پند تو بشنوده‌ام
&&
که طمع کردی که من زین دوده‌ام
\\
خوش نیامد گفت تو هرگز مرا
&&
من بری‌ام از تو در هر دو سرا
\\
هین مکن بابا که روز ناز نیست
&&
مر خدا را خویش وانباز نیست
\\
تا کنون کردی واین دم نازکیست
&&
اندرین درگاه گیرا ناز کیست
\\
لم یلد لم یولدست او از قدم
&&
نه پدر دارد نه فرزند و نه عم
\\
ناز فرزندان کجا خواهد کشید
&&
ناز بابایان کجا خواهد شنید
\\
نیستم مولود پیراکم بناز
&&
نیستم والد جوانا کم گراز
\\
نیستم شوهر نیم من شهوتی
&&
ناز را بگذار اینجا ای ستی
\\
جز خضوع و بندگی و اضطرار
&&
اندرین حضرت ندارد اعتبار
\\
گفت بابا سالها این گفته‌ای
&&
باز می‌گویی بجهل آشفته‌ای
\\
چند ازینها گفته‌ای با هرکسی
&&
تا جواب سرد بشنودی بسی
\\
این دم سرد تو در گوشم نرفت
&&
خاصه اکنون که شدم دانا و زفت
\\
گفت بابا چه زیان دارد اگر
&&
بشنوی یکبار تو پند پدر
\\
همچنین می‌گفت او پند لطیف
&&
همچنان می‌گفت او دفع عنیف
\\
نه پدر از نصح کنعان سیر شد
&&
نه دمی در گوش آن ادبیر شد
\\
اندرین گفتن بدند و موج تیز
&&
بر سر کنعان زد وشد ریز ریز
\\
نوح گفت ای پادشاه بردبار
&&
مر مرا خر مرد و سیلت برد بار
\\
وعده کردی مر مرا تو بارها
&&
که بیابد اهلت از طوفان رها
\\
دل نهادم بر امیدت من سلیم
&&
پس چرا بربود سیل از من گلیم
\\
گفت او از اهل و خویشانت نبود
&&
خود ندیدی تو سپیدی او کبود
\\
چونک دندان تو کرمش در فتاد
&&
نیست دندان بر کنش ای اوستاد
\\
تا که باقی تن نگردد زار ازو
&&
گرچه بود آن تو شو بیزار ازو
\\
گفت بیزارم ز غیر ذات تو
&&
غیر نبود آنک او شد مات تو
\\
تو همی دانی که چونم با تو من
&&
بیست چندانم که با باران چمن
\\
زنده از تو شاد از تو عایلی
&&
مغتذی بی واسطه و بی حایلی
\\
متصل نه منفصل نه ای کمال
&&
بلک بی چون و چگونه و اعتلال
\\
ماهیانیم و تو دریای حیات
&&
زنده‌ایم از لطفت ای نیکو صفات
\\
تو نگنجی در کنار فکرتی
&&
نی به معلولی قرین چون علتی
\\
پیش ازین طوفان و بعد این مرا
&&
تو مخاطب بوده‌ای در ماجرا
\\
با تو می‌گفتم نه با ایشان سخن
&&
ای سخن‌بخش نو و آن کهن
\\
نه که عاشق روز و شب گوید سخن
&&
گاه با اطلال و گاهی با دمن
\\
روی با اطلال کرده ظاهرا
&&
او کرا می‌گوید آن مدحت کرا
\\
شکر طوفان را کنون بگماشتی
&&
واسطهٔ اطلال را بر داشتی
\\
زانک اطلال لئیم و بد بدند
&&
نه ندایی نه صدایی می‌زدند
\\
من چنان اطلال خواهم در خطاب
&&
کز صدا چون کوه واگوید جواب
\\
تا مثنا بشنوم من نام تو
&&
عاشقم برنام جان آرام تو
\\
هرنبی زان دوست دارد کوه را
&&
تا مثنا بشنود نام ترا
\\
آن که پست مثال سنگ لاخ
&&
موش را شاید نه ما را در مناخ
\\
من بگویم او نگردد یار من
&&
بی صدا ماند دم گفتار من
\\
با زمین آن به که هموارش کنی
&&
نیست همدم با قدم یارش کنی
\\
گفت ای نوح ار تو خواهی جمله را
&&
حشر گردانم بر آرم از ثری
\\
بهر کنعانی دل تو نشکنم
&&
لیکت از احوال آگه می‌کنم
\\
گفت نه نه راضیم که تو مرا
&&
هم کنی غرقه اگر باید ترا
\\
هر زمانم غرقه می‌کن من خوشم
&&
حکم تو جانست چون جان می‌کشم
\\
ننگرم کس را وگر هم بنگرم
&&
او بهانه باشد و تو منظرم
\\
عاشق صنع توم در شکر و صبر
&&
عاشق مصنوع کی باشم چو گبر
\\
عاشق صنع خدا با فر بود
&&
عاشق مصنوع او کافر بود
\\
\end{longtable}
\end{center}
