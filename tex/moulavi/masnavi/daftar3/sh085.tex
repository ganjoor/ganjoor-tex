\begin{center}
\section*{بخش ۸۵ - قصهٔ دقوقی رحمة الله علیه و کراماتش}
\label{sec:sh085}
\addcontentsline{toc}{section}{\nameref{sec:sh085}}
\begin{longtable}{l p{0.5cm} r}
آن دقوقی داشت خوش دیباجه‌ای
&&
عاشق و صاحب کرامت خواجه‌ای
\\
در زمین می‌شد چو مه بر آسمان
&&
شب‌روان راگشته زو روشن روان
\\
در مقامی مسکنی کم ساختی
&&
کم دو روز اندر دهی انداختی
\\
گفت در یک خانه گر باشم دو روز
&&
عشق آن مسکن کند در من فروز
\\
غرة المسکن احاذره انا
&&
انقلی یا نفس سیری للغنا
\\
لا اعود خلق قلبی بالمکان
&&
کی یکون خالصا فی الامتحان
\\
روز اندر سیر بد شب در نماز
&&
چشم اندر شاه باز او همچو باز
\\
منقطع از خلق نه از بد خوی
&&
منفرد از مرد و زن نه از دوی
\\
مشفقی خلق و نافع همچو آب
&&
خوش شفعیی و دعااش مستجاب
\\
نیک و بد را مهربان و مستقر
&&
بهتر از مادر شهی‌تر از پدر
\\
گفت پیغامبر شما را ای مهان
&&
چون پدر هستم شفیق و مهربان
\\
زان سبب که جمله اجزای منید
&&
جزو را از کل چرا بر می‌کنید
\\
جزو از کل قطع شد بی کار شد
&&
عضو از تن قطع شد مردار شد
\\
تا نپیوندد بکل بار دگر
&&
مرده باشد نبودش از جان خبر
\\
ور بجنبد نیست آن را خود سند
&&
عضو نو ببریده هم جنبش کند
\\
جزو ازین کل گر برد یکسو رود
&&
این نه آن کلست کو ناقص شود
\\
قطع و وصل او نیاید در مقال
&&
چیز ناقص گفته شد بهر مثال
\\
\end{longtable}
\end{center}
