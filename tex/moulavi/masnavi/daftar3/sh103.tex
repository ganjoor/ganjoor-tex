\begin{center}
\section*{بخش ۱۰۳ - انکار کردن آن جماعت بر دعا و شفاعت دقوقی و پریدن ایشان و ناپیدا شدن در پردهٔ غیب و حیران شدن دقوقی کی در هوا رفتند یا در زمین}
\label{sec:sh103}
\addcontentsline{toc}{section}{\nameref{sec:sh103}}
\begin{longtable}{l p{0.5cm} r}
چون رهید آن کشتی و آمد بکام
&&
شد نماز آن جماعت هم تمام
\\
فجفجی افتادشان با همدگر
&&
کین فضولی کیست از ما ای پدر
\\
هر یکی با آن دگر گفتند سر
&&
از پس پشت دقوقی مستتر
\\
گفت هر یک من نکردستم کنون
&&
این دعا نه از برون نه از درون
\\
گفت مانا این امام ما ز درد
&&
بوالفضولانه مناجاتی بکرد
\\
گفت آن دیگر که ای یار یقین
&&
مر مرا هم می‌نماید این چنین
\\
او فضولی بوده است از انقباض
&&
کرد بر مختار مطلق اعتراض
\\
چون نگه کردم سپس تا بنگرم
&&
که چه می‌گویند آن اهل کرم
\\
یک ازیشان را ندیدم در مقام
&&
رفته بودند از مقام خود تمام
\\
نه به چپ نه راست نه بالا نه زیر
&&
چشم تیز من نشد بر قوم چیر
\\
درها بودند گویی آب گشت
&&
نه نشان پا و نه گردی بدشت
\\
در قباب حق شدند آن دم همه
&&
در کدامین روضه رفتند آن رمه
\\
درتحیر ماندم کین قوم را
&&
چون بپوشانید حق بر چشم ما
\\
آنچنان پنهان شدند از چشم او
&&
مثل غوطهٔ ماهیان در آب جو
\\
سالها درحسرت ایشان بماند
&&
عمرها در شوق ایشان اشک راند
\\
تو بگویی مرد حق اندر نظر
&&
کی در آرد با خدا ذکر بشر
\\
خر ازین می‌خسپد اینجا ای فلان
&&
که بشر دیدی تو ایشان را نه جان
\\
کار ازین ویران شدست ای مرد خام
&&
که بشر دیدی مر ایشان را چو عام
\\
تو همان دیدی که ابلیس لعین
&&
گفت من از آتشم آدم ز طین
\\
چشم ابلیسانه را یک دم ببند
&&
چند بینی صورت آخر چند چند
\\
ای دقوقی با دو چشم همچو جو
&&
هین مبر اومید ایشان را بجو
\\
هین بجو که رکن دولت جستن است
&&
هر گشادی در دل اندر بستن است
\\
از همه کار جهان پرداخته
&&
کو و کو می‌گو بجان چون فاخته
\\
نیک بنگر اندرین ای محتجب
&&
که دعا را بست حق در استجب
\\
هر که را دل پاک شد از اعتلال
&&
آن دعااش می‌رود تا ذوالجلال
\\
\end{longtable}
\end{center}
