\begin{center}
\section*{بخش ۲۱۳ - جذب هر عنصری جنس خود را کی در ترکیب آدمی محتبس شده است به غیر جنس}
\label{sec:sh213}
\addcontentsline{toc}{section}{\nameref{sec:sh213}}
\begin{longtable}{l p{0.5cm} r}
خاک گوید خاک تن را باز گرد
&&
ترک جان کن سوی ما آ همچو گرد
\\
جنس مایی پیش ما اولیتری
&&
به که زان تن وا رهی و زان تری
\\
گوید آری لیک من پابسته‌ام
&&
گرچه همچون تو ز هجران خسته‌ام
\\
تری تن را بجویند آبها
&&
کای تری باز آ ز غربت سوی ما
\\
گرمی تن را همی‌خواند اثیر
&&
که ز ناری راه اصل خویش گیر
\\
هست هفتاد و دو علت در بدن
&&
از کششهای عناصر بی رسن
\\
علت آید تا بدن را بسکلد
&&
تا عناصر همدگر را وا هلد
\\
چار مرغ‌اند این عناصر بسته‌پا
&&
مرگ و رنجوری و علت پاگشا
\\
پایشان از همدگر چون باز کرد
&&
مرغ هر عنصر یقین پرواز کرد
\\
جذبهٔ این اصلها و فرعها
&&
هر دمی رنجی نهد در جسم ما
\\
تا که این ترکیبها را بر درد
&&
مرغ هر جزوی به اصل خود پرد
\\
حکمت حق مانع آید زین عجل
&&
جمعشان دارد بصحت تا اجل
\\
گوید ای اجزا اجل مشهود نیست
&&
پر زدن پیش از اجلتان سود نیست
\\
چونک هر جزوی بجوید ارتفاق
&&
چون بود جان غریب اندر فراق
\\
\end{longtable}
\end{center}
