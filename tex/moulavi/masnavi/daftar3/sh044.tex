\begin{center}
\section*{بخش ۴۴ - فرستادن فرعون به مداین در طلب ساحران}
\label{sec:sh044}
\addcontentsline{toc}{section}{\nameref{sec:sh044}}
\begin{longtable}{l p{0.5cm} r}
چونک موسی بازگشت و او بماند
&&
اهل رای و مشورت را پیش خواند
\\
آنچنان دیدند کز اطراف مصر
&&
جمع آردشان شه و صراف مصر
\\
او بسی مردم فرستاد آن زمان
&&
هر نواحی بهر جمع جادوان
\\
هر طرف که ساحری بد نامدار
&&
کرد پران سوی او ده پیک کار
\\
دو جوان بودند ساحر مشتهر
&&
سحر ایشان در دل مه مستمر
\\
شیر دوشیده ز مه فاش آشکار
&&
در سفرها رفته بر خمی سوار
\\
شکل کرباسی نموده ماهتاب
&&
آن بپیموده فروشیده شتاب
\\
سیم برده مشتری آگه شده
&&
دست از حسرت به رخها بر زده
\\
صد هزاران همچنین در جادوی
&&
بوده منشی و نبوده چون روی
\\
چون بدیشان آمد آن پیغام شاه
&&
کز شما شاهست اکنون چاره‌خواه
\\
از پی آنک دو درویش آمدند
&&
بر شه و بر قصر او موکب زدند
\\
نیست با ایشان بغیر یک عصا
&&
که همی‌گردد به امرش اژدها
\\
شاه و لشکر جمله بیچاره شدند
&&
زین دو کس جمله به افغان آمدند
\\
چاره‌ای می‌باید اندر ساحری
&&
تا بود که زین دو ساحر جان بری
\\
آن دو ساحر را چو این پیغام داد
&&
ترس و مهری در دل هر دو فتاد
\\
عرق جنسیت چو جنبیدن گرفت
&&
سر به زانو بر نهادند از شگفت
\\
چون دبیرستان صوفی زانوست
&&
حل مشکل را دو زانو جادوست
\\
\end{longtable}
\end{center}
