\begin{center}
\section*{بخش ۱۷۰ - تشبیه دنیا کی بظاهر فراخست و بمعنی تنگ و تشبیه خواب کی خلاص است ازین تنگی}
\label{sec:sh170}
\addcontentsline{toc}{section}{\nameref{sec:sh170}}
\begin{longtable}{l p{0.5cm} r}
همچو گرمابه که تفسیده بود
&&
تنگ آیی جانت پخسیده شود
\\
گرچه گرمابه عریضست و طویل
&&
زان تبش تنگ آیدت جان و کلیل
\\
تا برون نایی بنگشاید دلت
&&
پس چه سود آمد فراخی منزلت
\\
یا که کفش تنگ پوشی ای غوی
&&
در بیابان فراخی می‌روی
\\
آن فراخی بیابان تنگ گشت
&&
بر تو زندان آمد آن صحرا و دشت
\\
هر که دید او مر ترا از دور گفت
&&
کو در آن صحرا چو لاله تر شکفت
\\
او نداند که تو همچون ظالمان
&&
از برون در گلشنی جان در فغان
\\
خواب تو آن کفش بیرون کردنست
&&
که زمانی جانت آزاد از تنست
\\
اولیا را خواب ملکست ای فلان
&&
همچو آن اصحاب کهف اندر جهان
\\
خواب می‌بینند و آنجا خواب نه
&&
در عدم در می‌روند و باب نه
\\
خانهٔ تنگ و درون جان چنگ‌لوک
&&
کرد ویران تا کند قصر ملوک
\\
چنگ‌لوکم چون جنین اندر رحم
&&
نه‌مهه گشتم شد این نقلان مهم
\\
گر نباشد درد زه بر مادرم
&&
من درین زندان میان آذرم
\\
مادر طبعم ز درد مرگ خویش
&&
می‌کند ره تا رهد بره ز میش
\\
تا چرد آن بره در صحرای سبز
&&
هین رحم بگشا که گشت این بره گبز
\\
درد زه گر رنج آبستان بود
&&
بر جنین اشکستن زندان بود
\\
حامله گریان ز زه کاین المناص
&&
و آن جنین خندان که پیش آمد خلاص
\\
هرچه زیر چرخ هستند امهات
&&
از جماد و از بهیمه وز نبات
\\
هر یکی از درد غیری غافل اند
&&
جز کسانی که نبیه و کامل‌اند
\\
آنچ کوسه داند از خانهٔ کسان
&&
بلمه از خانه خودش کی داند آن
\\
آنچ صاحب‌دل بداند حال تو
&&
تو ز حال خود ندانی ای عمو
\\
\end{longtable}
\end{center}
