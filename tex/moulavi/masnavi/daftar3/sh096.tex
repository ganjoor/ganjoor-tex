\begin{center}
\section*{بخش ۹۶ - پیش رفتن دقوقی رحمة الله علیه به امامت}
\label{sec:sh096}
\addcontentsline{toc}{section}{\nameref{sec:sh096}}
\begin{longtable}{l p{0.5cm} r}
این سخن پایان ندارد تیز دو
&&
هین نماز آمد دقوقی پیش رو
\\
ای یگانه هین دوگانه بر گزار
&&
تا مزین گردد از تو روزگار
\\
ای امام چشم‌روشن در صلا
&&
چشم روشن باید ایدر پیشوا
\\
در شریعت هست مکروه ای کیا
&&
در امامت پیش کردن کور را
\\
گرچه حافظ باشد و چست و فقیه
&&
چشم‌روشن به وگر باشد سفیه
\\
کور را پرهیز نبود از قذر
&&
چشم باشد اصل پرهیز و حذر
\\
او پلیدی را نبیند در عبور
&&
هیچ مؤمن را مبادا چشم کور
\\
کور ظاهر در نجاسهٔ ظاهرست
&&
کور باطن در نجاسات سرست
\\
این نجاسهٔ ظاهر از آبی رود
&&
آن نجاسهٔ باطن افزون می‌شود
\\
جز به آب چشم نتوان شستن آن
&&
چون نجاسات بواطن شد عیان
\\
چون نجس خواندست کافر را خدا
&&
آن نجاست نیست بر ظاهر ورا
\\
ظاهر کافر ملوث نیست زین
&&
آن نجاست هست در اخلاق و دین
\\
این نجاست بویش آید بیست گام
&&
و آن نجاست بویش از ری تا بشام
\\
بلک بویش آسمانها بر رود
&&
بر دماغ حور و رضوان بر شود
\\
اینچ می‌گویم به قدر فهم تست
&&
مردم اندر حسرت فهم درست
\\
فهم آبست و وجود تن سبو
&&
چون سبو بشکست ریزد آب ازو
\\
این سبو را پنج سوراخست ژرف
&&
اندرو نه آب ماند خود نه برف
\\
امر غضوا غضة ابصارکم
&&
هم شنیدی راست ننهادی تو سم
\\
از دهانت نطق فهمت را برد
&&
گوش چون ریگست فهمت را خورد
\\
همچنین سوراخهای دیگرت
&&
می‌کشاند آب فهم مضمرت
\\
گر ز دریا آب را بیرون کنی
&&
بی عوض آن بحر را هامون کنی
\\
بیگهست ار نه بگویم حال را
&&
مدخل اعواض را و ابدال را
\\
کان عوضها و آن بدلها بحر را
&&
از کجا آید ز بعد خرجها
\\
صد هزاران جانور زو می‌خورند
&&
ابرها هم از برونش می‌برند
\\
باز دریا آن عوضها می‌کشد
&&
از کجا دانند اصحاب رشد
\\
قصه‌ها آغاز کردیم از شتاب
&&
ماند بی مخلص درون این کتاب
\\
ای ضیاء الحق حسام الدین راد
&&
که فلک و ارکان چو تو شاهی نزاد
\\
تو بنادر آمدی در جان و دل
&&
ای دل و جان از قدوم تو خجل
\\
چند کردم مدح قوم ما مضی
&&
قصد من زانها تو بودی ز اقتضا
\\
خانهٔ خود را شناسد خود دعا
&&
تو بنام هر که خواهی کن ثنا
\\
بهر کتمان مدیح از نا محل
&&
حق نهادست این حکایات و مثل
\\
گر چه آن مدح از تو هم آمد خجل
&&
لیک بپذیرد خدا جهد المقل
\\
حق پذیرد کسره‌ای دارد معاف
&&
کز دو دیدهٔ کور دو قطره کفاف
\\
مرغ و ماهی داند آن ابهام را
&&
که ستودم مجمل این خوش‌نام را
\\
تا برو آه حسودان کم وزد
&&
تا خیالش را به دندان کم گزد
\\
خود خیالش را کجا یابد حسود
&&
در وثاق موش طوطی کی غنود
\\
آن خیال او بود از احتیال
&&
موی ابروی ویست آن نه هلال
\\
مدح تو گویم برون از پنج و هفت
&&
بر نویس اکنون دقوقی پیش رفت
\\
\end{longtable}
\end{center}
