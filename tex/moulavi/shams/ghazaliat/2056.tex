\begin{center}
\section*{غزل شماره ۲۰۵۶: خواجه غلط کرده‌ای در روش یار من}
\label{sec:2056}
\addcontentsline{toc}{section}{\nameref{sec:2056}}
\begin{longtable}{l p{0.5cm} r}
خواجه غلط کرده‌ای در روش یار من
&&
صد چو تو هم گم شود در من و در کار من
\\
نبود هر گردنی لایق شمشیر عشق
&&
خون سگان کی خورد ضیغم خون خوار من
\\
قلزم من کی کشد تخته هر کشتیی
&&
شوره تو کی چرد ز ابر گهربار من
\\
سر بمگردان چنین پوز مجنبان چنان
&&
چون تو خری کی رسد در جو انبار من
\\
خواجه به خویش آ یکی چشم گشا اندکی
&&
گر چه نه بر پای توست اندک و بسیار من
\\
گفت که عاشق چرا مست شد و بی‌حیا
&&
باده حیا کی هلد خاصه ز خمار من
\\
فتنه گرگی شده هم دغل و مکر او
&&
دام وی از وی کند قانص عیار من
\\
بر سر بازار او گرگ کهن کی خرند
&&
هر طرفی یوسفی زنده به بازار من
\\
همچو تو جغدی کجا باغ ارم را سزد
&&
بلبل جان هم نیافت راه به گلزار من
\\
مفخر تبریزیان شمس حق و دین بگو
&&
بلک صدای تو است این همه گفتار من
\\
\end{longtable}
\end{center}
