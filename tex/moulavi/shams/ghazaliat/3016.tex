\begin{center}
\section*{غزل شماره ۳۰۱۶: بازرهان خلق را از سر و از سرکشی}
\label{sec:3016}
\addcontentsline{toc}{section}{\nameref{sec:3016}}
\begin{longtable}{l p{0.5cm} r}
بازرهان خلق را از سر و از سرکشی
&&
ای که درون دلی چند ز دل درکشی
\\
ای دل دل جان جان آمد هنگام آن
&&
زنده کنی مرده را جانب محشر کشی
\\
پیرهن یوسفی هدیه فرستی به ما
&&
تا بدرد آفتاب پیرهن زرکشی
\\
نیزه کشی بردری تو کمر کوه را
&&
چونک ز دریای غیب آیی و لشکر کشی
\\
خاک در فقر را سرمه کش دل کنی
&&
چارق درویش را بر سر سنجر کشی
\\
سینه تاریک را گلشن جنت کنی
&&
تشنه دلان را سوار جانب کوثر کشی
\\
در شکم ماهیی حجره یونس کنی
&&
یوسف صدیق را از بن چه برکشی
\\
نفس شکم خواره را روزه مریم دهی
&&
تا سوی بهرام عشق مرکب لاغر کشی
\\
از غزل و شعر و بیت توبه دهی طبع را
&&
تا دل و جان را به غیب بی‌دم و دفتر کشی
\\
سنبله آتشین رسته کنی بر فلک
&&
زهره مه روی را گوشه چادر کشی
\\
مفخر تبریزیان شمس حق ای وای من
&&
گر تو مرا سوی خویش یک دم کمتر کشی
\\
\end{longtable}
\end{center}
