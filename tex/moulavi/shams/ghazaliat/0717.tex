\begin{center}
\section*{غزل شماره ۷۱۷: آن کس که به بندگیت آید}
\label{sec:0717}
\addcontentsline{toc}{section}{\nameref{sec:0717}}
\begin{longtable}{l p{0.5cm} r}
آن کس که به بندگیت آید
&&
با او تو چنین کنی نشاید
\\
ای روی تو خوب و خوی تو خوش
&&
چون تو گهری فلک نزاید
\\
روی تو و خوی تو لطیفست
&&
سر دل تو لطیف باید
\\
آن شخص که مردنیست فردا
&&
امروز چرا جفا نماید
\\
چیزی که به خود نمی‌پسندد
&&
آن بر دگری چه آزماید
\\
از خشم مخای هیچ کس را
&&
تا خشم خدا تو را نخاید
\\
برخیز ز قصد خون خلقان
&&
تا بر سر تو فرونیاید
\\
آن گاه قضا ز تو بگردد
&&
کان وسوسه در دلت نیاید
\\
ای گفته که مردم این چه مردیست
&&
کابلیس تو را چنین بگاید
\\
\end{longtable}
\end{center}
