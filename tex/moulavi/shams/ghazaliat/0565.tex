\begin{center}
\section*{غزل شماره ۵۶۵: اگر صد همچو من گردد هلاک او را چه غم دارد}
\label{sec:0565}
\addcontentsline{toc}{section}{\nameref{sec:0565}}
\begin{longtable}{l p{0.5cm} r}
اگر صد همچو من گردد هلاک او را چه غم دارد
&&
که نی عاشق نمی‌یابد که نی دلخسته کم دارد
\\
مرا گوید چرا چشمت رقیب روی من باشد
&&
بدان در پیش خورشیدش همی‌دارم که نم دارد
\\
چو اسماعیل پیش او بنوشم زخم نیش او
&&
خلیلم را خریدارم چه گر قصد ستم دارد
\\
اگر مشهور شد شورم خدا داند که معذورم
&&
کاسیر حکم آن عشقم که صد طبل و علم دارد
\\
مرا یار شکرناکم اگر بنشاند بر خاکم
&&
چرا غم دارد آن مفلس که یار محتشم دارد
\\
غمش در دل چو گنجوری دلم نور علی نوری
&&
مثال مریم زیبا که عیسی در شکم دارد
\\
چو خورشیدست یار من نمی‌گردد به جز تنها
&&
سپه سالار مه باشد کز استاره حشم دارد
\\
مسلمان نیستم گبرم اگر ماندست یک صبرم
&&
چه دانی تو که درد او چه دستان و قدم دارد
\\
ز درد او دهان تلخست هر دریا که می‌بینی
&&
ز داغ او نکو بنگر که روی مه رقم دارد
\\
به دوران‌ها چو من عاشق نرست از مغرب و مشرق
&&
بپرس از پیر گردونی که چون من پشت خم دارد
\\
خنک جانی که از خوابش به مالش‌ها برانگیزد
&&
بدان مالش بود شادان و آن را مغتنم دارد
\\
طبیبی چون دهد تلخش بنوشد تلخ او را خوش
&&
طبیبان را نمی‌شاید که عاقل متهم دارد
\\
اگر شان متهم داری بمانی بند بیماری
&&
کسی برخورد از استا که او را محترم دارد
\\
خمش کن کاندر این دریا نشاید نعره و غوغا
&&
که غواص آن کسی باشد که او امساک دم دارد
\\
\end{longtable}
\end{center}
