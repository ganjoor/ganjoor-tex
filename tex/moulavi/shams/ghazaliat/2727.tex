\begin{center}
\section*{غزل شماره ۲۷۲۷: گر وسوسه ره دهی به گوشی}
\label{sec:2727}
\addcontentsline{toc}{section}{\nameref{sec:2727}}
\begin{longtable}{l p{0.5cm} r}
گر وسوسه ره دهی به گوشی
&&
افسرده شوی بدان ز جوشی
\\
آن گرمی چشم را که داری
&&
نیش زهر است و شکل نوشی
\\
انبار نعیم را زیان چیست
&&
گر خشم گرفت کورموشی
\\
آخر چه زیان اگر بیفتد
&&
یک دو مگس از شکرفروشی
\\
مر ناقه شیر را چه نقصان
&&
گر دیگ شکست شیردوشی
\\
شب بود و زمانه خفته بودند
&&
در هیچ سری نبود هوشی
\\
آن شاه ز روی لطف برداشت
&&
سرنای و در او بزد خروشی
\\
در خون خودی اگر بمانی
&&
زین پس زان رو به روی پوشی
\\
ماییم ز عشق شمس تبریز
&&
هم ناطق عشق هم خموشی
\\
\end{longtable}
\end{center}
