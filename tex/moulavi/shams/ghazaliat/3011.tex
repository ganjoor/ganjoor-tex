\begin{center}
\section*{غزل شماره ۳۰۱۱: هر نفسی از درون دلبر روحانیی}
\label{sec:3011}
\addcontentsline{toc}{section}{\nameref{sec:3011}}
\begin{longtable}{l p{0.5cm} r}
هر نفسی از درون دلبر روحانیی
&&
عربده آرد مرا از ره پنهانیی
\\
فتنه و ویرانیم شور و پریشانیم
&&
برد مسلمانیم وای مسلمانیی
\\
گفت مرا می خوری یا چه گمان می‌بری
&&
کیست برون از گمان جز دل ربانیی
\\
بر سر افسانه رو مست سوی خانه رو
&&
جان بفشان کان نگار کرد گل افشانیی
\\
یک دم ای خوش عذار حال مرا گوش دار
&&
مست غمت را بیار رسم نگهبانیی
\\
عابد و معبود من شاهد و مشهود من
&&
عشق شناس ای حریف در دل انسانیی
\\
کعبه ما کوی او قبله ما روی او
&&
رهبر ما بوی او در ره سلطانیی
\\
خواجه صاحب نظر الحذر از ما حذر
&&
تا ننهد خواجه سر در خطر جانیی
\\
نی غلطم سر بیار تا ببری صد هزار
&&
گل ندمد جز ز خار گنج به ویرانیی
\\
آمد آن شیر من عاشق جان سیر من
&&
در کف او شیشه‌ای شکل پری خوانیی
\\
گفتم ای روح قدس آخر ما را بپرس
&&
گفت چه پرسم دریغ حال مرا دانیی
\\
مستم و گم کرده راه تن زن و پرسش مخواه
&&
مست چه‌ام بوی گیر باده جانانیی
\\
کی بود آن ای خدا ما شده از ما جدا
&&
برده قماشات ما غارت سبحانیی
\\
هر کی ورا کار کیست در کف او خارکیست
&&
هر کی ورا یار کیست هست چو زندانیی
\\
کارک تو هم تویی یارک تو هم تویی
&&
هر کی ز خود دور شد نیست به جز فانیی
\\
\end{longtable}
\end{center}
