\begin{center}
\section*{غزل شماره ۴۲۷: در دل و جان خانه کردی عاقبت}
\label{sec:0427}
\addcontentsline{toc}{section}{\nameref{sec:0427}}
\begin{longtable}{l p{0.5cm} r}
در دل و جان خانه کردی عاقبت
&&
هر دو را دیوانه کردی عاقبت
\\
آمدی کاتش در این عالم زنی
&&
وانگشتی تا نکردی عاقبت
\\
ای ز عشقت عالمی ویران شده
&&
قصد این ویرانه کردی عاقبت
\\
من تو را مشغول می‌کردم دلا
&&
یاد آن افسانه کردی عاقبت
\\
عشق را بی‌خویش بردی در حرم
&&
عقل را بیگانه کردی عاقبت
\\
یا رسول الله ستون صبر را
&&
استن حنانه کردی عاقبت
\\
شمع عالم بود لطف چاره گر
&&
شمع را پروانه کردی عاقبت
\\
یک سرم این سوست یک سر سوی تو
&&
دوسرم چون شانه کردی عاقبت
\\
دانه‌ای بیچاره بودم زیر خاک
&&
دانه را دردانه کردی عاقبت
\\
دانه‌ای را باغ و بستان ساختی
&&
خاک را کاشانه کردی عاقبت
\\
ای دل مجنون و از مجنون بتر
&&
مردی و مردانه کردی عاقبت
\\
کاسه سر از تو پر از تو تهی
&&
کاسه را پیمانه کردی عاقبت
\\
جان جانداران سرکش را به علم
&&
عاشق جانانه کردی عاقبت
\\
شمس تبریزی که مر هر ذره را
&&
روشن و فرزانه کردی عاقبت
\\
\end{longtable}
\end{center}
