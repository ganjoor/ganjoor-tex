\begin{center}
\section*{غزل شماره ۳۰۸۱: تو عاشقی چه کسی از کجا رسیدستی}
\label{sec:3081}
\addcontentsline{toc}{section}{\nameref{sec:3081}}
\begin{longtable}{l p{0.5cm} r}
تو عاشقی چه کسی از کجا رسیدستی
&&
مرا چه می‌نگری کژ به شب خریدستی
\\
چه ظلم کردم بر تو که چون ستم زدگان
&&
کله زدی به زمین بر قبا دریدستی
\\
تظلمی به سلف می‌کنی مگر پیشین
&&
که داغ و درد و غم عاشقان شنیدستی
\\
غلط ز رنگ تو پیداست ز آل یعقوبی
&&
بدیده رخ یوسف که کف بریدستی
\\
ز تیر غمزه دلدار اگر نخست دلت
&&
چرا ز غصه و غم چون کمان خمیدستی
\\
ز آه و ناله تو بوی مشک می‌آید
&&
یقین تو آهوی نافی سمن چریدستی
\\
تو هر چه هستی می‌باش یک سخن بشنو
&&
اگر چه میوه حکمت بسی بچیدستی
\\
حدیث جان توست این و گفت من چو صداست
&&
اگر تو شیخ شیوخی وگر مریدستی
\\
تو خویش درد گمان برده‌ای و درمانی
&&
تو خویش قفل گمان برده‌ای کلیدستی
\\
اگر ز وصف تو دزدم تو شحنه عقلی
&&
وگر تمام بگویم ابایزیدستی
\\
دریغ از تو که در آرزوی غیری تو
&&
جمال خویش ندیدی که بی‌ندیدستی
\\
تو را کسی بشناسد که اوت کسی کرده‌ست
&&
دگر کیست نداند که ناپدیدستی
\\
دلا برو بر یار و مباش بسته خویش
&&
که سایح و سبک و چابک و جریدستی
\\
به ترک مصر بگفتی ز شومی فرعون
&&
بر شعیب چو موسی فروخزیدستی
\\
چون عمر ماست حدیثش دراز اولیتر
&&
چنین درازسخن را بدان کشیدستی
\\
همی‌دوم پی ظل تو شمس تبریزی
&&
مگر منم عرفه تو مگر که عیدستی
\\
\end{longtable}
\end{center}
