\begin{center}
\section*{غزل شماره ۸۲۰: هر چه آن خسرو کند شیرین کند}
\label{sec:0820}
\addcontentsline{toc}{section}{\nameref{sec:0820}}
\begin{longtable}{l p{0.5cm} r}
هر چه آن خسرو کند شیرین کند
&&
چون درخت تین که جمله تین کند
\\
هر کجا خطبه بخواند بر دو ضد
&&
همچو شیر و شهدشان کابین کند
\\
با دم او می‌رود عین الحیات
&&
مرده جان یابد چو او تلقین کند
\\
مرغ جان‌ها با قفس‌ها برپرند
&&
چونک بنده پروری آیین کند
\\
عالمی بخشد به هر بنده جدا
&&
کیست کو اندر دو عالم این کند
\\
گر به قعر چاه نام او بری
&&
قعر چه را صدر علیین کند
\\
من بر آنم که شکرریزی کنم
&&
از شکر گر قسم من تعیین کند
\\
کافری گر لاف عشق او زند
&&
کفر او را جمله نور دین کند
\\
خار عالم در ره عاشق نهاد
&&
تا که جمله خار را نسرین کند
\\
تو نمی‌دانی که هر که مرغ اوست
&&
از سعادت بیضه‌ها زرین کند
\\
بس کنم زین پس نهان گویم دعا
&&
کی نهان ماند چو شه آمین کند
\\
\end{longtable}
\end{center}
