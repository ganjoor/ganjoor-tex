\begin{center}
\section*{غزل شماره ۲۲۰۳: جسم و جان با خود نخواهم خانه خمار کو}
\label{sec:2203}
\addcontentsline{toc}{section}{\nameref{sec:2203}}
\begin{longtable}{l p{0.5cm} r}
جسم و جان با خود نخواهم خانه خمار کو
&&
لایق این کفر نادر در جهان زنار کو
\\
هر زمان چون مست گردد از نسیم خمر جان
&&
تا در خمخانه می‌تازد ولیکن بار کو
\\
سوی بی‌گوشی سماع چنگ می‌آید ولیک
&&
چنگ جانان است آن را چوب یا اوتار کو
\\
چونک او بی‌تن شود پس خلعت جان آورند
&&
کاندر او دستان حایک یا که پود و تار کو
\\
کبر عاشق بوی کن کان خود به معنی خاکیی است
&&
در چنان دریا تکبر یا که ننگ و عار کو
\\
چون مشامت برگشاید آیدت از غار عشق
&&
طرفه بویی پس دوی هر سو که آخر غار کو
\\
رنگ بی‌رنگی است از رخسار عاشق آن صفا
&&
آن وفا و آن صفا و لطف خوش رخسار کو
\\
آمدت مژده ز عمر سرمدی پس حمد کو
&&
کاندر آن عمرت غم امسال و یاد پار کو
\\
صحبت ابرار و هم اشرار کان جا زحمت است
&&
در حریم سایه آن مهتر اخیار کو
\\
شمس حق و دین خداوند صفاهای ابد
&&
در شعاع آفتابش ذره هشیار کو
\\
\end{longtable}
\end{center}
