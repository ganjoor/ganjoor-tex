\begin{center}
\section*{غزل شماره ۹۴۹: مرا وصال تو باید صبا چه سود کند}
\label{sec:0949}
\addcontentsline{toc}{section}{\nameref{sec:0949}}
\begin{longtable}{l p{0.5cm} r}
مرا وصال تو باید صبا چه سود کند
&&
چو من زمین تو گشتم سما چه سود کند
\\
ایا بتان شکرلب چو روی شه دیدم
&&
مرا جمال و کمال شما چه سود کند
\\
دلم نماند و گدازید چون شکر در آب
&&
جمال ماه رخ دلربا چه سود کند
\\
فلک ببست میان مرا ز فضل کمر
&&
ولیک بی‌شه شهره قبا چه سود کند
\\
هزار حیله کنم من دغا و شیوه عشق
&&
چو شه حریف نباشد دغا چه سود کند
\\
مرا بقا و فنا از برای خدمت اوست
&&
مرا چو آن نبود این بقا چه سود کند
\\
سقا و آب برای حرارت جگرست
&&
جگر چو خون شد ای دل سقا چه سود کند
\\
فلک به ناله شد از بس دعا و زاری من
&&
چو بخت یار نباشد دعا چه سود کند
\\
مگو چنین تو چه دانی بلادریست نهان
&&
خدای داند و بس کاین بلا چه سود کند
\\
چو خونبهای تو ای دل هوای عشق ویست
&&
مگو که کشته شدم خونبها چه سود کند
\\
تو هان و هان به دل و دیده خاک این ره شو
&&
چو خاک باشی باید علا چه سود کند
\\
در آن فلک که شعاعات آفتاب دلست
&&
هزار سایه و ظل هما چه سود کند
\\
هما و سایه‌اش آن جا چو ظلمتی باشد
&&
ز نور ظلمت غیر فنا چه سود کند
\\
دلا تو چند زنی لاف از وفاداری
&&
برو به بحر وفا این وفا چه سود کند
\\
صفای باقی باید که بر رخت تابد
&&
تو جندره زده گیر این صفا چه سود کند
\\
چو کبر را بگذاری صفا ز حق یابی
&&
بدانی آنگه کاین کبریا چه سود کند
\\
برو به نزد خداوند شمس تبریزی
&&
فقیر او شو جانا غنا چه سود کند
\\
\end{longtable}
\end{center}
