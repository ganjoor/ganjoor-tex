\begin{center}
\section*{غزل شماره ۶۷: از آن مایی ای مولا اگر امروز اگر فردا}
\label{sec:0067}
\addcontentsline{toc}{section}{\nameref{sec:0067}}
\begin{longtable}{l p{0.5cm} r}
از آن مایی ای مولا اگر امروز اگر فردا
&&
شب و روزم ز تو روشن زهی رعنا زهی زیبا
\\
تو پاک پاکی از صورت ولیک از پرتو نورت
&&
نمایی صورتی هر دم چه باحسن و چه بابالا
\\
چو ابرو را چنین کردی چه صورت‌های چین کردی
&&
مرا بی‌عقل و دین کردی بر آن نقش و بر آن حورا
\\
مرا گویی چه عشقست این که نی بالا نه پستست این
&&
چه صیدی بی ز شستست این درون موج این دریا
\\
ایا معشوق هر قدسی چو می‌دانی چه می‌پرسی
&&
که سر عرش و صد کرسی ز تو ظاهر شود پیدا
\\
زدی در من یکی آتش که شد جان مرا مفرش
&&
که تا آتش شود گل خوش که تا یکتا شود صد تا
\\
فرست آن عشق ساقی را بگردان جام باقی را
&&
که از مزج و تلاقی را ندانم جامش از صهبا
\\
بکن این رمز را تعیین بگو مخدوم شمس الدین
&&
به تبریز نکوآیین ببر این نکته غرا
\\
\end{longtable}
\end{center}
