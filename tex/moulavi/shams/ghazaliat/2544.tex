\begin{center}
\section*{غزل شماره ۲۵۴۴: شنیدم کاشتری گم شد ز کردی در بیابانی}
\label{sec:2544}
\addcontentsline{toc}{section}{\nameref{sec:2544}}
\begin{longtable}{l p{0.5cm} r}
شنیدم کاشتری گم شد ز کردی در بیابانی
&&
بسی اشتر بجست از هر سوی کرد بیابانی
\\
چو اشتر را ندید از غم بخفت اندر کنار ره
&&
دلش از حسرت اشتر میان صد پریشانی
\\
در آخر چون درآمد شب بجست از خواب و دل پرغم
&&
برآمد گوی مه تابان ز روی چرخ چوگانی
\\
به نور مه بدید اشتر میان راه استاده
&&
ز شادی آمدش گریه به سان ابر نیسانی
\\
رخ اندر ماه روشن کرد و گفتا چون دهم شرحت
&&
که هم خوبی و نیکویی و هم زیبا و تابانی
\\
خداوندا در این منزل برافروز از کرم نوری
&&
که تا گم کرده خود را بیابد عقل انسانی
\\
شب قدر است در جانب چرا قدرش نمی‌دانی
&&
تو را می‌شورد او هر دم چرا او را نشورانی
\\
تو را دیوانه کرده‌ست او قرار جانت برده‌ست او
&&
غم جان تو خورده‌ست او چرا در جانش ننشانی
\\
چو او آب است و تو جویی چرا خود را نمی‌جویی
&&
چو او مشک است و تو بویی چرا خود را نیفشانی
\\
\end{longtable}
\end{center}
