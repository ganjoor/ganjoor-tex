\begin{center}
\section*{غزل شماره ۲۰۶۴: باز فروریخت عشق از در و دیوار من}
\label{sec:2064}
\addcontentsline{toc}{section}{\nameref{sec:2064}}
\begin{longtable}{l p{0.5cm} r}
باز فروریخت عشق از در و دیوار من
&&
باز ببرید بند اشتر کین دار من
\\
بار دگر شیر عشق پنجه خونین گشاد
&&
تشنه خون گشت باز این دل سگسار من
\\
باز سر ماه شد نوبت دیوانگی است
&&
آه که سودی نکرد دانش بسیار من
\\
بار دگر فتنه زاد جمره دیگر فتاد
&&
خواب مرا بست باز دلبر بیدار من
\\
صبر مرا خواب برد عقل مرا آب برد
&&
کار مرا یار برد تا چه شود کار من
\\
سلسله عاشقان با تو بگویم که چیست
&&
آنک مسلسل شود طره دلدار من
\\
خیز دگربار خیز خیز که شد رستخیز
&&
مایه صد رستخیز شور دگربار من
\\
گر ز خزان گلستان چون دل عاشق بسوخت
&&
نک رخ آن گلستان گلشن و گلزار من
\\
باغ جهان سوخته باغ دل افروخته
&&
سوخته اسرار باغ ساخته اسرار من
\\
نوبت عشرت رسید ای تن محبوس من
&&
خلعت صحت رسید ای دل بیمار من
\\
پیر خرابات هین از جهت شکر این
&&
رو گرو می‌بنه خرقه و دستار من
\\
خرقه و دستار چیست این نه ز دون همتی است
&&
جان و جهان جرعه‌ای است از شه خمار من
\\
داد سخن دادمی سوسن آزادمی
&&
لیک ز غیرت گرفت دل ره گفتار من
\\
شکر که آن ماه را هر طرفی مشتری است
&&
نیست ز دلال گفت رونق بازار من
\\
عربده قال نیست حاجت دلال نیست
&&
جعفر طرار نیست جعفر طیار من
\\
\end{longtable}
\end{center}
