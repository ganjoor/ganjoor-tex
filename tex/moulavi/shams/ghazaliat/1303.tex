\begin{center}
\section*{غزل شماره ۱۳۰۳: گر تو تنگ آیی ز ما زوتر برون رو ای حریف}
\label{sec:1303}
\addcontentsline{toc}{section}{\nameref{sec:1303}}
\begin{longtable}{l p{0.5cm} r}
گر تو تنگ آیی ز ما زوتر برون رو ای حریف
&&
کز ترش رویی همی‌رنجد دلارام ظریف
\\
گر همی انکار خود پنهان کنی بر روی تو
&&
می‌نماید دشمنی‌ها بر رخ تو لیف لیف
\\
روز گردک بر رخ داماد می‌باشد نشان
&&
از جمال او که نامش کرد رومی نیف نیف
\\
چون خداوند شمس دین چوگان زند یارش کجاست
&&
ور بر اسب فضل بنشیند کجا دارد ردیف
\\
خوان و بزم هر دو عالم نزد بزم شمس دین
&&
چون یکی کاسه پرآش و بر سر او یک رغیف
\\
وان رغیف و آش و کاسه صدقه تبریز دان
&&
از کمال و حرمت شهر شهنشاه شریف
\\
\end{longtable}
\end{center}
