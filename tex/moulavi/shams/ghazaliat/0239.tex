\begin{center}
\section*{غزل شماره ۲۳۹: کرانی ندارد بیابان ما}
\label{sec:0239}
\addcontentsline{toc}{section}{\nameref{sec:0239}}
\begin{longtable}{l p{0.5cm} r}
کرانی ندارد بیابان ما
&&
قراری ندارد دل و جان ما
\\
جهان در جهان نقش و صورت گرفت
&&
کدامست از این نقش‌ها آن ما
\\
چو در ره ببینی بریده سری
&&
که غلطان رود سوی میدان ما
\\
از او پرس از او پرس اسرار ما
&&
کز او بشنوی سر پنهان ما
\\
چه بودی که یک گوش پیدا شدی
&&
حریف زبان‌های مرغان ما
\\
چه بودی که یک مرغ پران شدی
&&
برو طوق سر سلیمان ما
\\
چه گویم چه دانم که این داستان
&&
فزونست از حد و امکان ما
\\
چگونه زنم دم که هر دم به دم
&&
پریشانترست این پریشان ما
\\
چه کبکان و بازان ستان می‌پرند
&&
میان هوای کهستان ما
\\
میان هوایی که هفتم هواست
&&
که بر اوج آنست ایوان ما
\\
از این داستان بگذر از من مپرس
&&
که درهم شکستست دستان ما
\\
صلاح الحق و دین نماید تو را
&&
جمال شهنشاه و سلطان ما
\\
\end{longtable}
\end{center}
