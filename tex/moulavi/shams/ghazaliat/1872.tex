\begin{center}
\section*{غزل شماره ۱۸۷۲: ای در غم بیهوده رو کم ترکوا برخوان}
\label{sec:1872}
\addcontentsline{toc}{section}{\nameref{sec:1872}}
\begin{longtable}{l p{0.5cm} r}
ای در غم بیهوده رو کم ترکوا برخوان
&&
وی حرص تو افزوده رو کم ترکوا برخوان
\\
از اسپک و از زینک پربادک و پرکینک
&&
وز غصه بیالوده رو کم ترکوا برخوان
\\
در روده و سرگینی باد هوس و کینی
&&
ای غافل آلوده رو کم ترکوا برخوان
\\
ای شیخ پر از دعوی وی صورت بی‌معنی
&&
نابوده و بنموده رو کم ترکوا برخوان
\\
منگر که شه و میری بنگر که همی‌میری
&&
در زیر یکی توده رو کم ترکوا برخوان
\\
آن نازک و آن مشتک آن ما و من زشتک
&&
پوسیده و فرسوده رو کم ترکوا برخوان
\\
رخ بر رخ زیبایان کم نه بنگر پایان
&&
رخسار تو فرسوده رو کم ترکوا برخوان
\\
گر باغ و سرا داری با مرگ چه پا داری
&&
در گور گل اندوده رو کم ترکوا برخوان
\\
رفتند جهان داران خون خواره و عیاران
&&
بر خلق نبخشوده رو کم ترکوا برخوان
\\
تابوت کسان دیده وز دور بخندیده
&&
وان چشم تو نگشوده رو کم ترکوا برخوان
\\
بس کن ز سخن گویی از گفت چه می جویی
&&
ای بادبپیموده رو کم ترکوا برخوان
\\
\end{longtable}
\end{center}
