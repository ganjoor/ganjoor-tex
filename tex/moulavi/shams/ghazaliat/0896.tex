\begin{center}
\section*{غزل شماره ۸۹۶: غره مشو گر ز چرخ کار تو گردد بلند}
\label{sec:0896}
\addcontentsline{toc}{section}{\nameref{sec:0896}}
\begin{longtable}{l p{0.5cm} r}
غره مشو گر ز چرخ کار تو گردد بلند
&&
زانک بلندت کند تا بتواند فکند
\\
قطره آب منی کز حیوان می‌زهد
&&
لایق قربان نشد تا نشد آن گوسفند
\\
توده ذرات ریگ تا نشود کوه سخت
&&
کس نزند بر سرش بیهده زخم کلند
\\
تا نشود گردنی گردن کس غل ندید
&&
تا نشود پا روان کس نشود پای بند
\\
پس سبقت رحمتی در غضبی شد پدید
&&
زهر بدان کس دهند کوست معود به قند
\\
برگ که رست از زمین تا که درختی نشد
&&
آتش نفروزد او شعله نگردد بلند
\\
باش چو رز میوه دار زور و بلندی مجو
&&
از پی خرما بدانک خار ورا کس نکند
\\
از پی میوه ضعیف رسته درختان زفت
&&
نقش درختان شگرف صورت میوه نژند
\\
دل مثل اولیاست استن جسم جهان
&&
جسم به دل قایمست بی‌خلل و بی‌گزند
\\
قوت جسم پدید هست دل ناپدید
&&
تا به کی انکار غیب غیب نگر چند چند
\\
\end{longtable}
\end{center}
