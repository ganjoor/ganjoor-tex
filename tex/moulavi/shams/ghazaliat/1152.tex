\begin{center}
\section*{غزل شماره ۱۱۵۲: ببین دلی که نگردد ز جان سپاری سیر}
\label{sec:1152}
\addcontentsline{toc}{section}{\nameref{sec:1152}}
\begin{longtable}{l p{0.5cm} r}
ببین دلی که نگردد ز جان سپاری سیر
&&
اسیر عشق نگردد ز رنج و خواری سیر
\\
ز زخم‌های نهانی که عاشقان دانند
&&
به خون درست و نگردد ز زخم کاری سیر
\\
مقیم شد به خرابات و جمله رندان را
&&
خراب کرد و نشد از شراب باری سیر
\\
هزار جان مقدس سپرد هر نفسی
&&
در آن شکار و نشد جان از آن شکاری سیر
\\
مثال نی ز لب یار کام پرشکرست
&&
ولیک نیست چو نی از فغان و زاری سیر
\\
بگفت تو ز چه سیری بگفتم از جز تو
&&
ولیک هیچ نگردم از آنچ داری سیر
\\
نه شهر و یار شناسیم ای مسلمانان
&&
از آنک نیست دل از جام شهریاری سیر
\\
هوای تو چو بهارست و دل ز توست چو باغ
&&
که باغ می‌نشود از دم بهاری سیر
\\
چو شرمسارم از احسان شمس تبریزی
&&
که جان مباد از این شرم و شرمساری سیر
\\
\end{longtable}
\end{center}
