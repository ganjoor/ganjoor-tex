\begin{center}
\section*{غزل شماره ۱۱۷: از دور بدیده شمس دین را}
\label{sec:0117}
\addcontentsline{toc}{section}{\nameref{sec:0117}}
\begin{longtable}{l p{0.5cm} r}
از دور بدیده شمس دین را
&&
فخر تبریز و رشک چین را
\\
آن چشم و چراغ آسمان را
&&
آن زنده کننده زمین را
\\
ای گشته چنان و آن چنانتر
&&
هر جان که بدیده او چنین را
\\
گفتا که که را کشم به زاری
&&
گفتمش که بنده کمین را
\\
این گفتن بود و ناگهانی
&&
از غیب گشاد او کمین را
\\
آتش درزد به هست بنده
&&
وز بیخ بکند کبر و کین را
\\
بی دل سیهی لاله زان می
&&
سرمست بکرد یاسمین را
\\
در دامن اوست عین مقصود
&&
بر ما بفشاند آستین را
\\
شاهی که چو رخ نمود مه را
&&
بر اسب فلک نهاد زین را
\\
بنشین کژ و راست گو که نبود
&&
همتا شه روح راستین را
\\
والله که از او خبر نباشد
&&
جبریل مقدس امین را
\\
حالی چه زند به قال آورد
&&
او چرخ بلند هفتمین را
\\
چون چشم دگر در او گشادیم
&&
یک جو نخریم ما یقین را
\\
آوه که بکرد بازگونه
&&
آن دولت وصل پوستین را
\\
ای مطرب عشق شمس دینم
&&
جان تو که بازگو همین را
\\
چون می‌نرسم به دستبوسش
&&
بر خاک همی‌زنم جبین را
\\
\end{longtable}
\end{center}
