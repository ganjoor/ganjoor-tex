\begin{center}
\section*{غزل شماره ۴۳۸: هر دم سلام آرد کاین نامه از فلانست}
\label{sec:0438}
\addcontentsline{toc}{section}{\nameref{sec:0438}}
\begin{longtable}{l p{0.5cm} r}
هر دم سلام آرد کاین نامه از فلانست
&&
گویی سلام و کاغذ در شهر ما گرانست
\\
زین مرگ هیچ کوسه ارزان نبرد بوسه
&&
بینی دراز کردن آیین نر خرانست
\\
هر جا که سیمبر بد می‌دانک سیم بر بد
&&
جان و جهان مگویش کان جان ز تو جهانست
\\
بتراش زر به ناخن از کان و چاره‌ای کن
&&
پنهان مدار زر را بی‌زر صنم نهانست
\\
گر حلقه زر نبودی در گوش او نرفتی
&&
در گوش حلقه زر بر طمع او نشانست
\\
ور زانک نازنینی بی‌سیم و زر ببینی
&&
چونک عنایت آمد اقبال رایگانست
\\
این یار زر نگیرد جانی بیار زرین
&&
زیرا که زر مرده آن سوی ناروانست
\\
سنگی است سرخ گشته صد تخم فتنه کشته
&&
مغرور زر پخته خام است و قلتبانست
\\
خامش سخن چه باید آن جا که عشق آید
&&
کمتر ز زر نباشی معشوق بی‌زبانست
\\
\end{longtable}
\end{center}
