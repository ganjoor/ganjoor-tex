\begin{center}
\section*{غزل شماره ۵۸۸: صلا رندان دگرباره که آن شاه قمار آمد}
\label{sec:0588}
\addcontentsline{toc}{section}{\nameref{sec:0588}}
\begin{longtable}{l p{0.5cm} r}
صلا رندان دگرباره که آن شاه قمار آمد
&&
اگر تلبیس نو دارد همانست او که پار آمد
\\
ز رندان کیست این کاره که پیش شاه خون خواره
&&
میان بندد دگرباره که اینک وقت کار آمد
\\
بیا ساقی سبک دستم که من باری میان بستم
&&
به جان تو که تا هستم مرا عشق اختیار آمد
\\
چو گلزار تو را دیدم چو خار و گل بروییدم
&&
چو خارم سوخت در عشقت گلم بر تو نثار آمد
\\
پیاپی فتنه انگیزی ز فتنه بازنگریزی
&&
ولیک این بار دانستم که یار من عیار آمد
\\
اگر بر رو زند یارم رخی دیگر به پیش آرم
&&
ازیرا رنگ رخسارم ز دستش آبدار آمد
\\
تویی شاها و دیرینه مقام تست این سینه
&&
نمی‌گویی کجا بودی که جان بی‌تو نزار آمد
\\
شهم گوید در این دشتم تو پنداری که گم گشتم
&&
نمی‌دانی که صبر من غلاف ذوالفقار آمد
\\
مرا برید و خون آمد غزل پرخون برون آمد
&&
برید از من صلاح الدین به سوی آن دیار آمد
\\
\end{longtable}
\end{center}
