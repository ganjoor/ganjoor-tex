\begin{center}
\section*{غزل شماره ۶۴۵: بار دگر آن آب به دولاب درآمد}
\label{sec:0645}
\addcontentsline{toc}{section}{\nameref{sec:0645}}
\begin{longtable}{l p{0.5cm} r}
بار دگر آن آب به دولاب درآمد
&&
وان چرخه گردنده در اشتاب درآمد
\\
بار دگر آن جان پر از آتش و از آب
&&
در لرزه چو خورشید و چو سیماب درآمد
\\
بار دگر آن صورت پنهانی عالم
&&
از روزن جان دوش چو مهتاب درآمد
\\
خورشید که می‌درد از او مشرق و مغرب
&&
از لطف بود گر به سطرلاب درآمد
\\
بار دگر آن صبح بخندید و بتابید
&&
تا خفته صدساله هم از خواب درآمد
\\
بار دگر آن قاضی حاجات ندا کرد
&&
خیزید که آن فاتح ابواب درآمد
\\
بار دگر از قبله روان گشت رسالت
&&
در گوش محمد چو به محراب درآمد
\\
چون رفت محمد به در خیبر ناسوت
&&
نقبی بزد از نصرت و نقاب درآمد
\\
از بیم ملک جمله فلک رخنه و در شد
&&
وز بیم مسبب همه اسباب درآمد
\\
آری لقبش بود سعادت بک عالم
&&
زان پیش که اشخاص به القاب درآمد
\\
بگشاد محمد در خمخانه غیبی
&&
بسیار کسادی به می ناب درآمد
\\
از بهر دل تشنه و تسکین چنین خون
&&
آن جام می لعل چو عناب درآمد
\\
خاموش کن امروز که این روز سخن نیست
&&
زحمت مده آن ساقی اصحاب درآمد
\\
\end{longtable}
\end{center}
