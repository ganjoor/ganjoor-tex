\begin{center}
\section*{غزل شماره ۲۶۰۲: ای دشمن عقل من وی داروی بی‌هوشی}
\label{sec:2602}
\addcontentsline{toc}{section}{\nameref{sec:2602}}
\begin{longtable}{l p{0.5cm} r}
ای دشمن عقل من وی داروی بی‌هوشی
&&
من خابیه تو در من چون باده همی‌جوشی
\\
اول تو و آخر تو بیرون تو و در سر تو
&&
هم شاهی و سلطانی هم حاجب و چاووشی
\\
خوش خویی و بدخویی دلسوزی و دلجویی
&&
هم یوسف مه رویی هم مانع و روپوشی
\\
بس تازه و بس سبزی بس شاهد و بس نغزی
&&
چون عقل در این مغزی چون حلقه در این گوشی
\\
هم دوری و هم خویشی هم پیشی و هم بیشی
&&
هم مار بداندیشی هم نیشی و هم نوشی
\\
ای رهزن بی‌خویشان ای مخزن درویشان
&&
یا رب چه خوشند ایشان آن دم که در آغوشی
\\
آن روز که هشیارم من عربده‌ها دارم
&&
و آن روز که خمارم چه صبر و چه خاموشی
\\
\end{longtable}
\end{center}
