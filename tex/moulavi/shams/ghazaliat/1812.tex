\begin{center}
\section*{غزل شماره ۱۸۱۲: ای نور افلاک و زمین چشم و چراغ غیب بین}
\label{sec:1812}
\addcontentsline{toc}{section}{\nameref{sec:1812}}
\begin{longtable}{l p{0.5cm} r}
ای نور افلاک و زمین چشم و چراغ غیب بین
&&
ای تو چنین و صد چنین مخدوم جانم شمس دین
\\
تا غمزه‌ات خون ریز شد وان زلف عنبربیز شد
&&
جان بنده تبریز شد مخدوم جانم شمس دین
\\
خورشید جان همچون شفق در مکتب تو نوسبق
&&
ای بنده‌ات خاصان حق مخدوم جانم شمس دین
\\
ای بحر اقبال و شرف صد ماه و شاهت در کنف
&&
برداشتم پیش تو کف مخدوم جانم شمس دین
\\
ای هم ملوک و هم ملک در پیشت ای نور فلک
&&
از همدگر مسکینترک مخدوم جانم شمس دین
\\
مطلوب جمله جان‌ها جان را سوی اجلال‌ها
&&
تو داده پر و بال‌ها مخدوم جانم شمس دین
\\
دل را ز تو حالی دگر در سلطنت قالی دگر
&&
تا پرد از بالی دگر مخدوم جانم شمس دین
\\
\end{longtable}
\end{center}
