\begin{center}
\section*{غزل شماره ۲۸۱۳: اقتلونی یا ثقاتی ان فی قتلی حیاتی}
\label{sec:2813}
\addcontentsline{toc}{section}{\nameref{sec:2813}}
\begin{longtable}{l p{0.5cm} r}
اقتلونی یا ثقاتی ان فی قتلی حیاتی
&&
و مماتی فی حیاتی و حیاتی فی مماتی
\\
اقتلونی ذاب جسمی قدح القهوه قسمی
&&
هله بشکن قفس ای جان چو طلبکار نجاتی
\\
ز سفر بدر شوی تو چو یقین ماه نوی تو
&&
ز شکست از چه تو تلخی چو همه قند و نباتی
\\
چو تویی یار مرا تو به از این دار مرا تو
&&
برسان قوت حیاتم که تو یاقوت زکاتی
\\
چو بسی قحط کشیدم بنما دعوت عیدم
&&
که نشد سیر دو چشمم به تره و نان براتی
\\
حرکت کن حرکت‌هاست کلید در روزی
&&
مگرت نیست خبر تو که چه زیباحرکاتی
\\
به چنین رخ که تو داری چه کشی ناز سپیده
&&
که نگنجد به صفت در که چه محمودصفاتی
\\
بنه‌ای ساقی اسعد تو یکی بزم مخلد
&&
که خماری است جهان را ز می و بزم نباتی
\\
به حق بحر کف تو گهر باشرف تو
&&
که به لطف و به گوارش تو به از آب فراتی
\\
مثل ساغر آخر تو خرابی عقولی
&&
که چو تحریمه اول سر ارکان صلاتی
\\
کرمت مست برآید کف چون بحر گشاید
&&
بدهد صدقه نپرسد که تو اهل صدقاتی
\\
به کرم فاتح عقدی به عطا نقده نقدی
&&
برهان منتظران را ز تمنای سباتی
\\
نه در ابروی تو چینی نه در آن خوی تو کینی
&&
به عدو گوید لطفت که بنینی و بناتی
\\
رسی از ساغر مردان به خیالات مصور
&&
ز ره سینه خرامان کنساء خفرات
\\
و جوار ساقیات و سواق جاریات
&&
تو بگو باقی این را انا فی سکر سقاتی
\\
\end{longtable}
\end{center}
