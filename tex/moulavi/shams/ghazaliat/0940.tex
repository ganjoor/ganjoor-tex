\begin{center}
\section*{غزل شماره ۹۴۰: ربود عشق تو تسبیح و داد بیت و سرود}
\label{sec:0940}
\addcontentsline{toc}{section}{\nameref{sec:0940}}
\begin{longtable}{l p{0.5cm} r}
ربود عشق تو تسبیح و داد بیت و سرود
&&
بسی بکردم لاحول و توبه دل نشنود
\\
غزل سرا شدم از دست عشق و دست زنان
&&
بسوخت عشق تو ناموس و شرم و هر چم بود
\\
عفیف و زاهد و ثابت قدم بدم چون کوه
&&
کدام کوه که باد توش چو که نربود
\\
اگر کهم هم از آواز تو صدا دارم
&&
وگر کهم همه در آتش توم که دود
\\
وجود تو چو بدیدم شدم ز شرم عدم
&&
ز عشق این عدم آمد جهان جان به وجود
\\
به هر کجا عدم آید وجود کم گردد
&&
زهی عدم که چو آمد از او وجود افزود
\\
فلک کبود و زمین همچو کور راه نشین
&&
کسی که ماه تو بیند رهد ز کور و کبود
\\
مثال جان بزرگی نهان به جسم جهان
&&
مثال احمد مرسل میان گبر و جهود
\\
ستایشت به حقیقت ستایش خویش است
&&
که آفتاب ستا چشم خویش را بستود
\\
ستایش تو چو دریا زبان ما کشتی
&&
روان مسافر دریا و عاقبت محمود
\\
مرا عنایت دریا چو بخت بیدارست
&&
مرا چه غم اگرم هست چشم خواب آلود
\\
\end{longtable}
\end{center}
