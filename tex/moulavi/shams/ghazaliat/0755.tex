\begin{center}
\section*{غزل شماره ۷۵۵: شاه ما از جمله شاهان پیش بود و بیش بود}
\label{sec:0755}
\addcontentsline{toc}{section}{\nameref{sec:0755}}
\begin{longtable}{l p{0.5cm} r}
شاه ما از جمله شاهان پیش بود و بیش بود
&&
زانک شاهنشاه ما هم شاه و هم درویش بود
\\
شاه ما از پرده برجان چو خود را جلوه کرد
&&
جان ما بی‌خویش شد زیرا که شه بی‌خویش بود
\\
شاه ما از جان ما هم دور و هم نزدیک بود
&&
جان ما با شاه ما نزدیک و دوراندیش بود
\\
صاف او بی‌درد بود و راحتش بی‌درد بود
&&
گلشن بی‌خار بود و نوش او بی‌نیش بود
\\
یک صفت از لطف شه آن جا که پرده برگرفت
&&
آب و آتش صلح کرد و گرگ دایه میش بود
\\
جان مطلق شد ز نورش صورتی کو جان نداشت
&&
گشت قربان رهش آن کس که او بدکیش بود
\\
نیست می‌گفتیم اندر هست گفت آری بیا
&&
هست شد عالم از او موقوف یک آریش بود
\\
\end{longtable}
\end{center}
