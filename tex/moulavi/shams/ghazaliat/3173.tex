\begin{center}
\section*{غزل شماره ۳۱۷۳: باده ده، ای ساقی هر متقی}
\label{sec:3173}
\addcontentsline{toc}{section}{\nameref{sec:3173}}
\begin{longtable}{l p{0.5cm} r}
باده ده، ای ساقی هر متقی
&&
بادهٔ شاهنشهی راوقی
\\
جام سخن بخش که از تف او
&&
گردد دیوار سیه منطقی
\\
بردر و بشکن غم و اندیشه را
&&
حاکم و سلطان و شه مطلقی
\\
چون بگریزی نرسد در تو کس
&&
ور بگریزیم تو خود سابقی
\\
جنت حسنت چو تجلی کند
&&
باغ شود دوزخ بر هر شقی
\\
ظلمت و نور از تو تحیر درند
&&
تا تو حقی یا که تو نور حقی
\\
گشت شب و روز ز تو غرق نور
&&
نیست مهت مغربی و مشرقی
\\
لابه کنی، باده دهی رایگان
&&
ساقی دریا صفت مشفقی
\\
مست قبول آمد قلب و سلیم
&&
زیرکی اینجاست همه احمقی
\\
زیرکی ار شرط خوشیها بدی
&&
باده نجستی خرد و موسقی
\\
فرد چرایی تو اگر یار کی؟
&&
از چه تو عذرایی اگر وامقی؟
\\
غنچه صفت خویش ز گل درکشی
&&
رو بکش آن خار، بدان لایقی
\\
خار کشانند، اگر چه شهند
&&
جز تو که بر گلشن جان عاشقی
\\
خامش باش و بنگر فتح باب
&&
چند پی هر سخن مغلقی
\\
\end{longtable}
\end{center}
