\begin{center}
\section*{غزل شماره ۲۳۱۲: دیدم رخ ترسا را با ما چو گل اشکفته}
\label{sec:2312}
\addcontentsline{toc}{section}{\nameref{sec:2312}}
\begin{longtable}{l p{0.5cm} r}
دیدم رخ ترسا را با ما چو گل اشکفته
&&
هم خلوت و هم بی‌گه در دیر صفا رفته
\\
با آن مه بی‌نقصان سرمست شده رقصان
&&
دستی سر زلف او دستی می بگرفته
\\
در رسته بازاری هر جا بده اغیاری
&&
در جانش زده ناری آن خونی آشفته
\\
و آن لعل چو بگشاید تا قند شکر خاید
&&
از عرش نثار آید بس گوهر ناسفته
\\
دل دزدد و بستاند وز سر دلت داند
&&
تا جمله فروخواند پنهانی ناگفته
\\
از حسن پری زاده صد بی‌دل و دل داده
&&
در هر طرف افتاده هم یک یک و هم جفته
\\
نوری که از او تابد هر چشم که برتابد
&&
بیدار ابد یابد در کالبد خفته
\\
از هفت فلک بیرون وز هر دو جهان افزون
&&
وین طرفه که آن بی‌چون اندر دل بنهفته
\\
از بهر چنین مشکل تبریز شده حاصل
&&
و اندر پی شمس الدین پای دل من کفته
\\
\end{longtable}
\end{center}
