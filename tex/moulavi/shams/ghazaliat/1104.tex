\begin{center}
\section*{غزل شماره ۱۱۰۴: باز شد در عاشقی بابی دگر}
\label{sec:1104}
\addcontentsline{toc}{section}{\nameref{sec:1104}}
\begin{longtable}{l p{0.5cm} r}
باز شد در عاشقی بابی دگر
&&
بر جمال یوسفی تابی دگر
\\
مژده بیداران راه عشق را
&&
آنک دیدم دوش من خوابی دگر
\\
ساخته شد از برای طالبان
&&
غیر این اسباب اسبابی دگر
\\
ابرها گر می‌نبارد نقد شد
&&
از برای زندگی آبی دگر
\\
یارکان سرکش شدند و حق بداد
&&
غیر این اصحاب اصحابی دگر
\\
سبزه زار عشق را معمور کرد
&&
عاشقان را دشت و دولابی دگر
\\
وین جگرهایی که بد پرزخم عشق
&&
شد درآویزان به قلابی دگر
\\
عشق اگر بدنام گردد غم مخور
&&
عشق دارد نام و القابی دگر
\\
کفشگر گر خشم گیرد چاره شد
&&
صوفیان را نعل و قبقابی دگر
\\
گر نداند حرف صوفی دان که هست
&&
دردهای عشق را بابی دگر
\\
از هوای شمس دین آموختم
&&
جانب تبریز آدابی دگر
\\
\end{longtable}
\end{center}
