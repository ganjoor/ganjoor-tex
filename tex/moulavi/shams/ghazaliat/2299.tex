\begin{center}
\section*{غزل شماره ۲۲۹۹: سراندازان همی‌آیی ز راه سینه در دیده}
\label{sec:2299}
\addcontentsline{toc}{section}{\nameref{sec:2299}}
\begin{longtable}{l p{0.5cm} r}
سراندازان همی‌آیی ز راه سینه در دیده
&&
فسونگرم می‌خوانی حکایت‌های شوریده
\\
به دم در چرخ می‌آری فلک‌ها را و گردون را
&&
چه باشد پیش افسونت یکی ادراک پوسیده
\\
گناه هر دو عالم را به یک توبه فروشویی
&&
چرایی زلت ما را تو در انگشت پیچیده
\\
تو را هر گوشه ایوبی به هر اطراف یعقوبی
&&
شکسته عشق درهاشان قماش از خانه دزدیده
\\
خرامان شو به گورستان ندایی کن بدان بستان
&&
که خیز ای مرده کهنه برقص ای جسم ریزیده
\\
همان دم جمله گورستان شود چون شهر آبادان
&&
همه رقصان همه شادان قضا از جمله گردیده
\\
گزافه این نمی‌لافم خیالی بر نمی‌بافم
&&
که صد ره دیده‌ام این را نمی‌گویم ز نادیده
\\
کسی کز خلق می‌گوید که من بگریختم رفتم
&&
صدق گو گر گریبانش پس پشت است بدریده
\\
خمش کن بشنو ای ناطق غم معشوق با عاشق
&&
که تا طالب بود جویان بود مطلب ستیزیده
\\
\end{longtable}
\end{center}
