\begin{center}
\section*{غزل شماره ۲۲: چندان بنالم ناله‌ها چندان برآرم رنگ‌ها}
\label{sec:0022}
\addcontentsline{toc}{section}{\nameref{sec:0022}}
\begin{longtable}{l p{0.5cm} r}
چندان بنالم ناله‌ها چندان برآرم رنگ‌ها
&&
تا برکنم از آینه هر منکری من زنگ‌ها
\\
بر مرکب عشق تو دل می‌راند و این مرکبش
&&
در هر قدم می‌بگذرد زان سوی جان فرسنگ‌ها
\\
بنما تو لعل روشنت بر کوری هر ظلمتی
&&
تا بر سر سنگین دلان از عرش بارد سنگ‌ها
\\
با این چنین تابانیت دانی چرا منکر شدند
&&
کاین دولت و اقبال را باشد از ایشان ننگ‌ها
\\
گر نی که کورندی چنین آخر بدیدندی چنان
&&
آن سو هزاران جان ز مه چون اختران آونگ‌ها
\\
چون از نشاط نور تو کوران همی بینا شوند
&&
تا از خوشی راه تو رهوار گردد لنگ‌ها
\\
اما چو اندر راه تو ناگاه بیخود می‌شود
&&
هر عقل زیرا رسته شد در سبزه زارت بنگ‌ها
\\
زین رو همی‌بینم کسان نالان چو نی وز دل تهی
&&
زین رو دو صد سرو روان خم شد ز غم چون چنگ‌ها
\\
زین رو هزاران کاروان بشکسته شد از ره روان
&&
زین ره بسی کشتی پر بشکسته شد بر گنگ‌ها
\\
اشکستگان را جان‌ها بستست بر اومید تو
&&
تا دانش بی‌حد تو پیدا کند فرهنگ‌ها
\\
تا قهر را برهم زند آن لطف اندر لطف تو
&&
تا صلح گیرد هر طرف تا محو گردد جنگ‌ها
\\
تا جستنی نوعی دگر ره رفتنی طرزی دگر
&&
پیدا شود در هر جگر در سلسله آهنگ‌ها
\\
وز دعوت جذب خوشی آن شمس تبریزی شود
&&
هر ذره انگیزنده‌ای هر موی چون سرهنگ‌ها
\\
\end{longtable}
\end{center}
