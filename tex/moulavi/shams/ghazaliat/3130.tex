\begin{center}
\section*{غزل شماره ۳۱۳۰: تماشا مرو نک تماشا تویی}
\label{sec:3130}
\addcontentsline{toc}{section}{\nameref{sec:3130}}
\begin{longtable}{l p{0.5cm} r}
تماشا مرو نک تماشا تویی
&&
جهان و نهان و هویدا تویی
\\
چه این جا روی و چه آن جا روی
&&
که مقصود از این جا و آن جا تویی
\\
به فردا میفکن فراق و وصال
&&
که سرخیل امروز و فردا تویی
\\
تو گویی گرفتار هجرم مگر
&&
که واصل تویی هجر گیرا تویی
\\
ز آدم بزایید حوا و گفت
&&
که آدم تو بودی و حوا تویی
\\
ز نخلی بزایید خرما و گفت
&&
که هم دخل و هم نخل خرما تویی
\\
تو مجنون و لیلی به بیرون مباش
&&
که رامین تویی ویس رعنا تویی
\\
تو درمان غم‌ها ز بیرون مجو
&&
که پازهر و درمان غم‌ها تویی
\\
اگر مه سیه شد همو صیقلست
&&
تو صیقل کنی خود مه ما تویی
\\
وگر مه سیه شد برو تو ملرز
&&
که مه را خطر نیست ترسا تویی
\\
ز هر زحمت افزا فزایش مجو
&&
که هم روح و هم راحت افزا تویی
\\
چو جمعی تو از جمع‌ها فارغی
&&
که با جمع و بی‌جمع و تنها تویی
\\
یکی برگشا پر بافر خویش
&&
که هم صاف و هم قاف و عنقا تویی
\\
چو درد سرت نیست سر را مبند
&&
که سرفتنه روز غوغا تویی
\\
اگرعالمی منکر ما شود
&&
غمی نیست ما را که ما را تویی
\\
مرو زیر و ما را ز بالا مگیر
&&
به پستی بمنشین که بالا تویی
\\
من و ما رها کن ز خواری مترس
&&
که با ما تویی شاه و بی‌ما تویی
\\
بشو رو و سیمای خود درنگر
&&
که آن یوسف خوب سیما تویی
\\
غلط یوسفی تو و یعقوب نیز
&&
مترس و بگو هم زلیخا تویی
\\
گمان می‌بری و این یقین و گمان
&&
گمان می‌برم من که مانا تویی
\\
از این ساحل آب و گل درگذر
&&
به گوهر سفر کن که دریا تویی
\\
از این چاه هستی چو یوسف برآ
&&
که بستان و ریحان و صحرا تویی
\\
اگر تا قیامت بگویم ز تو
&&
به پایان نیاید سر و پا تویی
\\
\end{longtable}
\end{center}
