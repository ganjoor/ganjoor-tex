\begin{center}
\section*{غزل شماره ۲۷۸۴: در جهان گر بازجویی نیست بی‌سودا سری}
\label{sec:2784}
\addcontentsline{toc}{section}{\nameref{sec:2784}}
\begin{longtable}{l p{0.5cm} r}
در جهان گر بازجویی نیست بی‌سودا سری
&&
لیک این سودا غریب آمد به عالم نادری
\\
جمله سوداها بر این فن عاقبت حسرت خورند
&&
ز آنک صد پر دارد این و نیست آن‌ها را پری
\\
پیش باغش باغ عالم نقش گرمابه‌ست و بس
&&
نی در او میوه بقایی نی در او شاخ تری
\\
آن ز سحری تر نماید چون بگیری شاخ او
&&
می‌برد شاخش تو را با خواجه قارون تا ثری
\\
صورت او چون عصا و باطن او اژدها
&&
چون نه‌ای موسی مرو بر اژدهای قاهری
\\
کف موسی کو که تا گردد عصا آن اژدها
&&
گردن آن اژدها را گیرد او چون لمتری
\\
گر کشیده می‌شوی آن سو ز جذب اژدهاست
&&
ز آنک او بس گرسنه‌ست و تو مر او را چون خوری
\\
جذب او چون آتشی آمد درافکن خود در آب
&&
دفع هر ضدی به ضدی دفع ناری کوثری
\\
چون تو در بلخی روان شو سوی بغداد ای پدر
&&
تا به هر دم دورتر باشی ز مرو و از هری
\\
تو مری باشی و چاکر اندر این حضرت به است
&&
ای افندی هین مگو این را مری و آن را مری
\\
ور فسردی در تکبر آفتابی را بجو
&&
در گداز هر فسرده شمس باشد ماهری
\\
آفتاب حشر را ماند گدازد هر جماد
&&
از زمین و آسمان و کوه و سنگ و گوهری
\\
تا بداند اهل محشر کاین همه یخ بوده است
&&
عقل جز وی ننگ مانده بر سر یخ چون خری
\\
ای خر لرزان شده بر روی یخ در زیر بار
&&
پوز بردارد به بالا خر که یا رب آخری
\\
شمس تبریزی چو عقل جزو را یاری دهد
&&
بال و پر یابد خر او برپرد چون جعفری
\\
\end{longtable}
\end{center}
