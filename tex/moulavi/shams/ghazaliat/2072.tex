\begin{center}
\section*{غزل شماره ۲۰۷۲: جفای تلخ تو گوهر کند مرا ای جان}
\label{sec:2072}
\addcontentsline{toc}{section}{\nameref{sec:2072}}
\begin{longtable}{l p{0.5cm} r}
جفای تلخ تو گوهر کند مرا ای جان
&&
که بحر تلخ بود جای گوهر و مرجان
\\
وفای توست یکی بحر دیگر خوش خوار
&&
که چارجوی بهشت است از تکش جوشان
\\
منم سکندر این دم به مجمع البحرین
&&
که تا رهانم جان را ز علت و بحران
\\
که تا ببندم سدی عظیم بر یأجوج
&&
که تا رهند خلایق ز حمله ایشان
\\
از آنک ایشان مر بحر را درآشامند
&&
که هیچ آب نماند ز تابشان به جهان
\\
از آنک آتشی‌اند وز عنصر دوزخ
&&
عدو لطف جنان و حجاب نور جنان
\\
ز هر شمار برونند از آنک از قهرند
&&
که قهر وصف حق است و ندارد آن پایان
\\
برهنه‌اند و همه سترپوششان گوش است
&&
نه سترپوش دلانه که دیدن است عیان
\\
لحاف گوش چپستش فراش گوش راست
&&
به شب نتیجه یأجوج را یقین می‌دان
\\
لحاف و فرش مقلد چون علم تقلید است
&&
یقین به معنی یأجوجی است نی انسان
\\
از آنک دل مثل روزن است کاندر وی
&&
ز شمس نورفشان است و ذره دست افشان
\\
هزار نام و صفت دارد این دل و هر نام
&&
به نسبتی دگر آمد خلاف و دیگر سان
\\
چنانک شخصی نسبت به تو پدر باشد
&&
به نسبت دگری یا پسر و یا اخوان
\\
چو نام‌های خدا در عدد به نسبت شد
&&
ز روی کافر قاهر ز روی ما رحمان
\\
بسا کسا که به نسبت به تو که معتقدی
&&
فرشته است و به نسبت به دیگری شیطان
\\
چنانک سر تو نسبت به تو بود مکشوف
&&
به نسبت دگری حال سر تو پنهان
\\
\end{longtable}
\end{center}
