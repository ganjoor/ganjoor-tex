\begin{center}
\section*{غزل شماره ۲۶۳۲: ای جان گذرکرده از این گنبد ناری}
\label{sec:2632}
\addcontentsline{toc}{section}{\nameref{sec:2632}}
\begin{longtable}{l p{0.5cm} r}
ای جان گذرکرده از این گنبد ناری
&&
در سلطنت فقر و فنا کار تو داری
\\
ای رخت کشیده به نهان خانه بینش
&&
وی کشته وجود همه و خویش به زاری
\\
پوشیده قباهای صفت‌های مقدس
&&
وز دلق دو صدپاره آدم شده عاری
\\
از شرم تو گل ریخته در پای جمالت
&&
وز لطف تو هر خار برون رفته ز خاری
\\
بی‌برگ نشاید که دگر غوره فشارد
&&
در میکده اکنون که تو انگور فشاری
\\
اقبال کف پای تو بر چشم نهاده
&&
اندر طمعی که سرش از لطف بخاری
\\
از غار به نور تو به باغ ازل آیند
&&
ای یار چه یاری تو و ای غار چه غاری
\\
بر کار شود در خود و بی‌کار ز عالم
&&
آن کز تو بنوشید یکی شربت کاری
\\
در باغ صفا زیر درختی به نگاری
&&
افتاد مرا چشم و بگفتم چه نگاری
\\
کز لذت حسن تو درختان به شکوفه
&&
آبستن تو گشته مگر جان بهاری
\\
در سجده شدم بیخود و گفتم که نگارا
&&
آخر ز کجایی تو علی الله چه یاری
\\
او گفت که از پرتو شمس الحق تبریز
&&
کاوصاف جمال رخ او نیست شماری
\\
\end{longtable}
\end{center}
