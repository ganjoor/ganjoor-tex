\begin{center}
\section*{غزل شماره ۴۰۴: هله ای آنک بخوردی سحری باده که نوشت}
\label{sec:0404}
\addcontentsline{toc}{section}{\nameref{sec:0404}}
\begin{longtable}{l p{0.5cm} r}
هله ای آنک بخوردی سحری باده که نوشت
&&
هله پیش آ که بگویم سخن راز به گوشت
\\
می روح آمد نادر رو از آن هم بچش آخر
&&
که به یک جرعه بپرد همه طراری و هوشت
\\
چو از این هوش برستی به مساقات و به مستی
&&
دهدت صد هش دیگر کرم باده فروشت
\\
چو در اسرار درآیی کندت روح سقایی
&&
به فلک غلغله افتد ز هیاهوی و خروشت
\\
بستان باده دیگر جز از آن احمر و اصفر
&&
کندت خواجه معنی برهاند ز نقوشت
\\
دهد آن کان ملاحت قدحی وقت صباحت
&&
به از آن صد قدح می که بخوردی شب دوشت
\\
تو اگرهای نگویی و اگر هوی نگویی
&&
همه اموات و جمادات بجوشند ز جوشت
\\
چو در آن حلقه بگنجی زبر معدن و گنجی
&&
هوس کسب بیفتد ز دل مکسبه کوشت
\\
تو که از شر اعادی به دو صد چاه فتادی
&&
برهانید به آخر کرم مظلمه پوشت
\\
همه آهنگ لقا کن خمش و صید رها کن
&&
به خموشیت میسر شود این صید وحوشت
\\
تو دهان را چو ببندی خمشی را بپسندی
&&
کشش و جذب ندیمان نگذارند خموشت
\\
\end{longtable}
\end{center}
