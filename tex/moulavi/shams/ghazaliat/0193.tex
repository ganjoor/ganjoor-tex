\begin{center}
\section*{غزل شماره ۱۹۳: جانا قبول گردان این جست و جوی ما را}
\label{sec:0193}
\addcontentsline{toc}{section}{\nameref{sec:0193}}
\begin{longtable}{l p{0.5cm} r}
جانا قبول گردان این جست و جوی ما را
&&
بنده و مرید عشقیم برگیر موی ما را
\\
بی ساغر و پیاله درده میی چو لاله
&&
تا گل سجود آرد سیمای روی ما را
\\
مخمور و مست گردان امروز چشم ما را
&&
رشک بهشت گردان امروز کوی ما را
\\
ما کان زر و سیمیم دشمن کجاست زر را
&&
از ما رسد سعادت یار و عدوی ما را
\\
شمع طراز گشتیم گردن دراز گشتیم
&&
فحل و فراخ کردی زین می گلوی ما را
\\
ای آب زندگانی ما را ربود سیلت
&&
اکنون حلال بادت بشکن سبوی ما را
\\
گر خوی ما ندانی از لطف باده واجو
&&
همخوی خویش کردست آن باده خوی ما را
\\
گر بحر می بریزی ما سیر و پر نگردیم
&&
زیرا نگون نهادی در سر کدوی ما را
\\
مهمان دیگر آمد دیکی دگر به کف کن
&&
کاین دیگ بس نیاید یک کاسه شوی ما را
\\
نک جوق جوق مستان در می‌رسند بستان
&&
مخمور چون نیابد چون یافت بوی ما را
\\
ترک هنر بگوید دفتر همه بشوید
&&
گر بشنود عطارد این طرقوی ما را
\\
سیلی خورند چون دف در عشق فخرجویان
&&
زخمه به چنگ آور می‌زن سه توی ما را
\\
بس کن که تلخ گردد دنیا بر اهل دنیا
&&
گر بشنوند ناگه این گفت و گوی ما را
\\
\end{longtable}
\end{center}
