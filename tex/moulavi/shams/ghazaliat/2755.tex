\begin{center}
\section*{غزل شماره ۲۷۵۵: چون عشق کند شکرفشانی}
\label{sec:2755}
\addcontentsline{toc}{section}{\nameref{sec:2755}}
\begin{longtable}{l p{0.5cm} r}
چون عشق کند شکرفشانی
&&
در جلوه شود مه نهانی
\\
بینی که شکر کران ندارد
&&
خوش می‌خوری و همی‌رسانی
\\
می‌غلط به هر طرف که غلطی
&&
بر سبزه سبز بوستانی
\\
گر ز آنک کله نهی وگر نی
&&
شاهنشه جمله خسروانی
\\
آن را بینی که من نگویم
&&
زیرا که بگویمت بدانی
\\
چون چشم تو وا کنند ناگه
&&
بر شهر عظیم آن جهانی
\\
ماننده طفل نوبزاده
&&
خیره نگری و خیره مانی
\\
تا چشم بر آن جهان نشیند
&&
چاره نبود از این نشانی
\\
بگریز به نور شمس تبریز
&&
تا کشف شود همه معانی
\\
\end{longtable}
\end{center}
