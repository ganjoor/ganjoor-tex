\begin{center}
\section*{غزل شماره ۱۶۴: چو مرا به سوی زندان بکشید تن ز بالا}
\label{sec:0164}
\addcontentsline{toc}{section}{\nameref{sec:0164}}
\begin{longtable}{l p{0.5cm} r}
چو مرا به سوی زندان بکشید تن ز بالا
&&
ز مقربان حضرت بشدم غریب و تنها
\\
به میان حبس ناگه قمری مرا قرین شد
&&
که فکند در دماغم هوسش هزار سودا
\\
همه کس خلاص جوید ز بلا و حبس من نی
&&
چه روم چه روی آرم به برون و یار این جا
\\
که به غیر کنج زندان نرسم به خلوت او
&&
که نشد به غیر آتش دل انگبین مصفا
\\
نظری به سوی خویشان نظری برو پریشان
&&
نظری بدان تمنا نظری بدین تماشا
\\
چو بود حریف یوسف نرمد کسی چو دارد
&&
به میان حبس بستان و که خاصه یوسف ما
\\
بدود به چشم و دیده سوی حبس هر کی او را
&&
ز چنین شکرستانی برسد چنین تقاضا
\\
من از اختران شنیدم که کسی اگر بیابد
&&
اثری ز نور آن مه خبری کنید ما را
\\
چو بدین گهر رسیدی رسدت که از کرامت
&&
بنهی قدم چو موسی گذری ز هفت دریا
\\
خبرش ز رشک جان‌ها نرسد به ماه و اختر
&&
که چو ماه او برآید بگدازد آسمان‌ها
\\
خجلم ز وصف رویش به خدا دهان ببندم
&&
چه برد ز آب دریا و ز بحر مشک سقا
\\
\end{longtable}
\end{center}
