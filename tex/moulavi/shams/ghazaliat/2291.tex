\begin{center}
\section*{غزل شماره ۲۲۹۱: بر آنم کز دل و دیده شوم بیزار یک باره}
\label{sec:2291}
\addcontentsline{toc}{section}{\nameref{sec:2291}}
\begin{longtable}{l p{0.5cm} r}
بر آنم کز دل و دیده شوم بیزار یک باره
&&
چو آمد آفتاب جان نخواهم شمع و استاره
\\
دلا نقاش را بنگر چه بینی نقش گرمابه
&&
مه و خورشید را بنگر چه گردی گرد مه پاره
\\
نهادی سیر بر بینی نسیم گل همی‌جویی
&&
زهی بی‌رزق کو جوید ز هر بیچاره‌ای چاره
\\
بجز نقاش را منگر که نقش غم کند شادی
&&
که از اکسیر لطف او عقیق و لعل شد خاره
\\
اگر مخمور اگر مستی به بزم او رو و رستی
&&
که شد عمری که در غربت ز خان و مانی آواره
\\
مگر غول بیابانی ره مدین نمی‌دانی
&&
که فوق سقف گردونی تو را قصر است و درساره
\\
نه هر قصری که تو دیدی از آن قیصری بود آن
&&
نه هر بامی و هر برجی ز بنایی است همواره
\\
هزاران گل در این پستی به وعده شاد می‌خندد
&&
هزاران شمع بر بالا به امر او است سیاره
\\
زهی سلطان زهی نجده سری بخشد به یک سجده
&&
اسیر او شوی بهتر کاسیر نفس مکاره
\\
ز علم او است هر مغزی پر از اندیشه و حیله
&&
ز لطف او است هر چشمی که مخمور است و سحاره
\\
خری کو در کلم زاری درافتاد و نمی‌ترسد
&&
برون رانندش از حایط بریده دم و لت خواره
\\
مگو ای عشق با تن تو حدیث عشق زیرا او
&&
نفاقی می‌کند با تو ولیکن نیست این کاره
\\
به پیشت دست می‌بندد ولیکن بر تو می‌خندد
&&
به گورستان رو و بنگر فغان از نفس اماره
\\
\end{longtable}
\end{center}
