\begin{center}
\section*{غزل شماره ۳۶۷: دل آمد و دی به گوش جان گفت}
\label{sec:0367}
\addcontentsline{toc}{section}{\nameref{sec:0367}}
\begin{longtable}{l p{0.5cm} r}
دل آمد و دی به گوش جان گفت
&&
ای نام تو این که می‌نتان گفت
\\
درنده آنک گفت پیدا
&&
سوزنده آنک در نهان گفت
\\
چه عذر و بهانه دارد ای جان
&&
آن کس که ز بی‌نشان نشان گفت
\\
گل داند و بلبل معربد
&&
رازی که میان گلستان گفت
\\
آن کس نه که از طریق تحصیل
&&
آموخت ز بانگ بلبلان گفت
\\
صیادی تیر غمزه‌ها را
&&
آن ابروهای چون کمان گفت
\\
صد گونه زبان زمین برآورد
&&
در پاسخ آن چه آسمان گفت
\\
ای عاشق آسمان قرین شو
&&
با او که حدیث نردبان گفت
\\
زان شاهد خانگی نشان کو
&&
هر کس سخنی ز خاندان گفت
\\
کو شعشعه‌های قرص خورشید
&&
هر سایه نشین ز سایه بان گفت
\\
با این همه گوش و هوش مستست
&&
زان چند سخن که این زبان گفت
\\
چون یافت زبان دو سه قراضه
&&
مشغول شد و به ترک کان گفت
\\
وز ننگ قراضه جان عاشق
&&
ترک بازار و این دکان گفت
\\
در گوشم گفت عشق بس کن
&&
خاموش کنم چو او چنان گفت
\\
\end{longtable}
\end{center}
