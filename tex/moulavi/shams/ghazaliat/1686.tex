\begin{center}
\section*{غزل شماره ۱۶۸۶: گفتم که عهد بستم وز عهد بد برستم}
\label{sec:1686}
\addcontentsline{toc}{section}{\nameref{sec:1686}}
\begin{longtable}{l p{0.5cm} r}
گفتم که عهد بستم وز عهد بد برستم
&&
گفتا چگونه بندی چیزی که من شکستم
\\
با وی چو شهد و شیرم هم دامنش بگیرم
&&
اما چگونه گیرم چون من شکسته دستم
\\
خود دامنش نگیرد الا شکسته دستی
&&
اکنون بلند گردم کز جور کرد پستم
\\
تا من بلند باشم پستم کند به داور
&&
چون نیست کرد آنگه بازآورد به هستم
\\
ای حلقه‌های زلفش پیچیده گرد حلقم
&&
افغان ز چشم مستش کان مست کرد مستم
\\
آمد خیال مستش مستانه حمله آورد
&&
چندان بهانه کردم وز دست او نرستم
\\
حلقه زدم به در بر آواز داد دلبر
&&
گفتا که نیست این جا یعنی بدان که هستم
\\
گفتم که بنده آمد گفت این دم تو دام است
&&
من کی شکار دامم من کی اسیر شستم
\\
گفتم اگر بسوزی جان مرا سزایم
&&
ای بت مرا بسوزان زیرا که بت پرستم
\\
من خشک از آن شدستم تا خوش مرا بسوزی
&&
چون تو مرا بسوزی از سوختن برستم
\\
هر جا روی بیایم هر جا روم بیایی
&&
در مرگ و زندگانی با تو خوشم خوشستم
\\
ای آب زندگانی با تو کجاست مردن
&&
در سایه تو بالله جستم ز مرگ جستم
\\
\end{longtable}
\end{center}
