\begin{center}
\section*{غزل شماره ۴۵۸: امروز چرخ را ز مه ما تحیریست}
\label{sec:0458}
\addcontentsline{toc}{section}{\nameref{sec:0458}}
\begin{longtable}{l p{0.5cm} r}
امروز چرخ را ز مه ما تحیریست
&&
خورشید را ز غیرت رویش تغیریست
\\
صبح وجود را به جز این آفتاب نیست
&&
بر ذره ذره وحدت حسنش مقرریست
\\
اما بدان سبب که به هر شام و هر صبوح
&&
اشکال نو نماید گویی که دیگریست
\\
اشکال نو به نو چو مناقض نمایدت
&&
اندر مناقضات خلافی مستریست
\\
در تو چو جنگ باشد گویی دو لشکر است
&&
در تو چو جنگ نبود دانی که لشکریست
\\
اندر خلیل لطف بد آتش نمود آب
&&
نمرود قهر بود بر او آب آذریست
\\
گرگی نمود یوسف در چشم حاسدان
&&
پنهان شد آنک خوب و شکرلب برادریست
\\
این دست خود همی‌برد از عشق روی او
&&
وان قصد جانش کرده که بس زشت و منکریست
\\
آن پرده از نمد نبود از حسد بود
&&
زان پرده دوست را منگر زشت منظریست
\\
دیویست نفس تو که حسد جزو وصف اوست
&&
تا کل او چگونه قبیحی و مقذریست
\\
آن مار زشت را تو کنون شیر می‌دهی
&&
نک اژدها شود که به طبع آدمی خوریست
\\
ای برق اژدهاکش از آسمان فضل
&&
برتاب و برکشش که از او روح مضطریست
\\
بی حرف شو چو دل اگرت صدر آرزوست
&&
کز گفت این زبانت چو خواهنده بر دریست
\\
\end{longtable}
\end{center}
