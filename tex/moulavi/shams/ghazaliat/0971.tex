\begin{center}
\section*{غزل شماره ۹۷۱: عشق تو مست و کف زنانم کرد}
\label{sec:0971}
\addcontentsline{toc}{section}{\nameref{sec:0971}}
\begin{longtable}{l p{0.5cm} r}
عشق تو مست و کف زنانم کرد
&&
مستم و بیخودم چه دانم کرد
\\
غوره بودم کنون شدم انگور
&&
خویشتن را ترش نتانم کرد
\\
شکرینست یار حلوایی
&&
مشت حلوا در این دهانم کرد
\\
تا گشاد او دکان حلوایی
&&
خانه‌ام برد و بی‌دکانم کرد
\\
خلق گوید چنان نمی‌باید
&&
من نبودم چنین چنانم کرد
\\
اولا خم شکست و سرکه بریخت
&&
نوحه کردم که او زیانم کرد
\\
صد خم می به جای آن یک خم
&&
درخورم داد و شادمانم کرد
\\
در تنور بلا و فتنه خویش
&&
پخته و سرخ رو چو نانم کرد
\\
چون زلیخا ز غم شدم من پیر
&&
کرد یوسف دعا جوانم کرد
\\
می‌پریدم ز دست او چون تیر
&&
دست در من زد و کمانم کرد
\\
پر کنم شکر آسمان و زمین
&&
چون زمین بودم آسمانم کرد
\\
از ره کهکشان گذشت دلم
&&
زان سوی کهکشان کشانم کرد
\\
نردبان‌ها و بام‌ها دیدم
&&
فارغ از بام و نردبانم کرد
\\
چون جهان پر شد از حکایت من
&&
در جهان همچو جان نهانم کرد
\\
چون مرا نرم یافت همچو زبان
&&
چون زبان زود ترجمانم کرد
\\
چون زبان متصل به دل بودم
&&
راز دل یک به یک بیانم کرد
\\
چون زبانم گرفت خون ریزی
&&
همچو شمشیر در میانم کرد
\\
بس کن ای دل که در بیان ناید
&&
آن چه آن یار مهربانم کرد
\\
\end{longtable}
\end{center}
