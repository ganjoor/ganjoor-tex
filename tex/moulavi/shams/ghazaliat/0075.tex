\begin{center}
\section*{غزل شماره ۷۵: ای خواجه نمی‌بینی این روز قیامت را}
\label{sec:0075}
\addcontentsline{toc}{section}{\nameref{sec:0075}}
\begin{longtable}{l p{0.5cm} r}
ای خواجه نمی‌بینی این روز قیامت را
&&
این یوسف خوبی را این خوش قد و قامت را
\\
ای شیخ نمی‌بینی این گوهر شیخی را
&&
این شعشعه نو را این جاه و جلالت را
\\
ای میر نمی‌بینی این مملکت جان را
&&
این روضه دولت را این تخت و سعادت را
\\
این خوشدل و خوش دامن دیوانه تویی یا من
&&
درکش قدحی با من بگذار ملامت را
\\
ای ماه که در گردش هرگز نشوی لاغر
&&
انوار جلال تو بدریده ضلالت را
\\
چون آب روان دیدی بگذار تیمم را
&&
چون عید وصال آمد بگذار ریاضت را
\\
گر ناز کنی خامی ور ناز کشی رامی
&&
در بارکشی یابی آن حسن و ملاحت را
\\
خاموش که خاموشی بهتر ز عسل نوشی
&&
درسوز عبارت را بگذار اشارت را
\\
شمس الحق تبریزی ای مشرق تو جان‌ها
&&
از تابش تو یابد این شمس حرارت را
\\
\end{longtable}
\end{center}
