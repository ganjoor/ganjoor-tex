\begin{center}
\section*{غزل شماره ۱۹۹۴: بشنو از بوالهوسان قصه میر عسسان}
\label{sec:1994}
\addcontentsline{toc}{section}{\nameref{sec:1994}}
\begin{longtable}{l p{0.5cm} r}
بشنو از بوالهوسان قصه میر عسسان
&&
رندی از حلقه ما گشت در این کوی نهان
\\
مدتی هست که ما در طلبش سوخته‌ایم
&&
شب و روز از طلبش هر طرفی جامه دران
\\
هم در این کوی کسی یافت ز ناگه اثرش
&&
جامه پرخون شده او است ببینید نشان
\\
خون عشاق کهن خود نشود تازه بود
&&
خون چو تازه است بدانید که هست آن فلان
\\
همه خون‌ها چو شود کهنه سیه گردد و خشک
&&
خون عشاق ابد تازه بجوشد ز روان
\\
تو مگو دفع که این دعوی خون کهن است
&&
خون عشاق نخفته‌ست و نخسبد به جهان
\\
غمزه توست که خونی است در این گوشه و بس
&&
نرگس توست که ساقی است دهد رطل گران
\\
غمزه توست که مست آید و دل‌ها دزدد
&&
قصد جان‌ها کند آن سخت دل سخته کمان
\\
داد آن است که آن گمشده را بازدهی
&&
یا چو او شد ز میانه تو درآیی به میان
\\
گر ز میر شکران داد بیابی ای دل
&&
شکر کن شو تو گدازان چو شکر با شکران
\\
گر چنان کشته شوی زنده جاوید شوی
&&
خدمت از جان چنین کشته به تبریز رسان
\\
\end{longtable}
\end{center}
