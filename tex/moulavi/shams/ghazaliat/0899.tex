\begin{center}
\section*{غزل شماره ۸۹۹: یار مرا عارض و عذار نه این بود}
\label{sec:0899}
\addcontentsline{toc}{section}{\nameref{sec:0899}}
\begin{longtable}{l p{0.5cm} r}
یار مرا عارض و عذار نه این بود
&&
باغ مرا نخل و برگ و بار نه این بود
\\
عهدشکن گشته‌اند خاصه و عامه
&&
قاعده اهل این دیار نه این بود
\\
روح در این غار غوره وار ترش چیست
&&
پرورش و عهد یار غار نه این بود
\\
سیل غم بی‌شمار بار و خرم برد
&&
طمع من از یار بردبار نه این بود
\\
از جهت من چه دیگ می‌پزد آن یار
&&
راتبه میر پخته کار نه این بود
\\
دام نهان کرد و دانه ریخت به پیشم
&&
کینه نهان داشت و آشکار نه این بود
\\
ناصح من کژ نهاد و برد ز راهم
&&
شرط امینی و مستشار نه این بود
\\
در چمن عیش خار از چه شکفته‌ست
&&
منبت آن شهره نوبهار نه این بود
\\
شحنه شد آن دزد من ببست دو دستم
&&
سایسی و عدل شهریار نه این بود
\\
مهل ندادی که عذر خویش بگویم
&&
خوی چو تو کوه باوقار نه این بود
\\
می‌رسدم بوی خون ز گفت درشتش
&&
رایحه ناف مشکبار نه این بود
\\
نوش تو را ذوق و طعم و لطف نه این بود
&&
وان شتر مست خوش عیار نه این بود
\\
پیش شه افغان کنم ز خدعه قلاب
&&
زر من آن نقد خوش عیار نه این بود
\\
شاه چو دریا خزینه‌اش همه گوهر
&&
لیک شهم را خزینه دار نه این بود
\\
بس که گله‌ست این نثار و جمله شکایت
&&
شاه شکور مرا نثار نه این بود
\\
\end{longtable}
\end{center}
