\begin{center}
\section*{غزل شماره ۱۸: ای یوسف خوش نام ما خوش می‌روی بر بام ما}
\label{sec:0018}
\addcontentsline{toc}{section}{\nameref{sec:0018}}
\begin{longtable}{l p{0.5cm} r}
ای یوسف خوش نام ما خوش می‌روی بر بام ما
&&
انا فتحنا الصلا بازآ ز بام از در درآ
\\
ای بحر پرمرجان من والله سبک شد جان من
&&
این جان سرگردان من از گردش این آسیا
\\
ای ساربان با قافله مگذر مرو زین مرحله
&&
اشتر بخوابان هین هله نه از بهر من بهر خدا
\\
نی نی برو مجنون برو خوش در میان خون برو
&&
از چون مگو بی‌چون برو زیرا که جان را نیست جا
\\
گر قالبت در خاک شد جان تو بر افلاک شد
&&
گر خرقه تو چاک شد جان تو را نبود فنا
\\
از سر دل بیرون نه‌ای بنمای رو کایینه‌ای
&&
چون عشق را سرفتنه‌ای پیش تو آید فتنه‌ها
\\
گویی مرا چون می‌روی گستاخ و افزون می‌روی
&&
بنگر که در خون می‌روی آخر نگویی تا کجا
\\
گفتم کز آتش‌های دل بر روی مفرش‌های دل
&&
می غلط در سودای دل تا بحر یفعل ما یشا
\\
هر دم رسولی می‌رسد جان را گریبان می‌کشد
&&
بر دل خیالی می‌دود یعنی به اصل خود بیا
\\
دل از جهان رنگ و بو گشته گریزان سو به سو
&&
نعره زنان کان اصل کو جامه دران اندر وفا
\\
\end{longtable}
\end{center}
