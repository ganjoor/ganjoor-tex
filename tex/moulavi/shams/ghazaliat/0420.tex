\begin{center}
\section*{غزل شماره ۴۲۰: ذوق روی ترشش بین که ز صد قند گذشت}
\label{sec:0420}
\addcontentsline{toc}{section}{\nameref{sec:0420}}
\begin{longtable}{l p{0.5cm} r}
ذوق روی ترشش بین که ز صد قند گذشت
&&
گفت بس چند بود گفتمش از چند گذشت
\\
چون چنین است صنم پند مده عاشق را
&&
آهن سرد چه کوبی که وی از پند گذشت
\\
تو چه پرسیش که چونی و چگونه است دلت
&&
منزل عشق از آن حال که پرسند گذشت
\\
آن چه روی است که ترکان همه هندوی ویند
&&
ترک تاز غم سودای وی از چند گذشت
\\
آن کف بحر گهربخش وراء النهر است
&&
روضه خوی وی از سغد سمرقند گذشت
\\
خارش حرص و طمع در جگر و جانش افکند
&&
چون نسیم کرمش بر دل خرسند گذشت
\\
ذوق دشنام وی از شهد ثنا بیش آمد
&&
لطف خار غم او را گل خوش خند گذشت
\\
گر در بسته کند منع ز هفتاد بلا
&&
تا که این سیل بلا آمد و از بند گذشت
\\
هر کی عقد و حل احوال دل خویش بدید
&&
بند هستی بشکست او و ز پیوند گذشت
\\
مرد چونک به کف آورد چنین در یتیم
&&
خاطر او ز وفای زن و فرزند گذشت
\\
بس که از قصه خوبش همه در فتنه فتند
&&
کاین مقالات خوش از فهم خردمند گذشت
\\
\end{longtable}
\end{center}
