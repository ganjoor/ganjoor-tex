\begin{center}
\section*{غزل شماره ۱۹۱: بیدار کن طرب را بر من بزن تو خود را}
\label{sec:0191}
\addcontentsline{toc}{section}{\nameref{sec:0191}}
\begin{longtable}{l p{0.5cm} r}
بیدار کن طرب را بر من بزن تو خود را
&&
چشمی چنین بگردان کوری چشم بد را
\\
خود را بزن تو بر من اینست زنده کردن
&&
بر مرده زن چو عیسی افسون معتمد را
\\
ای رویت از قمر به آن رو به روی من نه
&&
تا بنده دیده باشد صد دولت ابد را
\\
در واقعه بدیدم کز قند تو چشیدم
&&
با آن نشان که گفتی این بوسه نام زد را
\\
جان فرشته بودی یا رب چه گشته بودی
&&
کز چهره می‌نمودی لم یتخذ ولد را
\\
چون دست تو کشیدم صورت دگر ندیدم
&&
بی هوشیی بدیدم گم کرده مر خرد را
\\
جام چو نار درده بی‌رحم وار درده
&&
تا گم شوم ندانم خود را و نیک و بد را
\\
این بار جام پر کن لیکن تمام پر کن
&&
تا چشم سیر گردد یک سو نهد حسد را
\\
درده میی ز بالا در لا اله الا
&&
تا روح اله بیند ویران کند جسد را
\\
از قالب نمدوش رفت آینه خرد خوش
&&
چندانک خواهی اکنون می‌زن تو این نمد را
\\
\end{longtable}
\end{center}
