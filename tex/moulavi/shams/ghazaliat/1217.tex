\begin{center}
\section*{غزل شماره ۱۲۱۷: الحذر از عشق حذر هر کی نشانی بودش}
\label{sec:1217}
\addcontentsline{toc}{section}{\nameref{sec:1217}}
\begin{longtable}{l p{0.5cm} r}
الحذر از عشق حذر هر کی نشانی بودش
&&
گر بستیزد برود عشق تو برهم زندش
\\
از دل و جان برکندش لولی و منبل کندش
&&
سیل درآید چو گیا هر طرفی می‌بردش
\\
اوست یقین رهزن تو خون تو در گردن تو
&&
دور شو از خیر و شرش دور شو از نیک و بدش
\\
باده خوری مست شوی بی‌دل و بی‌دست شوی
&&
بیست سلامت بودش درکشدش خوش خوردش
\\
پای در این جوی نهی تا به قیامت نرهی
&&
هر که در این موج فتد تا لب دریا کشدش
\\
گول شود هول شود وز همه معزول شود
&&
دست نگیرد هنرش سود ندارد خردش
\\
ای دم تو دام خمش بی‌گنهان را بمکش
&&
ای رخ تو باده هش مست کند تا ابدش
\\
\end{longtable}
\end{center}
