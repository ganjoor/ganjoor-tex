\begin{center}
\section*{غزل شماره ۳۹۸: از سقاهم ربهم بین جمله ابرار مست}
\label{sec:0398}
\addcontentsline{toc}{section}{\nameref{sec:0398}}
\begin{longtable}{l p{0.5cm} r}
از سقاهم ربهم بین جمله ابرار مست
&&
وز جمال لایزالی هفت و پنج و چار مست
\\
این قیامت بین که گویی آشکارا شد ز غیب
&&
خم و کوزه حوض کوثر از می جبار مست
\\
تن چو سایه بر زمین و جان پاک عاشقان
&&
در بهشت عشق تجری تحتها الانهار مست
\\
چون فزون گردد تجلی از جمال حق ببین
&&
ذره ذره هر دو عالم گشته موسی وار مست
\\
از تقاضاهای مستان وز جواب لن تران
&&
در شفاعت مو به موی احمد مختار مست
\\
او سر است و ما چو دستار اندر او پیچیده‌ایم
&&
از شراب آن سری گردد سر و دستار مست
\\
یوسف مصری فروکن سر به مصر اندرنگر
&&
شهر پرآشوب بین و جمله بازار مست
\\
گر بگویم ای برادر خیره مانی زین عجب
&&
عرش و کرسی آسمان‌ها این همه کردار مست
\\
شمس تبریزی برآمد در دلم بزمی نهاد
&&
از شراب عشق گشتست این در و دیوار مست
\\
\end{longtable}
\end{center}
