\begin{center}
\section*{غزل شماره ۲۷: آن خواجه را در کوی ما در گل فرورفتست پا}
\label{sec:0027}
\addcontentsline{toc}{section}{\nameref{sec:0027}}
\begin{longtable}{l p{0.5cm} r}
آن خواجه را در کوی ما در گل فرورفتست پا
&&
با تو بگویم حال او برخوان اذا جاء القضا
\\
جباروار و زفت او دامن کشان می‌رفت او
&&
تسخرکنان بر عاشقان بازیچه دیده عشق را
\\
بس مرغ پران بر هوا از دام‌ها فرد و جدا
&&
می‌آید از قبضه قضا بر پر او تیر بلا
\\
ای خواجه سرمستک شدی بر عاشقان خنبک زدی
&&
مست خداوندی خود کشتی گرفتی با خدا
\\
بر آسمان‌ها برده سر وز سرنبشت او بی‌خبر
&&
همیان او پرسیم و زر گوشش پر از طال بقا
\\
از بوسه‌ها بر دست او وز سجده‌ها بر پای او
&&
وز لورکند شاعران وز دمدمه هر ژاژخا
\\
باشد کرم را آفتی کان کبر آرد در فتی
&&
از وهم بیمارش کند در چاپلوسی هر گدا
\\
بدهد درم‌ها در کرم او نافریدست آن درم
&&
از مال و ملک دیگری مردی کجا باشد سخا
\\
فرعون و شدادی شده خیکی پر از بادی شده
&&
موری بده ماری شده وان مار گشته اژدها
\\
عشق از سر قدوسیی همچون عصای موسیی
&&
کو اژدها را می‌خورد چون افکند موسی عصا
\\
بر خواجه روی زمین بگشاد از گردون کمین
&&
تیری زدش کز زخم او همچون کمانی شد دوتا
\\
در رو فتاد او آن زمان از ضربت زخم گران
&&
خرخرکنان چون صرعیان در غرغره مرگ و فنا
\\
رسوا شده عریان شده دشمن بر او گریان شده
&&
خویشان او نوحه کنان بر وی چو اصحاب عزا
\\
فرعون و نمرودی بده انی انا الله می‌زده
&&
اشکسته گردن آمده در یارب و در ربنا
\\
او زعفرانی کرده رو زخمی نه بر اندام او
&&
جز غمزه غمازه‌ای شکرلبی شیرین لقا
\\
تیرش عجبتر یا کمان چشمش تهیتر یا دهان
&&
او بی‌وفاتر یا جهان او محتجبتر یا هما
\\
اکنون بگویم سر جان در امتحان عاشقان
&&
از قفل و زنجیر نهان هین گوش‌ها را برگشا
\\
کی برگشایی گوش را کو گوش مر مدهوش را
&&
مخلص نباشد هوش را جز یفعل الله ما یشا
\\
این خواجه باخرخشه شد پرشکسته چون پشه
&&
نالان ز عشق عایشه کابیض عینی من بکا
\\
انا هلکنا بعدکم یا ویلنا من بعدکم
&&
مقت الحیوه فقدکم عودوا الینا بالرضا
\\
العقل فیکم مرتهن هل من صدا یشفی الحزن
&&
و القلب منکم ممتحن فی وسط نیران النوی
\\
ای خواجه با دست و پا پایت شکستست از قضا
&&
دل‌ها شکستی تو بسی بر پای تو آمد جزا
\\
این از عنایت‌ها شمر کز کوی عشق آمد ضرر
&&
عشق مجازی را گذر بر عشق حقست انتها
\\
غازی به دست پور خود شمشیر چوبین می‌دهد
&&
تا او در آن استا شود شمشیر گیرد در غزا
\\
عشقی که بر انسان بود شمشیر چوبین آن بود
&&
آن عشق با رحمان شود چون آخر آید ابتلا
\\
عشق زلیخا ابتدا بر یوسف آمد سال‌ها
&&
شد آخر آن عشق خدا می‌کرد بر یوسف قفا
\\
بگریخت او یوسف پیش زد دست در پیراهنش
&&
بدریده شد از جذب او برعکس حال ابتدا
\\
گفتش قصاص پیرهن بردم ز تو امروز من
&&
گفتا بسی زین‌ها کند تقلیب عشق کبریا
\\
مطلوب را طالب کند مغلوب را غالب کند
&&
ای بس دعاگو را که حق کرد از کرم قبله دعا
\\
باریک شد این جا سخن دم می‌نگنجد در دهن
&&
من مغلطه خواهم زدن این جا روا باشد دغا
\\
او می‌زند من کیستم من صورتم خاکیستم
&&
رمال بر خاکی زند نقش صوابی یا خطا
\\
این را رها کن خواجه را بنگر که می‌گوید مرا
&&
عشق آتش اندر ریش زد ما را رها کردی چرا
\\
ای خواجه صاحب قدم گر رفتم اینک آمدم
&&
تا من در این آخرزمان حال تو گویم برملا
\\
آخر چه گوید غره‌ای جز ز آفتابی ذره‌ای
&&
از بحر قلزم قطره‌ای زین بی‌نهایت ماجرا
\\
چون قطره‌ای بنمایدت باقیش معلوم آیدت
&&
ز انبار کف گندمی عرضه کنند اندر شرا
\\
کفی چو دیدی باقیش نادیده خود می‌دانیش
&&
دانیش و دانی چون شود چون بازگردد ز آسیا
\\
هستی تو انبار کهن دستی در این انبار کن
&&
بنگر چگونه گندمی وانگه به طاحون بر هلا
\\
هست آن جهان چون آسیا هست آن جهان چون خرمنی
&&
آن جا همین خواهی بدن گر گندمی گر لوبیا
\\
رو ترک این گو ای مصر آن خواجه را بین منتظر
&&
کو نیم کاره می‌کند تعجیل می‌گوید صلا
\\
ای خواجه تو چونی بگو خسته در این پرفتنه کو
&&
در خاک و خون افتاده‌ای بیچاره وار و مبتلا
\\
گفت الغیاث ای مسلمین دل‌ها نگهدارید هین
&&
شد ریخته خود خون من تا این نباشد بر شما
\\
من عاشقان را در تبش بسیار کردم سرزنش
&&
با سینه پرغل و غش بسیار گفتم ناسزا
\\
ویل لکل همزه بهر زبان بد بود
&&
هماز را لماز را جز چاشنی نبود دوا
\\
کی آن دهان مردم است سوراخ مار و کژدم است
&&
کهگل در آن سوراخ زن کزدم منه بر اقربا
\\
در عشق ترک کام کن ترک حبوب و دام کن
&&
مر سنگ را زر نام کن شکر لقب نه بر جفا
\\
\end{longtable}
\end{center}
