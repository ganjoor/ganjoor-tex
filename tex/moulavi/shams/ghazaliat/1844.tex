\begin{center}
\section*{غزل شماره ۱۸۴۴: مرا در دل همی‌آید که من دل را کنم قربان}
\label{sec:1844}
\addcontentsline{toc}{section}{\nameref{sec:1844}}
\begin{longtable}{l p{0.5cm} r}
مرا در دل همی‌آید که من دل را کنم قربان
&&
نباید بددلی کردن بباید کردن این فرمان
\\
دل من می نیارامد که من با دل بیارامم
&&
بباید کرد ترک دل نباید خصم شد با جان
\\
زهی میدان زهی مردان همه در مرگ خود شادان
&&
سر خود گوی باید کرد وانگه رفت در میدان
\\
زهی سر دل عاشق قضای سر شده او را
&&
خنک این سر خنک آن سر که دارد این چنین جولان
\\
اگر جانباز و عیاری وگر در خون خود یاری
&&
پس گردن چه می خاری چه می ترسی چو ترسایان
\\
اگر مجنون زنجیری سر زنجیر می گیری
&&
وگر از شیر زادستی چپی چون گربه در انبان
\\
مرا گفت آن جگرخواره که مهمان توام امشب
&&
جگر در سیخ کش ای دل کبابی کن پی مهمان
\\
کباب است و شراب امشب حرام و کفر خواب امشب
&&
که امشب همچو چتر آمد نهان در چتر شب سلطان
\\
ربابی چشم بربسته رباب و زخمه بر دسته
&&
کمانچه رانده آهسته مرا از خواب او افغان
\\
کشاکش‌هاست در جانم کشنده کیست می دانم
&&
دمی خواهم بیاسایم ولیکن نیستم امکان
\\
به هر روزم جنون آرد دگر بازی برون آرد
&&
که من بازیچه اویم ز بازی‌های او حیران
\\
چو جامم گه بگرداند چو ساغر گه بریزد خون
&&
چو خمرم گه بجوشاند چو مستم گه کند ویران
\\
گهی صرفم بنوشاند چو چنگم درخروشاند
&&
به شامم می بپوشاند به صبحم می کند یقظان
\\
گر این از شمس تبریز است زهی بنده نوازی‌ها
&&
وگر از دور گردون است زهی دور و زهی دوران
\\
\end{longtable}
\end{center}
