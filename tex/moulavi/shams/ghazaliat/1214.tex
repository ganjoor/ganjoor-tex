\begin{center}
\section*{غزل شماره ۱۲۱۴: بیا که دانه لطیفست رو ز دام مترس}
\label{sec:1214}
\addcontentsline{toc}{section}{\nameref{sec:1214}}
\begin{longtable}{l p{0.5cm} r}
بیا که دانه لطیفست رو ز دام مترس
&&
قمارخانه درآ و ز ننگ وام مترس
\\
بیا بیا که حریفان همه به گوش تواند
&&
بیا بیا که حریفان تو را غلام مترس
\\
بیا بیا به شرابی و ساقیی که مپرس
&&
درآ درآ بر آن شاه خوش سلام مترس
\\
شنیده‌ای که در این راه بیم جان و سر است
&&
چو یار آب حیاتست از این پیام مترس
\\
چو عشق عیسی وقتست و مرده می‌جوید
&&
بمیر پیش جمالش چو من تمام مترس
\\
اگر چه رطل گرانست او سبک روحست
&&
ز دست دوست فروکش هزار جام مترس
\\
غلام شیر شدی بی‌کباب کی مانی
&&
چو پخته خوار نباشی ز هیچ خام مترس
\\
حریف ماه شدی از عسس چه غم داری
&&
صبوح روح چو دیدی ز صبح و شام مترس
\\
خیال دوست بیاورد سوی من جامی
&&
که گیر باده خاص و ز خاص و عام مترس
\\
بگفتمش مه روزه‌ست و روز گفت خموش
&&
که نشکند می جان روزه و صیام مترس
\\
در این مقام خلیلست و بایزید حریف
&&
بگیر جام مقیم و در این مقام مترس
\\
\end{longtable}
\end{center}
