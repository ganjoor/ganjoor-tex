\begin{center}
\section*{غزل شماره ۴۷۸: وجود من به کف یار جز که ساغر نیست}
\label{sec:0478}
\addcontentsline{toc}{section}{\nameref{sec:0478}}
\begin{longtable}{l p{0.5cm} r}
وجود من به کف یار جز که ساغر نیست
&&
نگاه کن به دو چشمم اگرت باور نیست
\\
چو ساغرم دل پرخون من و تن لاغر
&&
به دست عشق که زرد و نزار و لاغر نیست
\\
به غیر خون مسلمان نمی‌خورد این عشق
&&
بیا به گوش تو گویم عجب که کافر نیست
\\
هزار صورت زاید چو آدم و حوا
&&
جهان پرست ز نقش وی او مصور نیست
\\
صلاح ذره صحرا و قطره دریا
&&
بداند و مدد آرد که علم او کر نیست
\\
به هر دمی دل ما را گشاید و بندد
&&
چرا دلش نشناسد به فعلش ار خر نیست
\\
خر از گشادن و بستن به دست خربنده
&&
شدست عارف و داند که اوست دیگر نیست
\\
چو بیندش سر و گوش خرانه جنباند
&&
ندای او بشناسد که او منکر نیست
\\
ز دست او علف و آب‌های خوش خوردست
&&
عجب عجب ز خدا مر تو را چنان خور نیست
\\
هزار بار ببستت به درد و ناله زدی
&&
چه منکری که خدا در خلاص مضطر نیست
\\
چو کافران ننهی سر مگر به وقت بلا
&&
به نیم حبه نیرزد سری کز آن سر نیست
\\
هزار صورت جان در هوا همی‌پرد
&&
مثال جعفر طیار اگر چه جعفر نیست
\\
ولیک مرغ قفس از هوا کجا داند
&&
گمان برد ز نژندی که خود مرا پر نیست
\\
سر از شکاف قفس هر نفس کند بیرون
&&
سرش بگنجد و تن نی از آنک کل سر نیست
\\
شکاف پنج حس تو شکاف آن قفس است
&&
هزار منظر بینی و ره به منظر نیست
\\
تن تو هیزم خشکست و آن نظر آتش
&&
چو نیک درنگری جمله جز که آذر نیست
\\
نه هیزمست که آتش شدست در سوزش
&&
بدانک هیزم نورست اگر چه انور نیست
\\
برای گوش کسانی که بعد ما آیند
&&
بگویم و بنهم عمر ما مأخر نیست
\\
که گوششان بگرفتست عشق و می‌آرد
&&
ز راه‌های نهانی که عقل رهبر نیست
\\
بخفت چشم محمد ضعیف گشت رباب
&&
مخسب گنج زرست این سخن اگر زر نیست
\\
خلایق اختر و خورشید شمس تبریزی
&&
کدام اختر کز شمس او منور نیست
\\
\end{longtable}
\end{center}
