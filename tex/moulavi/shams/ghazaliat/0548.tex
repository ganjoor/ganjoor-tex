\begin{center}
\section*{غزل شماره ۵۴۸: چشم تو ناز می‌کند ناز جهان تو را رسد}
\label{sec:0548}
\addcontentsline{toc}{section}{\nameref{sec:0548}}
\begin{longtable}{l p{0.5cm} r}
چشم تو ناز می‌کند ناز جهان تو را رسد
&&
حسن و نمک تو را بود ناز دگر که را رسد
\\
چشم تو ناز می‌کند لعل تو داد می‌دهد
&&
کشتن و حشر بندگان لاجرم از خدا رسد
\\
چشم کشید خنجری لعل نمود شکری
&&
بو که میان کش مکش هدیه به آشنا رسد
\\
سلطنتست و سروری خوبی و بنده پروری
&&
و آنچ بگفت ناید آن کز تو به جان عطا رسد
\\
نطق عطاردانه‌ام مستی بی‌کرانه‌ام
&&
گر نبود ز خوان تو راتبه از کجا رسد
\\
چرخ سجود می‌کند خرقه کبود می‌کند
&&
چرخ زنان چو صوفیان چونک ز تو صلا رسد
\\
جز تو خلیفه خدا کیست بگو به دور ما
&&
سجده کند ملک تو را چون ملک از سما رسد
\\
دولت خاکیان نگر کز ملکند پاکتر
&&
پرورش این چنین بود کز بر شاه ما رسد
\\
سر مکش از چنین سری کید تاج از آن سرش
&&
کبر مکن بر آن کسی کز سوی کبریا رسد
\\
نقد الست می‌رسد دست به دست می‌رسد
&&
زود بکن بلی بلی ور نکنی بلا رسد
\\
من که خریده ویم پرده دریده ویم
&&
رگ به رگ مرا از او لطف جدا جدا رسد
\\
گر به تمام مستمی راز غمش بگفتمی
&&
گفت تمام چون شکر زان مه خوش لقا رسد
\\
\end{longtable}
\end{center}
