\begin{center}
\section*{غزل شماره ۷۰۶: روزم به عیادت شب آمد}
\label{sec:0706}
\addcontentsline{toc}{section}{\nameref{sec:0706}}
\begin{longtable}{l p{0.5cm} r}
روزم به عیادت شب آمد
&&
جانم به زیارت لب آمد
\\
از بس که شنید یاربم چرخ
&&
از یارب من به یارب آمد
\\
یار آمد و جام باده بر کف
&&
زان می که خلاف مذهب آمد
\\
هر بار ز جرعه مست بودم
&&
این بار قدح لبالب آمد
\\
عالم به خمار اوست معجب
&&
پس وی چه عجب که معجب آمد
\\
بر هر فلکی که ماه او تافت
&&
خورشید کمینه کوکب آمد
\\
گویی مه نو سواره دیدش
&&
کز عشق چو نعل مرکب آمد
\\
این بس نبود شرف جهان را
&&
کو روح و جهان چو قالب آمد
\\
شاد آن دل روشنی که بیند
&&
دل را که چه سان مقرب آمد
\\
از پرتو دل جهان پرگل
&&
زیبا و خوش و مؤدب آمد
\\
هر میوه به وقت خویش سر کرد
&&
هر فصل چه سان مرتب آمد
\\
بس کن که به پیش ناطق کل
&&
گویای خمش مهذب آمد
\\
بس کن که عروس جان ز جلوه
&&
با نامحرم معذب آمد
\\
من بس نکنم که بی‌دلان را
&&
این کلبشکر مجرب آمد
\\
من بس نکنم به کوری آنک
&&
اندر ره دین مذبذب آمد
\\
خامش که به گفت حاجتی نیست
&&
چون جذب فرغت فانصب آمد
\\
خود گفتن بنده جذب حقست
&&
کز بنده به بنده اقرب آمد
\\
\end{longtable}
\end{center}
