\begin{center}
\section*{غزل شماره ۴۹۶: طرب ای بحر اصل آب حیات}
\label{sec:0496}
\addcontentsline{toc}{section}{\nameref{sec:0496}}
\begin{longtable}{l p{0.5cm} r}
طرب ای بحر اصل آب حیات
&&
ای تو ذات و دگر مهان چو صفات
\\
اه چه گفتم کجاست تا به کجا
&&
کو یکی وصف لایق چو تو ذات
\\
هر که در عشق روت غوطی خورد
&&
ریش خندی زند به هست و فوات
\\
شرق تا غرب شکرین گردد
&&
گر نماید بدو شکرت نبات
\\
جان من جام عشق دلبر دید
&&
لعل چون خون خویش گفت که‌هات
\\
جان بنوشید و از سرش تا پای
&&
آتشی برفروخت از شررات
\\
مست شد جان چنان که نشناسد
&&
خویشتن را ز می جز از طاعات
\\
بانگ آمد ز عرش مژده تو را
&&
که ز من درگذشت نور عطات
\\
مژده از بخششی که نتوان یافت
&&
به دو صد سال خون چشم و عنات
\\
که به هر قطره از پیاله او
&&
مرده زنده شود عجوز فتات
\\
گرش از عشق دوست بو بودی
&&
کی نگوسار گشتی هرگز لات
\\
چون شدی مست او کجا دانی
&&
تو رکوع و سجود در صلوات
\\
چونک بیخود شدی ز پرتو عشق
&&
جسم آن شاه ماست جان صلات
\\
چو بمردی به پای شمس الدین
&&
زنده گشتی تو ایمنی ز ممات
\\
داد مخدوم از خداوندیش
&&
بهر ملک ابد مثال و برات
\\
\end{longtable}
\end{center}
