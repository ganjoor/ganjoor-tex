\begin{center}
\section*{غزل شماره ۲۵۷۸: ای بود تو از کی نی وی ملک تو تا کی نی}
\label{sec:2578}
\addcontentsline{toc}{section}{\nameref{sec:2578}}
\begin{longtable}{l p{0.5cm} r}
ای بود تو از کی نی وی ملک تو تا کی نی
&&
عشق تو و جان من جز آتش و جز نی نی
\\
بر کشته دیت باشد ای شادی این کشته
&&
صد کشته هو دیدم امکان یکی هی نی
\\
ای دیده عجایب‌ها بنگر که عجب این است
&&
معشوق بر عاشق با وی نی و بی‌وی نی
\\
امروز به بستان آ در حلقه مستان آ
&&
مستان خرف از مستی آن جا قدح و می نی
\\
مستند نه از ساغر بنگر به شتر بنگر
&&
برخوان افلا ینظر معنیش بر این پی نی
\\
در مؤمن و در کافر بنگر تو به چشم سر
&&
جز نعره یا رب نی جز ناله یا حی نی
\\
آن جا که همی‌پویی زان است کز او سیری
&&
زان جا که گریزانی جز لطف پیاپی نی
\\
از ابجد اندیشه یا رب تو بشو لوحم
&&
در مکتب درویشان خود ابجد و حطی نی
\\
شمس الحق تبریزی آن جا که تو پیروزی
&&
از تابش خورشیدت هرگز خطری دی نی
\\
\end{longtable}
\end{center}
