\begin{center}
\section*{غزل شماره ۲۹۳۵: گر چه به زیر دلقی شاهی و کیقبادی}
\label{sec:2935}
\addcontentsline{toc}{section}{\nameref{sec:2935}}
\begin{longtable}{l p{0.5cm} r}
گر چه به زیر دلقی شاهی و کیقبادی
&&
ور چه ز چشم دوری در جان و سینه یادی
\\
گر چه به نقش پستی بر آسمان شدستی
&&
قندیل آسمانی نه چرخ را عمادی
\\
بستی تو هست ما را بر نیستی مطلق
&&
بستی مراد ما را بر شرط بی‌مرادی
\\
تا هیچ سست پایی در کوی تو نیاید
&&
پیش تو شیر آید شیری و شیرزادی
\\
سر را نهد به بیرون بی‌سر بر تو آید
&&
تا بشنود ز گردون بی‌گوش یا عبادی
\\
یک ماهه راه را تو بگذر برو به روزی
&&
زیرا که چون سلیمان بر بارگیر بادی
\\
دینار و زر چه باشد انبار جان بیاور
&&
جان ده درم رها کن گر عاشق جوادی
\\
حاجت نیاید ای جان در راه تو قلاوز
&&
چون نور و ماهتاب است این مهتدی و هادی
\\
مه نور و تاب خود را از جا به جا کشاند
&&
چون اشتر عرب را از جا به جای حادی
\\
از صد هزار توبه بشناخت جان مجنون
&&
چون بوی گور لیلی برداشت در منادی
\\
چون مه پی فزایش غمگین مشو ز کاهش
&&
زیرا ز بعد کاهش چون مه در ازدیادی
\\
هر لحظه دسته دسته ریحان به پیشت آید
&&
رسته ز دست رنجت وز خوب اعتقادی
\\
تشنیع بر سلیمان آری که گم شدم من
&&
گم شو چو هدهد ار تو دربند افتقادی
\\
یا صاحبی هذا دیباجه الرشاد
&&
الصبح قد تجلی حولوا عن الرقاد
\\
الشمس قد تلالا من غیر احتجاب
&&
و النصر قد توالی من غیر اجتهاد
\\
الروح فی المطار و الکأس فی الدوار
&&
و الهم فی الفرار و السکر فی امتداد
\\
\end{longtable}
\end{center}
