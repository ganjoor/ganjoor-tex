\begin{center}
\section*{غزل شماره ۵۰۸: شیر خدا بند گسستن گرفت}
\label{sec:0508}
\addcontentsline{toc}{section}{\nameref{sec:0508}}
\begin{longtable}{l p{0.5cm} r}
شیر خدا بند گسستن گرفت
&&
ساقی جان شیشه شکستن گرفت
\\
دزد دلم گشت گرفتار یار
&&
دزد مرا دست ببستن گرفت
\\
دوش چه شب بود که در نیم شب
&&
برق ز رخسار تو جستن گرفت
\\
عشق تو آورد شراب و کباب
&&
عقل به یک گوشه نشستن گرفت
\\
ساغر می قهقهه آغاز کرد
&&
خابیه خونابه گرستن گرفت
\\
در دل خم باده چو انداخت تیر
&&
بال و پر غصه گسستن گرفت
\\
پیر خرد دید که سرده توی
&&
دست ز مستان تو شستن گرفت
\\
طفل دلم را به کرم شیر ده
&&
چون سر پستان تو جستن گرفت
\\
جان من از شیر تو شد شیرگیر
&&
وز سگی نفس برستن گرفت
\\
ساقی باقی چو به جان باده داد
&&
عمر ابد یافت و بزستن گرفت
\\
بیش مگو راز که دلبر به خشم
&&
جانب من کژ نگرستن گرفت
\\
\end{longtable}
\end{center}
