\begin{center}
\section*{غزل شماره ۱۸۶۱: ای قاعده مستان در همدگر افتادن}
\label{sec:1861}
\addcontentsline{toc}{section}{\nameref{sec:1861}}
\begin{longtable}{l p{0.5cm} r}
ای قاعده مستان در همدگر افتادن
&&
استیزه گری کردن در شور و شر افتادن
\\
عاشق بتر از مست است عاشق هم از آن دست است
&&
گویم که چه باشد عشق در کان زر افتادن
\\
زر خود چه بود عاشق سلطان سلاطین است
&&
ایمن شدن از مردن وز تاج سر افتادن
\\
درویش به دلق اندر و اندر بغلش گوهر
&&
او ننگ چرا دارد از در به در افتادن
\\
مست آمد دوش آن مه افکنده کمر در ره
&&
آگه نبد از مستی او از کمر افتادن
\\
گفتم که دلا برجه می بر کف جان برنه
&&
کافتاد چنین وقتی وقت است درافتادن
\\
با بلبل بستانی همدست شدن دستی
&&
با طوطی روحانی اندر شکر افتادن
\\
من بی‌دل و دل داده در راه تو افتاده
&&
والله که نمی‌دانم جای دگر افتادن
\\
گر جام تو بشکستم مستم صنما مستم
&&
مستم مهل از دستم و اندر خطر افتادن
\\
این قاعده نوزاد است وین رسم نو افتاده‌ست
&&
شیشه شکنی کردن در شیشه گر افتادن
\\
\end{longtable}
\end{center}
