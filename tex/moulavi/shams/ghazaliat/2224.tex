\begin{center}
\section*{غزل شماره ۲۲۲۴: ای بمرده هر چه جان در پای او}
\label{sec:2224}
\addcontentsline{toc}{section}{\nameref{sec:2224}}
\begin{longtable}{l p{0.5cm} r}
ای بمرده هر چه جان در پای او
&&
هر چه گوهر غرقه در دریای او
\\
آتش عشقش خدایی می‌کند
&&
ای خدا هیهای او هیهای او
\\
جبرئیل و صد چو او گر سر کشد
&&
از سجود درگهش ای وای او
\\
چون مثالی برنویسد در فراق
&&
خون ببارد از خم طغرای او
\\
هر کی ماند زین قیامت بی‌خبر
&&
تا قیامت وای او ای وای او
\\
هر کی ناگه از چنان مه دور ماند
&&
ای خدایا چون بود شب‌های او
\\
در نظاره عاشقان بودیم دوش
&&
بر شمار ریگ در صحرای او
\\
خیمه در خیمه طناب اندر طناب
&&
پیش شاه عشق و لشکرهای او
\\
خیمه جان را ستون از نور پاک
&&
نور پاک از تابش سیمای او
\\
آب و آتش یک شده ز امروز او
&&
روز و شب محو است در فردای او
\\
عشق شیر و عاشقان اطفال شیر
&&
در میان پنجه صدتای او
\\
طفل شیر از زخم شیر ایمن بود
&&
بر سر پستان شیرافزای او
\\
در کدامین پرده پنهان بود عشق
&&
کس نداند کس نبیند جای او
\\
عشق چون خورشید ناگه سر کند
&&
برشود تا آسمان غوغای او
\\
\end{longtable}
\end{center}
