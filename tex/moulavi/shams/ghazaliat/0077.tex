\begin{center}
\section*{غزل شماره ۷۷: آب حیوان باید مر روح فزایی را}
\label{sec:0077}
\addcontentsline{toc}{section}{\nameref{sec:0077}}
\begin{longtable}{l p{0.5cm} r}
آب حیوان باید مر روح فزایی را
&&
ماهی همه جان باید دریای خدایی را
\\
ویرانه آب و گل چون مسکن بوم آمد
&&
این عرصه کجا شاید پرواز همایی را
\\
صد چشم شود حیران در تابش این دولت
&&
تو گوش مکش این سو هر کور عصایی را
\\
گر نقد درستی تو چون مست و قراضه ستی
&&
آخر تو چه پنداری این گنج عطایی را
\\
دلتنگ همی‌دانند کان جای که انصافست
&&
صد دل به فدا باید آن جان بقایی را
\\
دل نیست کم از آهن آهن نه که می‌داند
&&
آن سنگ که پیدا شد پولادربایی را
\\
عقل از پی عشق آمد در عالم خاک ار نی
&&
عقلی بنمی باید بی‌عهد و وفایی را
\\
خورشید حقایق‌ها شمس الحق تبریز است
&&
دل روی زمین بوسد آن جان سمایی را
\\
\end{longtable}
\end{center}
