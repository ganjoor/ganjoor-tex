\begin{center}
\section*{غزل شماره ۶۸۵: دل با دل دوست در حنین باشد}
\label{sec:0685}
\addcontentsline{toc}{section}{\nameref{sec:0685}}
\begin{longtable}{l p{0.5cm} r}
دل با دل دوست در حنین باشد
&&
گویای خموش همچنین باشد
\\
گویم سخن و زبان نجنبانم
&&
چون گوش حسود در کمین باشد
\\
دانم که زبان و گوش غمازند
&&
با دل گویم که دل امین باشد
\\
صد شعلهٔ آتش است در دیده
&&
از نکته دل که آتشین باشد
\\
خود طرفه‌تر این که در دل آتش
&&
چندین گل و سرو و یاسمین باشد
\\
زان آتش باغ سبزتر گردد
&&
تا آتش و آب همنشین باشد
\\
ای روح مقیم مرغزاری تو
&&
کان جا دل و عقل دانه چین باشد
\\
آن سوی که کفر و دین نمی‌گنجد
&&
کی ما و من فلان دین باشد
\\
\end{longtable}
\end{center}
