\begin{center}
\section*{غزل شماره ۲۱۶۳: چو شیرینتر نمود ای جان مها شور و بلای تو}
\label{sec:2163}
\addcontentsline{toc}{section}{\nameref{sec:2163}}
\begin{longtable}{l p{0.5cm} r}
چو شیرینتر نمود ای جان مها شور و بلای تو
&&
بهشتم جان شیرین را که می‌سوزد برای تو
\\
روان از تو خجل باشد دلم را پا به گل باشد
&&
مرا چه جای دل باشد چو دل گشته‌ست جای تو
\\
تو خورشیدی و دل در چه بتاب از چه به دل گه گه
&&
که می‌کاهد چو ماه ای مه به عشق جان فزای تو
\\
ز خود مسم به تو زرم به خود سنگم به تو درم
&&
کمر بستم به عشق اندر به اومید قبای تو
\\
گرفتم عشق را در بر کله بنهاده‌ام از سر
&&
منم محتاج و می‌گویم ز بی‌خویشی دعای تو
\\
دلا از حد خود مگذر برون کن باد را از سر
&&
به خاک کوی او بنگر ببین صد خونبهای تو
\\
اگر ریزم وگر رویم چه محتاج تو مه رویم
&&
چو برگ کاه می‌پرم به عشق کهربای تو
\\
ایا تبریز خوش جایم ز شمس الدین به هیهایم
&&
زنم لبیک و می‌آیم بدان کعبه لقای تو
\\
\end{longtable}
\end{center}
