\begin{center}
\section*{غزل شماره ۲۴۴۳: ای تو ملول از کار من من تشنه تر هر ساعتی}
\label{sec:2443}
\addcontentsline{toc}{section}{\nameref{sec:2443}}
\begin{longtable}{l p{0.5cm} r}
ای تو ملول از کار من من تشنه تر هر ساعتی
&&
آخر چه کم گردد ز تو کز تو برآید حاجتی
\\
بر تو زیانی کی شود از تو عدم گر شیء شود
&&
معدوم یابد خلعتی گیرد ز هستی رایتی
\\
یا مستحق مرحمت یابد مقام و مرتبت
&&
برخواند اندر مکتبت از لوح محفوظ آیتی
\\
ای رحمة للعالمین بخشی ز دریای یقین
&&
مر خاکیان را گوهری مر ماهیان را راحتی
\\
موجش گهی گوهر دهد لطفش گهی کشتی کشد
&&
چندین خلایق اندر او مر هر یکی را حالتی
\\
خود پیشتر اجزای او در سجده همچون شاکران
&&
وز بهر خدمت موج او گه گه نماید قامتی
\\
در پیش دریای نهان این هفت دریای جهان
&&
چون واهب اندر بخششی چون راهب اندر طاعتی
\\
دریای پرمرجان ما عمر دراز و جان ما
&&
پس عمر ما بی‌حد بود ما را نباشد غایتی
\\
ای قطره گر آگه شوی با سیل‌ها همره شوی
&&
سیلت سوی دریا برد پیشت نباشد آفتی
\\
ور سرکشی غافل شوی آن سیل عشق مستوی
&&
گوش تو گیرد می‌کشد کو بر تو دارد رافتی
\\
مستفعلن مستفعلن اکنون شکر پنهان کنم
&&
کز غیب جوقی طوطیان آورده اندم غارتی
\\
شکر نگر تو نو به نو آواز خاییدن شنو
&&
نی این شکر را صورتی نی طوطیان را آلتی
\\
دارد خدا قندی دگر کان ناید اندر نیشکر
&&
طوطی و حلقوم بشر آن را ندارد طاقتی
\\
چون شمس تبریزی که او گنجا ندارد در فلک
&&
کان مطلع خورشید او دارد عجایب ساحتی
\\
\end{longtable}
\end{center}
