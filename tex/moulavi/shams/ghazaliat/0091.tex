\begin{center}
\section*{غزل شماره ۹۱: در آب فکن ساقی بط زاده آبی را}
\label{sec:0091}
\addcontentsline{toc}{section}{\nameref{sec:0091}}
\begin{longtable}{l p{0.5cm} r}
در آب فکن ساقی بط زاده آبی را
&&
بشتاب و شتاب اولی مستان شبابی را
\\
ای جان بهار و دی وی حاتم نقل و می
&&
پر کن ز شکر چون نی بوبکر ربابی را
\\
ای ساقی شور و شر هین عیش بگیر از سر
&&
پر کن ز می احمر سغراق و شرابی را
\\
بنما ز می فرخ این سو اخ و آن سو اخ
&&
بربای نقاب از رخ معشوق نقابی را
\\
احسنت زهی یار او شاخ گل بی‌خار او
&&
شاباش زهی دارو دل‌های کبابی را
\\
صد حلقه نگر شیدا زان باده ناپیدا
&&
کاسد کند این صهبا صد خمر لعابی را
\\
مستان چمن پنهان اشکوفه ز شاخ افشان
&&
صد کوه چو که غلطان سیلاب حبابی را
\\
گر آن قدح روشن جانست نهان از تن
&&
پنهان نتوان کردن مستی و خرابی را
\\
ماییم چو کشت ای جان سرسبز در این میدان
&&
تشنه شده و جویان باران سحابی را
\\
چون رعد نه‌ای خامش چون پرده تست این هش
&&
وز صبر و فنا می‌کش طوطی خطابی را
\\
\end{longtable}
\end{center}
