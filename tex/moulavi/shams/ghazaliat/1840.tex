\begin{center}
\section*{غزل شماره ۱۸۴۰: مطرب خوش نوای من عشق نواز همچنین}
\label{sec:1840}
\addcontentsline{toc}{section}{\nameref{sec:1840}}
\begin{longtable}{l p{0.5cm} r}
مطرب خوش نوای من عشق نواز همچنین
&&
نغنغه دگر بزن پرده تازه برگزین
\\
مطرب روح من تویی کشتی نوح من تویی
&&
فتح و فتوح من تویی یار قدیم و اولین
\\
ای ز تو شاد جان من بی‌تو مباد جان من
&&
دل به تو داد جان من با غم توست همنشین
\\
تلخ بود غم بشر وین غم عشق چون شکر
&&
این غم عشق را دگر بیش به چشم غم مبین
\\
چون غم عشق ز اندرون یک نفسی رود برون
&&
خانه چو گور می شود خانگیان همه حزین
\\
سرمه ماست گرد تو راحت ماست درد تو
&&
کیست حریف و مرد تو ای شه مردآفرین
\\
تا که تو را شناختم همچو نمک گداختم
&&
شکم و شک فنا شود چون برسد بر یقین
\\
من شبم از سیه دلی تو مه خوب و مفضلی
&&
ظلمت شب عدم شود در رخ ماه راه بین
\\
عشق ز توست همچو جان عقل ز توست لوح خوان
&&
کان و مکان قراضه جو بحر ز توست دانه چین
\\
مست تو بوالفضول شد وز دو جهان ملول شد
&&
عشق تو را رسول شد او است نکال هر زمین
\\
در تبریز شمس دین دارد مطلعی دگر
&&
نیست ز مشرق او مبین نیست به مغرب او دفین
\\
\end{longtable}
\end{center}
