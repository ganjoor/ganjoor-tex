\begin{center}
\section*{غزل شماره ۱۶۹۹: دل را ز من بپوشی یعنی که من ندانم}
\label{sec:1699}
\addcontentsline{toc}{section}{\nameref{sec:1699}}
\begin{longtable}{l p{0.5cm} r}
دل را ز من بپوشی یعنی که من ندانم
&&
خط را کنی مسلسل یعنی که من نخوانم
\\
بر تخته خیالات آن را نه من نبشتم
&&
چون سر دل ندانم کاندر میان جانم
\\
از آفتاب بیشم ذرات روح پیشم
&&
رقصان و ذکرگویان سوی گهرفشانم
\\
گر نور خود نبودی ذرات کی نمودی
&&
ای ذره چون گریزی از جذبه عیانم
\\
پروانه وار عالم پران به گرد شمعم
&&
فریش می فرستم پریش می ستانم
\\
در خلوت است عشقی زین شرح شرحه شرحه
&&
گر شرح عشق خواهی پیش ویت نشانم
\\
ور زان که در گمانی نقش گمان ز من دان
&&
زان نقش منکران را در قعر می کشانم
\\
ور زان که در یقینی دام یقین ز من بین
&&
زان دام مقبلان را از کفر می رهانم
\\
ور درد و رنج داری در من نظر کن از وی
&&
کان تیر رنج نجهد الا که از کمانم
\\
ور رنج گشت راحت در من نگر همان دم
&&
می بین که آن نشانه‌ست از لطف بی‌نشانم
\\
هر جا که این جمال است داد و ستد حلال است
&&
وان جا که ذوالجلال است من دم زدن نتانم
\\
\end{longtable}
\end{center}
