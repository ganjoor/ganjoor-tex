\begin{center}
\section*{غزل شماره ۱۲۴۶: دوش رفتم در میان مجلس سلطان خویش}
\label{sec:1246}
\addcontentsline{toc}{section}{\nameref{sec:1246}}
\begin{longtable}{l p{0.5cm} r}
دوش رفتم در میان مجلس سلطان خویش
&&
بر کف ساقی بدیدم در صراحی جان خویش
\\
گفتمش ای جان جان ساقیان بهر خدا
&&
پر کنی پیمانه و نشکنی پیمان خویش
\\
خوش بخندید و بگفت ای ذوالکرم خدمت کنم
&&
حرمتت دارم به حق و حرمت ایمان خویش
\\
ساغری آورد و بوسید و نهاد او بر کفم
&&
پرمی رخشنده همچون چهره رخشان خویش
\\
سجده کردم پیش او و درکشیدم جام را
&&
آتشی افکند در من می ز آتشدان خویش
\\
چون پیاپی کرد و بر من ریخت زان سان جام چند
&&
آن می چون زر سرخم برد اندر کان خویش
\\
از گل رخسار او سرسبز دیدم باغ خویش
&&
ز ابروی چون سنبل او پخته دیدم نان خویش
\\
بخت و روزی هر کسی اندر خراباتی روید
&&
من کیم غمخوارگی را یافتم من آن خویش
\\
بولهب را دیدم آن جا دست می‌خایید سخت
&&
بوهریره دست کرده در دل انبان خویش
\\
بولهب چون پشت بود و رو نبیند هیچ پشت
&&
بوهریره روی کرده در مه و کیوان خویش
\\
بولهب در فکر رفته حجت و برهان طلب
&&
بوهریره حجت خویش است و هم برهان خویش
\\
نیست هر خم لایق می هین سر خم را ببند
&&
تا برآرد خم دیگر ساقی از خمدان خویش
\\
بس کنم تا میر مجلس بازگوید با شما
&&
داستان صد هزاران مجلس پنهان خویش
\\
\end{longtable}
\end{center}
