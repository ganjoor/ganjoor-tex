\begin{center}
\section*{غزل شماره ۲۸۳۰: هله پاسبان منزل تو چگونه پاسبانی}
\label{sec:2830}
\addcontentsline{toc}{section}{\nameref{sec:2830}}
\begin{longtable}{l p{0.5cm} r}
هله پاسبان منزل تو چگونه پاسبانی
&&
که ببرد رخت ما را همه دزد شب نهانی
\\
بزن آب سرد بر رو بجه و بکن علالا
&&
که ز خوابناکی تو همه سود شد زیانی
\\
که چراغ دزد باشد شب و خواب پاسبانان
&&
به دمی چراغشان را ز چه رو نمی‌نشانی
\\
بگذار کاهلی را چو ستاره شب روی کن
&&
ز زمینیان چه ترسی که سوار آسمانی
\\
دو سه عوعو سگانه نزند ره سواران
&&
چه برد ز شیر شرزه سگ و گاو کاهدانی
\\
سگ خشم و گاو شهوت چه زنند پیش شیری
&&
که به بیشه حقایق بدرد صف عیانی
\\
نه دو قطره آب بودی که سفینه‌ای و نوحی
&&
به میان موج طوفان چپ و راست می‌دوانی
\\
چو خدا بود پناهت چه خطر بود ز راهت
&&
به فلک رسد کلاهت که سر همه سرانی
\\
چه نکو طریق باشد که خدا رفیق باشد
&&
سفر درشت گردد چو بهشت جاودانی
\\
تو مگو که ارمغانی چه برم پی نشانی
&&
که بس است مهر و مه را رخ خویش ارمغانی
\\
تو اگر روی وگر نی بدود سعادت تو
&&
همه کار برگزارد به سکون و مهربانی
\\
چو غلام توست دولت کندت هزار خدمت
&&
که ندارد از تو چاره و گرش ز در برانی
\\
تو بخسپ خوش که بختت ز برای تو نخسپد
&&
تو بگیر سنگ در کف که شود عقیق کانی
\\
به فلک برآ چو عیسی ارنی بگو چو موسی
&&
که خدا تو را نگوید که خموش لن ترانی
\\
خمش ای دل و چه چاره سر خم اگر بگیری
&&
دل خنب برشکافد چو بجوشد این معانی
\\
دو هزار بار هر دم تو بخوانی این غزل را
&&
اگر آن سوی حقایق سیران او بدانی
\\
\end{longtable}
\end{center}
