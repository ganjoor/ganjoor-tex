\begin{center}
\section*{غزل شماره ۱۹۸۷: هله نیم مست گشتم قدحی دگر مدد کن}
\label{sec:1987}
\addcontentsline{toc}{section}{\nameref{sec:1987}}
\begin{longtable}{l p{0.5cm} r}
هله نیم مست گشتم قدحی دگر مدد کن
&&
چو حریف نیک داری تو به ترک نیک و بد کن
\\
منگر که کیست گریان ز جفا و کیست عریان
&&
نه وصی آدمی تو بنشین و کار خود کن
\\
نظری به سوی می کن به نوای چنگ و نی کن
&&
نظری دگر به سوی رخ یار سروقد کن
\\
شکرت چو آرزو شد ز لب شکرفروشش
&&
چو عباس دبس زودتر ز شکرفروش کدکن
\\
نه که کودکم که میلم به مویز و جوز باشد
&&
تو مویز و جوز خود را بستان در آن سبد کن
\\
شکر خوش تبرزد که هزار جان به ارزد
&&
حسد ار کنی تو باری پی آن شکر حسد کن
\\
به بت شکرفشان شو ز لبش شکرستان شو
&&
جهت قران ماهش چو منجمان رصد کن
\\
چو رسید ماه روزه نه ز کاسه گو نه کوزه
&&
پس از این نشاط و مستی ز صراحی ابد کن
\\
به سماع و طوی بنشین به میان کوی بنشین
&&
که کسی خورت نبیند طرب از می احد کن
\\
چو عروس جان ز مستی برسد به کوی هستی
&&
خورشش از این طبق ده تتقش هم از خرد کن
\\
ز سخن ملول گشتی که کسیت نیست محرم
&&
سبک آینه بیان را تو بگیر و در نمد کن
\\
\end{longtable}
\end{center}
