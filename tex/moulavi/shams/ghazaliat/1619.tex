\begin{center}
\section*{غزل شماره ۱۶۱۹: تو گواه باش خواجه که ز توبه توبه کردم}
\label{sec:1619}
\addcontentsline{toc}{section}{\nameref{sec:1619}}
\begin{longtable}{l p{0.5cm} r}
تو گواه باش خواجه که ز توبه توبه کردم
&&
بشکست جام توبه چو شراب عشق خوردم
\\
به جمال بی‌نظیرت به شراب شیرگیرت
&&
که به گرد عهد و توبه نروم دگر نگردم
\\
به لب شکرفشانت به ضمیر غیب دانت
&&
که نه سخره جهانم نه زبون سرخ و زردم
\\
به رخ چو آفتابت به حلاوت خطابت
&&
که هزارساله ره من ز ورای گرم و سردم
\\
به هوای همچو رخشت به لوای روح بخشت
&&
که به جز تو کس نداند که کیم چگونه مردم
\\
به سعادت صباحت به قیامت صبوحت
&&
که سجل آسمان را به فر تو درنوردم
\\
هله ای شه مخلد تو بگو به ساقی خود
&&
چو کسی ترش درآید دهدش ز درد در دم
\\
هله تا دوی نباشد کهن و نوی نباشد
&&
که در این مقام عشرت من از آن جمع فردم
\\
بدهش از آن رحیقی که شود خوشی عشیقی
&&
که ز مستی و خرابی برهد ز عکس و طردم
\\
نه در او حسد بماند نه غم جسد بماند
&&
خوش و پاک بازآید به سوی بساط نردم
\\
به صفا مثال زهره به رضا به سان مهره
&&
نه نصیبه جو نه بهره که ببردم و نبردم
\\
بپریده از زمانه ز هوای دام و دانه
&&
که در این قمارخانه چو گواه بی‌نبردم
\\
پس از این خموش باشم همه گوش و هوش باشم
&&
که نه بلبلم نه طوطی همه قند و شاخ وردم
\\
\end{longtable}
\end{center}
