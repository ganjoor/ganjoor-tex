\begin{center}
\section*{غزل شماره ۳۵۰: ز بعد وقت نومیدی امیدیست}
\label{sec:0350}
\addcontentsline{toc}{section}{\nameref{sec:0350}}
\begin{longtable}{l p{0.5cm} r}
ز بعد وقت نومیدی امیدیست
&&
به زیر کوری اندر سینه دیدیست
\\
نبینی نور چون دانی تو کوری
&&
سیه نادیده کی داند سپیدیست
\\
قرین صد هزاران نقش و معنی
&&
نهان تصریف سلطان وحیدیست
\\
که جنباننده این نقش و معنی‌ست
&&
چو بادی رقص‌های شاخ بیدیست
\\
مشو نومید از دشنام دلدار
&&
که بعد رنج روزه روز عیدیست
\\
که یبقی الحب ما بقی العتاب
&&
که هر نقصی کشاننده مزیدیست
\\
رها کن گفت به از گفت یابی
&&
یقین هر حادثی را خود ندیدیست
\\
\end{longtable}
\end{center}
