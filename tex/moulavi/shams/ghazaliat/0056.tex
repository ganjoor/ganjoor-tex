\begin{center}
\section*{غزل شماره ۵۶: عطارد مشتری باید متاع آسمانی را}
\label{sec:0056}
\addcontentsline{toc}{section}{\nameref{sec:0056}}
\begin{longtable}{l p{0.5cm} r}
عطارد مشتری باید متاع آسمانی را
&&
مهی مریخ چشم ارزد چراغ آن جهانی را
\\
چو چشمی مقترن گردد بدان غیبی چراغ جان
&&
ببیند بی‌قرینه او قرینان نهانی را
\\
یکی جان عجب باید که داند جان فدا کردن
&&
دو چشم معنوی باید عروسان معانی را
\\
یکی چشمیست بشکفته صقال روح پذرفته
&&
چو نرگس خواب او رفته برای باغبانی را
\\
چنین باغ و چنین شش جو پس این پنج و این شش جو
&&
قیاسی نیست کمتر جو قیاس اقترانی را
\\
به صف‌ها رایت نصرت به شب‌ها حارس امت
&&
نهاده بر کف وحدت در سبع المثانی را
\\
شکسته پشت شیطان را بدیده روی سلطان را
&&
که هر خس از بنا داند به استدلال بانی را
\\
زهی صافی زهی حری مثال می خوشی مری
&&
کسی دزدد چنین دری که بگذارد عوانی را
\\
الی البحر توجهنا و من عذب تفکهنا
&&
لقینا الدر مجانا فلا نبغی الدنانی را
\\
لقیت الماء عطشانا لقیت الرزق عریانا
&&
صحبت اللیث احیانا فلا اخشی السنانی را
\\
توی موسی عهد خود درآ در بحر جزر و مد
&&
ره فرعون باید زد رها کن این شبانی را
\\
الا ساقی به جان تو به اقبال جوان تو
&&
به ما ده از بنان تو شراب ارغوانی را
\\
بگردان باده شاهی که همدردی و همراهی
&&
نشان درد اگر خواهی بیا بنگر نشانی را
\\
بیا درده می احمر که هم بحر است و هم گوهر
&&
برهنه کن به یک ساغر حریف امتحانی را
\\
برو ای رهزن مستان رها کن حیله و دستان
&&
که ره نبود در این بستان دغا و قلتبانی را
\\
جواب آنک می‌گوید به زر نخریده‌ای جان را
&&
که هندو قدر نشناسد متاع رایگانی را
\\
\end{longtable}
\end{center}
