\begin{center}
\section*{غزل شماره ۲۶۶۱: کجا شد عهد و پیمانی که کردی}
\label{sec:2661}
\addcontentsline{toc}{section}{\nameref{sec:2661}}
\begin{longtable}{l p{0.5cm} r}
کجا شد عهد و پیمانی که کردی
&&
کجا شد قول و سوگندی که خوردی
\\
نگفتی چرخ تا گردان بود گرد
&&
از این سرگشته هرگز برنگردی
\\
نگفتی تا بود خورشید دلگرم
&&
نکاهد گرم ما را هیچ سردی
\\
نگفتی یک دل و مردانه باشیم
&&
به جان جمله مردان و بمردی
\\
مرا گویی اگر من جور کردم
&&
بدان کردم که پیش از من تو کردی
\\
چرا شاید که با چون من گدایی
&&
چو تو شاهنشهی گیرد نبردی
\\
میان ما و تو سرکنگبین است
&&
ز من سرکه ز تو شکرنوردی
\\
چو من سرکه فروشم پس تو شکر
&&
بیفزا چون به شیرینی تو فردی
\\
منم خاک و چو خاکی باد یابد
&&
تو عذرش نه مگویش گرد کردی
\\
نباشد راه را عار از چو من گرد
&&
که زر را عار نبود رنگ زردی
\\
شهاب آتش ما زنده بادا
&&
چو القاب شهاب سهروردی
\\
\end{longtable}
\end{center}
