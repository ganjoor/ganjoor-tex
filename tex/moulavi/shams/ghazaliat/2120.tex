\begin{center}
\section*{غزل شماره ۲۱۲۰: دگرباره چو مه کردیم خرمن}
\label{sec:2120}
\addcontentsline{toc}{section}{\nameref{sec:2120}}
\begin{longtable}{l p{0.5cm} r}
دگرباره چو مه کردیم خرمن
&&
خرامیدیم بر کوری دشمن
\\
دگربار آفتاب اندر حمل شد
&&
بخندانید عالم را چو گلشن
\\
ز طنازی شکوفه لب گشاده‌ست
&&
به غمازی زبان گشته‌ست سوسن
\\
چه اطلس‌ها که پوشیدند در باغ
&&
از آن خیاط بی‌مقراض و سوزن
\\
طبق بر سر نهاده هر درختی
&&
پر از حلوای بی‌دوشاب و روغن
\\
دهل کردیم اشکم را دگربار
&&
چو طبال ربیعی شد دهلزن
\\
ز ره گشته ز باد آن روی آبی
&&
که بود اندر زمستان همچو آهن
\\
بهار نو مگر داوود وقت است
&&
کز آن آهن ببافیده‌ست جوشن
\\
ندا زد در عدم حق کای ریاحین
&&
برون رفتند آن سردان ز مسکن
\\
به سربالای هستی روی آرید
&&
چو مرغان خلیلی از نشیمن
\\
رسید آن لک لک عارف ز غربت
&&
مسبح گرد او مرغان الکن
\\
هزیمتیان که پنهان گشته بودند
&&
برون کردند سر یک یک ز روزن
\\
برون کردند سرها سبزپوشان
&&
پر از طوق و جواهر گوش و گردن
\\
سماع است و هزاران حور در باغ
&&
همی‌کوبند پا بر گور بهمن
\\
هلا ای بید گوش و سر بجنبان
&&
اگر داری چو نرگس چشم روشن
\\
همی‌گویم سخن را ترک من کن
&&
ستیزه رو است می‌آید پی من
\\
نخواهم من برای روی سختش
&&
حدیث عاشقان را فاش کردن
\\
ینادی الورد یا اصحاب مدین
&&
الا فافرح بنا من کان یحزن
\\
فان الارض اخضرت بنور
&&
و قال الله للعاری تزین
\\
و عاد الهاربون الی حیاه
&&
و دیوان النشور غدا مدون
\\
بامر الله ماتوا ثم جاؤا
&&
و ابلاهم زمانا ثم احسن
\\
و شمس الله طالعه به فضل
&&
و برهان صنایعه مبرهن
\\
و صبغنا النبات بغیر صبغ
&&
نقدر حجمها من غیر ملبن
\\
جنان فی جنان فی جنان
&&
الا یا حایرا فیها توطن
\\
و هیجنا النفوس الی المعالی
&&
فذا نال الوصال و ذا تفرعن
\\
الا فاسکت و کلمهم به صمت
&&
فان الصمت للاسرار ابین
\\
\end{longtable}
\end{center}
