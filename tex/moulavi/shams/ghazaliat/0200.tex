\begin{center}
\section*{غزل شماره ۲۰۰: نام شتر به ترکی چه بود بگو دوا}
\label{sec:0200}
\addcontentsline{toc}{section}{\nameref{sec:0200}}
\begin{longtable}{l p{0.5cm} r}
نام شتر به ترکی چه بود بگو دوا
&&
نام بچه ش چه باشد او خود پیش دوا
\\
ما زاده قضا و قضا مادر همه‌ست
&&
چون کودکان دوان شده‌ایم از پی قضا
\\
ما شیر از او خوریم و همه در پیش پریم
&&
گر شرق و غرب تازد ور جانب سما
\\
طبل سفر ز دست قدم در سفر نهیم
&&
در حفظ و در حمایت و در عصمت خدا
\\
در شهر و در بیابان همراه آن مهیم
&&
ای جان غلام و بنده آن ماه خوش لقا
\\
آن جاست شهر کان شه ارواح می‌کشد
&&
آن جاست خان و مان که بگوید خدا بیا
\\
کوته شود بیابان چون قبله او بود
&&
پیش و سپس چمن بود و سرو دلربا
\\
کوهی که در ره آید هم پشت خم دهد
&&
کای قاصدان معدن اجلال مرحبا
\\
همچون حریر نرم شود سنگلاخ راه
&&
چون او بود قلاوز آن راه و پیشوا
\\
ما سایه وار در پی آن مه دوان شدیم
&&
ای دوستان همدل و همراه الصلا
\\
دل را رفیق ما کند آن کس که عذر هست
&&
زیرا که دل سبک بود و چست و تیزپا
\\
دل مصر می‌رود که به کشتیش وهم نیست
&&
دل مکه می‌رود که نجوید مهاره را
\\
از لنگی تنست و ز چالاکی دلست
&&
کز تن نجست حق و ز دل جست آن وفا
\\
اما کجاست آن تن همرنگ جان شده
&&
آب و گلی شده‌ست بر ارواح پادشا
\\
ارواح خیره مانده که این شوره خاک بین
&&
از حد ما گذشت و ملک گشت و مقتدا
\\
چه جای مقتدا که بدان جا که او رسید
&&
گر پا نهیم پیش بسوزیم در شقا
\\
این در گمان نبود در او طعن می‌زدیم
&&
در هیچ آدمی منگر خوار ای کیا
\\
ما همچو آب در گل و ریحان روان شویم
&&
تا خاک‌های تشنه ز ما بر دهد گیا
\\
بی دست و پاست خاک جگرگرم بهر آب
&&
زین رو دوان دوان رود آن آب جوی‌ها
\\
پستان آب می خلد ایرا که دایه اوست
&&
طفل نبات را طلبد دایه جا به جا
\\
ما را ز شهر روح چنین جذب‌ها کشید
&&
در صد هزار منزل تا عالم فنا
\\
باز از جهان روح رسولان همی‌رسند
&&
پنهان و آشکار بازآ به اقربا
\\
یاران نو گرفتی و ما را گذاشتی
&&
ما بی‌تو ناخوشیم اگر تو خوشی ز ما
\\
ای خواجه این ملالت تو ز آه اقرباست
&&
با هر کی جفت گردی آنت کند جدا
\\
خاموش کن که همت ایشان پی توست
&&
تأثیر همت‌ست تصاریف ابتلا
\\
\end{longtable}
\end{center}
