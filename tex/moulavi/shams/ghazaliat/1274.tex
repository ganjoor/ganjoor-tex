\begin{center}
\section*{غزل شماره ۱۲۷۴: خواجه چرا کرده‌ای روی تو بر ما ترش}
\label{sec:1274}
\addcontentsline{toc}{section}{\nameref{sec:1274}}
\begin{longtable}{l p{0.5cm} r}
خواجه چرا کرده‌ای روی تو بر ما ترش
&&
زین شکرستان برو هست کس این جا ترش
\\
در شکرستان دل قند بود هم خجل
&&
تو ز کجا آمدی ابرو و سیما ترش
\\
بر فلک آن طوطیان جمله شکر می‌خورند
&&
گر نپری بر فلک منگر بالا ترش
\\
رستم میدان فکر پیش عروسان بکر
&&
هیچ بود در وصال وقت تماشا ترش
\\
هر کی خورد می صبوح روز بود شیرگیر
&&
هر کی خورد دوغ هست امشب و فردا ترش
\\
مؤمن و ایمان و دین ذوق و حلاوت بود
&&
تو به کجا دیده‌ای طبله حلوا ترش
\\
این ترشی‌ها همه پیش تو زان جمع شد
&&
جنس رود سوی جنس ترش رود با ترش
\\
والله هر میوه‌ای کو نپزد ز آفتاب
&&
گر چه بود نیشکر نبود الا ترش
\\
سوزش خورشید عشق صبر بود صبر کن
&&
روز دو سه صبر به مذهب تو با ترش
\\
هر کی ترش بینیش دانک ز آتش گریخت
&&
غوره که در سایه ماند هست سر و پا ترش
\\
دعوه دل کرده‌ای وعده وفا کن مباش
&&
در صف دعوی چو شیر وقت تقاضا ترش
\\
بنگر در مصطفی چونک ترش شد دمی
&&
کرده عتابش عبس خواند مر او را ترش
\\
خامش و تهمت منه خواجه ترش نیست لیک
&&
گه گه قاصد کند مردم دانا ترش
\\
او چو شکر بوده است دل ز شکر پر ولیک
&&
در ادب کودکان باشد لالا ترش
\\
\end{longtable}
\end{center}
