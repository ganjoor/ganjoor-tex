\begin{center}
\section*{غزل شماره ۲۸۱۶: که شکیبد ز تو ای جان که جگرگوشه جانی}
\label{sec:2816}
\addcontentsline{toc}{section}{\nameref{sec:2816}}
\begin{longtable}{l p{0.5cm} r}
که شکیبد ز تو ای جان که جگرگوشه جانی
&&
چه تفکر کند از مکر و ز دستان که ندانی
\\
نه درونی نه برونی که از این هر دو فزونی
&&
نه ز شیری نه ز خونی نه از اینی نه از آنی
\\
برود فکرت جادو نهدت دام به هر سو
&&
تو همه دام و فنش را به یکی فن بدرانی
\\
چه بود باطن کبکی که دل باز نداند
&&
چه حبوب است زمین در که ز چرخ است نهانی
\\
کلهش بنهی وآنگه فکنی باز به سیلی
&&
چه کند بره مسکین چو کند شیر شبانی
\\
کله و تاج سرم را پی سیلی تو باید
&&
که مرا تاج تویی و جز تو جمله گرانی
\\
به کجا اسب دواند به کجا رخت کشاند
&&
ز تو چون جان بجهاند که تو صد جان جهانی
\\
به چه نقصان نگرندت به چه عیبی شکنندت
&&
به کی مانند کنندت که به مخلوق نمانی
\\
به ملاقات نشان ده ز خیالات امان ده
&&
مکشش زود زمان ده که تو قسام زمانی
\\
هله ای جان گشاده قدم صدق نهاده
&&
همه از پای فتاده تو خوش و دست زنانی
\\
شه و شاهین جلالی که چنین باپر و بالی
&&
نه گمانی نه خیالی همه عینی و عیانی
\\
چه بود طبع و رموزش به یکی شعله بسوزش
&&
به یکی تیر بدوزش که بسی سخته کمانی
\\
هله بر قوس بنه زه ز کمینگاه برون جه
&&
برهان خویش از این ده که تو زان شهر کلانی
\\
چو همه خانه دل را بگرفت آتش بالا
&&
بود اظهار زبانه به از اظهار زبانی
\\
\end{longtable}
\end{center}
