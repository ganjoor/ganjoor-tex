\begin{center}
\section*{غزل شماره ۱۱۲۶: سست مکن زه که من تیر توام چارپر}
\label{sec:1126}
\addcontentsline{toc}{section}{\nameref{sec:1126}}
\begin{longtable}{l p{0.5cm} r}
سست مکن زه که من تیر توام چارپر
&&
روی مگردان که من یک دله‌ام نی دوسر
\\
از تو زدن تیغ تیز وز دل و جان صد رضا
&&
یک سخنم چون قضا نی اگرم نی مگر
\\
گر بکشی ذوالفقار ثابتم و پایدار
&&
نی بگریزم چو باد نی بمرم چون شرر
\\
جان بسپارم به تیغ هیچ نگویم دریغ
&&
از جهت زخم تیغ ساخت حقم چون سپر
\\
تیغ زن ای آفتاب گردن شب را به تاب
&&
ظلمت شب‌ها ز چیست کوره خاک کدر
\\
معدن صبرست تن معدن شکر است دل
&&
معدن خنده‌ست شش معدن رحمت جگر
\\
بر سر من چون کلاه ساز شها تختگاه
&&
در بر خود چون قبا تنگ بگیرم به بر
\\
گفت کسی عشق را صورت و دست از کجا
&&
منبت هر دست و پا عشق بود در صور
\\
نی پدر و مادرت یک دمه‌ای عشق باخت
&&
چونک یگانه شدند چون تو کسی کرد سر
\\
عشق که بی‌دست او دست تو را دست ساخت
&&
بی‌سر و دستش مبین شکل دگر کن نظر
\\
رنگ همه روی‌ها آب همه جوی‌ها
&&
مفخر تبریز دان شمس حق ای دیده ور
\\
\end{longtable}
\end{center}
