\begin{center}
\section*{غزل شماره ۳۰۱۸: ای تو ز خوبی خویش آینه را مشتری}
\label{sec:3018}
\addcontentsline{toc}{section}{\nameref{sec:3018}}
\begin{longtable}{l p{0.5cm} r}
ای تو ز خوبی خویش آینه را مشتری
&&
سوخته باد آینه تا تو در او ننگری
\\
جان من از بحر عشق آب چو آتش بخورد
&&
در قدح جان من آب کند آذری
\\
خار شد این جان و دل در حسد آینه
&&
کو چو گلستان شده‌ست از نظر عبهری
\\
گم شده‌ام من ز خویش گر تو بیابی مرا
&&
زود سلامش رسان گو که خوشی خوشتری
\\
گر تو بیابی مرا از من من را بگو
&&
که من آواره‌ای گشته نهان چون پری
\\
مست نیم ای حریف عقل نرفت از سرم
&&
غمزه جادوش کرد جان مرا ساحری
\\
گر تو به عقلی بیا یک نظری کن در او
&&
تا تو بدانی که نیست کار بتم سرسری
\\
بر لب دریای عشق دیدم من ماهیی
&&
کرد یکی شیوه‌ای شیوه او برتری
\\
گر چه که ماهی نمود لیک خود او بحر بود
&&
صورت گوساله‌ای بود دو صد سامری
\\
ماهی ترک زبان کرد که گفته‌ست بحر
&&
نطق زبان را که تو حلقه برون دری
\\
دم زدن ماهیان آب بود نی هوا
&&
زانک هوا آتشیست نیست حریف تری
\\
بنگر در ماهیی نان وی و رزق او
&&
بحر بود پس تو در عشق از او کمتری
\\
دام فکندم که تا صید کنم ماهیی
&&
صید سلیمان وقت جان من انگشتری
\\
این چه بهانست خود زود بگو بحر کیست
&&
از حسد کس مترس در طلب مهتری
\\
روشن و مطلق بگو تا نشود از دلت
&&
مفخر تبریز ما شمس حق و دین بری
\\
\end{longtable}
\end{center}
