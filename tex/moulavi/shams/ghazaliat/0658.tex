\begin{center}
\section*{غزل شماره ۶۵۸: بگو دل را که گرد غم نگردد}
\label{sec:0658}
\addcontentsline{toc}{section}{\nameref{sec:0658}}
\begin{longtable}{l p{0.5cm} r}
بگو دل را که گرد غم نگردد
&&
ازیرا غم به خوردن کم نگردد
\\
نبات آب و گل جمله غم آمد
&&
که سور او به جز ماتم نگردد
\\
مگرد ای مرغ دل پیرامن غم
&&
که در غم پر و پا محکم نگردد
\\
دل اندر بی‌غمی پری بیابد
&&
که دیگر گرد این عالم نگردد
\\
دلا این تن عدو کهنه تست
&&
عدو کهنه خال و عم نگردد
\\
دلا سر سخت کن کم کن ملولی
&&
ملول اسرار را محرم نگردد
\\
چو ماهی باش در دریای معنی
&&
که جز با آب خوش همدم نگردد
\\
ملالی نیست ماهی را ز دریا
&&
که بی‌دریا خود او خرم نگردد
\\
یکی دریاست در عالم نهانی
&&
که در وی جز بنی آدم نگردد
\\
ز حیوان تا که مردم وانبرد
&&
درون آب حیوان هم نگردد
\\
خموش از حرف زیرا مرد معنی
&&
بگرد حرف لا و لم نگردد
\\
\end{longtable}
\end{center}
