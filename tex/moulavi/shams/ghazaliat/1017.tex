\begin{center}
\section*{غزل شماره ۱۰۱۷: آمد ترش رویی دگر یا زمهریر است او مگر}
\label{sec:1017}
\addcontentsline{toc}{section}{\nameref{sec:1017}}
\begin{longtable}{l p{0.5cm} r}
آمد ترش رویی دگر یا زمهریر است او مگر
&&
برریز جامی بر سرش ای ساقی همچون شکر
\\
یا می دهش از بلبله یا خود به راهش کن هله
&&
زیرا میان گلرخان خوش نیست عفریت ای پسر
\\
درده می پیغامبری تا خر نماند در خری
&&
خر را بروید در زمان از باده عیسی دو پر
\\
در مجلس مستان دل هشیار اگر آید مهل
&&
دانی که مستان را بود در حال مستی خیر و شر
\\
ای پاسبان بر در نشین در مجلس ما ره مده
&&
جز عاشقی آتش دلی کآید از او بوی جگر
\\
گر دست خواهی پا دهد ور پای خواهی سر نهد
&&
ور بیل خواهی عاریت بر جای بیل آرد تبر
\\
تا در شراب آغشته‌ام بی‌شرم و بی‌دل گشته‌ام
&&
اسپر سلامت نیستم در پیش تیغم چون سپر
\\
خواهم یکی گوینده‌ای آب حیاتی زنده‌ای
&&
کآتش به خواب اندرزند وین پرده گوید تا سحر
\\
اندر تن من گر رگی هشیار یابی بردرش
&&
چون شیرگیر حق نشد او را در این ره سگ شمر
\\
قومی خراب و مست و خوش قومی غلام پنج و شش
&&
آن‌ها جدا وین‌ها جدا آن‌ها دگر وین‌ها دگر
\\
ز اندازه بیرون خورده‌ام کاندازه را گم کرده‌ام
&&
شدوا یدی شدوا فمی هذا حفاظ ذی السکر
\\
هین نیش ما را نوش کن افغان ما را گوش کن
&&
ما را چو خود بی‌هوش کن بی‌هوش سوی ما نگر
\\
\end{longtable}
\end{center}
