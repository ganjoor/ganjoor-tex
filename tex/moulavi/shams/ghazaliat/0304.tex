\begin{center}
\section*{غزل شماره ۳۰۴: هیچ می‌دانی چه می‌گوید رباب}
\label{sec:0304}
\addcontentsline{toc}{section}{\nameref{sec:0304}}
\begin{longtable}{l p{0.5cm} r}
هیچ می‌دانی چه می‌گوید رباب
&&
ز اشک چشم و از جگرهای کباب
\\
پوستی‌ام دور مانده من ز گوشت
&&
چون ننالم در فراق و در عذاب
\\
چوب هم گوید بدم من شاخ سبز
&&
زین من بشکست و بدرید آن رکاب
\\
ما غریبان فراقیم ای شهان
&&
بشنوید از ما الی الله المأب
\\
هم ز حق رستیم اول در جهان
&&
هم بدو وا می‌رویم از انقلاب
\\
بانگ ما همچون جرس در کاروان
&&
یا چو رعدی وقت سیران سحاب
\\
ای مسافر دل منه بر منزلی
&&
که شوی خسته به گاه اجتذاب
\\
زانک از بسیار منزل رفته‌ای
&&
تو ز نطفه تا به هنگام شباب
\\
سهل گیرش تا به سهلی وارهی
&&
هم دهی آسان و هم یابی ثواب
\\
سخت او را گیر کو سختت گرفت
&&
اول او و آخر او او را بیاب
\\
خوش کمانچه می‌کشد کان تیر او
&&
در دل عشاق دارد اضطراب
\\
ترک و رومی و عرب گر عاشق است
&&
همزبان اوست این بانگ صواب
\\
باد می‌نالد همی‌خواند تو را
&&
که بیا اندر پیم تا جوی آب
\\
آب بودم باد گشتم آمدم
&&
تا رهانم تشنگان را زین سراب
\\
نطق آن بادست کآبی بوده است
&&
آب گردد چون بیندازد نقاب
\\
از برون شش جهت این بانگ خاست
&&
کز جهت بگریز و رو از ما متاب
\\
عاشقا کمتر ز پروانه نه‌ای
&&
کی کند پروانه ز آتش اجتناب
\\
شاه در شهرست بهر جغد من
&&
کی گذارم شهر و کی گیرم خراب
\\
گر خری دیوانه شد نک کیر گاو
&&
بر سرش چندان بزن کید لباب
\\
گر دلش جویم خسیش افزون شود
&&
کافران را گفت حق ضرب الرقاب
\\
\end{longtable}
\end{center}
