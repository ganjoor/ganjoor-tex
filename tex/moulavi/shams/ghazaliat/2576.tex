\begin{center}
\section*{غزل شماره ۲۵۷۶: ای خواجه سلام علیک از زحمت ما چونی}
\label{sec:2576}
\addcontentsline{toc}{section}{\nameref{sec:2576}}
\begin{longtable}{l p{0.5cm} r}
ای خواجه سلام علیک از زحمت ما چونی
&&
ای معدن زیبایی وی کان وفا چونی
\\
در جنت و در دوزخ پرسان تواند ای جان
&&
کای جنت روحانی وی بحر صفا چونی
\\
هر نور تو را گوید ای چشم و چراغ من
&&
هر رنج تو را گوید کی دفع بلا چونی
\\
ای خدمت تو کردن چون گلبشکر خوردن
&&
زین خدمت پوسیده زین طال بقا چونی
\\
در وقت جفا دل را صد تاج و کمر بخشی
&&
در وقت جفا اینی تا وقت وفا چونی
\\
ای موسی این دوران چونی تو ز فرعونان
&&
وی شاه ید بیضا با اهل عمی چونی
\\
گوید به تو هر گلشن هر نرگس و هر سوسن
&&
کز زحمت و رنج ما ای باد صبا چونی
\\
ای آب خضر چونی از گردش چرخ آخر
&&
وی تاج همه جان‌ها دربند قبا چونی
\\
ای جان عنادیده خامش که عنایت‌ها
&&
پرسند تو را هر دم کز رنج و عنا چونی
\\
\end{longtable}
\end{center}
