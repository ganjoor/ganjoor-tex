\begin{center}
\section*{غزل شماره ۲۳۵۲: یک جام ز صد هزار جان به}
\label{sec:2352}
\addcontentsline{toc}{section}{\nameref{sec:2352}}
\begin{longtable}{l p{0.5cm} r}
یک جام ز صد هزار جان به
&&
برخیز و قماش ما گرو نه
\\
ما از خود خویش توبه کردیم
&&
ما هیچ نمی‌رویم از این ده
\\
یک رنگ کند شراب ما را
&&
تا هر دو یکی شود که و مه
\\
درویش ز خویشتن تهی شد
&&
پر ده تو شراب فقر پر ده
\\
برخیز و به زه کن آن کمان را
&&
ماییم کمان و باده چون زه
\\
برجای بماند عقل پرفعل
&&
این است سزای پیر فربه
\\
ما غم نخوریم خود کی دیده‌ست
&&
تو بار کشی و او کند عه
\\
بگریز ز غم به سوی شه رو
&&
وز خانه عاریت برون جه
\\
\end{longtable}
\end{center}
