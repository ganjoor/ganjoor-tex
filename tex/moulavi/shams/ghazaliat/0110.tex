\begin{center}
\section*{غزل شماره ۱۱۰: تو بشکن چنگ ما را ای معلا}
\label{sec:0110}
\addcontentsline{toc}{section}{\nameref{sec:0110}}
\begin{longtable}{l p{0.5cm} r}
تو بشکن چنگ ما را ای معلا
&&
هزاران چنگ دیگر هست این جا
\\
چو ما در چنگ عشق اندرفتادیم
&&
چه کم آید بر ما چنگ و سرنا
\\
رباب و چنگ عالم گر بسوزد
&&
بسی چنگی پنهانیست یارا
\\
ترنگ و تنتنش رفته به گردون
&&
اگر چه ناید آن در گوش صما
\\
چراغ و شمع عالم گر بمیرد
&&
چو غم چون سنگ و آهن هست برجا
\\
به روی بحر خاشاکست اغانی
&&
نیاید گوهری بر روی دریا
\\
ولیکن لطف خاشاک از گهر دان
&&
که عکس عکس برق اوست بر ما
\\
اغانی جمله فرع شوق وصلیست
&&
برابر نیست فرع و اصل اصلا
\\
دهان بربند و بگشا روزن دل
&&
از آن ره باش با ارواح گویا
\\
\end{longtable}
\end{center}
