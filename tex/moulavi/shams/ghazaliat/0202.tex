\begin{center}
\section*{غزل شماره ۲۰۲: هر روز بامداد سلام علیکما}
\label{sec:0202}
\addcontentsline{toc}{section}{\nameref{sec:0202}}
\begin{longtable}{l p{0.5cm} r}
هر روز بامداد سلام علیکما
&&
آن جا که شه نشیند و آن وقت مرتضا
\\
دل ایستاد پیشش بسته دو دست خویش
&&
تا دست شاه بخشد پایان زر و عطا
\\
جان مست کاس و تا ابدالدهر گه گهی
&&
بر خوان جسم کاسه نهد دل نصیب ما
\\
تا زان نصیب بخشد دست مسیح عشق
&&
مر مرده را سعادت و بیمار را دوا
\\
برگ تمام یابد از او باغ عشرتی
&&
هم بانوا شود ز طرب چنگل دوتا
\\
در رقص گشته تن ز نواهای تن به تن
&&
جان خود خراب و مست در آن محو و آن فنا
\\
زندان شده بهشت ز نای و ز نوش عشق
&&
قاضی عقل مست در آن مسند قضا
\\
سوی مدرس خرد آیند در سؤال
&&
کاین فتنه عظیم در اسلام شد چرا
\\
مفتی عقل کل به فتوی دهد جواب
&&
کاین دم قیامت‌ست روا کو و ناروا
\\
در عیدگاه وصل برآمد خطیب عشق
&&
با ذوالفقار و گفت مر آن شاه را ثنا
\\
از بحر لامکان همه جان‌های گوهری
&&
کرده نثار گوهر و مرجان جان‌ها
\\
خاصان خاص و پردگیان سرای عشق
&&
صف صف نشسته در هوسش بر در سرا
\\
چون از شکاف پرده بر ایشان نظر کند
&&
بس نعره‌های عشق برآید که مرحبا
\\
می‌خواست سینه‌اش که سنایی دهد به چرخ
&&
سینای سینه‌اش بنگنجید در سما
\\
هر چار عنصرند در این جوش همچو دیک
&&
نی نار برقرار و نه خاک و نم هوا
\\
گه خاک در لباس گیا رفت از هوس
&&
گه آب خود هوا شد از بهر این ولا
\\
از راه روغناس شده آب آتشی
&&
آتش شده ز عشق هوا هم در این فضا
\\
ارکان به خانه خانه بگشته چو بیذقی
&&
از بهر عشق شاه نه از لهو چون شما
\\
ای بی‌خبر برو که تو را آب روشنی‌ست
&&
تا وارهد ز آب و گلت صفوت صفا
\\
زیرا که طالب صفت صفوت‌ست آب
&&
وان نیست جز وصال تو با قلزم ضیا
\\
ز آدم اگر بگردی او بی‌خدای نیست
&&
ابلیس وار سنگ خوری از کف خدا
\\
آری خدای نیست ولیکن خدای را
&&
این سنتی‌ست رفته در اسرار کبریا
\\
چون پیش آدم از دل و جان و بدن کنی
&&
یک سجده‌ای به امر حق از صدق بی‌ریا
\\
هر سو که تو بگردی از قبله بعد از آن
&&
کعبه بگردد آن سو بهر دل تو را
\\
مجموع چون نباشم در راه پس ز من
&&
مجموع چون شوند رفیقان باوفا
\\
دیوارهای خانه چو مجموع شد به نظم
&&
آن گاه اهل خانه در او جمع شد دلا
\\
چون کیسه جمع نبود باشد دریده درز
&&
پس سیم جمع چون شود از وی یکی بیا
\\
مجموع چون شوم چو به تبریز شد مقیم
&&
شمس الحقی که او شد سرجمع هر علا
\\
\end{longtable}
\end{center}
