\begin{center}
\section*{غزل شماره ۱۶۳۲: هله رفتیم و گرانی ز جمالت بردیم}
\label{sec:1632}
\addcontentsline{toc}{section}{\nameref{sec:1632}}
\begin{longtable}{l p{0.5cm} r}
هله رفتیم و گرانی ز جمالت بردیم
&&
جهت توشه ره ذکر وصالت بردیم
\\
تا که ما را و تو را تذکره‌ای باشد یاد
&&
دل خسته به تو دادیم و خیالت بردیم
\\
آن خیال رخ خوبت که قمر بنده اوست
&&
وان خم ابروی مانند هلالت بردیم
\\
وان شکرخنده خوبت که شکر تشنه اوست
&&
ز شکرخانه مجموع خصالت بردیم
\\
چون کبوتر چو بپریم به تو بازآییم
&&
زانک ما این پر و بال از پر و بالت بردیم
\\
هر کجا پرد فرعی به سوی اصل آید
&&
هر چه داریم همه از عز و جلالت بردیم
\\
شمس تبریز شنو خدمت ما را ز صبا
&&
گر شمال است و صبا هم ز شمالت بردیم
\\
\end{longtable}
\end{center}
