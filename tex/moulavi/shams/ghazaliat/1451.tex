\begin{center}
\section*{غزل شماره ۱۴۵۱: گر تو بنمی خسپی بنشین تو که من خفتم}
\label{sec:1451}
\addcontentsline{toc}{section}{\nameref{sec:1451}}
\begin{longtable}{l p{0.5cm} r}
گر تو بنمی خسپی بنشین تو که من خفتم
&&
تو قصه خود می گو من قصه خود گفتم
\\
بس کردم از دستان زیرا مثل مستان
&&
از خواب به هر سویی می جنبم و می افتم
\\
من تشنه آن یارم گر خفته و بیدارم
&&
با نقش خیال او همراهم و هم جفتم
\\
چون صورت آیینه من تابع آن رویم
&&
زان رو صفت او را بنمودم و بنهفتم
\\
آن دم که بخندید او من نیز بخندیدم
&&
وان دم که برآشفت او من نیز برآشفتم
\\
باقیش بگو تو هم زیرا که ز بحر توست
&&
درهای معانی که در رشته دم سفتم
\\
\end{longtable}
\end{center}
