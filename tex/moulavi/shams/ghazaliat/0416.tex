\begin{center}
\section*{غزل شماره ۴۱۶: مطرب و نوحه گر عاشق و شوریده خوش است}
\label{sec:0416}
\addcontentsline{toc}{section}{\nameref{sec:0416}}
\begin{longtable}{l p{0.5cm} r}
مطرب و نوحه گر عاشق و شوریده خوش است
&&
نبود بسته بود رسته و روییده خوش است
\\
تف و بوی جگر سوخته و جوشش خون
&&
گرد زیر و بم مطرب به چه پیچیده خوش است
\\
ز ابر پرآب دو چشمش ز تصاریف فراق
&&
بر شکوفه رخ پژمرده بباریده خوش است
\\
بنگر جان و جهان ور نتوانی دیدن
&&
این جهان در هوسش درهم و شوریده خوش است
\\
پیش دلبر بنهادن سر سرمست سزا است
&&
سر او را کف معشوق بمالیده خوش است
\\
دیدن روی دلارام عیان سلطانی است
&&
هم خیال صنم نادره در دیده خوش است
\\
این سعادت ندهد دست همیشه اما
&&
دیدن آن مه جان ناگه و دزدیده خوش است
\\
عشق اگر رخت تو را برد به غارت خوش باش
&&
پیش آن یوسف زیبا کف ببریده خوش است
\\
بس کن ار چه که اراجیف بشیر وصل است
&&
وصل همچون شکر ناگه بشنیده خوش است
\\
\end{longtable}
\end{center}
