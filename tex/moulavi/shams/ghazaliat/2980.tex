\begin{center}
\section*{غزل شماره ۲۹۸۰: آن لحظه کفتاب و چراغ جهان شوی}
\label{sec:2980}
\addcontentsline{toc}{section}{\nameref{sec:2980}}
\begin{longtable}{l p{0.5cm} r}
آن لحظه کآفتاب و چراغ جهان شوی
&&
اندر جهان مرده درآیی و جان شوی
\\
اندر دو چشم کور درآیی نظر دهی
&&
و اندر دهان گنگ درآیی زبان شوی
\\
در دیو زشت درروی و یوسفش کنی
&&
و اندر نهاد گرگ درآیی شبان شوی
\\
هر روز سر برآری از چارطاق نو
&&
چون رو بدان کنند از آن جا نهان شوی
\\
گاهی چو بوی گل مدد مغزها شوی
&&
گاهی انیس دیده شوی گلستان شوی
\\
فرزین کژروی و رخ راست رو شها
&&
در لعب کس نداند تا خود چه سان شوی
\\
رو رو ورق بگردان ای عشق بی‌نشان
&&
بر یک ورق قرار نمایی نشان شوی
\\
در عدل دوست محو شو ای دل به وقت غم
&&
هم محو لطف او شو چون شادمان شوی
\\
آبی که محو کل شد او نیز کل شود
&&
هم تو صفات پاک شوی گر چنان شوی
\\
آن بانگ چنگ را چو هوا هر طرف بری
&&
و آن سوز قهر را تو گوا چون دخان شوی
\\
ای عشق این همه بشوی و تو پاک از این
&&
بی صورتی چو خشم اگر چه سنان شوی
\\
این دم خموش کرده‌ای و من خمش کنم
&&
آنگه بیان کنم که تو نطق و بیان شوی
\\
\end{longtable}
\end{center}
