\begin{center}
\section*{غزل شماره ۲۰: چندانک خواهی جنگ کن یا گرم کن تهدید را}
\label{sec:0020}
\addcontentsline{toc}{section}{\nameref{sec:0020}}
\begin{longtable}{l p{0.5cm} r}
چندانک خواهی جنگ کن یا گرم کن تهدید را
&&
می‌دان که دود گولخن هرگز نیاید بر سما
\\
ور خود برآید بر سما کی تیره گردد آسمان
&&
کز دود آورد آسمان چندان لطیفی و ضیا
\\
خود را مرنجان ای پدر سر را مکوب اندر حجر
&&
با نقش گرمابه مکن این جمله چالیش و غزا
\\
گر تو کنی بر مه تفو بر روی تو بازآید آن
&&
ور دامن او را کشی هم بر تو تنگ آید قبا
\\
پیش از تو خامان دگر در جوش این دیگ جهان
&&
بس برطپیدند و نشد درمان نبود الا رضا
\\
بگرفت دم مار را یک خارپشت اندر دهن
&&
سر درکشید و گرد شد مانند گویی آن دغا
\\
آن مار ابله خویش را بر خار می‌زد دم به دم
&&
سوراخ سوراخ آمد او از خود زدن بر خارها
\\
بی صبر بود و بی‌حیل خود را بکشت او از عجل
&&
گر صبر کردی یک زمان رستی از او آن بدلقا
\\
بر خارپشت هر بلا خود را مزن تو هم هلا
&&
ساکن نشین وین ورد خوان جاء القضا ضاق الفضا
\\
فرمود رب العالمین با صابرانم همنشین
&&
ای همنشین صابران افرغ علینا صبرنا
\\
رفتم به وادی دگر باقی تو فرما ای پدر
&&
مر صابران را می‌رسان هر دم سلامی نو ز ما
\\
\end{longtable}
\end{center}
