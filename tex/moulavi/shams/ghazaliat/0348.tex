\begin{center}
\section*{غزل شماره ۳۴۸: صدایی کز کمان آید نذیریست}
\label{sec:0348}
\addcontentsline{toc}{section}{\nameref{sec:0348}}
\begin{longtable}{l p{0.5cm} r}
صدایی کز کمان آید نذیریست
&&
که اغلب با صدایش زخم تیریست
\\
مؤثر را نگر در آب آثار
&&
کاثر جستن عصای هر ضریریست
\\
پس لا تبصرونت تبصرونی‌ست
&&
بصر جستن ز الهام بصیریست
\\
تو هر چه داری نه جویانش بودی
&&
طلب‌ها گوش گیری و بشیریست
\\
چنان کن که طلب‌ها بیش گردد
&&
کثیرالزرع را طمع وفیریست
\\
مشو نومید از ظلمی که کردی
&&
که دریای کرم توبه پذیریست
\\
گناهت را کند تسبیح و طاعات
&&
که در توبه پذیری بی‌نظیریست
\\
شکسته باش و خاکی باش این جا
&&
که می‌جوید کرم هر جا فقیریست
\\
کرم دامن پر از زر کرد و آورد
&&
که تا وا می‌خرد هر جا اسیریست
\\
عزیزی بخشد آن کس را که خواری‌ست
&&
بزرگی بخشد آن را که حقیریست
\\
که هستی نیستی جوید همیشه
&&
زکات آن جا نیاید که امیریست
\\
ازیرا مظهر چیزیست ضدش
&&
از این دو ضد را ضد خود ظهیریست
\\
تو بر تخته سیاهی گر نویسی
&&
نهان گردد که هر دو همچو قیریست
\\
بود فرقی ز تری تا ترست خط
&&
چو گردد خشک پنهان چون ضمیریست
\\
خمش کن گر چه شرحش بی‌شمارست
&&
طبیعت‌ها عدو هر کثیریست
\\
\end{longtable}
\end{center}
