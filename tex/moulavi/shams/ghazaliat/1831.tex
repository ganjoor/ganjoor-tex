\begin{center}
\section*{غزل شماره ۱۸۳۱: راز تو فاش می کنم صبر نماند بیش از این}
\label{sec:1831}
\addcontentsline{toc}{section}{\nameref{sec:1831}}
\begin{longtable}{l p{0.5cm} r}
راز تو فاش می کنم صبر نماند بیش از این
&&
بیش فلک نمی‌کشد درد مرا و نی زمین
\\
این دل من چه پرغم است وان دل تو چه فارغ است
&&
آن رخ تو چو خوب چین وین رخ من پر است چین
\\
تا که بسوزد این جهان چند بسوزد این دلم
&&
چند بود بتا چنان چند گهی بود چنین
\\
سر هزارساله را مستم و فاش می کنم
&&
خواه ببند دیده را خواه گشا و خوش ببین
\\
شور مرا چو دید مه آمد سوی من ز ره
&&
گفت مده ز من نشان یار توایم و همنشین
\\
خیره بماند جان من در رخ او دمی و گفت
&&
ای صنم خوش خوشین ای بت آب و آتشین
\\
ای رخ جان فزای او بهر خدا همان همان
&&
مطرب دلربای من بهر خدا همین همین
\\
عشق تو را چو مفرشم آب بزن بر آتشم
&&
ای مه غیب آن جهان در تبریز شمس دین
\\
\end{longtable}
\end{center}
