\begin{center}
\section*{غزل شماره ۲۵۹۶: نظاره چه می‌آیی در حلقه بیداری}
\label{sec:2596}
\addcontentsline{toc}{section}{\nameref{sec:2596}}
\begin{longtable}{l p{0.5cm} r}
نظاره چه می‌آیی در حلقه بیداری
&&
گر سینه نپوشانی تیری بخوری کاری
\\
در حلقه سر اندرکن دل را تو قویتر کن
&&
شاهی است تو باور کن بر کرسی جباری
\\
تا بازرهی زان دم تا مست شوی هر دم
&&
گاهی ز لب لعلش گاهی ز می ناری
\\
بگشای دهانت را خاشاک مجو در می
&&
خاشاک کجا باشد در ساغر هشیاری
\\
ای خواجه چرا جویی دلداری از آن جانان
&&
بس نیست رخ خوبش دلجویی و دلداری
\\
دی نامه او خواندم در قصه بی‌خویشی
&&
بنوشتم از عالم صد نامه بیزاری
\\
نقش تو چو نقش من رخ بر رخ خود کرده‌ست
&&
با ما غم دل گویی یا قصه جان آری
\\
من با صنم معنی تن جامه برون کردم
&&
چون عشق بزد آتش در پرده ستاری
\\
در رنگ رخم عشقش چون عکس جمالش دید
&&
افتاد به پایم عشق در عذر گنه کاری
\\
شمس الحق تبریزی آیی و نبینندت
&&
زیرا که چو جان آیی بی‌رنگ صباواری
\\
\end{longtable}
\end{center}
