\begin{center}
\section*{غزل شماره ۹۷۳: صوفیان در دمی دو عید کنند}
\label{sec:0973}
\addcontentsline{toc}{section}{\nameref{sec:0973}}
\begin{longtable}{l p{0.5cm} r}
صوفیان در دمی دو عید کنند
&&
عنکبوتان مگس قدید کنند
\\
شمع‌ها می‌زنند خورشیدند
&&
تا که ظلمات را شهید کنند
\\
باز هر ذره شد چو نفخه صور
&&
تا شهید تو را سعید کنند
\\
چرخ کهنه به گردشان گردد
&&
تا کهنه‌هاش را جدید کنند
\\
رغم آن حاسدان که می‌خواهند
&&
تا قریب تو را بعید کنند
\\
حاسدان را هم از حسد بخرند
&&
همه را طالب و مرید کنند
\\
کیمیای سعادت همه‌اند
&&
در همه فعل خود بدید کنند
\\
کیمیایی کنند همه افلاک
&&
لیک در مدتی مدید کنند
\\
وان هم از ماه غیب دزدیدند
&&
که گهی پاک و گه پلید کنند
\\
خنک آن دم که جمله اجزا را
&&
بی ز ترکیب‌ها وحید کنند
\\
بس کن این و سر تنور ببند
&&
تا که نان‌هات را ثرید کنند
\\
\end{longtable}
\end{center}
