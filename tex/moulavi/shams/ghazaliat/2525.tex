\begin{center}
\section*{غزل شماره ۲۵۲۵: اگر گل‌های رخسارش از آن گلشن بخندیدی}
\label{sec:2525}
\addcontentsline{toc}{section}{\nameref{sec:2525}}
\begin{longtable}{l p{0.5cm} r}
اگر گل‌های رخسارش از آن گلشن بخندیدی
&&
بهار جان شدی تازه نهال تن بخندیدی
\\
وگر آن جان جان جان به تن‌ها روی بنمودی
&&
تنم از لطف جان گشتی و جان من بخندیدی
\\
ور آن نور دو صد فردوس گفتی هی قنق گلدم
&&
شدی این خانه فردوسی چو گل مسکن بخندیدی
\\
وگر آن ناطق کلی زبان نطق بگشادی
&&
تن مرده شدی گویا دل الکن بخندیدی
\\
گر آن معشوق معشوقان بدیدستی به مکر و فن
&&
روان‌ها ذوفنون گشتی و هر یک فن بخندیدی
\\
دریدی پرده‌ها از عشق و آشوبی درافتادی
&&
شدندی فاش مستوران گر او معلن بخندیدی
\\
گر آن سلطان خوبی از گریبان سر برآوردی
&&
همه دراعه‌های حسن تا دامن بخندیدی
\\
ور آن ماه دو صد گردون به ناگه خرمنی کردی
&&
طرب چون خوشه‌ها کردی و چون خرمن بخندیدی
\\
ور او یک لطف بنمودی گشادی چشم جان‌ها را
&&
خشونت‌ها گرفتی لطف و هر اخشن بخندیدی
\\
شهنشاه شهنشاهان و قانان چون عطا دادی
&&
به مسکینی شدی او گنج و بر مخزن بخندیدی
\\
از آن می‌های لعل او ز پرده غیب رو دادی
&&
حسن مستک شدی بی‌می و بر احسن بخندیدی
\\
ور آن لعل لبان او گهرها دادی از حکمت
&&
شدی مرمر مثال لعل و بر معدن بخندیدی
\\
ور آن قهار عاشق کش به مهر آمیزشی کردی
&&
که خارا بدادی شیر و تا آهن بخندیدی
\\
وگر زالی از آن رستم بیابیدی نظر یک دم
&&
به حق بر رستم دستان صف اشکن بخندیدی
\\
در آن روزی که آن شیر وغا مردی کند پیدا
&&
نه بر شیران مست آن روز مرد و زن بخندیدی
\\
پیاپی ساقی دولت روان کردی می خلت
&&
که تا ساغر شدی سرمست وز می دن بخندیدی
\\
هر آن جانی که دست شمس تبریزی ببوسیدی
&&
حیاتش جاودان گشتی و بر مردن بخندیدی
\\
بدیدی زود امن او ز مردی جنگ می‌جستی
&&
کراهت داشتی بر امن و بر مؤمن بخندیدی
\\
\end{longtable}
\end{center}
