\begin{center}
\section*{غزل شماره ۲۴۳۳: ای آنک اندر باغ جان آلاجقی برساختی}
\label{sec:2433}
\addcontentsline{toc}{section}{\nameref{sec:2433}}
\begin{longtable}{l p{0.5cm} r}
ای آنک اندر باغ جان آلاجقی برساختی
&&
آتش زدی در جسم و جان روح مصور ساختی
\\
پای درختان بسته بد تو برگشادی پایشان
&&
صحن گلستان خاک بد فرشش ز گوهر ساختی
\\
مرغ معماگوی را رسم سخن آموختی
&&
باز دل پژمرده را صد بال و صد پر ساختی
\\
ای عمر بی‌مرگی ز تو وی برگ بی‌برگی ز تو
&&
الحق خدنگ مرگ را پاینده اسپر ساختی
\\
عاشق در این ره چون قلم کژمژ همی‌رفتش قدم
&&
بر دفتر جان بهر او پاکیزه مسطر ساختی
\\
حیوان و گاوی را اگر مردم کنی نبود عجب
&&
سرگین گاوی را چو تو در بحر عنبر ساختی
\\
آن کو جهان گیری کند چون آفتاب از بهر تو
&&
او را هم از اجزای او صد تیغ و لشکر ساختی
\\
در پیش آدم گر ملک سجده کند نبود عجب
&&
کز بهر خاکی چرخ را سقا و چاکر ساختی
\\
از اختران در سنگ و گل تأثیرها درریختی
&&
وز راه دل تا آسمان معراج معبر ساختی
\\
در خاک تیره خارشی انداختی از بهر زه
&&
یک خاک را کردی پدر یک خاک مادر ساختی
\\
از گور در جنت اگر درها گشایی قادری
&&
در گور تن از پنج حس بشکافتی در ساختی
\\
در آتش خشم پدر صد آب رحمت می‌نهی
&&
و اندر دل آب منی صد گونه آذر ساختی
\\
از بلغم و صفرای ما وز خون و از سودای ما
&&
زین چار خرقه روح را ای شاه چادر ساختی
\\
روزی بیاید کاین سخن خصمی کند با مستمع
&&
کب حیاتم خواندمت تو خویشتن کر ساختی
\\
ای شمس تبریزی بگو شرح معانی مو به مو
&&
دستش بده پایش بده چون صورت سر ساختی
\\
\end{longtable}
\end{center}
