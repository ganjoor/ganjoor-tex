\begin{center}
\section*{غزل شماره ۲۸۸۹: ای که تو چشمه حیوان و بهار چمنی}
\label{sec:2889}
\addcontentsline{toc}{section}{\nameref{sec:2889}}
\begin{longtable}{l p{0.5cm} r}
ای که تو چشمه حیوان و بهار چمنی
&&
چو منی تو خود خود را کی بگوید چو منی
\\
من شبم تو مه بدری مگریز از شب خویش
&&
مه کی باشد که تو خورشید دو صد انجمی
\\
پاسبان در تو ماه برین بام فلک
&&
تو که در مقعد صدقی چو شه اندر وطنی
\\
ماه پیمانه عمر است گهی پر گه نیم
&&
تو به پیمانه نگنجی تو نه عمر زمنی
\\
هر کی در عهد تو از جور زمانه گله کرد
&&
سزد ار کفش جفا بر دهن او بزنی
\\
کاین زمانه چو تن است و تو در او چون جانی
&&
جان بود تن نبود تن چو تو جان جان تنی
\\
سجده کردند ملایک تن آدم را زود
&&
پرتو جان تو دیدند در آن جسم سنی
\\
اهرمن صورت گل دید و سرش سجده نکرد
&&
چوب رد بر سرش آمد که برو اهرمنی
\\
\end{longtable}
\end{center}
