\begin{center}
\section*{غزل شماره ۲۲۰۵: دوش خوابی دیده‌ام خود عاشقان را خواب کو}
\label{sec:2205}
\addcontentsline{toc}{section}{\nameref{sec:2205}}
\begin{longtable}{l p{0.5cm} r}
دوش خوابی دیده‌ام خود عاشقان را خواب کو
&&
کاندرون کعبه می‌جستم که آن محراب کو
\\
کعبه جان‌ها نه آن کعبه که چون آن جا رسی
&&
در شب تاریک گویی شمع یا مهتاب کو
\\
بلک بنیادش ز نوری کز شعاع جان تو
&&
نور گیرد جمله عالم لیک جان را تاب کو
\\
خانقاهش جمله از نور است فرشش علم و عقل
&&
صوفیانش بی‌سر و پا غلبه قبقاب کو
\\
تاج و تختی کاندرون داری نهان ای نیکبخت
&&
در گمان کیقباد و سنجر و سهراب کو
\\
در میان باغ حسنش می‌پر ای مرغ ضمیر
&&
کایمن آباد است آن جا دام یا مضراب کو
\\
در درون عاریت‌های تن تو بخششی است
&&
در میان جان طلب کان بخشش وهاب کو
\\
در صفت کردن ز دور اطناب شد گفت زمان
&&
چون رسیدم در طناب خود کنون اطناب کو
\\
چون برون رفتی ز گل زود آمدی در باغ دل
&&
پس از آن سو جز سماع و جز شراب ناب کو
\\
چون ز شورستان تن رفتی سوی بستان جان
&&
جز گل و ریحان و لاله و چشمه‌های آب کو
\\
چون هزاران حسن دیدی کان نبد از کالبد
&&
پس چرا گویی جمال فاتح الابواب کو
\\
ای فقیه از بهر لله علم عشق آموز تو
&&
ز آنک بعد از مرگ حل و حرمت و ایجاب کو
\\
چون به وقت رنج و محنت زود می‌یابی درش
&&
بازگویی او کجا درگاه او را باب کو
\\
باش تا موج وصالش دررباید مر تو را
&&
غیب گردی پس بگویی عالم اسباب کو
\\
ار چه خط این بوابت هوس شد در رقاع
&&
رقعه عشقش بخوان بنمایدت بواب کو
\\
هر کسی را نایب حق تا نگویی زینهار
&&
در بساط قاضی آ آنگه ببین نواب کو
\\
تا نمالی گوش خود را خلق بینی کار و بار
&&
چون بمالی چشم خود را گویی آن را تاب کو
\\
در خرابات حقیقت پیش مستان خراب
&&
در چنان صافی نبینی درد و خس و انساب کو
\\
در حساب فانیی عمرت تلف شد بی‌حساب
&&
در صفای یار بنگر شبهت حساب کو
\\
چون میت پردل کند در بحر دل غوطی خوری
&&
این ترانه می‌زنی کاین بحر را پایاب کو
\\
\end{longtable}
\end{center}
