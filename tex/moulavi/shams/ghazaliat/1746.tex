\begin{center}
\section*{غزل شماره ۱۷۴۶: بر آن شده‌ست دلم کتشی بگیرانم}
\label{sec:1746}
\addcontentsline{toc}{section}{\nameref{sec:1746}}
\begin{longtable}{l p{0.5cm} r}
بر آن شده‌ست دلم کآتشی بگیرانم
&&
که هر کی او نمرد پیش تو بمیرانم
\\
کمان عشق بدرم که تا بداند عقل
&&
که بی‌نظیرم و سلطان بی‌نظیرانم
\\
که رفت در نظر تو که بی‌نظیر نشد
&&
مقام گنج شده‌ست این نهاد ویرانم
\\
من از کجا و مباهات سلطنت ز کجا
&&
فقیر فقرم و افتاده فقیرانم
\\
من آن کسم که تو نامم نهی نمی‌دانم
&&
چو من اسیر توام پس امیر میرانم
\\
جز از اسیری و میری مقام دیگر هست
&&
چو من فنا شوم از هر دو کس نفیرانم
\\
چو شب بیاید میر و اسیر محو شوند
&&
اسیر هیچ نداند که از اسیرانم
\\
به خواب شب گرو آمد امیری میران
&&
چو عشق هیچ نخسبد ز عشق گیرانم
\\
به آفتاب نگر پادشاه یک روزه‌ست
&&
همی‌گدازد مه منیر کز وزیرانم
\\
منم که پخته عشقم نه خام و خام طمع
&&
خدای کرد خمیری از آن خمیرانم
\\
خمیرکرده یزدان کجا بماند خام
&&
خمیرمایه پذیرم نه از فطیرانم
\\
فطیر چون کند او فاطرالسموات است
&&
چو اختران سماوات از منیرانم
\\
تو چند نام نهی خویش را خمش می باش
&&
که کودکی است که گویی که من ز پیرانم
\\
\end{longtable}
\end{center}
