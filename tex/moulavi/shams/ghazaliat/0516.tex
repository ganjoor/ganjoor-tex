\begin{center}
\section*{غزل شماره ۵۱۶: بازرسیدیم ز میخانه مست}
\label{sec:0516}
\addcontentsline{toc}{section}{\nameref{sec:0516}}
\begin{longtable}{l p{0.5cm} r}
بازرسیدیم ز میخانه مست
&&
بازرهیدیم ز بالا و پست
\\
جمله مستان خوش و رقصان شدند
&&
دست زنید ای صنمان دست دست
\\
ماهی و دریا همه مستی کنند
&&
چونک سر زلف تو افتاده شست
\\
زیر و زبر گشت خرابات ما
&&
خنب نگون گشت و قرابه شکست
\\
پیر خرابات چو آن شور دید
&&
بر سر بام آمد و از بام جست
\\
جوش برآورد یکی می کز او
&&
هست شود نیست شود نیست هست
\\
شیشه چو بشکست و به هر سوی ریخت
&&
چند کف پای حریفان که خست
\\
آن که سر از پای نداند کجاست
&&
مست فتادست به کوی الست
\\
باده پرستان همه در عشرتند
&&
تنتن تنتن شنو ای تن پرست
\\
\end{longtable}
\end{center}
