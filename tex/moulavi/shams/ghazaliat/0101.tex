\begin{center}
\section*{غزل شماره ۱۰۱: بسوزانیم سودا و جنون را}
\label{sec:0101}
\addcontentsline{toc}{section}{\nameref{sec:0101}}
\begin{longtable}{l p{0.5cm} r}
بسوزانیم سودا و جنون را
&&
درآشامیم هر دم موج خون را
\\
حریف دوزخ آشامان مستیم
&&
که بشکافند سقف سبزگون را
\\
چه خواهد کرد شمع لایزالی
&&
فلک را وین دو شمع سرنگون را
\\
فروبریم دست دزد غم را
&&
که دزدیدست عقل صد زبون را
\\
شراب صرف سلطانی بریزیم
&&
بخوابانیم عقل ذوفنون را
\\
چو گردد مست حد بر وی برانیم
&&
که از حد برد تزویر و فسون را
\\
اگر چه زوبع و استاد جمله‌ست
&&
چه داند حیله ریب المنون را
\\
چنانش بیخود و سرمست سازیم
&&
که چون آید نداند راه چون را
\\
چنان پیر و چنان عالم فنا به
&&
که تا عبرت شود لایعلمون را
\\
کنون عالم شود کز عشق جان داد
&&
کنون واقف شود علم درون را
\\
درون خانه دل او ببیند
&&
ستون این جهان بی‌ستون را
\\
که سرگردان بدین سرهاست گر نه
&&
سکون بودی جهان بی‌سکون را
\\
تن باسر نداند سر کن را
&&
تن بی‌سر شناسد کاف و نون را
\\
یکی لحظه بنه سر ای برادر
&&
چه باشد از برای آزمون را
\\
یکی دم رام کن از بهر سلطان
&&
چنین سگ را چنین اسب حرون را
\\
تو دوزخ دان خودآگاهی عالم
&&
فنا شو کم طلب این سرفزون را
\\
چنان اندر صفات حق فرورو
&&
که برنایی نبینی این برون را
\\
چه جویی ذوق این آب سیه را
&&
چه بویی سبزه این بام تون را
\\
خمش کردم نیارم شرح کردن
&&
ز رشک و غیرت هر خام دون را
\\
نما ای شمس تبریزی کمالی
&&
که تا نقصی نباشد کاف و نون را
\\
\end{longtable}
\end{center}
