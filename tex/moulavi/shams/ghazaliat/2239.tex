\begin{center}
\section*{غزل شماره ۲۲۳۹: رفتم به کوی خواجه و گفتم که خواجه کو}
\label{sec:2239}
\addcontentsline{toc}{section}{\nameref{sec:2239}}
\begin{longtable}{l p{0.5cm} r}
رفتم به کوی خواجه و گفتم که خواجه کو
&&
گفتند خواجه عاشق و مست است و کو به کو
\\
گفتم فریضه دارم آخر نشان دهید
&&
من دوستدار خواجه‌ام آخر نیم عدو
\\
گفتند خواجه عاشق آن باغبان شده‌ست
&&
او را به باغ‌ها جو یا بر کنار جو
\\
مستان و عاشقان بر دلدار خود روند
&&
هر کس که گشت عاشق رو دست از او بشو
\\
ماهی که آب دید نپاید به خاکدان
&&
عاشق کجا بماند در دور رنگ و بو
\\
برف فسرده کو رخ آن آفتاب دید
&&
خورشید پاک خوردش اگر هست تو به تو
\\
خاصه کسی که عاشق سلطان ما بود
&&
سلطان بی‌نظیر وفادار قندخو
\\
آن کیمیای بی‌حد و بی‌عد و بی‌قیاس
&&
بر هر مسی که برزد زر شد به ارجعوا
\\
در خواب شو ز عالم وز شش جهت گریز
&&
تا چند گول گردی و آواره سو به سو
\\
ناچار می برندت باری به اختیار
&&
تا پیش شاه باشدت اعزاز و آبرو
\\
گر ز آنک در میانه نبودی سرخری
&&
اسرار کشف کردی عیسیت مو به مو
\\
بستم ره دهان و گشادم ره نهان
&&
رستم به یک قنینه ز سودای گفت و گو
\\
\end{longtable}
\end{center}
