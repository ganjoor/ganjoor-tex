\begin{center}
\section*{غزل شماره ۲۰۴۱: پروانه شد در آتش گفتا که همچنین کن}
\label{sec:2041}
\addcontentsline{toc}{section}{\nameref{sec:2041}}
\begin{longtable}{l p{0.5cm} r}
پروانه شد در آتش گفتا که همچنین کن
&&
می‌سوخت و پر همی‌زد بر جا که همچنین کن
\\
شمع و فتیله بسته با گردن شکسته
&&
می‌گفت نرم نرمک با ما که همچنین کن
\\
مومی که می‌گدازد با سوز می بسازد
&&
در تف و تاب داده خود را که همچنین کن
\\
گر سیم و زر فشانی در سود این جهانی
&&
سودت ندارد آن‌ها الا که همچنین کن
\\
دامان پر ز گوهر کرد و نشست بر سر
&&
وز رشک تلخ گشته دریا که همچنین کن
\\
از نیک و بد بریده وز دام‌ها پریده
&&
بر کوه قاف رفته عنقا که همچنین کن
\\
رخساره پاک کرده دراعه چاک کرده
&&
با خار صبر کرده گل‌ها که همچنین کن
\\
صد ننگ و نام هشته با عقل خصم گشته
&&
بر مغزها دویده صهبا که همچنین کن
\\
خالی شده‌ست و ساده نه چشم برگشاده
&&
لب بر لبش نهاده سرنا که همچنین کن
\\
چل سال چشم آدم در عذر داشت ماتم
&&
گفته به کودکانش بابا که همچنین کن
\\
خاموش باش و صابر عبرت بگیر آخر
&&
خامش شده‌ست و گریان خارا که همچنین کن
\\
تبریز شمس دین را بین کز ضیای جانی
&&
پر کرده از جلالت صحرا که همچنین کن
\\
\end{longtable}
\end{center}
