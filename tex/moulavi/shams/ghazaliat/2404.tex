\begin{center}
\section*{غزل شماره ۲۴۰۴: ای مه و ای آفتاب پیش رخت مسخره}
\label{sec:2404}
\addcontentsline{toc}{section}{\nameref{sec:2404}}
\begin{longtable}{l p{0.5cm} r}
ای مه و ای آفتاب پیش رخت مسخره
&&
تا چه زند زهره از آینه و جندره
\\
پیش تو افتاده ماه بر ره سودای عشق
&&
ریخته گلگونه‌اش یاوه شده قنجره
\\
پنجره‌ای شد سماع سوی گلستان تو
&&
گوش و دل عاشقان بر سر این پنجره
\\
آه که این پنجره هست حجابی عظیم
&&
رو که حجابی خوش است هیچ مگو ای سره
\\
از شکرینی که هست بهر بخاییدنش
&&
لب همه دندان شده‌ست بر مثل دستره
\\
دست دل خویش را دیدم در خمره‌ای
&&
گفتم خواجه حکیم چیست در این خنبره
\\
گفت شراب کسی کو همگی چرخ را
&&
با همه دولاب جان می نخرد یک تره
\\
کره گردون تند پیشش پالانیی
&&
بر سر میدان او جان خر باتوبره
\\
ای شه فارغ از آن باشد در لشکرت
&&
نصرت بر میمنه دولت بر میسره
\\
ای که ز تبریز تو عید جهان شمس دین
&&
هین که رسید آفتاب جانب برج بره
\\
\end{longtable}
\end{center}
