\begin{center}
\section*{غزل شماره ۶۱۵: خواب از پی آن آید تا عقل تو بستاند}
\label{sec:0615}
\addcontentsline{toc}{section}{\nameref{sec:0615}}
\begin{longtable}{l p{0.5cm} r}
خواب از پی آن آید تا عقل تو بستاند
&&
دیوانه کجا خسبد دیوانه چه شب داند
\\
نی روز بود نی شب در مذهب دیوانه
&&
آن چیز که او دارد او داند او داند
\\
از گردش گردون شد روز و شب این عالم
&&
دیوانه آن جا را گردون بنگر داند
\\
گر چشم سرش خسپد بی‌سر همه چشمست او
&&
کز دیده جان خود لوح ازلی خواند
\\
دیوانگی ار خواهی چون مرغ شو و ماهی
&&
با خواب چو همراهی آن با تو کجا ماند
\\
شب رو شو و عیاری در عشق چنان یاری
&&
تا باز شود کاری زان طره که بفشاند
\\
دیوانه دگر سانست او حامله جانست
&&
چشمش چو به جانانست حملش نه بدو ماند
\\
زین شرح اگر خواهی از شمس حق و شاهی
&&
تبریز همه عالم زو نور نو افشاند
\\
\end{longtable}
\end{center}
