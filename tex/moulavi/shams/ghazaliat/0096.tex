\begin{center}
\section*{غزل شماره ۹۶: لب را تو به هر بوسه و هر لوت میالا}
\label{sec:0096}
\addcontentsline{toc}{section}{\nameref{sec:0096}}
\begin{longtable}{l p{0.5cm} r}
لب را تو به هر بوسه و هر لوت میالا
&&
تا از لب دلدار شود مست و شکرخا
\\
تا از لب تو بوی لب غیر نیاید
&&
تا عشق مجرد شود و صافی و یکتا
\\
آن لب که بود کون خری بوسه گه او
&&
کی یابد آن لب شکربوس مسیحا
\\
می‌دانک حدث باشد جز نور قدیمی
&&
بر مزبله پرحدث آن گاه تماشا
\\
آنگه که فنا شد حدث اندر دل پالیز
&&
رست از حدثی و شود او چاشنی افزا
\\
تا تو حدثی لذت تقدیس چه دانی
&&
رو از حدثی سوی تبارک و تعالی
\\
زان دست مسیح آمد داروی جهانی
&&
کو دست نگه داشت ز هر کاسه سکبا
\\
از نعمت فرعون چه موسی کف و لب شست
&&
دریای کرم داد مر او را ید بیضا
\\
خواهی که ز معده و لب هر خام گریزی
&&
پرگوهر و روتلخ همی‌باش چو دریا
\\
هین چشم فروبند که آن چشم غیورست
&&
هین معده تهی دار که لوتیست مهیا
\\
سگ سیر شود هیچ شکاری بنگیرد
&&
کز آتش جوعست تک و گام تقاضا
\\
کو دست و لب پاک که گیرد قدح پاک
&&
کو صوفی چالاک که آید سوی حلوا
\\
بنمای از این حرف تصاویر حقایق
&&
یا من قسم القهوه و الکاس علینا
\\
\end{longtable}
\end{center}
