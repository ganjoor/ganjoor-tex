\begin{center}
\section*{غزل شماره ۲۲۱: مرا تو گوش گرفتی همی‌کشی به کجا}
\label{sec:0221}
\addcontentsline{toc}{section}{\nameref{sec:0221}}
\begin{longtable}{l p{0.5cm} r}
مرا تو گوش گرفتی همی‌کشی به کجا
&&
بگو که در دل تو چیست چیست عزم تو را
\\
چه دیگ پخته‌ای از بهر من عزیزا دوش
&&
خدای داند تا چیست عشق را سودا
\\
چو گوش چرخ و زمین و ستاره در کف توست
&&
کجا روند همان جا که گفته‌ای که بیا
\\
مرا دو گوش گرفتی و جمله را یک گوش
&&
که می‌زنم ز بن هر دو گوش طال بقا
\\
غلام پیر شود خواجه‌اش کند آزاد
&&
چو پیر گشتم از آغاز بنده کرد مرا
\\
نه کودکان به قیامت سپیدمو خیزند
&&
قیامت تو سیه موی کرد پیران را
\\
چو مرده زنده کنی پیر را جوان سازی
&&
خموش کردم و مشغول می‌شوم به دعا
\\
\end{longtable}
\end{center}
