\begin{center}
\section*{غزل شماره ۱۰۹۴: پر ده آن جام می را ساقیا بار دیگر}
\label{sec:1094}
\addcontentsline{toc}{section}{\nameref{sec:1094}}
\begin{longtable}{l p{0.5cm} r}
پر ده آن جام می را ساقیا بار دیگر
&&
نیست در دین و دنیا همچو تو یار دیگر
\\
کفر دان در طریقت جهل دان در حقیقت
&&
جز تماشای رویت پیشه و کار دیگر
\\
تا تو آن رخ نمودی عقل و ایمان ربودی
&&
هست منصور جان را هر طرف دار دیگر
\\
جان ز تو گشت شیدا دل ز تو گشت دریا
&&
کی کند التفاتی دل به دلدار دیگر
\\
جز به بغداد کویت یا خوش آباد رویت
&&
نیست هر دم فلک را جز که پیکار دیگر
\\
در خرابات مردان جام جانست گردان
&&
نیست مانند ایشان هیچ خمار دیگر
\\
همتی دار عالی کان شه لاابالی
&&
غیر انبار دنیا دارد انبار دیگر
\\
پاره‌ای چون برانی اندر این ره بدانی
&&
غیر این گلستان‌ها باغ و گلزار دیگر
\\
پا به مردی فشردی سر سلامت ببردی
&&
رفت دستار بستان شصت دستار دیگر
\\
دل مرا برد ناگه سوی آن شهره خرگه
&&
من گرفتار گشتم دل گرفتار دیگر
\\
روز چون عذر آری شب سر خواب خاری
&&
پای ما تا چه گردد هر دم از خار دیگر
\\
جز که در عشق صانع عمر هرزه‌ست و ضایع
&&
ژاژ دان در طریقت فعل و گفتار دیگر
\\
بخت اینست و دولت عیش اینست و عشرت
&&
کو جز این عشق و سودا سود و بازار دیگر
\\
گفتمش دل ببردی تا کجاها سپردی
&&
گفت نی من نبردم برد عیار دیگر
\\
گفتمش من نترسم من هم از دل بپرسم
&&
دل بگوید نماند شک و انکار دیگر
\\
راستی گوی ای جان عاشقان را مرنجان
&&
جز تو در دلربایان کو دل افشار دیگر
\\
چون کمالات فانی هستشان این امانی
&&
که به هر دم نمایند لطف و ایثار دیگر
\\
پس کمالات آن را کو نگارد جهان را
&&
چون تقاضا نباشد عشق و هنجار دیگر
\\
بحر از این روی جوشد مرغ از این رو خروشد
&&
تا در این دام افتد هر دم آشکار دیگر
\\
چون خدا این جهان را کرد چون گنج پیدا
&&
هر سری پر ز سودا دارد اظهار دیگر
\\
هر کجا خوش نگاری روز و شب بی‌قراری
&&
جوید او حسن خود را نوخریدار دیگر
\\
هر کجا ماه رویی هر کجا مشک بویی
&&
مشتری وار جوید عاشقی زار دیگر
\\
این نفس مست اویم روز دیگر بگویم
&&
هم بر این پرده تر با تو اسرار دیگر
\\
بس کن و طبل کم زن کاندر این باغ و گلشن
&&
هست پهلوی طبلت بیست نعار دیگر
\\
\end{longtable}
\end{center}
