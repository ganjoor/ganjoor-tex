\begin{center}
\section*{غزل شماره ۱۲۷۲: باز فرود آمدیم بر در سلطان خویش}
\label{sec:1272}
\addcontentsline{toc}{section}{\nameref{sec:1272}}
\begin{longtable}{l p{0.5cm} r}
باز فرود آمدیم بر در سلطان خویش
&&
بازگشادیم خوش بال و پر جان خویش
\\
باز سعادت رسید دامن ما را کشید
&&
بر سر گردون زدیم خیمه و ایوان خویش
\\
دیده دیو و پری دید ز ما سروری
&&
هدهد جان بازگشت سوی سلیمان خویش
\\
ساقی مستان ما شد شکرستان ما
&&
یوسف جان برگشاد جعد پریشان خویش
\\
دوش مرا گفت یار چونی از این روزگار
&&
چون بود آن کس که دید دولت خندان خویش
\\
آن شکری را که هیچ مصر ندیدش به خواب
&&
شکر که من یافتم در بن دندان خویش
\\
بی‌زر و سر سروریم بی‌حشمی مهتریم
&&
قند و شکر می‌خوریم در شکرستان خویش
\\
تو زر بس نادری نیست کست مشتری
&&
صنعت آن زرگری رو به سوی کان خویش
\\
دور قمر عمرها ناقص و کوته بود
&&
عمر درازی نهاد یار به دوران خویش
\\
دل سوی تبریز رفت در هوس شمس دین
&&
رو رو ای دل بجو زر به حرمدان خویش
\\
\end{longtable}
\end{center}
