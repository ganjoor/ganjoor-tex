\begin{center}
\section*{غزل شماره ۱۹۱۵: برو ای دل به سوی دلبر من}
\label{sec:1915}
\addcontentsline{toc}{section}{\nameref{sec:1915}}
\begin{longtable}{l p{0.5cm} r}
برو ای دل به سوی دلبر من
&&
بدان خورشید شرق و شمع روشن
\\
مرو هر سو به سوی بی‌سویی رو
&&
که هر مسکین بدان سو یافت مسکن
\\
بنه سر چون قلم بر خط امرش
&&
که هر بی‌سر از او افراشت گردن
\\
که جز در ظل آن سلطان خوبان
&&
دل ترسندگان را نیست مؤمن
\\
به دستت او دهد سرمایه زر
&&
ز پایت او گشاید بند آهن
\\
ور از انبوهی از در ره نیابی
&&
چو گنجشکان درآ از راه روزن
\\
وگر زان خرمن گل بو نیابی
&&
چه سود عنبرینه و مشک و لادن
\\
وگر سبلت ز شیرش تر نکردی
&&
برو ای قلتبان و ریش می کن
\\
چو دیدی روی او در دل بروید
&&
گل و نسرین و بید و سرو و سوسن
\\
درآمیزد دلت با آب حسنش
&&
چو آتش که درآویزد به روغن
\\
درآ در آتشش زیرا خلیلی
&&
مرم ز آتش نه‌ای نمرود بدظن
\\
درآ در بحر او تا همچو ماهی
&&
بروید مر تو را از خویش جوشن
\\
ز کاه غم جدا کن حب شادی
&&
که آن مه را برای ماست خرمن
\\
بهار آمد برون آ همچو سبزه
&&
به کوری دی و بر رغم بهمن
\\
نخمی چون کمان گر تیر اویی
&&
به قاب قوس رستستی ز مکمن
\\
زهی بر کار و ساکن تو به ظاهر
&&
مثال مرهمی در کار کردن
\\
خمش کن شد خموشی چون بلادر
&&
بلادر گر ننوشی باش کودن
\\
\end{longtable}
\end{center}
