\begin{center}
\section*{غزل شماره ۳۱۶۳: عشق در کفر کرد اظهاری}
\label{sec:3163}
\addcontentsline{toc}{section}{\nameref{sec:3163}}
\begin{longtable}{l p{0.5cm} r}
عشق در کفر کرد اظهاری
&&
بست ایمان ز ترس زناری
\\
بانگ زنهار از جهان برخاست
&&
هیچ کس را نداد زنهاری
\\
هیچ کنجی نبود بی‌خصمی
&&
هیچ گنجی نبود بی‌ماری
\\
نی که یوسف خزید در چاهی
&&
نه محمد گریخت در غاری
\\
پای ذاالنون کشید در زنجیر
&&
سر منصور رفت بر داری
\\
جز به کنج عدم نیاسایی
&&
در عدم درگریز یک باری
\\
جهت خرقه‌ای چنین زخمی
&&
این چنین درد سر ز دستاری
\\
کفن از خلعت و قبا خوشتر
&&
گور از این شهر به به بسیاری
\\
کی بود کز وجود بازرهم
&&
در عدم درپرم چو طیاری
\\
کی بود کز قفس برون پرد
&&
مرغ جانم به سوی گلزاری
\\
بچشد او غریب چاشت خوری
&&
بگشاید عجیب منقاری
\\
چون دل و چشم معده نور خورد
&&
ز آن که اصل غذا بد انواری
\\
بل هم احیاء عند ربهم
&&
بخورد یرزقون در اسراری
\\
آهوی مشک ناف من برهد
&&
ناگه از دام چرخ مکاری
\\
جان بر جان‌های پاک رود
&&
در جهانی که نیست بی‌کاری
\\
مشت گندم که اندر این دامست
&&
هست آن را مدد ز انباری
\\
باغ دنیا که تازه می‌گردد
&&
آخر آبش بود ز جوباری
\\
خاکیان را کی هوش می‌بخشد
&&
پادشاه قدیم و جباری
\\
گر نکردی نثار دانش و هوش
&&
کی بدی در زمانه هشیاری
\\
خاک خفته نداشت بیداری
&&
شاه کردش ز لطف بیداری
\\
خون و سرگین نداشت زیبایی
&&
پرده‌اش داد حسن ستاری
\\
جانب خرمن کرم بگریز
&&
هین قناعت مکن به ایثاری
\\
جامه از اطلسی بساز که هست
&&
بر سر عقل از او کله واری
\\
این کله را بده سری بستان
&&
کان سرت دارد از کله عاری
\\
ای دل من به برج شمس گریز
&&
زو قناعت مکن به دیداری
\\
شمس تبریز کز شعاع ویست
&&
شمس همراه چرخ دواری
\\
\end{longtable}
\end{center}
