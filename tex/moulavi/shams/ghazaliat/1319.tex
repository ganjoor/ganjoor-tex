\begin{center}
\section*{غزل شماره ۱۳۱۹: بباید عشق را ای دوست دردک}
\label{sec:1319}
\addcontentsline{toc}{section}{\nameref{sec:1319}}
\begin{longtable}{l p{0.5cm} r}
بباید عشق را ای دوست دردک
&&
دل پردرد و رخساران زردک
\\
ای بی‌درد دل و بی‌سوز سینه
&&
بود دعوی مشتاقیت سردک
\\
جهان عشق بس بی‌حد جهانست
&&
تو داری دیدگان نیک خردک
\\
چه داند روستایی مخزن شاه
&&
کماج و دوغ داند جان کردک
\\
بجز بانگ دفت نبود نصیبی
&&
چو هستی چون خصی در روز گردک
\\
اگر خواهی که مرد کار گردی
&&
ز کار و بار خود شو زود فردک
\\
چو چیزی یافتی خود را تو مفروش
&&
به پیش هر دکان مانند قردک
\\
که دعوی مردیت بی‌جان مردان
&&
بدان آرد که گویندت که مردک
\\
اگر ناگاه مردی پیش افتد
&&
به خون خود دری کاری نبردک
\\
تو دیده بسته‌ای در زهد می‌باش
&&
به تسبیح و به ذکر چند وردک
\\
مکن شیخی دروغی بر مریدان
&&
ار آن ناز و کرشمه ای فسردک
\\
شه شطرنجی ار تو کژ ببازی
&&
به شمس الدین تبریزی تو نردک
\\
\end{longtable}
\end{center}
