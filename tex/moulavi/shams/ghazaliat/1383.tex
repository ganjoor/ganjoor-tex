\begin{center}
\section*{غزل شماره ۱۳۸۳: تا من بدیدم روی تو ای ماه و شمع روشنم}
\label{sec:1383}
\addcontentsline{toc}{section}{\nameref{sec:1383}}
\begin{longtable}{l p{0.5cm} r}
تا من بدیدم روی تو ای ماه و شمع روشنم
&&
هر جا نشینم خرمم هر جا روم در گلشنم
\\
هر جا خیال شه بود باغ و تماشاگه بود
&&
در هر مقامی که روم بر عشرتی بر می تنم
\\
درها اگر بسته شود زین خانقاه شش دری
&&
آن ماه رو از لامکان سر درکند در روزنم
\\
گوید سلام علیک هی آوردمت صد نقل و می
&&
من شاهم و شاهنشهم پرده سپاهان می زنم
\\
من آفتاب انورم خوش پرده‌ها را بردرم
&&
من نوبهارم آمدم تا خارها را برکنم
\\
هر کس که خواهد روز و شب عیش و تماشا و طرب
&&
من قندها را لذتم بادام‌ها را روغنم
\\
گویم سخن را بازگو مردی کرم ز آغاز گو
&&
هین بی‌ملولی شرح کن من سخت کند و کودنم
\\
گوید که آن گوش گران بهتر ز هوش دیگران
&&
صد فضل دارد این بر آن کان جا هوا این جا منم
\\
رو رو که صاحب دولتی جان حیات و عشرتی
&&
رضوان و حور و جنتی زیرا گرفتی دامنم
\\
هم کوه و هم عنقا تویی هم عروه الوثقی تویی
&&
هم آب و هم سقا تویی هم باغ و سرو و سوسنم
\\
افلاک پیشت سر نهد املاک پیشت پر نهد
&&
دل گویدت مومم تو را با دیگران چون آهنم
\\
\end{longtable}
\end{center}
