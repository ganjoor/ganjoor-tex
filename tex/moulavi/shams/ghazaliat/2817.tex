\begin{center}
\section*{غزل شماره ۲۸۱۷: مکن ای دوست نشاید که بخوانند و نیایی}
\label{sec:2817}
\addcontentsline{toc}{section}{\nameref{sec:2817}}
\begin{longtable}{l p{0.5cm} r}
مکن ای دوست نشاید که بخوانند و نیایی
&&
و اگر نیز بیایی بروی زود نپایی
\\
هله ای دیده و نورم گه آن شد که بشورم
&&
پی موسی تو طورم شدی از طور کجایی
\\
اگرم خصم بخندد و گرم شحنه ببندد
&&
تو اگر نیز به قاصد به غضب دست بخایی
\\
به تو سوگند بخوردم که از این شیوه نگردم
&&
بکنم شور و بگردم به خدا و به خدایی
\\
بکن ای دوست چراغی که به از اختر و چرخی
&&
بکن ای دوست طبیبی که به هر درد دوایی
\\
دل ویران من اندر غلط ار جغد درآید
&&
بزند عکس تو بر وی کند آن جغد همایی
\\
هله یک قوم بگریند و یکی قوم بخندند
&&
ره عشق تو ببندند به استیزه نمایی
\\
اگر از خشم بجنگی وگر از خصم بلنگی
&&
و اگر شیر و پلنگی تو هم از حلقه مایی
\\
به بد و نیک زمانه نجهد عشق ز خانه
&&
نبود عشق فسانه که سمایی است سمایی
\\
چو مرا درد دوا شد چو مرا جور وفا شد
&&
چو مرا ارض سما شد چه کنم طال بقایی
\\
سحرالعین چه باشد که جهان خشک نماید
&&
بر عام و بر عارف چو گلستان رضایی
\\
هله این ناز رها کن نفسی روی به ما کن
&&
نفسی ترک دغا کن چه بود مکر و دغایی
\\
هله خاموش که تا او لب شیرین بگشاید
&&
بکند هر دو جهان را خضر وقت سقایی
\\
\end{longtable}
\end{center}
