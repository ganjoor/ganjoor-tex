\begin{center}
\section*{غزل شماره ۲۳۷۶: صنما از آنچ خوردی بهل اندکی به ما ده}
\label{sec:2376}
\addcontentsline{toc}{section}{\nameref{sec:2376}}
\begin{longtable}{l p{0.5cm} r}
صنما از آنچ خوردی بهل اندکی به ما ده
&&
غم تو به توی ما را تو به جرعه‌ای صفا ده
\\
که غم تو خورد ما را چه خراب کرد ما را
&&
به شراب شادی افزا غم و غصه را سزا ده
\\
ز شراب آسمانی که خدا دهد نهانی
&&
بنهان ز دست خصمان تو به دست آشنا ده
\\
بنشان تو جنگ‌ها را بنواز چنگ‌ها را
&&
ز عراق و از سپاهان تو به چنگ ما نوا ده
\\
سر خم چو برگشایی دو هزار مست تشنه
&&
قدح و کدو بیارند که مرا ده و مرا ده
\\
صنما ببین خزان را بنگر برهنگان را
&&
ز شراب همچو اطلس به برهنگان قبا ده
\\
به نظاره جوانان بنشسته‌اند پیران
&&
به می جوان تازه دو سه پیر را عصا ده
\\
به صلاح دین به زاری برسی که شهریاری
&&
ملک و شراب داری ز شراب جان عطا ده
\\
\end{longtable}
\end{center}
