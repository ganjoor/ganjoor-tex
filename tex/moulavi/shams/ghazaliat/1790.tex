\begin{center}
\section*{غزل شماره ۱۷۹۰: دلدار من در باغ دی می گشت و می گفت ای چمن}
\label{sec:1790}
\addcontentsline{toc}{section}{\nameref{sec:1790}}
\begin{longtable}{l p{0.5cm} r}
دلدار من در باغ دی می گشت و می گفت ای چمن
&&
صد حور خوش داری ولی بنگر یکی داری چو من
\\
گفتم صلای ماجرا ما را نمی‌پرسی چرا
&&
گفتا که پرسش‌های ما بیرون ز گوش است و دهن
\\
گفتم ز پرسش تو بحل باری اشارت را مهل
&&
گفت از اشارت‌های دل هم جان بسوزد هم بدن
\\
گفتم که چونی در سفر گفتا که چون باشد قمر
&&
سیمین بر و زرین کمر چشم و چراغ مرد و زن
\\
گشتن به گرد خود خطا الا جمال قطب را
&&
او را روا باشد روا کو ره رو است اندر وطن
\\
هم ساربان هم اشتران مستند از آن صاحب قران
&&
ای ساربان منزل مکن جز بر در آن یار من
\\
ای عشرت و ای ناز ما ای اصل و ای آغاز ما
&&
آخر چه داند راز ما جان حسن یا بوالحسن
\\
ای عشق تو در جان من چون آفتاب اندر حمل
&&
وی صورتت در چشم من همچون عقیق اندر یمن
\\
چون اولین و آخرین در حشر جمع آید یقین
&&
از تو نباشد خوبتر در جمله آن انجمن
\\
مجنون چو بیند مر تو را لیلی بر او کاسد شود
&&
لیلی چو بیند مر تو را گردد چو مجنون ممتحن
\\
در جست و جوی روی تو در پای گل بس خارها
&&
ای یاس من گوید همی‌اندر فراقت یاسمن
\\
گر آفتاب روی تو روزی ده ما نیستی
&&
ذرات کونین از طمع کی باز کردندی دهن
\\
حیوان چو قربانی بود جسمش ز جان فانی بود
&&
پس شرحه‌های گوشتش زنده شود زین بابزن
\\
آتش بگوید شرحه را سر حیاتات بقا
&&
کای رسته از جان فنا بر جان بی‌آزار زن
\\
نعره زنند آن شرحه‌ها یا لیت قومی یعلمون
&&
گر نعره شان این سو رسد نی گبر ماند نی وثن
\\
نی ترش ماند در دلی نی پای ماند در گلی
&&
لبیک لبیک و بلی می گوی و می رو تا وطن
\\
هست این سخن را باقیی در پرده مشتاقیی
&&
پیدا شود گر ساقیی ما را کند بی‌خویشتن
\\
\end{longtable}
\end{center}
