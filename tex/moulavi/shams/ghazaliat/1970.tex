\begin{center}
\section*{غزل شماره ۱۹۷۰: مطربا بردار چنگ و لحن موسیقار زن}
\label{sec:1970}
\addcontentsline{toc}{section}{\nameref{sec:1970}}
\begin{longtable}{l p{0.5cm} r}
مطربا بردار چنگ و لحن موسیقار زن
&&
آتش از جرمم بیار و اندر استغفار زن
\\
ای کلیم عشق بر فرعون هستی حمله بر
&&
بر سر او تو عصای محو موسی وار زن
\\
عقل از بهر هوس‌ها دارداری می کند
&&
زود چشمش را ببند و بهر او تو دار زن
\\
ور بگوید من به دانش نظم کاری می کنم
&&
آتشی دست آور و در نظم و اندر کار زن
\\
در غریبستان جان تا کی شوی مهمان خاک
&&
خاک اندر چشم این مهمان و مهمان دار زن
\\
مطربا حسنت ز پرگار خرد بیرونتر است
&&
خیمه عشرت برون از عقل و از پرگار زن
\\
تار چنگت را ز پود صرف می جانی بده
&&
زان حراره کهنه نوبخت بر اوتار زن
\\
بر در مخدوم شمس الدین ز دیده آب زن
&&
در همه هستی ز نار چهره او نار زن
\\
از یکی دستان او خورشید و مه را خفته کن
&&
پس نهان زو چنگ اندر دولت بیدار زن
\\
عقل هشیارت قبایی دوخت بهر شمس دین
&&
تو ز عشق او به چشم منکران مسمار زن
\\
بر براق عشق بنشین جانب تبریز رو
&&
و آنگهی زانو ز بهر غمزه خون خوار زن
\\
\end{longtable}
\end{center}
