\begin{center}
\section*{غزل شماره ۸۵۲: جز لطف و جز حلاوت خود از شکر چه آید}
\label{sec:0852}
\addcontentsline{toc}{section}{\nameref{sec:0852}}
\begin{longtable}{l p{0.5cm} r}
جز لطف و جز حلاوت خود از شکر چه آید
&&
جز نور بخش کردن خود از قمر چه آید
\\
جز رنگ‌های دلکش از گلستان چه خیزد
&&
جز برگ و جز شکوفه از شاخ تر چه آید
\\
جز طالع مبارک از مشتری چه یابی
&&
جز نقدهای روشن از کان زر چه آید
\\
آن آفتاب تابان مر لعل را چه بخشد
&&
وز آب زندگانی اندر جگر چه آید
\\
از دیدن جمالی کو حسن آفریند
&&
بالله یکی نظر کن کاندر نظر چه آید
\\
ماییم و شور مستی مستی و بت پرستی
&&
زین سان که ما شدستیم از ما دگر چه آید
\\
مستی و مستتر شو بی‌زیر و بی‌زبر شو
&&
بی خویش و بی‌خبر شو خود از خبر چه آید
\\
چیزی ز ماست باقی مردانه باش ساقی
&&
درده می رواقی زین مختصر چه آید
\\
چون گل رویم بیرون با جامه‌های گلگون
&&
مجنون شویم مجنون از خواب و خور چه آید
\\
ای شه صلاح دین تو بیرون مشو ز صورت
&&
بنما فرشتگان را تو کز بشر چه آید
\\
\end{longtable}
\end{center}
