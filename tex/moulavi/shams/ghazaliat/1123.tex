\begin{center}
\section*{غزل شماره ۱۱۲۳: پرده خوش آن بود کز پس آن پرده دار}
\label{sec:1123}
\addcontentsline{toc}{section}{\nameref{sec:1123}}
\begin{longtable}{l p{0.5cm} r}
پرده خوش آن بود کز پس آن پرده دار
&&
با رخ چون آفتاب سایه نماید نگار
\\
آید خورشیدوار ذره شود بی‌قرار
&&
کان رخ همچون بهار از پس پرده مدار
\\
خیز که این روز ماست روز دلفروز ماست
&&
از جهت سوز ماست عشق چنین پرشرار
\\
خیز که رستیم ما بند شکستیم ما
&&
خیز که مستیم ما تا به ابد بی‌خمار
\\
خیز که جان آمدست جان و جهان آمده است
&&
دست زنان آمدست ای دل دستی برآر
\\
آب حیات آمدست روز نجات آمدست
&&
قند و نبات آمدست ای صنم قندبار
\\
بنده آن پرده‌ام گوش گران کرده‌ام
&&
تا که به گوشم دهان آرد آن پرده دار
\\
مکر مرا چون بدید مکر دگر او پزید
&&
آمد و گوشم گزید گفت هلا ای عیار
\\
بی‌ادبی هم نکوست کان سبب جنگ اوست
&&
سر نکشم من ز دوست بهر چنین کار و بار
\\
جنگ تو است این حیات زانک ندارد ثبات
&&
جنگ تو خوش چون نبات صلح تو خود زینهار
\\
\end{longtable}
\end{center}
