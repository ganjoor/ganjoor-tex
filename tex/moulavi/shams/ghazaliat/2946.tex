\begin{center}
\section*{غزل شماره ۲۹۴۶: دی عهد و توبه کردی امروز درشکستی}
\label{sec:2946}
\addcontentsline{toc}{section}{\nameref{sec:2946}}
\begin{longtable}{l p{0.5cm} r}
دی عهد و توبه کردی امروز درشکستی
&&
دی بحر تلخ بودی امروز گوهرستی
\\
دی بایزید بودی و اندر مزید بودی
&&
و امروز در خرابی دردی فروش و مستی
\\
دردی بنوش ای جان بسکل ز هوش ای جان
&&
ازرق مپوش ای جان تا که صنم پرستی
\\
امروز بس خرابی هم جام آفتابی
&&
نی کدخدای ماهی نی شوهر مهستی
\\
افزونی از مساکن بیرونی از معادن
&&
آن نیستی ولیکن هستی چنانک هستی
\\
یک گوشه بسته بودی زان گوشه خسته بودی
&&
آن بسته را گشودی رستی تمام رستی
\\
حیوان سوار نبود جز بهر کار نبود
&&
حیوان نه‌ای تو حیی جستی ز کار جستی
\\
تو پیک آسمانی چون ماه کی توانی
&&
تا تو سوار پایی تا تو به دست شستی
\\
خامش مده نشانی گر چه ز هر بیانی
&&
شد مرهم جهانی هر خسته‌ای که خستی
\\
\end{longtable}
\end{center}
