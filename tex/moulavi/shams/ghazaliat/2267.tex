\begin{center}
\section*{غزل شماره ۲۲۶۷: هم صدوا هم عتبوا عتابا ما له سبب}
\label{sec:2267}
\addcontentsline{toc}{section}{\nameref{sec:2267}}
\begin{longtable}{l p{0.5cm} r}
هم صدوا هم عتبوا عتابا ما له سبب
&&
تن و دل ما مسخر او که می‌نپرد به جز بر او
\\
فما طلبوا سوی سقمی فطاب علی ما طلبوا
&&
عجب خبری که می‌دهدم دم و غم او کر و فر او
\\
فنی جلدی اذا عبسوا فکیف تری اذا طربوا
&&
مرا غم او چو زنده کند چگونه شوم ز منظر او
\\
فلا هرب اذا طلبوا و لا طرب اذا هربوا
&&
عجب چه بود بهر دو جهان که آن نبود میسر او
\\
اری امما به سکروا و لا قدح و لا عنب
&&
حدث نشود شکر که خوری شکر چو چشد ز شکر او
\\
لقد ملئت خواطرنا بهم عجبا و ما العجب
&&
سحر اثری ز طلعت او شبم نفسی ز عنبر او
\\
سکت او ناوهم سکتوا و لا سئمو و لا عتبوا
&&
خبر نکنم دگر که مرا رسید خبر ز مخبر او
\\
فوا حزنی اذا حجبوا و یا طربی اذا قربوا
&&
درم بزند سری نکند که سر نبرد کس از سر او
\\
\end{longtable}
\end{center}
