\begin{center}
\section*{غزل شماره ۱۴۵۲: ساقی چو شه من بد بیش از دگران خوردم}
\label{sec:1452}
\addcontentsline{toc}{section}{\nameref{sec:1452}}
\begin{longtable}{l p{0.5cm} r}
ساقی چو شه من بد بیش از دگران خوردم
&&
برگشت سر از مستی تخلیط و خطا کردم
\\
آن ساقی بایستم چون دید که سرمستم
&&
بگرفت سر دستم بوسید رخ زردم
\\
گفتم که تو سلطانی جانی و دو صد جانی
&&
تو خود نمکستانی شوری دگر آوردم
\\
از جام می خالص پرعربده شد مجلس
&&
از عربده کی ترسم من عربده پروردم
\\
بی‌او نکنم عشرت گر تشنه و مخمورم
&&
جفت نظرش باشم گر جفتم وگر فردم
\\
من شاخ ترم اما بی‌باد کجا رقصم
&&
من سایه آن سروم بی‌سرو کجا گردم
\\
نور دل ابر آمد آن ماه اگر ابرم
&&
شاه همه مردان است آن شاه اگر مردم
\\
می رفت شه شیرین گفتم نفسی بنشین
&&
ای مستی هر جزوم ای داروی هر دردم
\\
خورشید حمل کی بود ای گرمی تو بی‌حد
&&
ای محو شده در تو هم گرمم و هم سردم
\\
در کاس تو افتادم کز باده تو شادم
&&
در طاس تو افتادم چون مهره آن نردم
\\
ساکن شوم از گفتن گر اوم نشوراند
&&
زیرا که سوار است او من در قدمش گردم
\\
\end{longtable}
\end{center}
