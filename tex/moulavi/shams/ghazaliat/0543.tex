\begin{center}
\section*{غزل شماره ۵۴۳: یار مرا می‌نهلد تا که بخارم سر خود}
\label{sec:0543}
\addcontentsline{toc}{section}{\nameref{sec:0543}}
\begin{longtable}{l p{0.5cm} r}
یار مرا می‌نهلد تا که بخارم سر خود
&&
هیکل یارم که مرا می‌فشرد در بر خود
\\
گاه چو قطار شتر می‌کشدم از پی خود
&&
گاه مرا پیش کند شاه چو سرلشکر خود
\\
گه چو نگینم به مزد تا که به من مهر نهد
&&
گاه مرا حلقه کند دوزد او بر در خود
\\
خون ببرد نطفه کند نطفه برد خلق کند
&&
خلق کشد عقل کند فاش کند محشر خود
\\
گاه براند به نیم همچو کبوتر ز وطن
&&
گاه به صد لابه مرا خواند تا محضر خود
\\
گاه چو کشتی بردم بر سر دریا به سفر
&&
گاه مرا لنگ کند بندد بر لنگر خود
\\
گاه مرا آب کند از پی پاکی طلبان
&&
گاه مرا خار کند در ره بداختر خود
\\
هشت بهشت ابدی منظر آن شاه نشد
&&
تا چه خوش است این دل من کو کندش منظر خود
\\
من به شهادت نشدم مؤمن آن شاهد جان
&&
مؤمنش آن گاه شدم که بشدم کافر خود
\\
هر کی درآمد به صفش یافت امان از تلفش
&&
تیغ بدیدم به کفش سوختم آن اسپر خود
\\
همپر جبریل بدم ششصد پر بود مرا
&&
چونک رسیدم بر او تا چه کنم من پر خود
\\
حارس آن گوهر جان بودم روزان و شبان
&&
در تک دریای گهر فارغم از گوهر خود
\\
چند صفت می‌کنیش چونک نگنجد به صفت
&&
بس کن تا من بروم بر سر شور و شر خود
\\
\end{longtable}
\end{center}
