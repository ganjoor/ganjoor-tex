\begin{center}
\section*{غزل شماره ۱۹۵۸: ساقیا چون مست گشتی خویش را بر من بزن}
\label{sec:1958}
\addcontentsline{toc}{section}{\nameref{sec:1958}}
\begin{longtable}{l p{0.5cm} r}
ساقیا چون مست گشتی خویش را بر من بزن
&&
ذکر فردا نسیه باشد نسیه را گردن بزن
\\
سال سال ماست و طالع طالع زهره‌ست و ماه
&&
ای دل این عیش و طرب حدی ندارد تن بزن
\\
تا درون سنگ و آهن تابش و شادی رسید
&&
گر تو را باور نیاید سنگ بر آهن بزن
\\
بنگر اندر میزبان و در رخش شادی ببین
&&
بر سر این خوان نشین و کاسه در روغن بزن
\\
عقل زیرک را برآر و پهلوی شادی نشان
&&
جان روشن را سبک بر باده روشن بزن
\\
شاخه‌ها سرمست و رقصانند از باد بهار
&&
ای سمن مستی کن و ای سرو بر سوسن بزن
\\
جامه‌های سبز ببریدند بر دکان غیب
&&
خیز ای خیاط بنشین بر دکان سوزن بزن
\\
\end{longtable}
\end{center}
