\begin{center}
\section*{غزل شماره ۳۰۶: رغبت به عاشقان کن ای جان صدر غایب}
\label{sec:0306}
\addcontentsline{toc}{section}{\nameref{sec:0306}}
\begin{longtable}{l p{0.5cm} r}
رغبت به عاشقان کن ای جان صدر غایب
&&
بنشین میان مستان اینک مه و کواکب
\\
آن روز پرعجایب وان محشر قیامت
&&
گشتست پیش حسنت مستغرق عجایب
\\
چون طیبات خواندی بر طیبین فشاندی
&&
طیبتر از تو کی بود ای معدن اطایب
\\
جان را ز تست هر دم سلطانیی مسلم
&&
این شکر از کی گویم از شاه یا ز صاحب
\\
در جیب خاک کردی ارواح پاک جیبان
&&
سر کرده در گریبان چون صوفیان مراقب
\\
عشق تو چون درآمد اندیشه مرد پیشش
&&
عشق تو صبح صادق اندیشه صبح کاذب
\\
ای عقل باش حیران نی وصل جو نه هجران
&&
چون وصل گوش داری زان کس که نیست غایب
\\
جان چیست فقر و حاجت جان بخش کیست جز تو
&&
ای قبله حوایج معشوقه مطالب
\\
نک نقد شد قیامت اینک یکی علامت
&&
طالع شد آفتابت از جانب مغارب
\\
درکش رمیدگان را محنت رسیدگان را
&&
زان جذبه‌های جانی ای جذبه تو غالب
\\
تا بیند این دو دیده صبح خدا دمیده
&&
دام طلب دریده مطلوب گشته طالب
\\
عشق و طلب چه باشد آیینه تجلی
&&
نقش و حسد چه باشد آیینه معایب
\\
کو بلبل چمن‌ها تا گفتمی سخن‌ها
&&
نگذشت بر دهان‌ها یا دست هیچ کاتب
\\
نه از نقش‌های صورت نه از صاف و نه از کدورت
&&
نه از ماضی و نه حالی نه از زهد نه از مراتب
\\
عقلم برفت از جا باقیش را تو فرما
&&
ای از درت نرفته کس ناامید و غایب
\\
\end{longtable}
\end{center}
