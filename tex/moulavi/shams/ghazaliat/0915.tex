\begin{center}
\section*{غزل شماره ۹۱۵: بیا که ساقی عشق شراب باره رسید}
\label{sec:0915}
\addcontentsline{toc}{section}{\nameref{sec:0915}}
\begin{longtable}{l p{0.5cm} r}
بیا که ساقی عشق شراب باره رسید
&&
خبر ببر بر بیچارگان که چاره رسید
\\
امیر عشق رسید و شرابخانه گشاد
&&
شراب همچو عقیقش به سنگ خاره رسید
\\
هزار چشمه شیر و شکر روان شد از او
&&
شکاف کرد و به طفلان گاهواره رسید
\\
هزار مسجد پر شد چو عشق گشت امام
&&
صلوه خیر من النوم از آن مناره رسید
\\
بریز دیگ حلیماب را که کاسه رسید
&&
گشاده هل سر خم را که دردخواه رسید
\\
چو آفتاب جمالش به خاکیان درتافت
&&
زحل ز پرده هفتم پی نظاره رسید
\\
شدیم جمله فریدون چو تاج او دیدیم
&&
شدیم جمله منجم چو آن ستاره رسید
\\
شدیم جمله برهنه چو عشق او زد راه
&&
شدیم جمله پیاده چو او سواره رسید
\\
چو پاره پاره درآمد به لطف آن دلبر
&&
بدان طمع دل پرخون پاره پاره رسید
\\
بده زبان و همه گوش شو در این حضرت
&&
شتاب کن که پی گوش گوشواره رسید
\\
\end{longtable}
\end{center}
