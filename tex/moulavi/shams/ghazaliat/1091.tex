\begin{center}
\section*{غزل شماره ۱۰۹۱: نه که مهمان غریبم تو مرا یار مگیر}
\label{sec:1091}
\addcontentsline{toc}{section}{\nameref{sec:1091}}
\begin{longtable}{l p{0.5cm} r}
نه که مهمان غریبم تو مرا یار مگیر
&&
نه که فلاح توام سرور و سالار مگیر
\\
نه که همسایه آن سایه احسان توام
&&
تو مرا همسفر و مشفق و غمخوار مگیر
\\
شربت رحمت تو بر همگان گردانست
&&
تو مرا تشنه و مستسقی و بیمار مگیر
\\
نه که هر سنگ ز خورشید نصیبی دارد
&&
تو مرا منتظر و کشته دیدار مگیر
\\
نه که لطف تو گنه سوز گنه کارانست
&&
تو مرا تایب و مستغفر غفار مگیر
\\
نه که هر مرغ به بال و پر تو می‌پرد
&&
تو مرا صعوه شمر جعفر طیار مگیر
\\
به دو صد پر نتوان بی‌مددت پریدن
&&
تو مرا زیر چنین دام گرفتار مگیر
\\
خفتگان را نه تماشای نهان می‌بخشی
&&
تو مرا خفته شمر حاضر و بیدار مگیر
\\
نه که بوی جگر پخته ز من می‌آید
&&
مدد اشک من و زردی رخسار مگیر
\\
نه که مجنون ز تو زان سوی خرد باغی یافت
&&
از جنون خوش شد و می‌گفت خرد زار مگیر
\\
با جنون تو خوشم تا که فنون را چه کنم
&&
چون تو همخوابه شدی بستر هموار مگیر
\\
چشم مست تو خرابی دل و عقل همه‌ست
&&
عارض چون قمر و رنگ چو گلنار مگیر
\\
قامت عرعریت قامت ما دوتا کرد
&&
نادری ذقن و زلف چو زنار مگیر
\\
این تصاویر همه خود صور عشق بود
&&
عشق بی‌صورت چون قلزم زخار مگیر
\\
خرمن خاکم و آن ماه بگردم گردان
&&
تو مرا همتک این گنبد دوار مگیر
\\
من به کوی تو خوشم خانه من ویران گیر
&&
من به بوی تو خوشم نافه تاتار مگیر
\\
میکده‌ست این سر من ساغر می گو بشکن
&&
چون زرست این رخ من زر به خروار مگیر
\\
چون دلم بتکده شد آزر گو بت متراش
&&
چون سرم معصره شد خانه خمار مگیر
\\
کفر و اسلام کنون آمد و عشق از ازلست
&&
کافری را که کشد عشق ز کفار مگیر
\\
بانگ بلبل شنو ای گوش بهل نعره خر
&&
در گلستان نگر ای چشم و پی خار مگیر
\\
بس کن و طبل مزن گفت برای غیرست
&&
من خود اغیار خودم دامن اغیار مگیر
\\
\end{longtable}
\end{center}
