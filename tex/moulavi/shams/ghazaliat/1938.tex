\begin{center}
\section*{غزل شماره ۱۹۳۸: نازنینی را رها کن با شهان نازنین}
\label{sec:1938}
\addcontentsline{toc}{section}{\nameref{sec:1938}}
\begin{longtable}{l p{0.5cm} r}
نازنینی را رها کن با شهان نازنین
&&
ناز گازر برنتابد آفتاب راستین
\\
سایه خویشی فنا شو در شعاع آفتاب
&&
چند بینی سایه خود نور او را هم ببین
\\
درفکنده‌ای خویش غلطی بی‌خبر همچون ستور
&&
آدمی شو در ریاحین غلط و اندر یاسمین
\\
از خیال خویش ترسد هر کی در ظلمت بود
&&
زان که در ظلمت نماید نقش‌های سهمگین
\\
از ستاره روز باشد ایمنی کاروان
&&
زانک با خورشید آمد هم قران و هم قرین
\\
مرغ شب چون روز بیند گوید این ظلمت ز چیست
&&
زانک او گشته‌ست با شب آشنا و همنشین
\\
شاد آن مرغی که مهر شب در او محکم نگشت
&&
سوی تبریز آید او اندر هوای شمس دین
\\
\end{longtable}
\end{center}
