\begin{center}
\section*{غزل شماره ۱۱۵۱: قدح شکست و شرابم نماند و من مخمور}
\label{sec:1151}
\addcontentsline{toc}{section}{\nameref{sec:1151}}
\begin{longtable}{l p{0.5cm} r}
قدح شکست و شرابم نماند و من مخمور
&&
خراب کار مرا شمس دین کند معمور
\\
خدیو عالم بینش چراغ عالم کشف
&&
که روح‌هاش به جان سجده می‌کنند از دور
\\
که تا ز بحر تحیر برآورد دستش
&&
هزار جان و روان‌های غرقه مغمور
\\
گر آسمان و زمین پر شود ز ظلمت کفر
&&
چو او بتابد پرتو بگیرد آن همه نور
\\
از آن صفا که ملایک از او همی‌یابند
&&
اگر رسد به شیاطین شوند هر یک حور
\\
وگر نباشد آن نور دیو را روزی
&&
به پرده‌های کرم دیو را کند مستور
\\
به روز عیدی کو بخش کردن آغازد
&&
به هر سویست عروسی به هر نواحی سور
\\
ز سوی تبریز آن آفتاب درتابد
&&
شوند زنده ذرایر مثال نفخه صور
\\
ایا صبا به خدا و به حق نان و نمک
&&
که هر سحر من و تو گشته‌ایم از او مسرور
\\
که چون رسی به نهایت کران عالم غیب
&&
از آن گذر کن و کاهل مباش چون رنجور
\\
از آن پری که از او یافتی بکن پرواز
&&
هزارساله ره اندر پرت نباشد دور
\\
بپر چو خسته شود آن پرت سجودی کن
&&
برای حال من خسته جان و دل مهجور
\\
به آب چشم بگویش که از زمان فراق
&&
شدست روز سیاه و شدست مو کافور
\\
تو آن کسی که همه مجرمان عالم را
&&
به بحر رحمت غوطی دهی کنی مغفور
\\
چو چشم بینا در جان تو همی‌نرسد
&&
کسی که چشم ندارد یقین بود معذور
\\
چنان بکن تو به لابه که خاک پایش را
&&
بدیده آری کاین درد می‌شود ناسور
\\
وزین سفر به سعادت صبا چو بازآیی
&&
درافکنی به وجود و عدم شرار و شرور
\\
چو سرمه‌اش به من آری هزار رحمت نو
&&
به جانت بادا تا قرن‌های نامحصور
\\
\end{longtable}
\end{center}
