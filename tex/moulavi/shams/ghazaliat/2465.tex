\begin{center}
\section*{غزل شماره ۲۴۶۵: آمده‌ای که راز من بر همگان بیان کنی}
\label{sec:2465}
\addcontentsline{toc}{section}{\nameref{sec:2465}}
\begin{longtable}{l p{0.5cm} r}
آمده‌ای که راز من بر همگان بیان کنی
&&
و آن شه بی‌نشانه را جلوه دهی نشان کنی
\\
دوش خیال مست تو آمد و جام بر کفش
&&
گفتم می نمی‌خورم گفت مکن زیان کنی
\\
گفتم ترسم ار خورم شرم بپرد از سرم
&&
دست برم به جعد تو باز ز من کران کنی
\\
دید که ناز می‌کنم گفت بیا عجب کسی
&&
جان به تو روی آورد روی بدو گران کنی
\\
با همگان پلاس و کم با چو منی پلاس هم
&&
خاصبک نهان منم راز ز من نهان کنی
\\
گنج دل زمین منم سر چه نهی تو بر زمین
&&
قبله آسمان منم رو چه به آسمان کنی
\\
سوی شهی نگر که او نور نظر دهد تو را
&&
ور به ستیزه سر کشی روز اجل چنان کنی
\\
رنگ رخت که داد روز رد شو از برای او
&&
چون ز پی سیاهه‌ای روی چو زعفران کنی
\\
همچو خروس باش نر وقت شناس و پیش رو
&&
حیف بود خروس را ماده چو ماکیان کنی
\\
کژ بنشین و راست گو راست بود سزا بود
&&
جان و روان تو منم سوی دگر روان کنی
\\
گر به مثال اقرضوا قرض دهی قراضه‌ای
&&
نیم قراضه قلب را گنج کنی و کان کنی
\\
ور دو سه روز چشم را بند کنی باتقوا
&&
چشمه چشم حس را بحر در عیان کنی
\\
ور به نشان ما روی راست چو تیر ساعتی
&&
قامت تیر چرخ را بر زه خود کمان کنی
\\
بهتر از این کرم بود جرم تو را گنه تو را
&&
شرح کنم که پیش من بر چه نمط فغان کنی
\\
بس که نگنجد آن سخن کو بنبشت در دهان
&&
گر همه ذره ذره را بازکشی دهان کنی
\\
\end{longtable}
\end{center}
