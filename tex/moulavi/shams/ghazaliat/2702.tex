\begin{center}
\section*{غزل شماره ۲۷۰۲: مگیر ای ساقی از مستان کرانی}
\label{sec:2702}
\addcontentsline{toc}{section}{\nameref{sec:2702}}
\begin{longtable}{l p{0.5cm} r}
مگیر ای ساقی از مستان کرانی
&&
که کم یابی گرانی بی‌گرانی
\\
بیا ای سرو گلرخ سوی گلشن
&&
که به از سرو نبود سایه بانی
\\
چو نور از ناودان چشم ریزد
&&
یقین بی‌بام نبود ناودانی
\\
عجب آن بام بالای چه خانه‌ست
&&
مبارک جا مبارک خاندانی
\\
که را بود این گمان که بازیابیم
&&
نشانی زین چنین فتنه نشانی
\\
دلی که چون شفق غرقاب خون بود
&&
پر از خورشید شد چون آسمانی
\\
ز حرص این شکم پهلو تهی کن
&&
که تا پهلو زنی با پهلوانی
\\
عجب ننگت نمی‌آید برادر
&&
ز جانی کو بود محتاج نانی
\\
که آب زندگانی گفت ما را
&&
که جز دکان نان داری دکانی
\\
\end{longtable}
\end{center}
