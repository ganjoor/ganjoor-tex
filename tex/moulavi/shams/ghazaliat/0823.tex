\begin{center}
\section*{غزل شماره ۸۲۳: عمر بر اومید فردا می‌رود}
\label{sec:0823}
\addcontentsline{toc}{section}{\nameref{sec:0823}}
\begin{longtable}{l p{0.5cm} r}
عمر بر اومید فردا می‌رود
&&
غافلانه سوی غوغا می‌رود
\\
روزگار خویش را امروز دان
&&
بنگرش تا در چه سودا می‌رود
\\
گه به کیسه گه به کاسه عمر رفت
&&
هر نفس از کیسه ما می‌رود
\\
مرگ یک یک می‌برد وز هیبتش
&&
عاقلان را رنگ و سیما می‌رود
\\
مرگ در ره ایستاده منتظر
&&
خواجه بر عزم تماشا می‌رود
\\
مرگ از خاطر به ما نزدیکتر
&&
خاطر غافل کجاها می‌رود
\\
تن مپرور زانک قربانیست تن
&&
دل بپرور دل به بالا می‌رود
\\
چرب و شیرین کم ده این مردار را
&&
زانک تن پرورد رسوا می‌رود
\\
چرب و شیرین ده ز حکمت روح را
&&
تا قوی گردد که آن جا می‌رود
\\
حکمتت از شه صلاح الدین رسد
&&
آنک چون خورشید یکتا می‌رود
\\
\end{longtable}
\end{center}
