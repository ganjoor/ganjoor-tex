\begin{center}
\section*{غزل شماره ۶۱: هلا ای زهره زهرا بکش آن گوش زهرا را}
\label{sec:0061}
\addcontentsline{toc}{section}{\nameref{sec:0061}}
\begin{longtable}{l p{0.5cm} r}
هلا ای زهره زهرا بکش آن گوش زهرا را
&&
تقاضایی نهادستی در این جذبه دل ما را
\\
منم ناکام کام تو برای صید و دام تو
&&
گهی بر رکن بام تو گهی بگرفته صحرا را
\\
چه داند دام بیچاره فریب مرغ آواره
&&
چه داند یوسف مصری نتیجه شور و غوغا را
\\
گریبان گیر و این جا کش کسی را که تو خواهی خوش
&&
که من دامم تو صیادی چه پنهان صنعتی یارا
\\
چو شهر لوط ویرانم چو چشم لوط حیرانم
&&
سبب خواهم که واپرسم ندارم زهره و یارا
\\
اگر عطار عاشق بد سنایی شاه و فایق بد
&&
نه اینم من نه آنم من که گم کردم سر و پا را
\\
یکی آهم کز این آهم بسوزد دشت و خرگاهم
&&
یکی گوشم که من وقفم شهنشاه شکرخا را
\\
خمش کن در خموشی جان کشد چون کهربا آن را
&&
که جانش مستعد باشد کشاکش‌های بالا را
\\
\end{longtable}
\end{center}
