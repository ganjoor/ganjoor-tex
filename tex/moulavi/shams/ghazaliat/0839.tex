\begin{center}
\section*{غزل شماره ۸۳۹: خشمین بر آن کسی شو کز وی گزیر باشد}
\label{sec:0839}
\addcontentsline{toc}{section}{\nameref{sec:0839}}
\begin{longtable}{l p{0.5cm} r}
خشمین بر آن کسی شو کز وی گزیر باشد
&&
یا غیر خاک پایش کس دستگیر باشد
\\
گیرم کز او بگردی شاه و امیر و فردی
&&
ناچار مرگ روزی بر تو امیر باشد
\\
گر فاضلی و فردی آب خضر نخوردی
&&
هر کو نخورد آبش در مرگ اسیر باشد
\\
ای پیر جان فطرت پیر عیان نه فکرت
&&
پیری نه کز قدیدی مویش چو شیر باشد
\\
پیری مکن بر آن کس کز مکر و از فضولی
&&
خواهد که بازگونه بر پیر پیر باشد
\\
پیری بر آن کسی کن کو مرده تو باشد
&&
پیش جلالت تو خوار و حقیر باشد
\\
چون موی ابروی را وهمش هلال بیند
&&
بر چشمش آفتابت کی مستدیر باشد
\\
آن کس که از تکبر مالد سبال خود را
&&
از نور کبریایی چون مستنیر باشد
\\
عرضه گری رها کن ای خواجه خویش لا کن
&&
تا ذره وجودت شمس منیر باشد
\\
جلوه مکن جمالت مگشای پر و بالت
&&
تا با پر خدایی جان مستطیر باشد
\\
بربند پنج حس را زین سیل‌های تیره
&&
تا عقل کل ز شش سو بر تو مطیر باشد
\\
بی آن خمیرمایه گر تو خمیر تن را
&&
صد سال گرم داری نانش فطیر باشد
\\
گر قاب قوس خواهی دل راست کن چو تیری
&&
در قوس او درآید کو همچو تیر باشد
\\
خاموش اگر توانی بی‌حرف گو معانی
&&
تا بر بساط گفتن حاکم ضمیر باشد
\\
\end{longtable}
\end{center}
