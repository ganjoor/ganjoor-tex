\begin{center}
\section*{غزل شماره ۲۹۴۸: ای کرده رو چو سرکه چه گردد ار بخندی}
\label{sec:2948}
\addcontentsline{toc}{section}{\nameref{sec:2948}}
\begin{longtable}{l p{0.5cm} r}
ای کرده رو چو سرکه چه گردد ار بخندی
&&
والله ز سرکه رویی تو هیچ برنبندی
\\
تلخی ستان شکر ده سیلی بنوش و سر ده
&&
خندان بمیر چون گل گر ز آنک ارجمندی
\\
چون مو شده‌ست آن مه در خنده است و قهقه
&&
چت کم شود که گه گه از خوی ماه رندی
\\
بشکفته است شوره تو غوره‌ای و غوره
&&
آخر تو جان نداری تا چند مستمندی
\\
با کان غم نشینی شادی چگونه بینی
&&
از موش و موش خانه کی یافت کس بلندی
\\
بالای چرخ نیلی یابند جبرئیلی
&&
وز خاک پای پاکان یابند بی‌گزندی
\\
زان رنگ روی و سیما اسرار توست پیدا
&&
کاندر کدام کویی چه یار می‌پسندی
\\
چون چشم می‌گشاید در چشم می‌نماید
&&
گر ز آنک ریش گاوی ور شیر هوشمندی
\\
قارون مثال دلوی در قعر چه فروشد
&&
عیسی به بام گردون بنمود خوش کمندی
\\
گر دلو سر برآرد جز آب چه ندارد
&&
پاره شود بپوسد در ظلمت و نژندی
\\
ای لولیان لالا بالا پریده بالا
&&
وارسته زین هیولا فارغ ز چون و چندی
\\
\end{longtable}
\end{center}
