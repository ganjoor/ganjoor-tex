\begin{center}
\section*{غزل شماره ۱۰۶۱: عرض لشکر می‌دهد مر عاشقان را عشق یار}
\label{sec:1061}
\addcontentsline{toc}{section}{\nameref{sec:1061}}
\begin{longtable}{l p{0.5cm} r}
عرض لشکر می‌دهد مر عاشقان را عشق یار
&&
زندگان آن جا پیاده کشتگان آن جا سوار
\\
عارض رخسار او چون عارض لشکر شدست
&&
زخم چشم و چشم زخم عاشقان را گوش دار
\\
آفتابا شرم دار از روی او در ابر رو
&&
ماه تابان از چنان رخ الحذار و الحذار
\\
چون به لشکرگاه عشق آیی دو دیده وام کن
&&
وانگهان از یک نظر آن وام‌ها را می‌گزار
\\
جز خمار باده جان چشم را تدبیر نیست
&&
باده جان از که گیری زان دو چشم پرخمار
\\
چون تو پای لنگ داری گو پر از خلخال باش
&&
گوش کر را سود نبود از هزاران گوشوار
\\
گر عصا را تو بدزدی از کف موسی چه سود
&&
بازوی حیدر بباید تا براند ذوالفقار
\\
دست عیسی را بگیر و سرمه چوب از وی مدزد
&&
تا ببینی کار دست و تا ببینی دست کار
\\
گر ندانی کرد آن سو زیرزیرک می‌نگر
&&
نی به چشم امتحانی بل به چشم اعتبار
\\
زانک آن سو در نوازش رحمتی جوشیده است
&&
شمس تبریزیش گویم یا جمال کردگار
\\
\end{longtable}
\end{center}
