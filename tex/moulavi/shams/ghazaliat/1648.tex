\begin{center}
\section*{غزل شماره ۱۶۴۸: ساقیا عربده کردیم که در جنگ شویم}
\label{sec:1648}
\addcontentsline{toc}{section}{\nameref{sec:1648}}
\begin{longtable}{l p{0.5cm} r}
ساقیا عربده کردیم که در جنگ شویم
&&
می گلرنگ بده تا همه یک رنگ شویم
\\
صورت لطف سقی الله تویی در دو جهان
&&
رخ می رنگ نما تا همگان دنگ شویم
\\
باده منسوخ شود چون به صفت باده شویم
&&
بنگ منسوخ شود چون همگی بنگ شویم
\\
هین که اندیشه و غم پهلوی ما خانه گرفت
&&
باده ده تا که از او ما به دو فرسنگ شویم
\\
مطربا بهر خدا زخمه مستانه بزن
&&
تا ز زخمه خوش تو ساخته چون چنگ شویم
\\
مجلس قیصر روم است بده صیقل دل
&&
تا که چون آینه جان همه بی‌رنگ شویم
\\
یک جهان تنگ دل و ما ز فراخی نشاط
&&
یک نفس عاشق آنیم که دلتنگ شویم
\\
دشمن عقل کی دیده‌ست کز آمیزش او
&&
همه عقل و همه علم و همه فرهنگ شویم
\\
شمس تبریز چو در باغ صفا رو بنمود
&&
زود در گردن عشقش همه آونگ شویم
\\
\end{longtable}
\end{center}
