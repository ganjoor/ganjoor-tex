\begin{center}
\section*{غزل شماره ۱۴۸۰: خیزید مخسپید که نزدیک رسیدیم}
\label{sec:1480}
\addcontentsline{toc}{section}{\nameref{sec:1480}}
\begin{longtable}{l p{0.5cm} r}
خیزید مخسپید که نزدیک رسیدیم
&&
آواز خروس و سگ آن کوی شنیدیم
\\
والله که نشان‌های قروی ده یارست
&&
آن نرگس و نسرین و قرنفل که چریدیم
\\
از ذوق چراگاه و ز اشتاب چریدن
&&
وز حرص زبان و لب و پدفوز گزیدیم
\\
چون تیر پریدیم و بسی صید گرفتیم
&&
گر چه چو کمان از زه احکام خمیدیم
\\
ما عاشق مستیم به صد تیغ نگردیم
&&
شیریم که خون دل فغفور چشیدیم
\\
مستان الستیم به جز باده ننوشیم
&&
بر خوان جهان نی ز پی آش و ثریدیم
\\
حق داند و حق دید که در وقت کشاکش
&&
از ما چه کشیدید وز ایشان چه کشیدیم
\\
خیزید مخسپید که هنگام صبوح است
&&
استاره روز آمد و آثار بدیدیم
\\
شب بود و همه قافله محبوس رباطی
&&
خیزید کز آن ظلمت و آن حبس رهیدیم
\\
خورشید رسولان بفرستاد در آفاق
&&
کاینک یزک مشرق و ما جیش عتیدیم
\\
هین رو به شفق آر اگر طایر روزی
&&
کز سوی شفق چون نفس صبح دمیدیم
\\
هر کس که رسولی شفق را بشناسد
&&
ما نیز در اظهار بر او فاش و پدیدیم
\\
وان کس که رسولی شفق را نپذیرد
&&
هم محرم ما نیست بر او پرده تنیدیم
\\
خفاش نپذرفت فرودوخت از او چشم
&&
ما پرده آن دوخته را هم بدریدیم
\\
تریاق جهان دید و گمان برد که زهر است
&&
ای مژده دلی را که ز پندار خریدیم
\\
خامش کن تا واعظ خورشید بگوید
&&
کو بر سر منبر شد و ما جمله مریدیم
\\
\end{longtable}
\end{center}
