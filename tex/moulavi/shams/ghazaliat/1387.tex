\begin{center}
\section*{غزل شماره ۱۳۸۷: هین خیره خیره می نگر اندر رخ صفراییم}
\label{sec:1387}
\addcontentsline{toc}{section}{\nameref{sec:1387}}
\begin{longtable}{l p{0.5cm} r}
هین خیره خیره می نگر اندر رخ صفراییم
&&
هر کس که او مکی بود داند که من بطحاییم
\\
زان لاله روی دلستان روید ز رویم زعفران
&&
هر لحظه زان شادی فزا بیش است کارافزاییم
\\
مانند برف آمد دلم هر لحظه می کاهد دلم
&&
آن جا همی‌خواهد دلم زیرا که من آن جاییم
\\
هر جا حیاتی بیشتر مردم در او بی‌خویشتر
&&
خواهی بیا در من نگر کز شید جان شیداییم
\\
آن برف گوید دم به دم بگذارم و سیلی شوم
&&
غلطان سوی دریا روم من بحری و دریاییم
\\
تنها شدم راکد شدم بفسردم و جامد شدم
&&
تا زیر دندان بلا چون برف و یخ می خاییم
\\
چون آب باش و بی‌گره از زخم دندان‌ها بجه
&&
من تا گره دارم یقین می کوبی و می ساییم
\\
برف آب را بگذار هین فقاع‌های خاص بین
&&
می جوشد و بر می جهد که تیزم و غوغاییم
\\
هر لحظه بخروشانترم برجسته و جوشانترم
&&
چون عقل بی‌پر می پرم زیرا چو جان بالاییم
\\
بسیار گفتم ای پدر دانم که دانی این قدر
&&
که چون نیم بی‌پا و سر در پنجه آن ناییم
\\
گر تو ملولستی ز من بنگر در آن شاه زمن
&&
تا گرم و شیرینت کند آن دلبر حلواییم
\\
ای بی‌نوایان را نوا جان ملولان را دوا
&&
پران کننده جان که من از قافم و عنقاییم
\\
من بس کنم بس از حنین او بس نخواهد کرد از این
&&
من طوطیم عشقش شکر هست از شکر گویاییم
\\
\end{longtable}
\end{center}
