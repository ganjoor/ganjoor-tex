\begin{center}
\section*{غزل شماره ۱۰۵۸: آخر کی شود از آن لقا سیر}
\label{sec:1058}
\addcontentsline{toc}{section}{\nameref{sec:1058}}
\begin{longtable}{l p{0.5cm} r}
آخر کی شود از آن لقا سیر
&&
آخر کی شود ز باغ ما سیر
\\
ای عدل تو کرده چرخ را سبز
&&
وی لطف تو کرده باغ را سیر
\\
رو بنمایید ای ظریفان
&&
کز جان خودیم بی‌شما سیر
\\
آن نقل هزارمن بریزید
&&
تا گردد هر کجا گدا سیر
\\
در بزم رضای تست نقلی
&&
وز وی دل و چشم انبیا سیر
\\
کی گردد سیر ماهی از آب
&&
کی گردد خلق از خدا سیر
\\
مشتاب مرو که کیمیایی
&&
تا مس بچرد ز کیمیا سیر
\\
خوانی دگرست غیر این خوان
&&
تا لوت خورند اولیا سیر
\\
تا ذوق جفاش دید جانم
&&
در عشق جفاست از وفا سیر
\\
کز ملکت سیر شد سلیمان
&&
و ایوب نگشت از بلا سیر
\\
چه مکر و چه نعل باژگونه‌ست
&&
خود گرسنه نادرست یا سیر
\\
خاموش کن و دغا رها کن
&&
آخر نشدی از این دغا سیر
\\
\end{longtable}
\end{center}
