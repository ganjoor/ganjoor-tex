\begin{center}
\section*{غزل شماره ۱۹۴۹: آنچ می آید ز وصفت این زمانم در دهن}
\label{sec:1949}
\addcontentsline{toc}{section}{\nameref{sec:1949}}
\begin{longtable}{l p{0.5cm} r}
آنچ می آید ز وصفت این زمانم در دهن
&&
بر مرید مرده خوانم اندراندازد کفن
\\
خود مرید من نمیرد کآب حیوان خورده است
&&
وانگهان از دست کی از ساقیان ذوالمنن
\\
ای نجات زندگان و ای حیات مردگان
&&
از درونم بت تراشی وز برونم بت شکن
\\
ور براندازد ز رویت باد دولت پرده‌ای
&&
از حیا گل آب گردد نی چمن ماند نه من
\\
ور می لب بازگیری از گلستان ساعتی
&&
از خمار و سرگرانی هر سمن گردد سه من
\\
ور زمانی بی‌دلان را دم دهی و دل دهی
&&
جان رهد از ننگ ما و ما رهیم از خویشتن
\\
گر ندزدید از تو چیزی دل چرا آویخته‌ست
&&
چاره نبود دزد را در عاقبت ز آویختن
\\
گر چنین آویختن حاصل شدی هر دزد را
&&
از حریصی دزد گشتی جمله عالم مرد و زن
\\
اندر این آویختن کمتر کراماتی که هست
&&
آب حیوان خوردن است و تا ابد باقی شدن
\\
چاشنی سوز شمعت گر به عنقا برزدی
&&
پر چو پروانه بدادی سر نهادی در لگن
\\
صورت صنع تو آمد ساعتی در بتکده
&&
گه شمن بت می شد آن دم گاه بت می شد شمن
\\
هر زمانی نقش می شد نعت احمد بر صلیب
&&
سر وحدت می شنیدند آشکارا از وثن
\\
عشقت ای خوب ختن بر دل سواره گشت گفت
&&
این چنین مرکب بباید تاختن را تا ختن
\\
شور تو عقلم ستد با فتنه‌ها دربافتم
&&
شور و بی‌عقلی بباید بافتن را با فتن
\\
من کجا شعر از کجا لیکن به من در می دمد
&&
آن یکی ترکی که آید گویدم هی کیمسن
\\
ترک کی تاجیک کی زنگی کی رومی کی
&&
مالک الملکی که داند مو به مو سر و علن
\\
جامه شعر است شعر و تا درون شعر کیست
&&
یا که حوری جامه زیب و یا که دیوی جامه کن
\\
شعرش از سر برکشیم و حور را در بر کشیم
&&
فاعلاتن فاعلاتن فاعلاتن فاعلن
\\
\end{longtable}
\end{center}
