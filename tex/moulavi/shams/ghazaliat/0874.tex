\begin{center}
\section*{غزل شماره ۸۷۴: امروز مرده بین که چه سان زنده می‌شود}
\label{sec:0874}
\addcontentsline{toc}{section}{\nameref{sec:0874}}
\begin{longtable}{l p{0.5cm} r}
امروز مرده بین که چه سان زنده می‌شود
&&
آزاد سرو بین که چه سان بنده می‌شود
\\
پوسیده استخوان و کفن‌های مرده بین
&&
کز روح و علم و عشق چه آکنده می‌شود
\\
آن حلق و آن دهان که دریدست در لحد
&&
چون عندلیب مست چه گوینده می‌شود
\\
آن جان به شیشه‌ای که ز سوزن همی‌گریخت
&&
جان را به تیغ عشق فروشنده می‌شود
\\
بسیار دیده‌ای که بجوشد ز سنگ آب
&&
از شهد شیر بین که چه جوشنده می‌شود
\\
امروز کعبه بین که روان شد به سوی حاج
&&
کز وی هزار قافله فرخنده می‌شود
\\
امروز غوره بین که شکر بست از نشاط
&&
امروز شوره بین که چه روینده می‌شود
\\
می‌خند ای زمین که بزادی خلیفه‌ای
&&
کز وی کلوخ و سنگ تو جنبنده می شود
\\
غم مرد و گریه رفت بقای من و تو باد
&&
هر جا که گریه ایست کنون خنده می‌شود
\\
آن گلشنی شکفت که از فر بوی او
&&
بی داس و تیش خار تو برکنده می‌شود
\\
پاینده گشت خضر که آب حیات دید
&&
پاینده گشت و دید که پاینده می‌شود
\\
پاینده عمر باد روان لطیف ما
&&
جان را بقاست تن چو قبا ژنده می‌شود
\\
خاموش و خوش بخسپ در این خرمن شکر
&&
زیرا شکر به گفت پراکنده می‌شود
\\
من خامشم ولیک ز هیهای طوطیان
&&
هم نیشکر ز لطف خروشنده می‌شود
\\
\end{longtable}
\end{center}
