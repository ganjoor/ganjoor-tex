\begin{center}
\section*{غزل شماره ۳۳۳: اندر دل هر کس که از این عشق اثر نیست}
\label{sec:0333}
\addcontentsline{toc}{section}{\nameref{sec:0333}}
\begin{longtable}{l p{0.5cm} r}
اندر دل هر کس که از این عشق اثر نیست
&&
تو ابر در او کش که به جز خصم قمر نیست
\\
ای خشک درختی که در آن باغ نرستست
&&
وی خوار عزیزی که در این ظل شجر نیست
\\
بسکل ز جز این عشق اگر در یتیمی
&&
زیرا که جز این عشق تو را خویش و پدر نیست
\\
در مذهب عشاق به بیماری مرگست
&&
هر جان که به هر روز از این رنج بتر نیست
\\
در صورت هر کس که از آن رنگ بدیدی
&&
می‌دان تو به تحقیق که از جنس بشر نیست
\\
هر نی که بدیدی به میانش کمر عشق
&&
تنگش تو به بر گیر که جز تنگ شکر نیست
\\
شمس الحق تبریز چو در دام کشیدت
&&
منگر به چپ و راست که امکان حذر نیست
\\
\end{longtable}
\end{center}
