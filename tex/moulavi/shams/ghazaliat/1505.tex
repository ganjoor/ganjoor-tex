\begin{center}
\section*{غزل شماره ۱۵۰۵: یکی مطرب همی‌خواهم در این دم}
\label{sec:1505}
\addcontentsline{toc}{section}{\nameref{sec:1505}}
\begin{longtable}{l p{0.5cm} r}
یکی مطرب همی‌خواهم در این دم
&&
که نشناسد ز مستی زیر از بم
\\
حریفی نیز خواهم غمگساری
&&
ز بی‌خویشی نداند شادی از غم
\\
همه اجزای او مستی گرفته
&&
مبدل گشته از اولاد آدم
\\
مسلمانی منور گشته از وی
&&
مسلم گشته از هستی مسلم
\\
چو با نه کس بیاید بشمری ده
&&
ده تو نه بود از ده یکی کم
\\
خدایا نوبتی مست بفرست
&&
که ما از می دهل کردیم اشکم
\\
دهل کوبان برون آییم از خویش
&&
که ما را عزم ساقی شد مصمم
\\
دهلزن گر نباشد عید عید است
&&
جهان پرعید شد والله اعلم
\\
پراکنده بخواهم گفت امروز
&&
چه گوید مرد درهم جز که درهم
\\
مگر ساقی بینداید دهانم
&&
از آن جام و از آن رطل دمادم
\\
مرادم کیست زین‌ها شمس تبریز
&&
ازیرا شمس آمد جان عالم
\\
\end{longtable}
\end{center}
