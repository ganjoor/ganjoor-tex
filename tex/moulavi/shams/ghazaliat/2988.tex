\begin{center}
\section*{غزل شماره ۲۹۸۸: هر چند شیر بیشه و خورشیدطلعتی}
\label{sec:2988}
\addcontentsline{toc}{section}{\nameref{sec:2988}}
\begin{longtable}{l p{0.5cm} r}
هر چند شیر بیشه و خورشیدطلعتی
&&
بر گرد حوض گردی و در حوض درفتی
\\
اسپت بیاورند که چالاک فارسی
&&
شربت بیاورند که مخمور شربتی
\\
بی خواب و بی‌قراری شب‌های تا به روز
&&
خواب تو بخت بست که بسته سعادتی
\\
از پای درفتادی و از دست رفته‌ای
&&
بی دست و پای باش چه دربند آلتی
\\
بی دست و پا چو گوی به میدان حق بپوی
&&
میدان از آن توست به چوگان تو بابتی
\\
ای رو به قبله من و الحمدخوان من
&&
می‌خوانمت به خویش که تو پنج آیتی
\\
ای عقل جان بباز چرا جان به شیشه‌ای
&&
وی جان بیار باده چرا بی‌مروتی
\\
رو کان مشک باش که بس پاک نافه‌ای
&&
رو جمله سود باش که فرخ تجارتی
\\
بر مغز من برآی که چون می مفرحی
&&
در چشم من درآی که نور بصارتی
\\
در مغزها نگنجی بس بی‌کرانه‌ای
&&
در جسم‌ها نگنجی ز ایشان زیادتی
\\
ای دف زخم خواره چه مظلوم و صابری
&&
وی نای رازگوی چه صاحب کرامتی
\\
خامش مساز بیت که مهمان بیت تو
&&
در بیت‌ها نگنجد چه در عمارتی
\\
چون غنچه لب ببند و چو گل بی‌دو لب بخند
&&
تا هیچ کس نداند کاندر چه نعمتی
\\
ای شاه شاد مفخر تبریز شمس دین
&&
تبلیغ راز کن که تو اهل سفارتی
\\
\end{longtable}
\end{center}
