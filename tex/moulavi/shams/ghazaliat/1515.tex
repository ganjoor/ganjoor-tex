\begin{center}
\section*{غزل شماره ۱۵۱۵: چه نزدیک است جان تو به جانم}
\label{sec:1515}
\addcontentsline{toc}{section}{\nameref{sec:1515}}
\begin{longtable}{l p{0.5cm} r}
چه نزدیک است جان تو به جانم
&&
که هر چیزی که اندیشی بدانم
\\
از این نزدیکتر دارم نشانی
&&
بیا نزدیک و بنگر در نشانم
\\
به درویشی بیا اندر میانه
&&
مکن شوخی مگو کاندر میانم
\\
میان خانه‌ات همچون ستونم
&&
ز بامت سرفرو چون ناودانم
\\
منم همراز تو در حشر و در نشر
&&
نه چون یاران دنیا میزبانم
\\
میان بزم تو گردان چو خمرم
&&
گه رزم تو سابق چون سنانم
\\
اگر چون برق مردن پیشه سازم
&&
چو برق خوبی تو بی‌زبانم
\\
همیشه سرخوشم فرقی نباشد
&&
اگر من جان دهم یا جان ستانم
\\
به تو گر جان دهم باشد تجارت
&&
که بدهی به هر جانی صد جهانم
\\
در این خانه هزاران مرده بیش اند
&&
تو بنشسته که اینک خان و مانم
\\
یکی کف خاک گوید زلف بودم
&&
یکی کف خاک گوید استخوانم
\\
شوی حیران و ناگه عشق آید
&&
که پیشم آ که زنده جاودانم
\\
بکش در بر بر سیمین ما را
&&
که از خویشت همین دم وارهانم
\\
خمش کن خسروا هم گو ز شیرین
&&
ز شیرینی همی‌سوزد دهانم
\\
\end{longtable}
\end{center}
