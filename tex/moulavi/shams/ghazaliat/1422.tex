\begin{center}
\section*{غزل شماره ۱۴۲۲: طواف حاجیان دارم بگرد یار می گردم}
\label{sec:1422}
\addcontentsline{toc}{section}{\nameref{sec:1422}}
\begin{longtable}{l p{0.5cm} r}
طواف حاجیان دارم بگرد یار می گردم
&&
نه اخلاق سگان دارم نه بر مردار می گردم
\\
مثال باغبانانم نهاده بیل بر گردن
&&
برای خوشه خرما به گرد خار می گردم
\\
نه آن خرما که چون خوردی شود بلغم کند صفرا
&&
ولیکن پر برویاند که چون طیار می گردم
\\
جهان مارست و زیر او یکی گنجی است بس پنهان
&&
سر گنجستم و بر وی چو دم مار می گردم
\\
ندارم غصه دانه اگر چه گرد این خانه
&&
فرورفته به اندیشه چو بوتیمار می گردم
\\
نخواهم خانه‌ای در ده نه گاو و گله فربه
&&
ولیکن مست سالارم پی سالار می گردم
\\
رفیق خضرم و هر دم قدوم خضر را جویان
&&
قدم برجا و سرگردان که چون پرگار می گردم
\\
نمی‌دانی که رنجورم که جالینوس می جویم
&&
نمی‌بینی که مخمورم که بر خمار می گردم
\\
نمی‌دانی که سیمرغم که گرد قاف می پرم
&&
نمی‌دانی که بو بردم که بر گلزار می گردم
\\
مرا زین مردمان مشمر خیالی دان که می گردد
&&
خیال ار نیستم ای جان چه بر اسرار می گردم
\\
چرا ساکن نمی‌گردم بر این و آن همی‌گویم
&&
که عقلم برد و مستم کرد ناهموار می گردم
\\
مرا گویی مرو شپشپ که حرمت را زیان دارد
&&
ز حرمت عار می دارم از آن بر عار می گردم
\\
بهانه کرده‌ام نان را ولیکن مست خبازم
&&
نه بر دینار می گردم که بر دیدار می گردم
\\
هر آن نقشی که پیش آید در او نقاش می بینم
&&
برای عشق لیلی دان که مجنون وار می گردم
\\
در این ایوان سربازان که سر هم در نمی‌گنجد
&&
من سرگشته معذورم که بی‌دستار می گردم
\\
نیم پروانه آتش که پر و بال خود سوزم
&&
منم پروانه سلطان که بر انوار می گردم
\\
چه لب را می گزی پنهان که خامش باش و کمتر گوی
&&
نه فعل و مکر توست این هم که بر گفتار می گردم
\\
بیا ای شمس تبریزی شفق وار ار چه بگریزی
&&
شفق وار از پی شمست بر این اقطار می گردم
\\
\end{longtable}
\end{center}
