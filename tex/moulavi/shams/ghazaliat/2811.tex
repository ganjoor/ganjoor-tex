\begin{center}
\section*{غزل شماره ۲۸۱۱: از هوای شمس دین بنگر تو این دیوانگی}
\label{sec:2811}
\addcontentsline{toc}{section}{\nameref{sec:2811}}
\begin{longtable}{l p{0.5cm} r}
از هوای شمس دین بنگر تو این دیوانگی
&&
با همه خویشان گرفته شیوه بیگانگی
\\
وحش صحرا گشته و رسوای بازاری شده
&&
از هوای خانه او صد هزاران خانگی
\\
صاعقه هجرش زده برسوخته یک بارگی
&&
عقل و شرم و فهم و تقوا دانش و فرزانگی
\\
من ز شمع عشق او نان پاره‌ای می‌خواستم
&&
گفت بنویسید توقیعش پی پروانگی
\\
ای گشاده قلعه‌های جان به چشم آتشین
&&
ای هزاران صف دریده عشقت از مردانگی
\\
ای خداوند شمس دین صد گنج خاک است پیش تو
&&
تا چه باشد عاشق بیچاره‌ای یک دانگی
\\
صد غریو و بانگ اندر سقف گردون افکنیم
&&
من نیم در عشق پابرجای تو یک بانگی
\\
عقل را گفتم میان جان و جانان فرق کن
&&
شانه عقلم ز فرقش یاوه کرده شانگی
\\
\end{longtable}
\end{center}
