\begin{center}
\section*{غزل شماره ۱۷۲۵: نگفتمت مرو آن جا که آشنات منم}
\label{sec:1725}
\addcontentsline{toc}{section}{\nameref{sec:1725}}
\begin{longtable}{l p{0.5cm} r}
نگفتمت مرو آن جا که آشنات منم
&&
در این سراب فنا چشمه حیات منم
\\
وگر به خشم روی صد هزار سال ز من
&&
به عاقبت به من آیی که منتهات منم
\\
نگفتمت که به نقش جهان مشو راضی
&&
که نقش بند سراپرده رضات منم
\\
نگفتمت که منم بحر و تو یکی ماهی
&&
مرو به خشک که دریای باصفات منم
\\
نگفتمت که چو مرغان به سوی دام مرو
&&
بیا که قدرت پرواز و پرّ و پات منم
\\
نگفتمت که تو را ره زنند و سرد کنند
&&
که آتش و تبش و گرمی هوات منم
\\
نگفتمت که صفت‌های زشت در تو نهند
&&
که گم کنی که سر چشمه صفات منم
\\
نگفتمت که مگو کار بنده از چه جهت
&&
نظام گیرد خلاق بی‌جهات منم
\\
اگر چراغ دلی دان که راه خانه کجاست
&&
وگر خداصفتی دان که کدخدات منم
\\
\end{longtable}
\end{center}
