\begin{center}
\section*{غزل شماره ۱۳۰۱: ما دو سه رند عشرتی جمع شدیم این طرف}
\label{sec:1301}
\addcontentsline{toc}{section}{\nameref{sec:1301}}
\begin{longtable}{l p{0.5cm} r}
ما دو سه رند عشرتی جمع شدیم این طرف
&&
چون شتران رو به رو پوز نهاده در علف
\\
از چپ و راست می‌رسد مست طمع هر اشتری
&&
چون شتران فکنده لب مست و برآوریده کف
\\
غم مخورید هر شتر ره نبرد بدین اغل
&&
زانک به پستی‌اند و ما بر سر کوه بر شرف
\\
کس به درازگردنی بر سر کوه کی رسد
&&
ور چه کنند عف عفی غم نخوریم ما ز عف
\\
بحر اگر شود جهان کشتی نوح اندرآ
&&
کشتی نوح کی بود سخره غرقه و تلف
\\
کان زمردیم ما آفت چشم اژدها
&&
آنک لدیغ غم بود حصه اوست وااسف
\\
جمله جهان پرست غم در پی منصب و درم
&&
ما خوش و نوش و محترم مست طرب در این کنف
\\
مست شدند عارفان مطرب معرفت بیا
&&
زود بگو رباعیی پیش درآ بگیر دف
\\
باد به بیشه درفکن در سر سرو و بید زن
&&
تا که شوند سرفشان بید و چنار صف به صف
\\
بید چو خشک و کل بود برگ ندارد و ثمر
&&
جنبش کی کند سرش از دم و باد لاتخف
\\
چاره خشک و بی‌مدد نفخه ایزدی بود
&&
کوست به فعل یک به یک نیست ضعیف و مستخف
\\
نخله خشک ز امر حق داد ثمر به مریمی
&&
یافت ز نفخ ایزدی مرده حیات مؤتنف
\\
ابله اگر زنخ زند تو ره عشق گم مکن
&&
پیشه عشق برگزین هرزه شمر دگر حرف
\\
چون غزلی به سر بری مدحت شمس دین بگو
&&
وز تبریز یاد کن کوری خصم ناخلف
\\
\end{longtable}
\end{center}
