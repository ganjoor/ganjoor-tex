\begin{center}
\section*{غزل شماره ۲۰۸۰: چهار شعر بگفتم بگفت نی به از این}
\label{sec:2080}
\addcontentsline{toc}{section}{\nameref{sec:2080}}
\begin{longtable}{l p{0.5cm} r}
چهار شعر بگفتم بگفت نی به از این
&&
بلی ولیک بده اولا شراب گزین
\\
بده به خمس مبارک مرا ششم جامی
&&
بگو بگیر و درآشام خمس با خمسین
\\
غزال خویش به من ده غزل ز من بستان
&&
نمای چهره شعریت و شعر تازه ببین
\\
خمار شعر نگویم خمار من بشکن
&&
بدان میی که نگنجد در آسمان و زمین
\\
ستیزه روی مرا لطف و دلبری تو کرد
&&
وگر نه سخت ادبناک بودم و مسکین
\\
هزارساله ادب را به یک قدح ببری
&&
خمار عشق تو نگذاشت دیده شرمین
\\
ز سایه تو جهان پر ز لیلی و مجنون
&&
هزار ویسه بسازد هزار گون رامین
\\
وگر نه سایه نمودی جمال وحدت تو
&&
در این جهان نه قران هست آمدی نه قرین
\\
تو آفتابی و جز تو چو سایه تابع توست
&&
گهی رود به شمال و گهی دود به یمین
\\
گهی محیط جهان و گهی به کل فانی
&&
به دست توست مسخر چو مهره تکوین
\\
جمال و حسن تو ساکن چو عشق ما پیچان
&&
جبین هجر تو بی‌چین چو سفره ما پرچین
\\
سکون حسن عجبتر که بی‌قراری ما
&&
و باز از این دو عجبتر چو سر کنی ز کمین
\\
\end{longtable}
\end{center}
