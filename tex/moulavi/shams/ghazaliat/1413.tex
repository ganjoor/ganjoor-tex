\begin{center}
\section*{غزل شماره ۱۴۱۳: کشید این دل گریبانم به سوی کوی آن یارم}
\label{sec:1413}
\addcontentsline{toc}{section}{\nameref{sec:1413}}
\begin{longtable}{l p{0.5cm} r}
کشید این دل گریبانم به سوی کوی آن یارم
&&
در آن کویی که می خوردم گرو شد کفش و دستارم
\\
ز عقل خود چو رفتم من سر زلفش گرفتم من
&&
کنون در حلقه زلفش گرفتارم گرفتارم
\\
چو هر دم می فزون باشد ببین حالم که چون باشد
&&
چنان می‌های صدساله چنین عقلی که من دارم
\\
بگوید در چنان مستی نهان کن سر ز من رستی
&&
مسلمانان در آن حالت چه پنهان ماند اسرارم
\\
مرا می گوید آن دلبر که از عاشق فنا خوشتر
&&
نگارا چند بشتابی نه آخر اندر این کارم
\\
چو ابر نوبهاری من چه خوش گریان و خندانم
&&
از آن می‌های کاری من چه خوش بی‌هوش هشیارم
\\
چو عنقا کوه قافی را تو پران بینی از عشقش
&&
اگر آن که خبر یابد ز لعل یار عیارم
\\
منم چو آسمان دوتو ز عشق شمس تبریزی
&&
بزن تو زخمه آهسته که تا برنسکلد تارم
\\
\end{longtable}
\end{center}
