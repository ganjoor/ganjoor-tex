\begin{center}
\section*{غزل شماره ۲۳۷۱: کی بود خاک صنم با خون ما آمیخته}
\label{sec:2371}
\addcontentsline{toc}{section}{\nameref{sec:2371}}
\begin{longtable}{l p{0.5cm} r}
کی بود خاک صنم با خون ما آمیخته
&&
خوش بود این جسم‌ها با جان‌ها آمیخته
\\
این صدف‌های دل ما با چنین درد فراق
&&
با گهرهای صفای باوفا آمیخته
\\
روز و شب با هم نشسته آب و آتش هم قرین
&&
لطف و قهری جفت و دردی با صفا آمیخته
\\
وصل و هجران صلح کرده کفر ایمان یک شده
&&
بوی وصل شاه ما اندر صبا آمیخته
\\
گرگ یوسف خلق گشته گرگی از وی گم شده
&&
بوی پیراهن رسیده با عما آمیخته
\\
خاک خاکی ترک کرده تیرگی از وی شده
&&
آب همچون باده با نور صفا آمیخته
\\
شادیا روزی که آن معشوق جان‌های لقا
&&
آمده در بزم مست و با شما آمیخته
\\
مست کرده جمله را زان غمزه مخمور خویش
&&
تا ز مستی اجنبی با آشنا آمیخته
\\
تا ز بسیاری شراب ابلیس چون آدم شده
&&
لعنت ابلیس هم با اصطفا آمیخته
\\
آن در بسته ابد بگشاده از مفتاح لطف
&&
قفل‌های بی‌وفایی با وفا آمیخته
\\
سر سر شمس دین مخدوم ما پیدا شده
&&
تا ببینی بنده با وصف خدا آمیخته
\\
ای خداوند شمس دین فریاد از این حرف رهی
&&
ز آنک هر حرفی از این با اژدها آمیخته
\\
یک دمی مهلت دهم تا پستتر گیرم سخن
&&
ز آنک تند است این سخن با کبریا آمیخته
\\
در ره عشاق حضرت گو که از هر محنتش
&&
صد هزاران لطف باشد با بلا آمیخته
\\
قطره زهر و هزاران تنگ تریاق شفا
&&
نفخه عیسی دولت با وبا آمیخته
\\
خواری آن جا با عزیزی عهد بسته یک شده
&&
پستی آن جا از طبیعت با علا آمیخته
\\
جان بود ارزان به نرخ خاک پیش جان جان
&&
گر چه این جا هست جان‌ها با غلا آمیخته
\\
از پی آن جان جان جان‌ها چنان گوهر شده
&&
مس جان با جان جان چون کیمیا آمیخته
\\
آخر دور جهان با اولش یک سر شده
&&
ابتدای ابتدا با انتها آمیخته
\\
در سرای بخت رو یعنی که تبریز صفا
&&
تا ببینی این سرا با آن سرا آمیخته
\\
\end{longtable}
\end{center}
