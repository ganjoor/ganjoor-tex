\begin{center}
\section*{غزل شماره ۳۹۴: چشمه‌ای خواهم که از وی جمله را افزایش است}
\label{sec:0394}
\addcontentsline{toc}{section}{\nameref{sec:0394}}
\begin{longtable}{l p{0.5cm} r}
چشمه‌ای خواهم که از وی جمله را افزایش است
&&
دلبری خواهم که از وی مرده را آسایش است
\\
بنده بحر محیطم کز محیطی برتر است
&&
سنگ و گوهر هر دو را از فضل او بخشایش است
\\
باغ و طاووسند هر یک از جمالش بانصیب
&&
زاغ را خالی ندارد گر چه بی‌آرایش است
\\
صورت ار نقصان پذیرد نیست معنی را کمی
&&
عاشق اندر ذوق باشد گر چه در پالایش است
\\
بنگر اندر جان که هست او از بلندی بی‌خبر
&&
گر چه اندر قالب او در خانه آلایش است
\\
شمس تبریزی قدومت خانه اقبال را
&&
صحن را افروزش است و بام را اندایش است
\\
\end{longtable}
\end{center}
