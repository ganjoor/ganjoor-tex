\begin{center}
\section*{غزل شماره ۱۲۸۵: شکست نرخ شکر را بتم به روی ترش}
\label{sec:1285}
\addcontentsline{toc}{section}{\nameref{sec:1285}}
\begin{longtable}{l p{0.5cm} r}
شکست نرخ شکر را بتم به روی ترش
&&
چه باده‌هاست بتم را در آن کدوی ترش
\\
به قاصد او ترشست و به جان شیرینش
&&
که نیست در همه اجزاش تای موی ترش
\\
هزار خمره سرکه عسل شدست از او
&&
که هست دلبر شیرین دوای خوی ترش
\\
زهای و هوی ترش‌های ماش خنده گرفت
&&
حلاوت عجبی یافت‌های و هوی ترش
\\
ترش چگونه نخندد به زیر لب چو شنید
&&
که جوی شیر و شکر شد روان به سوی ترش
\\
ربود سیل ویم دوش و خلق نعره زنان
&&
میان جوی عسل چیست آن سبوی ترش
\\
پریر یار مرا جست کان ترش رو کو
&&
خمار نیست چرا بودش آرزوی ترش
\\
شتاب و تیز همی‌رفت کو به کو پی من
&&
چرا کند شکرقند جست و جوی ترش
\\
گرفته طبله حلوا و بنده را جویان
&&
که تا ز جایزه شیرین کند گلوی ترش
\\
عجب نباشد اگر قصد او فنای منست
&&
همیشه شیرین باشد یقین عدوی ترش
\\
غلط مکن ترشی نی برای دفع توست
&&
ز رشک چون تو شکاریست رنگ و بوی ترش
\\
ز رشک جاه امیرست روترش دربان
&&
ز رشک روی عروس است روی شوی ترش
\\
هزار خانه چو زنبور پرعسل داری
&&
به جان تو که گذر کن ز گفت و گوی ترش
\\
\end{longtable}
\end{center}
