\begin{center}
\section*{غزل شماره ۲۲۶۲: پرده بگردان و بزن ساز نو}
\label{sec:2262}
\addcontentsline{toc}{section}{\nameref{sec:2262}}
\begin{longtable}{l p{0.5cm} r}
پرده بگردان و بزن ساز نو
&&
هین که رسید از فلک آواز نو
\\
تازه و خندان نشود گوش و هوش
&&
تا ز خرد درنرسد راز نو
\\
این بکند زهره که چون ماه دید
&&
او بزند چنگ طرب ساز نو
\\
خیز سبک رطل گران را بیار
&&
تا ببرم شرم ز هنباز نو
\\
برجه ساقی طرب آغاز کن
&&
وز می کهنه بنه آغاز نو
\\
در عوض آنک گزیدی رخم
&&
بوسه بده بر سر این گاز نو
\\
از تو رخ همچو زرم گاز یافت
&&
می‌رسدم گر بکنم ناز نو
\\
چون نکنم ناز که پنهان و فاش
&&
می‌رسدم خلعت و اعزاز نو
\\
خلعت نو بین که به هر گوشه‌اش
&&
تازه طرازی است ز طراز نو
\\
پر همایی بگشا در وفا
&&
بر سر عشاق به پرواز نو
\\
مرد قناعت که کرم‌های تو
&&
حرص دهد هر نفس و آز نو
\\
می به سبو ده که به تو تشنه شد
&&
این قنق خابیه پرداز نو
\\
رنگ رخ و اشک روانم بس است
&&
سر مرا هر یک غماز نو
\\
گرم درآ گرم که آن گرمدار
&&
صنعت نو دارد و انگاز نو
\\
بس کن کاین گفت تو نسبت به عشق
&&
جامه کهنه‌ست ز بزاز نو
\\
\end{longtable}
\end{center}
