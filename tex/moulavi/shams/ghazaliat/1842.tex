\begin{center}
\section*{غزل شماره ۱۸۴۲: چهره شرمگین تو بستد شرمگان من}
\label{sec:1842}
\addcontentsline{toc}{section}{\nameref{sec:1842}}
\begin{longtable}{l p{0.5cm} r}
چهره شرمگین تو بستد شرمگان من
&&
شور تو کرد عاقبت فتنه و شر مکان من
\\
مه که نشانده تو است لابه کنان به پیش تو
&&
پیش خودم نشان دمی ای شه خوش نشان من
\\
در ره تو کمین خسم از ره دور می رسم
&&
ای دل من به دست تو بشنو داستان من
\\
گرد فلک همی‌دوم پر و تهی همی‌شوم
&&
زانک قرار برده‌ای ای دل و جان ز جان من
\\
گرد تو گشتمی ولی گرد کجاست مر تو را
&&
گرد در تو می دوم ای در تو امان من
\\
عشق برید ناف من بر تو بود طواف من
&&
لاف من و گزاف من پیش تو ترجمان من
\\
گه همه لعل می شوم گاه چو نعل می شوم
&&
تا کرمت بگویدم باز درآ به کان من
\\
گفت مرا که چند چند سیر نگشتی از سخن
&&
زانک سوی تو می رود این سخن روان من
\\
\end{longtable}
\end{center}
