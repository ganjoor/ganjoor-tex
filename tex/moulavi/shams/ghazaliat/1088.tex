\begin{center}
\section*{غزل شماره ۱۰۸۸: سر فروکن به سحر کز سر بازار نظر}
\label{sec:1088}
\addcontentsline{toc}{section}{\nameref{sec:1088}}
\begin{longtable}{l p{0.5cm} r}
سر فروکن به سحر کز سر بازار نظر
&&
طبله کالبد آورده‌ام آخر بنگر
\\
بر سر کوی تو پرطبله من بین و بخر
&&
شانه‌ها و شبه‌ها و سره روغن‌ها تر
\\
شبه من غم تو روغن من مرهم تو
&&
شانه‌ام محرم آن زلف پر از فتنه و شر
\\
از فراقت تلفم گشته خیالت علفم
&&
که دلم را شکمی شد ز تو پرجوع بقر
\\
من ندانم چه کسم کز شکرت پرهوسم
&&
ای مگس‌ها شده از ذوق شکرهات شکر
\\
پرده بردار صبا از بر آن شهره قبا
&&
تا ز سیمین بر او گردد کارم همه زر
\\
چند گویی تو بجو یار وزو دست بشو
&&
در دو عالم نبود یار مرا یار دگر
\\
چون خرد ماند و دل با من ای خواجه بهل
&&
ماه و خورشید که دیدست در اعضای بشر
\\
چون که در جان منی شسته به چشمان منی
&&
شمس تبریز خداوند تو چونی به سفر
\\
\end{longtable}
\end{center}
