\begin{center}
\section*{غزل شماره ۱۱۴۶: مرا بگاه ده ای ساقی کریم عقار}
\label{sec:1146}
\addcontentsline{toc}{section}{\nameref{sec:1146}}
\begin{longtable}{l p{0.5cm} r}
مرا بگاه ده ای ساقی کریم عقار
&&
که دوش هیچ نخفتم ز تشنگی و خمار
\\
لبم که نام تو گوید به باده‌اش خوش کن
&&
سرم خمار تو دارد به مستیش تو بخار
\\
بریز باده بر اجسامم و بر اعراضم
&&
چنانک هیچ نماند ز من رگی هشیار
\\
وگر خراب شوم من بود رگی باقی
&&
چو جغد هل که بگردد در این خراب دیار
\\
چو لاله زار کن این دشت را به باده لعل
&&
روا مدار که موقوف داریم به بهار
\\
ز توست این شجره و خرقه‌اش تو دادستی
&&
که از شراب تو اشکوفه کرده‌اند اشجار
\\
مرا چو مست کنی زین شجر برآرم سر
&&
به خنده دل بنمایم به خلق همچو انار
\\
مرا چو وقف خرابات خویش کردستی
&&
توام خراب کنی هم تو باشیم معمار
\\
بیار رطل گران تا خمش کنم پی آن
&&
نه لایقست که باشد غلام تو مکثار
\\
\end{longtable}
\end{center}
