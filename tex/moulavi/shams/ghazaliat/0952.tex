\begin{center}
\section*{غزل شماره ۹۵۲: ز شمس دین طرب نوبهار بازآید}
\label{sec:0952}
\addcontentsline{toc}{section}{\nameref{sec:0952}}
\begin{longtable}{l p{0.5cm} r}
ز شمس دین طرب نوبهار بازآید
&&
نشاط بلبله و سبزه زار بازآید
\\
کرانه کرد دلم از نبیذ و از ساقی
&&
چو وصل او بگشاید کنار بازآید
\\
کبوتر دل من در شکار باز پرید
&&
خنک زمانی کو از شکار بازآید
\\
بگردد این رخ زردم چو صد هزار نگار
&&
ز طبل دعوت من گر نگار بازآید
\\
چو ملک حسن بر وی مهم قرار گرفت
&&
بود که سوی دلم زو قرار بازآید
\\
چو خارخار دلم می‌نشیند از هوسش
&&
که گلشنش بر این خار خار بازآید
\\
چو مهرها که شود محو نطع آن گوهر
&&
دغای عشق چو خانه قمار بازآید
\\
ز مستی‌اش چه گمان بردمی که بعد از می
&&
ز هجر عربده کن آن خمار بازآید
\\
از این خمار مرا نیست غم اگر روزی
&&
به دستم آن قدح پرشرار بازآید
\\
هزار چشمه حیوان چه در شمار آید
&&
اگر از او لطف بی‌شمار بازآید
\\
سؤال کردم رخ را که چند زر باشی
&&
که جان من ز زری تو زار بازآید
\\
مرا جواب چو زر داد من زرم دایم
&&
مگر که سیمبر خوش عیار بازآید
\\
بگفتمش چو بماندی تو زنده بی آن جان
&&
چه عذر آری چون آن عذار بازآید
\\
من آن ندانم دانم که آه از تبریز
&&
کز آتشش ز دلم الحذار بازآید
\\
\end{longtable}
\end{center}
