\begin{center}
\section*{غزل شماره ۲۱۳۱: حیلت رها کن عاشقا دیوانه شو دیوانه شو}
\label{sec:2131}
\addcontentsline{toc}{section}{\nameref{sec:2131}}
\begin{longtable}{l p{0.5cm} r}
حیلت رها کن عاشقا دیوانه شو دیوانه شو
&&
و اندر دل آتش درآ پروانه شو پروانه شو
\\
هم خویش را بیگانه کن هم خانه را ویرانه کن
&&
وآنگه بیا با عاشقان هم خانه شو هم خانه شو
\\
رو سینه را چون سینه‌ها هفت آب شو از کینه‌ها
&&
وآنگه شراب عشق را پیمانه شو پیمانه شو
\\
باید که جمله جان شوی تا لایق جانان شوی
&&
گر سوی مستان می‌روی مستانه شو مستانه شو
\\
آن گوشوار شاهدان هم صحبت عارض شده
&&
آن گوش و عارض بایدت دردانه شو دردانه شو
\\
چون جان تو شد در هوا ز افسانه شیرین ما
&&
فانی شو و چون عاشقان افسانه شو افسانه شو
\\
تو لیلة القبری برو تا لیلة القدری شوی
&&
چون قدر مر ارواح را کاشانه شو کاشانه شو
\\
اندیشه‌ات جایی رود وآنگه تو را آن جا کشد
&&
ز اندیشه بگذر چون قضا پیشانه شو پیشانه شو
\\
قفلی بود میل و هوا بنهاده بر دل‌های ما
&&
مفتاح شو مفتاح را دندانه شو دندانه شو
\\
بنواخت نور مصطفی آن استن حنانه را
&&
کمتر ز چوبی نیستی حنانه شو حنانه شو
\\
گوید سلیمان مر تو را بشنو لسان الطیر را
&&
دامی و مرغ از تو رمد رو لانه شو رو لانه شو
\\
گر چهره بنماید صنم پر شو از او چون آینه
&&
ور زلف بگشاید صنم رو شانه شو رو شانه شو
\\
تا کی دوشاخه چون رخی تا کی چو بیذق کم تکی
&&
تا کی چو فرزین کژ روی فرزانه شو فرزانه شو
\\
شکرانه دادی عشق را از تحفه‌ها و مال‌ها
&&
هل مال را خود را بده شکرانه شو شکرانه شو
\\
یک مدتی ارکان بدی یک مدتی حیوان بدی
&&
یک مدتی چون جان شدی جانانه شو جانانه شو
\\
ای ناطقه بر بام و در تا کی روی در خانه پر
&&
نطق زبان را ترک کن بی‌چانه شو بی‌چانه شو
\\
\end{longtable}
\end{center}
