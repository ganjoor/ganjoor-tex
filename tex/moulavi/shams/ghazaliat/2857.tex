\begin{center}
\section*{غزل شماره ۲۸۵۷: چه جمال جان فزایی که میان جان مایی}
\label{sec:2857}
\addcontentsline{toc}{section}{\nameref{sec:2857}}
\begin{longtable}{l p{0.5cm} r}
چه جمال جان فزایی که میان جان مایی
&&
تو به جان چه می‌نمایی تو چنین شکر چرایی
\\
چو بدان تو راه یابی چو هزار مه بتابی
&&
تو چه آتش و چه آبی تو چنین شکر چرایی
\\
غم عشق تو پیاده شده قلعه‌ها گشاده
&&
به سپاه نور ساده تو چنین شکر چرایی
\\
همه زنگ را شکسته شده دست جمله بسته
&&
شه چین بس خجسته تو چنین شکر چرایی
\\
تو چراغ طور سینا تو هزار بحر و مینا
&&
بجز از تو جان مبینا تو چنین شکر چرایی
\\
تو برسته از فزونی ز قیاس‌ها برونی
&&
به دو چشم مست خونی تو چنین شکر چرایی
\\
به دلم چه آذر آمد چو خیال تو درآمد
&&
دو جهان به هم برآمد تو چنین شکر چرایی
\\
تو در آن دو رخ چه داری که فکندی از عیاری
&&
دو هزار بی‌قراری تو چنین شکر چرایی
\\
چو بدان لطیف خنده همه را بکرده بنده
&&
ز دم تو مرده زنده تو چنین شکر چرایی
\\
چو صفات حسن ایزد عرقت به بحر ریزد
&&
دو هزار موج خیزد تو چنین شکر چرایی
\\
چو دو زلف توست طوقم ز شراب توست شوقم
&&
بنگر که در چه ذوقم تو چنین شکر چرایی
\\
ز گلت سمن فنا شد همه مکر و فن فنا شد
&&
من و صد چو من فنا شد تو چنین شکر چرایی
\\
\end{longtable}
\end{center}
