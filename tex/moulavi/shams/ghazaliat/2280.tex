\begin{center}
\section*{غزل شماره ۲۲۸۰: ای عاشقان ای عاشقان دیوانه‌ام کو سلسله}
\label{sec:2280}
\addcontentsline{toc}{section}{\nameref{sec:2280}}
\begin{longtable}{l p{0.5cm} r}
ای عاشقان ای عاشقان دیوانه‌ام کو سلسله
&&
ای سلسله جنبان جان عالم ز تو پرغلغله
\\
زنجیر دیگر ساختی در گردنم انداختی
&&
وز آسمان درتاختی تا رهزنی بر قافله
\\
برخیز ای جان از جهان برپر ز خاک خاکدان
&&
کز بهر ما بر آسمان گردان شده‌ست این مشعله
\\
آن را که باشد درد دل کی رهزند باران گل
&&
از عشق باشد او بحل کو را نشد که خردله
\\
روزی مخنث بانگ زد گفتا که ای چوبان بد
&&
آن بز عجب ما را گزد در من نظر کرد از گله
\\
گفتا مخنث را گزد هم بکشدش زیر لگد
&&
اما چه غم زو مرد را گفتا نکو گفتی هله
\\
کو عقل تا گویا شوی کو پای تا پویا شوی
&&
وز خشک در دریا شوی ایمن شوی از زلزله
\\
سلطان سلطانان شوی در ملک جاویدان شوی
&&
بالاتر از کیوان شوی بیرون شوی زین مزبله
\\
چون عقل کل صاحب عمل جوشان چو دریای عسل
&&
چون آفتاب اندر حمل چون مه به برج سنبله
\\
صد زاغ و جغد و فاخته در تو نواها ساخته
&&
بشنیدیی اسرار دل گر کم شدی این مشغله
\\
بی‌دل شو ار صاحب دلی دیوانه شو گر عاقلی
&&
کاین عقل جزوی می‌شود در چشم عشقت آبله
\\
تا صورت غیبی رسد وز صورتت بیرون کشد
&&
کز جعد پیچاپیچ او مشکل شده‌ست این مسله
\\
اما در این راه از خوشی باید که دامن برکشی
&&
زیرا ز خون عاشقان آغشته‌ست این مرحله
\\
رو رو دلا با قافله تنها مرو در مرحله
&&
زیرا که زاید فتنه‌ها این روزگار حامله
\\
از رنج‌ها مطلق روی اندر امان حق روی
&&
در بحر چون زورق روی رفتی دلا رو بی‌گله
\\
چون دل ز جان برداشتی رستی ز جنگ و آشتی
&&
آزاد و فارغ گشته‌ای هم از دکان هم از غله
\\
ز اندیشه جانت رسته شد راه خطرها بسته شد
&&
آن کو به تو پیوسته شد پیوسته باشد در چله
\\
در روز چون ایمن شدی زین رومی باعربده
&&
شب هم مکن اندیشه‌ای زین زنگی پرزنگله
\\
خامش کن ای شیرین لقا رو مشک بربند ای سقا
&&
زیرا نگنجد موج‌ها اندر سبو و بلبله
\\
\end{longtable}
\end{center}
