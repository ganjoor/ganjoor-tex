\begin{center}
\section*{غزل شماره ۷۵۱: هر زمان کز غیب عشق یار ما خنجر کشد}
\label{sec:0751}
\addcontentsline{toc}{section}{\nameref{sec:0751}}
\begin{longtable}{l p{0.5cm} r}
هر زمان کز غیب عشق یار ما خنجر کشد
&&
گر بخواهم ور نخواهم او مرا اندرکشد
\\
همچو پره و قفل من چون جفت گردم با کسی
&&
همچو مرغ کشته آن دم پرم از من برکشد
\\
کفر و دین عاشقانش هم رقوم عشق اوست
&&
حاش لله کان رقم بر طایفه دیگر کشد
\\
چون گشاید باگشادم چون ببندد بسته‌ام
&&
گوی میدان خود کی باشد تا ز چوگان سر کشد
\\
همچو ابراهیم گاهم جانب آتش برد
&&
همچو احمد گاهم از آتش سوی کوثر کشد
\\
گویی آتش خوشتر آید مر تو را یا کوثرش
&&
خوشترم آنست کان سلطان مرا خوشتر کشد
\\
آب و آتش خوشتر آمد رنج و راحت داد اوست
&&
زین سبب‌ها ساخت تا بر دیده‌ها چادر کشد
\\
دوست را دشمن نماید آب را آتش کند
&&
مؤمنی را ناگهان در حلقه کافر کشد
\\
سرخوشان و سرکشان را عشق او بند و گشاست
&&
سرکشان را موکشان آن عشق در چنبر کشد
\\
بر حذر باید بدن گر چه حذر هم داد اوست
&&
آن حذر او داد کز بهر بچه مادر کشد
\\
\end{longtable}
\end{center}
