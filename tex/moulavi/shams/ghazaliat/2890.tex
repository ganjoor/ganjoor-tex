\begin{center}
\section*{غزل شماره ۲۸۹۰: سخن تلخ مگو ای لب تو حلوایی}
\label{sec:2890}
\addcontentsline{toc}{section}{\nameref{sec:2890}}
\begin{longtable}{l p{0.5cm} r}
سخن تلخ مگو ای لب تو حلوایی
&&
سر فروکن به کرم ای که بر این بالایی
\\
هر چه گویی تو اگر تلخ و اگر شور خوش است
&&
گوهر دیده و دل جانی و جان افزایی
\\
نه به بالا نه به زیری و نه جان در جهت است
&&
شش جهت را چه کنم در دل خون پالایی
\\
سر فروکن که از آن روز که رویت دیدم
&&
دل و جان مست شد و عقل و خرد سودایی
\\
هر کی او عاشق جسم است ز جان محروم است
&&
تلخ آید شکر اندر دهن صفرایی
\\
ای که خورشید تو را سجده کند هر شامی
&&
کی بود کز دل خورشید به بیرون آیی
\\
آفتابی که ز هر ذره طلوعی داری
&&
کوه‌ها را جهت ذره شدن می‌سایی
\\
چه لطیفی و ز آغاز چنان جباری
&&
چه نهانی و عجب این که در این غوغایی
\\
گر خطا گفتم و مقلوب و پراکنده مگیر
&&
ور بگیری تو مرا بخت نوم افزایی
\\
صورت عشق تویی صورت ما سایه تو
&&
یک دمم زشت کنی باز توام آرایی
\\
می‌نماید که مگر دوش به خوابت دیدم
&&
که من امروز ندارم به جهان گنجایی
\\
ساربانا بمخوابان شتر این منزل نیست
&&
همرهان پیش شدستند که را می‌پایی
\\
هین خمش کن که ز دم آتش دل شعله زند
&&
شعله دم می‌زند این دم تو چه می‌فرمایی
\\
شمس تبریز چو در شمس فلک درتابد
&&
تابش روز شود از وی نابینایی
\\
\end{longtable}
\end{center}
