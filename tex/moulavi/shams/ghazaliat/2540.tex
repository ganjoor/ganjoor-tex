\begin{center}
\section*{غزل شماره ۲۵۴۰: چو شیر و انگبین جانا چه باشد گر درآمیزی}
\label{sec:2540}
\addcontentsline{toc}{section}{\nameref{sec:2540}}
\begin{longtable}{l p{0.5cm} r}
چو شیر و انگبین جانا چه باشد گر درآمیزی
&&
عسل از شیر نگریزد تو هم باید که نگریزی
\\
اگر نالایقم جانا شوم لایق به فر تو
&&
وگر ناچیز و معدومم بیابم از تو من چیزی
\\
یکی قطره شود گوهر چو یابد او علف از تو
&&
که قافی شود ذره چو دربندی و بستیزی
\\
همه خاکیم روینده ز آب ذکر و باد دم
&&
گلی که خندد و گرید کز او فکری بینگیزی
\\
گلستانی کنش خندان و فرمانی به دستش ده
&&
که ای گلشن شدی ایمن ز آفت‌های پاییزی
\\
گهی در صورت آبی بیایی جان دهی گل را
&&
گهی در صورت بادی به هر شاخی درآویزی
\\
درختی بیخ او بالا نگونه شاخه‌های او
&&
به عکس آن درختانی که سعدی‌اند و شونیزی
\\
گهی گویی به گوش دل که در دوغ من افتادی
&&
منم جان همه عالم تو چون از جان بپرهیزی
\\
گهی زانوت بربندم چو اشتر تا فروخسپی
&&
گهی زانوت بگشایم که تا از جای برخیزی
\\
منال ای اشتر و خامش به من بنگر به چشم هش
&&
که تمییز نوت بخشم اگر چه کان تمییزی
\\
تویی شمع و منم آتش چو افتم در دماغت خوش
&&
یکی نیمه فروسوزی یکی نیمه فروریزی
\\
به هر سوزی چو پروانه مشو قانع بسوزان سر
&&
به پیش شمع چون لافی این سودای دهلیزی
\\
اگر داری سر مستان کله بگذار و سر بستان
&&
کله دارند و سرها نی کلهداران پالیزی
\\
سر آن‌ها راست که با او درآوردند سر با سر
&&
کم از خاری که زد با گل ز چالاکی و سرتیزی
\\
تو هر چیزی که می‌جویی مجویش جز ز کان او
&&
که از زر هم زری یابند و از ارزیز ارزیزی
\\
خمش کن قصه عمری به روزی کی توان گفتن
&&
کجا آید ز یک خشتک گریبانی و تیریزی
\\
\end{longtable}
\end{center}
