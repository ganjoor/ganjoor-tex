\begin{center}
\section*{غزل شماره ۷۶: آخر بشنید آن مه آه سحر ما را}
\label{sec:0076}
\addcontentsline{toc}{section}{\nameref{sec:0076}}
\begin{longtable}{l p{0.5cm} r}
آخر بشنید آن مه آه سحر ما را
&&
تا حشر دگر آمد امشب حشر ما را
\\
چون چرخ زند آن مه در سینه من گویم
&&
ای دور قمر بنگر دور قمر ما را
\\
کو رستم دستان تا دستان بنماییمش
&&
کو یوسف تا بیند خوبی و فر ما را
\\
تو لقمه شیرین شو در خدمت قند او
&&
لقمه نتوان کردن کان شکر ما را
\\
ما را کرمش خواهد تا در بر خود گیرد
&&
زین روی دوا سازد هر لحظه گر ما را
\\
چون بی‌نمکی نتوان خوردن جگر بریان
&&
می‌زن به نمک هر دم بریان جگر ما را
\\
بی پای طواف آریم بی‌سر به سجود آییم
&&
چون بی‌سر و پا کرد او این پا و سر ما را
\\
بی پای طواف آریم گرد در آن شاهی
&&
کو مست الست آمد بشکست در ما را
\\
چون زر شد رنگ ما از سینه سیمینش
&&
صد گنج فدا بادا این سیم و زر ما را
\\
در رنگ کجا آید در نقش کجا گنجد
&&
نوری که ملک سازد جسم بشر ما را
\\
تشبیه ندارد او وز لطف روا دارد
&&
زیرا که همی‌داند ضعف نظر ما را
\\
فرمود که نور من ماننده مصباح است
&&
مشکات و زجاجه گفت سینه و بصر ما را
\\
خامش کن تا هر کس در گوش نیارد این
&&
خود کیست که دریابد او خیر و شر ما را
\\
\end{longtable}
\end{center}
