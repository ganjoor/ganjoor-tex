\begin{center}
\section*{غزل شماره ۲۵۰: هین که منم بر در در برگشا}
\label{sec:0250}
\addcontentsline{toc}{section}{\nameref{sec:0250}}
\begin{longtable}{l p{0.5cm} r}
هین که منم بر در در برگشا
&&
بستن در نیست نشان رضا
\\
در دل هر ذره تو را درگهیست
&&
تا نگشایی بود آن در خفا
\\
فالق اصباحی و رب الفلق
&&
باز کنی صد در و گویی درآ
\\
نی که منم بر در بلک توی
&&
راه بده در بگشا خویش را
\\
آمد کبریت بر آتشی
&&
گفت برون آ بر من دلبرا
\\
صورت من صورت تو نیست لیک
&&
جمله توام صورت من چون غطا
\\
صورت و معنی تو شوم چون رسی
&&
محو شود صورت من در لقا
\\
آتش گفتش که برون آمدم
&&
از خود خود روی بپوشم چرا
\\
هین بستان از من تبلیغ کن
&&
بر همه اصحاب و همه اقربا
\\
کوه اگر هست چو کاهش بکش
&&
داده امت من صفت کهربا
\\
کاه ربای من که می‌کشد
&&
نه از عدم آوردم کوه حرا
\\
در دل تو جمله منم سر به سر
&&
سوی دل خویش بیا مرحبا
\\
دلبرم و دل برم ایرا که هست
&&
جوهر دل زاده ز دریای ما
\\
نقل کنم ور نکنم سایه را
&&
سایه من کی بود از من جدا
\\
لیک ز جایش ببرم تا شود
&&
وصلت او ظاهر وقت جلا
\\
تا که بداند که او فرع ماست
&&
تا که جدا گردد او از عدا
\\
رو بر ساقی و شنو باقیش
&&
تات بگوید به زبان بقا
\\
\end{longtable}
\end{center}
