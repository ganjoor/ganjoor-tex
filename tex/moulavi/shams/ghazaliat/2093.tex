\begin{center}
\section*{غزل شماره ۲۰۹۳: تازه شد از او باغ و بر من}
\label{sec:2093}
\addcontentsline{toc}{section}{\nameref{sec:2093}}
\begin{longtable}{l p{0.5cm} r}
تازه شد از او باغ و بر من
&&
شاخ گل من نیلوفر من
\\
گشته است روان در جوی وفا
&&
آب حیوان از کوثر من
\\
ای روی خوشت دین و دل من
&&
ای بوی خوشت پیغامبر من
\\
هر لحظه مرا در پیش رخت
&&
آیینه کند آهنگر من
\\
من خشک لبم من چشم ترم
&&
این است مها خشک و تر من
\\
آن کس که منم خاک در او
&&
می‌کوبد او بام و در من
\\
آن کس که منم پابسته او
&&
می‌گردد او گرد سر من
\\
باده نخورم ور ز آنک خورم
&&
او بوسه دهد بر ساغر من
\\
پستان وفا کی کرد سیه
&&
آن دایه جان آن مادر من
\\
از من دو جهان صد بر بخورد
&&
چون آید او اندر بر من
\\
دزدار فلک قلعه بدهد
&&
چون گردد او سرلشکر من
\\
بربند دهان غماز مشو
&&
غماز بس است آن گوهر من
\\
\end{longtable}
\end{center}
