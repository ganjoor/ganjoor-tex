\begin{center}
\section*{غزل شماره ۳۳۵: همه خوف آدمی را از درونست}
\label{sec:0335}
\addcontentsline{toc}{section}{\nameref{sec:0335}}
\begin{longtable}{l p{0.5cm} r}
همه خوف آدمی را از درونست
&&
ولیکن هوش او دایم برونست
\\
برون را می‌نوازد همچو یوسف
&&
درون گرگی‌ست کو در قصد خونست
\\
بدرد زهره او گر نبیند
&&
درون را کو به زشتی شکل چونست
\\
بدان زشتی به یک حمله بمیرد
&&
ولیکن آدمی او را زبونست
\\
الف گشت‌ست نون می‌بایدش ساخت
&&
که تا گردد الف چیزی که نونست
\\
اگر نه خود عنایات خداوند
&&
بدیدستی چه امکان سکون‌ست
\\
نه عالم بد نه آدم بد نه روحی
&&
که صافی و لطیف و آبگون‌ست
\\
که او را بود حکم و پادشاهی
&&
نپنداری که این کار از کنونست
\\
نمی‌گویم که در تقدیر شه بود
&&
حقیقت بود و صد چندین فزونست
\\
خداوندی شمس الدین تبریز
&&
ورای هفت چرخ نیلگونست
\\
به زیر ران او تقدیر رامست
&&
اگر چه نیک تندست و حرونست
\\
چو عقل کل بویی برد از وی
&&
شب و روز از هوس اندر جنونست
\\
که پیش همت او عقل دیده‌ست
&&
که همت‌های عالی جمله دونست
\\
کدامین سوی جویم خدمتش را
&&
که منزلگاه او بالای سونست
\\
هر آن مشکل که شیران حل نکردند
&&
بر او جمله بازی و فسونست
\\
نگفتم هیچ رمزی تا بدانی
&&
ز عین حال او این‌ها شجونست
\\
ایا تبریز خاک توست کحلم
&&
که در خاکت عجایب‌ها فنونست
\\
\end{longtable}
\end{center}
