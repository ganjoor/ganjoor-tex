\begin{center}
\section*{غزل شماره ۳۰۷۴: مسلم آمد یار مرا دل افروزی}
\label{sec:3074}
\addcontentsline{toc}{section}{\nameref{sec:3074}}
\begin{longtable}{l p{0.5cm} r}
مسلم آمد یار مرا دل افروزی
&&
چه عشق داد مرا فضل حق زهی روزی
\\
اگر سرم برود گو برو مرا سر اوست
&&
رهیدم از کله و از سر و کله دوزی
\\
دهان به گوش من آورد و گفت در گوشم
&&
یکی حدیث بیاموزمت بیاموزی
\\
چو آهوی ختنی خون تو شود همه مشک
&&
اگر دمی بچری تو ز ما به خوش پوزی
\\
چو جان جان شده‌ای ننگ جان و تن چه کشی
&&
چو کان زر شده‌ای حبه‌ای چه اندوزی
\\
به سوی مجلس خوبان بکش حریفان را
&&
به خضر و چشمه حیوان بکن قلاوزی
\\
شراب لعل رسیده‌ست نیست انگوری
&&
شکر نثار شد و نیست این شکر خوزی
\\
هوا و حرص یکی آتشیست تو بازی
&&
بپر گزاف پر و بال را چه می‌سوزی
\\
خمش که خلق ندانند بانگ را ز صدا
&&
تویی که دانی پیروزه را ز پیروزی
\\
\end{longtable}
\end{center}
