\begin{center}
\section*{غزل شماره ۱۰۵۴: ای یار شگرف در همه کار}
\label{sec:1054}
\addcontentsline{toc}{section}{\nameref{sec:1054}}
\begin{longtable}{l p{0.5cm} r}
ای یار شگرف در همه کار
&&
عیاره و عاشق تو عیار
\\
تو روز قیامتی که از تو
&&
زیر و زبرست شهر و بازار
\\
من زاری عاشقان چه گویم
&&
ای معشوقان ز عشق تو زار
\\
در روز اجل چو من بمیرم
&&
در گور مکن مرا نگهدار
\\
ور می‌خواهی که زنده گردیم
&&
ما را به نسیم وصل بسپار
\\
آخر تو کجا و ما کجاییم
&&
ای بی‌تو حیات و عیش بی‌کار
\\
از من رگ جان بریده بادا
&&
گر بی‌تو رگیم هست هشیار
\\
اندر ره تو دو صد کمین بود
&&
نزدیک نمود راه و هموار
\\
از گلشن روی تو شدم مست
&&
بنهادم مست پای بر خار
\\
رفتم سوی دانه تو چون مرغ
&&
پرخون دیدم جناح و منقار
\\
این طرفه که خوشترست زخمت
&&
از هر دانه که دارد انبار
\\
ای بی‌تو حرام زندگانی
&&
ای بی‌تو نگشته بخت بیدار
\\
خود بخت تویی و زندگی تو
&&
باقی نامی و لاف و آزار
\\
ای کرده ز دل مرا فراموش
&&
آخر چه شود مرا به یاد آر
\\
یک بار چو رفت آب در جوی
&&
کی گردد چرخ طمع یک بار
\\
خامش که ستیزه می‌فزاید
&&
آن خواجه عشق را ز گفتار
\\
\end{longtable}
\end{center}
