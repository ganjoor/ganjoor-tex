\begin{center}
\section*{غزل شماره ۱۳۰۵: کعبه جان‌ها تویی گرد تو آرم طواف}
\label{sec:1305}
\addcontentsline{toc}{section}{\nameref{sec:1305}}
\begin{longtable}{l p{0.5cm} r}
کعبه جان‌ها تویی گرد تو آرم طواف
&&
جغد نیم بر خراب هیچ ندارم طواف
\\
پیشه ندارم جز این کار ندارم جز این
&&
چون فلکم روز و شب پیشه و کارم طواف
\\
بهتر از این یار کیست خوشتر از این کار چیست
&&
پیش بت من سجود گرد نگارم طواف
\\
رخت کشیدم به حج تا کنم آن جا قرار
&&
برد عرب رخت من برد قرارم طواف
\\
تشنه چه بیند به خواب چشمه و حوض و سبو
&&
تشنه وصل توام کی بگذارم طواف
\\
چونک برآرم سجود بازرهم از وجود
&&
کعبه شفیعم شود چونک گزارم طواف
\\
حاجی عاقل طواف چند کند هفت هفت
&&
حاجی دیوانه‌ام من نشمارم طواف
\\
گفتم گل را که خار کیست ز پیشش بران
&&
گفت بسی کرد او گرد عذارم طواف
\\
گفت به آتش هوا دود نه درخورد توست
&&
گفت بهل تا کند گرد شرارم طواف
\\
عشق مرا می‌ستود کو همه شب همچو ماه
&&
بر سر و رو می‌کند گرد غبارم طواف
\\
همچو فلک می‌کند بر سر خاکم سجود
&&
همچو قدح می‌کند گرد خمارم طواف
\\
خواجه عجب نیست اینک من بدوم پیش صید
&&
طرفه که بر گرد من کرد شکارم طواف
\\
چار طبیعت چو چار گردن حمال دان
&&
همچو جنازه مبا بر سر چارم طواف
\\
هست اثرهای یار در دمن این دیار
&&
ور نه نبودی بر این تیره دیارم طواف
\\
عاشق مات ویم تا ببرد رخت من
&&
ور نه نبودی چنین گرد قمارم طواف
\\
سرو بلندم که من سبز و خوشم در خزان
&&
نی چو حشیشم بود گرد بهارم طواف
\\
از سپه رشک ما تیر قضا می‌رسد
&&
تا نکنی بی‌سپر گرد حصارم طواف
\\
خشت وجود مرا خرد کن ای غم چو گرد
&&
تا که کنم همچو گرد گرد سوارم طواف
\\
بس کن و چون ماهیان باش خموش اندر آب
&&
تا نه چو تابه شود بر سر نارم طواف
\\
\end{longtable}
\end{center}
