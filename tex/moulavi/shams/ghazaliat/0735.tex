\begin{center}
\section*{غزل شماره ۷۳۵: دوش آمد پیل ما را باز هندستان به یاد}
\label{sec:0735}
\addcontentsline{toc}{section}{\nameref{sec:0735}}
\begin{longtable}{l p{0.5cm} r}
دوش آمد پیل ما را باز هندستان به یاد
&&
پرده شب می‌درید او از جنون تا بامداد
\\
دوش ساغرهای ساقی جمله مالامال بود
&&
ای که تا روز قیامت عمر ما چون دوش باد
\\
باده‌ها در جوش از او و عقل‌ها بی‌هوش از او
&&
جزو و کل و خار و گل از روی خوبش باد شاد
\\
بانگ نوشانوش مستان تا فلک بررفته بود
&&
بر کف ما باده بود و در سر ما بود باد
\\
در فلک افتاده ز ایشان صد هزاران غلغله
&&
در سجود افتاده آن جا صد هزاران کیقباد
\\
روز پیروزی و دولت در شب ما درج بود
&&
شب ز اخوان صفا ناگه چنین روزی بزاد
\\
موج زد دریا نشانی یافت زین شب آسمان
&&
آن نشان را از تفاخر بر سر و رو می‌نهاد
\\
هر چه ناسوتی ز ظلمت راه‌ها را بسته بود
&&
نور لاهوتی ز رحمت بسته‌ها را می‌گشاد
\\
کی بماند زان هوا اشکال حسی برقرار
&&
چون بماند برقرار آن کس که یابد این مراد
\\
عمر را از سر بگیرید ای مسلمانان که یار
&&
نیستان را هست کرد و عاشقان را داد داد
\\
یار ما افتادگان را زین سپس معذور داشت
&&
زان که هر جا کوست ساقی کس نماند بر سداد
\\
جوش دریای عنایت ای مسلمانان شکست
&&
طمطراق اجتهاد و بارنامه اعتقاد
\\
آن عنایت شه صلاح الدین بود کو یوسفیست
&&
هم عزیز مصر باید مشتریش اندر مزاد
\\
\end{longtable}
\end{center}
