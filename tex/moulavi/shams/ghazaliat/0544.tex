\begin{center}
\section*{غزل شماره ۵۴۴: ای که ز یک تابش تو کوه احد پاره شود}
\label{sec:0544}
\addcontentsline{toc}{section}{\nameref{sec:0544}}
\begin{longtable}{l p{0.5cm} r}
ای که ز یک تابش تو کوه احد پاره شود
&&
چه عجب ار مشت گلی عاشق و بیچاره شود
\\
چونک به لطفش نگری سنگ حجر موم شود
&&
چونک به قهرش نگری موم تو خود خاره شود
\\
نوحه کنی نوحه کنی مرده دل زنده شود
&&
کار کنی کار کنی جان تو این کاره شود
\\
عزم سفر دارد جان می‌نهیش بند گران
&&
برسکلد بند تو را عاقبت آواره شود
\\
چونک سلیمان برود دیو شهنشاه شود
&&
چون برود صبر و خرد نفس تو اماره شود
\\
عشق گرفتست جهان رنگ نبینی تو از او
&&
لیک چو بر تن بزند زردی رخساره شود
\\
شه بچه‌ای باید کو مشتری لعل بود
&&
نادره‌ای باید کو بهر تو غمخواره شود
\\
بشنو از قل خدا هست زمین مهد شما
&&
گر نبود طفل چرا بسته گهواره شود
\\
چون بجهی از غضبش دامن حلمش بکشی
&&
آتش سوزنده تو را لطف و کرم باره شود
\\
گردش این سایه من سخره خورشید حق است
&&
نی چو منجم که دلش سخره استاره شود
\\
\end{longtable}
\end{center}
