\begin{center}
\section*{غزل شماره ۹۱۸: چو کارزار کند شاه روم با شمشاد}
\label{sec:0918}
\addcontentsline{toc}{section}{\nameref{sec:0918}}
\begin{longtable}{l p{0.5cm} r}
چو کارزار کند شاه روم با شمشاد
&&
چگونه گردم خرم چگونه باشم شاد
\\
جهان عقل چو روم و جهان طبع چو زنگ
&&
میان هر دو فتاده‌ست کارزار و جهاد
\\
شما و هر چه مراد شماست در عالم
&&
من و طریق خداوند مبدا و ایجاد
\\
به اختلاف دو شمشیر نیست امن طریق
&&
که اختلاف مقرر ز شورش اضداد
\\
ولیک ملک مقرر نصیبه خردست
&&
که امن و خوف نداند کلوخ و سنگ و جماد
\\
چراغ عقل در این خانه نور می‌ندهد
&&
ز پیچ پیچ که دارد لهب ز یاغی باد
\\
فرشته رست به علم و بهیمه رست به جهل
&&
میان دو به تنازغ بماند مردم زاد
\\
گهی همی‌کشدش علم سوی علیین
&&
گهیش جهل به پستی که هر چه بادا باد
\\
نشسته جان که به یک سو کند ظفر این را
&&
که تا رهم ز کشاکش شوم خوش و منقاد
\\
چو نیم کاره شد این قصه چون دهان بستی
&&
ز بیم ولوله و شر و فتنه و فریاد
\\
\end{longtable}
\end{center}
