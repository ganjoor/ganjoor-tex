\begin{center}
\section*{غزل شماره ۵۸۱: مه دی رفت و بهمن هم بیا که نوبهار آمد}
\label{sec:0581}
\addcontentsline{toc}{section}{\nameref{sec:0581}}
\begin{longtable}{l p{0.5cm} r}
مه دی رفت و بهمن هم بیا که نوبهار آمد
&&
زمین سرسبز و خرم شد زمان لاله زار آمد
\\
درختان بین که چون مستان همه گیجند و سرجنبان
&&
صبا برخواند افسونی که گلشن بی‌قرار آمد
\\
سمن را گفت نیلوفر که پیچاپیچ من بنگر
&&
چمن را گفت اشکوفه که فضل کردگار آمد
\\
بنفشه در رکوع آمد چو سنبل در خشوع آمد
&&
چو نرگس چشمکش می‌زد که وقت اعتبار آمد
\\
چه گفت آن بید سرجنبان که از مستی سبک سر شد
&&
چه دید آن سرو خوش قامت که رفت و پایدار آمد
\\
قلم بگرفته نقاشان که جانم مست کف‌هاشان
&&
که تصویرات زیباشان جمال شاخسار آمد
\\
هزاران مرغ شیرین پر نشسته بر سر منبر
&&
ثنا و حمد می‌خواند که وقت انتشار آمد
\\
چو گوید مرغ جان یاهو بگوید فاخته کوکو
&&
بگوید چون نبردی بو نصیبت انتظار آمد
\\
بفرمودند گل‌ها را که بنمایید دل‌ها را
&&
نشاید دل نهان کردن چو جلوه یار غار آمد
\\
به بلبل گفت گل بنگر به سوی سوسن اخضر
&&
که گر چه صد زبان دارد صبور و رازدار آمد
\\
جوابش داد بلبل رو به کشف راز من بگرو
&&
که این عشقی که من دارم چو تو بی‌زینهار آمد
\\
چنار آورد رو در رز که ای ساجد قیامی کن
&&
جوابش داد کاین سجده مرا بی‌اختیار آمد
\\
منم حامل از آن شربت که بر مستان زند ضربت
&&
مرا باطن چو نار آمد تو را ظاهر چنان آمد
\\
برآمد زعفران فرخ نشان عاشقان بر رخ
&&
بر او بخشود و گل گفت اه که این مسکین چه زار آمد
\\
رسید این ماجرای او به سیب لعل خندان رو
&&
به گل گفت او نمی‌داند که دلبر بردبار آمد
\\
چو سیب آورد این دعوی که نیکو ظنم از مولی
&&
برای امتحان آن ز هر سو سنگسار آمد
\\
کسی سنگ اندر او بندد چو صادق بود می‌خندد
&&
چرا شیرین نخندد خوش کش از خسرو نثار آمد
\\
کلوخ انداز خوبان را برای خواندن باشد
&&
جفای دوستان با هم نه از بهر نفار آمد
\\
زلیخا گر درید آن دم گریبان و زه یوسف
&&
پی تجمیش و بازی دان که کشاف سرار آمد
\\
خورد سنگ و فروناید که من آویخته شادم
&&
که این تشریف آویزش مرا منصوروار آمد
\\
که من منصورم آویزان ز شاخ دار الرحمان
&&
مرا دور از لب زشتان چنین بوس و کنار آمد
\\
هلا ختم است بر بوسه نهان کن دل چو سنبوسه
&&
درون سینه زن پنهان دمی که بی‌شمار آمد
\\
\end{longtable}
\end{center}
