\begin{center}
\section*{غزل شماره ۱۹۱۰: تو هر جزو جهان را بر گذر بین}
\label{sec:1910}
\addcontentsline{toc}{section}{\nameref{sec:1910}}
\begin{longtable}{l p{0.5cm} r}
تو هر جزو جهان را بر گذر بین
&&
تو هر یک را رسیده از سفر بین
\\
تو هر یک را به طمع روزی خود
&&
به پیش شاه خود بنهاده سر بین
\\
مثال اختران از بهر تابش
&&
فتاده عاجز اندر پای خور بین
\\
مثال سیل‌ها در جستن آب
&&
به سوی بحرشان زیر و زبر بین
\\
برای هر یکی از مطبخ شاه
&&
به قدر او تو خوان معتبر بین
\\
به پیش جام بحرآشام ایشان
&&
تو دریای جهان را مختصر بین
\\
وان‌ها را که روزی روی شاه است
&&
ز حسن شه دهانش پرشکر بین
\\
به چشم شمس تبریزی تو بنگر
&&
یکی دریای دیگر پرگهر بین
\\
\end{longtable}
\end{center}
