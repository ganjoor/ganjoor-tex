\begin{center}
\section*{غزل شماره ۱۷۸: می‌شدی غافل ز اسرار قضا}
\label{sec:0178}
\addcontentsline{toc}{section}{\nameref{sec:0178}}
\begin{longtable}{l p{0.5cm} r}
می‌شدی غافل ز اسرار قضا
&&
زخم خوردی از سلحدار قضا
\\
این چه کار افتاد آخر ناگهان
&&
این چنین باشد چنین کار قضا
\\
هیچ گل دیدی که خندد در جهان
&&
کو نشد گرینده از خار قضا
\\
هیچ بختی در جهان رونق گرفت
&&
کو نشد محبوس و بیمار قضا
\\
هیچ کس دزدیده روی عیش دید
&&
کو نشد آونگ بر دار قضا
\\
هیچ کس را مکر و فن سودی نکرد
&&
پیش بازی‌های مکار قضا
\\
این قضا را دوستان خدمت کنند
&&
جان کنند از صدق ایثار قضا
\\
گر چه صورت مرد جان باقی بماند
&&
در عنایت‌های بسیار قضا
\\
جوز بشکست و بمانده مغز روح
&&
رفت در حلوا ز انبار قضا
\\
آنک سوی نار شد بی‌مغز بود
&&
مغز او پوسید از انکار قضا
\\
آنک سوی یار شد مسعود بود
&&
مغز جان بگزید و شد یار قضا
\\
\end{longtable}
\end{center}
