\begin{center}
\section*{غزل شماره ۲۰۵۷: یار شو و یار بین دل شو و دلدار بین}
\label{sec:2057}
\addcontentsline{toc}{section}{\nameref{sec:2057}}
\begin{longtable}{l p{0.5cm} r}
یار شو و یار بین دل شو و دلدار بین
&&
در پی سرو روان چشمه و گلزار بین
\\
برجه و کاهل مباش در ره عیش و معاش
&&
پیشکشی کن قماش رونق تجار بین
\\
جمله تجار ما اهل دل و انبیا
&&
همره این کاروان خالق غفار بین
\\
آمد محمود باز بر در حجره ایاز
&&
عشق گزین عشقباز دولت بسیار بین
\\
خاک ایازم که او هست چو من عشق خو
&&
عشق شود عشق جو دلبر عیار بین
\\
سنت نیکو است این چارق با پوستین
&&
قبله کنش بهر شکر باقی از ایثار بین
\\
ساعت رنج و بلا چارق بین می‌شوی
&&
بی‌مرضی خویش را خسته و بیمار بین
\\
چارق ما نطفه دان خون رحم پوستین
&&
گوهر عقل و بصر از شه بیدار بین
\\
گوهر پیشین بنه تا کندت میر ده
&&
کهنه ده و نو ستان دانه ده انبار بین
\\
تا نگری در زمین هیچ نبینی فلک
&&
یک دمه خود را مبین خلعت دیدار بین
\\
این سخن درنثار هم به سخن ده سپار
&&
پس تو ز هر جزو خویش نکته و گفتار بین
\\
\end{longtable}
\end{center}
