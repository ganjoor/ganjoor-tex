\begin{center}
\section*{غزل شماره ۱۲۶۳: سرمست شد نگارم بنگر به نرگسانش}
\label{sec:1263}
\addcontentsline{toc}{section}{\nameref{sec:1263}}
\begin{longtable}{l p{0.5cm} r}
سرمست شد نگارم بنگر به نرگسانش
&&
مستانه شد حدیثش پیچیده شد زبانش
\\
گه می‌فتد از این سو گه می‌فتد از آن سو
&&
آن کس که مست گردد خود این بود نشانش
\\
چشمش بلای مستان ما را از او مترسان
&&
من مستم و نترسم از چوب شحنگانش
\\
ای عشق الله الله سرمست شد شهنشه
&&
برجه بگیر زلفش درکش در این میانش
\\
اندیشه‌ای که آید در دل ز یار گوید
&&
جان بر سرش فشانم پرزر کنم دهانش
\\
آن روی گلستانش وان بلبل بیانش
&&
وان شیوه‌هاش یا رب تا با کیست آنش
\\
این صورتش بهانه‌ست او نور آسمانست
&&
بگذر ز نقش و صورت جانش خوشست جانش
\\
دی را بهار بخشد شب را نهار بخشد
&&
پس این جهان مرده زنده‌ست از آن جهانش
\\
\end{longtable}
\end{center}
