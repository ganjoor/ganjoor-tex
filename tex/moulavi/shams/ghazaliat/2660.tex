\begin{center}
\section*{غزل شماره ۲۶۶۰: چرا ز اندیشه ای بیچاره گشتی}
\label{sec:2660}
\addcontentsline{toc}{section}{\nameref{sec:2660}}
\begin{longtable}{l p{0.5cm} r}
چرا ز اندیشه ای بیچاره گشتی
&&
فرورفتی به خود غمخواره گشتی
\\
تو را من پاره پاره جمع کردم
&&
چرا از وسوسه صدپاره گشتی
\\
ز دارالملک عشقم رخت بردی
&&
در این غربت چنین آواره گشتی
\\
زمین را بهر تو گهواره کردم
&&
فسرده تخته گهواره گشتی
\\
روان کردم ز سنگت آب حیوان
&&
به سوی خشک رفتی خاره گشتی
\\
تویی فرزند جان کار تو عشق است
&&
چرا رفتی تو و هرکاره گشتی
\\
از آن خانه که تو صد زخم خوردی
&&
به گرد آن در و درساره گشتی
\\
در آن خانه که صد حلوا چشیدی
&&
نگشتی مطمئن اماره گشتی
\\
خمش کن گفت هشیاریت آرد
&&
نه مست غمزه خماره گشتی
\\
\end{longtable}
\end{center}
