\begin{center}
\section*{غزل شماره ۲۰۵۲: ای آنک از میانه کران می‌کنی مکن}
\label{sec:2052}
\addcontentsline{toc}{section}{\nameref{sec:2052}}
\begin{longtable}{l p{0.5cm} r}
ای آنک از میانه کران می‌کنی مکن
&&
با ما ز خشم روی گران می‌کنی مکن
\\
دربند سود خویشی و اندر زیان ما
&&
کس زین نکرد سود زیان می‌کنی مکن
\\
راضی شدی که بیش نجویی زیان ما
&&
این از پی رضای کیان می‌کنی مکن
\\
بر جای باده سرکه غم می‌دهی مده
&&
در جوی آب خون چه روان می‌کنی مکن
\\
از چهره‌ام نشاط طرب می‌بری مبر
&&
بر چهره‌ام ز دیده نشان می‌کنی مکن
\\
مظلوم می‌کشی و تظلم همی‌کنی
&&
خود راه می‌زنی و فغان می‌کنی مکن
\\
پایم به کار نیست که سرمست دلبرم
&&
مر مست را بهل چه کشان می‌کنی مکن
\\
گویی بیا که بر تو کنم صبر را شبان
&&
بر بره گرگ را چه شبان می‌کنی مکن
\\
در روز زاهدی و به شب زاهدان کشی
&&
امشب که آشتی است همان می‌کنی مکن
\\
ای دوستان ز رشک تو خصمان همدگر
&&
این دوست را چه دشمن آن می‌کنی مکن
\\
گویی که می مخور پس اگر می همی‌دهی
&&
مخمور را چه خشک دهان می‌کنی مکن
\\
گویی چو تیر راست رو اندر هوای ما
&&
پس تیر راست را چه کمان می‌کنی مکن
\\
گویی خموش کن تو خموشم نمی‌هلی
&&
هر موی را ز عشق زبان می‌کنی مکن
\\
\end{longtable}
\end{center}
