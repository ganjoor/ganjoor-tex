\begin{center}
\section*{غزل شماره ۲۹۶۵: ای برده اختیارم تو اختیار مایی}
\label{sec:2965}
\addcontentsline{toc}{section}{\nameref{sec:2965}}
\begin{longtable}{l p{0.5cm} r}
ای برده اختیارم تو اختیار مایی
&&
من شاخ زعفرانم تو لاله زار مایی
\\
گفتم غمت مرا کشت گفتا چه زهره دارد
&&
غم این قدر نداند کآخر تو یار مایی
\\
من باغ و بوستانم سوزیده خزانم
&&
باغ مرا بخندان کآخر بهار مایی
\\
گفتا تو چنگ مایی و اندر ترنگ مایی
&&
پس چیست زاری تو چون در کنار مایی
\\
گفتم ز هر خیالی درد سر است ما را
&&
گفتا ببر سرش را تو ذوالفقار مایی
\\
سر را گرفته بودم یعنی که در خمارم
&&
گفت ار چه در خماری نی در خمار مایی
\\
گفتم چو چرخ گردان والله که بی‌قرارم
&&
گفت ار چه بی‌قراری نی بی‌قرار مایی
\\
شکرلبش بگفتم لب را گزید یعنی
&&
آن راز را نهان کن چون رازدار مایی
\\
ای بلبل سحرگه ما را بپرس گه گه
&&
آخر تو هم غریبی هم از دیار مایی
\\
تو مرغ آسمانی نی مرغ خاکدانی
&&
تو صید آن جهانی وز مرغزار مایی
\\
از خویش نیست گشته وز دوست هست گشته
&&
تو نور کردگاری یا کردگار مایی
\\
از آب و گل بزادی در آتشی فتادی
&&
سود و زیان یکی دان چون در قمار مایی
\\
این جا دوی نگنجد این ما و تو چه باشد
&&
این هر دو را یکی دان چون در شمار مایی
\\
خاموش کن که دارد هر نکته تو جانی
&&
مسپار جان به هر کس چون جان سپار مایی
\\
\end{longtable}
\end{center}
