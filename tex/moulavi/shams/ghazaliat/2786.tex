\begin{center}
\section*{غزل شماره ۲۷۸۶: آتشینا آب حیوان از کجا آورده‌ای}
\label{sec:2786}
\addcontentsline{toc}{section}{\nameref{sec:2786}}
\begin{longtable}{l p{0.5cm} r}
آتشینا آب حیوان از کجا آورده‌ای
&&
دانم این باری که الحق جان فزا آورده‌ای
\\
مشرق و مغرب بدرد همچو ابر از یک دگر
&&
چون چنین خورشید از نور خدا آورده‌ای
\\
خیره گان روی خود را از ره و منزل مپرس
&&
چون بر ایشان شعله‌های کبریا آورده‌ای
\\
احمقی باشد اگر جانی بمیرد بعد از این
&&
چون چنین دریای جوشان از بقا آورده‌ای
\\
از قضا و از قدر مر عاشقان را خوف نیست
&&
چون قدر را مست گشته با قضا آورده‌ای
\\
می‌نگنجد جان ما در پوست از شادی تو
&&
کاین جمال جان فزا از بهر ما آورده‌ای
\\
شمس تبریزی جفا کردی و دانم این قدر
&&
کز میان هر جفایی صد وفا آورده‌ای
\\
\end{longtable}
\end{center}
