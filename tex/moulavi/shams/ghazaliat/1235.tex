\begin{center}
\section*{غزل شماره ۱۲۳۵: نگاری را که می‌جویم به جانش}
\label{sec:1235}
\addcontentsline{toc}{section}{\nameref{sec:1235}}
\begin{longtable}{l p{0.5cm} r}
نگاری را که می‌جویم به جانش
&&
نمی‌بینم میان حاضرانش
\\
کجا رفت او میان حاضران نیست
&&
در این مجلس نمی‌بینم نشانش
\\
نظر می‌افکنم هر سو و هر جا
&&
نمی‌بینم اثر از گلستانش
\\
مسلمانان کجا شد نامداری
&&
که می‌دیدم چو شمع اندر میانش
\\
بگو نامش که هر کی نام او گفت
&&
به گور اندر نپوسد استخوانش
\\
خنک آن را که دست او ببوسید
&&
به وقت مرگ شیرین شد دهانش
\\
ز رویش شکر گویم یا ز خویش
&&
که کفو او نمی‌بیند جهانش
\\
زمینی گر نیابد شکل او چیست
&&
که می‌گردد در این عشق آسمانش
\\
بگو القاب شمس الدین تبریز
&&
مدار از گوش مشتاقان نهانش
\\
\end{longtable}
\end{center}
