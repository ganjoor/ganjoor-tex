\begin{center}
\section*{غزل شماره ۲۱۵۴: هین کژ و راست می‌روی باز چه خورده‌ای بگو}
\label{sec:2154}
\addcontentsline{toc}{section}{\nameref{sec:2154}}
\begin{longtable}{l p{0.5cm} r}
هین کژ و راست می‌روی باز چه خورده‌ای بگو
&&
مست و خراب می‌روی خانه به خانه کو به کو
\\
با کی حریف بوده‌ای بوسه ز کی ربوده‌ای
&&
زلف که را گشوده‌ای حلقه به حلقه مو به مو
\\
نی تو حریف کی کنی ای همه چشم و روشنی
&&
خفیه روی چو ماهیان حوض به حوض جو به جو
\\
راست بگو به جان تو ای دل و جانم آن تو
&&
ای دل همچو شیشه‌ام خورده میت کدو کدو
\\
راست بگو نهان مکن پشت به عاشقان مکن
&&
چشمه کجاست تا که من آب کشم سبو سبو
\\
در طلبم خیال تو دوش میان انجمن
&&
می‌نشناخت بنده را می‌نگریست رو به رو
\\
چون بشناخت بنده را بنده کژرونده را
&&
گفت بیا به خانه هی چند روی تو سو به سو
\\
عمر تو رفت در سفر با بد و نیک و خیر و شر
&&
همچو زنان خیره سر حجره به حجره شو به شو
\\
گفتمش ای رسول جان ای سبب نزول جان
&&
ز آنک تو خورده‌ای بده چند عتاب و گفت و گو
\\
گفت شراره‌ای از آن گر ببری سوی دهان
&&
حلق و دهان بسوزدت بانگ زنی گلو گلو
\\
لقمه هر خورنده را درخور او دهد خدا
&&
آنچ گلو بگیردت حرص مکن مجو مجو
\\
گفتم کو شراب جان ای دل و جان فدای آن
&&
من نه‌ام از شتردلان تا برمم به های و هو
\\
حلق و گلوبریده با کو برمد از این ابا
&&
هر کی بلنگد او از این هست مرا عدو عدو
\\
دست کز آن تهی بود گر چه شهنشهی بود
&&
دست بریده‌ای بود مانده به دیر بر سمو
\\
خامش باش و معتمد محرم راز نیک و بد
&&
آنک نیازمودیش راز مگو به پیش او
\\
\end{longtable}
\end{center}
