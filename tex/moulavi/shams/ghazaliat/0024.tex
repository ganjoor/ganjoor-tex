\begin{center}
\section*{غزل شماره ۲۴: چون نالد این مسکین که تا رحم آید آن دلدار را}
\label{sec:0024}
\addcontentsline{toc}{section}{\nameref{sec:0024}}
\begin{longtable}{l p{0.5cm} r}
چون نالد این مسکین که تا رحم آید آن دلدار را
&&
خون بارد این چشمان که تا بینم من آن گلزار را
\\
خورشید چون افروزدم تا هجر کمتر سوزدم
&&
دل حیلتی آموزدم کز سر بگیرم کار را
\\
ای عقل کل ذوفنون تعلیم فرما یک فسون
&&
کز وی بخیزد در درون رحمی نگارین یار را
\\
چون نور آن شمع چگل می‌درنیابد جان و دل
&&
کی داند آخر آب و گل دلخواه آن عیار را
\\
جبریل با لطف و رشد عجل سمین را چون چشد
&&
این دام و دانه کی کشد عنقای خوش منقار را
\\
عنقا که یابد دام کس در پیش آن عنقامگس
&&
ای عنکبوت عقل بس تا کی تنی این تار را
\\
کو آن مسیح خوش دمی بی‌واسطه مریم یمی
&&
کز وی دل ترسا همی پاره کند زنار را
\\
دجال غم چون آتشی گسترد ز آتش مفرشی
&&
کو عیسی خنجرکشی دجال بدکردار را
\\
تن را سلامت‌ها ز تو جان را قیامت‌ها ز تو
&&
عیسی علامت‌ها ز تو وصل قیامت وار را
\\
ساغر ز غم در سر فتد چون سنگ در ساغر فتد
&&
آتش به خار اندرفتد چون گل نباشد خار را
\\
ماندم ز عذرا وامقی چون من نبودم لایقی
&&
لیکن خمار عاشقی در سر دل خمار را
\\
شطرنج دولت شاه را صد جان به خرجش راه را
&&
صد که حمایل کاه را صد درد دردی خوار را
\\
بینم به شه واصل شده می از خودی فاصل شده
&&
وز شاه جان حاصل شده جان‌ها در او دیوار را
\\
باشد که آن شاه حرون زان لطف از حدها برون
&&
منسوخ گرداند کنون آن رسم استغفار را
\\
جانی که رو این سو کند با بایزید او خو کند
&&
یا در سنایی رو کند یا بو دهد عطار را
\\
مخدوم جان کز جام او سرمست شد ایام او
&&
گاهی که گویی نام او لازم شمر تکرار را
\\
عالی خداوند شمس دین تبریز از او جان زمین
&&
پرنور چون عرش مکین کو رشک شد انوار را
\\
ای صد هزاران آفرین بر ساعت فرخترین
&&
کان ناطق روح الامین بگشاید آن اسرار را
\\
در پاکی بی‌مهر و کین در بزم عشق او نشین
&&
در پرده منکر ببین آن پرده صدمسمار را
\\
\end{longtable}
\end{center}
