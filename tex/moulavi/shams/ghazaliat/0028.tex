\begin{center}
\section*{غزل شماره ۲۸: ای شاه جسم و جان ما خندان کن دندان ما}
\label{sec:0028}
\addcontentsline{toc}{section}{\nameref{sec:0028}}
\begin{longtable}{l p{0.5cm} r}
ای شاه جسم و جان ما خندان کن دندان ما
&&
سرمه کش چشمان ما ای چشم جان را توتیا
\\
ای مه ز اجلالت خجل عشقت ز خون ما بحل
&&
چون دیدمت می‌گفت دل جاء القضا جاء القضا
\\
ما گوی سرگردان تو اندر خم چوگان تو
&&
گه خوانیش سوی طرب گه رانیش سوی بلا
\\
گه جانب خوابش کشی گه سوی اسبابش کشی
&&
گه جانب شهر بقا گه جانب دشت فنا
\\
گه شکر آن مولی کند گه آه واویلی کند
&&
گه خدمت لیلی کند گه مست و مجنون خدا
\\
جان را تو پیدا کرده‌ای مجنون و شیدا کرده‌ای
&&
گه عاشق کنج خلا گه عاشق رو و ریا
\\
گه قصد تاج زر کند گه خاک‌ها بر سر کند
&&
گه خویش را قیصر کند گه دلق پوشد چون گدا
\\
طرفه درخت آمد کز او گه سیب روید گه کدو
&&
گه زهر روید گه شکر گه درد روید گه دوا
\\
جویی عجایب کاندرون گه آب رانی گاه خون
&&
گه باده‌های لعل گون گه شیر و گه شهد شفا
\\
گه علم بر دل برتند گه دانش از دل برکند
&&
گه فضل‌ها حاصل کند گه جمله را روبد بلا
\\
روزی محمدبک شود روزی پلنگ و سگ شود
&&
گه دشمن بدرگ شود گه والدین و اقربا
\\
گه خار گردد گاه گل گه سرکه گردد گاه مل
&&
گاهی دهلزن گه دهل تا می‌خورد زخم عصا
\\
گه عاشق این پنج و شش گه طالب جان‌های خوش
&&
این سوش کش آن سوش کش چون اشتری گم کرده جا
\\
گاهی چو چه کن پست رو مانند قارون سوی گو
&&
گه چون مسیح و کشت نو بالاروان سوی علا
\\
تا فضل تو راهش دهد وز شید و تلوین وارهد
&&
شیاد ما شیدا شود یک رنگ چون شمس الضحی
\\
چون ماهیان بحرش سکن بحرش بود باغ و وطن
&&
بحرش بود گور و کفن جز بحر را داند وبا
\\
زین رنگ‌ها مفرد شود در خنب عیسی دررود
&&
در صبغه الله رو نهد تا یفعل الله ما یشا
\\
رست از وقاحت وز حیا وز دور وز نقلان جا
&&
رست از برو رست از بیا چون سنگ زیر آسیا
\\
انا فتحنا بابکم لا تهجروا اصحابکم
&&
نلحق بکم اعقابکم هذا مکافات الولا
\\
انا شددنا جنبکم انا غفرنا ذنبکم
&&
مما شکرتم ربکم و الشکر جرار الرضا
\\
مستفعلن مستفعلن مستفعلن مستفعلن
&&
باب البیان مغلق قل صمتنا اولی بنا
\\
\end{longtable}
\end{center}
