\begin{center}
\section*{غزل شماره ۳۰۶۶: رسید ترکم با چهره‌های گل وردی}
\label{sec:3066}
\addcontentsline{toc}{section}{\nameref{sec:3066}}
\begin{longtable}{l p{0.5cm} r}
رسید ترکم با چهره‌های گل وردی
&&
بگفتمش چه شد آن عهد گفت اول وردی
\\
بگفتمش که یکی نامه‌ای به دست صبا
&&
بدادمی عجب آورد گفت گستردی
\\
بگفتمش که چرا بی‌گه آمدی ای دوست
&&
بگفت سیرو یدی یلده یلدشم اردی
\\
بگفتمش ز رخ توست شهر جان روشن
&&
ز آفتاب درآموختی جوامردی
\\
بگفت طرح نهد رخ رخم دو صد خور را
&&
تو چون مرا تبع او کنی زهی سردی
\\
بقای من چو بدید و زوال خود خورشید
&&
گرفت در طلبم عادت جهان گردی
\\
سجود کردم و مستغفرانه نالیدم
&&
بدید اشک مرا در فغان و پردردی
\\
بگفت نی که به قاصد مخالفی گفتی
&&
به عشق گفت من و گفتنم درآوردی
\\
بگفتمش گل بی‌خار و صبح بی‌شامی
&&
که بندگان را با شیر و شهد پروردی
\\
ز لطف‌های توست آنک سرخ می‌گویند
&&
به عرف حیله زر را بدان همه زردی
\\
بگفت باش کم آزار و دم مزن خامش
&&
که زرد گفتی زر را به فن و آزردی
\\
\end{longtable}
\end{center}
