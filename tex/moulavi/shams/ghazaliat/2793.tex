\begin{center}
\section*{غزل شماره ۲۷۹۳: این چه چتر است این که بر ملک ابد برداشتی}
\label{sec:2793}
\addcontentsline{toc}{section}{\nameref{sec:2793}}
\begin{longtable}{l p{0.5cm} r}
این چه چتر است این که بر ملک ابد برداشتی
&&
یادآوری جهان را ز آنک در سر داشتی
\\
زلف کفر و روی ایمان را چرا درساختی
&&
ز آنک قصد مؤمن و ترسا و کافر داشتی
\\
جان همی‌تابید از نور جلالت موج موج
&&
ز آنک تو در بحر جان دریا و گوهر داشتی
\\
پیش حیرتگاه عشقت جمله شیران در طلب
&&
بس که لرزیدند و افتادند و تو برداشتی
\\
هم تو جان را گاه مسکین و اسیر انداختی
&&
هم تواش سلطان و شاهنشاه و سنجر داشتی
\\
صد هزاران را میان آب دریا سوختی
&&
صد هزاران را میان آتشی تر داشتی
\\
در یکی جسم طلسم آدمی اندر نهان
&&
ای بسی خورشید و ماه و چرخ و اختر داشتی
\\
در چنین جسم چو تابوتی میان خون و خاک
&&
این شهید روح را هر لحظه خوشتر داشتی
\\
آفتابا پیش تو هر ذره‌ای کو شکر کرد
&&
مر دهان شکر او را پر ز شکر داشتی
\\
از نمک‌های حیاتت این وجود مرده را
&&
تازه و خوش بو چو ورد و مشک و عنبر داشتی
\\
شمس تبریزی ز عشقت من همه زر می‌زنم
&&
ز آنک تو بالا و پست عشق پرزر داشتی
\\
\end{longtable}
\end{center}
