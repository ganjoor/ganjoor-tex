\begin{center}
\section*{غزل شماره ۱۱۲۴: تاخت رخ آفتاب گشت جهان مست وار}
\label{sec:1124}
\addcontentsline{toc}{section}{\nameref{sec:1124}}
\begin{longtable}{l p{0.5cm} r}
تاخت رخ آفتاب گشت جهان مست وار
&&
بر مثل ذره‌ها رقص کنان پیش یار
\\
شاه نشسته به تخت عشق گرو کرده رخت
&&
رقص کنان هر درخت دست زنان هر چنار
\\
از قدح جام وی مست شده کو و کی
&&
گرم شده جام دی سرد شده جان نار
\\
روح بشارت شنید پرده جان بردرید
&&
رایت احمد رسید کفر بشد زار زار
\\
بانگ زده آن هما هر کی که هست از شما
&&
دور شو از عشق ما تا نشوی دلفکار
\\
گفته دل من بدو کای صنم تندخو
&&
چون برهد آن که او گشت به زخمت شکار
\\
عشق چو ابر گران ریخت بر این و بر آن
&&
شد طرفی زعفران شد طرفی لاله زار
\\
آب منی همچو شیر بعد زمانی یسیر
&&
زاد یکی همچو قیر وان دگری همچو قار
\\
منکر شه کور زاد بی‌خبر و کور باد
&&
از شه ما شمس دین در تبریز افتخار
\\
\end{longtable}
\end{center}
