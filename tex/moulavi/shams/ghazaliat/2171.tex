\begin{center}
\section*{غزل شماره ۲۱۷۱: گشته‌ست طپان جانم ای جان و جهان برگو}
\label{sec:2171}
\addcontentsline{toc}{section}{\nameref{sec:2171}}
\begin{longtable}{l p{0.5cm} r}
گشته‌ست طپان جانم ای جان و جهان برگو
&&
هین سلسله درجنبان ای ساقی جان برگو
\\
سلطان خوشان آمد و آن شاه نشان آمد
&&
تا چند کشی گوشم ای گوش کشان برگو
\\
سری است سمندر را ز آتش بنمی سوزد
&&
جانی است قلندر را نادرتر از آن برگو
\\
بنگر حشر مستان از دست بنه دستان
&&
با رطل گران پیش آ با ضرب گران برگو
\\
زان غمزه چون تیرش و ابروی کمان گیرش
&&
اسرار سلحشوری با تیر و کمان برگو
\\
برگو هله جان برگو پیش همگان برگو
&&
و آن نکته که می‌دانی با او پنهان برگو
\\
از جام رحیق او مست است عشیق او
&&
پیغام عقیق او ای گوهر کان برگو
\\
من بی‌زبر و زیرم در پنجه آن شیرم
&&
ز احوال جهان سیرم ز احوال فلان برگو
\\
زیر است نوای غم و اندرخور شادی بم
&&
یک لحظه چنین برگو یک لحظه چنان برگو
\\
خورشید معینت شد اقبال قرینت شد
&&
مقصود یقینت شد بی‌شک و گمان برگو
\\
چون بگذری ای عارف زین آب و گل ناشف
&&
زان سو مثل هاتف بی‌نام و نشان برگو
\\
در عالم جان جا کن در غیب تماشا کن
&&
رویی به روان‌ها کن زین گرم روان برگو
\\
من بیخود و سرمستم اینک سر خم بستم
&&
ای شاه زبردستم بی‌کام و دهان برگو
\\
\end{longtable}
\end{center}
