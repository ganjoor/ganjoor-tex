\begin{center}
\section*{غزل شماره ۲۷۹۵: ای تو جان صد گلستان از سمن پنهان شدی}
\label{sec:2795}
\addcontentsline{toc}{section}{\nameref{sec:2795}}
\begin{longtable}{l p{0.5cm} r}
ای تو جان صد گلستان از سمن پنهان شدی
&&
ای تو جان جان جانم چون ز من پنهان شدی
\\
چون فلک از توست روشن پس تو را محجوب چیست
&&
چونک تن از توست زنده چون ز تن پنهان شدی
\\
از کمال غیرت حق وز جمال حسن خویش
&&
ای شه مردان چنین از مرد و زن پنهان شدی
\\
ای تو شمع نه فلک کز نه فلک بگذشته‌ای
&&
تا چه سر است اینک تو اندر لگن پنهان شدی
\\
ای سهیلی کآفتاب از روی تو بیخود شده‌ست
&&
خیر باشد خیر باشد کز یمن پنهان شدی
\\
مشک تاتاری به هر دم می‌کند غمزی به خلق
&&
چونک سلطان خطایی وز ختن پنهان شدی
\\
گر ز ما پنهان شوی وز هر دو عالم چه عجب
&&
ای مه بی‌خویشتن کز خویشتن پنهان شدی
\\
آن چنان پنهان شدی ای آشکار جان‌ها
&&
تا ز بس پنهانی از پنهان شدن پنهان شدی
\\
شمس تبریزی به چاهی رفته‌ای چون یوسفی
&&
ای تو آب زندگی چون از رسن پنهان شدی
\\
\end{longtable}
\end{center}
