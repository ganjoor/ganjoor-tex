\begin{center}
\section*{غزل شماره ۲۱۲: اسیر شیشه کن آن جنیان دانا را}
\label{sec:0212}
\addcontentsline{toc}{section}{\nameref{sec:0212}}
\begin{longtable}{l p{0.5cm} r}
اسیر شیشه کن آن جنیان دانا را
&&
بریز خون دل آن خونیان صهبا را
\\
ربوده‌اند کلاه هزار خسرو را
&&
قبای لعل ببخشیده چهره ما را
\\
به گاه جلوه چو طاووس عقل‌ها برده
&&
گشاده چون دل عشاق پر رعنا را
\\
ز عکسشان فلک سبز رنگ لعل شود
&&
قیاس کن که چگونه کنند دل‌ها را
\\
درآورند به رقص و طرب به یک جرعه
&&
هزار پیر ضعیف بمانده برجا را
\\
چه جای پیر که آب حیات خلاقند
&&
که جان دهند به یک غمزه جمله اشیاء را
\\
شکرفروش چنین چست هیچ کس دیده‌ست
&&
سخن شناس کند طوطی شکرخا را
\\
زهی لطیف و ظریف و زهی کریم و شریف
&&
چنین رفیق بباید طریق بالا را
\\
صلا زدند همه عاشقان طالب را
&&
روان شوید به میدان پی تماشا را
\\
اگر خزینه قارون به ما فروریزند
&&
ز مغز ما نتوانند برد سودا را
\\
بیار ساقی باقی که جان جان‌هایی
&&
بریز بر سر سودا شراب حمرا را
\\
دلی که پند نگیرد ز هیچ دلداری
&&
بر او گمار دمی آن شراب گیرا را
\\
زهی شراب که عشقش به دست خود پخته‌ست
&&
زهی گهر که نبوده‌ست هیچ دریا را
\\
ز دست زهره به مریخ اگر رسد جامش
&&
رها کند به یکی جرعه خشم و صفرا را
\\
تو مانده‌ای و شراب و همه فنا گشتیم
&&
ز خویشتن چه نهان می‌کنی تو سیما را
\\
ولیک غیرت لالاست حاضر و ناظر
&&
هزار عاشق کشتی برای لالا را
\\
به نفی لا لا گوید به هر دمی لالا
&&
بزن تو گردن لا را بیار الا را
\\
بده به لالا جامی از آنک می‌دانی
&&
که علم و عقل رباید هزار دانا را
\\
و یا به غمزه شوخت به سوی او بنگر
&&
که غمزه تو حیاتی‌ست ثانی احیا را
\\
به آب ده تو غبار غم و کدورت را
&&
به خواب درکن آن جنگ را و غوغا را
\\
خدای عشق فرستاد تا در او پیچیم
&&
که نیست لایق پیچش ملک تعالی را
\\
بماند نیم غزل در دهان و ناگفته
&&
ولی دریغ که گم کرده‌ام سر و پا را
\\
برآ بتاب بر افلاک شمس تبریزی
&&
به مغز نغز بیارای برج جوزا را
\\
\end{longtable}
\end{center}
