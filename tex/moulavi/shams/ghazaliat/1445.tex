\begin{center}
\section*{غزل شماره ۱۴۴۵: من دلق گرو کردم عریان خراباتم}
\label{sec:1445}
\addcontentsline{toc}{section}{\nameref{sec:1445}}
\begin{longtable}{l p{0.5cm} r}
من دلق گرو کردم عریان خراباتم
&&
خوردم همه رخت خود مهمان خراباتم
\\
ای مطرب زیبارو دستی بزن و برگو
&&
تو آن مناجاتی من آن خراباتم
\\
خواهی که مرا بینی ای بسته نقش تن
&&
جان را نتوان دیدن من جان خراباتم
\\
نی مرد شکم خوارم نی درد شکم دارم
&&
زین مایده بیزارم بر خوان خراباتم
\\
من همدم سلطانم حقا که سلیمانم
&&
کلی همه ایمانم ایمان خراباتم
\\
با عشق در این پستی کردم طرب و مستی
&&
گفتم چه کسی گفتا سلطان خراباتم
\\
هر جا که همی‌باشم همکاسه اوباشم
&&
هر گوشه که می گردم گردان خراباتم
\\
گویی بنما معنی برهان چنین دعوی
&&
روشنتر از این برهان برهان خراباتم
\\
گر رفت زر و سیمم با سینه سیمینم
&&
ور بی‌سر و سامانم سامان خراباتم
\\
ای ساقی جان جانی شمع دل ویرانی
&&
ویران دلم را بین ویران خراباتم
\\
گویی که تو را شیطان افکند در این ویران
&&
خوبی ملک دارد شیطان خراباتم
\\
هر گه که خمش باشم من خم خراباتم
&&
هر گه که سخن گویم دربان خراباتم
\\
\end{longtable}
\end{center}
