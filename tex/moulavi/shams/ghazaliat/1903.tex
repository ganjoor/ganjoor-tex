\begin{center}
\section*{غزل شماره ۱۹۰۳: شنیدی تو که خط آمد ز خاقان}
\label{sec:1903}
\addcontentsline{toc}{section}{\nameref{sec:1903}}
\begin{longtable}{l p{0.5cm} r}
شنیدی تو که خط آمد ز خاقان
&&
که از پرده برون آیند خوبان
\\
چنین فرموده است خاقان که امسال
&&
شکر خواهم که باشد سخت ارزان
\\
زهی سال و زهی روز مبارک
&&
زهی خاقان زهی اقبال خندان
\\
درون خانه بنشستن حرام است
&&
که سلطان می خرامد سوی میدان
\\
بیا با ما به میدان تا ببینی
&&
یکی بزم خوش پیدای پنهان
\\
نهاده خوان و نعمت‌های بسیار
&&
ز حلواها و از مرغان بریان
\\
غلامان چو مه در پیش ساقی
&&
نوای مطربان خوشتر از جان
\\
ولیک از عشق شه جان‌های مستان
&&
فراغت دارد از ساقی و از خوان
\\
تو گویی این کجا باشد همان جا
&&
که اندیشه کجا گشته‌ست جویان
\\
\end{longtable}
\end{center}
