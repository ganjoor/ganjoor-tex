\begin{center}
\section*{غزل شماره ۲۰۶: ای همه خوبی تو را پس تو کرایی که را}
\label{sec:0206}
\addcontentsline{toc}{section}{\nameref{sec:0206}}
\begin{longtable}{l p{0.5cm} r}
ای همه خوبی تو را پس تو که رایی که را
&&
ای گل در باغ ما پس تو کجایی کجا
\\
سوسن با صد زبان از تو نشانم نداد
&&
گفت رو از من مجو غیر دعا و ثنا
\\
از کف تو ای قمر باغ دهان پرشکر
&&
وز کف تو بی‌خبر با همه برگ و نوا
\\
سرو اگر سر کشید در قد تو کی رسید
&&
نرگس اگر چشم داشت هیچ ندید او تو را
\\
مرغ اگر خطبه خواند شاخ اگر گل فشاند
&&
سبزه اگر تیز راند هیچ ندارد دوا
\\
شرب گل از ابر بود شرب دل از صبر بود
&&
ابر حریف گیاه صبر حریف صبا
\\
هر طرفی صف زده مردم و دیو و دده
&&
لیک در این میکده پای ندارند پا
\\
هر طرفی‌ام بجو هر چه بخواهی بگو
&&
ره نبری تار مو تا ننمایم هدی
\\
گرم شود روی آب از تپش آفتاب
&&
باز همش آفتاب برکشد اندر علا
\\
بربردش خرد خرد تا که ندانی چه برد
&&
صاف بدزدد ز درد شعشعه دلربا
\\
زین سخن بوالعجب بستم من هر دو لب
&&
لیک فلک جمله شب می‌زندت الصلا
\\
\end{longtable}
\end{center}
