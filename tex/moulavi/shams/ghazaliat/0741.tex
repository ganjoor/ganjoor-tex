\begin{center}
\section*{غزل شماره ۷۴۱: پنج در چه فایده چون هجر را شش تو کند}
\label{sec:0741}
\addcontentsline{toc}{section}{\nameref{sec:0741}}
\begin{longtable}{l p{0.5cm} r}
پنج در چه فایده چون هجر را شش تو کند
&&
خون بدان شد دل که طالب خون دل را بو کند
\\
چنگ را در عشق او از بهر آن آموختم
&&
کس نداند حالت من ناله من او کند
\\
ای به هر سویی دویده کار تو یک سو نشد
&&
آنک در شش سو نگنجد کار او یک سو کند
\\
شیر آهو می‌دراند شیر ما بس نادرست
&&
نقش آهو را بگیرد دردمد آهو کند
\\
باطنت را لاله سازد ظاهرت را ارغوان
&&
یک دمت سازد قزلبک یک دمت صارو کند
\\
موج آن دریا مجو کو را مدد از جو بود
&&
آن بجو کز نور جان دو پیه را دو جو کند
\\
خوش قمررویی کز این غم می‌گذارد چون هلال
&&
خوش شکرخویی که با آن شکرستان خو کند
\\
آهنی کو موم شد بهر قبول مهر عشق
&&
خاک را عنبر کند او سنگ را لؤلؤ کند
\\
دل کباب و خون دیده پیشکش پیشش برم
&&
گر تقاضای شراب و یخنی و طرغو کند
\\
لکلک آن حق شناسد ملک را لکلک کند
&&
فاخته محجوب باشد لاجرم کوکو کند
\\
آب و روغن کم کن و خامش چو روغن می‌گداز
&&
خرم آن کاندر غم آن روی تن چون مو کند
\\
\end{longtable}
\end{center}
