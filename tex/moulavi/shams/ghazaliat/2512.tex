\begin{center}
\section*{غزل شماره ۲۵۱۲: رها کن ماجرا ای جان فروکن سر ز بالایی}
\label{sec:2512}
\addcontentsline{toc}{section}{\nameref{sec:2512}}
\begin{longtable}{l p{0.5cm} r}
رها کن ماجرا ای جان فروکن سر ز بالایی
&&
که آمد نوبت عشرت زمان مجلس آرایی
\\
چه باشد جرم و سهو ما به پیش یرلغ لطفت
&&
کجا تردامنی ماند چو تو خورشید ما رایی
\\
درآ ای تاج و تخت ما برون انداز رخت ما
&&
بسوزان هر چه می‌سوزی بفرما هر چه فرمایی
\\
اگر آتش زنی سوزی تو باغ عقل کلی را
&&
هزاران باغ برسازی ز بی‌عقلی و شیدایی
\\
وگر رسوا شود عاشق به صد مکروه و صد تهمت
&&
از این سویش بیالایی وزان سویش بیارایی
\\
نه تو اجزای آبی را بدادی تابش جوهر
&&
نه تو اجزای خاکی را بدادی حله خضرایی
\\
نه از اجزای یک آدم جهان پرآدمی کردی
&&
نه آنی که مگس را تو بدادی فر عنقایی
\\
طبیبی دید کوری را نمودش داروی دیده
&&
بگفتش سرمه ساز این را برای نور بینایی
\\
بگفتش کور اگر آن را که من دیدم تو می‌دیدی
&&
دو چشم خویش می‌کندی و می‌گشتی تماشایی
\\
زهی لطفی که بر بستان و گورستان همی‌ریزی
&&
زهی نوری که اندر چشم و در بی‌چشم می‌آیی
\\
اگر بر زندگان ریزی برون پرند از گردون
&&
وگر بر مردگان ریزی شود مرده مسیحایی
\\
غذای زاغ سازیدی ز سرگینی و مرداری
&&
چه داند زاغ کان طوطی چه دارد در شکرخایی
\\
چه گفت آن زاغ بیهوده که سرگینش خورانیدی
&&
نگهدار ای خدا ما را از آن گفتار و بدرایی
\\
چه گفت آن طوطی اخضر که شکر دادیش درخور
&&
به فضل خود زبان ما بدان گفتار بگشایی
\\
کیست آن زاغ سرگین چش کسی کو مبتلا گردد
&&
به علمی غیر علم دین برای جاه دنیایی
\\
کیست آن طوطی و شکرضمیر منبع حکمت
&&
که حق باشد زبان او چو احمد وقت گویایی
\\
مرا در دل یکی دلبر همی‌گوید خمش بهتر
&&
که بس جان‌های نازک را کند این گفت سودایی
\\
\end{longtable}
\end{center}
