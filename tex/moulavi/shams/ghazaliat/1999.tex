\begin{center}
\section*{غزل شماره ۱۹۹۹: مکن ای دوست ز جور این دلم آواره مکن}
\label{sec:1999}
\addcontentsline{toc}{section}{\nameref{sec:1999}}
\begin{longtable}{l p{0.5cm} r}
مکن ای دوست ز جور این دلم آواره مکن
&&
جان پی پاره بگیر و جگرم پاره مکن
\\
مر تو را عاشق دل داده و غمخوار بسی است
&&
جان و سر قصد سر این دل غمخواره مکن
\\
نظر رحم بکن بر من و بیچارگیم
&&
جز تو ار چاره گری هست مرا چاره مکن
\\
پیش آتشکده عشق تو دل شیشه گر است
&&
دل خود بر دل چون شیشه من خاره مکن
\\
هر دمی هجر ستمکار تو دم می دهدم
&&
هر دمم دم ده بی‌باک ستمکاره مکن
\\
تن پربند چو گهواره و دل چون طفل است
&&
در کنارش کش و وابسته گهواره مکن
\\
پیش خورشید رخت جان مرا رقصان دار
&&
همچو شب جان مرا بند هر استاره مکن
\\
ز دغل عالم غدار دو صد سر دارد
&&
سر من در سر این عالم غداره مکن
\\
صد چو هاروت و چو ماروت ز سحرش بسته‌ست
&&
مر مرا بسته این جادوی سحاره مکن
\\
خمر یک روزه این نفس خمار ابد است
&&
هین مرا تشنه این خاین خماره مکن
\\
لعب اول چو مرا بست میفزا بازی
&&
ز آنچ یک باره شدم مات تو ده باره مکن
\\
جمله عیاری ناسوت ز لاهوت تو است
&&
تو دگر یاری این کافر عیاره مکن
\\
\end{longtable}
\end{center}
