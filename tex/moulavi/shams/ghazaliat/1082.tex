\begin{center}
\section*{غزل شماره ۱۰۸۲: عقل بند ره روان و عاشقانست ای پسر}
\label{sec:1082}
\addcontentsline{toc}{section}{\nameref{sec:1082}}
\begin{longtable}{l p{0.5cm} r}
عقل بند ره روان و عاشقانست ای پسر
&&
بند بشکن ره عیان اندر عیانست ای پسر
\\
عقل بند و دل فریب و تن غرور و جان حجاب
&&
راه از این جمله گرانی‌ها نهانست ای پسر
\\
چون ز عقل و جان و دل برخاستی بیرون شدی
&&
این یقین و این عیان هم در گمانست ای پسر
\\
مرد کو از خود نرفتست او نه مردست ای پسر
&&
عشق کان از جان نباشد آفسانست ای پسر
\\
سینه خود را هدف کن پیش تیر حکم او
&&
هین که تیر حکم او اندر کمانست ای پسر
\\
سینه‌ای کز زخم تیر جذبه او خسته شد
&&
بر جبین و چهره او صد نشانست ای پسر
\\
گر روی بر آسمان هفتمین ادریس وار
&&
عشق جانان سخت نیکونردبانست ای پسر
\\
هر طرف که کاروانی نازنازان می‌رود
&&
عشق را بنگر که قبله کاروانست ای پسر
\\
سایه افکندست عشقش همچو دامی بر زمین
&&
عشق چون صیاد او بر آسمانست ای پسر
\\
عشق را از من مپرس از کس مپرس از عشق پرس
&&
عشق در گفتن چو ابر درفشانست ای پسر
\\
ترجمانی من و صد چون منش محتاج نیست
&&
در حقایق عشق خود را ترجمانست ای پسر
\\
عشق کار خفتگان و نازکان نرم نیست
&&
عشق کار پردلان و پهلوانست ای پسر
\\
هر کی او مر عاشقان و صادقان را بنده شد
&&
خسرو و شاهنشه و صاحب قرانست ای پسر
\\
این جهان پرفسون از عشق تا نفریبدت
&&
کاین جهان بی‌وفا از تو جهانست ای پسر
\\
بیت‌های این غزل گر شد دراز از وصل‌ها
&&
پرده دیگر شد ولی معنی همانست ای پسر
\\
هین دهان بربند و خامش کن از این پس چون صدف
&&
کاین زیانت در حقیقت خصم جانست ای پسر
\\
\end{longtable}
\end{center}
