\begin{center}
\section*{غزل شماره ۲۴۳۱: این عشق گردان کو به کو بر سر نهاده طبله‌ای}
\label{sec:2431}
\addcontentsline{toc}{section}{\nameref{sec:2431}}
\begin{longtable}{l p{0.5cm} r}
این عشق گردان کو به کو بر سر نهاده طبله‌ای
&&
که هر کجا مرده بود زنده کنم بی‌حیله‌ای
\\
خوان روانم از کرم زنده کنم مرده بدم
&&
کو نرگدایی تا برد از خوان لطفم زله‌ای
\\
گاهی تو را در بر کنم گاهی ز زهرت پر کنم
&&
آگاه شو آخر ز من ای در کفم چون کیله‌ای
\\
گر حبه‌ای آید به من صد کان پرزرش کنم
&&
دریای شیرینش کنم هر چند باشد قله‌ای
\\
از تو عدم وز من کرم وز تو رضا وز من قسم
&&
صد اطلس و اکسون نهم در پیش کرم پیله‌ای
\\
هر لحظه نومید را خرمن دهم بی‌کشتنی
&&
هر لحظه درویش را قربت دهم بی‌چله‌ای
\\
چشمه شکر جوشان کنم اندر دل تنگ نیی
&&
اندیشه‌های خوش نهم اندر دماغ و کله‌ای
\\
می‌ران فرس در دین فقط ور اسب تو گردد سقط
&&
بر جای اسب لاغری هر سو بیابی گله‌ای
\\
خاموش باش و لا مگو جز آن که حق بخشد مجو
&&
جوشان ز حلوای رضا بر جمره چون پاتیله‌ای
\\
تبریز شد خلد برین از عکس روی شمس دین
&&
هر نقش در وی حور عین هر جامه از وی حله‌ای
\\
\end{longtable}
\end{center}
