\begin{center}
\section*{غزل شماره ۲۹۰۳: باوفا یارا جفا آموختی}
\label{sec:2903}
\addcontentsline{toc}{section}{\nameref{sec:2903}}
\begin{longtable}{l p{0.5cm} r}
باوفا یارا جفا آموختی
&&
این جفا را از کجا آموختی
\\
کو وفاهای لطیفت کز نخست
&&
در شکار جان ما آموختی
\\
هر کجا زشتی جفاکاری رسید
&&
خوبیش دادی وفا آموختی
\\
ای دل از عالم چنین بیگانگی
&&
هم ز یار آشنا آموختی
\\
جانت گر خواهد صنم گویی بلی
&&
این بلی را زان بلا آموختی
\\
عشق را گفتم فروخوردی مرا
&&
این مگر از اژدها آموختی
\\
آن عصای موسی اژدرها بخورد
&&
تو مگر هم زان عصا آموختی
\\
ای دل ار از غمزه‌اش خسته شدی
&&
از لبش آخر دوا آموختی
\\
شکر هشتی و شکایت می‌کنی
&&
از یکی باری خطا آموختی
\\
زان شکرخانه مگو الا که شکر
&&
آن چنان کز انبیا آموختی
\\
این صفا را از گله تیره مکن
&&
کاین صفا از مصطفی آموختی
\\
هر چه خلق آموختت زان لب ببند
&&
جمله آن شو کز خدا آموختی
\\
عاشقا از شمس تبریزی چو ابر
&&
سوختی لیکن ضیا آموختی
\\
\end{longtable}
\end{center}
