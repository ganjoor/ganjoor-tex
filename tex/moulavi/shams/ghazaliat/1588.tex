\begin{center}
\section*{غزل شماره ۱۵۸۸: من سر خم را ببستم باز شد پهلوی خم}
\label{sec:1588}
\addcontentsline{toc}{section}{\nameref{sec:1588}}
\begin{longtable}{l p{0.5cm} r}
من سر خم را ببستم باز شد پهلوی خم
&&
آنک خم را ساخت هم او می شناسد خوی خم
\\
کوزه‌ها محتاج خم و خم‌ها محتاج جو
&&
در میان خم چه باشد آنچ دارد جوی خم
\\
مستیان بس پدید و خمشان را کس ندید
&&
عالمی زیر و زبر پیچان شده از بوی خم
\\
گر نبودی بوی آن خم در دماغ خاص و عام
&&
پس به هر محفل چرا دارند گفت و گوی خم
\\
بوی خمش خلق را در کوزه فقاع کرد
&&
شد هزاران ترک و رومی بنده و هندوی خم
\\
جادوی بر خم نشیند می دواند شهر شهر
&&
جادوان را ریش خندی می کند جادوی خم
\\
در سر خود پیچ ای دل مست و بیخود چون شراب
&&
همچنین می رو خراب از بوی خم تا روی خم
\\
تا ببینی ناگهان مستی رمیده از جهان
&&
نزد خم ای جان عمم که منم خالوی خم
\\
روی از آن سو کن کز این سو گفت و گو را راه نیست
&&
چون ز شش سو وارهیدی بازیابی سوی خم
\\
\end{longtable}
\end{center}
