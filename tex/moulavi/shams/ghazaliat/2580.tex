\begin{center}
\section*{غزل شماره ۲۵۸۰: ای خیره نظر در جو پیش آ و بخور آبی}
\label{sec:2580}
\addcontentsline{toc}{section}{\nameref{sec:2580}}
\begin{longtable}{l p{0.5cm} r}
ای خیره نظر در جو پیش آ و بخور آبی
&&
بیهوده چه می‌گردی بر آب چو دولابی
\\
صحراست پر از شکر دریاست پر از گوهر
&&
یک جو نبری زین دو بی‌کوشش و اسبابی
\\
گر مرد تماشایی چون دیده بنگشایی
&&
بگشادن چشم ارزد تا بانی مهتابی
\\
محراب بسی دیدی در وی بنگنجیدی
&&
اندر نظر حربی بشکافد محرابی
\\
ما تشنه و هر جانب یک چشمه حیوانی
&&
ما طامع و پیش و پس دریا کف وهابی
\\
ره چیست میان ما جز نقص عیان ما
&&
کو پرده میان ما جز چشم گران خوابی
\\
شش نور همی‌بارد زان ابر که حق آرد
&&
جسمت مثل بامی هر حس تو میزانی
\\
شش چشمه پیوسته می‌گردد شب بسته
&&
زان سوش روان کرده آن فاتح ابوابی
\\
خورشید و قمر گاهی شب افتد در چاهی
&&
بیرون کشدش زان چه بی‌آلت و قلابی
\\
صد صنعت سلطانی دارد ز تو پنهانی
&&
زیرا که ضعیفی تو بی‌طاقت و بی‌تابی
\\
این مفرش و آن کیوان افلاک ورای آن
&&
بر کف خدا لرزان ماننده سیمابی
\\
دریا چو چنان باشد کف درخور آن باشد
&&
اندر صفتش خاطر هست احول و کذابی
\\
بگریزد عقل و جان از هیبت آن سلطان
&&
چون دیو که بگریزد از عمر خطابی
\\
بکری برمد از شو معشوق جهانش او
&&
از جان عزیز خود بیگانه و صخابی
\\
ره داده به دام خود صد زاغ پی بازی
&&
چون باز به دام آمد برداشته مضرابی
\\
خاموش که آن اسعد این را به از این گوید
&&
بی‌صفقه صفاقی بی‌شرفه دبابی
\\
\end{longtable}
\end{center}
