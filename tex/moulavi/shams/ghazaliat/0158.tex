\begin{center}
\section*{غزل شماره ۱۵۸: امتزاج روح‌ها در وقت صلح و جنگ‌ها}
\label{sec:0158}
\addcontentsline{toc}{section}{\nameref{sec:0158}}
\begin{longtable}{l p{0.5cm} r}
امتزاج روح‌ها در وقت صلح و جنگ‌ها
&&
با کسی باید که روحش هست صافی صفا
\\
چون تغییر هست در جان وقت جنگ و آشتی
&&
آن نه یک روحست تنها بلک گشتستند جدا
\\
چون بخواهد دل سلام آن یکی همچون عروس
&&
مر زفاف صحبت داماد دشمن روی را
\\
باز چون میلی بود سویی بدان ماند که او
&&
میل دارد سوی داماد لطیف دلربا
\\
از نظرها امتزاج و از سخن‌ها امتزاج
&&
وز حکایت امتزاج و از فکر آمیزها
\\
همچنانک امتزاج ظاهرست اندر رکوع
&&
وز تصافح وز عناق و قبله و مدح و دعا
\\
بر تفاوت این تمازج‌ها ز میل و نیم میل
&&
وز سر کره و کراهت وز پی ترس و حیا
\\
آن رکوع باتأنی وان ثنای نرم نرم
&&
هم مراتب در معانی در صورها مجتبا
\\
این همه بازیچه گردد چون رسیدی در کسی
&&
کش سما سجده‌اش برد وان عرش گوید مرحبا
\\
آن خداوند لطیف بنده پرور شمس دین
&&
کو رهاند مر شما را زین خیال بی‌وفا
\\
با عدم تا چند باشی خایف و امیدوار
&&
این همه تأثیر خشم اوست تا وقت رضا
\\
هستی جان اوست حقا چونک هستی زو بتافت
&&
لاجرم در نیستی می‌ساز با قید هوا
\\
گه به تسبیع هوا و گه به تسبیع خیال
&&
گه به تسبیع کلام و گه به تسبیع لقا
\\
گه خیال خوش بود در طنز همچون احتلام
&&
گه خیال بد بود همچون که خواب ناسزا
\\
وانگهی تخییل‌ها خوشتر از این قوم رذیل
&&
اینت هستی کو بود کمتر ز تخییل عما
\\
پس از آن سوی عدم بدتر از این از صد عدم
&&
این عدم‌ها بر مراتب بود همچون که بقا
\\
تا نیاید ظل میمون خداوندی او
&&
هیچ بندی از تو نگشاید یقین می‌دان دلا
\\
\end{longtable}
\end{center}
