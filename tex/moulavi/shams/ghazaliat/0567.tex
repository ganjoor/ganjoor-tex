\begin{center}
\section*{غزل شماره ۵۶۷: نباشد عیب پرسیدن تو را خانه کجا باشد}
\label{sec:0567}
\addcontentsline{toc}{section}{\nameref{sec:0567}}
\begin{longtable}{l p{0.5cm} r}
نباشد عیب پرسیدن تو را خانه کجا باشد
&&
نشانی ده اگر یابیم وان اقبال ما باشد
\\
تو خورشید جهان باشی ز چشم ما نهان باشی
&&
تو خود این را روا داری وانگه این روا باشد
\\
نگفتی من وفادارم وفا را من خریدارم
&&
ببین در رنگ رخسارم بیندیش این وفا باشد
\\
بیا ای یار لعلین لب دلم گم گشت در قالب
&&
دلم داغ شما دارد یقین پیش شما باشد
\\
در این آتش کبابم من خراب اندر خرابم من
&&
چه باشد ای سر خوبان تنی کز سر جدا باشد
\\
دل من در فراق جان چو ماری سرزده پیچان
&&
بگرد نقش تو گردان مثال آسیا باشد
\\
بگفتم ای دل مسکین بیا بر جای خود بنشین
&&
حذر کن ز آتش پرکین دل من گفت تا باشد
\\
فروبستست تدبیرم بیا ای یار شبگیرم
&&
بپرس از شاه کشمیرم کسی را کشنا باشد
\\
خود او پیدا و پنهانست جهان نقش است و او جانست
&&
بیندیش این چه سلطانست مگر نور خدا باشد
\\
خروش و جوش هر مستی ز جوش خم می باشد
&&
سبکساری هر آهن ز تو آهن ربا باشد
\\
خریدی خانه دل را دل آن توست می‌دانی
&&
هر آنچ هست در خانه از آن کدخدا باشد
\\
قماشی کان تو نبود برون انداز از خانه
&&
درون مسجد اقصی سگ مرده چرا باشد
\\
مسلم گشت دلداری تو را ای تو دل عالم
&&
مسلم گشت جان بخشی تو را وان دم تو را باشد
\\
که دریا را شکافیدن بود چالاکی موسی
&&
قبای مه شکافیدن ز نور مصطفی باشد
\\
برآرد عشق یک فتنه که مردم راه که گیرد
&&
به شهر اندر کسی ماند که جویای فنا باشد
\\
زند آتش در این بیشه که بگریزند نخجیران
&&
ز آتش هر که نگریزد چو ابراهیم ما باشد
\\
خمش کوته کن ای خاطر که علم اول و آخر
&&
بیان کرده بود عاشق چو پیش شاه لا باشد
\\
\end{longtable}
\end{center}
