\begin{center}
\section*{غزل شماره ۵۶۱: طوطی جان مست من از شکری چه می‌شود}
\label{sec:0561}
\addcontentsline{toc}{section}{\nameref{sec:0561}}
\begin{longtable}{l p{0.5cm} r}
طوطی جان مست من از شکری چه می‌شود
&&
زهره می پرست من از قمری چه می‌شود
\\
بحر دلم که موج او از فلک نهم گذشت
&&
خیره بمانده‌ام که او از گهری چه می‌شود
\\
باغ دلم که صد ارم در نظرش بود عدم
&&
نرگس تازه خیره شد کز شجری چه می‌شود
\\
جان سپهست و من علم جان سحرست و من شبم
&&
این دل آفتاب من هر سحری چه می‌شود
\\
دل شده پاره پاره‌ها در نظر و نظاره‌ها
&&
کاین همه کون هر زمان از نظری چه می‌شود
\\
از غلبات عشق او عقل چه شور می‌کند
&&
وز لمعان جان او جانوری چه می‌شود
\\
من همگی چو شیشه‌ام شیشه گریست پیشه‌ام
&&
آه که شیشه دلم از حجری چه می‌شود
\\
باخبران و زیرکان گر چه شوند لعل کان
&&
بی خبرند از این کز او بی‌خبری چه می‌شود
\\
از تبریز شمس دین راست شود دل و نظر
&&
آن نظر خوش از کژ و کژنگری چه می‌شود
\\
\end{longtable}
\end{center}
