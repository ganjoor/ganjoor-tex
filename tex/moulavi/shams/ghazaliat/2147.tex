\begin{center}
\section*{غزل شماره ۲۱۴۷: چیست که هر دمی چنین می‌کشدم به سوی او}
\label{sec:2147}
\addcontentsline{toc}{section}{\nameref{sec:2147}}
\begin{longtable}{l p{0.5cm} r}
چیست که هر دمی چنین می‌کشدم به سوی او
&&
عنبر نی و مشک نی بوی وی است بوی او
\\
سلسله‌ای است بی‌بها دشمن جمله توبه‌ها
&&
توبه شکست من کیم سنگ من و سبوی او
\\
توبه شکست او بسی توبه و این چنین کسی
&&
پرده دری و دلبری خوی وی است خوی او
\\
توبه من برای او توبه شکن هوای او
&&
توبه من گناه من سوخته پیش روی او
\\
شاخ و درخت عقل و جان نیست مگر به باغ او
&&
آب حیات جاودان نیست مگر به جوی او
\\
عشق و نشاط گستری با می و رطل ساغری
&&
می‌رسد از کنارها غلغل وهای هوی او
\\
مرد که خودپسند شد همچو کدو بلند شد
&&
تا نشود ز خود تهی پر نشود کدوی او
\\
سایه که باز می‌شود جمع و دراز می‌شود
&&
هست ز آفتاب جان قوت جست و جوی او
\\
سایه وی است و نور او جمع وی است و دور او
&&
نور ز عکس روی او سایه ز عکس موی او
\\
ای مه و آفتاب جان پرده دری مکن عیان
&&
تا ز فلک فرودرد پرده هفت توی او
\\
چیست درون جیب من جز تو و من حجاب من
&&
ای من و تو فنا شده پیش بقای اوی او
\\
\end{longtable}
\end{center}
