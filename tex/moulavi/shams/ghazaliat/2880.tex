\begin{center}
\section*{غزل شماره ۲۸۸۰: به حق و حرمت آنک همگان را جانی}
\label{sec:2880}
\addcontentsline{toc}{section}{\nameref{sec:2880}}
\begin{longtable}{l p{0.5cm} r}
به حق و حرمت آنک همگان را جانی
&&
قدحی پر کن از آنک صفتش می‌دانی
\\
همه را زیر و زبر کن نه زبر مان و نه زیر
&&
تا بدانند که امروز در این میدانی
\\
آتش باده بزن در بنه شرم و حیا
&&
دل مستان بگرفت از طرب پنهانی
\\
وقت آن شد که دل رفته به ما بازآری
&&
عقل‌ها را چو کبوتربچگان پرانی
\\
نکته می‌گویی در حلقه مستان خراب
&&
خوش بود گنج که درتابد در ویرانی
\\
می جوشیده بر این سوختگان گردان کن
&&
پیش خامان بنه آن قلیه و آن بورانی
\\
چه شدم من تو بگو هم که چه دانم شده‌ای
&&
کی بگوید لب تو حرف بدین آسانی
\\
\end{longtable}
\end{center}
