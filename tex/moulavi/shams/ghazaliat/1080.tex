\begin{center}
\section*{غزل شماره ۱۰۸۰: ساقیا هستند خلقان از می ما دور دور}
\label{sec:1080}
\addcontentsline{toc}{section}{\nameref{sec:1080}}
\begin{longtable}{l p{0.5cm} r}
ساقیا هستند خلقان از می ما دور دور
&&
زان جمال و زان کمال و فر و سیما دور دور
\\
گر چه پیر کهنه‌ای در حکمت و ذوق و صفا
&&
از شراب صاف ما هستی تو پیرا دور دور
\\
چونک بینایان نمی‌بینند رنگ جام را
&&
عقل خود داند که باشد جان اعمی دور دور
\\
چون صریح و رمز قاضی می‌نداند جان او
&&
دور باشد از دل او رمز و ایما دور دور
\\
تا نبرد تیغ شمس الحق زنار تو را
&&
جان تو باشد از آن لطف و چلیپا دور دور
\\
تا ز خوبی بتان خالی نگردد جان تو
&&
باشی از رخسار آن دلدار زیبا دور دور
\\
گر چه اندر بزم شاهان تو بدی سرده ولیک
&&
چون در این بزم اندرآیی باشی این جا دور دور
\\
تو شنیدی قرب موسی طور سینا نور حق
&&
در حضور خضر بود آن طور سینا دور دور
\\
سقف مینا گر چه بس عالیست پیش چشم تو
&&
لیک پیش رفعتش بد سقف مینا دور دور
\\
ای گران جان یا سبک شو یا برو از بزم ما
&&
یا مکن مانند خود از عیش ما را دور دور
\\
مطرب عشاق بهر من زن این نادر نوا
&&
زانک هست از گوش کر این بانگ سرنا دور دور
\\
\end{longtable}
\end{center}
