\begin{center}
\section*{غزل شماره ۲۸۹۷: بوی باغ و گلستان آید همی}
\label{sec:2897}
\addcontentsline{toc}{section}{\nameref{sec:2897}}
\begin{longtable}{l p{0.5cm} r}
بوی باغ و گلستان آید همی
&&
بوی یار مهربان آید همی
\\
از نثار جوهر یارم مرا
&&
آب دریا تا میان آید همی
\\
با خیال گلستانش خارزار
&&
نرمتر از پرنیان آید همی
\\
از چنین نجار یعنی عشق او
&&
نردبان آسمان آید همی
\\
جوع کلبم را ز مطبخ‌های جان
&&
لحظه لحظه بوی نان آید همی
\\
زان در و دیوارهای کوی دوست
&&
عاشقان را بوی جان آید همی
\\
یک وفا می‌آر و می‌بر صد هزار
&&
این چنین را آن چنان آید همی
\\
هر که میرد پیش حسن روی دوست
&&
نابمرده در جنان آید همی
\\
کاروان غیب می‌آید به عین
&&
لیک از این زشتان نهان آید همی
\\
نغزرویان سوی زشتان کی روند
&&
بلبل اندر گلبنان آید همی
\\
پهلوی نرگس بروید یاسمین
&&
گل به غنچه خوش دهان آید همی
\\
این همه رمز است و مقصود این بود
&&
کان جهان اندر جهان آید همی
\\
همچو روغن در میان جان شیر
&&
لامکان اندر مکان آید همی
\\
همچو عقل اندر میان خون و پوست
&&
بی نشان اندر نشان آید همی
\\
وز ورای عقل عشق خوبرو
&&
می‌به کف دامن کشان آید همی
\\
وز ورای عشق آن کش شرح نیست
&&
جز همین گفتن که آن آید همی
\\
بیش از این شرحش توان کردن ولیک
&&
از سوی غیرت سنان آید همی
\\
تن زنم زیرا ز حرف مشکلش
&&
هر کسی را صد گمان آید همی
\\
\end{longtable}
\end{center}
