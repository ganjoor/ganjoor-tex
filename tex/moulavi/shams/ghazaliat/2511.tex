\begin{center}
\section*{غزل شماره ۲۵۱۱: به باغ و چشمه حیوان چرا این چشم نگشایی}
\label{sec:2511}
\addcontentsline{toc}{section}{\nameref{sec:2511}}
\begin{longtable}{l p{0.5cm} r}
به باغ و چشمه حیوان چرا این چشم نگشایی
&&
چرا بیگانه‌ای از ما چو تو در اصل از مایی
\\
تو طوطی زاده‌ای جانم مکن ناز و مرنجانم
&&
ز اصل آورده‌ای دانم تو قانون شکرخایی
\\
بیا در خانه خویش آ مترس از عکس خود پیش آ
&&
بهل طبع کژاندیشی که او یاوه‌ست و هرجایی
\\
بیا ای شاه یغمایی مرو هر جا که ما رایی
&&
اگر بر دیگران تلخی به نزد ما چو حلوایی
\\
نباشد عیب در نوری کز او غافل بود کوری
&&
نباشد عیب حلوا را به طعن شخص صفرایی
\\
برآر از خاک جانی را ببین جان آسمانی را
&&
کز آن گردان شده‌ست ای جان مه و این چرخ خضرایی
\\
قدم بر نردبانی نه دو چشم اندر عیانی نه
&&
بدن را در زیانی نه که تا جان را بیفزایی
\\
درختی بین بسی بابر نه خشکش بینی و نی تر
&&
به سایه آن درخت اندر بخسپی و بیاسایی
\\
یکی چشمه عجب بینی که نزدیکش چو بنشینی
&&
شوی همرنگ او در حین به لطف و ذوق و زیبایی
\\
ندانی خویش را از وی شوی هم شیء و هم لاشی
&&
نماند کو نماند کی نماند رنگ و سیمایی
\\
چو با چشمه درآمیزی نماید شمس تبریزی
&&
درون آب همچون مه ز بهر عالم آرایی
\\
\end{longtable}
\end{center}
