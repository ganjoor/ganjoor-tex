\begin{center}
\section*{غزل شماره ۳۲۲: آمده‌ام که تا به خود گوش کشان کشانمت}
\label{sec:0322}
\addcontentsline{toc}{section}{\nameref{sec:0322}}
\begin{longtable}{l p{0.5cm} r}
آمده‌ام که تا به خود گوش کشان کشانمت
&&
بی دل و بیخودت کنم در دل و جان نشانمت
\\
آمده‌ام بهار خوش پیش تو ای درخت گل
&&
تا که کنار گیرمت خوش خوش و می‌فشانمت
\\
آمده‌ام که تا تو را جلوه دهم در این سرا
&&
همچو دعای عاشقان فوق فلک رسانمت
\\
آمده‌ام که بوسه‌ای از صنمی ربوده‌ای
&&
بازبده به خوشدلی خواجه که واستانمت
\\
گل چه بود که گل تویی ناطق امر قل تویی
&&
گر دگری نداندت چون تو منی بدانمت
\\
جان و روان من تویی فاتحه خوان من تویی
&&
فاتحه شو تو یک سری تا که به دل بخوانمت
\\
صید منی شکار من گر چه ز دام جسته‌ای
&&
جانب دام بازرو ور نروی برانمت
\\
شیر بگفت مر مرا نادره آهوی برو
&&
در پی من چه می‌دوی تیز که بردرانمت
\\
زخم پذیر و پیش رو چون سپر شجاعتی
&&
گوش به غیر زه مده تا چو کمان خمانمت
\\
از حد خاک تا بشر چند هزار منزلست
&&
شهر به شهر بردمت بر سر ره نمانمت
\\
هیچ مگو و کف مکن سر مگشای دیگ را
&&
نیک بجوش و صبر کن زانک همی‌پرانمت
\\
نی که تو شیرزاده‌ای در تن آهوی نهان
&&
من ز حجاب آهوی یک رهه بگذرانمت
\\
گوی منی و می‌دوی در چوگان حکم من
&&
در پی تو همی‌دوم گر چه که می‌دوانمت
\\
\end{longtable}
\end{center}
