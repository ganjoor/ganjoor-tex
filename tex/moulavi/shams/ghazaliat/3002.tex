\begin{center}
\section*{غزل شماره ۳۰۰۲: شوری فتاد در فلک ای مه چه شسته‌ای}
\label{sec:3002}
\addcontentsline{toc}{section}{\nameref{sec:3002}}
\begin{longtable}{l p{0.5cm} r}
شوری فتاد در فلک ای مه چه شسته‌ای
&&
پرنور کن تو خیمه و خرگه چه شسته‌ای
\\
آگاه نیستند مگر این فسردگان
&&
از آتش تو ای بت آگه چه شسته‌ای
\\
آتش خوران ره به سر کوی منتظر
&&
با مردمان زیرک ابله چه شسته‌ای
\\
دل شیر بیشه‌ست ولیکن سرش تویی
&&
دل لشکر حقست و تویی شه چه شسته‌ای
\\
ای جان تیزگوش تو بشنو هم از درون
&&
هم ره به توست بر سر هر ره چه شسته‌ای
\\
هین کز فراخنای دلت تا به عرش رفت
&&
هیهای وصل و خنده و قهقه چه شسته‌ای
\\
دی بامداد دامن جانم گرفت دل
&&
کان جان و دل رسید تو آوه چه شسته‌ای
\\
دولاب دولتست ز تبریز شمس دین
&&
درزن تو دست‌ها و در این ره چه شسته‌ای
\\
\end{longtable}
\end{center}
