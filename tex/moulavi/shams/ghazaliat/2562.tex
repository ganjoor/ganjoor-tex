\begin{center}
\section*{غزل شماره ۲۵۶۲: یکی فرهنگ دیگر نو برآر ای اصل دانایی}
\label{sec:2562}
\addcontentsline{toc}{section}{\nameref{sec:2562}}
\begin{longtable}{l p{0.5cm} r}
یکی فرهنگ دیگر نو برآر ای اصل دانایی
&&
ببین تو چاره‌ای از نو که الحق سخت بینایی
\\
بسی دل‌ها چو گوهرها ز نور لعل تو تابان
&&
بسی طوطی که آموزند از قندت شکرخایی
\\
زدی طعنه که دود تو ندارد آتش عاشق
&&
گر آتش نیستش حقی وگر دارد چه فرمایی
\\
برو ای جان دولت جو چه خواهم کرد دولت را
&&
من و عشق و شب تیره نگار و باده پیمایی
\\
بیا ای مونس روزم نگفتم دوش در گوشت
&&
که عشرت در کمی خندد تو کم زن تا بیفزایی
\\
دلا آخر نمی‌گویی کجا شد مکر و دستانت
&&
چو جام از دست جان نوشی از آن بی‌دست و بی‌پایی
\\
به هر شب شمس تبریزی چه گوهرها که می‌بیزی
&&
چه سلطانی چه جان بخشی چه خورشیدی چه دریایی
\\
\end{longtable}
\end{center}
