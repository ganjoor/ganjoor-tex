\begin{center}
\section*{غزل شماره ۲۶۵۳: گر این سلطان ما را بنده باشی}
\label{sec:2653}
\addcontentsline{toc}{section}{\nameref{sec:2653}}
\begin{longtable}{l p{0.5cm} r}
گر این سلطان ما را بنده باشی
&&
همه گریند و تو در خنده باشی
\\
وگر غم پر شود اطراف عالم
&&
تو شاد و خرم و فرخنده باشی
\\
وگر چرخ و زمین از هم بدرد
&&
ورای هر دو جانی زنده باشی
\\
به هفتم چرخ نوبت پنج داری
&&
چو خیمه شش جهت برکنده باشی
\\
همه مشتاق دیدار تو باشند
&&
تو صد پرده فروافکنده باشی
\\
چو اندیشه به جاسوسی اسرار
&&
درون سینه‌ها گردنده باشی
\\
دلا بر چشم خوبان چهره بگشا
&&
که اندیشد که تو شرمنده باشی
\\
بدیشان صدقه می‌ده چون هلالند
&&
تو بدری از کجا گیرنده باشی
\\
اگر خالی شوی از خویش چون نی
&&
چو نی پر از شکر آکنده باشی
\\
برو خرقه گرو کن در خرابات
&&
چو سالوسان چرا در ژنده باشی
\\
به عشق شمس تبریزی بده جان
&&
که تا چون عشق او پاینده باشی
\\
\end{longtable}
\end{center}
