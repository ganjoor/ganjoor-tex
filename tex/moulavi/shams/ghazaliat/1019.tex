\begin{center}
\section*{غزل شماره ۱۰۱۹: ما را خدا از بهر چه آورد بهر شور و شر}
\label{sec:1019}
\addcontentsline{toc}{section}{\nameref{sec:1019}}
\begin{longtable}{l p{0.5cm} r}
ما را خدا از بهر چه آورد بهر شور و شر
&&
دیوانگان را می‌کند زنجیر او دیوانه‌تر
\\
ای عشق شوخ بوالعجب آورده جان را در طرب
&&
آری درآ هر نیم شب بر جان مست بی‌خبر
\\
ما را کجا باشد امان کز دست این عشق آسمان
&&
ماندست اندر خرکمان چون عاشقان زیر و زبر
\\
ای عشق خونم خورده‌ای صبر و قرارم برده‌ای
&&
از فتنه روز و شبت پنهان شدستم چون سحر
\\
در لطف اگر چون جان شوم از جان کجا پنهان شوم
&&
گر در عدم غلطان شوم اندر عدم داری نظر
\\
ما را که پیدا کرده‌ای نی از عدم آورده‌ای
&&
ای هر عدم صندوق تو ای در عدم بگشاده در
\\
هستی خوش و سرمست تو گوش عدم در دست تو
&&
هر دو طفیل هست تو بر حکم تو بنهاده سر
\\
کاشانه را ویرانه کن فرزانه را دیوانه کن
&&
وان باده در پیمانه کن تا هر دو گردد بی‌خطر
\\
ای عشق چست معتمد مستی سلامت می‌کند
&&
بشنو سلام مست خود دل را مکن همچون حجر
\\
چون دست او بشکسته‌ای چون خواب او بربسته‌ای
&&
بشکن خمار مست را بر کوی مستان برگذر
\\
\end{longtable}
\end{center}
