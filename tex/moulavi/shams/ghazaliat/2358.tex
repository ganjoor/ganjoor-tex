\begin{center}
\section*{غزل شماره ۲۳۵۸: ماییم و دو چشم و جان خیره}
\label{sec:2358}
\addcontentsline{toc}{section}{\nameref{sec:2358}}
\begin{longtable}{l p{0.5cm} r}
ماییم و دو چشم و جان خیره
&&
بنگر تو به عاشقان خیره
\\
تو چون مه و ما به گرد رویت
&&
سرگشته چو آسمان خیره
\\
عقل است شبان به گرد احوال
&&
فریاد از این شبان خیره
\\
در دیده هزار شمع رخشان
&&
وین دیده چو شمعدان خیره
\\
از شرق به غرب موج نور است
&&
سر می‌کند از نهان خیره
\\
بیرون ز جهان مرده شاهی است
&&
وز عشق یکی جهان خیره
\\
گویی که مرا از او نشان ده
&&
خیره چه دهد نشان خیره
\\
از چشم سیه سپید پرخون
&&
کز چشم بود زبان خیره
\\
در روی صلاح دین تو بنگر
&&
تا دریابی بیان خیره
\\
\end{longtable}
\end{center}
