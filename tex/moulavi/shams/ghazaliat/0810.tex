\begin{center}
\section*{غزل شماره ۸۱۰: باز شیری با شکر آمیختند}
\label{sec:0810}
\addcontentsline{toc}{section}{\nameref{sec:0810}}
\begin{longtable}{l p{0.5cm} r}
باز شیری با شکر آمیختند
&&
عاشقان با همدگر آمیختند
\\
روز و شب را از میان برداشتند
&&
آفتابی با قمر آمیختند
\\
رنگ معشوقان و رنگ عاشقان
&&
جمله همچون سیم و زر آمیختند
\\
چون بهار سرمدی حق رسید
&&
شاخ خشک و شاخ تر آمیختند
\\
رافضی انگشت در دندان گرفت
&&
هم علی و هم عمر آمیختند
\\
بر یکی تختند این دم هر دو شاه
&&
بلک خود در یک کمر آمیختند
\\
هم شب قدر آشکارا شد چو عید
&&
هم فرشته با بشر آمیختند
\\
هم زبان همدگر آموختند
&&
بی نفور این دو نفر آمیختند
\\
نفس کل و هر چه زاد از نفس کل
&&
همچو طفلان با پدر آمیختند
\\
خیر و شر و خشک و تر زان هست شد
&&
کز طبیعت خیر و شر آمیختند
\\
من دهان بستم تو باقی را بدان
&&
کاین نظر با آن نظر آمیختند
\\
بهر نور شمس تبریزی تنم
&&
شمع وارش با شرر آمیختند
\\
\end{longtable}
\end{center}
