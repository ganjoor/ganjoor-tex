\begin{center}
\section*{غزل شماره ۲۱۸۵: چو بگشادم نظر از شیوه تو}
\label{sec:2185}
\addcontentsline{toc}{section}{\nameref{sec:2185}}
\begin{longtable}{l p{0.5cm} r}
چو بگشادم نظر از شیوه تو
&&
بشد کارم چو زر از شیوه تو
\\
تویی خورشید و من چون میوه خام
&&
به هر دم پخته‌تر از شیوه تو
\\
چو زهره می‌نوازم چنگ عشرت
&&
شب و روز ای قمر از شیوه تو
\\
به هر دم صد هزار اجزای مرده
&&
شود چون جانور از شیوه تو
\\
چرا ازرق قبای چرخ گردون
&&
چنین بندد کمر از شیوه تو
\\
چرا روی شفق سرخ است هر شام
&&
به خونابه جگر از شیوه تو
\\
ز شیوه ماهت استاره همی‌جست
&&
گرفتم من بصر از شیوه تو
\\
به خوبی همچو تو خود این محال است
&&
چنان خوبی به سر از شیوه تو
\\
ز انبوهی نباشد جان سوزن
&&
ز عاشق وین حشر از شیوه تو
\\
عجب چون آمد اندر عالم عشق
&&
هزاران شور و شر از شیوه تو
\\
اگر نه پرده آویزی به هر دم
&&
بدرد این بشر از شیوه تو
\\
اگر غفلت نباشد جمله عالم
&&
شود زیر و زبر از شیوه تو
\\
چرایم شمس تبریزی چو شیدا
&&
به گرد بام و در از شیوه تو
\\
\end{longtable}
\end{center}
