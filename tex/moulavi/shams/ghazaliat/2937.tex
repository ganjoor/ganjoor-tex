\begin{center}
\section*{غزل شماره ۲۹۳۷: از بهر مرغ خانه چون خانه‌ای بسازی}
\label{sec:2937}
\addcontentsline{toc}{section}{\nameref{sec:2937}}
\begin{longtable}{l p{0.5cm} r}
از بهر مرغ خانه چون خانه‌ای بسازی
&&
اشتر در او نگنجد با آن همه درازی
\\
آن مرغ خانه عقل است و آن خانه این تن تو
&&
اشتر جمال عشق است با قد و سرفرازی
\\
رطل گران شه را این مرغ برنتابد
&&
بویی کز او بیابی صد مغز را ببازی
\\
از ما مجوی جانا اسرار این حقیقت
&&
زیرا که غرق غرقم از نکته مجازی
\\
من هیکلی بدیدم اسرار عشق در وی
&&
کردم حمایل آن را از روی لاغ و بازی
\\
تا شد گرانترک شد آن هیکل خدایی
&&
تا برنتابد آن را پشت هزار تازی
\\
شد پرده‌ام دریده تا پرده‌ها بسوزم
&&
از آتشی که خیزد در پرده حجازی
\\
چون عشق او بغرد وین پرده‌ها بدرد
&&
با شمس حق تبریز در وقت عشقبازی
\\
\end{longtable}
\end{center}
