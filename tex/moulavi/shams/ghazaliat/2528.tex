\begin{center}
\section*{غزل شماره ۲۵۲۸: مروت نیست در سرها که اندازند دستاری}
\label{sec:2528}
\addcontentsline{toc}{section}{\nameref{sec:2528}}
\begin{longtable}{l p{0.5cm} r}
مروت نیست در سرها که اندازند دستاری
&&
کجا گیرد نظام ای جان به صرفه خشک بازاری
\\
رها کن گرگ خونی را که رو نارد بدان صیدی
&&
رها کن صرفه جویی را که برناید بدین کاری
\\
چه باشد زر چه باشد جان چه باشد گوهر و مرجان
&&
چو نبود خرج سودایی فدای خوبی یاری
\\
ز بخل ار طوق زر دارم مرا غلی بود غلی
&&
وگر خلخال زر دارم مرا خاری بود خاری
\\
برو ای شاخ بی‌میوه تهی می‌گرد چون چرخی
&&
شدستی پاسبان زر هلا می‌پیچ چون ماری
\\
تو زر سرخ می‌گویش که او زرد است و رنجوری
&&
تو خواجه شهر می‌خوانش که او را نیست شلواری
\\
چرا از بهر همدردان نبازم سیم چون مردان
&&
چرا چون شربت شافی نباشم نوش بیماری
\\
نتانم بد کم از چنگی حریف هر دل تنگی
&&
غذای گوش‌ها گشته به هر زخمی و هر تاری
\\
نتانم بد کم از باده ز ینبوع طرب زاده
&&
صلای عیش می‌گوید به هر مخمور و خماری
\\
کرم آموز تو یارا ز سنگ مرمر و خارا
&&
که می‌جوشد ز هر عرقش عطابخشی و ایثاری
\\
چگونه میر و سرهنگی که ننگ صخره و سنگی
&&
چگونه شیر حق باشد اسیر نفس سگساری
\\
خمش کردم که رب دین نهان‌ها را کند تعیین
&&
نماید شاخ زشتش را وگر چه هست ستاری
\\
\end{longtable}
\end{center}
