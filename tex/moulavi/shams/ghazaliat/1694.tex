\begin{center}
\section*{غزل شماره ۱۶۹۴: بازآمدم خرامان تا پیش تو بمیرم}
\label{sec:1694}
\addcontentsline{toc}{section}{\nameref{sec:1694}}
\begin{longtable}{l p{0.5cm} r}
بازآمدم خرامان تا پیش تو بمیرم
&&
ای بارها خریده از غصه و زحیرم
\\
من چون زمین خشکم لطف تو ابر و مشکم
&&
جز رعد تو نخواهم جز جعد تو نگیرم
\\
خوشتر اسیری تو صد بار از امیری
&&
خاصه دمی که گویی ای خسته دل اسیرم
\\
خاکی به تو رسیده به از زری رمیده
&&
خاصه دمی که گویی ای بی‌نوا فقیرم
\\
از ماجرا گذر کن گو عقل ماجرا را
&&
چنگ است ورد و ذکرم باده‌ست شیخ و پیرم
\\
ای جان جان مستان ای گنج تنگدستان
&&
در جنت جمالت من غرق شهد و شیرم
\\
من رستخیز دیدم وز خویش نابدیدم
&&
گر چون کمان خمیدم پرنده همچو تیرم
\\
خاکی بدم ز بادت بالا گرفت خاکم
&&
بی‌تو کجا روم من ای از تو ناگزیرم
\\
ای نور دیده و دین گفتی به عقل بنشین
&&
ای پرده‌ها دریده کی می هلی ستیزم
\\
من بنده الستم آن تو بوده استم
&&
آن خیره کش فراقت می راند خیر خیرم
\\
کی خندد این درختم بی‌نوبهار رویت
&&
کی دررسد فطیرم تا نسرشی خمیرم
\\
تا خوان تو بدیدم آزاد از ثریدم
&&
تا خویش تو بدیدم از خویش خود نفیرم
\\
از من گذر چو کردی از عقل و جان گذشتم
&&
در من اثر چو کردی بر گنبد اثیرم
\\
در قعده‌ام سلامی ای جان گزین من کن
&&
تا بی‌سلام نبود این قعده اخیرم
\\
من کف چرا نکوبم چون در کف است خوبم
&&
من پا چرا نکوبم چون بم شده‌ست زیرم
\\
تبریز شمس دین را از ما رسان تو خدمت
&&
خدمت به مشرقی به کز روش مستنیرم
\\
\end{longtable}
\end{center}
