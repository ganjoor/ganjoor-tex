\begin{center}
\section*{غزل شماره ۱۸۸۰: در زیر نقاب شب این زنگیکان را بین}
\label{sec:1880}
\addcontentsline{toc}{section}{\nameref{sec:1880}}
\begin{longtable}{l p{0.5cm} r}
در زیر نقاب شب این زنگیکان را بین
&&
با زنگیکان امشب در عشرت جان بنشین
\\
خلقان همه خوش خفته عشاق درآشفته
&&
اسرار به هم گفته شاباش زهی آیین
\\
یاران بشوریده با جان بسوزیده
&&
بگشاده دل و دیده در شاهد بی‌کابین
\\
چون عشق تو رامم شد این عشق حرامم شد
&&
چون زلف تو دامم شد شب گشت مرا مشکین
\\
شد زنگی شب مستی دستی همگان دستی
&&
در دیده هر هستی از دیده زنگی بین
\\
آن چرخ فرومانده کآبش بنگرداند
&&
این چرخ چه می داند کز چیست ورا تسکین
\\
می گردد آن مسکین نی مهر در او نی کین
&&
که کندن آن فرهاد از چیست جز از شیرین
\\
شه هندوی بنگی را آن مایه شنگی را
&&
آن خسرو زنگی را کرد حشری بر چین
\\
شمعی تو برافروزی شمس الحق تبریزی
&&
تا هندوی شب سوزی از روی چو صد پروین
\\
\end{longtable}
\end{center}
