\begin{center}
\section*{غزل شماره ۳۹۶: در ره معشوق ما ترسندگان را کار نیست}
\label{sec:0396}
\addcontentsline{toc}{section}{\nameref{sec:0396}}
\begin{longtable}{l p{0.5cm} r}
در ره معشوق ما ترسندگان را کار نیست
&&
جمله شاهانند آن جا بندگان را بار نیست
\\
گر تو نازی می‌کنی یعنی که من فرخنده‌ام
&&
نزد این اقبال ما فرخندگی جز عار نیست
\\
گر به فقرت ناز باشد ژنده برگیر و برو
&&
نزد این سلطان ما آن جمله جز زنار نیست
\\
گر تو نور حق شدی از شرق تا مغرب برو
&&
زانک ما را زین صفت پروای آن انوار نیست
\\
گر تو سر حق بدانستی برو با سر باش
&&
زانک این اسرار ما را خوی آن اسرار نیست
\\
راست شو در راه ما وین مکر را یک سوی نه
&&
زان که این میدان ما جولانگه مکار نیست
\\
شمس دین و شمس دین آن جان ما اینک بدان
&&
جز به سوی راه تبریز اسب ما رهوار نیست
\\
مست بودم فاش کردم سر خود با یارکان
&&
زانک هشیاری مرا خود مذهب آزار نیست
\\
گر نهی پرگار بر تن تا بدانی حد ما
&&
حد ما خود ای برادر لایق پرگار نیست
\\
خاک پاشی می‌کنی تو ای صنم در راه ما
&&
خاک پاشی دو عالم پیش ما در کار نیست
\\
صوفیان عشق را خود خانقاهی دیگر است
&&
جان ما را اندر آن جا کاسه و ادرار نیست
\\
در تک دوزخ نشستم ترک کردم بخت را
&&
زانک ما را اشتهای جنت و ابرار نیست
\\
\end{longtable}
\end{center}
