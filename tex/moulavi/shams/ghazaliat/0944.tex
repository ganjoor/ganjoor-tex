\begin{center}
\section*{غزل شماره ۹۴۴: به باغ بلبل از این پس نوای ما گوید}
\label{sec:0944}
\addcontentsline{toc}{section}{\nameref{sec:0944}}
\begin{longtable}{l p{0.5cm} r}
به باغ بلبل از این پس نوای ما گوید
&&
حدیث عشق شکرریز جان فزا گوید
\\
اگر ز رنگ رخ یار ما خبر دارد
&&
ز لاله زار و ز نسرین و گل چرا گوید
\\
ز راه غیرت گوید که تا بپوشاند
&&
رها کند سر چشمه حدیث پا گوید
\\
که پاره پاره به تدریج ذره که گردد
&&
فنا شود که اگر تند و بر ولا گوید
\\
کهی که ذره بود پیش او دو صد که قاف
&&
دوان دوان شود آن دم که او بیا گوید
\\
چو گوش کوه شنید آن بیای فرخ او
&&
به سر بیاید و لبیک را دو تا گوید
\\
به حق گلشن اقبال کاندر او مستی
&&
چو گل خموش که تا بلبلت ثنا گوید
\\
\end{longtable}
\end{center}
