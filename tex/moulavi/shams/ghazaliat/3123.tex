\begin{center}
\section*{غزل شماره ۳۱۲۳: به حیلت تو خواهی که در را ببندی}
\label{sec:3123}
\addcontentsline{toc}{section}{\nameref{sec:3123}}
\begin{longtable}{l p{0.5cm} r}
به حیلت تو خواهی که در را ببندی
&&
بنالی چو رنجور و سر را ببندی
\\
چو رنجور والله که آن زور داری
&&
که بر چرخ آیی قمر را ببندی
\\
گر آن روی چون مه به گردون نمایی
&&
به صبح جمالت سحر را ببندی
\\
غلام صبوحم ولی خصم صبحم
&&
که از بهر رفتن کمر را ببندی
\\
اگر گاو آرند پیشت سفیهان
&&
به یک نکته صد گاو و خر را ببندی
\\
به یک غمزه آهوان دو چشمت
&&
چو روبه کنی شیر نر را ببندی
\\
زمستان هجر آمد و ترسم آنست
&&
که سیلاب این چشم تر را ببندی
\\
وگر همچو خورشید ناگه بتابی
&&
بدین آب هر رهگذر را ببندی
\\
خموشم ولیکن روا نیست جانا
&&
که از حال زارم نظر را ببندی
\\
\end{longtable}
\end{center}
