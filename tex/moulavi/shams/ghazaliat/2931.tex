\begin{center}
\section*{غزل شماره ۲۹۳۱: از دلبر نهانی گر بوی جان بیابی}
\label{sec:2931}
\addcontentsline{toc}{section}{\nameref{sec:2931}}
\begin{longtable}{l p{0.5cm} r}
از دلبر نهانی گر بوی جان بیابی
&&
در صد جهان نگنجی گر یک نشان بیابی
\\
چون مهر جان پذیری بی‌لشکری امیری
&&
هم ملک غیب گیری هم غیب دان بیابی
\\
گنجی که تو شنیدی سودای آن گزیدی
&&
گر در زمین ندیدی در آسمان بیابی
\\
در عشق اگر امینی ای بس بتان چینی
&&
هم رایگان ببینی هم رایگان بیابی
\\
در آینه مبارک آن صاف صاف بی‌شک
&&
نقش بهشت یک یک هم در جهان بیابی
\\
چون تیر عشق خستت معشوق کرد مستت
&&
گر جان بشد ز دستت صد همچنان بیابی
\\
قفل طلسم مشکل سهلت شود به حاصل
&&
گر از وساوس دل یک دم امان بیابی
\\
درهم شکن بتان را از بهر شاه جان را
&&
تا نقش بند آن را اندر عیان بیابی
\\
تبریز در محقق از شمس ملت و حق
&&
در رمزهای مطلق صد ترجمان بیابی
\\
\end{longtable}
\end{center}
