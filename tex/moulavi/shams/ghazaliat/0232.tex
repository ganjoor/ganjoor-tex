\begin{center}
\section*{غزل شماره ۲۳۲: چو عشق را تو ندانی بپرس از شب‌ها}
\label{sec:0232}
\addcontentsline{toc}{section}{\nameref{sec:0232}}
\begin{longtable}{l p{0.5cm} r}
چو عشق را تو ندانی بپرس از شب‌ها
&&
بپرس از رخ زرد و ز خشکی لب‌ها
\\
چنان که آب حکایت کند ز اختر و ماه
&&
ز عقل و روح حکایت کنند قالب‌ها
\\
هزار گونه ادب جان ز عشق آموزد
&&
که آن ادب نتوان یافتن ز مکتب‌ها
\\
میان صد کس عاشق چنان بدید بود
&&
که بر فلک مه تابان میان کوکب‌ها
\\
خرد نداند و حیران شود ز مذهب عشق
&&
اگر چه واقف باشد ز جمله مذهب‌ها
\\
خضردلی که ز آب حیات عشق چشید
&&
کساد شد بر آن کس زلال مشرب‌ها
\\
به باغ رنجه مشو در درون عاشق بین
&&
دمشق و غوطه و گلزارها و نیرب‌ها
\\
دمشق چه که بهشتی پر از فرشته و حور
&&
عقول خیره در آن چهره‌ها و غبغب‌ها
\\
نه از نبیذ لذیذش شکوفه‌ها و خمار
&&
نه از حلاوت حلواش دمل و تب‌ها
\\
ز شاه تا به گدا در کشاکش طمعند
&&
به عشق بازرهد جان ز طمع و مطلب‌ها
\\
چه فخر باشد مر عشق را ز مشتریان
&&
چه پشت باشد مر شیر را ز ثعلب‌ها
\\
فراز نخل جهان پخته‌ای نمی‌یابم
&&
که کند شد همه دندانم از مذنب‌ها
\\
به پر عشق بپر در هوا و بر گردون
&&
چو آفتاب منزه ز جمله مرکب‌ها
\\
نه وحشتی دل عشاق را چو مفردها
&&
نه خوف قطع و جداییست چون مرکب‌ها
\\
عنایتش بگزیدست از پی جان‌ها
&&
مسببش بخریدست از مسبب‌ها
\\
وکیل عشق درآمد به صدر قاضی کاب
&&
که تا دلش برمد از قضا و از گب‌ها
\\
زهی جهان و زهی نظم نادر و ترتیب
&&
هزار شور درافکند در مرتب‌ها
\\
گدای عشق شمر هر چه در جهان طربیست
&&
که عشق چون زر کانست و آن مذهب‌ها
\\
سلبت قلبی یا عشق خدعه و دها
&&
کذبت حاشا لکن ملاحه و بها
\\
ارید ذکرک یا عشق شاکرا لکن
&&
و لهت فیک و شوشت فکرتی و نها
\\
به صد هزار لغت گر مدیح عشق کنم
&&
فزونترست جمالش ز جمله دب‌ها
\\
\end{longtable}
\end{center}
