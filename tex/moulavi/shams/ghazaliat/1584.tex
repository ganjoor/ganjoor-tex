\begin{center}
\section*{غزل شماره ۱۵۸۴: هر که گوید کان چراغ دیده‌ها را دیده‌ام}
\label{sec:1584}
\addcontentsline{toc}{section}{\nameref{sec:1584}}
\begin{longtable}{l p{0.5cm} r}
هر که گوید کان چراغ دیده‌ها را دیده‌ام
&&
پیش من نه دیده‌اش را کامتحان دیده‌ام
\\
چشم بد دور از خیالش دوشمان بس لطف کرد
&&
من پس گوش از خجالت تا سحر خاریده‌ام
\\
گر چه او عیار و مکار است گرد خویشتن
&&
از میان رخت او من نقدها دزدیده‌ام
\\
پای از دزدی کشیدم چونک دست از کار شد
&&
زانک دزدی دزدتر از خویشتن بشنیده‌ام
\\
جمله مرغان به پر و بال خود پریده‌اند
&&
من ز بال و پر خود بی‌بال و پر پریده‌ام
\\
من به سنگ خود همیشه جام خود بشکسته‌ام
&&
من به چنگ خود همیشه پرده‌ام بدریده‌ام
\\
من به ناخن‌های خود هم اصل خود برکنده‌ام
&&
من ز ابر چشم خود بر کشت جان باریده‌ام
\\
ای سیه دل لاله بر کشتم چرا خندیده‌ای
&&
نوبهارت وانماید آنچ من کاریده‌ام
\\
چون بهارم از بهار شمس تبریزی خدیو
&&
از درونم جمله خنده وز برون زاریده‌ام
\\
\end{longtable}
\end{center}
