\begin{center}
\section*{غزل شماره ۱۵۸۹: چشم بگشا جان نگر کش سوی جانان می برم}
\label{sec:1589}
\addcontentsline{toc}{section}{\nameref{sec:1589}}
\begin{longtable}{l p{0.5cm} r}
چشم بگشا جان نگر کش سوی جانان می برم
&&
پیش آن عید ازل جان بهر قربان می برم
\\
چون کبوترخانه جان‌ها از او معمور گشت
&&
پس چرا این زیره را من سوی کرمان می برم
\\
زانک هر چیزی به اصلش شاد و خندان می رود
&&
سوی اصل خویش جان را شاد و خندان می برم
\\
زیر دندان تا نیاید قند شیرین کی بود
&&
جان همچون قند را من زیر دندان می برم
\\
تا که زر در کان بود او را نباشد رونقی
&&
سوی زرگر اندک اندک زودش از کان می برم
\\
دود آتش کفر باشد نور او ایمان بود
&&
شمع جان را من ورای کفر و ایمان می برم
\\
سوی هر ابری که او منکر شود خورشید را
&&
آفتابی زیر دامن بهر برهان می برم
\\
شمس تبریز ارمغانم گوهر بحر دل است
&&
من ز شرم جان پاکت همچو عمان می برم
\\
\end{longtable}
\end{center}
