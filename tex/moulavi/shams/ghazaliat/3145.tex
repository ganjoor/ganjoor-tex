\begin{center}
\section*{غزل شماره ۳۱۴۵: آوخ آوخ چو من وفاداری}
\label{sec:3145}
\addcontentsline{toc}{section}{\nameref{sec:3145}}
\begin{longtable}{l p{0.5cm} r}
آوخ آوخ چو من وفاداری
&&
در تمنای چون تو خون خواری
\\
آوخ آوخ طبیب خون ریزی
&&
بر سر زار زار بیماری
\\
آن جفاها که کرده‌ای با من
&&
نکند هیچ یار با یاری
\\
گفتمش قصد خون من داری
&&
بی خطا و گناه گفت آری
\\
عشق جز بی‌گناه می‌نکشد
&&
نکشد عشق او گنه کاری
\\
هر زمان گلشنی همی‌سوزم
&&
تو چه باشی به پیش من خاری
\\
بشکستم هزار چنگ طرب
&&
تو چه باشی به چنگ من تاری
\\
شهرها از سپاه من ویران
&&
تو چه باشی شکسته دیواری
\\
گفتمش از کمینه بازی تو
&&
جان نبرده‌ست هیچ عیاری
\\
ای ز هر تار موی طره تو
&&
سرنگون سار بسته طراری
\\
گر ببازم وگر نه زین شه رخ
&&
ماتم و مات مات من باری
\\
آن که نخرید و آن که او بخرید
&&
شد پشیمان غریب بازاری
\\
و آن که بخرید گوید آن همه را
&&
کاش من بودمی خریداری
\\
و آن که نخرید دست می‌خاید
&&
ناامید و فتاده و خواری
\\
فرع بگرفته اصل افکنده
&&
جان بداده گرفته مرداری
\\
پا بریده به عشق نعلینی
&&
سر بداده به عشق دستاری
\\
با چنین مشتری کند صرفه
&&
از چنین باده مانده هشیاری
\\
خر علف زار تن گزید و بماند
&&
خر مردار در علف زاری
\\
\end{longtable}
\end{center}
