\begin{center}
\section*{غزل شماره ۴۱: شمع جهان دوش نبد نور تو در حلقه ما}
\label{sec:0041}
\addcontentsline{toc}{section}{\nameref{sec:0041}}
\begin{longtable}{l p{0.5cm} r}
شمع جهان دوش نبد نور تو در حلقه ما
&&
راست بگو شمع رخت دوش کجا بود کجا
\\
سوی دل ما بنگر کز هوس دیدن تو
&&
دولت آن جا که در او حسن تو بگشاد قبا
\\
دوش به هر جا که بدی دانم کامروز ز غم
&&
گشته بود همچو دلم مسجد لا حول و لا
\\
دوش همی‌گشتم من تا به سحر ناله کنان
&&
بدرک بالصبح بدا هیج نومی‌و نفی
\\
سایه نوری تو و ما جمله جهان سایه تو
&&
نور کی دیدست که او باشد از سایه جدا
\\
گاه بود پهلوی او گاه شود محو در او
&&
پهلوی او هست خدا محو در او هست لقا
\\
سایه زده دست طلب سخت در آن نور عجب
&&
تا چو بکاهد بکشد نور خدایش به خدا
\\
شرح جدایی و درآمیختگی سایه و نور
&&
لا یتناهی و لئن جئت بضعف مددا
\\
نور مسبب بود و هر چه سبب سایه او
&&
بی سببی قد جعل الله لکل سببا
\\
آینه همدگر افتاد مسبب و سبب
&&
هر کی نه چون آینه گشتست ندید آینه را
\\
\end{longtable}
\end{center}
