\begin{center}
\section*{غزل شماره ۶۲۱: در تابش خورشیدش رقصم به چه می‌باید}
\label{sec:0621}
\addcontentsline{toc}{section}{\nameref{sec:0621}}
\begin{longtable}{l p{0.5cm} r}
در تابش خورشیدش رقصم به چه می‌باید
&&
تا ذره چو رقص آید از منش به یاد آید
\\
شد حامله هر ذره از تابش روی او
&&
هر ذره از آن لذت صد ذره همی‌زاید
\\
در هاون تن بنگر کز عشق سبک روحی
&&
تا ذره شود خود را می‌کوبد و می‌ساید
\\
گر گوهر و مرجانی جز خرد مشو این جا
&&
زیرا که در این حضرت جز ذره نمی‌شاید
\\
در گوهر جان بنگر اندر صدف این تن
&&
کز دست گران جانی انگشت همی‌خاید
\\
چون جان بپرد از تو این گوهر زندانی
&&
چون ذره به اصلش شد خوانیش ولی ناید
\\
ور سخت شود بندش در خون بزند نقبی
&&
عمری برود در خون موییش نیالاید
\\
جز تا به چه بابل او را نبود منزل
&&
تا جان نشود جادو جایی بنیاساید
\\
تبریز ز برج تو گر تابد شمس الدین
&&
هم ابر شود چون مه هم ماه درافزاید
\\
\end{longtable}
\end{center}
