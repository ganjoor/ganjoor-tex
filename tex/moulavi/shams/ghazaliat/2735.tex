\begin{center}
\section*{غزل شماره ۲۷۳۵: برجه که بهار زد صلایی}
\label{sec:2735}
\addcontentsline{toc}{section}{\nameref{sec:2735}}
\begin{longtable}{l p{0.5cm} r}
برجه که بهار زد صلایی
&&
در باغ خرام چون صبایی
\\
از شاخ درخت گیر رقصی
&&
وز لاله و که شنو صدایی
\\
ریحان گوید به سبزه رازی
&&
بلبل طلبد ز گل نوایی
\\
از باد زند گیاه موجی
&&
در بحر هوای آشنایی
\\
وز ابر که حامله‌ست از بحر
&&
چون چشم عروس بین بکایی
\\
وز گریه ابر و خنده برق
&&
در سنبل و سرو ارتقایی
\\
فخ شسته به پیش گوش قمری
&&
کموزدش او بهانه‌هایی
\\
نرگس گوید به سوسن آخر
&&
برگوی تو هجو یا ثنایی
\\
ای سوسن صدزبان فروخوان
&&
بر مرغ حکایت همایی
\\
سوسن گوید خمش که مستم
&&
از جام میی گران بهایی
\\
سرمستم و بیخودم مبادا
&&
بجهد ز دهان من خطایی
\\
رو کن به شهی کز او بپوشید
&&
اشکوفه بریشمین قبایی
\\
می‌گوید بید سرفشانان
&&
رستیم ز دست اژدهایی
\\
ای سرو برای شکر این را
&&
تو نیز چنین بکوب پایی
\\
ای جان و جهان به تو رهیدیم
&&
ز اشکنجه جان جان نمایی
\\
از وسوسه چنین حریفی
&&
وز دغدغه چنین دغایی
\\
زان دی که بسی قفا بخوردیم
&&
رفت و بنمودمان قفایی
\\
ظاهر مشواد او که آمد
&&
از شوم ظهور او خفایی
\\
خاموش کن و نظاره می‌کن
&&
بی زحمت خوف در رجایی
\\
\end{longtable}
\end{center}
