\begin{center}
\section*{غزل شماره ۲۱۰: گر نه تهی باشدی بیشترین جوی‌ها}
\label{sec:0210}
\addcontentsline{toc}{section}{\nameref{sec:0210}}
\begin{longtable}{l p{0.5cm} r}
گر نه تهی باشدی بیشترین جوی‌ها
&&
خواجه چرا می‌دود تشنه در این کوی‌ها
\\
خم که در او باده نیست هست خم از باد پر
&&
خم پر از باد کی سرخ کند روی‌ها
\\
هست تهی خارها نیست در او بوی گل
&&
کور بجوید ز خار لطف گل و بوی‌ها
\\
با طلب آتشین روی چو آتش ببین
&&
بر پی دودش برو زود در این سوی‌ها
\\
در حجب مشک موی روی ببین اه چه روی
&&
آنک خدایش بشست دور ز روشوی‌ها
\\
بر رخ او پرده نیست جز که سر زلف او
&&
گاه چو چوگان شود گاه شود گوی‌ها
\\
از غلط عاشقان از تبش روی او
&&
صورت او می‌شود بر سر آن موی‌ها
\\
هی که بسی جان‌ها موی به مو بسته‌اند
&&
چون مگسان شسته‌اند بر سر چربوی‌ها
\\
باده چو از عقل برد رنگ ندارد رواست
&&
حسن تو چون یوسفیست تا چه کنم خوی‌ها
\\
آهوی آن نرگسش صید کند جز که شیر
&&
راست شود روح چون کژ کند ابروی‌ها
\\
مفخر تبریزیان شمس حق بی‌زیان
&&
توی به تو عشق توست باز کن این توی‌ها
\\
\end{longtable}
\end{center}
