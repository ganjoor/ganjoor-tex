\begin{center}
\section*{غزل شماره ۴۱۵: تشنه بر لب جو بین که چه در خواب شدست}
\label{sec:0415}
\addcontentsline{toc}{section}{\nameref{sec:0415}}
\begin{longtable}{l p{0.5cm} r}
تشنه بر لب جو بین که چه در خواب شدست
&&
بر سر گنج گدا بین که چه پرتاب شدست
\\
ای بسا خشک لبا کز گره سحر کسی
&&
در ارس بی‌خبر از آب چو دولاب شدست
\\
چشم بند ار نبدی که گرو شمع شدی
&&
کآفتاب سحری ناسخ مهتاب شدست
\\
ترسد ار شمع نباشد بنبیند مه را
&&
دل آن گول از این ترس چو سیماب شدست
\\
چون سلیمان نهان است که دیوانش دل است
&&
جان محجوب از او مفخر حجاب شدست
\\
ای بسا سنگ دلا که حجرش لعل شدست
&&
ای بسا غوره در این معصره دوشاب شدست
\\
این چه مشاطه و گلگونه غیب است کز او
&&
زعفرانی رخ عشاق چو عناب شدست
\\
چند عثمان پر از شرم که از مستی او
&&
چون عمر شرم شکن گشته و خطاب شدست
\\
طرفه قفال کز انفاس کند قفل و کلید
&&
من دکان بستم کو فاتح ابواب شدست
\\
\end{longtable}
\end{center}
