\begin{center}
\section*{غزل شماره ۱۵۴۶: رفتم تصدیع از جهان بردم}
\label{sec:1546}
\addcontentsline{toc}{section}{\nameref{sec:1546}}
\begin{longtable}{l p{0.5cm} r}
رفتم تصدیع از جهان بردم
&&
بیرون شدم از زحیر و جان بردم
\\
کردم بدرود همنشینان را
&&
جان را به جهان بی‌نشان بردم
\\
زین خانه شش دری برون رفتم
&&
خوش رخت به سوی لامکان بردم
\\
چون میر شکار غیب را دیدم
&&
چون تیر پریدم و کمان بردم
\\
چوگان اجل چو سوی من آمد
&&
من گوی سعادت از میان بردم
\\
از روزن من مهی عجب درتافت
&&
رفتم سوی بام و نردبان بردم
\\
این بام فلک که مجمع جان‌هاست
&&
ز آن خوشتر بد که من گمان بردم
\\
شاخ گل من چو گشت پژمرده
&&
بازش سوی باغ و گلستان بردم
\\
چون مشتریی نبود نقدم را
&&
زودش سوی اصل اصل کان بردم
\\
زین قلب زنان قراضه جان را
&&
هم جانب زرگر ارمغان بردم
\\
در غیب جهان بی‌کران دیدم
&&
آلاجق خود بدان کران بردم
\\
بر من مگری که زین سفر شادم
&&
چون راه به خطه جنان بردم
\\
این نکته نویس بر سر گورم
&&
که سر ز بلا و امتحان بردم
\\
خوش خسپ تنا در این زمین که من
&&
پیغام تو سوی آسمان بردم
\\
بربند زنخ که من فغان‌ها را
&&
سرجمله به خالق فغان بردم
\\
زین بیش مگو غم دل ایرا من
&&
دل را به جناب غیب دان بردم
\\
\end{longtable}
\end{center}
