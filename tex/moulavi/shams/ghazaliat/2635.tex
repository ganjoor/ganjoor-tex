\begin{center}
\section*{غزل شماره ۲۶۳۵: امروز سماع است و مدام است و سقایی}
\label{sec:2635}
\addcontentsline{toc}{section}{\nameref{sec:2635}}
\begin{longtable}{l p{0.5cm} r}
امروز سماع است و مدام است و سقایی
&&
گردان شده بر جمع قدح‌های عطایی
\\
فرمان سقی الله رسیده‌ست بنوشید
&&
ای تن همه جان شو نه که ز اخوان صفایی
\\
ای دور چه دوری تو و ای روز چه روزی
&&
وی گلشن اقبال چه بابرگ و نوایی
\\
از خاک برویند در این دور خلایق
&&
کاین نفخه صور است که کرده‌ست صدایی
\\
از کوه شنو نعره صد ناقه صالح
&&
وز چرخ شنو بانگ سرافیل صلایی
\\
هین رخت فروگیر و بخوابان شتران را
&&
آخر بگشا چشم که در دست رضایی
\\
ای مرده بشو زنده و ای پیر جوان شو
&&
وی منکر محشر هله تا ژاژ نخایی
\\
خواهم سخنی گفت دهانم بمبندید
&&
کامروز حلال است ورا رازگشایی
\\
ور ز آنک ز غیرت ره این گفت ببندید
&&
ره باز کنم سوی خیالات هوایی
\\
ما نیز خیالات بدستیم و از این دم
&&
هستی پذرفتیم ز دم‌های خدایی
\\
صد هستی دیگر به جز این هست بگیری
&&
کاین را تو فراموش کنی خواجه کجایی
\\
\end{longtable}
\end{center}
