\begin{center}
\section*{غزل شماره ۲۵۲۴: اگر آب و گل ما را چو جان و دل پری بودی}
\label{sec:2524}
\addcontentsline{toc}{section}{\nameref{sec:2524}}
\begin{longtable}{l p{0.5cm} r}
اگر آب و گل ما را چو جان و دل پری بودی
&&
به تبریز آمدی این دم بیابان را بپیمودی
\\
بپر ای دل که پر داری برو آن جا که بیماری
&&
نماندی هیچ بیماری گر او رخسار بنمودی
\\
چه کردی آن دل مسکین اگر چون تن گران بودی
&&
اگر پرش ببخشیدی بر او دلبر ببخشودی
\\
دریغا قالبم را هم ز بخشش نیم پر بودی
&&
که بر تبریزیان در ره دواسپه او برافزودی
\\
مبارک بادشان این ره به توفیق و امان الله
&&
به هر شهری و هر جایی به هر دشتی و هر رودی
\\
دلم همراه ایشان شد که شبشان پاسبان باشد
&&
اگر پیدا بدی پاسش یکی همراه نغنودی
\\
بپرید ای شهان آن سو که یابید آنچ قسمت شد
&&
نحاسی را ز اکسیری ایازی را ز محمودی
\\
روید ای عاشقان حق به اقبال ابد ملحق
&&
روان باشید همچون مه به سوی برج مسعودی
\\
به برج عاشقان شه میان صادقان ره
&&
که از سردان و مردودان شود جوینده مردودی
\\
بپر ای دل به پنهانی به پر و بال روحانی
&&
گرت طالب نبودی شه چنین پرهات نگشودی
\\
در احسان سابق است آن شه به وعده صادق است آن شه
&&
اگر نه خالق است آن شه تو را از خلق نربودی
\\
برون از نور و دود است او که افروزید این آتش
&&
از این آتش خرد نوری از این آذر هوا دودی
\\
دلا اندر چه وسواسی که دود از نور نشناسی
&&
بسوز از عشق نور او درون نار چون عودی
\\
نه از اولاد نمرودی که بسته آتش و دودی
&&
چو فرزند خلیلی تو مترس از دود نمرودی
\\
در آتش باش جان من یکی چندی چو نرم آهن
&&
که گر آتش نبودی خود رخ آیینه که زدودی
\\
چه آسان می‌شود مشکل به نور پاک اهل دل
&&
چنانک آهن شود مومی ز کف شمع داوودی
\\
ز شمس الدین شناس ای دل چو بر تو حل شود مشکل
&&
تجلی بهر موسی دان به جودی که رسد جودی
\\
\end{longtable}
\end{center}
