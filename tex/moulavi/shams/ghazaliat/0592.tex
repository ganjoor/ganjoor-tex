\begin{center}
\section*{غزل شماره ۵۹۲: اگر چرخ وجود من از این گردش فروماند}
\label{sec:0592}
\addcontentsline{toc}{section}{\nameref{sec:0592}}
\begin{longtable}{l p{0.5cm} r}
اگر چرخ وجود من از این گردش فروماند
&&
بگرداند مرا آن کس که گردون را بگرداند
\\
اگر این لشکر ما را ز چشم بد شکست افتد
&&
به امر شاه لشکرها از آن بالا فروآید
\\
اگر باد زمستانی کند باغ مرا ویران
&&
بهار شهریار من ز دی انصاف بستاند
\\
شمار برگ اگر باشد یکی فرعون جباری
&&
کف موسی یکایک را به جای خویش بنشاند
\\
مترسان دل مترسان دل ز سختی‌های این منزل
&&
که آب چشمه حیوان بتا هرگز نمیراند
\\
رایناکم رایناکم و اخرجنا خفایاکم
&&
فان لم تنتهوا عنها فایانا و ایاکم
\\
و ان طفتم حوالینا و انتم نور عینانا
&&
فلا تستیاسوا منان فان العیش احیاکم
\\
شکسته بسته تازی‌ها برای عشقبازی‌ها
&&
بگویم هر چه من گویم شهی دارم که بستاند
\\
چو من خود را نمی‌یابم سخن را از کجا یابم
&&
همان شمعی که داد این را همو شمعم بگیراند
\\
\end{longtable}
\end{center}
