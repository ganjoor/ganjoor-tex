\begin{center}
\section*{غزل شماره ۱۲۷۱: باز درآمد طبیب از در رنجور خویش}
\label{sec:1271}
\addcontentsline{toc}{section}{\nameref{sec:1271}}
\begin{longtable}{l p{0.5cm} r}
باز درآمد طبیب از در رنجور خویش
&&
دست عنایت نهاد بر سر مهجور خویش
\\
بار دگر آن حبیب رفت بر آن غریب
&&
تا جگر او کشید شربت موفور خویش
\\
شربت او چون ربود گشت فنا از وجود
&&
ساقی وحدت بماند ناظر و منظور خویش
\\
نوش ورا نیش نیست ور بودش راضیم
&&
نیست عسل خواره را چاره ز زنبور خویش
\\
این شب هجران دراز با تو بگویم چراست
&&
فتنه شد آن آفتاب بر رخ مستور خویش
\\
غفلت هر دلبری از رخ خود رحمتست
&&
ور نه ببستی نقاب بر رخ مشهور خویش
\\
عاشق حسن خودی لیک تو پنهان ز خود
&&
خلعت وصلت بپوش بر تن این عور خویش
\\
شکر که خورشید عشق رفت به برج حمل
&&
در دل و جان‌ها فکند پرورش نور خویش
\\
شکر که موسی برست از همه فرعونیان
&&
باز به میقات وصل آمد بر طور خویش
\\
عیسی جان دررسید بر سر عازر دمید
&&
عازر از افسون او حشر شد از گور خویش
\\
باز سلیمان رسید دیو و پری جمع شد
&&
بر همه شان عرضه کرد خاتم و منشور خویش
\\
ساقی اگر بایدت تا کنم این را تمام
&&
باده گویا بنه بر لب مخمور خویش
\\
\end{longtable}
\end{center}
