\begin{center}
\section*{غزل شماره ۱۳۰۷: ای مونس و غمگسار عاشق}
\label{sec:1307}
\addcontentsline{toc}{section}{\nameref{sec:1307}}
\begin{longtable}{l p{0.5cm} r}
ای مونس و غمگسار عاشق
&&
وی چشم و چراغ و یار عاشق
\\
ای داروی فربهی و صحت
&&
از بهر تن نزار عاشق
\\
ای رحمت و پادشاهی تو
&&
بربوده دل و قرار عاشق
\\
ای کرده خیال را رسولی
&&
در واسطه یادگار عاشق
\\
آن را که به خویش بار ندهی
&&
کی بیند کار و بار عاشق
\\
از جذب و کشیدن تو باشد
&&
آن ناله زار زار عاشق
\\
تعلیم و اشارت تو باشد
&&
آن حیله گری و کار عاشق
\\
از راه نمودن تو باشد
&&
آن رفتن راهوار عاشق
\\
ای بند تو دلگشای عاشق
&&
وی پند تو گوشوار عاشق
\\
دیرست که خواب شب نمانده است
&&
در دیده شرمسار عاشق
\\
دیرست که اشتها برفتست
&&
از معده لقمه خوار عاشق
\\
دیرست که زعفران برستست
&&
از چهره لاله زار عاشق
\\
دیرست کز آب‌های دیده
&&
دریا کردی کنار عاشق
\\
زین‌ها چه زیانش چون تو باشی
&&
چاره گر و غمگسار عاشق
\\
صد گنج فروشیش به دانگی
&&
وان دانگ کنی نثار عاشق
\\
ای لاف ابیت عند ربی
&&
آرایش و افتخار عاشق
\\
لو لاک لما خلقت الافلاک
&&
نه چرخ به اختیار عاشق
\\
بس کن که عنایتش بسنده است
&&
برهان و سخن گزار عاشق
\\
\end{longtable}
\end{center}
