\begin{center}
\section*{غزل شماره ۳۷۵: مر عاشق را ز ره چه بیمست}
\label{sec:0375}
\addcontentsline{toc}{section}{\nameref{sec:0375}}
\begin{longtable}{l p{0.5cm} r}
مر عاشق را ز ره چه بیمست
&&
چون همره عاشق آن قدیمست
\\
از رفتن جان چه خوف باشد
&&
او را که خدای جان ندیمست
\\
اندر سفرست لیک چون مه
&&
در طلعت خوب خود مقیمست
\\
کی منتظر نسیم باشد
&&
آن کس که سبکتر از نسیمست
\\
عشق و عاشق یکی‌ست ای جان
&&
تا ظن نبری که آن دو نیمست
\\
چون گشت درست عشق عاشق
&&
هم منعم خویش و هم نعیمست
\\
او در طلب چنین درستی
&&
در پیش سهیل چون ادیمست
\\
چون رفت در این طلب به دریا
&&
دری‌ست اگر چه او یتیمست
\\
ای دیده کرم ز شمس تبریز
&&
مر حاتم را مگو کریمست
\\
\end{longtable}
\end{center}
