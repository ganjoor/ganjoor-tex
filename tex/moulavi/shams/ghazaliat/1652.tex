\begin{center}
\section*{غزل شماره ۱۶۵۲: ای دریغا که شب آمد همه از هم ببریم}
\label{sec:1652}
\addcontentsline{toc}{section}{\nameref{sec:1652}}
\begin{longtable}{l p{0.5cm} r}
ای دریغا که شب آمد همه از هم ببریم
&&
مجلس آخر شد و ما تشنه و مخمورسریم
\\
رفت این روز دراز و در حس گشت فراز
&&
ز اول روز خماریم به شب زان بتریم
\\
باطن ما چو فلک تا به ابد مستسقی است
&&
گر چه روزی دو سه در نقش و نگار بشریم
\\
معده گاو گرفته‌ست ره معده دل
&&
ور نه در مرج بقا صاحب جوع بقریم
\\
نزد یزدان نه صباح است برادر نه مسا
&&
چیز دیگر بود و ما تبع آن دگریم
\\
همه زندان جهان پر ز نگارست و نقوش
&&
همه محبوس نقوش و وثنات صوریم
\\
کوزه‌ها دان تو صور را و ز هر شربت فکر
&&
همچو کوزه همه هر لحظه تهی ایم و پریم
\\
نفسی پر ز سماع و نفسی پر ز نزاع
&&
نفسی لست ابالی نفسی نفع و ضریم
\\
شربت از کوزه نروید بود از جای دگر
&&
همچو کوزه ز اصول مددش بی‌خبریم
\\
از دهنده نظر ار چه که نظر محجوب است
&&
زان است محجوب که ما غرق دهنده نظریم
\\
آن چنانک نتوان دید ز بعد مفرط
&&
سبب قربت مفرط معزول از بصریم
\\
گه ز تمزیج جمادات چو یخ منجمدیم
&&
گه در آن شیر گدازنده مثال شکریم
\\
اگر این یخ نرود زان است که خورشید رمید
&&
وگر آن مه نرسد زان است که بند اگریم
\\
گر چه دل را ز لقا بر جگرش آبی نیست
&&
متصل با کرم دوست چو آب و جگریم
\\
چو مهندس جهت جان وطن غیبی ساخت
&&
با مهندس ز درون هندسه‌ای برشمریم
\\
چو سلیمان اگر او تاج نهد بر سر ما
&&
همچو مور از پی شکرش همه بسته کمریم
\\
از زکاتی که فرستد بر ما آن خورشید
&&
قمر اندر قمر اندر قمر اندر قمریم
\\
وز سحابی که فرستد بر ما آن دریا
&&
گهر اندر گهر اندر گهر اندر گهریم
\\
زان بهاری که خزانی نبود در پی او
&&
همه سرسبز و فزاینده چو سرو و شجریم
\\
جان چو روز است و تن ما چو شب و ما به میان
&&
واسطه روز و شب خویش مثال سحریم
\\
من خمش کردم ای خواجه ولیکن زنهار
&&
هله منگر سوی ما سست که احدی الکبریم
\\
\end{longtable}
\end{center}
