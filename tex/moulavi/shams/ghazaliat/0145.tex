\begin{center}
\section*{غزل شماره ۱۴۵: ای وصالت یک زمان بوده فراقت سال‌ها}
\label{sec:0145}
\addcontentsline{toc}{section}{\nameref{sec:0145}}
\begin{longtable}{l p{0.5cm} r}
ای وصالت یک زمان بوده فراقت سال‌ها
&&
ای به زودی بار کرده بر شتر احمال‌ها
\\
شب شد و درچین ز هجران رخ چون آفتاب
&&
درفتاده در شب تاریک بس زلزال‌ها
\\
چون همی‌رفتی به سکته حیرتی حیران بدم
&&
چشم باز و من خموش و می‌شد آن اقبال‌ها
\\
ور نه سکته بخت بودی مر مرا خود آن زمان
&&
چهره خون آلود کردی بردریدی شال‌ها
\\
بر سر ره جان و صد جان در شفاعت پیش تو
&&
در زمان قربان بکردی خود چه باشد مال‌ها
\\
تا بگشتی در شب تاریک ز آتش نال‌ها
&&
تا چو احوال قیامت دیده شد اهوال‌ها
\\
تا بدیدی دل عذابی گونه گونه در فراق
&&
سنگ خون گرید اگر زان بشنود احوال‌ها
\\
قدها چون تیر بوده گشته در هجران کمان
&&
اشک خون آلود گشت و جمله دل‌ها دال‌ها
\\
چون درستی و تمامی شاه تبریزی بدید
&&
در صف نقصان نشست است از حیا مثقال‌ها
\\
از برای جان پاک نورپاش مه وشت
&&
ای خداوند شمس دین تا نشکنی آمال‌ها
\\
از مقال گوهرین بحر بی‌پایان تو
&&
لعل گشته سنگ‌ها و ملک گشته حال‌ها
\\
حال‌های کاملانی کان ورای قال‌هاست
&&
شرمسار از فر و تاب آن نوادر قال‌ها
\\
ذره‌های خاک‌هامون گر بیابد بوی او
&&
هر یکی عنقا شود تا برگشاید بال‌ها
\\
بال‌ها چون برگشاید در دو عالم ننگرد
&&
گرد خرگاه تو گردد واله اجمال‌ها
\\
دیده نقصان ما را خاک تبریز صفا
&&
کحل بادا تا بیابد زان بسی اکمال‌ها
\\
چونک نورافشان کنی درگاه بخشش روح را
&&
خود چه پا دارد در آن دم رونق اعمال‌ها
\\
خود همان بخشش که کردی بی‌خبر اندر نهان
&&
می‌کند پنهان پنهان جمله افعال‌ها
\\
ناگهان بیضه شکافد مرغ معنی برپرد
&&
تا هما از سایه آن مرغ گیرد فال‌ها
\\
هم تو بنویس ای حسام الدین و می‌خوان مدح او
&&
تا به رغم غم ببینی بر سعادت خال‌ها
\\
گر چه دست افزار کارت شد ز دستت باک نیست
&&
دست شمس الدین دهد مر پات را خلخال‌ها
\\
\end{longtable}
\end{center}
