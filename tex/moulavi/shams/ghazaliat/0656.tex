\begin{center}
\section*{غزل شماره ۶۵۶: مرغان که کنون از قفص خویش جدایید}
\label{sec:0656}
\addcontentsline{toc}{section}{\nameref{sec:0656}}
\begin{longtable}{l p{0.5cm} r}
مرغان که کنون از قفس خویش جدایید
&&
رخ باز نمایید و بگویید کجایید
\\
کشتی شما ماند بر این آب شکسته
&&
ماهی صفتان یک دم از این آب برآیید
\\
یا قالب بشکست و بدان دوست رسیدست
&&
یا دام بشد از کف و از صید جدایید
\\
امروز شما هیزم آن آتش خویشید
&&
یا آتشتان مرد شما نور خدایید
\\
آن باد وبا گشت شما را فسرانید
&&
یا باد صبا گشت به هر جا که درآیید
\\
در هر سخن از جان شما هست جوابی
&&
هر چند دهان را به جوابی نگشایید
\\
در هاون ایام چه درها که شکستید
&&
آن سرمه دیدست بسایید بسایید
\\
ای آنک بزادیت چو در مرگ رسیدید
&&
این زادن ثانیست بزایید بزایید
\\
گر هند وگر ترک بزادیت دوم بار
&&
پیدا شود آن روز که روبند گشایید
\\
ور زانک سزیدیت به شمس الحق تبریز
&&
والله که شما خاصبک روز سزایید
\\
\end{longtable}
\end{center}
