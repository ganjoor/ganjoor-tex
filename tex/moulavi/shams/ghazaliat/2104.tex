\begin{center}
\section*{غزل شماره ۲۱۰۴: شب که جهان است پر از لولیان}
\label{sec:2104}
\addcontentsline{toc}{section}{\nameref{sec:2104}}
\begin{longtable}{l p{0.5cm} r}
شب که جهان است پر از لولیان
&&
زهره زند پرده شنگولیان
\\
بیند مریخ که بزم است و عیش
&&
خنجر و شمشیر کند در میان
\\
ماه فشاند پر خود چون خروس
&&
پیش و پسش اختر چون ماکیان
\\
دیده غماز بدوزد فلک
&&
تا که گواهی ندهد بر کیان
\\
خفته گروهی و گروهی به صید
&&
تا کی کند سود و کی دارد زیان
\\
پنج و شش است امشب مهره قمار
&&
سست میفکن لب چون ناشیان
\\
جام بقا گیر و بهل جام خواب
&&
پرده بود خواب و حجاب عیان
\\
ساقی باقی است خوش و عاشقان
&&
خاک سیه بر سر این باقیان
\\
زهر از آن دست کریمش بنوش
&&
تا که شوی مهتر حلواییان
\\
عشق چو مغز است جهان همچو پوست
&&
عشق چو حلوا و جهان چون تیان
\\
حلق من از لذت حلوا بسوخت
&&
تا نکنم حلیه حلوا بیان
\\
\end{longtable}
\end{center}
