\begin{center}
\section*{غزل شماره ۱۷۷۰: آمد سرمست سحر دلبرم}
\label{sec:1770}
\addcontentsline{toc}{section}{\nameref{sec:1770}}
\begin{longtable}{l p{0.5cm} r}
آمد سرمست سحر دلبرم
&&
بیخود و بنشست به مجلس برم
\\
گرم شد و عربده آغاز کرد
&&
گفت که تو نقشی و من آزرم
\\
تو به دو پر می پری و من به صد
&&
تو ز دو کس من ز دو صد خوشترم
\\
گر چه فروتر بنشستم ز لطف
&&
من ز حریفان به دو سر برترم
\\
یک قدحم بیست چو جام شماست
&&
تا همه دانند که من دیگرم
\\
ساغر من تا لب و باقی به نیم
&&
جان و دلم زفت و به تن لاغرم
\\
صورت من ناید در چشم سر
&&
زان که از این سر نیم و زان سرم
\\
من پنهان در دل و دل هم نهان
&&
زانک در این هر دو صدف گوهرم
\\
گر قدحی بیشتر از من خوری
&&
من دو سبو بیشتر از تو خورم
\\
گر به دو صد کوه چو بز بردوی
&&
من که و بز را دو شکم بردرم
\\
چون بدوم مه نبود همتکم
&&
چون بجهم چرخ بود چنبرم
\\
چون ببرم دست به سوی سلاح
&&
دشنه خورشید بود خنجرم
\\
خشک نماید بر تو این غزل
&&
چون نشدی تر ز نم کوثرم
\\
کور نه‌ام لیک مرا کیمیاست
&&
این درم قلب از آن می خرم
\\
جزو و کلم یار مرا درخور است
&&
نی خوردم غم و نه من غم خورم
\\
\end{longtable}
\end{center}
