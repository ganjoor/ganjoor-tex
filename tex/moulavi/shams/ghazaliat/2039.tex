\begin{center}
\section*{غزل شماره ۲۰۳۹: رو سر بنه به بالین تنها مرا رها کن}
\label{sec:2039}
\addcontentsline{toc}{section}{\nameref{sec:2039}}
\begin{longtable}{l p{0.5cm} r}
رو سر بنه به بالین تنها مرا رها کن
&&
ترک من خراب شب گرد مبتلا کن
\\
ماییم و موج سودا شب تا به روز تنها
&&
خواهی بیا ببخشا خواهی برو جفا کن
\\
از من گریز تا تو هم در بلا نیفتی
&&
بگزین ره سلامت ترک ره بلا کن
\\
ماییم و آب دیده در کنج غم خزیده
&&
بر آب دیده ما صد جای آسیا کن
\\
خیره کشی است ما را دارد دلی چو خارا
&&
بکشد کسش نگوید تدبیر خونبها کن
\\
بر شاه خوبرویان واجب وفا نباشد
&&
ای زردروی عاشق تو صبر کن وفا کن
\\
دردی است غیر مردن آن را دوا نباشد
&&
پس من چگونه گویم کاین درد را دوا کن
\\
در خواب دوش پیری در کوی عشق دیدم
&&
با دست اشارتم کرد که عزم سوی ما کن
\\
گر اژدهاست بر ره عشقی است چون زمرد
&&
از برق این زمرد هی دفع اژدها کن
\\
بس کن که بیخودم من ور تو هنرفزایی
&&
تاریخ بوعلی گو تنبیه بوالعلا کن
\\
\end{longtable}
\end{center}
