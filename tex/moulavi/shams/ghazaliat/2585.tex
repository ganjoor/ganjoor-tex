\begin{center}
\section*{غزل شماره ۲۵۸۵: گر نرگس خون خوارش دربند امانستی}
\label{sec:2585}
\addcontentsline{toc}{section}{\nameref{sec:2585}}
\begin{longtable}{l p{0.5cm} r}
گر نرگس خون خوارش دربند امانستی
&&
هم زهر شکر گشتی هم گرگ شبانستی
\\
هم دور قمر یارا چون بنده بدی ما را
&&
هم ساغر سلطانی اندر دورانستی
\\
هم کوه بدان سختی چون شیره و شیرستی
&&
هم بحر بدان تلخی آب حیوانستی
\\
از طلعت مستورش بر خلق زدی نورش
&&
هم نرگس مخمورش بر ما نگرانستی
\\
با هیچ دل مست او تقصیر نکرده‌ست او
&&
پس چیست ز ناشکری تشنیع چنانستی
\\
وصلش به میان آید از لطف و کرم لیکن
&&
کفو کمر وصلش ای کاش میانستی
\\
صورتگر بی‌صورت گر ز آنک عیان بودی
&&
در مردن این صورت کس را چه زیانستی
\\
راه نظر ار بودی بی‌رهزن پنهانی
&&
با هر مژه و ابرو کی تیر و کمانستی
\\
بربند دهان زیرا دریا خمشی خواهد
&&
ور نی دهن ماهی پرگفت و زیانستی
\\
\end{longtable}
\end{center}
