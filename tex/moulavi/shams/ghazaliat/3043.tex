\begin{center}
\section*{غزل شماره ۳۰۴۳: ز حد چون بگذشتی بیا بگوی که چونی}
\label{sec:3043}
\addcontentsline{toc}{section}{\nameref{sec:3043}}
\begin{longtable}{l p{0.5cm} r}
ز حد چون بگذشتی بیا بگوی که چونی
&&
ز عشق جیب دریدی در ابتدای جنونی
\\
شکست کشتی صبرم هزار بار ز موجت
&&
سری برآر ز موجی که موج قلزم خونی
\\
که خون بهینه شرابست جگر بهینه کبابست
&&
همین دوم تو فزون کن که از فزونه فزونی
\\
چو از الست تو مستم چو در فنای تو هستم
&&
چو مهر عشق شکستم چه غم خورم ز حرونی
\\
برون بسیت بجستم درون بدیدم و رستم
&&
چه میل و عشق شدستم به جست و جوی درونی
\\
دلی ز من بربودی که دل نبود و تو بودی
&&
چه آتشی و چه دودی چه جادوی چه فسونی
\\
نمای چهره زیبا تو شمس مفخر تبریز
&&
که نقش‌ها تو نمایی ز روح آینه گونی
\\
\end{longtable}
\end{center}
