\begin{center}
\section*{غزل شماره ۲۶۴۲: زان جای بیا خواجه بدین جای نه جایی}
\label{sec:2642}
\addcontentsline{toc}{section}{\nameref{sec:2642}}
\begin{longtable}{l p{0.5cm} r}
زان جای بیا خواجه بدین جای نه جایی
&&
کاین جاست تو را خانه کجایی تو کجایی
\\
آن جا که نه جای است چراگاه تو بوده‌ست
&&
زین شهره چراگاه تو محروم چرایی
\\
جاندار سراپرده سلطان عدم باش
&&
تا بازرهی از دم این جان هوایی
\\
گه پای مشو گه سر بگریز از این سو
&&
مستی و خرابی نگر و بی‌سر و پایی
\\
ای راه نمای از می و منزل چو شوی مست
&&
نی راه به خود دانی و نی راه نمایی
\\
مستان ازل در عدم و محو چریدند
&&
کز نیست بود قاعده هست نمایی
\\
جان بر زبر همدگر افتاده ز مستی
&&
همچون ختن غیب پر از ترک خطایی
\\
این نعره زنان گشته که هیهای چه خوبی
&&
و آن سجده کنان گشته که بس روح فزایی
\\
مخدوم خداوندی شمس الحق تبریز
&&
هم نور زمینی تو و خورشید سمایی
\\
\end{longtable}
\end{center}
