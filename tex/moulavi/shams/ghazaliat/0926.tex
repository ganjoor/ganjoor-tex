\begin{center}
\section*{غزل شماره ۹۲۶: حبیب کعبه جانست اگر نمی‌دانید}
\label{sec:0926}
\addcontentsline{toc}{section}{\nameref{sec:0926}}
\begin{longtable}{l p{0.5cm} r}
حبیب کعبه جانست اگر نمی‌دانید
&&
به هر طرف که بگردید رو بگردانید
\\
که جان ویست به عالم اگر شما جسمید
&&
که جان جمله جان‌هاست اگر شما جانید
\\
ندا برآمد امشب که جان کیست فدا
&&
بجست جان من از جا که نقد بستانید
\\
هزار نکته نبشتست عشق بر رویم
&&
ز حال دل چو شما عاشقید برخوانید
\\
چه ساغرست که هر دم به عاشقان آید
&&
شما کشید چنین ساغری که مردانید
\\
که عشق باغ و تماشاست اگر ملول شوید
&&
هواش مرکب تازیست اگر فرومانید
\\
چو آب و نان همه ماهیان ز بحر بود
&&
چو ماهیید چرا عاشق لب نانید
\\
قرابه ایست پر از رنج و نام او جسمست
&&
به سنگ بربزنید و تمام برهانید
\\
چو مرغ در قفسم بهر شمس تبریزی
&&
ز دشمنی قفسم بشکنید و بدرانید
\\
\end{longtable}
\end{center}
