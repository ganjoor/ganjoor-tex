\begin{center}
\section*{غزل شماره ۹۸: ای از نظرت مست شده اسم و مسما}
\label{sec:0098}
\addcontentsline{toc}{section}{\nameref{sec:0098}}
\begin{longtable}{l p{0.5cm} r}
ای از نظرت مست شده اسم و مسما
&&
ای یوسف جان گشته ز لب‌های شکرخا
\\
ما را چه از آن قصه که گاو آمد و خر رفت
&&
هین وقت لطیفست از آن عربده بازآ
\\
ای شاه تو شاهی کن و آراسته کن بزم
&&
ای جان ولی نعمت هر وامق و عذرا
\\
هم دایه جان‌هایی و هم جوی می و شیر
&&
هم جنت فردوسی و هم سدره خضرا
\\
جز این بنگوییم وگر نیز بگوییم
&&
گویید خسیسان که محالست و علالا
\\
خواهی که بگویم بده آن جام صبوحی
&&
تا چرخ به رقص آید و صد زهره زهرا
\\
هر جا ترشی باشد اندر غم دنیی
&&
می‌غرد و می‌برد از آن جای دل ما
\\
برخیز بخیلانه در خانه فروبند
&&
کان جا که تویی خانه شود گلشن و صحرا
\\
این مه ز کجا آمد وین روی چه رویست
&&
این نور خداییست تبارک و تعالی
\\
هم قادر و هم قاهر و هم اول و آخر
&&
اول غم و سودا و به آخر ید بیضا
\\
هر دل که نلرزیدت و هر چشم که نگریست
&&
یا رب خبرش ده تو از این عیش و تماشا
\\
تا شید برآرد وی و آید به سر کوی
&&
فریاد برآرد که تمنیت تمنا
\\
نگذاردش آن عشق که سر نیز بخارد
&&
شاباش زهی سلسله و جذب و تقاضا
\\
در شهر چو من گول مگر عشق ندیدست
&&
هر لحظه مرا گیرد این عشق ز بالا
\\
هر داد و گرفتی که ز بالاست لطیفست
&&
گر حاذق جدست وگر عشوه تیبا
\\
\end{longtable}
\end{center}
