\begin{center}
\section*{غزل شماره ۷۰۵: ای کز تو همه جفا وفا شد}
\label{sec:0705}
\addcontentsline{toc}{section}{\nameref{sec:0705}}
\begin{longtable}{l p{0.5cm} r}
ای کز تو همه جفا وفا شد
&&
آن عهد و وفای تو کجا شد
\\
با روی تو سور شد عزاها
&&
بی روی تو سورها عزا شد
\\
شد بی‌قدمت سرا خرابه
&&
باز از تو خرابه‌ها سرا شد
\\
از دعوت تو فنا شود هست
&&
وز هجر تو هست‌ها فنا شد
\\
ای کشته مرا به جرم آنک
&&
از من راضی به جان چرا شد
\\
آن تخم عطای تست در جان
&&
کو را کف دست باسخا شد
\\
اعنات مهیجست جان را
&&
ور نی ز چه روی جان گدا شد
\\
گر عاشق داد نیست جودت
&&
پس جان ز چه عاشق دعا شد
\\
زد پرتو ساقییت بر ابر
&&
کز عکس تو ابرها سقا شد
\\
زد عکس صبوری تو بر کوه
&&
تسکین زمین و متکا شد
\\
زد عکس بلندی تو بر چرخ
&&
معنی تو صورت سما شد
\\
از حسن تو خاک هم خبر یافت
&&
شد یوسف خوب و دلربا شد
\\
از گفت بدار چنگ کز وی
&&
بی گفت تو فهم بانوا شد
\\
\end{longtable}
\end{center}
