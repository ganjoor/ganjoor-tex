\begin{center}
\section*{غزل شماره ۲۳۱۶: امروز بت خندان می‌بخش کند خنده}
\label{sec:2316}
\addcontentsline{toc}{section}{\nameref{sec:2316}}
\begin{longtable}{l p{0.5cm} r}
امروز بت خندان می‌بخش کند خنده
&&
عالم همه خندان شد بگذشت ز حد خنده
\\
پیوسته حسد بودی پرغصه ولیک این دم
&&
می‌جوشد و می‌روید از عین حسد خنده
\\
در من بنگر ای جان تا هر دو سلف خندیم
&&
کان خنده بی‌پایان آورد مدد خنده
\\
بربسته و بررسته غرقند در این رسته
&&
تا با همگان باشد از عین ابد خنده
\\
تا چند نهان خندم پنهان نکنم زین پس
&&
هر چند نهان دارم از من بجهد خنده
\\
ور تو پنهان داری ناموس تو من دانم
&&
کاندر سر هر مویت درجست دو صد خنده
\\
هر ذره که می‌پوید بی‌خنده نمی‌روید
&&
از نیست سوی هستی ما را کی کشد خنده
\\
خنده پدر و مادر در چرخ درآوردت
&&
بنمود به هر طورت الطاف احد خنده
\\
آن دم که دهان خندد در خنده جان بنگر
&&
کان خنده بی‌دندان در لب بنهد خنده
\\
\end{longtable}
\end{center}
