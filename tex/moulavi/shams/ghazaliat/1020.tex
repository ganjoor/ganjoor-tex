\begin{center}
\section*{غزل شماره ۱۰۲۰: ای تو نگار خانگی خانه درآ از این سفر}
\label{sec:1020}
\addcontentsline{toc}{section}{\nameref{sec:1020}}
\begin{longtable}{l p{0.5cm} r}
ای تو نگار خانگی خانه درآ از این سفر
&&
پسته لعل برگشا تا نشود گران شکر
\\
ساقی روح چون تویی کشتی نوح چون تویی
&&
تا که تهیست ساغرم خون چه پرست این جگر
\\
طعنه زند مرا ز کین رو صنمی دگر گزین
&&
در دو جهان یکی بگو کو صنمی کجا دگر
\\
آن قلمی که نقش کرد چونک بدید نقش تو
&&
گفت که‌های گم شدم این ملکست یا بشر
\\
جان و جهان چرا چنین عیب و ملامتم کنی
&&
در دل من درآ ببین هر نفسی یکی حشر
\\
عشق بگوید الصلا مایده دو صد بلا
&&
خشک لبی و چشم تر مایده بین ز خشک و تر
\\
چونک چشیدی این دو را جلوه شود بتی تو را
&&
شهره یکی ستاره‌ای بنده او دو صد قمر
\\
فاش بگو که شمس دین خاصبک و شه یقین
&&
در تبریز همچو دین اوست نهان و مشتهر
\\
\end{longtable}
\end{center}
