\begin{center}
\section*{غزل شماره ۷۷۵: هله هش دار که در شهر دو سه طرارند}
\label{sec:0775}
\addcontentsline{toc}{section}{\nameref{sec:0775}}
\begin{longtable}{l p{0.5cm} r}
هله هش دار که در شهر دو سه طرارند
&&
که به تدبیر کلاه از سر مه بردارند
\\
دو سه رندند که هشیاردل و سرمستند
&&
که فلک را به یکی عربده در چرخ آرند
\\
سردهانند که تا سر ندهی سر ندهند
&&
ساقیانند که انگور نمی‌افشارند
\\
یار آن صورت غیبند که جان طالب اوست
&&
همچو چشم خوش او خیره کش و بیمارند
\\
صورتی‌اند ولی دشمن صورت‌هااند
&&
در جهانند ولی از دو جهان بیزارند
\\
همچو شیران بدرانند و به لب می‌خندند
&&
دشمن همدگرند و به حقیقت یارند
\\
خرفروشانه یکی با دگری در جنگند
&&
لیک چون وانگری متفق یک کارند
\\
همچو خورشید همه روز نظر می‌بخشند
&&
مثل ماه و ستاره همه شب سیارند
\\
گر به کف خاک بگیرند زر سرخ شود
&&
روز گندم دروند ار چه به شب جو کارند
\\
دلبرانند که دل بر ندهد بی‌برشان
&&
سرورانند که بیرون ز سر و دستارند
\\
شکرانند که در معده نگردند ترش
&&
شاکرانند و از آن یار چه برخوردارند
\\
مردمی کن برو از خدمتشان مردم شو
&&
زانک این مردم دیگر همه مردم خوارند
\\
بس کن و بیش مگو گر چه دهان پرسخنست
&&
زانک این حرف و دم و قافیه هم اغیارند
\\
\end{longtable}
\end{center}
