\begin{center}
\section*{غزل شماره ۱۶۴۳: چند خسپیم صبوح است صلا برخیزیم}
\label{sec:1643}
\addcontentsline{toc}{section}{\nameref{sec:1643}}
\begin{longtable}{l p{0.5cm} r}
چند خسپیم صبوح است صلا برخیزیم
&&
آب رحمت بستانیم و بر آتش ریزیم
\\
آن کمیت عربی را که فلک پیمای است
&&
وقت زین است و لگام است چرا ننگیزیم
\\
خوش برانیم سوی بیشه شیران سیاه
&&
شیرگیرانه ز شیران سیه نگریزیم
\\
در زندان جهان را به شجاعت بکنیم
&&
شحنه عشق چو با ماست ز کی پرهیزیم
\\
زنگیان شب غم را همه سر برداریم
&&
زنگ و رومی چه بود چون به وغا یستیزیم
\\
قدح باده نسازیم جز از کاسه سر
&&
گرد هر دیگ نگردیم نه ما کفلیزیم
\\
ز آخور ثور برانیم سوی برج اسد
&&
چو اسد هست چه با گله گاو آمیزیم
\\
اندر این منزل هر دم حشری گاو آرد
&&
چاره نبود ز سر خر چو در این پالیزیم
\\
موج دریای حقایق که زند بر که قاف
&&
زان ز ما جوش برآورد که ما کاریزیم
\\
بدر ما راست اگر چه چو هلالیم نزار
&&
صدر ما راست اگر چه که در این دهلیزیم
\\
گلرخان روی نمایند چو رو بنماییم
&&
که بهاریم در آن باغ نه ما پاییزیم
\\
وز سر ناز بگوییم چه چیزید شما
&&
سجده آرند که ما پیش شما ناچیزیم
\\
گلعذاریم ولی پیش رخ خوب شما
&&
روی ناشسته و آلوده و بی‌تمییزیم
\\
آهوان تبتی بهر چرا آمده‌اند
&&
زانک امروز همه مشک و عبر می بیزیم
\\
چون دهد جام صفا بر همه ایثار کنیم
&&
ور زند سیخ بلا همچو خران نسکیزیم
\\
تاب خورشید ازل بر سر ما می تابد
&&
می زند بر سر ما تیز از آن سرتیزیم
\\
طالع شمس چو ما راست چه باشد اختر
&&
روز و شب در نظر شمس حق تبریزیم
\\
\end{longtable}
\end{center}
