\begin{center}
\section*{غزل شماره ۱۸۴۳: دوش چه خورده‌ای دلا راست بگو نهان مکن}
\label{sec:1843}
\addcontentsline{toc}{section}{\nameref{sec:1843}}
\begin{longtable}{l p{0.5cm} r}
دوش چه خورده‌ای دلا راست بگو نهان مکن
&&
همچو کسان بی‌گنه روی به آسمان مکن
\\
رو ترش و گران کنی تا سر خود نهان کنی
&&
بار دگر گرفتمت بار دگر همان مکن
\\
باده خاص خورده‌ای جام خلاص خورده‌ای
&&
بوی شراب می زند لخلخه در دهان مکن
\\
چون سر عشق نیستت عقل مبر ز عاشقان
&&
چشم خمار کم گشا روی به ارغوان مکن
\\
چون سر صید نیستت دام منه میان ره
&&
چونک گلی نمی‌دهی جلوه گلستان مکن
\\
غم نخورد ز رهزنی آه کسی نگیردش
&&
نیست چنان کسی کی او حکم کند چنان مکن
\\
خشم گرفت ابلهی رفت ز مجلس شهی
&&
گفت شهش که شاد رو جانب ما روان مکن
\\
خشم کسی کند کی او جان و جهان ما بود
&&
خشم مکن تو خویش را مسخره جهان مکن
\\
بند برید جوی دل آب سمن روا نشد
&&
مشعله‌های جان نگر مشغله زبان مکن
\\
\end{longtable}
\end{center}
