\begin{center}
\section*{غزل شماره ۱۳۳: غمزه عشقت بدان آرد یکی محتاج را}
\label{sec:0133}
\addcontentsline{toc}{section}{\nameref{sec:0133}}
\begin{longtable}{l p{0.5cm} r}
غمزه عشقت بدان آرد یکی محتاج را
&&
کو به یک جو برنسنجد هیچ صاحب تاج را
\\
اطلس و دیباج بافد عاشق از خون جگر
&&
تا کشد در پای معشوق اطلس و دیباج را
\\
در دل عاشق کجا یابی غم هر دو جهان
&&
پیش مکی قدر کی باشد امیر حاج را
\\
عشق معراجیست سوی بام سلطان جمال
&&
از رخ عاشق فروخوان قصه معراج را
\\
زندگی ز آویختن دارد چو میوه از درخت
&&
زان همی‌بینی درآویزان دو صد حلاج را
\\
گر نه علم حال فوق قال بودی کی بدی
&&
بنده احبار بخارا خواجه نساج را
\\
بلمه ای‌هان تا نگیری ریش کوسه در نبرد
&&
هندوی ترکی میاموز آن ملک تمغاج را
\\
همچو فرزین کژروست و رخ سیه بر نطع شاه
&&
آنک تلقین می‌کند شطرنج مر لجلاج را
\\
ای که میرخوان به غراقان روحانی شدی
&&
بر چنین خوانی چه چینی خرده تتماج را
\\
عاشق آشفته از آن گوید که اندر شهر دل
&&
عشق دایم می‌کند این غارت و تاراج را
\\
بس کن ایرا بلبل عشقش نواها می‌زند
&&
پیش بلبل چه محل باشد دم دراج را
\\
\end{longtable}
\end{center}
