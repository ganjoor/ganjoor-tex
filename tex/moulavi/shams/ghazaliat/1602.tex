\begin{center}
\section*{غزل شماره ۱۶۰۲: می بسازد جان و دل را بس عجایب کان صیام}
\label{sec:1602}
\addcontentsline{toc}{section}{\nameref{sec:1602}}
\begin{longtable}{l p{0.5cm} r}
می بسازد جان و دل را بس عجایب کان صیام
&&
گر تو خواهی تا عجب گردی عجایب دان صیام
\\
گر تو را سودای معراج است بر چرخ حیات
&&
دانک اسب تازی تو هست در میدان صیام
\\
هیچ طاعت در جهان آن روشنی ندهد تو را
&&
چونک بهر دیده دل کوری ابدان صیام
\\
چونک هست این صوم نقصان حیات هر ستور
&&
خاص شد بهر کمال معنی انسان صیام
\\
چون حیات عاشقان از مطبخ تن تیره بود
&&
پس مهیا کرد بهر مطبخ ایشان صیام
\\
چیست آن اندر جهان مهلکتر و خون ریزتر
&&
بر دل و جان و جا خون خواره شیطان صیام
\\
خدمت خاص نهانی تیزنفع و زودسود
&&
چیست پیش حضرت درگاه این سلطان صیام
\\
ماهی بیچاره را آب آن چنان تازه نکرد
&&
آنچ کرد اندر دل و جان‌های مشتاقان صیام
\\
در تن مرد مجاهد در ره مقصود دل
&&
هست بهتر از حیات صد هزاران جان صیام
\\
گر چه ایمان هست مبنی بر بنای پنج رکن
&&
لیک والله هست از آن‌ها اعظم الارکان صیام
\\
لیک در هر پنج پنهان کرده قدر صوم را
&&
چون شب قدر مبارک هست خود پنهان صیام
\\
سنگ بی‌قیمت که صد خروار از او کس ننگرد
&&
لعل گرداند چو خورشیدش درون کان صیام
\\
شیر چون باشی که تو از روبهی لرزان شوی
&&
چیره گرداند تو را بر بیشه شیران صیام
\\
بس شکم خاری کند آن کو شکم خواری کند
&&
نیست اندر طالع جمع شکم خواران صیام
\\
خاتم ملک سلیمان است یا تاجی که بخت
&&
می نهد بر تارک سرهای مختاران صیام
\\
خنده صایم به است از حال مفطر در سجود
&&
زانک می بنشاندت بر خوان الرحمان صیام
\\
در خورش آن بام تون از تو به آلایش بود
&&
همچو حمامت بشوید از همه خذلان صیام
\\
شهوت خوردن ستاره نحس دان تاریک دل
&&
نور گرداند چو ماهت در همه کیوان صیام
\\
هیچ حیوانی تو دیدی روشن و پرنور علم
&&
تن چو حیوان است مگذار از پی حیوان صیام
\\
شهوت تن را تو همچون نیشکر درهم شکن
&&
تا درون جان ببینی شکر ارزان صیام
\\
قطره‌ای تو سوی بحری کی توانی آمدن
&&
سوی بحرت آورد چون سیل و چون باران صیام
\\
پای خود را از شرف مانند سر گردان به صوم
&&
زانک هست آرامگاه مرد سرگردان صیام
\\
خویشتن را بر زمین زن در گه غوغای نفس
&&
دست و پایی زن که بفروشم چنین ارزان صیام
\\
گر چه نفست رستمی باشد مسلط بر دلت
&&
لرز بر وی افکند چون بر گل لرزان صیام
\\
ظلمتی کز اندرونش آب حیوان می زهد
&&
هست آن ظلمت به نزد عقل هشیاران صیام
\\
گر تو خواهی نور قرآن در درون جان خویش
&&
هست سر نور پاک جمله قرآن صیام
\\
بر سر خوان‌های روحانی که پاکان شسته اند
&&
مر تو را همکاسه گرداند بدان پاکان صیام
\\
روزه چون روزت کند روشن دل و صافی روان
&&
روز عید وصل شه را ساخته قربان صیام
\\
در صیام ار پا نهی شادی کنان نه با گشاد
&&
چون حرام است و نشاید پیش غمناکان صیام
\\
زود باشد کز گریبان بقا سر برزند
&&
هر که در سر افکند ماننده دامان صیام
\\
\end{longtable}
\end{center}
