\begin{center}
\section*{غزل شماره ۲۳۸۱: عشق بین با عاشقان آمیخته}
\label{sec:2381}
\addcontentsline{toc}{section}{\nameref{sec:2381}}
\begin{longtable}{l p{0.5cm} r}
عشق بین با عاشقان آمیخته
&&
روح بین با خاکدان آمیخته
\\
چند بینی این و آن و نیک و بد
&&
بنگر آخر این و آن آمیخته
\\
چند گویی بی‌نشان و بانشان
&&
بی‌نشان بین با نشان آمیخته
\\
چند گویی این جهان و آن جهان
&&
آن جهان بین وین جهان آمیخته
\\
دل چو شاه آمد زبان چون ترجمان
&&
شاه بین با ترجمان آمیخته
\\
اندرآمیزید زیرا بهر ماست
&&
این زمین با آسمان آمیخته
\\
آب و آتش بین و خاک و باد را
&&
دشمنان چون دوستان آمیخته
\\
گرگ و میش و شیر و آهو چار ضد
&&
از نهیب قهرمان آمیخته
\\
آن چنان شاهی نگر کز لطف او
&&
خار و گل در گلستان آمیخته
\\
آن چنان ابری نگر کز فیض او
&&
آب چندین ناودان آمیخته
\\
اتحاد اندر اثر بین و بدان
&&
نوبهار و مهرگان آمیخته
\\
گر چه کژبازند و ضدانند لیک
&&
همچو تیرند و کمان آمیخته
\\
قند خا خاموش باش و حیف دان
&&
قند و پند اندر دهان آمیخته
\\
شمس تبریزی همی‌روید ز دل
&&
کس نباشد آن چنان آمیخته
\\
\end{longtable}
\end{center}
