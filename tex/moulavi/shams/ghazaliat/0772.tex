\begin{center}
\section*{غزل شماره ۷۷۲: صنما سپاه عشقت به حصار دل درآمد}
\label{sec:0772}
\addcontentsline{toc}{section}{\nameref{sec:0772}}
\begin{longtable}{l p{0.5cm} r}
صنما سپاه عشقت به حصار دل درآمد
&&
بگذر بدین حوالی که جهان به هم برآمد
\\
به دو چشم نرگسینت به دو لعل شکرینت
&&
به دو زلف عنبرینت که کساد عنبر آمد
\\
به پلنگ عزت تو به نهنگ غیرت تو
&&
به خدنگ غمزه تو که هزار لشکر آمد
\\
به حق دل لطیفی خوش و مقبل و ظریفی
&&
که بر او وظیفه تو ابدا مقرر آمد
\\
که خلیل حق که دستش همه سال بت شکستی
&&
به خیال خانه تو شب و روز بتگر آمد
\\
تو مپرس حال مجنون که ز دست رفت لیلی
&&
تو مپرس حال آزر که خلیل آزر آمد
\\
به جهانیان نماید تن مرده زنده کردن
&&
چو مسیح خوبی تو سوی گور عازر آمد
\\
چه خوش است داغ عشقت که ز داغ عشق هر جان
&&
ز خراج و عشر و سخره ابدا محرر آمد
\\
به سوار روح بنگر منگر به گرد قالب
&&
که غبار از سواری حسن و منور آمد
\\
ز حجاب گل دلا تو به جهان نظاره‌ای کن
&&
که پس گل مشبک دو هزار منظر آمد
\\
دو سه بیت ماند باقی تو بگو که از تو خوشتر
&&
که ز ابر منطق تو دل و سینه اخضر آمد
\\
\end{longtable}
\end{center}
