\begin{center}
\section*{غزل شماره ۱۲۸۱: ز هدهدان تفکر چو دررسید نشانش}
\label{sec:1281}
\addcontentsline{toc}{section}{\nameref{sec:1281}}
\begin{longtable}{l p{0.5cm} r}
ز هدهدان تفکر چو دررسید نشانش
&&
مراست ملک سلیمان چو نقد گشت عیانش
\\
پری و دیو نداند ز تختگاه بلندش
&&
که تخت او نظرست و بصیرتست جهانش
\\
زبان جمله مرغان بداند او به بصیرت
&&
که هیچ مرغ نداند به وهم خویش زبانش
\\
نشان سکه او بین به هر درست که نقدست
&&
ولیک نقد نیابی که بو بری سوی کانش
\\
مگر که حلقه رندان بی‌نشان تو ببینی
&&
که عشق پیش درآید درآورد به میانش
\\
ز تیر او بود آن دل که برپرید از آن سو
&&
وگر نه کیست ز مردان که او کشید کمانش
\\
کسی که خورد شرابش ز دست ساقی عشقش
&&
همان شراب مقدم تو پر کن و برسانش
\\
از آنک هیچ شرابی خمار او ننشاند
&&
دغل میار تو ساقی مده از این و از آنش
\\
ز شمس مفخر تبریز باده گشت وظیفه
&&
چگونه بنده نباشد به هر دمی دل و جانش
\\
\end{longtable}
\end{center}
