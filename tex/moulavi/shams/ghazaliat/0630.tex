\begin{center}
\section*{غزل شماره ۶۳۰: گر دیو و پری حارس باتیغ و سپر باشد}
\label{sec:0630}
\addcontentsline{toc}{section}{\nameref{sec:0630}}
\begin{longtable}{l p{0.5cm} r}
گر دیو و پری حارس باتیغ و سپر باشد
&&
چون حکم خدا آید آن زیر و زبر باشد
\\
بر هر چه امیدستت کی گیرد او دستت
&&
بر شکل عصا آید وان مار دوسر باشد
\\
وان غصه که می‌گویی آن چاره نکردم دی
&&
هر چاره که پنداری آن نیز غرر باشد
\\
خودکرده شمر آن را چه خیزد از آن سودا
&&
اندر پی صد چون آن صد دام دگر باشد
\\
آن چاره همی‌کردم آن مات نمی‌آمد
&&
آن چاره لنگت را آخر چه اثر باشد
\\
از مات تو قوتی کن یاقوت شو او را تو
&&
تا او تو شوی تو او این حصن و مفر باشد
\\
\end{longtable}
\end{center}
