\begin{center}
\section*{غزل شماره ۸۳۷: هر کجا بوی خدا می‌آید}
\label{sec:0837}
\addcontentsline{toc}{section}{\nameref{sec:0837}}
\begin{longtable}{l p{0.5cm} r}
هر کجا بوی خدا می‌آید
&&
خلق بین بی‌سر و پا می‌آید
\\
زانک جان‌ها همه تشنه‌ست به وی
&&
تشنه را بانگ سقا می‌آید
\\
شیرخوار کرمند و نگران
&&
تا که مادر ز کجا می‌آید
\\
در فراقند و همه منتظرند
&&
کز کجا وصل و لقا می‌آید
\\
از مسلمان و جهود و ترسا
&&
هر سحر بانگ دعا می‌آید
\\
خنک آن هوش که در گوش دلش
&&
ز آسمان بانگ صلا می‌آید
\\
گوش خود را ز جفا پاک کنید
&&
زانک بانگی ز سما می‌آید
\\
گوش آلوده ننوشد آن بانگ
&&
هر سزایی به سزا می‌آید
\\
چشم آلوده مکن از خد و خال
&&
کان شهنشاه بقا می‌آید
\\
ور شد آلوده به اشکش می‌شوی
&&
زانک از آن اشک دوا می‌آید
\\
کاروان شکر از مصر رسید
&&
شرفه گام و درا می‌آید
\\
هین خمش کز پی باقی غزل
&&
شاه گوینده ما می‌آید
\\
\end{longtable}
\end{center}
