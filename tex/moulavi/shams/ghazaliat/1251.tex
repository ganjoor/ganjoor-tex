\begin{center}
\section*{غزل شماره ۱۲۵۱: گر لب او شکند نرخ شکر می‌رسدش}
\label{sec:1251}
\addcontentsline{toc}{section}{\nameref{sec:1251}}
\begin{longtable}{l p{0.5cm} r}
گر لب او شکند نرخ شکر می‌رسدش
&&
ور رخش طعنه زند بر گل تر می‌رسدش
\\
گر فلک سجده برد بر در او می‌سزدش
&&
ور ستاند گرو از قرص قمر می‌رسدش
\\
ور شه عقل که عالم همگی چاکر اوست
&&
جهت خدمت او بست کمر می‌رسدش
\\
شاه خورشید که بر زنگی شب تیغ کشید
&&
گر پی هیبتش افکند سپر می‌رسدش
\\
گر عطارد ز پی دایره و نقطه او
&&
همچو پرگار دوانست به سر می‌رسدش
\\
آن جمالی که فرشته نبود محرم او
&&
گر ندارد سر دیدار بشر می‌رسدش
\\
کار و بار ملکانی که زبردست شدند
&&
نکند ور بکند زیر و زبر می‌رسدش
\\
می‌شمردم من از این نوع شنودم ز فلک
&&
که از این‌ها بگذر چیز دگر می‌رسدش
\\
\end{longtable}
\end{center}
