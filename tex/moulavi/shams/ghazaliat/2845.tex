\begin{center}
\section*{غزل شماره ۲۸۴۵: بت من به طعنه گوید چه میان ره فتادی}
\label{sec:2845}
\addcontentsline{toc}{section}{\nameref{sec:2845}}
\begin{longtable}{l p{0.5cm} r}
بت من به طعنه گوید چه میان ره فتادی
&&
صنما چرا نیفتم ز چنان میی که دادی
\\
صنما چنان فتادم که به حشر هم نخیزم
&&
چو چنان قدح گرفتی سر مشک را گشادی
\\
شده‌ام خراب لیکن قدری وقوف دارم
&&
که سرم تو برگرفتی به کنار خود نهادی
\\
صنما ز چشم مستت که شرابدار عشق است
&&
بدهی می و قدح نی چه عظیم اوستادی
\\
کرم تو است این هم که شراب برد عقلم
&&
که اگر به عقل بودی شکافدی ز شادی
\\
قدحی به من بدادی که همی‌زنم دو دستک
&&
که به یک قدح برستم ز هزار بی‌مرادی
\\
به دو چشم شوخ مستت که طرب بزاد از وی
&&
که تو روح اولینی و ز هیچ کس نزادی
\\
\end{longtable}
\end{center}
