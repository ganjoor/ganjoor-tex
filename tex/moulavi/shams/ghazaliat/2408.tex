\begin{center}
\section*{غزل شماره ۲۴۰۸: چو آفتاب برآمد ز قعر آب سیاه}
\label{sec:2408}
\addcontentsline{toc}{section}{\nameref{sec:2408}}
\begin{longtable}{l p{0.5cm} r}
چو آفتاب برآمد ز قعر آب سیاه
&&
ز ذره ذره شنو لا اله الا الله
\\
چه جای ذره که چون آفتاب جان آمد
&&
ز آفتاب ربودند خود قبا و کلاه
\\
ز آب و گل چو برآمد مه دل آدم وار
&&
صد آفتاب چو یوسف فروشود در چاه
\\
سری ز خاک برآور که کم ز مور نه‌ای
&&
خبر ببر بر موران ز دشت و خرمنگاه
\\
از آن به دانه پوسیده مور قانع شد
&&
که او ز سنبل سرسبز ما نبود آگاه
\\
بگو به مور بهار است و دست و پا داری
&&
چرا ز گور نسازی به سوی صحرا راه
\\
چه جای مور سلیمان درید جامه شوق
&&
مرا مگیر خدا زین مثال‌های تباه
\\
ولی به قد خریدار می‌برند قبا
&&
اگر چه جامه دراز است هست قد کوتاه
\\
بیار قد درازی که تا فروبریم
&&
قبا که پیش درازیش بسکلد زه ماه
\\
خموش کردم از این پس که از خموشی من
&&
جدا شود حق و باطل چنانک دانه ز کاه
\\
\end{longtable}
\end{center}
