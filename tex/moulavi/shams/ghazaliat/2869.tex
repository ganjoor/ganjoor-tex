\begin{center}
\section*{غزل شماره ۲۸۶۹: هست اندر غم تو دلشده دانشمندی}
\label{sec:2869}
\addcontentsline{toc}{section}{\nameref{sec:2869}}
\begin{longtable}{l p{0.5cm} r}
هست اندر غم تو دلشده دانشمندی
&&
همچو نقره‌ست در آتشکده دانشمندی
\\
بر امید کرم و رحمت بخشایش تو
&&
از ره دور به سر آمده دانشمندی
\\
هست ز اوباش خیالات تو اندر ره عشق
&&
خسته و شیفته و ره زده دانشمندی
\\
چه زیان دارد خوبی تو را دوست اگر
&&
قوت یابد ز چنین مایده دانشمندی
\\
با چنین جام جنونی که تو گردان کردی
&&
کی بماند به سر قاعده دانشمندی
\\
کی روا دارد انصاف و جوانمردی تو
&&
که به غم کشته شود بیهده دانشمندی
\\
کی روا دارد خورشید حق گرمی بخش
&&
که فسرده شود از مجمده دانشمندی
\\
جانب مدرسه عشق کشیدش لطفت
&&
تا ز درس تو برد فایده دانشمندی
\\
نحس تربیع عناصر بگرفتش رحمی
&&
تا منور شود از منقده دانشمندی
\\
بس سخن دارد وز بیم ملال دل تو
&&
لب ببسته‌ست در این معبده دانشمندی
\\
\end{longtable}
\end{center}
