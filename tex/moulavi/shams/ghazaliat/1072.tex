\begin{center}
\section*{غزل شماره ۱۰۷۲: گر خورد آن شیر عشقت خون ما را خورده گیر}
\label{sec:1072}
\addcontentsline{toc}{section}{\nameref{sec:1072}}
\begin{longtable}{l p{0.5cm} r}
گر خورد آن شیر عشقت خون ما را خورده گیر
&&
ور سپارم هر دمی جان دگر بسپرده گیر
\\
سردهم این دم توی می بی‌محابا می‌خورم
&&
گر کسی آید برد دستار و کفشم برده گیر
\\
گر بگوید هوشیاری زرق را پرورده‌ای
&&
با چنین برقی پیاپی زرق را پرورده گیر
\\
جان من طغرای باقی دارد اندر دست خویش
&&
صورتم امروز و فرداییست او را مرده گیر
\\
از خدا دریا همی‌خواهی و مار خشکیی
&&
چون تو ماهی نیستی دریا به دست آورده گیر
\\
غوره افشاری و گویی من ریاضت می‌کنم
&&
چونک میخواره نه‌ای رو شیره افشرده گیر
\\
صوفیان صاف را گویی که دردی خورده‌اند
&&
صوفیان را صاف می‌دارد تو بستان درده گیر
\\
هر شکوفه کز می ما نیست خندان بر درخت
&&
گر چه او تازه‌ست و خندان هم کنون پژمرده گیر
\\
شمس تبریزی تو خورشیدی و از تو چاره نیست
&&
چونک بی‌تو شب بود استاره‌ها بشمرده گیر
\\
\end{longtable}
\end{center}
