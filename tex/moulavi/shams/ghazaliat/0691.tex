\begin{center}
\section*{غزل شماره ۶۹۱: هر چند که بلبلان گزینند}
\label{sec:0691}
\addcontentsline{toc}{section}{\nameref{sec:0691}}
\begin{longtable}{l p{0.5cm} r}
هر چند که بلبلان گزینند
&&
مرغان دگر خمش نشینند
\\
خود گیر که خرمنی ندارند
&&
نه از خرمن فقر دانه چینند
\\
از حلقه برون نه‌ایم ما نیز
&&
هر چند که آن شهان نگینند
\\
گر ولوله مرا نخواهند
&&
از بهر چه کارم آفرینند
\\
شیرین و ترش مراد شاهست
&&
دو دیگ نهاده بهر اینند
\\
بایست بود ترش به مطبخ
&&
چون مخموران بدان رهینند
\\
هر حالت ما غذای قومیست
&&
زین اغذیه غیبیان سمینند
\\
مرغان ضمیر از آسمانند
&&
روزی دو سه بسته زمینند
\\
زانشان ز فلک گسیل کردند
&&
هر چند ستارگان دینند
\\
تا قدر وصال حق بدانند
&&
تا درد فراق حق بینند
\\
بر خاک قراضه گر بریزند
&&
آن را نهلند و برگزینند
\\
شمس تبریز کم سخن بود
&&
شاهان همه صابر و امینند
\\
\end{longtable}
\end{center}
