\begin{center}
\section*{غزل شماره ۱۸۶۰: الا ای باد شبگیرم بیار اخبار شمس الدین}
\label{sec:1860}
\addcontentsline{toc}{section}{\nameref{sec:1860}}
\begin{longtable}{l p{0.5cm} r}
الا ای باد شبگیرم بیار اخبار شمس الدین
&&
خداوندم ولی دانی تو از اسرار شمس الدین
\\
کسی کز نام او بر بحر بی‌کشتی عبر یابی
&&
چو سامندر ز مهر او روی در نار شمس الدین
\\
کرامت‌ها که مردان از تفاخر یاد آن آرند
&&
به ذات حق کز آن دارد هماره عار شمس الدین
\\
یکی غاری است کاندر وی ز سر سرها وحی است
&&
برون غار حق حارس درون غار شمس الدین
\\
ز جسم و روح‌ها بگذر حجاب عشق هم بردر
&&
دو صد منزل از آن سوتر ببین بازار شمس الدین
\\
ایا روحی ترفرف فی فضاء العشق و استشرف
&&
و طرفی جنه الاسرار من انوار شمس الدین
\\
قلایدهای در دارد بناگوش ضمیر من
&&
از آن الفاظ وحی آسای شکربار شمس الدین
\\
ایا ای دل تو آن جایی که نوشت باد وصل او
&&
ولیکن زحمتش کم ده مکن آزار شمس الدین
\\
بصر در دیده بفزاید اگر در دیده ره یابد
&&
به جای توتیا و کحل ناگه خار شمس الدین
\\
به هر سویی چو تو ای دل هزاران زار دارد او
&&
مپندار از سر نخوت تویی بس زار شمس الدین
\\
به لطف خویش یک چندی مهار اشترش دادت
&&
وگر نه خود کی یارد آن که باشد یار شمس الدین
\\
زهی فرقی از آن روزی که پیشش سجده می کردم
&&
که آن روزی که می گفتم بد این جا پار شمس الدین
\\
خرابی دین و دنیا را نباشد هیچ اصلاحی
&&
مگر از لطف بی‌پایان وز هنجار شمس الدین
\\
شب تاریک تو ای دل نبیند روز را هرگز
&&
مگر از نور و از اشراق آن رخسار شمس الدین
\\
عجب باشد که روزی من بگیرم جام وصل او
&&
شوم مست و همی‌گویم که من خمار شمس الدین
\\
که بخت من چنان خفته‌ست که بیداری ندارد رو
&&
مگر از بخت و اقبال چنان بیدار شمس الدین
\\
نبودت پیش از این مثلش نباشد بعد از این دانم
&&
ز لوح سرها واقف و زان هشیار شمس الدین
\\
بزد خود بر در امکان که مانندش برون ناید
&&
ز اوصاف بدیع خویش خود مسمار شمس الدین
\\
یکی جوبار روحانی است که جان‌ها جان از او یابند
&&
شده حاکم به کلیه بر آن جوبار شمس الدین
\\
سمعت القوم کل القوم اعلاهم و اصفاهم
&&
علی تفضیله جدا علی الاخیار شمس الدین
\\
و ان کانت ایادیه و افضالا اتانیه
&&
و احیی الروح مجانا لمن ادرار شمس الدین
\\
فروحی خط اقرارا برق الف اقرار
&&
و ان کان قد استغنی من الاقرار شمس الدین
\\
هدی قلبی الی واد کثیر خصبه جدا
&&
علیه الغیث موصولا لمن مدرار شمس الدین
\\
ایا تبریز سلمنا علی نادیک تسلیما
&&
فبلغ صبوتی و الهجر بالاعذار شمس الدین
\\
\end{longtable}
\end{center}
