\begin{center}
\section*{غزل شماره ۳۸۰: آن خواجه اگر چه تیزگوش است}
\label{sec:0380}
\addcontentsline{toc}{section}{\nameref{sec:0380}}
\begin{longtable}{l p{0.5cm} r}
آن خواجه اگر چه تیزگوش است
&&
استیزه کن و گران فروش است
\\
من غره به سست خنده او
&&
ایمن گشتم که او خموش است
\\
هش دار که آب زیر کاه است
&&
بحری است که زیر که به جوش است
\\
هر جا که روی هش است مفتاح
&&
این جا چه کنی که قفل هوش است
\\
در روی تو بنگرد بخندد
&&
مغرور مشو که روی پوش است
\\
هر دل که به چنگ او درافتاد
&&
چون چنگ همیشه در خروش است
\\
با این همه روح‌ها چه زنبور
&&
طواف ویند زانک نوش است
\\
شیری است که غم ز هیبت او
&&
در گور مقیم همچو موش است
\\
شمس تبریز روز نقد است
&&
عالم به چه در حدیث دوش است
\\
\end{longtable}
\end{center}
