\begin{center}
\section*{غزل شماره ۱۱۵: ای جان و قوام جمله جان‌ها}
\label{sec:0115}
\addcontentsline{toc}{section}{\nameref{sec:0115}}
\begin{longtable}{l p{0.5cm} r}
ای جان و قوام جمله جان‌ها
&&
پر بخش و روان کن روان‌ها
\\
با تو ز زیان چه باک داریم
&&
ای سودکن همه زیان‌ها
\\
فریاد ز تیرهای غمزه
&&
وز ابروهای چون کمان‌ها
\\
در لعل بتان شکر نهادی
&&
بگشاده به طمع آن دهان‌ها
\\
ای داده به دست ما کلیدی
&&
بگشاده بدان در جهان‌ها
\\
گر زانک نه در میان مایی
&&
برجسته چراست این میان‌ها
\\
ور نیست شراب بی‌نشانیت
&&
پس شاهد چیست این نشان‌ها
\\
ور تو ز گمان ما برونی
&&
پس زنده ز کیست این گمان‌ها
\\
ور تو ز جهان ما نهانی
&&
پیدا ز کی می‌شود نهان‌ها
\\
بگذار فسانه‌های دنیا
&&
بیزار شدیم ما از آن‌ها
\\
جانی که فتاد در شکرریز
&&
کی گنجد در دلش چنان‌ها
\\
آن کو قدم تو را زمین شد
&&
کی یاد کند ز آسمان‌ها
\\
بربند زبان ما به عصمت
&&
ما را مفکن در این زبان‌ها
\\
\end{longtable}
\end{center}
