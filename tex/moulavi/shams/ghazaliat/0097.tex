\begin{center}
\section*{غزل شماره ۹۷: رفتم به سوی مصر و خریدم شکری را}
\label{sec:0097}
\addcontentsline{toc}{section}{\nameref{sec:0097}}
\begin{longtable}{l p{0.5cm} r}
رفتم به سوی مصر و خریدم شکری را
&&
خود فاش بگو یوسف زرین کمری را
\\
در شهر کی دیدست چنین شهره بتی را
&&
در بر کی کشیدست سهیل و قمری را
\\
بنشاند به ملکت ملکی بنده بد را
&&
بخرید به گوهر کرمش بی‌گهری را
\\
خضر خضرانست و از هیچ عجب نیست
&&
کز چشمه جان تازه کند او جگری را
\\
از بهر زبردستی و دولت دهی آمد
&&
نی زیر و زبر کردن زیر و زبری را
\\
شاید که نخسپیم به شب چونک نهانی
&&
مه بوسه دهد هر شب انجم شمری را
\\
آثار رساند دل و جان را به مؤثر
&&
حمال دل و جان کند آن شه اثری را
\\
اکسیر خداییست بدان آمد کاین جا
&&
هر لحظه زر سرخ کند او حجری را
\\
جان‌های چو عیسی به سوی چرخ برانند
&&
غم نیست اگر ره نبود لاشه خری را
\\
هر چیز گمان بردم در عالم و این نی
&&
کاین جاه و جلالست خدایی نظری را
\\
سوز دل شاهانه خورشید بباید
&&
تا سرمه کشد چشم عروس سحری را
\\
ما عقل نداریم یکی ذره وگر نی
&&
کی آهوی عاقل طلبد شیر نری را
\\
بی عقل چو سایه پیت ای دوست دوانیم
&&
کان روی چو خورشید تو نبود دگری را
\\
خورشید همه روز بدان تیغ گزارد
&&
تا زخم زند هر طرفی بی‌سپری را
\\
بر سینه نهد عقل چنان دل شکنی را
&&
در خانه کشد روح چنان رهگذی را
\\
در هدیه دهد چشم چنان لعل لبی را
&&
رخ زر زند از بهر چنین سیمبری را
\\
رو صاحب آن چشم شو ای خواجه چو ابرو
&&
کو راست کند چشم کژ کژنگری را
\\
ای پاک دلان با جز او عشق مبازید
&&
نتوان دل و جان دادن هر مختصری را
\\
خاموش که او خود بکشد عاشق خود را
&&
تا چند کشی دامن هر بی‌هنری را
\\
\end{longtable}
\end{center}
