\begin{center}
\section*{غزل شماره ۲: ای طایران قدس را عشقت فزوده بال‌ها}
\label{sec:0002}
\addcontentsline{toc}{section}{\nameref{sec:0002}}
\begin{longtable}{l p{0.5cm} r}
ای طایران قدس را عشقت فزوده بال‌ها
&&
در حلقه سودای تو روحانیان را حال‌ها
\\
در لا احب الآفلین پاکی ز صورت‌ها یقین
&&
در دیده‌های غیب بین هر دم ز تو تمثال‌ها
\\
افلاک از تو سرنگون خاک از تو چون دریای خون
&&
ماهت نخوانم ای فزون از ماه‌ها و سال‌ها
\\
کوه از غمت بشکافته وان غم به دل درتافته
&&
یک قطره خونی یافته از فضلت این افضال‌ها
\\
ای سروران را تو سند بشمار ما را زان عدد
&&
دانی سران را هم بود اندر تبع دنبال‌ها
\\
سازی ز خاکی سیدی بر وی فرشته حاسدی
&&
با نقد تو جان کاسدی پامال گشته مال‌ها
\\
آن کو تو باشی بال او ای رفعت و اجلال او
&&
آن کو چنین شد حال او بر روی دارد خال‌ها
\\
گیرم که خارم خار بد خار از پی گل می‌زهد
&&
صراف زر هم می‌نهد جو بر سر مثقال‌ها
\\
فکری بدست افعال‌ها خاکی بدست این مال‌ها
&&
قالی بدست این حال‌ها حالی بدست این قال‌ها
\\
آغاز عالم غلغله پایان عالم زلزله
&&
عشقی و شکری با گله آرام با زلزال‌ها
\\
توقیع شمس آمد شفق طغرای دولت عشق حق
&&
فال وصال آرد سبق کان عشق زد این فال‌ها
\\
از رحمة للعالمین اقبال درویشان ببین
&&
چون مه منور خرقه‌ها چون گل معطر شال‌ها
\\
عشق امر کل ما رقعه‌ای او قلزم و ما جرعه‌ای
&&
او صد دلیل آورده و ما کرده استدلال‌ها
\\
از عشق گردون مؤتلف بی‌عشق اختر منخسف
&&
از عشق گشته دال الف بی‌عشق الف چون دال‌ها
\\
آب حیات آمد سخن کاید ز علم من لدن
&&
جان را از او خالی مکن تا بردهد اعمال‌ها
\\
بر اهل معنی شد سخن اجمال‌ها تفصیل‌ها
&&
بر اهل صورت شد سخن تفصیل‌ها اجمال‌ها
\\
گر شعرها گفتند پر پر به بود دریا ز در
&&
کز ذوق شعر آخر شتر خوش می‌کشد ترحال‌ها
\\
\end{longtable}
\end{center}
