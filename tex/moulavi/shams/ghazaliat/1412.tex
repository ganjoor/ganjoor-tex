\begin{center}
\section*{غزل شماره ۱۴۱۲: بیا هر کس که می خواهد که تا با وی گرو بندم}
\label{sec:1412}
\addcontentsline{toc}{section}{\nameref{sec:1412}}
\begin{longtable}{l p{0.5cm} r}
بیا هر کس که می خواهد که تا با وی گرو بندم
&&
که سنگ خاره جان گیرد بپیوند خداوندم
\\
همی‌گفتم به گل روزی زهی خندان قلاوزی
&&
مرا گل گفت می دانی تو باری کز چه می خندم
\\
خیال شاه خوش خویم تبسم کرد در رویم
&&
چنین شد نسل بر نسلم چنین فرزند فرزندم
\\
شه من گفت هر مسکین که عمرش نیست من عمرم
&&
بدین وعده من مسکین امید از عمر برکندم
\\
دل من بانگ بر من زد چه باشد قدر عمری خود
&&
چه منت می نهی بر من تو خود چندی و من چندم
\\
شهی کز لطف می آید اگر منت نهد شاید
&&
که چاهی پرحدث بودی منت از زر درآگندم
\\
کمر نابسته در خدمت مرا تاج خرد داد او
&&
تو خود اندیشه کن با خود چه بخشد گر بپیوندم
\\
یقول العشق لی سرا تنافس و اغتنم برا
&&
و لا تفجر و لا تهجر و الا تبتئس تندم
\\
همه شاهان غلامان را به خرسندی ثنا گفته
&&
همه خشم خداوندی بر من این که خرسندم
\\
مضی فی صحوتی یومی و فاض السکر فی قومی
&&
فاسرع و اسقنی خمرا حمیرا تشبه العندم
\\
بیا درده یکی جامی پر از شادی و آرامی
&&
که بنمایم سرانجامی چو مخموران بپرسندم
\\
میازارید از خویم که من بسیار می گویم
&&
جهانی طوطیان دارم اگر بسیار شد قندم
\\
\end{longtable}
\end{center}
