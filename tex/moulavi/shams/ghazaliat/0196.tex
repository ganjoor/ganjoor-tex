\begin{center}
\section*{غزل شماره ۱۹۶: در جنبش اندرآور زلف عبرفشان را}
\label{sec:0196}
\addcontentsline{toc}{section}{\nameref{sec:0196}}
\begin{longtable}{l p{0.5cm} r}
در جنبش اندرآور زلف عبرفشان را
&&
در رقص اندرآور جان‌های صوفیان را
\\
خورشید و ماه و اختر رقصان بگرد چنبر
&&
ما در میان رقصیم رقصان کن آن میان را
\\
لطف تو مطربانه از کمترین ترانه
&&
در چرخ اندرآرد صوفی آسمان را
\\
باد بهار پویان آید ترانه گویان
&&
خندان کند جهان را خیزان کند خزان را
\\
بس مار یار گردد گل جفت خار گردد
&&
وقت نثار گردد مر شاه بوستان را
\\
هر دم ز باغ بویی آید چو پیک سویی
&&
یعنی که الصلا زن امروز دوستان را
\\
در سر خود روان شد بستان و با تو گوید
&&
در سر خود روان شو تا جان رسد روان را
\\
تا غنچه برگشاید با سرو سر سوسن
&&
لاله بشارت آرد مر بید و ارغوان را
\\
تا سر هر نهالی از قعر بر سر آید
&&
معراجیان نهاده در باغ نردبان را
\\
مرغان و عندلیبان بر شاخه‌ها نشسته
&&
چون بر خزینه باشد ادرار پاسبان را
\\
این برگ چون زبان‌ها وین میوه‌ها چو دل‌ها
&&
دل‌ها چو رو نماید قیمت دهد زبان را
\\
\end{longtable}
\end{center}
