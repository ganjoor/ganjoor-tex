\begin{center}
\section*{غزل شماره ۲۶۵۵: تو نقشی نقش بندان را چه دانی}
\label{sec:2655}
\addcontentsline{toc}{section}{\nameref{sec:2655}}
\begin{longtable}{l p{0.5cm} r}
تو نقشی نقش بندان را چه دانی
&&
تو شکلی پیکری جان را چه دانی
\\
تو خود می‌نشنوی بانگ دهل را
&&
رموز سر پنهان را چه دانی
\\
هنوز از کات کفرت خود خبر نیست
&&
حقایق‌های ایمان را چه دانی
\\
هنوزت خار در پای است بنشین
&&
تو سرسبزی بستان را چه دانی
\\
تو نامی کرده‌ای این را و آن را
&&
از این نگذشته‌ای آن را چه دانی
\\
چه صورت‌هاست مر بی‌صورتان را
&&
تو صورت‌های ایشان را چه دانی
\\
زنخ کم زن که اندر چاه نفسی
&&
تو آن چاه زنخدان را چه دانی
\\
درخت سبز داند قدر باران
&&
تو خشکی قدر باران را چه دانی
\\
سیه کاری مکن با باز چون زاغ
&&
تو باز چتر سلطان را چه دانی
\\
سلیمانی نکردی در ره عشق
&&
زبان جمله مرغان را چه دانی
\\
نگهبانی است حاضر بر تو سبحان
&&
تو حیوانی نگهبان را چه دانی
\\
تو را در چرخ آورده‌ست ماهی
&&
تو ماه چرخ گردان را چه دانی
\\
تجلی کرد این دم شمس تبریز
&&
تو دیوی نور رحمان را چه دانی
\\
\end{longtable}
\end{center}
