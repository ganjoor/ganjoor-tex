\begin{center}
\section*{غزل شماره ۲۸۰۵: ای خوشا عیشی که باشد ای خوشا نظاره‌ای}
\label{sec:2805}
\addcontentsline{toc}{section}{\nameref{sec:2805}}
\begin{longtable}{l p{0.5cm} r}
ای خوشا عیشی که باشد ای خوشا نظاره‌ای
&&
چون به اصل اصل خویش آید چنین هر پاره‌ای
\\
هر طرف آید به دستش بی‌صراحی باده‌ای
&&
هر طرف آید به چشمش دلبری عیاره‌ای
\\
دلبری که سنگ خارا گر ز لعلش بو برد
&&
جان پذیرد سنگ خارا تا شود هشیاره‌ای
\\
باده دزدید از لبان دلبر من یک صفت
&&
لاجرم در عشق آن لب جان شده میخواره‌ای
\\
صبحدم بر راه دیری راهبم همراه شد
&&
دیدمش هم درد خویش و دیدمش هم کاره‌ای
\\
یک صراحی پیشم آورد آن حریف نیک خو
&&
گشت جانم زان صراحی بیخودی خماره‌ای
\\
در میان بیخودی تبریز شمس الدین نمود
&&
از پی بیچارگان سوی وصالش چاره‌ای
\\
\end{longtable}
\end{center}
