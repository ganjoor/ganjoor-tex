\begin{center}
\section*{غزل شماره ۹۹۶: گفت کسی خواجه سنایی بمرد}
\label{sec:0996}
\addcontentsline{toc}{section}{\nameref{sec:0996}}
\begin{longtable}{l p{0.5cm} r}
گفت کسی خواجه سنایی بمرد
&&
مرگ چنین خواجه نه کاریست خرد
\\
کاه نبود او که به بادی پرید
&&
آب نبود او که به سرما فسرد
\\
شانه نبود او که به مویی شکست
&&
دانه نبود او که زمینش فشرد
\\
گنج زری بود در این خاکدان
&&
کو دو جهان را بجوی می‌شمرد
\\
قالب خاکی سوی خاکی فکند
&&
جان خرد سوی سماوات برد
\\
جان دوم را که ندانند خلق
&&
مغلطه گوییم به جانان سپرد
\\
صاف درآمیخت به دردی می
&&
بر سر خم رفت جدا شد ز درد
\\
در سفر افتند به هم ای عزیز
&&
مرغزی و رازی و رومی و کرد
\\
خانه خود بازرود هر یکی
&&
اطلس کی باشد همتای برد
\\
خامش کن چون نقط ایرا ملک
&&
نام تو از دفتر گفتن سترد
\\
\end{longtable}
\end{center}
