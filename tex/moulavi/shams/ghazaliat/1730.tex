\begin{center}
\section*{غزل شماره ۱۷۳۰: چه روز باشد کاین جسم و رسم بنوردیم}
\label{sec:1730}
\addcontentsline{toc}{section}{\nameref{sec:1730}}
\begin{longtable}{l p{0.5cm} r}
چه روز باشد کاین جسم و رسم بنوردیم
&&
میان مجلس جان حلقه حلقه می گردیم
\\
همی‌خوریم می جان به حضرت سلطان
&&
چنانک بی‌لب و ساغر نخست می خوردیم
\\
خراب و مست به ساقی جان همی‌گوییم
&&
برآر دست که ما دست‌ها برآوردیم
\\
بیار نقل که ما نقل کرده‌ایم این سو
&&
بیار باده احمر که زار و رخ زردیم
\\
بکن سلام که تسلیم ابتلای توییم
&&
بپرس گرم که افسرده دم سردیم
\\
جوابمان دهد آن ساقیم که نوش خورید
&&
که ما به نورفشانی چو مه جوامردیم
\\
تو ملک کدکن وهب لی بگو سلیمان وار
&&
که ما به منع عطا مور را نیازردیم
\\
ز هجر و فرقت ما درد و غم بسی دیدیم
&&
درآی در بر ما ما دوای هر دردیم
\\
دل آر خسته به خار جفا و گل بستان
&&
چه تحفه آری ماورد را که ما وردیم
\\
اگر ز مونس و جفتان خود جدا ماندی
&&
بیا که در کرم و حسن لطف ما فردیم
\\
اگر تو کار نکردی و مفلسی از خیر
&&
بیا که کار چو تو صد هزار ما کردیم
\\
بیار اشک چو مشتاق و گرد را بنشان
&&
که روی ماه نبینیم تا در این گردیم
\\
خمش گزاف مینداز مهره اندر طاس
&&
به ما گذار که ما اوستاد این نردیم
\\
\end{longtable}
\end{center}
