\begin{center}
\section*{غزل شماره ۱۹۳۳: برخیز و صبوح را برنجان}
\label{sec:1933}
\addcontentsline{toc}{section}{\nameref{sec:1933}}
\begin{longtable}{l p{0.5cm} r}
برخیز و صبوح را برنجان
&&
ای روی تو آفتاب رخشان
\\
جان‌ها که ز راه نو رسیدند
&&
بر مایده قدیم بنشان
\\
جان‌ها که پرید دوش در خواب
&&
در عالم غیب شد پریشان
\\
هر جان به ولایتی و شهری
&&
آواره شدند چون غریبان
\\
مرغان رمیده را فرازآر
&&
حراقه بزن صفیر برخوان
\\
هرچ آوردند از ره آورد
&&
بیخود کنشان و جمله بستان
\\
زیرا هر گل که برگ دارد
&&
او بر نخورد از این گلستان
\\
عقلی باید ز عقل بیزار
&&
خوش نیست قلاوزی زحیران
\\
جغد است قلاوز و همه راه
&&
در هر قدمی هزار ویران
\\
ای باز خدا درآ به آواز
&&
از کنگره‌های شهر سلطان
\\
این راه بزن که اندر این راه
&&
خفت اشتر و مست شد شتربان
\\
\end{longtable}
\end{center}
