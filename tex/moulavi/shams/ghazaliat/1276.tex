\begin{center}
\section*{غزل شماره ۱۲۷۶: باز درآمد ز راه بیخود و سرمست دوش}
\label{sec:1276}
\addcontentsline{toc}{section}{\nameref{sec:1276}}
\begin{longtable}{l p{0.5cm} r}
باز درآمد ز راه بیخود و سرمست دوش
&&
توبه کنان توبه را سیل ببردست دوش
\\
گرز برآورد عشق کوفت سر عقل را
&&
شد ز بلندی عشق چرخ فلک پست دوش
\\
دولت نو شد پدید دام جهان را درید
&&
مرغ ظریف از قفس شکر که وارست دوش
\\
آنچ به هفت آسمان جست فرشته و نیافت
&&
نک به زمین گاه خاک سهل برون جست دوش
\\
آنک دل جبرئیل از کف او خسته بود
&&
مرغ پراشکسته‌ای سینه او خست دوش
\\
عقل کمالی که او گردن شیران شکست
&&
عاشق بی‌دست و پا گردن او بست دوش
\\
از شرر آفتاب شیشه گردون نکفت
&&
سایه بی‌سایه‌ای دید دراشکست دوش
\\
ماه که چون عاشقان در پی خورشید بود
&&
بعد فراق دراز خفیه بپیوست دوش
\\
آنک در او عقل و وهم می‌نرسد از قصور
&&
گشت عیان تا که عشق کوفت بر او دست دوش
\\
هر چه بود آن خیال گردد روزی وصال
&&
چند خیال عدم آمد در هست دوش
\\
خامش باش ای دلیل خامشیت گفتنست
&&
شد سر و گوشت بلند از سخن پست دوش
\\
\end{longtable}
\end{center}
