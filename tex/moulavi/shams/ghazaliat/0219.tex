\begin{center}
\section*{غزل شماره ۲۱۹: چو اندرآید یارم چه خوش بود به خدا}
\label{sec:0219}
\addcontentsline{toc}{section}{\nameref{sec:0219}}
\begin{longtable}{l p{0.5cm} r}
چو اندرآید یارم چه خوش بود به خدا
&&
چو گیرد او به کنارم چه خوش بود به خدا
\\
چو شیر پنجه نهد بر شکسته آهوی خویش
&&
که ای عزیز شکارم چه خوش بود به خدا
\\
گریزپای رهش را کشان کشان ببرند
&&
بر آسمان چهارم چه خوش بود به خدا
\\
بدان دو نرگس مستش عظیم مخمورم
&&
چو بشکنند خمارم چه خوش بود به خدا
\\
چو جان زار بلادیده با خدا گوید
&&
که جز تو هیچ ندارم چه خوش بود به خدا
\\
جوابش آید از آن سو که من تو را پس از این
&&
به هیچ کس نگذارم چه خوش بود به خدا
\\
شب وصال بیاید شبم چو روز شود
&&
که روز و شب نشمارم چه خوش بود به خدا
\\
چو گل شکفته شوم در وصال گلرخ خویش
&&
رسد نسیم بهارم چه خوش بود به خدا
\\
بیابم آن شکرستان بی‌نهایت را
&&
که برد صبر و قرارم چه خوش بود به خدا
\\
امانتی که به نه چرخ در نمی‌گنجد
&&
به مستحق بسپارم چه خوش بود به خدا
\\
خراب و مست شوم در کمال بی‌خویشی
&&
نه بدروم نه بکارم چه خوش بود به خدا
\\
به گفت هیچ نیایم چو پر بود دهنم
&&
سر حدیث نخارم چه خوش بود به خدا
\\
\end{longtable}
\end{center}
