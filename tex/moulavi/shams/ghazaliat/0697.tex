\begin{center}
\section*{غزل شماره ۶۹۷: دل بی‌لطف تو جان ندارد}
\label{sec:0697}
\addcontentsline{toc}{section}{\nameref{sec:0697}}
\begin{longtable}{l p{0.5cm} r}
دل بی‌لطف تو جان ندارد
&&
جان بی‌تو سر جهان ندارد
\\
عقل ار چه شگرف کدخداییست
&&
بی خوان تو آب و نان ندارد
\\
خورشید چو دید خاک کویت
&&
هرگز سر آسمان ندارد
\\
گلنار چو دید گلشن جان
&&
زین پس سر بوستان ندارد
\\
در دولت تو سیه گلیمی
&&
گر سود کند زیان ندارد
\\
بی ماه تو شب سیه گلیمست
&&
این دارد و آن و آن ندارد
\\
دارد ز ستاره‌ها هزاران
&&
بی ماه چراغدان ندارد
\\
بی گفت تو گوش نیست جان را
&&
بی گوش تو جان زبان ندارد
\\
وان جان غریب در تظلم
&&
می‌نالد و ترجمان ندارد
\\
لیکن رخ زرد او گواهست
&&
و اشکی که غمش نهان ندارد
\\
غماز شوم بود دم سرد
&&
آن دم که دم خران ندارد
\\
اصل دم سرد مهر جانست
&&
کان را مه مهر جان ندارد
\\
چون دل سبکش کند بهارت
&&
صد گونه غمش گران ندارد
\\
آن عشق جوان چو نوبهارت
&&
جز پیران را جوان ندارد
\\
تا چند نشان دهی خمش کن
&&
کان اصل نشان نشان ندارد
\\
بگذار نشان چو شمس تبریز
&&
آن شمس که او کران ندارد
\\
\end{longtable}
\end{center}
