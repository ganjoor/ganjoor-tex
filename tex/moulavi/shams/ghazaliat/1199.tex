\begin{center}
\section*{غزل شماره ۱۱۹۹: یا مکثر الدلال علی الخلق بالنشوز}
\label{sec:1199}
\addcontentsline{toc}{section}{\nameref{sec:1199}}
\begin{longtable}{l p{0.5cm} r}
یا مکثر الدلال علی الخلق بالنشوز
&&
الفوز فی لقایک طوبی لمن یفوز
\\
من آتشین زبانم از عشق تو چو شمع
&&
گویی همه زبان شو و سر تا قدم بسوز
\\
غوغای روز بینی چون شمع مرده باش
&&
چون خلوت شب آمد چون شمع برفروز
\\
گفتم بسوز و سازش چشمم به سوی توست
&&
چشمم مدوز هر دم ای شیر همچو یوز
\\
ما را چو درکشیدی رو درمکش ز ما
&&
این پرده را دریدی آن پرده را مدوز
\\
ای آب زندگانی بخشا بر آن کسی
&&
کو پیش از این فراق در آن آب کرد پوز
\\
اول چنان نواز و در آخر چنین گداز
&&
اول یجوز آمد و امروز لایجوز
\\
ای جان و بخت خندان در روی ما بخند
&&
تا سرو و گل بخندد در موسم عجوز
\\
در موسم عجوز چو در باغ جان روی
&&
بنماید آن عجوز ز هر گوشه صد تموز
\\
گوید به باغ جان رو گویم که ره کجاست
&&
گوید که راه باغ نیاموختی هنوز
\\
آن سو که نکته‌ها و رموز چو جان رسد
&&
ای عمر باد داده تو در نکته و رموز
\\
تو غمز ما طلب کن خود رمزگو مباش
&&
با آن کمان دولت کو درمپیچ توز
\\
گر نفس پیر شد دل و جان تازه است و تر
&&
همچون بنفشه تر خوش روی پشت گوز
\\
ان لم یکن لقلبک فی ذاته غنی
&&
لم تغنه المناصب و المال و الکنوز
\\
ان کنت ذا غنی و غناک مکتم
&&
کم حبه مکتمه ترصد البروز
\\
یا طالب الجواهر و الدر و الحصی
&&
مثلان فی الظلام فهل تدر ما تحوز
\\
می‌چین تو سنگ ریزه و در زین نشیب بحر
&&
در شب مزن تو قلب که پیدا شود به روز
\\
استمحن النقود به میزان صادق
&&
ردا لما یضرک مدا لما یعوز
\\
\end{longtable}
\end{center}
