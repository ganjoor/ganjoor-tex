\begin{center}
\section*{غزل شماره ۱۹۹۱: همه خوردند و بخفتند و تهی گشت وطن}
\label{sec:1991}
\addcontentsline{toc}{section}{\nameref{sec:1991}}
\begin{longtable}{l p{0.5cm} r}
همه خوردند و بخفتند و تهی گشت وطن
&&
وقت آن شد که درآییم خرامان به چمن
\\
همه خوردند و برفتند بقای ما باد
&&
که دل و جان زمانیم و سپهدار زمن
\\
چو تویی آب حیاتی کی نماند باقی
&&
چو تو باشی بت زیبا همه گردند شمن
\\
کتب العشق علینا غمرات و محن
&&
و قضی الحجب علینا فتنا بعد فتن
\\
فرج آمد برهیدیم ز تشویش جهان
&&
بپرد جان مجرد به گلستان منن
\\
ناقتی نخ هنا فهو مناخ حسن
&&
فیه ماء و سخاء و رخاء و عطن
\\
یرزقون فرحین بخوریم آن می و نقل
&&
مقعد صدق چو شد منزل عشاق سکن
\\
دامن سیب کشانیم سوی شفتالو
&&
ببریم از گل تر چند سخن سوی سمن
\\
چو مرا می بدهی هیچ مجو شرط ادب
&&
مست را حد نزند شرع مرا نیز مزن
\\
ادب و بی‌ادبی نیست به دستم چه کنم
&&
چو شتر می کشدم مست شتربان به رسن
\\
بلبل از عشق ز گل بوسه طمع کرد و بگفت
&&
بشکن شاخ نبات و دل ما را مشکن
\\
گفت گل راز من اندرخور طفلان نبود
&&
بچه را ابجد و هوز به و حطی کلمن
\\
گفت گر می ندهی بوسه بده باده عشق
&&
گفت این هم ندهم باش حزین جفت حزن
\\
گفت من نیز تو را بر دف و بربط بزنم
&&
تنن تن تننن تن تننن تن تننن
\\
گفت شب طشت مزن که همه بیدار شوند
&&
که مگر ماه گرفته‌ست مجو شور و فتن
\\
طشت اگر من نزنم فتنه چو نه ماهه شده‌ست
&&
فتنه‌ها زاید ناچار شب آبستن
\\
برگ می لرزد بر شاخ و دلم می لرزد
&&
لرزه برگ ز باد و دلم از خوب ختن
\\
تاب رخسار گل و لاله خبر می دهدم
&&
که چراغی است نهان گشته در این زیر لگن
\\
جهد کن تا لگن جهل ز دل برداری
&&
تا که از مشرق جان صبح برآید روشن
\\
شمس تبریز طلوعی کن از مشرق روح
&&
که چو خورشید تو جانی و جهان جمله بدن
\\
\end{longtable}
\end{center}
