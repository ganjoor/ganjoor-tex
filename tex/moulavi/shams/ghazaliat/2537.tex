\begin{center}
\section*{غزل شماره ۲۵۳۷: مگر دانید با دلبر به حق صحبت و یاری}
\label{sec:2537}
\addcontentsline{toc}{section}{\nameref{sec:2537}}
\begin{longtable}{l p{0.5cm} r}
مگر دانید با دلبر به حق صحبت و یاری
&&
هر آنچ دوش می‌گفتم ز بی‌خویشی و بیماری
\\
وگر ناگه قضاء الله از این‌ها بشنود آن مه
&&
خود او داند که سودایی چه گوید در شب تاری
\\
چو نبود عقل در خانه پریشان باشد افسانه
&&
گهی زیر و گهی بالا گهی جنگ و گهی زاری
\\
اگر شور مرا یزدان کند توزیع بر عالم
&&
نبینی هیچ یک عاقل شوند از عقل‌ها عاری
\\
مگر ای عقل تو بر من همه وسواس می‌ریزی
&&
مگر ای ابر تو بر من شراب شور می‌باری
\\
مسلمانان مسلمانان شما دل‌ها نگهدارید
&&
مگردا کس به گرد من نه نظاره نه دلداری
\\
\end{longtable}
\end{center}
