\begin{center}
\section*{غزل شماره ۳۰۷۵: بیا بیا که تو از نادرات ایامی}
\label{sec:3075}
\addcontentsline{toc}{section}{\nameref{sec:3075}}
\begin{longtable}{l p{0.5cm} r}
بیا بیا که تو از نادرات ایامی
&&
برادری پدری مادری دلارامی
\\
به نام خوب تو مرده ز گور برخیزد
&&
گزاف نیست برادر چنین نکونامی
\\
تو فضل و رحمت حقی که هر که در تو گریخت
&&
قبول می کنیش با کژی و با خامی
\\
همی‌زیم به ستیزه و این هم از گولیست
&&
که تا مرا نکشی ای هوس نیارامی
\\
به هیچ نقش نگنجی ولیک تقدیرا
&&
اگر به نقش درآیی عجب گل اندامی
\\
گهی فراق نمایی و چاره آموزی
&&
گهی رسول فرستی و جان پیغامی
\\
درون روزن دل چون فتاد شعله شمع
&&
بداند این دل شب رو که بر سر بامی
\\
مرادم آنک شود سایه و آفتاب یکی
&&
که تا ز عشق نمایم تمام خوش کامی
\\
محال جوی و محالم بدین گناه مرا
&&
قبول می‌نکند هیچ عالم و عامی
\\
تو هم محال ننوشی و معتقد نشوی
&&
برو برو که مرید عقول و احلامی
\\
اگر ز خسرو جان‌ها حلاوتی یابی
&&
محال هر دو جهان را چو من درآشامی
\\
ور از طبیب طبیبان گوارشی یابی
&&
مکاشفی تو بخوان خدا نه اوهامی
\\
برآ ز مشرق تبریز شمس دین بخرام
&&
که بر ممالک هر دو جهان چو بهرامی
\\
\end{longtable}
\end{center}
