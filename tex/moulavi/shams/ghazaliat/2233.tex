\begin{center}
\section*{غزل شماره ۲۲۳۳: ای ترک ماه چهره چه گردد که صبح تو}
\label{sec:2233}
\addcontentsline{toc}{section}{\nameref{sec:2233}}
\begin{longtable}{l p{0.5cm} r}
ای ترک ماه چهره چه گردد که صبح تو
&&
آیی به حجره من و گویی که گل برو
\\
تو ماه ترکی و من اگر ترک نیستم
&&
دانم من این قدر که به ترکی است آب سو
\\
آب حیات تو گر از این بنده تیره شد
&&
ترکی مکن به کشتنم ای ترک ترک خو
\\
رزق مرا فراخی از آن چشم تنگ توست
&&
ای تو هزار دولت و اقبال تو به تو
\\
ای ارسلان قلج مکش از بهر خون من
&&
عشقت گرفت جمله اجزام مو به مو
\\
زخم قلج مبادا بر عشق تو رسد
&&
از بخل جان نمی‌کنم ای ترک گفت و گو
\\
بر ما فسون بخواند ککجک ای قشلرن
&&
ای سزدش تو سیرک سزدش قنی بجو
\\
نام تو ترک گفتم از بهر مغلطه
&&
زیرا که عشق دارد صد حاسد و عدو
\\
دکتر شنیدم از تو و خاموش ماندم
&&
غماز من بس است در این عشق رنگ و بو
\\
\end{longtable}
\end{center}
