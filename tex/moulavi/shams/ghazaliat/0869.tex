\begin{center}
\section*{غزل شماره ۸۶۹: خیاط روزگار به بالای هیچ مرد}
\label{sec:0869}
\addcontentsline{toc}{section}{\nameref{sec:0869}}
\begin{longtable}{l p{0.5cm} r}
خیاط روزگار به بالای هیچ مرد
&&
پیراهنی ندوخت که آن را قبا نکرد
\\
بنگر هزار گول سلیم اندر این جهان
&&
دامان زر دهند و خرند از بلیس درد
\\
گل‌های رنگ رنگ که پیش تو نقل‌هاست
&&
تو می خوری از آن و رخت می‌کنند زرد
\\
ای مرده را کنار گرفته که جان من
&&
آخر کنار مرده کند جان و جسم سرد
\\
خود با خدای کن که از این نقش‌های دیو
&&
خواهی شدن به وقت اجل بی‌مراد فرد
\\
پاها مکش دراز بر این خوش بساط خاک
&&
کاین بستریست عاریه می‌ترس از نورد
\\
مفکن گزافه مهره در این طاس روزگار
&&
پرهیز از آن حریف که هست اوستاد نرد
\\
منگر به گرد تن بنگر در سوار روح
&&
می‌جو سوار را به نظر در میان گرد
\\
رخسارها چون گل لابد ز گلشنیست
&&
گلزار اگر نباشد پس از کجاست ورد
\\
سیب زنخ چو دیدی می‌دان درخت سیب
&&
بهر نمونه آمد این نیست بهر خورد
\\
همت بلند دار که با همت خسیس
&&
چاوش پادشاه براند تو را که برد
\\
خاموش کن ز حرف و سخن بی‌حروف گوی
&&
چون ناطقه ملایکه بر سقف لاجورد
\\
\end{longtable}
\end{center}
