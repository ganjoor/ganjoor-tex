\begin{center}
\section*{غزل شماره ۱۹۶۶: جام پر کن ساقیا آتش بزن اندر غمان}
\label{sec:1966}
\addcontentsline{toc}{section}{\nameref{sec:1966}}
\begin{longtable}{l p{0.5cm} r}
جام پر کن ساقیا آتش بزن اندر غمان
&&
مست کن جان را که تا اندررسد در کاروان
\\
از خم آن می که گر سرپوش برخیزد از او
&&
بررود بر چرخ بویش مست گردد آسمان
\\
زان میی کز قطره جان بخش دل افروز او
&&
می شود دریای غم همچون مزاجش شادمان
\\
چون نهد پا در دماغ سرکشان روزگار
&&
در زمان سجده کنان گردند همچون خادمان
\\
جان اگر چه بس عزیز است نزد خاص و نزد عام
&&
لیک نزد خاص باشد بوی آن می جان جان
\\
جان و ماه و جان و قالب بی‌نشان شد از میی
&&
کید او از بی‌نشانی بردراند هر نشان
\\
خمخانه لم یزل جوشیده زان می کز کفش
&&
گشته ویرانه به عالم در هزاران خاندان
\\
گر به مغرب بوی آن می از عدم یابد گشاد
&&
مست گردند زاهدان اندر هری و طالقان
\\
دست مست خم او گر خار کارد در زمین
&&
شرق تا مغرب بروید از زمین‌ها گلستان
\\
بانگ چنگ چنگی سرمست عشقش دررسد
&&
در جهان خوف افتد صد امان اندر امان
\\
گر ز خم احمدی بویی برون ظاهر شود
&&
چون میش در جوش گردد چشم و جان کافران
\\
گر ز خمر احمدی خواهی تمام بوی و رنگ
&&
منزلی کن بر در تبریز یک دم ساربان
\\
تا شوی از بوی جان حق خصال می فعال
&&
وز تجلی‌های لطفش هم قرین و هم قران
\\
در درون مست عشقش چیست خورشید نهان
&&
آن که داند جز کسی جانا که آن دارد از آن
\\
گر چه می پرسید عقلم هر دم از استاد عشق
&&
سر آن می او نمی‌فرمود الا آن آن
\\
هر دمی از مصر آن یوسف سوی جان‌های ما
&&
تنگ‌های شکر می وش رسد صد کاروان
\\
جان من در خم عشقش می بجوشد جوش‌ها
&&
آه اگر بودی سوی ایوان عشقش نردبان
\\
چون جهد از جان من القاب او مانند برق
&&
چشم بیند از شعاعش صد درخش کاویان
\\
صد هزاران خانه‌ها سازد میش در صحن جان
&&
چون کند زیر و زبر سودای عشقش خاندان
\\
بوی عنبر می رود بر عرش و بر روحانیان
&&
گر چه جان تو خورد هم نیم شب از می نهان
\\
از ملولی هجر او چون سامری اندر جهان
&&
جانم از جمله جهان گشته‌ست صحرا بر کران
\\
چون شراب موسی افکن زان خضر کف دررسد
&&
صد چو جان من درآید چون کمر اندر میان
\\
ای خداوند شمس دین مقصود از این جمله تویی
&&
ای که خاک تو بود چون جان من دور زمان
\\
در پی آن می که خوردم از پیاله وصل تو
&&
این چنین زهرت ز جام هجر خوردم مزمزان
\\
همچو تبریز و چو ایام همایون تو شاه
&&
خود نبوده‌ست و نباشد بی‌مکان و بی‌اوان
\\
\end{longtable}
\end{center}
