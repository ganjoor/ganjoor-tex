\begin{center}
\section*{غزل شماره ۴۳۷: هر جور کز تو آید بر خود نهم غرامت}
\label{sec:0437}
\addcontentsline{toc}{section}{\nameref{sec:0437}}
\begin{longtable}{l p{0.5cm} r}
هر جور کز تو آید بر خود نهم غرامت
&&
جرم تو را و خود را بر خود نهم تمامت
\\
ای ماه روی از تو صد جور اگر بیاید
&&
تن را بود چو خلعت جان را بود سلامت
\\
هر کس ز جمله عالم از تو نصیب دارند
&&
عشق تو شد نصیبم احسنت ای کرامت
\\
گه جام مست گردد از لذت می تو
&&
گه می به جوش آید از چاشنی جامت
\\
معنی به سجده آید چون صورت تو بیند
&&
هر حرف رقص آرد چون بشنود کلامت
\\
عاشق چو مستتر شد بر وی ملامت آید
&&
زیرا که نقل این می نبود به جز ملامت
\\
\end{longtable}
\end{center}
