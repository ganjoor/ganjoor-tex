\begin{center}
\section*{غزل شماره ۱۴۱۱: گرم درآ و دم مده باده بیار ای صنم}
\label{sec:1411}
\addcontentsline{toc}{section}{\nameref{sec:1411}}
\begin{longtable}{l p{0.5cm} r}
گرم درآ و دم مده باده بیار ای صنم
&&
لابه بنده گوش کن گوش مخار ای صنم
\\
فوق فلک مکان تو جان و روان روان تو
&&
هل طربی که برکند بیخ خمار ای صنم
\\
این دو حریف دلستان باد قرین دوستان
&&
جیم جمال خوب تو جام عقار ای صنم
\\
مرغ دل علیل را شهپر جبرئیل را
&&
غیر بهشت روی تو نیست مطار ای صنم
\\
خمر عصیر روح را نیست نظیر در جهان
&&
ذوق کنار دوست را نیست کنار ای صنم
\\
معجز موسوی تویی چون سوی بحر غم روی
&&
از تک بحر برجهد گرد و غبار ای صنم
\\
جام پر از عقار کن جان مرا سوار کن
&&
زود پیاده را ببین گشته سوار ای صنم
\\
مرکب من چو می بود هر عدمیم شیء بود
&&
موجب حبس کی بود وام قمار ای صنم
\\
هین که فزود شور من هم تو بخوان زبور من
&&
کرد دل شکور من ترک شکار ای صنم
\\
\end{longtable}
\end{center}
