\begin{center}
\section*{غزل شماره ۱۷۸۹: ای عاشقان ای عاشقان هنگام کوچ است از جهان}
\label{sec:1789}
\addcontentsline{toc}{section}{\nameref{sec:1789}}
\begin{longtable}{l p{0.5cm} r}
ای عاشقان ای عاشقان هنگام کوچ است از جهان
&&
در گوش جانم می رسد طبل رحیل از آسمان
\\
نک ساربان برخاسته قطارها آراسته
&&
از ما حلالی خواسته چه خفته‌اید ای کاروان
\\
این بانگ‌ها از پیش و پس بانگ رحیل است و جرس
&&
هر لحظه‌ای نفس و نفس سر می کشد در لامکان
\\
زین شمع‌های سرنگون زین پرده‌های نیلگون
&&
خلقی عجب آید برون تا غیب‌ها گردد عیان
\\
زین چرخ دولابی تو را آمد گران خوابی تو را
&&
فریاد از این عمر سبک زنهار از این خواب گران
\\
ای دل سوی دلدار شو ای یار سوی یار شو
&&
ای پاسبان بیدار شو خفته نشاید پاسبان
\\
هر سوی شمع و مشعله هر سوی بانگ و مشغله
&&
کامشب جهان حامله زاید جهان جاودان
\\
تو گل بدی و دل شدی جاهل بدی عاقل شدی
&&
آن کو کشیدت این چنین آن سو کشاند کش کشان
\\
اندر کشاکش‌های او نوش است ناخوش‌های او
&&
آب است آتش‌های او بر وی مکن رو را گران
\\
در جان نشستن کار او توبه شکستن کار او
&&
از حیله بسیار او این ذره‌ها لرزان دلان
\\
ای ریش خند رخنه جه یعنی منم سالار ده
&&
تا کی جهی گردن بنه ور نی کشندت چون کمان
\\
تخم دغل می کاشتی افسوس‌ها می داشتی
&&
حق را عدم پنداشتی اکنون ببین ای قلتبان
\\
ای خر به کاه اولیتری دیگی سیاه اولیتری
&&
در قعر چاه اولیتری ای ننگ خانه و خاندان
\\
در من کسی دیگر بود کاین خشم‌ها از وی جهد
&&
گر آب سوزانی کند ز آتش بود این را بدان
\\
در کف ندارم سنگ من با کس ندارم جنگ من
&&
با کس نگیرم تنگ من زیرا خوشم چون گلستان
\\
پس خشم من زان سر بود وز عالم دیگر بود
&&
این سو جهان آن سو جهان بنشسته من بر آستان
\\
بر آستان آن کس بود کو ناطق اخرس بود
&&
این رمز گفتی بس بود دیگر مگو درکش زبان
\\
\end{longtable}
\end{center}
