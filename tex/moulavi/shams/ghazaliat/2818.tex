\begin{center}
\section*{غزل شماره ۲۸۱۸: صنما چونک فریبی همه عیار فریبی}
\label{sec:2818}
\addcontentsline{toc}{section}{\nameref{sec:2818}}
\begin{longtable}{l p{0.5cm} r}
صنما چونک فریبی همه عیار فریبی
&&
صنما چون همه جانی دل هشیار فریبی
\\
سحری چون قمر آیی به خرابات درآیی
&&
بت و بتخانه بسوزی دل و دلدار فریبی
\\
دل آشفته نگیری خرد خفته نگیری
&&
تو بدان نرگس خفته همه بیدار فریبی
\\
ز غمت سنگ گدازد رمه با گرگ بسازد
&&
رمه و گرگ و شبان را تو به یک بار فریبی
\\
چه کنم جان و بدن را چه کنم قوت تن را
&&
که تو جبار جهانی همه بیمار فریبی
\\
قمر زنگی شب را تو کنی رومی مه رو
&&
همه کوران سیه را تو به انوار فریبی
\\
همه را گوش بگیری شنوایی برسانی
&&
همه را چشم گشایی و به دیدار فریبی
\\
تو نه آنی که فریبی ز کسی صرفه بجویی
&&
تو همه لطف و عطایی تو به ایثار فریبی
\\
تو صلاح دل و دینی تو در این لطف چنینی
&&
که کمین خار فنا را سوی گلزار فریبی
\\
\end{longtable}
\end{center}
