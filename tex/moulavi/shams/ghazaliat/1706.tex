\begin{center}
\section*{غزل شماره ۱۷۰۶: برخیز تا شراب به رطل و سبو خوریم}
\label{sec:1706}
\addcontentsline{toc}{section}{\nameref{sec:1706}}
\begin{longtable}{l p{0.5cm} r}
برخیز تا شراب به رطل و سبو خوریم
&&
بزم شهنشه‌ست نه ما باده می خریم
\\
بحری است شهریار و شرابی است خوشگوار
&&
درده شراب لعل ببین ما چه گوهریم
\\
خورشید جام نور چو برریخت بر زمین
&&
ما ذره وار مست بر این اوج برپریم
\\
خورشید لایزال چو ما را شراب داد
&&
از کبر در پیاله خورشید ننگریم
\\
پیش آر آن شراب خردسوز دلفروز
&&
تا همچو دل ز آب و گل خویش بگذریم
\\
پرخواره‌ایم کز کرم شاه واقفیم
&&
در شرب سابقیم و به خدمت مقصریم
\\
زیرا که سکر مانع خدمت بود یقین
&&
زین سو چو فربهیم بدان سوی لاغریم
\\
نوری که در زجاجه و مشکات تافته‌ست
&&
بر ما بزن که ما ز شعاعش منوریم
\\
بس گرم و سرد شد دل از این باده چون تنور
&&
درسوزمان چو هیزم تا هیچ نفسریم
\\
چون شیشه فلک پر از آتش شده‌ست جان
&&
چون کوره بهر ما که مس و قلب یا زریم
\\
ای گلعذار جام چو لاله به مجلس آر
&&
کز ساغر چو لاله چو گل یاسمین بریم
\\
خوش خوش بیا و اصل خوشی را به بزم آر
&&
با جمله ما خوشیم ولی با تو خوشتریم
\\
ای مطرب آن ترانه تر بازگو ببین
&&
تو تری و لطیفی و ما از تو ترتریم
\\
اندرفکن ز بانگ و خروش خوشت صدا
&&
در ما که در وفای تو چون کوه مرمریم
\\
آن دم که از مسیح تو میراث برده‌ای
&&
در گوش ما بدم که چو سرنای مضطریم
\\
گر چه دهان پر است ز گفتار لب ببند
&&
خاموش کن که پیش حسودان منکریم
\\
\end{longtable}
\end{center}
