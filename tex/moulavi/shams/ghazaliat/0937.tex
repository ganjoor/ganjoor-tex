\begin{center}
\section*{غزل شماره ۹۳۷: فراغتی دهدم عشق تو ز خویشاوند}
\label{sec:0937}
\addcontentsline{toc}{section}{\nameref{sec:0937}}
\begin{longtable}{l p{0.5cm} r}
فراغتی دهدم عشق تو ز خویشاوند
&&
از آنک عشق تو بنیاد عافیت برکند
\\
از آنک عشق نخواهد به جز خرابی کار
&&
از آنک عشق نگیرد ز هیچ آفت پند
\\
چه جای مال و چه نام نکو و حرمت و بوش
&&
چه خان و مان و سلامت چه اهل و یا فرزند
\\
که جان عاشق چون تیغ عشق برباید
&&
هزار جان مقدس به شکر آن بنهند
\\
هوای عشق تو و آن گاه خوف ویرانی
&&
تو کیسه بسته و آن گاه عشق آن لب قند
\\
سرک فروکش و کنج سلامتی بنشین
&&
ز دست کوته ناید هوای سرو بلند
\\
برو ز عشق نبردی تو بوی در همه عمر
&&
نه عشق داری عقلیست این به خود خرسند
\\
چه صبر کردن و دامن ز فتنه بربودن
&&
نشسته تا که چه آید ز چرخ روزی چند
\\
درآمد آتش عشق و بسوخت هر چه جز اوست
&&
چو جمله سوخته شد شاد شین و خوش می‌خند
\\
و خاصه عشق کسی کز الست تا به کنون
&&
نبوده است چنو خود به حرمت پیوند
\\
اگر تو گویی دیدم ورا برای خدا
&&
گشای دیده دیگر و این دو را بربند
\\
کز این نظر دو هزاران هزار چون من و تو
&&
به هر دو عالم دایم هلاک و کور شدند
\\
اگر به دیده من غیر آن جمال آید
&&
بکنده باد مرا هر دو دیده‌ها به کلند
\\
بصیرت همه مردان مرد عاجز شد
&&
کجا رسد به جمال و جلال شاه لوند
\\
دریغ پرده هستی خدای برکندی
&&
چنانک آن در خیبر علی حیدر کند
\\
که تا بدیدی دیده که پنج نوبت او
&&
هزار ساله از آن سو که گفته شد بزنند
\\
\end{longtable}
\end{center}
