\begin{center}
\section*{غزل شماره ۲۵۸۳: آمد مه ما مستی دستی فلکا دستی}
\label{sec:2583}
\addcontentsline{toc}{section}{\nameref{sec:2583}}
\begin{longtable}{l p{0.5cm} r}
آمد مه ما مستی دستی فلکا دستی
&&
من نیست شدم باری در هست یکی هستی
\\
از یک قدح و از صد دل مست نمی‌گردد
&&
گر باده اثر کردی در دل تن از او رستی
\\
بار دگر آوردی زان می که سحر خوردی
&&
پر می‌دهیم گر نی این شیشه بنشکستی
\\
بر جام من از مستی سنگی زدی اشکستی
&&
از جز تو گر اشکستی بودی که نپیوستی
\\
زین باده چشید آدم کز خویش برون آمد
&&
گر مرده از این خوردی از گور برون جستی
\\
گر سیر نه ای از سر هین خوار و زبون منگر
&&
در ماه که از بالا آید به چه پستی
\\
ای برده نمازم را از وقت چه بی‌باکی
&&
گر رشک نبردی دل تن عشق پرستستی
\\
آن مست در آن مستی گر آمدی اندر صف
&&
هم قبله از او گشتی هم کعبه رخش خستی
\\
\end{longtable}
\end{center}
