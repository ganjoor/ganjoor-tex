\begin{center}
\section*{غزل شماره ۲۶۴۳: ای شاه تو ترکی عجمی وار چرایی}
\label{sec:2643}
\addcontentsline{toc}{section}{\nameref{sec:2643}}
\begin{longtable}{l p{0.5cm} r}
ای شاه تو ترکی عجمی وار چرایی
&&
تو جان و جهانی تو و بیمار چرایی
\\
گلزار چو رنگ از صدقات تو ببردند
&&
گلزار بده زان رخ و پرخار چرایی
\\
الحق تو نگفتی و دم باده او گفت
&&
ای خواجه منصور تو بر دار چرایی
\\
در غار فتم چون دل و دلدار حریفند
&&
دلدار چو شد ای دل در غار چرایی
\\
آن شاه نشد لیک پی چشم بد این گو
&&
گر شاه بشد مخزن اسرار چرایی
\\
گر بیخ دلت نیست در آن آب حیاتش
&&
ای باغ چنین تازه و پربار چرایی
\\
گر راه نبرده‌ست دلت جانب گلزار
&&
خوش بو و شکرخنده و دلدار چرایی
\\
گر دیو زند طعنه که خود نیست سلیمان
&&
ای دیو اگر نیست تو در کار چرایی
\\
بر چشمه دل گر نه پری خانه حسن است
&&
ای جان سراسیمه پری دار چرایی
\\
ای مریم جان گر تو نه‌ای حامل عیسی
&&
زان زلف چلیپا پی زنار چرایی
\\
گر از می شمس الحق تبریز نه مستی
&&
پس معتکف خانه خمار چرایی
\\
\end{longtable}
\end{center}
