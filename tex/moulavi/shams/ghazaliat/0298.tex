\begin{center}
\section*{غزل شماره ۲۹۸: آه از این زشتان که مه رو می‌نمایند از نقاب}
\label{sec:0298}
\addcontentsline{toc}{section}{\nameref{sec:0298}}
\begin{longtable}{l p{0.5cm} r}
آه از این زشتان که مه رو می‌نمایند از نقاب
&&
از درون سو کاه تاب و از برون سو ماهتاب
\\
چنگ دجال از درون و رنگ ابدال از برون
&&
دام دزدان در ضمیر و رمز شاهان در خطاب
\\
عاشق چادر مباش و خر مران در آب و گل
&&
تا نمانی ز آب و گل مانند خر اندر خلاب
\\
چون به سگ نان افکنی سگ بو کند آنگه خورد
&&
سگ نه‌ای شیری چه باشد بهر نان چندین شتاب
\\
در هر آن مردار بینی رنگکی گویی که جان
&&
جان کجا رنگ از کجا جان را بجو جان را بیاب
\\
تو سؤال و حاجتی دلبر جواب هر سؤال
&&
چون جواب آید فنا گردد سؤال اندر جواب
\\
از خطابش هست گشتی چون شراب از سعی آب
&&
وز شرابش نیست گشتی همچو آب اندر شراب
\\
او ز نازش سر کشیده همچو آتش در فروغ
&&
تو ز خجلت سر فکنده چون خطا پیش صواب
\\
گر خزان غارتی مر باغ را بی‌برگ کرد
&&
عدل سلطان بهار آمد برای فتح باب
\\
برگ‌ها چون نامه‌ها بر وی نبشته خط سبز
&&
شرح آن خط‌ها بجو از عنده‌ام الکتاب
\\
\end{longtable}
\end{center}
