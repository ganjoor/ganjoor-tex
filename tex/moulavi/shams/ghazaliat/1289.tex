\begin{center}
\section*{غزل شماره ۱۲۸۹: دلی کز تو سوزد چه باشد دوایش}
\label{sec:1289}
\addcontentsline{toc}{section}{\nameref{sec:1289}}
\begin{longtable}{l p{0.5cm} r}
دلی کز تو سوزد چه باشد دوایش
&&
چو تشنه تو باشد که باشد سقایش
\\
چو بیمار گردد به بازار گردد
&&
دکان تو جوید لب قندخایش
\\
تویی باغ و گلشن تویی روز روشن
&&
مکن دل چو آهن مران از لقایش
\\
به درد و به زاری به اندوه و خواری
&&
عجب چند داری برون سرایش
\\
مها از سر او چو تو سایه بردی
&&
چه سود و چه راحت ز سایه همایش
\\
چو یک دم نبیند جمال و جلالت
&&
بگیرد ملالی ز جان و ز جایش
\\
جهان از بهارش چو فردوس گردد
&&
چمن بی‌زبانی بگوید ثنایش
\\
جواهر که بخشد کف بحر خویش
&&
فزایش که بخشد رخ جان فزایش
\\
جهان سایه توست روش از تو دارد
&&
ز نور تو باشد بقا و فنایش
\\
منم مهره تو فتاده ز دستت
&&
از این طاس غربت بیا درربایش
\\
بگیرم ادب را ببندم دو لب را
&&
که تا راز گوید لب دلگشایش
\\
\end{longtable}
\end{center}
