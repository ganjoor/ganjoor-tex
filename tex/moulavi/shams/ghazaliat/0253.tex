\begin{center}
\section*{غزل شماره ۲۵۳: چند نهان داری آن خنده را}
\label{sec:0253}
\addcontentsline{toc}{section}{\nameref{sec:0253}}
\begin{longtable}{l p{0.5cm} r}
چند نهان داری آن خنده را
&&
آن مه تابنده فرخنده را
\\
بنده کند روی تو صد شاه را
&&
شاه کند خنده تو بنده را
\\
خنده بیاموز گل سرخ را
&&
جلوه کن آن دولت پاینده را
\\
بسته بدانست در آسمان
&&
تا بکشد چون تو گشاینده را
\\
دیده قطار شترهای مست
&&
منتظرانند کشاننده را
\\
زلف برافشان و در آن حلقه کش
&&
حلق دو صد حلقه رباینده را
\\
روز وصالست و صنم حاضرست
&&
هیچ مپا مدت آینده را
\\
عاشق زخمست دف سخت رو
&&
میل لبست آن نی نالنده را
\\
بر رخ دف چند طپانچه بزن
&&
دم ده آن نای سگالنده را
\\
ور به طمع ناله برآرد رباب
&&
خوش بگشا آن کف بخشنده را
\\
عیب مکن گر غزل ابتر بماند
&&
نیست وفا خاطر پرنده را
\\
\end{longtable}
\end{center}
