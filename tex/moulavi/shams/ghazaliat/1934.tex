\begin{center}
\section*{غزل شماره ۱۹۳۴: از ما مرو ای چراغ روشن}
\label{sec:1934}
\addcontentsline{toc}{section}{\nameref{sec:1934}}
\begin{longtable}{l p{0.5cm} r}
از ما مرو ای چراغ روشن
&&
تا زنده شود هزار چون من
\\
تا بشکفد از درون هر خار
&&
صد نرگس و یاسمین و سوسن
\\
بر هر شاخی هزار میوه
&&
در هر گل تر هزار گلشن
\\
جان شب را تو چون چراغی
&&
یا جان چراغ را چو روغن
\\
ای روزن خانه را چو خورشید
&&
یا خانه بسته را چو روزن
\\
ای جوشن را چو دست داوود
&&
یا رستم جنگ را چو جوشن
\\
خورشید پی تو غرق آتش
&&
وز بهر تو ساخت ماه خرمن
\\
نستاند هیچ کس به جز تو
&&
تاوان بهار را ز بهمن
\\
از شوق تو باغ و راغ در جوش
&&
وز عشق تو گل دریده دامن
\\
ای دوست مرا چو سر تو باشی
&&
من غم نخورم ز وام کردن
\\
روزی که گذر کنی به بازار
&&
هم مرد رود ز خویش و هم زن
\\
وان شب که صبوح او تو باشی
&&
هم روح بود خراب و هم تن
\\
ترکی کند آن صبوح و گوید
&&
با هندوی شب به خشم سن سن
\\
ترکیت به از خراج بلغار
&&
هر سن سن تو هزار رهزن
\\
گفتی که خموش من خموشم
&&
گر زانک نیاریم به گفتن
\\
ور گوش رباب دل بپیچی
&&
در گفت آیم که تن تنن تن
\\
خاکی بودم خموش و ساکن
&&
مستم کردی به هست کردن
\\
هستی بگذارم و شوم خاک
&&
تا هست کنی مرا دگر فن
\\
خاموش که گفت نیز هستی است
&&
باش از پی انصتواش الکن
\\
\end{longtable}
\end{center}
