\begin{center}
\section*{غزل شماره ۱۱۷۴: مرا می‌گفت دوش آن یار عیار}
\label{sec:1174}
\addcontentsline{toc}{section}{\nameref{sec:1174}}
\begin{longtable}{l p{0.5cm} r}
مرا می‌گفت دوش آن یار عیار
&&
سگ عاشق به از شیران هشیار
\\
جهان پر شد مگر گوشت گرفتست
&&
سگ اصحاب کهف و صاحب غار
\\
قرین شاه باشد آن سگی کو
&&
برای شاه جوید کبک و کفتار
\\
خصوصا آن سگی کو را به همت
&&
نباشد صید او جز شاه مختار
\\
ببوسد خاک پایش شیر گردون
&&
بدان لب که نیالاید به مردار
\\
دمی می‌خور دمی می‌گو به نوبت
&&
مده خود را به گفت و گو به یک بار
\\
نه آن مطرب که در مجلس نشیند
&&
گهی نوشد گهی کوشد به مزمار
\\
ملولان باز جنبیدن گرفتند
&&
همی‌جنگند و می‌لنگند ناچار
\\
بجنبان گوشه زنجیر خود را
&&
رگ دیوانگیشان را بیفشار
\\
ملول جمله عالم تازه گردد
&&
چو خندان اندرآید یار بی‌یار
\\
الفت السکر ادرکنی باسکار
&&
ایا جاری ایا جاری ایا جار
\\
و لا تسق بکاسات صغار
&&
فهذا یوم احسان و ایثار
\\
و قاتل فی سبیل الجود بخلا
&&
لیبقی منک منهاج و آثار
\\
فقل انا صببنا الماء صبا
&&
و نحن الماء لا ماء و لا نار
\\
و سیمائی شهید لی بانی
&&
قضیت عندهم فی العشق اوطار
\\
و طیبوا و اسکروا قومی فانی
&&
کریم فی کروم العصر عصار
\\
جنون فی جنون فی جنون
&&
تخفف عنک اثقالا و اوزار
\\
\end{longtable}
\end{center}
