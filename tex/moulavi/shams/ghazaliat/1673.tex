\begin{center}
\section*{غزل شماره ۱۶۷۳: امشب ای دلدار مهمان توییم}
\label{sec:1673}
\addcontentsline{toc}{section}{\nameref{sec:1673}}
\begin{longtable}{l p{0.5cm} r}
امشب ای دلدار مهمان توییم
&&
شب چه باشد روز و شب آن توییم
\\
هر کجا باشیم و هر جا که رویم
&&
حاضران کاسه و خوان توییم
\\
نقش‌های صنعت دست توییم
&&
پروریده نعمت و نان توییم
\\
چون کبوترزاده برج توییم
&&
در سفر طواف ایوان توییم
\\
حیث ما کنتم فولوا شطره
&&
با زجاجه دل پری خوان توییم
\\
هر زمان نقشی کنی در مغز ما
&&
ما صحیفه خط و عنوان توییم
\\
همچو موسی کم خوریم از دایه شیر
&&
زانک مست شیر و پستان توییم
\\
ایمنیم از دزد و مکر راه زن
&&
زانک چون زر در حرمدان توییم
\\
زان چنین مست است و دلخوش جان ما
&&
که سبکسار و گران جان توییم
\\
گوی زرین فلک رقصان ماست
&&
چون نباشد چون که چوگان توییم
\\
خواه چوگان ساز ما را خواه گوی
&&
دولت این بس که به میدان توییم
\\
خواه ما را مار کن خواهی عصا
&&
معجز موسی و برهان توییم
\\
گر عصا سازیم بیفشانیم برگ
&&
وقت خشم و جنگ ثعبان توییم
\\
عشق ما را پشت داری می کند
&&
زانک خندان روی بستان توییم
\\
سایه ساز ماست نور سایه سوز
&&
زانک همچون مه به میزان توییم
\\
هم تو بگشا این دهان را هم تو بند
&&
بند آن توست و انبان توییم
\\
\end{longtable}
\end{center}
