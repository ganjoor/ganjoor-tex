\begin{center}
\section*{غزل شماره ۳۱۸۸: ادر کاسی و دعنی عن فنونی}
\label{sec:3188}
\addcontentsline{toc}{section}{\nameref{sec:3188}}
\begin{longtable}{l p{0.5cm} r}
ادر کاسی و دعنی عن فنونی
&&
جننت فلا تحدث من جنونی
\\
نه چون ماندست ما را، نی چگونه
&&
ندانم تو دلاراما که چونی
\\
رایت الناس للدنیا زبونا
&&
و ذقت العشق فالدنیا زبونی
\\
مترس از خصم و تو فارغ همی باش
&&
که عاشق هست آن بحر فزونی
\\
فما للخلق یا صاحی ظهوری
&&
و ما للخلق یا صاحی کنونی
\\
اگر عشقم درون آرام گیرد
&&
کجا بیندم این خلق برونی
\\
و مادام الهوی تغلی فؤادی
&&
فلا تطمع قراری اوسکونی
\\
ایا نفس ملامت گر، خمش کن
&&
که هم تو در ضلالت رهنمونی
\\
ضلال العشق یا صاحی حلالی
&&
خراب العشق یا صاحی حصونی
\\
زهی کشتی شاهانه که عشق است
&&
که رانندش درین دریایی خونی
\\
فتبریز و شمس‌الدین قصدی
&&
انادیهم، خدونی اوصلونی
\\
\end{longtable}
\end{center}
