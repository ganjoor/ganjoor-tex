\begin{center}
\section*{غزل شماره ۱۱۲۹: عمر که بی‌عشق رفت هیچ حسابش مگیر}
\label{sec:1129}
\addcontentsline{toc}{section}{\nameref{sec:1129}}
\begin{longtable}{l p{0.5cm} r}
عمر که بی‌عشق رفت هیچ حسابش مگیر
&&
آب حیاتست عشق در دل و جانش پذیر
\\
هر که جز عاشقان ماهی بی‌آب دان
&&
مرده و پژمرده است گر چه بود او وزیر
\\
عشق چو بگشاد رخت سبز شود هر درخت
&&
برگ جوان بردمد هر نفس از شاخ پیر
\\
هر که شود صید عشق کی شود او صید مرگ
&&
چون سپرش مه بود کی رسدش زخم تیر
\\
سر ز خدا تافتی هیچ رهی یافتی
&&
جانب ره بازگرد یاوه مرو خیر خیر
\\
تنگ شکر خر بلاش ور نخری سرکه باش
&&
عاشق این میر شو ور نشوی رو بمیر
\\
جمله جان‌های پاک گشته اسیران خاک
&&
عشق فروریخت زر تا برهاند اسیر
\\
ای که به زنبیل تو هیچ کسی نان نریخت
&&
در بن زنبیل خود هم بطلب ای فقیر
\\
چست شو و مرد باش حق دهدت صد قماش
&&
خاک سیه گشت زر خون سیه گشت شیر
\\
مفخر تبریزیان شمس حق و دین بیا
&&
تا برهد پای دل ز آب و گل همچو قیر
\\
\end{longtable}
\end{center}
