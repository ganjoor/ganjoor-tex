\begin{center}
\section*{غزل شماره ۲۹۸۱: ای سیرگشته از ما ما سخت مشتهی}
\label{sec:2981}
\addcontentsline{toc}{section}{\nameref{sec:2981}}
\begin{longtable}{l p{0.5cm} r}
ای سیرگشته از ما ما سخت مشتهی
&&
وی پاکشیده از ره کو شرط همرهی
\\
مغز جهان تویی تو و باقی همه حشیش
&&
کی یابد آدمی ز حشیشات فربهی
\\
هر شهر کو خراب شد و زیر او زبر
&&
زان شد که دور ماند ز سایه شهنشهی
\\
چون رفت آفتاب چه ماند شب سیاه
&&
از سر چو رفت عقل چه ماند جز ابلهی
\\
ای عقل فتنه‌ای همه از رفتن تو بود
&&
وآنگه گناه بر تن بی‌عقل می‌نهی
\\
آن جا که پشت آری گمراهی است و جنگ
&&
و آن جا که رو نمایی مستی و والهی
\\
هجده هزار عالم دو قسم بیش نیست
&&
نیمش جماد مرده و نیمیش آگهی
\\
دریای آگهی که خردها همه از او است
&&
آن است منتهای خردهای منتهی
\\
ای جان آشنا که در آن بحر می‌روی
&&
وی آنک همچو تیر از این چرخ می‌جهی
\\
از خرگه تن تو جهانی منور است
&&
تا تو چگونه باشی ای روح خرگهی
\\
ای روح از شراب تو مست ابد شده
&&
وی خاک در کف تو شد زر ده دهی
\\
وصف تو بی‌مثال نیاید به فهم عام
&&
وافزاید از مثال خیال مشبهی
\\
از شوق عاشقی اگرت صورتی نهد
&&
آلایشی نیابد بحر منزهی
\\
گر نسبتی کنند به نعل آن هلال را
&&
زان ژاژ شاعران نفتد ماه از مهی
\\
دریا به پیش موسی کی ماند سد راه
&&
و اندر پناه عیسی کی ماند اکمهی
\\
او خواجه همه‌ست گرش نیست یک غلام
&&
آن سرو او سهی است گرش نشمری سهی
\\
تو موسیی ولیک شبانی دری هنوز
&&
تو یوسفی ولیک هنوز اندر این چهی
\\
زان مزد کار می‌نرسد مر تو را که هیچ
&&
پیوسته نیستی تو در این کار گه گهی
\\
خامش که بی‌طعام حق و بی‌شراب غیب
&&
این حرف و نقش هست دو سه کاسه تهی
\\
\end{longtable}
\end{center}
