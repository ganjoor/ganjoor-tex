\begin{center}
\section*{غزل شماره ۱۳۳۳: ای تو ولی احسان دل ای حسن رویت دام دل}
\label{sec:1333}
\addcontentsline{toc}{section}{\nameref{sec:1333}}
\begin{longtable}{l p{0.5cm} r}
ای تو ولی احسان دل ای حسن رویت دام دل
&&
ای از کرم پرسان دل وی پرسشت آرام دل
\\
ما زنده از اکرام تو ای هر دو عالم رام تو
&&
وی از حیات نام تو جانی گرفته نام دل
\\
بر گرد تن دل حلقه شد تن با دلم همخرقه شد
&&
وین هر دو در تو غرقه شد ای تو ولی انعام دل
\\
ای تن گرفته پای دل وی دل گرفته دامنت
&&
دامن ز دل اندرمکش تا تن رسد بر بام دل
\\
ای گوهر دریای دل چه جای جان چه جای دل
&&
روشن ز تو شب‌های دل خرم ز تو ایام دل
\\
ای عاشق و معشوق من در غیر عشق آتش بزن
&&
چون نقطه‌ای در جیم تن چون روشنی بر جام دل
\\
از بارگاه عقل کل آید همی بانگ دهل
&&
کآمد سپاه آسمان نک می‌رسد اعلام دل
\\
از زخم تیغ آن سپه در کشتن خصمان شه
&&
پرخون شده صحرا و ره ره گشته خون آشام دل
\\
زان حمله‌های صف شکن سرکوفته دیوان تن
&&
خطبه به نام شه شده دیوان پر از احکام دل
\\
ای قیل و قالت چون شکر وی گوشمالت چون شکر
&&
گر زین ادب خوارم کنی خواری منست اکرام دل
\\
گر سر تو ننهفتمی من گفتنی‌ها گفتمی
&&
تا از دلم واقف شدی امروز خاص و عام دل
\\
\end{longtable}
\end{center}
