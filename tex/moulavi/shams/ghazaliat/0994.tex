\begin{center}
\section*{غزل شماره ۹۹۴: دوست همان به که بلاکش بود}
\label{sec:0994}
\addcontentsline{toc}{section}{\nameref{sec:0994}}
\begin{longtable}{l p{0.5cm} r}
دوست همان به که بلاکش بود
&&
عود همان به که در آتش بود
\\
جام جفا باشد دشوارخوار
&&
چون ز کف دوست بود خوش بود
\\
زهر بنوش از قدحی کان قدح
&&
از کرم و لطف منقش بود
\\
عشق خلیلست درآ در میان
&&
غم مخور ار زیر تو آتش بود
\\
سرد شود آتش پیش خلیل
&&
بید و گل و سنبله کش بود
\\
در خم چوگانش یکی گوی شو
&&
تا که فلک زیر تو مفرش بود
\\
رقص کنان گوی اگر چه ز زخم
&&
در غم و در کوب و کشاکش بود
\\
سابق میدان بود او لاجرم
&&
قبله هر فارس مه وش بود
\\
چونک تراشیده شده‌ست او تمام
&&
رست از آن غم که تراشش بود
\\
هر کی مشوش بود او ایمنست
&&
گر دو جهان جمله مشوش بود
\\
مفخر تبریز تو را شمس دین
&&
شرق نه در پنج و نه در شش بود
\\
\end{longtable}
\end{center}
