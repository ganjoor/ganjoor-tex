\begin{center}
\section*{غزل شماره ۸۶۲: قومی که بر براق بصیرت سفر کنند}
\label{sec:0862}
\addcontentsline{toc}{section}{\nameref{sec:0862}}
\begin{longtable}{l p{0.5cm} r}
قومی که بر براق بصیرت سفر کنند
&&
بی ابر و بی‌غبار در آن مه نظر کنند
\\
در دانه‌های شهوتی آتش زنند زود
&&
وز دامگاه صعب به یک تک عبر کنند
\\
از خارخار این گر طبع آن طرف روند
&&
بزم و سرای گلشن جای دگر کنند
\\
بر پای لولیان طبیعت نهند بند
&&
شاهان روح زو سر از این کوی درکنند
\\
پای خرد ببسته و اوباش نفس را
&&
دستی چنین گشاده که تا شور و شر کنند
\\
اجزای ما بمرده در این گورهای تن
&&
کو صور عشق تا سر از این گور برکنند
\\
مسیست شهوت تو و اکسیر نور عشق
&&
از نور عشق مس وجود تو زر کنند
\\
انصاف ده که با نفس گرم عشق او
&&
سردا جماعتی که حدیث هنر کنند
\\
چون صوفیان گرسنه در مطبخ خرد
&&
آیند و زله‌های گران مایه جز کنند
\\
زاغان طبع را تو ز مردار روزه ده
&&
تا طوطیان شوند و شکار شکر کنند
\\
در ظل میرآب حیات شکرمزاج
&&
شاید که آتشان طبیعت شرر کنند
\\
از رشک نورها است که عقل کمال را
&&
از غیرت ملاحت او کور و کر کنند
\\
جز حق اگر به دیدن او غمزه‌ای کند
&&
آن دیده را به مهر ابد بی‌خبر کنند
\\
فخر جهان و دیده تبریز شمس دین
&&
کاجزای خاک از گذرش زیب و فر کنند
\\
اندر فضای روح نیابند مثل او
&&
گر صد هزار بارش زیر و زبر کنند
\\
خالی مباد از سر خورشید سایه‌اش
&&
تا روز را به دور حوادث سپر کنند
\\
\end{longtable}
\end{center}
