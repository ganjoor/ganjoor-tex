\begin{center}
\section*{غزل شماره ۲۵۰۴: اگر زهر است اگر شکر چه شیرین است بی‌خویشی}
\label{sec:2504}
\addcontentsline{toc}{section}{\nameref{sec:2504}}
\begin{longtable}{l p{0.5cm} r}
اگر زهر است اگر شکر چه شیرین است بی‌خویشی
&&
کله جویی نیابی سر چه شیرین است بی‌خویشی
\\
چو افتادی تو در دامش چو خوردی باده جامش
&&
برون آیی نیابی در چه شیرین است بی‌خویشی
\\
مترس آخر نه مردی تو بجنب آخر نمردی تو
&&
بده آن زر به سیمین بر چه شیرین است بی‌خویشی
\\
چرا تو سرد و برف آیی فنا شو تا شگرف آیی
&&
غم هستی تو کمتر خور چه شیرین است بی‌خویشی
\\
در این منگر که در دامم که پر گشت است این جامم
&&
به پیری عمر نو بنگر چه شیرین است بی‌خویشی
\\
چه هشیاری برادر هی ببین دریای پر از می
&&
مسلمان شو تو ای کافر چه شیرین است بی‌خویشی
\\
نمود آن زلف مشکینش که عنبر گشت مسکینش
&&
زهی مشک و زهی عنبر چه شیرین است بی‌خویشی
\\
بیا ای یار در بستان میان حلقه مستان
&&
به دست هر یکی ساغر چه شیرین است بی‌خویشی
\\
یکی شه بین تو بس حاضر به جمله روح‌ها ناظر
&&
ز بی‌خویشی از آن سوتر چه شیرین است بی‌خویشی
\\
\end{longtable}
\end{center}
