\begin{center}
\section*{غزل شماره ۲۹۴۳: گرمی مجوی الا از سوزش درونی}
\label{sec:2943}
\addcontentsline{toc}{section}{\nameref{sec:2943}}
\begin{longtable}{l p{0.5cm} r}
گرمی مجوی الا از سوزش درونی
&&
زیرا نگشت روشن دل ز آتش برونی
\\
بیمار رنج باید تا شاه غیب آید
&&
در سینه درگشاید گوید ز لطف چونی
\\
آن نافه‌های آهو و آن زلف یار خوش خو
&&
آن را تو در کمی جو کان نیست در فزونی
\\
تا آدمی نمیرد جان ملک نگیرد
&&
جز کشته کی پذیرد عشق نگار خونی
\\
عشقش بگفته با تو یا ما رویم یا تو
&&
ساکن مباش تا تو در جنبش و سکونی
\\
بر دل چو زخم راند دل سر جان بداند
&&
آنگه نه عیب ماند در نفس و نی حرونی
\\
غم چون تو را فشارد تا از خودت برآرد
&&
پس بر تو نور بارد از چرخ آبگونی
\\
در عین درد بنشین هر لحظه دوست می‌بین
&&
آخر چرا تو مسکین اندر پی فسونی
\\
تبریز جان فزودی چون شمس حق نمودی
&&
از وی خجسته بودی پیوسته نی کنونی
\\
\end{longtable}
\end{center}
