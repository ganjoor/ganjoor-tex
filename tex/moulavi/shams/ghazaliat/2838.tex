\begin{center}
\section*{غزل شماره ۲۸۳۸: صفت خدای داری چو به سینه‌ای درآیی}
\label{sec:2838}
\addcontentsline{toc}{section}{\nameref{sec:2838}}
\begin{longtable}{l p{0.5cm} r}
صفت خدای داری چو به سینه‌ای درآیی
&&
لمعان طور سینا تو ز سینه وانمایی
\\
صفت چراغ داری چو به خانه شب درآیی
&&
همه خانه نور گیرد ز فروغ روشنایی
\\
صفت شراب داری تو به مجلسی که باشی
&&
دو هزار شور و فتنه فکنی ز خوش لقایی
\\
چو طرب رمیده باشد چو هوس پریده باشد
&&
چه گیاه و گل بروید چو تو خوش کنی سقایی
\\
چو جهان فسرده باشد چو نشاط مرده باشد
&&
چه جهان‌های دیگر که ز غیب برگشایی
\\
ز تو است این تقاضا به درون بی‌قراران
&&
و اگر نه تیره گل را به صفا چه آشنایی
\\
فلکی به گرد خاکی شب و روز گشته گردان
&&
فلکا ز ما چه خواهی نه تو معدن ضیایی
\\
نفسی سرشک ریزی نفسی تو خاک بیزی
&&
نه قراضه جویی آخر همه کان و کیمیایی
\\
مثل قراضه جویان شب و روز خاک بیزی
&&
ز چه خاک می‌پرستی نه تو قبله دعایی
\\
چه عجب اگر گدایی ز شهی عطا بجوید
&&
عجب این که پادشاهی ز گدا کند گدایی
\\
و عجبتر اینک آن شه به نیاز رفت چندان
&&
که گدا غلط درافتد که مراست پادشاهی
\\
فلکا نه پادشاهی نه که خاک بنده توست
&&
تو چرا به خدمت او شب و روز در هوایی
\\
فلکم جواب گوید که کسی تهی نپوید
&&
که اگر کهی بپرد بود آن ز کهربایی
\\
سخنم خور فرشته‌ست من اگر سخن نگویم
&&
ملک گرسنه گوید که بگو خمش چرایی
\\
تو نه از فرشتگانی خورش ملک چه دانی
&&
چه کنی ترنگبین را تو حریف گندنایی
\\
تو چه دانی این ابا را که ز مطبخ دماغ است
&&
که خدا کند در آن جا شب و روز کدخدایی
\\
تبریز شمس دین را تو بگو که رو به ما کن
&&
غلطم بگو که شمسا همه روی بی‌قفایی
\\
\end{longtable}
\end{center}
