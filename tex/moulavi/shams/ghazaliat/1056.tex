\begin{center}
\section*{غزل شماره ۱۰۵۶: انجیرفروش را چه بهتر}
\label{sec:1056}
\addcontentsline{toc}{section}{\nameref{sec:1056}}
\begin{longtable}{l p{0.5cm} r}
انجیرفروش را چه بهتر
&&
انجیرفروشی ای برادر
\\
ماییم معاشران دولت
&&
هین بر کف ما نهید ساغر
\\
ای ساقی ماه روی زیبا
&&
ای جمله مراد تو میسر
\\
از روی تو تاب یافت خورشید
&&
وز بال تو برپرید جعفر
\\
ماییم بلای دی چشیده
&&
چون باغ ز زخم دی مزعفر
\\
بشنو ز بهار نو سقاهم
&&
در جام کن آن شراب احمر
\\
لوح دل را ز غم فروشوی
&&
ای شاه مطهر مطهر
\\
ای تو همه را ولی نعمت
&&
بر ما ز همه کسان فزونتر
\\
در سایه‌ات ای درخت طوبی
&&
ما راست سعادت مکرر
\\
بر عشق و جمال دوست وقفیم
&&
وز جمله کارها محرر
\\
بر هر که گزید خدمت تو
&&
شد منصب سلطنت مقرر
\\
آن کس که بود مرید خورشید
&&
چون نبود همچو مه منور
\\
مخمور شدند قوم و تشنه
&&
درده می و زین حدیث بگذر
\\
جان را بده از مزوره خویش
&&
تا نبود صحتش مزور
\\
یک قوم همی‌رسند مهمان
&&
امروز مقدم و مأخر
\\
ما گاو و شتر کنیم قربان
&&
از بهر قدوم هر برادر
\\
چه گاو که می‌سزد به قربان
&&
از بهر مبشر آن مبشر
\\
تو نیز شتردلی رها کن
&&
اشترواری فرست شکر
\\
شکر گفتم قدح نگفتم
&&
در نقل بود نبیذ مضمر
\\
ور این نکنی خموش گردم
&&
دانی چه کنم خموشی اندر
\\
\end{longtable}
\end{center}
