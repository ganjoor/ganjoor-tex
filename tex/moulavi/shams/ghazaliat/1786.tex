\begin{center}
\section*{غزل شماره ۱۷۸۶: دزدیده چون جان می روی اندر میان جان من}
\label{sec:1786}
\addcontentsline{toc}{section}{\nameref{sec:1786}}
\begin{longtable}{l p{0.5cm} r}
دزدیده چون جان می روی اندر میان جان من
&&
سرو خرامان منی ای رونق بستان من
\\
چون می روی بی‌من مرو ای جان جان بی‌تن مرو
&&
وز چشم من بیرون مشو ای شعله تابان من
\\
هفت آسمان را بردرم وز هفت دریا بگذرم
&&
چون دلبرانه بنگری در جان سرگردان من
\\
تا آمدی اندر برم شد کفر و ایمان چاکرم
&&
ای دیدن تو دین من وی روی تو ایمان من
\\
بی‌پا و سر کردی مرا بی‌خواب و خور کردی مرا
&&
سرمست و خندان اندرآ ای یوسف کنعان من
\\
از لطف تو چو جان شدم وز خویشتن پنهان شدم
&&
ای هست تو پنهان شده در هستی پنهان من
\\
گل جامه در از دست تو ای چشم نرگس مست تو
&&
ای شاخ‌ها آبست تو ای باغ بی‌پایان من
\\
یک لحظه داغم می کشی یک دم به باغم می کشی
&&
پیش چراغم می کشی تا وا شود چشمان من
\\
ای جان پیش از جان‌ها وی کان پیش از کان‌ها
&&
ای آن پیش از آن‌ها ای آن من ای آن من
\\
منزلگه ما خاک نی گر تن بریزد باک نی
&&
اندیشه‌ام افلاک نی ای وصل تو کیوان من
\\
مر اهل کشتی را لحد در بحر باشد تا ابد
&&
در آب حیوان مرگ کو ای بحر من عمان من
\\
ای بوی تو در آه من وی آه تو همراه من
&&
بر بوی شاهنشاه من شد رنگ و بو حیران من
\\
جانم چو ذره در هوا چون شد ز هر ثقلی جدا
&&
بی‌تو چرا باشد چرا ای اصل چار ارکان من
\\
ای شه صلاح الدین من ره دان من ره بین من
&&
ای فارغ از تمکین من ای برتر از امکان من
\\
\end{longtable}
\end{center}
