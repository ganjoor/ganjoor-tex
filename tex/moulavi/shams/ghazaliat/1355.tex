\begin{center}
\section*{غزل شماره ۱۳۵۵: دو چشم اگر بگشادی به آفتاب وصال}
\label{sec:1355}
\addcontentsline{toc}{section}{\nameref{sec:1355}}
\begin{longtable}{l p{0.5cm} r}
دو چشم اگر بگشادی به آفتاب وصال
&&
برآ به چرخ حقایق دگر مگو ز خیال
\\
ستاره‌ها بنگر از ورای ظلمت و نور
&&
چو ذره رقص کنان در شعاع نور جلال
\\
اگر چه ذره در آن آفتاب درنرسد
&&
ولی ز تاب شعاعش شوند نور خصال
\\
هر آن دلی که به خدمت خمید چون ابرو
&&
گشاد از نظرش صد هزار چشم کمال
\\
دهان ببند ز حال دلم که با لب دوست
&&
خدای داند کو را چه واقعه‌ست و چه حال
\\
مکن اشارت سوی دلم که دل آن نیست
&&
مپر به سوی همایان شه بدان پر و بال
\\
جراحت همه را از نمک بود فریاد
&&
مرا فراق نمک‌هاش شد وبال وبال
\\
چو ملک گشت وصالت ز شمس تبریزی
&&
نماند حیله حال و نه التفات به قال
\\
\end{longtable}
\end{center}
