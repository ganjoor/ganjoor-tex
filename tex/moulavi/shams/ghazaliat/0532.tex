\begin{center}
\section*{غزل شماره ۵۳۲: مر عاشقان را پند کس هرگز نباشد سودمند}
\label{sec:0532}
\addcontentsline{toc}{section}{\nameref{sec:0532}}
\begin{longtable}{l p{0.5cm} r}
مر عاشقان را پند کس هرگز نباشد سودمند
&&
نی آن چنان سیلیست این کش کس تواند کرد بند
\\
ذوق سر سرمست را هرگز نداند عاقلی
&&
حال دل بی‌هوش را هرگز نداند هوشمند
\\
بیزار گردند از شهی شاهان اگر بویی برند
&&
زان باده‌ها که عاشقان در مجلس دل می‌خورند
\\
خسرو وداع ملک خود از بهر شیرین می‌کند
&&
فرهاد هم از بهر او بر کوه می‌کوبد کلند
\\
مجنون ز حلقه عاقلان از عشق لیلی می‌رمد
&&
بر سبلت هر سرکشی کردست وامق ریش خند
\\
افسرده آن عمری که آن بگذشت بی آن جان خوش
&&
ای گنده آن مغزی که آن غافل بود زین لورکند
\\
این آسمان گر نیستی سرگشته و عاشق چو ما
&&
زین گردش او سیر آمدی گفتی بسستم چند چند
\\
عالم چو سرنایی و او در هر شکافش می‌دمد
&&
هر ناله‌ای دارد یقین زان دو لب چون قند قند
\\
می‌بین که چون در می‌دمد در هر گلی در هر دلی
&&
حاجت دهد عشقی دهد کافغان برآرد از گزند
\\
دل را ز حق گر برکنی بر کی نهی آخر بگو
&&
بی جان کسی که دل از او یک لحظه برتانست کند
\\
من بس کنم تو چست شو شب بر سر این بام رو
&&
خوش غلغلی در شهر زن ای جان به آواز بلند
\\
\end{longtable}
\end{center}
