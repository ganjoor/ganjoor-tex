\begin{center}
\section*{غزل شماره ۹۰۲: ز سر بگیرم عیشی چو پا به گنج فروشد}
\label{sec:0902}
\addcontentsline{toc}{section}{\nameref{sec:0902}}
\begin{longtable}{l p{0.5cm} r}
ز سر بگیرم عیشی چو پا به گنج فروشد
&&
ز روی پشت و پناهی که پشت‌ها همه رو شد
\\
دگر نشینم هرگز برای دل که برآید
&&
کجا برآید آن دل که کوی عشق فروشد
\\
موکلان چو آتش ز عشق سوی من آیند
&&
به سوی عشق گریزم که جمله فتنه از او شد
\\
که در سرم ز شرابش نه چشم ماند نه خوابش
&&
به دست ساقی نابش مگر سرم چو کدو شد
\\
به خوان عشق نشستم چشیدم از نمک او
&&
چو لقمه کردم خود را مرا چو عشق گلو شد
\\
سبو به دست دویدم به جویبار معانی
&&
که آب گشت سبویم چو آب جان به سبو شد
\\
نماز شام برفتم به سوی طرفه رومی
&&
چو دید بر در خویشم ز بام زود فروشد
\\
سر از دریچه برون کرد چو شعله‌های منور
&&
که بام و خانه و بنده به جملگی همه او شد
\\
نهیم دست دهان بر که نازکست معانی
&&
ز شمس مفخر تبریز سوخت جان و همو شد
\\
\end{longtable}
\end{center}
