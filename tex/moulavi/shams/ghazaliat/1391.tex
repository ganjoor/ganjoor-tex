\begin{center}
\section*{غزل شماره ۱۳۹۱: تا کی به حبس این جهان من خویش زندانی کنم}
\label{sec:1391}
\addcontentsline{toc}{section}{\nameref{sec:1391}}
\begin{longtable}{l p{0.5cm} r}
تا کی به حبس این جهان من خویش زندانی کنم
&&
وقت است جان پاک را تا میر میدانی کنم
\\
بیرون شدم ز آلودگی با قوت پالودگی
&&
اوراد خود را بعد از این مقرون سبحانی کنم
\\
نیزه به دستم داد شه تا نیزه بازی‌ها کنم
&&
تا کی به دست هر خسی من رسم چوگانی کنم
\\
آن پادشاه لم یزل داده‌ست ملک بی‌خلل
&&
باشد بتر از کافری گر یاد دربانی کنم
\\
چون این بنا برکنده شد آن گریه‌هامان خنده شد
&&
چون در بنا بستم نظر آهنگ دربانی کنم
\\
ای دل مرا در نیم شب دادی ز دانایی خبر
&&
اکنون به تو در خلوتم تا آنچ می دانی کنم
\\
در چاه تخمی کاشتن بی‌عقل را باشد روا
&&
این جا به داد عقل کل کشت بیابانی کنم
\\
دشوارها رفت از نظر هر سد شد زیر و زبر
&&
بر جای پا چون رست پر دوران به آسانی کنم
\\
در حضرت فرد صمد دل کی رود سوی عدد
&&
در خوان سلطان ابد چون غیر سرخوانی کنم
\\
تا چند گویم بس کنم کم یاد پیش و پس کنم
&&
اندر حضور شاه جان تا چند خط خوانی کنم
\\
\end{longtable}
\end{center}
