\begin{center}
\section*{غزل شماره ۳۳۰: بار دگر آن دلبر عیار مرا یافت}
\label{sec:0330}
\addcontentsline{toc}{section}{\nameref{sec:0330}}
\begin{longtable}{l p{0.5cm} r}
بار دگر آن دلبر عیار مرا یافت
&&
سرمست همی‌گشت به بازار مرا یافت
\\
پنهان شدم از نرگس مخمور مرا دید
&&
بگریختم از خانه خمار مرا یافت
\\
بگریختنم چیست کز او جان نبرد کس
&&
پنهان شدنم چیست چو صد بار مرا یافت
\\
گفتم که در انبوهی شهرم کی بیابد
&&
آن کس که در انبوهی اسرار مرا یافت
\\
ای مژده که آن غمزه غماز مرا جست
&&
وی بخت که آن طره طرار مرا یافت
\\
دستار ربود از سر مستان به گروگان
&&
دستار برو گوشه دستار مرا یافت
\\
من از کف پا خار همی‌کردم بیرون
&&
آن سرو دو صد گلشن و گلزار مرا یافت
\\
از گلشن خود بر سر من یار گل افشاند
&&
وان بلبل وان نادره تکرار مرا یافت
\\
من گم شدم از خرمن آن ماه چو کیله
&&
امروز مه اندر بن انبار مرا یافت
\\
از خون من آثار به هر راه چکیدست
&&
اندر پی من بود به آثار مرا یافت
\\
چون آهو از آن شیر رمیدم به بیابان
&&
آن شیر گه صید به کهسار مرا یافت
\\
آن کس که به گردون رود و گیرد آهو
&&
با صبر و تأنی و به هنجار مرا یافت
\\
در کام من این شست و من اندر تک دریا
&&
صاید به سررشته جرار مرا یافت
\\
جامی که برد از دلم آزار به من داد
&&
آن لحظه که آن یار کم آزار مرا یافت
\\
این جان گران جان سبکی یافت و بپرید
&&
کان رطل گران سنگ سبکسار مرا یافت
\\
امروز نه هوش است و نه گوش است و نه گفتار
&&
کان اصل هر اندیشه و گفتار مرا یافت
\\
\end{longtable}
\end{center}
