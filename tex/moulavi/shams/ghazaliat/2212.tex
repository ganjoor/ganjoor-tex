\begin{center}
\section*{غزل شماره ۲۲۱۲: تو بمال گوش بربط که عظیم کاهل است او}
\label{sec:2212}
\addcontentsline{toc}{section}{\nameref{sec:2212}}
\begin{longtable}{l p{0.5cm} r}
تو بمال گوش بربط که عظیم کاهل است او
&&
بشکن خمار را سر که سر همه شکست او
\\
بنواز نغمه تر به نشاط جام احمر
&&
صدفی است بحرپیما که در آورد به دست او
\\
چو درآمد آن سمن بر در خانه بسته بهتر
&&
که پریر کرد حیله ز میان ما بجست او
\\
چه بهانه گر بت است او چه بلا و آفت است او
&&
بگشاید و بدزدد کمر هزار مست او
\\
شده‌ایم آتشین پا که رویم مست آن جا
&&
تو برو نخست بنگر که کنون به خانه هست او
\\
به کسی نظر ندارد به جز آینه بت من
&&
که ز عکس چهره خود شده است بت پرست او
\\
هله ساقیا بیاور سوی من شراب احمر
&&
که سری که مست شد او ز خیال ژاژ رست او
\\
نه غم و نه غم پرستم ز غم زمانه رستم
&&
که حریف او شدستم که در ستم ببست او
\\
تو اگر چه سخت مستی برسان قدح به چستی
&&
مشکن تو شیشه گر چه دو هزار کف بخست او
\\
قدحی رسان به جانم که برد به آسمانم
&&
مدهم به دست فکرت که کشد به سوی پست او
\\
تو نه نیک گو و نی بد بپذیر ساغر خود
&&
بد و نیک او بگوید که پناه هر بد است او
\\
\end{longtable}
\end{center}
