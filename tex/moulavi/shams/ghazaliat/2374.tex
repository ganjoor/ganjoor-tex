\begin{center}
\section*{غزل شماره ۲۳۷۴: هله صیاد نگویی که چه دام است و چه دانه}
\label{sec:2374}
\addcontentsline{toc}{section}{\nameref{sec:2374}}
\begin{longtable}{l p{0.5cm} r}
هله صیاد نگویی که چه دام است و چه دانه
&&
که چو سیمرغ ببیند بجهد مست ز لانه
\\
بجز از دست فلانی مستان باده که آن می
&&
برهاند دل و جان را ز فسون و ز فسانه
\\
بخورد عشق جهان را چو عصا از کف موسی
&&
به زبانی که بسوزد همه را همچو زبانه
\\
نه سماع است نه بازی که کمندی است الهی
&&
منگر سست به نخوت تو در این بیت و ترانه
\\
نبود هیچ غری را غم دلاله و شاهد
&&
نبود هیچ کلی را غم شانه گر و شانه
\\
به دهان تو چنین تیغ نهاده‌ست نهنده
&&
مثل کارد که گیرد بر تیغی به دهانه
\\
که خیالات سفیهان همه دربان الهند
&&
نگذارند سگان را سوی درگاه و ستانه
\\
نگذارند غران را که درآیند به لشکر
&&
که بخندد لب دشمن ز کر و فر زنانه
\\
چو ندیده‌ست نشانه نبود اسپر و تیرش
&&
چو نخورده‌ست دوگانه نبود مرد یگانه
\\
\end{longtable}
\end{center}
