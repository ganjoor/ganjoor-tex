\begin{center}
\section*{غزل شماره ۵۲۴: بی گاه شد بی‌گاه شد خورشید اندر چاه شد}
\label{sec:0524}
\addcontentsline{toc}{section}{\nameref{sec:0524}}
\begin{longtable}{l p{0.5cm} r}
بی گاه شد بی‌گاه شد خورشید اندر چاه شد
&&
خورشید جان عاشقان در خلوت الله شد
\\
روزیست اندر شب نهان ترکی میان هندوان
&&
شب ترک تازی‌ها بکن کان ترک در خرگاه شد
\\
گر بو بری زین روشنی آتش به خواب اندرزنی
&&
کز شب روی و بندگی زهره حریف ماه شد
\\
ما شب گریزان و دوان و اندر پی ما زنگیان
&&
زیرا که ما بردیم زر تا پاسبان آگاه شد
\\
ما شب روی آموخته صد پاسبان را سوخته
&&
رخ‌ها چو شمع افروخته کان بیذق ما شاه شد
\\
ای شاد آن فرخ رخی کو رخ بدان رخ آورد
&&
ای کر و فر آن دلی کو سوی آن دلخواه شد
\\
آن کیست اندر راه دل کو را نباشد آه دل
&&
کار آن کسی دارد که او غرقابه آن آه شد
\\
چون غرق دریا می‌شود دریاش بر سر می‌نهد
&&
چون یوسف چاهی که او از چاه سوی جاه شد
\\
گویند اصل آدمی خاکست و خاکی می‌شود
&&
کی خاک گردد آن کسی کو خاک این درگاه شد
\\
یک سان نماید کشت‌ها تا وقت خرمن دررسد
&&
نیمیش مغز نغز شد وان نیم دیگر کاه شد
\\
\end{longtable}
\end{center}
