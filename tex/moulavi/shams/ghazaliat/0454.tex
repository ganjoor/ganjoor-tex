\begin{center}
\section*{غزل شماره ۴۵۴: جان سوی جسم آمد و تن سوی جان نرفت}
\label{sec:0454}
\addcontentsline{toc}{section}{\nameref{sec:0454}}
\begin{longtable}{l p{0.5cm} r}
جان سوی جسم آمد و تن سوی جان نرفت
&&
وان سو که تیر رفت حقیقت کمان نرفت
\\
جان چست شد که تا بپرد وین تن گران
&&
هم در زمین فروشد و بر آسمان نرفت
\\
جان میزبان تن شد در خانه گلین
&&
تن خانه دوست بود که با میزبان نرفت
\\
در وحشتی بماند که تن را گمان نبود
&&
جان رفت جانبی که بدان جا گمان نرفت
\\
پایان فراق بین که جهان آمد این جهان
&&
اندر جهان کی دید کسی کز جهان نرفت
\\
مرگت گلو بگیرد تو خیره سر شوی
&&
گویی رسول نامد وین را بیان نرفت
\\
در هر دهان که آب از آزادیم گشاد
&&
در گور هیچ مور ورا در دهان نرفت
\\
\end{longtable}
\end{center}
