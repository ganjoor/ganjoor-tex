\begin{center}
\section*{غزل شماره ۲۳۰۷: بربند دهان از نان کمد شکر روزه}
\label{sec:2307}
\addcontentsline{toc}{section}{\nameref{sec:2307}}
\begin{longtable}{l p{0.5cm} r}
بربند دهان از نان کآمد شکر روزه
&&
دیدی هنر خوردن بنگر هنر روزه
\\
آن شاه دو صد کشور تاجیت نهد بر سر
&&
بربند میان زوتر کآمد کمر روزه
\\
زین عالم چون سجین برپر سوی علیین
&&
بستان نظر حق بین زود از نظر روزه
\\
ای نقره باحرمت در کوره این مدت
&&
آتش کندت خدمت اندر شرر روزه
\\
روزه نم زمزم شد در عیسی مریم شد
&&
بر طارم چارم شد او در سفر روزه
\\
کو پر زدن مرغان کو پر ملک ای جان
&&
این هست پر چینه و آن هست پر روزه
\\
گر روزه ضرر دارد صد گونه هنر دارد
&&
سودای دگر دارد سودای سر روزه
\\
این روزه در این چادر پنهان شده چون دلبر
&&
از چادر او بگذر واجو خبر روزه
\\
باریک کند گردن ایمن کند از مردن
&&
تخمه اثر خوردن مستی اثر روزه
\\
سی روز در این دریا پا سر کنی و سر پا
&&
تا دررسی ای مولا اندر گهر روزه
\\
شیطان همه تدبیرش و آن حیله و تزویرش
&&
بشکست همه تیرش پیش سپر روزه
\\
روزه کر و فر خود خوشتر ز تو برگوید
&&
دربند در گفتن بگشای در روزه
\\
شمس الحق تبریزی هم صبری و پرهیزی
&&
هم عید شکرریزی هم کر و فر روزه
\\
\end{longtable}
\end{center}
