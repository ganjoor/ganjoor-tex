\begin{center}
\section*{غزل شماره ۸۲: معشوقه به سامان شد تا باد چنین بادا}
\label{sec:0082}
\addcontentsline{toc}{section}{\nameref{sec:0082}}
\begin{longtable}{l p{0.5cm} r}
معشوقه به سامان شد تا باد چنین بادا
&&
کفرش همه ایمان شد تا باد چنین بادا
\\
ملکی که پریشان شد از شومی شیطان شد
&&
باز آن سلیمان شد تا باد چنین بادا
\\
یاری که دلم خستی در بر رخ ما بستی
&&
غمخواره یاران شد تا باد چنین بادا
\\
هم باده جدا خوردی هم عیش جدا کردی
&&
نک سرده مهمان شد تا باد چنین بادا
\\
زان طلعت شاهانه زان مشعله خانه
&&
هر گوشه چو میدان شد تا باد چنین بادا
\\
زان خشم دروغینش زان شیوه شیرینش
&&
عالم شکرستان شد تا باد چنین بادا
\\
شب رفت صبوح آمد غم رفت فتوح آمد
&&
خورشید درخشان شد تا باد چنین بادا
\\
از دولت محزونان وز همت مجنونان
&&
آن سلسله جنبان شد تا باد چنین بادا
\\
عید آمد و عید آمد یاری که رمید آمد
&&
عیدانه فراوان شد تا باد چنین بادا
\\
ای مطرب صاحب دل در زیر مکن منزل
&&
کان زهره به میزان شد تا باد چنین بادا
\\
درویش فریدون شد هم کیسه قارون شد
&&
همکاسه سلطان شد تا باد چنین بادا
\\
آن باد هوا را بین ز افسون لب شیرین
&&
با نای در افغان شد تا باد چنین بادا
\\
فرعون بدان سختی با آن همه بدبختی
&&
نک موسی عمران شد تا باد چنین بادا
\\
آن گرگ بدان زشتی با جهل و فرامشتی
&&
نک یوسف کنعان شد تا باد چنین بادا
\\
شمس الحق تبریزی از بس که درآمیزی
&&
تبریز خراسان شد تا باد چنین بادا
\\
از اسلم شیطانی شد نفس تو ربانی
&&
ابلیس مسلمان شد تا باد چنین بادا
\\
آن ماه چو تابان شد کونین گلستان شد
&&
اشخاص همه جان شد تا باد چنین بادا
\\
بر روح برافزودی تا بود چنین بودی
&&
فر تو فروزان شد تا باد چنین بادا
\\
قهرش همه رحمت شد زهرش همه شربت شد
&&
ابرش شکرافشان شد تا باد چنین بادا
\\
از کاخ چه رنگستش وز شاخ چه تنگستش
&&
این گاو چو قربان شد تا باد چنین بادا
\\
ارضی چو سمایی شد مقصود سنایی شد
&&
این بود همه آن شد تا باد چنین بادا
\\
خاموش که سرمستم بربست کسی دستم
&&
اندیشه پریشان شد تا باد چنین بادا
\\
\end{longtable}
\end{center}
