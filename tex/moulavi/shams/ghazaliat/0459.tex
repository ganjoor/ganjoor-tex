\begin{center}
\section*{غزل شماره ۴۵۹: ای مرده‌ای که در تو ز جان هیچ بوی نیست}
\label{sec:0459}
\addcontentsline{toc}{section}{\nameref{sec:0459}}
\begin{longtable}{l p{0.5cm} r}
ای مرده‌ای که در تو ز جان هیچ بوی نیست
&&
رو رو که عشق زنده دلان مرده شوی نیست
\\
ماننده خزانی هر روز سردتر
&&
در تو ز سوز عشق یکی تای موی نیست
\\
هرگز خزان بهار شود این مجو محال
&&
حاشا بهار همچو خزان زشتخوی نیست
\\
روباه لنگ رفت که بر شیر عاشقم
&&
گفتم که این به دمدمه و های هوی نیست
\\
گیرم که سوز و آتش عشاق نیستت
&&
شرمت کجا شدست تو را هیچ روی نیست
\\
عاشق چو اژدها و تو یک کرم نیستی
&&
عاشق چو گنج‌ها و تو را یک تسوی نیست
\\
از من دو سه سخن شنو اندر بیان عشق
&&
گر چه مرا ز عشق سر گفت و گوی نیست
\\
اول بدان که عشق نه اول نه آخرست
&&
هر سو نظر مکن که از آن سوی سوی نیست
\\
گر طالب خری تو در این آخرجهان
&&
خر می‌طلب مسیح از این سوی جوی نیست
\\
یکتا شدست عیسی از آن خر به نور دل
&&
دل چون شکمبه پرحدث و توی توی نیست
\\
با خر میا به میدان زیرا که خرسوار
&&
از فارسان حمله و چوگان و گوی نیست
\\
هندوی ساقی دل خویشم که بزم ساخت
&&
تا ترک غم نتازد کامروز طوی نیست
\\
در شهر مست آیم تا جمله اهل شهر
&&
دانند کاین زهی ز گدایان کوی نیست
\\
آن عشق می‌فروش قیامت همی‌کند
&&
زان باده‌ای که درخور خم و سبوی نیست
\\
زان می زبان بیابد آن کس که الکنست
&&
زان می گلو گشاید آن کش گلوی نیست
\\
بس کن چه آرزوست تو را این سخنوری
&&
باری مرا ز مستی آن آرزوی نیست
\\
\end{longtable}
\end{center}
