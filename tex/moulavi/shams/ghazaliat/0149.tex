\begin{center}
\section*{غزل شماره ۱۴۹: خدمت شمس حق و دین یادگارت ساقیا}
\label{sec:0149}
\addcontentsline{toc}{section}{\nameref{sec:0149}}
\begin{longtable}{l p{0.5cm} r}
خدمت شمس حق و دین یادگارت ساقیا
&&
باده گردان چیست آخر داردارت ساقیا
\\
ساقی گلرخ ز می این عقل ما را خار نه
&&
تا بگردد جمله گل این خارخارت ساقیا
\\
جام چون طاووس پران کن به گرد باغ بزم
&&
تا چو طاووسی شود این زهر و مارت ساقیا
\\
کار را بگذار می را بار کن بر اسب جام
&&
تا ز کیوان بگذرد این کار و بارت ساقیا
\\
تا تو باشی در عزیزی‌ها به بند خود دری
&&
می‌کند ای سخت جان خاکی خوارت ساقیا
\\
چشمه رواق می را نحل بگشا سوی عیش
&&
تا ز چشمه می‌شود هر چشم و چارت ساقیا
\\
عقل نامحرم برون ران تو ز خلوت زان شراب
&&
تا نماید آن صنم رخسار نارت ساقیا
\\
بیخودی از می بگیر و از خودی رو بر کنار
&&
تا بگیرد در کنار خویش یارت ساقیا
\\
تو شوی از دست بینی عیش خود را بر کنار
&&
چون بگیرد در بر سیمین کنارت ساقیا
\\
گاه تو گیری به بر در یار را از بیخودی
&&
چونک بیخودتر شدی گیرد کنارت ساقیا
\\
از می تبریز گردان کن پیاپی رطل‌ها
&&
تا ببرد تارهای چنگ عارت ساقیا
\\
\end{longtable}
\end{center}
