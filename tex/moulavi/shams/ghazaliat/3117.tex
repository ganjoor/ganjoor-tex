\begin{center}
\section*{غزل شماره ۳۱۱۷: پذیرفت این دل ز عشقت خرابی}
\label{sec:3117}
\addcontentsline{toc}{section}{\nameref{sec:3117}}
\begin{longtable}{l p{0.5cm} r}
پذیرفت این دل ز عشقت خرابی
&&
درآ در خرابی چو تو آفتابی
\\
چه گویی دلم را که از من نترسی
&&
ز دریا نترسد چنین مرغ آبی
\\
منم دل سپرده برانداز پرده
&&
که عمریست ای جان که اندر حجابی
\\
چو پرده برانداخت گفتم دلا هی
&&
به بیداریست این عجب یا به خوابی
\\
بگفتم زمانی چنین باش پیدا
&&
بگفتا که شاید ولی برنتابی
\\
دلم صد هزاران سخن راند ز آن خوش
&&
مرا گفت بشنو گر اهل خطابی
\\
که گر او نه آبست باغ از چه خندد
&&
وگر آتشی نیست چون دل کبابی
\\
از این جنس باران و برقش جهان شد
&&
در اسرار عشقش چو ابر سحابی
\\
بگفتم خمش کن چو تو مست عشقی
&&
مثال صراحی پر از خون نابی
\\
دلا چند باشی تو سرمست گفتن
&&
چو در عین آبی چه مست سرابی
\\
بر این و بر آن تو منه این بهانه
&&
تو خود را برون کن که خود را عذابی
\\
من و ماست کهگل سر خم گرفته
&&
تو بردار کهگل که خم شرابی
\\
دلا خون نخسپد و دانم که تو دل
&&
تو آن سیل خونی که دریا بیابی
\\
بهانه‌ست این‌ها بیا شمس تبریز
&&
که مفتاح عرشی و فتاح بابی
\\
\end{longtable}
\end{center}
