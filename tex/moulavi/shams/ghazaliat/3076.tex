\begin{center}
\section*{غزل شماره ۳۰۷۶: بلندتر شده‌ست آفتاب انسانی}
\label{sec:3076}
\addcontentsline{toc}{section}{\nameref{sec:3076}}
\begin{longtable}{l p{0.5cm} r}
بلندتر شده‌ست آفتاب انسانی
&&
زهی حلاوت و مستی و عشق و آسانی
\\
جهان ز نور تو ناچیز شد چه چیزی تو
&&
طلسم دلبریی یا تو گنج جانانی
\\
زهی قلم که تو را نقش کرد در صورت
&&
که نامه همه را نانبشته می‌خوانی
\\
برون بری تو ز خرگاه شش جهت جان را
&&
چو جان نماند بر جاش عشق بنشانی
\\
دلا چو باز شهنشاه صید کرد تو را
&&
تو ترجمانبگ سر زبان مرغانی
\\
چه ترجمان که کنون بس بلند سیمرغی
&&
که آفت نظر جان صد سلیمانی
\\
درید چارق ایمان و کفر در طلبت
&&
هزارساله از آن سوی کفر و ایمانی
\\
به هر سحر که درخشی خروس جان گوید
&&
بیا که جان و جهانی برو که سلطانی
\\
چو روح من بفزوده‌ست شمس تبریزی
&&
به سوی او برم از باغ روح ریحانی
\\
\end{longtable}
\end{center}
