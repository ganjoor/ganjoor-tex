\begin{center}
\section*{غزل شماره ۲۰۰۵: چه نشستی دور چون بیگانگان}
\label{sec:2005}
\addcontentsline{toc}{section}{\nameref{sec:2005}}
\begin{longtable}{l p{0.5cm} r}
چه نشستی دور چون بیگانگان
&&
اندرآ در حلقه دیوانگان
\\
شرم چه بود عاشقی و آن گاه شرم
&&
جان چه باشد این هوس و آن گاه جان
\\
می‌فروشد او به جانی بوسه‌ای
&&
رو بخر کان رایگان است رایگان
\\
آنک عشقش خانه‌ها برهم زده‌ست
&&
آمد اندر خانه همسایگان
\\
کف برآورده‌ست این دریا ز عشق
&&
سر فروکرده‌ست آن مه ز آسمان
\\
ای ببسته خواب‌ها امشب بیا
&&
خواب ما را بین چو وصلت بی‌نشان
\\
هر شهی را بندگانش حارسند
&&
شاه ما مر بندگان را پاسبان
\\
شاه ما از خواب و بیداری برون
&&
در میان جان ما دامن کشان
\\
اندر این شب می‌نماید صورتی
&&
مشعله در دست یا رب کیست آن
\\
خواب جست و شورش افزودن گرفت
&&
یاد آمد پیل را هندوستان
\\
آتش عشق خدا بالا گرفت
&&
تیر تقدیر خدا جست از کمان
\\
دانه‌ای کان در زمین غیب بود
&&
سر زد و همچون درختی شد عیان
\\
برق جست و آتشی زد در درخت
&&
آتش و برق شگرف بی‌امان
\\
سبزتر می‌شد ز آتش آن درخت
&&
می‌شکفت از برق و آتش گلستان
\\
این درختان سبز از آتش شوند
&&
آب دارد این درختان را زبان
\\
تا تویی پیدا نهان گردد درخت
&&
او شود پیدا چو تو گردی نهان
\\
شمس تبریز است باغ عشق را
&&
هم طراوت هم نما هم باغبان
\\
\end{longtable}
\end{center}
