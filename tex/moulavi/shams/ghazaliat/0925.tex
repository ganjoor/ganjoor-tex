\begin{center}
\section*{غزل شماره ۹۲۵: دو ماه پهلوی همدیگرند بر در عید}
\label{sec:0925}
\addcontentsline{toc}{section}{\nameref{sec:0925}}
\begin{longtable}{l p{0.5cm} r}
دو ماه پهلوی همدیگرند بر در عید
&&
مه مصور یار و مه منور عید
\\
چو هر دو سر به هم آورده‌اند در اسرار
&&
هزار وسوسه افکنده‌اند در سر عید
\\
ز موج بحر برقصند خلق همچو صدف
&&
ولیک همچو صدف بی‌خبر ز گوهر عید
\\
ز عید باقی این عید آمده‌ست رسول
&&
چو دل به عید سپاری تو را برد بر عید
\\
به روز عید بگویم دهل چه می‌گوید
&&
اگر تو مردی برجه رسید لشکر عید
\\
قراضه دو که دادی برای حق بنگر
&&
جزای حسن عمل گیر گنج پرزر عید
\\
وگر چو شیشه شکستی ز سنگ صوم و جهاد
&&
می حلال سقا هم بکش ز ساغر عید
\\
از این شکار سوی شاه بازپر چون باز
&&
که درپرید به مژده ز شه کبوتر عید
\\
تو گاو فربه حرصت به روزه قربان کن
&&
که تا بری به تبرک هلال لاغر عید
\\
وگر نکردی قربان عنایت یزدان
&&
امید هست که ذبحش کند به خنجر عید
\\
\end{longtable}
\end{center}
