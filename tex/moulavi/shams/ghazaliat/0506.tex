\begin{center}
\section*{غزل شماره ۵۰۶: کار من اینست که کاریم نیست}
\label{sec:0506}
\addcontentsline{toc}{section}{\nameref{sec:0506}}
\begin{longtable}{l p{0.5cm} r}
کار من اینست که کاریم نیست
&&
عاشقم از عشق تو عاریم نیست
\\
تا که مرا شیر غمت صید کرد
&&
جز که همین شیر شکاریم نیست
\\
در تک این بحر چه خوش گوهری
&&
که مثل موج قراریم نیست
\\
بر لب بحر تو مقیمم مقیم
&&
مست لبم گر چه کناریم نیست
\\
وقف کنم اشکم خود بر میت
&&
کز می تو هیچ خماریم نیست
\\
می‌رسدم باده تو ز آسمان
&&
منت هر شیره فشاریم نیست
\\
باده‌ات از کوه سکونت برد
&&
عیب مکن زان که وقاریم نیست
\\
ملک جهان گیرم چون آفتاب
&&
گر چه سپاهی و سواریم نیست
\\
می‌کشم از مصر شکر سوی روم
&&
گر چه شتربان و قطاریم نیست
\\
گر چه ندارم به جهان سروری
&&
دردسر بیهده باریم نیست
\\
بر سر کوی تو مرا خانه گیر
&&
کز سر کوی تو گذاریم نیست
\\
همچو شکر با گلت آمیختم
&&
نیست عجب گر سر خاریم نیست
\\
قطب جهانی همه را رو به توست
&&
جز که به گرد تو دواریم نیست
\\
خویش من آنست که از عشق زاد
&&
خوشتر از این خویش و تباریم نیست
\\
چیست فزون از دو جهان شهر عشق
&&
بهتر از این شهر و دیاریم نیست
\\
گر ننگارم سخنی بعد از این
&&
نیست از آن رو که نگاریم نیست
\\
\end{longtable}
\end{center}
