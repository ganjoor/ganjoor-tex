\begin{center}
\section*{غزل شماره ۲۳۱۹: آن یار غریب من آمد به سوی خانه}
\label{sec:2319}
\addcontentsline{toc}{section}{\nameref{sec:2319}}
\begin{longtable}{l p{0.5cm} r}
آن یار غریب من آمد به سوی خانه
&&
امروز تماشا کن اشکال غریبانه
\\
یاران وفا را بین اخوان صفا را بین
&&
در رقص که بازآمد آن گنج به ویرانه
\\
ای چشم چمن می‌بین وی گوش سخن می‌چین
&&
بگشای لب نوشین ای یار خوش افسانه
\\
امروز می باقی بی‌صرفه ده ای ساقی
&&
از بحر چه کم گردد زین یک دو سه پیمانه
\\
پیمانه و پیمانه در باده دوی نبود
&&
خواهی که یکی گردد بشکن تو دو پیمانه
\\
من باز شکارم جان دربند مدارم جان
&&
زین بیش نمی‌باشم چون جغد به ویرانه
\\
قانع نشوم با تو صبر از دل من گم شد
&&
رو با دگری می‌گو من نشنوم افسانه
\\
من دانه افلاکم یک چند در این خاکم
&&
چون عدل بهار آمد سرسبز شود دانه
\\
تو آفت مرغانی زان دانه که می‌دانی
&&
یک مشت برافشانی ز انبار پر از دانه
\\
ای داده مرا رونق صد چون فلک ازرق
&&
ای دوست بگو مطلق این هست چنین یا نه
\\
بار دگر ای جان تو زنجیر بجنبان تو
&&
وز دور تماشا کن در مردم دیوانه
\\
خود گلشن بخت است این یا رب چه درخت است این
&&
صد بلبل مست این جا هر لحظه کند لانه
\\
جان گوش کشان آید دل سوی خوشان آید
&&
زیرا که بهار آمد شد آن دی بیگانه
\\
\end{longtable}
\end{center}
