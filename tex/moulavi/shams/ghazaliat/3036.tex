\begin{center}
\section*{غزل شماره ۳۰۳۶: گاه چو اشتر در وحل آیی}
\label{sec:3036}
\addcontentsline{toc}{section}{\nameref{sec:3036}}
\begin{longtable}{l p{0.5cm} r}
گاه چو اشتر در وحل آیی
&&
گه چو شکاری در عجل آیی
\\
کجکنن اغلن چند گریزی
&&
عاقبت آخر در عمل آیی
\\
در سوی بی‌سو می‌رو و می‌جو
&&
تا کی ای دل در علل آیی
\\
در طلبی تو در طرب افتی
&&
در نمدی تو در حلل آیی
\\
دردسر آید شور و شر آید
&&
عاشق شو تا بی‌خلل آیی
\\
نفخ کند جان در دل ترسان
&&
مطرب جویی در غزل آیی
\\
چونک قویتر دردمد آن نی
&&
در رخ دلبر مکتحل آیی
\\
چنگ بگیری ننگ پذیری
&&
فاعل نبوی مفتعل آیی
\\
از غم دلبر در برش افتی
&&
در کف اویی در بغل آیی
\\
فکر رها کن ترک نهی کن
&&
زانک ز حیرت با دول آیی
\\
فکر چو آید ضد ورا بین
&&
زین دو به حیرت محتمل آیی
\\
زانک تردد آرد به حیرت
&&
زین دو تحول در محل آیی
\\
ز اول فکرت آخر ره بین
&&
چند به گفتن منتقل آیی
\\
\end{longtable}
\end{center}
