\begin{center}
\section*{غزل شماره ۱۴: ای عاشقان ای عاشقان امروز ماییم و شما}
\label{sec:0014}
\addcontentsline{toc}{section}{\nameref{sec:0014}}
\begin{longtable}{l p{0.5cm} r}
ای عاشقان ای عاشقان امروز ماییم و شما
&&
افتاده در غرقابه‌ای تا خود که داند آشنا
\\
گر سیل عالم پر شود هر موج چون اشتر شود
&&
مرغان آبی را چه غم تا غم خورد مرغ هوا
\\
ما رخ ز شکر افروخته با موج و بحر آموخته
&&
زان سان که ماهی را بود دریا و طوفان جان فزا
\\
ای شیخ ما را فوطه ده وی آب ما را غوطه ده
&&
ای موسی عمران بیا بر آب دریا زن عصا
\\
این باد اندر هر سری سودای دیگر می‌پزد
&&
سودای آن ساقی مرا باقی همه آن شما
\\
دیروز مستان را به ره بربود آن ساقی کله
&&
امروز می در می‌دهد تا برکند از ما قبا
\\
ای رشک ماه و مشتری با ما و پنهان چون پری
&&
خوش خوش کشانم می‌بری آخر نگویی تا کجا
\\
هر جا روی تو با منی ای هر دو چشم و روشنی
&&
خواهی سوی مستیم کش خواهی ببر سوی فنا
\\
عالم چو کوه طور دان ما همچو موسی طالبان
&&
هر دم تجلی می‌رسد برمی‌شکافد کوه را
\\
یک پاره اخضر می‌شود یک پاره عبهر می‌شود
&&
یک پاره گوهر می‌شود یک پاره لعل و کهربا
\\
ای طالب دیدار او بنگر در این کهسار او
&&
ای که چه باد خورده‌ای ما مست گشتیم از صدا
\\
ای باغبان ای باغبان در ما چه درپیچیده‌ای
&&
گر برده‌ایم انگور تو تو برده‌ای انبان ما
\\
\end{longtable}
\end{center}
