\begin{center}
\section*{غزل شماره ۲۴۵۶: هم نظری هم خبری هم قران را قمری}
\label{sec:2456}
\addcontentsline{toc}{section}{\nameref{sec:2456}}
\begin{longtable}{l p{0.5cm} r}
هم نظری هم خبری هم قمران را قمری
&&
هم شکر اندر شکر اندر شکر اندر شکری
\\
هم سوی دولت درجی هم غم ما را فرجی
&&
هم قدحی هم فرحی هم شب ما را سحری
\\
هم گل سرخ و سمنی در دل گل طعنه زنی
&&
سوی فلک حمله کنی زهره و مه را ببری
\\
چند فلک گشت قمر تا به خودش راه دهی
&&
چند گدازید شکر تا تو بدو درنگری
\\
چند جنون کرد خرد در هوس سلسله‌ای
&&
چند صفت گشت دلم تا تو بر او برگذری
\\
آن قدح شاده بده دم مده و باده بده
&&
هین که خروس سحری مانده شد از ناله گری
\\
گر به خرابات بتان هر طرفی لاله رخی است
&&
لاله رخا تو ز یکی لاله ستان دگری
\\
هم تو جنون را مددی هم تو جمال خردی
&&
تیر بلا از تو رسد هم تو بلا را سپری
\\
چونک صلاح دل و دین مجلس دل را شد امین
&&
مادر دولت بکند دختر جان را پدری
\\
\end{longtable}
\end{center}
