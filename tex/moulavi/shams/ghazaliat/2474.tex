\begin{center}
\section*{غزل شماره ۲۴۷۴: ریگ ز آب سیر شد من نشدم زهی زهی}
\label{sec:2474}
\addcontentsline{toc}{section}{\nameref{sec:2474}}
\begin{longtable}{l p{0.5cm} r}
ریگ ز آب سیر شد من نشدم زهی زهی
&&
لایق خرکمان من نیست در این جهان زهی
\\
بحر کمینه شربتم کوه کمینه لقمه‌ام
&&
من چه نهنگم ای خدا بازگشا مرا رهی
\\
تشنه‌تر از اجل منم دوزخ وار می‌تنم
&&
هیچ رسد عجب مرا لقمه زفت فربهی
\\
نیست نزار عشق را جز که وصال داروی
&&
نیست دهان عشق را جز کف تو علف دهی
\\
عقل به دام تو رسد هم سر و ریش گم کند
&&
گر چه بود گران سری گر چه بود سبک جهی
\\
صدق نهنده هم تویی در دل هر موحدی
&&
نقش کننده هم تویی در دل هر مشبهی
\\
نوح ز اوج موج تو گشته حریف تخته‌ای
&&
روح ز بوی کوی تو مست و خراب و والهی
\\
خامش باش و بازرو جانب قصر خامشان
&&
باز به شهر عشق رو ای تو فکنده در دهی
\\
\end{longtable}
\end{center}
