\begin{center}
\section*{غزل شماره ۲۲۱۳: خنک آن جان که رود مست و خرامان بر او}
\label{sec:2213}
\addcontentsline{toc}{section}{\nameref{sec:2213}}
\begin{longtable}{l p{0.5cm} r}
خنک آن جان که رود مست و خرامان بر او
&&
برهد از خر تن در سفر مصدر او
\\
خلع نعلین کند وز خود و دنیا بجهد
&&
همچو موسی قدم صدق زند بر در او
\\
همچو جرجیس شود کشته عشقش صد بار
&&
یا چو اسحاق شود بسمل از آن خنجر او
\\
سر دیگر رسدش جز سر پردرد و صداع
&&
مغفرت بنهد بر فرق سرش مغفر او
\\
کیله رزقش اگر درشکند میکائیل
&&
عوضش گاه بود خلد و گهی کوثر او
\\
پدر و مادر و خویشان چو به خاکش بنهند
&&
شود او ماهی و دریا پدر و مادر او
\\
عشق دریای حیات است که او را تک نیست
&&
عمر جاوید بود موهبت کمتر او
\\
می‌رود شمس و قمر هر شب در گور غروب
&&
می‌دهدشان فر نو شعشعه گوهر او
\\
ملک الموت به صد ناز ستاند جانی
&&
که بود باخبر و دیده ور از محشر او
\\
تن ما خفته در آن خاک به چشم عامه
&&
روح چون سرو روان در چمن اخضر او
\\
نه به ظاهر تن ما معدن خون و خلط است
&&
هیچ جان را سقمی هست از این مقذر او
\\
در چنین مزبله جان را دو هزاران باغ است
&&
پس چرا ترسد جان از لحد و مقبر او
\\
آنک خون را چو می ناب غذای جان کرد
&&
بنگر در تن پرنور و رخ احمر او
\\
هله دلدار بخوان باقی این بر منکر
&&
تا دو صد چشمه روان گردد از مرمر او
\\
\end{longtable}
\end{center}
