\begin{center}
\section*{غزل شماره ۲۰۳۱: ای محو راه گشته از محو هم سفر کن}
\label{sec:2031}
\addcontentsline{toc}{section}{\nameref{sec:2031}}
\begin{longtable}{l p{0.5cm} r}
ای محو راه گشته از محو هم سفر کن
&&
چشمی ز دل برآور در عین دل نظر کن
\\
دل آینه است چینی با دل چو همنشینی
&&
صد تیغ اگر ببینی هم دیده را سپر کن
\\
دانم که برشکستی تو محو دل شدستی
&&
در عین نیست هستی یک حمله دگر کن
\\
تا بشکنی شکاری پهلوی چشمه ساری
&&
ای شیر بیشه دل چنگال در جگر کن
\\
چون شد گرو گلیمی بهر در یتیمی
&&
با فتنه عظیمی تو دست در کمر کن
\\
ماییم ذره ذره در آفتاب غره
&&
از ذره خاک بستان در دیده قمر کن
\\
از ما نماند برجا جان از جنون و سودا
&&
ای پادشاه بینا ما را ز خود خبر کن
\\
در عالم منقش ای عشق همچو آتش
&&
هر نقش را به خود کش وز خویش جانور کن
\\
ای شاه هر چه مردند رندان سلام کردند
&&
مستند و می نخوردند آن سو یکی گذر کن
\\
سیمرغ قاف خیزد در عشق شمس تبریز
&&
آن پر هست برکن وز عشق بال و پر کن
\\
\end{longtable}
\end{center}
