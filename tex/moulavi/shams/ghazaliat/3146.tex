\begin{center}
\section*{غزل شماره ۳۱۴۶: ای دلزار محنت و بلا داری}
\label{sec:3146}
\addcontentsline{toc}{section}{\nameref{sec:3146}}
\begin{longtable}{l p{0.5cm} r}
ای دلزار محنت و بلا داری
&&
بر خدا اعتمادها داری
\\
اینچنین حضرتی و تو نومید؟
&&
مکن ای دل، اگر خدا داری
\\
رخت اندیشه می‌کشی هرجا
&&
بنگر آخر، جز او کرا داری؟
\\
لطفهایی که کرد چندین گاه
&&
یاد آور اگر وفاداری
\\
چشم سر داد و چشم سر ایزد
&&
چشم جای دگر چرا داری؟!
\\
عمر ضایع مکن، که عمر گذشت
&&
زرگری کن، که کیمیا داری
\\
هر سحر مر ترا ندا آید
&&
سو ما آ، که داغ ما داری
\\
پیش ازین تن تو جان پاک بدی
&&
چند خود را ازان جدا داری؟!
\\
جان پاکی، میان خاک سیاه
&&
من نگویم، تو خود روا داری؟!
\\
خویشتن را تو از قبا بشناس
&&
که ازین آب و گل قبا داری
\\
می‌روی هر شب از قبا بیرون
&&
که جز این دست، دست و پا داری
\\
بس بود، این قدر بدان گفتم
&&
که درین کوچه آشنا داری
\\
\end{longtable}
\end{center}
