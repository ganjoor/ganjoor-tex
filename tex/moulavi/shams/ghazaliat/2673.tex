\begin{center}
\section*{غزل شماره ۲۶۷۳: سؤالی دارم ای خواجه خدایی}
\label{sec:2673}
\addcontentsline{toc}{section}{\nameref{sec:2673}}
\begin{longtable}{l p{0.5cm} r}
سؤالی دارم ای خواجه خدایی
&&
که امروز این چنین شیرین چرایی
\\
کی باشد مه که گویم ماه رویی
&&
کی باشد جان که گویم جان فزایی
\\
مثالی لایق آن روی خوبت
&&
بسی شب‌ها ز حق کردم گدایی
\\
رها کن این همه با ما تو چونی
&&
تو جانی و به چونی درنیایی
\\
تو صدساله ره از چونی گذشتی
&&
میان موج‌های کبریایی
\\
هوای خویشتن را سر بریدی
&&
ز میل نفس خود کردی جدایی
\\
همه میل دل معشوق گشتی
&&
به تسلیم و رضا و مرتضایی
\\
از این هم درگذشتم چونی ای جان
&&
که این دم رستخیز سحرهایی
\\
همی‌پیچی به صد گون چشم ما را
&&
به صد صورت جهان را می‌نمایی
\\
زمانی صورت زندان و چاهی
&&
زمانی گلستان و دلربایی
\\
همان یک چیز را گه مار سازی
&&
گهی بخشی درختی و عصایی
\\
به دست توست بوقلمون همه چیز
&&
ز انسان و ز حیوان و نمایی
\\
گهی نیل است و گاهی خون بسته
&&
گهی لیل است و گه صبح ضیایی
\\
بدین خوف و رجاها منعقد شد
&&
که از هر ضد ضد بر می‌گشایی
\\
سؤالی چند دارم از تو حل کن
&&
که مشکل‌های ما را مرتجایی
\\
سؤال اول آن است ای سخندان
&&
که هم اول هم آخر جان مایی
\\
چو اول هم تویی و آخر تویی هم
&&
ز کی دانم وفا و بی‌وفایی
\\
دوم آن است ای آن کت دوم نیست
&&
که رنج احولی را توتیایی
\\
\end{longtable}
\end{center}
