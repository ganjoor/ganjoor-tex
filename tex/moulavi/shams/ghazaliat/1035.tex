\begin{center}
\section*{غزل شماره ۱۰۳۵: ای عاشق بیچاره شده زار به زر بر}
\label{sec:1035}
\addcontentsline{toc}{section}{\nameref{sec:1035}}
\begin{longtable}{l p{0.5cm} r}
ای عاشق بیچاره شده زار به زر بر
&&
گویی که نزد مرگ تو را حلقه به در بر
\\
بندیش از آن روز که دم‌های شماری
&&
تو می‌زنی و وهم زنت شوی دگر بر
\\
خود را تو سپر کن به قبول همه احکام
&&
زان پیش که تیر اجل آید به سپر بر
\\
از آدمی ادراک و نظر باشد مقصود
&&
کای رحمت پیوسته به ادراک و نظر بر
\\
ای کان شکر فضل تو وین خلق چو طوطی
&&
طوطی چه کند که ننهد دل به شکر بر
\\
آن نیشکر از عشق تو صد جای کمر بست
&&
شکر تو نبشتست بر اطراف کمر بر
\\
جز شمس و قمر باصره را نور دگر ده
&&
ای نور تو وافر شده بر شمس و قمر بر
\\
از کار جهان سیر شده خاطر عارف
&&
عاشق شده بر شیوه و بر کار دگر بر
\\
دیدست که گر نوش کند آب جهان را
&&
بی‌حضرت تو آب ندارد به جگر بر
\\
گیرم همه شب پاس نداری و نزاری
&&
خود را بزن ای مخلص بر ورد سحر بر
\\
آن‌ها که شب و صبحدم آرام ندیدند
&&
ناگاه فتادند بر آن گنج گهر بر
\\
موسی همه شب نور همی‌جست و به آخر
&&
نوری عجبی دید به بالای شجر بر
\\
یعقوب وطن ساخت به جان طره شب را
&&
تا بوسه زد آخر به رخ و زلف پسر بر
\\
مقصود خدا بود و پسر بود بهانه
&&
عاشق نشود جان پیمبر به بشر بر
\\
او ز آل خلیلست و به آفل نکند میل
&&
چون خار بود آفل او را به بصر بر
\\
جز دوست خلیلی نپذیرفت خلیلش
&&
ور نه تن خود را نفکندی به شرر بر
\\
ای گشته بت جان تو نقشی و کلوخی
&&
انکار تو پس چیست به عباد حجر بر
\\
یک لحظه بنه گوش که خواهم سخنی گفت
&&
ای چشم خوشت طعنه زده نرگس تر بر
\\
بر نقد زن ای دوست که محبوب تو نقدست
&&
ای چشم نهاده همه بر بوک و مگر بر
\\
بربستم لب را ز ره چشم بگویم
&&
چیزی که رود مستی آن کله سر بر
\\
نی نی بنگویم که عجب صید شگرفست
&&
مرغ نظرست و ننشیند به خبر بر
\\
\end{longtable}
\end{center}
