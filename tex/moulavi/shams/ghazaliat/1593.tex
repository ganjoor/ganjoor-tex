\begin{center}
\section*{غزل شماره ۱۵۹۳: روی نیکت بد کند من نیک را بر بد نهم}
\label{sec:1593}
\addcontentsline{toc}{section}{\nameref{sec:1593}}
\begin{longtable}{l p{0.5cm} r}
روی نیکت بد کند من نیک را بر بد نهم
&&
عاشقی بس پخته‌ام این ننگ را بر خود نهم
\\
ننگ عاشق ننگ دارد از همه فخر جهان
&&
ننگ را من بر سر آن عشرت بی‌حد نهم
\\
علم چون چادر گشاید در برم گیرد به لطف
&&
حرف‌های علم را بر گردن ابجد نهم
\\
تاج زرین چون نهد از عاشقی بر فرق من
&&
تخت خود را من برآرم بر سر فرقد نهم
\\
چون در آب زندگانی صورتم پنهان شود
&&
صورت خود را به پیش صورت احمد نهم
\\
نام شمس الدین تبریزی چو بنویسم بدانک
&&
شکر دلخواه را در اشکم کاغذ نهم
\\
\end{longtable}
\end{center}
