\begin{center}
\section*{غزل شماره ۱۶۲۶: فلکا بگو که تا کی گله‌های یار گویم}
\label{sec:1626}
\addcontentsline{toc}{section}{\nameref{sec:1626}}
\begin{longtable}{l p{0.5cm} r}
فلکا بگو که تا کی گله‌های یار گویم
&&
نبود شبی که آیم ز میان کار گویم
\\
ز میان او مقامم کمر است و کوه و صحرا
&&
بجهم از این میان و سخن و کنار گویم
\\
ز فراق گلستانش چو در امتحان خارم
&&
برهم ز خار چون گل سخن از عذار گویم
\\
همه بانگ زاغ آید به خرابه‌های بهمن
&&
برهم از این چو بلبل صفت بهار گویم
\\
گرهی ز نقد غنچه بنهم به پیش سوسن
&&
صفتی ز رنگ لاله به بنفشه زار گویم
\\
بکشد ز کبر دامن دل من چو دلبر آید
&&
بدرد نظر گریبان چو ز انتظار گویم
\\
بنهد کلاه از سر خم خاص خسروانی
&&
بجهد ز مهر ساقی چو من از خمار گویم
\\
\end{longtable}
\end{center}
