\begin{center}
\section*{غزل شماره ۲۵۹۱: ای پرده در پرده بنگر که چه‌ها کردی}
\label{sec:2591}
\addcontentsline{toc}{section}{\nameref{sec:2591}}
\begin{longtable}{l p{0.5cm} r}
ای پرده در پرده بنگر که چه‌ها کردی
&&
دل بودی و جان بردی این جا چه رها کردی
\\
خورشید جهانی تو سلطان شهانی تو
&&
بی‌هوشی جانی تو گیرم که جفا کردی
\\
هم عاقبت ای سلطان بردی همه را مهمان
&&
در بخشش و در احسان حاجات روا کردی
\\
هر سنگ که بگرفتی لعل و گهرش کردی
&&
هر پشه که پروردی صد همچو هما کردی
\\
یک طایفه را ای جان منشور خطا دادی
&&
یک قافله را ناگه اصحاب صفا کردی
\\
آثار فلک‌ها را اجزای زمین کردی
&&
اجزای زمین‌ها را در لطف سما کردی
\\
پس من ز چه بشناسم از چرخ زمین‌ها را
&&
چون قاعده بشکستی وز درد دوا کردی
\\
\end{longtable}
\end{center}
