\begin{center}
\section*{غزل شماره ۲۵۶۵: ای دوست ز شهر ما ناگه به سفر رفتی}
\label{sec:2565}
\addcontentsline{toc}{section}{\nameref{sec:2565}}
\begin{longtable}{l p{0.5cm} r}
ای دوست ز شهر ما ناگه به سفر رفتی
&&
ما تلخ شدیم و تو در کان شکر رفتی
\\
نوری که بدو پرد جان از قفس قالب
&&
در تو نظری کرد او در نور نظر رفتی
\\
رفتی تو از این پستی در شادی و در مستی
&&
آن سوی زبردستی گر زیر و زبر رفتی
\\
مانند خیالی تو هر دم به یکی صورت
&&
زین شکل برون جستی در شکل دگر رفتی
\\
امروز چو جانستی در صدر جنانستی
&&
از دور قمر رستی بالای قمر رفتی
\\
اکنون ز تن گریان جانا شده‌ای عریان
&&
چون ترک کله کردی وز بند کمر رفتی
\\
از نان شده‌ای فارغ وز منت خبازان
&&
وز آب شدی فارغ کز تف جگر رفتی
\\
نانی دهدت جانان بی‌معده و بی‌دندان
&&
آبی دهدت صافی زان بحر که دررفتی
\\
از جان شریف خود وز حال لطیف خود
&&
بفرست خبر زیرا در عین خبر رفتی
\\
ور ز آنک خبر ندهی دانم که کجاهایی
&&
در دامن دریایی چون در و گهر رفتی
\\
هان ای سخن روشن درتاب در این روزن
&&
کز گوش گذر کردی در عقل و بصر رفتی
\\
\end{longtable}
\end{center}
