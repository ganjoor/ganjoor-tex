\begin{center}
\section*{غزل شماره ۴۶۷: آنک چنان می‌رود ای عجب او جان کیست}
\label{sec:0467}
\addcontentsline{toc}{section}{\nameref{sec:0467}}
\begin{longtable}{l p{0.5cm} r}
آنک چنان می‌رود ای عجب او جان کیست
&&
سخت روان می‌رود سرو خرامان کیست
\\
حلقه آن جعد او سلسله پای کیست
&&
زلف چلیپا و شش آفت ایمان کیست
\\
در دل ما صورتیست ای عجب آن نقش کیست
&&
وین همه بوهای خوش از سوی بستان کیست
\\
دیدم آن شاه را آن شه آگاه را
&&
گفتم این شاه کیست خسرو و سلطان کیست
\\
چون سخن من شنید گفت به خاصان خویش
&&
کاین همه درد از کجاست حال پریشان کیست
\\
عقل روان سو به سو روح دوان کو به کو
&&
دل همه در جست و جو یا رب جویان کیست
\\
دل چه نهی بر جهان باش در او میهمان
&&
بنده آن شو که او داند مهمان کیست
\\
در دل من دار و گیر هست دو صد شاه و میر
&&
این دل پرغلغله مجلس و ایوان کیست
\\
عرصه دل بی‌کران گم شده در وی جهان
&&
ای دل دریاصفت سینه بیابان کیست
\\
غم چه کند با کسی داند غم از کجاست
&&
شاد ابد گشت آنک داند شادان کیست
\\
ای زده لاف کرم گفته که من محسنم
&&
مرگ تو گوید تو را کاین همه احسان کیست
\\
آن دم کاین دوستان با تو دگرگون شوند
&&
پس تو بدانی که این جمله طلسم آن کیست
\\
نقد سخن را بمان سکه سلطان بجو
&&
کای زر کامل عیار نقد تو از کان کیست
\\
\end{longtable}
\end{center}
