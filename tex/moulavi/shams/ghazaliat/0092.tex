\begin{center}
\section*{غزل شماره ۹۲: زهی باغ زهی باغ که بشکفت ز بالا}
\label{sec:0092}
\addcontentsline{toc}{section}{\nameref{sec:0092}}
\begin{longtable}{l p{0.5cm} r}
زهی باغ زهی باغ که بشکفت ز بالا
&&
زهی قدر و زهی بدر تبارک و تعالی
\\
زهی فر زهی نور زهی شر زهی شور
&&
زهی گوهر منثور زهی پشت و تولا
\\
زهی ملک زهی مال زهی قال زهی حال
&&
زهی پر و زهی بال بر افلاک تجلی
\\
چو جان سلسله‌ها را بدرد به حرونی
&&
چه ذاالنون چه مجنون چه لیلی و چه لیلا
\\
علم‌های الهی ز پس کوه برآمد
&&
چه سلطان و چه خاقان چه والی و چه والا
\\
چه پیش آمد جان را که پس انداخت جهان را
&&
بزن گردن آن را که بگوید که تسلا
\\
چو بی‌واسطه جبار بپرورد جهان را
&&
چه ناقوس چه ناموس چه اهلا و چه سهلا
\\
گر اجزای زمینی وگر روح امینی
&&
چو آن حال ببینی بگو جل جلالا
\\
گر افلاک نباشد به خدا باک نباشد
&&
دل غمناک نباشد مکن بانگ و علالا
\\
فروپوش فروپوش نه بخروش نه بفروش
&&
تویی باده مدهوش یکی لحظه بپالا
\\
تو کرباسی و قصار تو انگوری و عصار
&&
بپالا و بیفشار ولی دست میالا
\\
خمش باش خمش باش در این مجمع اوباش
&&
مگو فاش مگو فاش ز مولی و ز مولا
\\
\end{longtable}
\end{center}
