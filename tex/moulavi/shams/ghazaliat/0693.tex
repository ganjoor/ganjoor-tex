\begin{center}
\section*{غزل شماره ۶۹۳: جانی که ز نور مصطفی زاد}
\label{sec:0693}
\addcontentsline{toc}{section}{\nameref{sec:0693}}
\begin{longtable}{l p{0.5cm} r}
جانی که ز نور مصطفی زاد
&&
با او تو مگو ز داد و بیداد
\\
هرگز ماهی سباحت آموخت
&&
آزادی جست سرو آزاد
\\
خاری که ز گلبن طرب رست
&&
گلزار به روی او شود شاد
\\
دورست رواق‌های شادی
&&
از آتش و آب و خاک و از باد
\\
زین چار بسیط چون چلیپا
&&
ترکیب موحدان برون باد
\\
زان سو فلکیست نیک روشن
&&
زان سو ملکیست بسته مرصاد
\\
کمتر بخشش دو چشم بخشد
&&
بینا و حکیم و تیز و استاد
\\
با دیده جان چو واپس آیی
&&
در عالم آب و گل به ارشاد
\\
بینی تو و دیگران نبینند
&&
هر سو نوری به رسم میلاد
\\
در هر ابری هزار خورشید
&&
در هر ویران بهشت آباد
\\
تختی بنهی به قصر مردان
&&
هم خیمه زنی به بام اوتاد
\\
بویی ببری ز شمس تبریز
&&
کو را است ملک مطیع و منقاد
\\
\end{longtable}
\end{center}
