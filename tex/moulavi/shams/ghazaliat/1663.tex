\begin{center}
\section*{غزل شماره ۱۶۶۳: نو به نو هر روز باری می کشم}
\label{sec:1663}
\addcontentsline{toc}{section}{\nameref{sec:1663}}
\begin{longtable}{l p{0.5cm} r}
نو به نو هر روز باری می کشم
&&
وین بلا از بهر کاری می کشم
\\
زحمت سرما و برف ماه دی
&&
بر امید نوبهاری می کشم
\\
پیش آن فربه کن هر لاغری
&&
این چنین جسم نزاری می کشم
\\
از دو صد شهرم اگر بیرون کنند
&&
بهر عشق شهریاری می کشم
\\
گر دکان و خانه‌ام ویران شود
&&
بر وفای لاله زاری می کشم
\\
عشق یزدان پس حصاری محکم است
&&
رخت جان اندر حصاری می کشم
\\
ناز هر بیگانه سنگین دلی
&&
بهر یاری بردباری می کشم
\\
بهر لعلش کوه و کانی می کنم
&&
بهر آن گل بار خاری می کشم
\\
بهر آن دو نرگس مخمور او
&&
همچو مخموران خاری می کشم
\\
بهر صیدی کو نمی‌گنجد به دام
&&
دام و داهول شکاری می کشم
\\
گفت ای غم تا قیامت می کشی
&&
می کشم ای دوست آری می کشم
\\
سینه غار و شمس تبریزی است یار
&&
سخره بهر یار غاری می کشم
\\
\end{longtable}
\end{center}
