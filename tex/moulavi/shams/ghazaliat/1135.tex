\begin{center}
\section*{غزل شماره ۱۱۳۵: بیار ساقی بادت فدا سر و دستار}
\label{sec:1135}
\addcontentsline{toc}{section}{\nameref{sec:1135}}
\begin{longtable}{l p{0.5cm} r}
بیار ساقی بادت فدا سر و دستار
&&
ز هر کجا که دهد دست جام جان دست آر
\\
درآی مست و خرامان و ساغر اندر دست
&&
روا مبین چو تو ساقی و ما چنین هشیار
\\
بیار جام که جانم ز آرزومندی
&&
ز خویش نیز برآمد چه جای صبر و قرار
\\
بیار جام حیاتی که هم مزاج توست
&&
که مونس دل خسته‌ست و محرم اسرار
\\
از آن شراب که گر جرعه‌ای از او بچکد
&&
ز خاک شوره بروید همان زمان گلزار
\\
شراب لعل که گر نیم شب برآرد جوش
&&
میان چرخ و زمین پر شود از او انوار
\\
زهی شراب و زهی ساغر و زهی ساقی
&&
که جان‌ها و روان‌ها نثار باد نثار
\\
بیا که در دل من رازهای پنهانست
&&
شراب لعل بگردان و پرده‌ای مگذار
\\
مرا چو مست کنی آنگهی تماشا کن
&&
که شیرگیر چگونست در میان شکار
\\
تبارک الله آن دم که پر شود مجلس
&&
ز بوی جام و ز نور رخ چنان دلدار
\\
هزار مست چو پروانه جانب آن شمع
&&
نهاده جان به طبق بر که این بگیر و بیار
\\
ز مطربان خوش آواز و نعره مستان
&&
شراب در رگ خمار گم کند رفتار
\\
ببین به حال جوانان کهف کان خوردند
&&
خراب سیصد و نه سال مست اندر غار
\\
چه باده بود که موسی به ساحران درریخت
&&
که دست و پای بدادند مست و بیخودوار
\\
زنان مصر چه دیدند بر رخ یوسف
&&
که شرحه شرحه بریدند ساعد چو نگار
\\
چه ریخت ساقی تقدیس بر سر جرجیس
&&
که غم نخورد و نترسید ز آتش کفار
\\
هزار بارش کشتند و پیشتر می‌رفت
&&
که مستم و خبرم نیست از یکی و هزار
\\
صحابیان که برهنه به پیش تیغ شدند
&&
خراب و مست بدند از محمد مختار
\\
غلط محمد ساقی نبود جامی بود
&&
پر از شراب و خدا بود ساقی ابرار
\\
کدام شربت نوشید پوره ادهم
&&
که مست وار شد از ملک و مملکت بیزار
\\
چه سکر بود که آواز داد سبحانی
&&
که گفت رمز اناالحق و رفت بر سر دار
\\
به بوی آن می‌شد آب روشن و صافی
&&
چو مست سجده کنان می‌رود به سوی بحار
\\
ز عشق این می خاکست گشته رنگ آمیز
&&
ز تف این می آتش فروخت خوش رخسار
\\
وگر نه باد چرا گشت همدم و غماز
&&
حیات سبزه و بستان و دفتر گفتار
\\
چه ذوق دارند این چار اصل ز آمیزش
&&
نبات و مردم و حیوان نتیجه این چار
\\
چه بی‌هشانه میی دارد این شب زنگی
&&
که خلق را به یکی جام می‌برد از کار
\\
ز لطف و صنعت صانع کدام را گویم
&&
که بحر قدرت او را پدید نیست کنار
\\
شراب عشق بنوشیم و بار عشق کشیم
&&
چنانک اشتر سرمست در میان قطار
\\
نه مستیی که تو را آرزوی عقل آید
&&
ز مستی که کند روح و عقل را بیدار
\\
ز هر چه دارد غیر خدا شکوفه کند
&&
از آنک غیر خدا نیست جز صداع و خمار
\\
کجا شراب طهور و کجا می انگور
&&
طهور آب حیاتست و آن دگر مردار
\\
دمی چو خوک و زمانی چو بوزنه کندت
&&
به آب سرخ سیه روی گردی آخر کار
\\
دلست خنب شراب خدا سرش بگشا
&&
سرش به گل بگرفتست طبع بدکردار
\\
چو اندکی سر خم را ز گل کنی خالی
&&
برآید از سر خم بو و صد هزار آثار
\\
اگر درآیم کثار آن فروشمرم
&&
شمار آن نتوان کرد تا به روز شمار
\\
چو عاجزیم بلا احصیی فرود آریم
&&
چو گشت وقت فروداشت جام جان بردار
\\
درآ به مجلس عشاق شمس تبریزی
&&
که آفتاب از آن شمس می‌برد انوار
\\
\end{longtable}
\end{center}
