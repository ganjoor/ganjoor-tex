\begin{center}
\section*{غزل شماره ۹۰۹: چه پادشاست که از خاک پادشا سازد}
\label{sec:0909}
\addcontentsline{toc}{section}{\nameref{sec:0909}}
\begin{longtable}{l p{0.5cm} r}
چه پادشاست که از خاک پادشا سازد
&&
ز بهر یک دو گدا خویشتن گدا سازد
\\
باقرضواالله کدیه کند چو مسکینان
&&
که تا تو را بدهد ملک و متکا سازد
\\
به مرده برگذرد مرده را حیات دهد
&&
به درد درنگرد درد را دوا سازد
\\
چو باد را فسراند ز باد آب کند
&&
چو آب را بدهد جوش از او هوا سازد
\\
نظر مکن به جهان خوار کاین جهان فانیست
&&
که او به عاقبتش عالم بقا سازد
\\
ز کیمیا عجب آید که زر کند مس را
&&
مسی نگر که به هر لحظه کیمیا سازد
\\
هزار قفل گر هست بر دلت مهراس
&&
دکان عشق طلب کن که دلگشا سازد
\\
کسی که بی‌قلم و آلتی به بتخانه
&&
هزار صورت زیبا برای ما سازد
\\
هزار لیلی و مجنون ز بهر ما برساخت
&&
چه صورتست که بهر خدا خدا سازد
\\
گر آهنست دل تو ز سختی‌اش مگری
&&
که صیقل کرمش آینه صفا سازد
\\
ز دوستان چو ببری به زیر خاک روی
&&
ز مار و مور حریفان خوش لقا سازد
\\
نه مار را مدد و پشت دار موسی ساخت
&&
نه لحظه لحظه ز عین جفا وفا سازد
\\
درون گور تن خود تو این زمان بنگر
&&
که دم به دم چه خیالات دلربا سازد
\\
چو سینه بازشکافی در او نبینی هیچ
&&
که تا زنخ نزند کس که او کجا سازد
\\
مثل شدست که انگور خور ز باغ مپرس
&&
که حق ز سنگ دو صد چشمه رضا سازد
\\
درون سنگ بجویی ز آب اثر نبود
&&
ز غیب سازد نه از پستی و علا سازد
\\
ز بی‌چگونه و چون آمد این چگونه و چون
&&
که صد هزار بلی گو خود از او لا سازد
\\
دو جوی نور نگر از دو پیه پاره روان
&&
عجب مدار عصا را که اژدها سازد
\\
در این دو گوش نگر کهربای نطق کجاست
&&
عجب کسی که ز سوراخ کهربا سازد
\\
سرای را بدهد جان و خواجه ایش کند
&&
چو خواجه را بکشد باز از او سرا سازد
\\
اگر چه صورت خواجه به زیر خاک شدست
&&
ضمیر خواجه وطنگه ز کبریا سازد
\\
به چشم مردم صورت پرست خواجه برفت
&&
ولیک خواجه ز نقش دگر قبا سازد
\\
خموش کن به زبان مدحت و ثنا کم گوی
&&
که تا خدای تو را مدحت و ثنا سازد
\\
\end{longtable}
\end{center}
