\begin{center}
\section*{غزل شماره ۲۱۶۹: ز مکر حق مباش ایمن اگر صد بخت بینی تو}
\label{sec:2169}
\addcontentsline{toc}{section}{\nameref{sec:2169}}
\begin{longtable}{l p{0.5cm} r}
ز مکر حق مباش ایمن اگر صد بخت بینی تو
&&
بمال این چشم‌ها را گر به پندار یقینی تو
\\
که مکر حق چنان تند است کز وی دیده جانت
&&
تو را عرشی نماید او و گر باشی زمینی تو
\\
گمان خاینی می بر تو بر جان امین شکلت
&&
که گر تو ساده دل باشی ندارد سود امینی تو
\\
خریدی هندوی زشتی قبیحی را تو در چادر
&&
تو ساده پوستین بر بوی زهره روی چینی تو
\\
چو شب در خانه آوردی بدیدی روش بی‌چادر
&&
ز رویش دیده بگرفتی ز بویش بستی بینی تو
\\
در این بازار طراران زاهدشکل بسیارند
&&
فریبندت اگر چه اهل و باعقل متینی تو
\\
مگر فضل خداوند خداوندان شمس الدین
&&
کند تنبیه جانت را کند هر دم معینی تو
\\
ببین آن آفتابی را کش اول نیست و نی پایان
&&
که اندر دین همی‌تابد اگر از اهل دینی تو
\\
به سوی باغ وحدت رو کز او شادی همی‌روید
&&
که هر جزوت شود خندان اگر در خود حزینی تو
\\
\end{longtable}
\end{center}
