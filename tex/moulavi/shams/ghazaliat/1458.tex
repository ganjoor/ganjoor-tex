\begin{center}
\section*{غزل شماره ۱۴۵۸: یک لحظه و یک ساعت دست از تو نمی‌دارم}
\label{sec:1458}
\addcontentsline{toc}{section}{\nameref{sec:1458}}
\begin{longtable}{l p{0.5cm} r}
یک لحظه و یک ساعت دست از تو نمی‌دارم
&&
زیرا که تویی کارم زیرا که تویی بارم
\\
از قند تو می نوشم با پند تو می کوشم
&&
من صید جگرخسته تو شیر جگرخوارم
\\
جان من و جان تو گویی که یکی بوده‌ست
&&
سوگند بدین یک جان کز غیر تو بیزارم
\\
از باغ جمال تو یک بند گیاهم من
&&
وز خلعت وصل تو یک پاره کلهوارم
\\
بر گرد تو این عالم خار سر دیوار است
&&
بر بوی گل وصلت خاری است که می خارم
\\
چون خار چنین باشد گلزار تو چون باشد
&&
ای خورده و ای برده اسرار تو اسرارم
\\
خورشید بود مه را بر چرخ حریف ای جان
&&
دانم که بنگذاری در مجلس اغیارم
\\
رفتم بر درویشی گفتا که خدا یارت
&&
گویی به دعای او شد چون تو شهی یارم
\\
دیدم همه عالم را نقش در گرمابه
&&
ای برده تو دستارم هم سوی تو دست آرم
\\
هر جنس سوی جنسش زنجیر همی‌درد
&&
من جنس کیم کاین جا در دام گرفتارم
\\
گرد دل من جانا دزدیده همی‌گردی
&&
دانم که چه می جویی ای دلبر عیارم
\\
در زیر قبا جانا شمعی پنهان داری
&&
خواهی که زنی آتش در خرمن و انبارم
\\
ای گلشن و گلزارم وی صحت بیمارم
&&
ای یوسف دیدارم وی رونق بازارم
\\
تو گرد دلم گردان من گرد درت گردان
&&
در دست تو در گردش سرگشته چو پرگارم
\\
در شادی روی تو گر قصه غم گویم
&&
گر غم بخورد خونم والله که سزاوارم
\\
بر ضرب دف حکمت این خلق همی‌رقصند
&&
بی‌پرده تو رقصد یک پرده نپندارم
\\
آواز دفت پنهان وین رقص جهان پیدا
&&
پنهان بود این خارش هر جای که می خارم
\\
خامش کنم از غیرت زیرا ز نبات تو
&&
ابر شکرافشانم جز قند نمی‌بارم
\\
در آبم و در خاکم در آتش و در بادم
&&
این چار بگرد من اما نه از این چارم
\\
گه ترکم و گه هندو گه رومی و گه زنگی
&&
از نقش تو است ای جان اقرارم و انکارم
\\
تبریز دل و جانم با شمس حق است این جا
&&
هر چند به تن اکنون تصدیع نمی‌آرم
\\
\end{longtable}
\end{center}
