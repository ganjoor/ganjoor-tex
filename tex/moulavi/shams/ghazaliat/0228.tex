\begin{center}
\section*{غزل شماره ۲۲۸: بیار آن که قرین را سوی قرین کشدا}
\label{sec:0228}
\addcontentsline{toc}{section}{\nameref{sec:0228}}
\begin{longtable}{l p{0.5cm} r}
بیار آن که قرین را سوی قرین کشدا
&&
فرشته را ز فلک جانب زمین کشدا
\\
به هر شبی چو محمد به جانب معراج
&&
براق عشق ابد را به زیر زین کشدا
\\
به پیش روح نشین زان که هر نشست تو را
&&
به خلق و خوی و صفت‌های همنشین کشدا
\\
شراب عشق ابد را که ساقیش روح است
&&
نگیرد و نکشد ور کشد چنین کشدا
\\
برو بدزد ز پروانه خوی جانبازی
&&
که آن تو را به سوی نور شمع دین کشدا
\\
رسید وحی خدایی که گوش تیز کنید
&&
که گوش تیز به چشم خدای بین کشدا
\\
خیال دوست تو را مژده وصال دهد
&&
که آن خیال و گمان جانب یقین کشدا
\\
در این چهی تو چو یوسف خیال دوست رسن
&&
رسن تو را به فلک‌های برترین کشدا
\\
به روز وصل اگر عقل ماندت گوید
&&
نگفتمت که چنان کن که آن به این کشدا
\\
بجه بجه ز جهان همچو آهوان از شیر
&&
گرفتمش همه کان است کان به کین کشدا
\\
به راستی برسد جان بر آستان وصال
&&
اگر کژی به حریر و قز کژین کشدا
\\
بکش تو خار جفاها از آن که خارکشی
&&
به سبزه و گل و ریحان و یاسمین کشدا
\\
بنوش لعنت و دشنام دشمنان پی دوست
&&
که آن به لطف و ثناها و آفرین کشدا
\\
دهان ببند و امین باش در سخن داری
&&
که شه کلید خزینه بر امین کشدا
\\
\end{longtable}
\end{center}
