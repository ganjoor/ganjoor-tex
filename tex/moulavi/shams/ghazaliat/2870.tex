\begin{center}
\section*{غزل شماره ۲۸۷۰: ای دریغا در این خانه دمی بگشودی}
\label{sec:2870}
\addcontentsline{toc}{section}{\nameref{sec:2870}}
\begin{longtable}{l p{0.5cm} r}
ای دریغا در این خانه دمی بگشودی
&&
مونس خویش بدیدی دل هر موجودی
\\
چشم یعقوب به دیدار پسر شاد شدی
&&
ساقی وصل شراب صمدی پیمودی
\\
رو نمودی که منم شاهد تو باک مدار
&&
از زیان هیچ میندیش چو دیدی سودی
\\
هیچ کس رشک نبردی که فلان دست ببرد
&&
هر کسی در چمن روح به کام آسودی
\\
نیست روزی که سپاه شبش آرد غارت
&&
نیست دینار و درم یا هوس معدودی
\\
حاجتت نیست که یاد طرب کهنه کنی
&&
کی بود در خضر خلد غم امرودی
\\
صد هزاران گره جمع شده بر دل ما
&&
از نصیب کرمش آب شدی بگشودی
\\
صورت حشو خیالات ره ما بستند
&&
تیغ خورشید رخش خفیه شده در خودی
\\
طالب جمله وی است و لقبش مطلوبی
&&
عابد جمله وی است و لقبش معبودی
\\
خادم و مؤذن این مسجد تن جان شماست
&&
ساجدی گشته نهان در صفت مسجودی
\\
ای ایازت دل و جان شمس حق تبریزی
&&
نیست در هر دو جهان چون تو شه محمودی
\\
\end{longtable}
\end{center}
