\begin{center}
\section*{غزل شماره ۱۱۰۲: عشق را با گفت و با ایما چه کار}
\label{sec:1102}
\addcontentsline{toc}{section}{\nameref{sec:1102}}
\begin{longtable}{l p{0.5cm} r}
عشق را با گفت و با ایما چه کار
&&
روح را با صورت اسما چه کار
\\
عاشقان گوی‌اند در چوگان یار
&&
گوی را با دست و یا با پا چه کار
\\
هر کجا چوگانش راند می‌رود
&&
گوی را با پست و با بالا چه کار
\\
آینه‌ست و مظهر روی بتان
&&
با نکوسیماش و بدسیما چه کار
\\
سوسمار از آب خوردن فارغست
&&
مر ورا با چشمه و سقا چه کار
\\
آن خیالی که ضمیر اوطان اوست
&&
پاش را با مسکن و با جا چه کار
\\
عیسیی که برگذشت او از اثیر
&&
با غم سرماش و یا گرما چه کار
\\
ای رسایل کشته با نادی غیب
&&
رو تو را با گفت و با غوغا چه کار
\\
\end{longtable}
\end{center}
