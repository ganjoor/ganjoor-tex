\begin{center}
\section*{غزل شماره ۱۷۲۱: خوش سوی ما آ دمی ز آنچ که ما هم خوشیم}
\label{sec:1721}
\addcontentsline{toc}{section}{\nameref{sec:1721}}
\begin{longtable}{l p{0.5cm} r}
خوش سوی ما آ دمی ز آنچ که ما هم خوشیم
&&
آب حیات توایم گر چه به شکل آتشیم
\\
تو جو کبوتربچه زاده این لانه‌ای
&&
گر تو نیایی به خود مات از این سو کشیم
\\
حاضر ما شو که ما حاضر آن شاهدیم
&&
مست می اش می شویم باده از او می چشیم
\\
تیزروان همچو سیل گر چه چو که ساکنیم
&&
نعره زنان همچو رعد گر چه چنین خامشیم
\\
جان چو دریا تو راست بر کف خود نه بیا
&&
گر چه که ما همچو چرخ بی‌گنهی می کشیم
\\
زان سوی این پنج حس نوبت ما پنج کن
&&
کان سوی این شش جهت خسرو این هر ششیم
\\
در پی سرنای عشق تیزدم و دلنواز
&&
کز رگ جان همچو چنگ بهر تو در نالشیم
\\
صحت دعوی عشق مسند و بالش مجو
&&
ما نه چو رنجورکان عاشق آن بالشیم
\\
نور فلک شمس دین مفخر تبریز ما
&&
از رخ آن آفتاب چرخ درون مه وشیم
\\
\end{longtable}
\end{center}
