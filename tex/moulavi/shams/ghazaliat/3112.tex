\begin{center}
\section*{غزل شماره ۳۱۱۲: تو خدای خویی تو صفات هویی}
\label{sec:3112}
\addcontentsline{toc}{section}{\nameref{sec:3112}}
\begin{longtable}{l p{0.5cm} r}
تو خدای خویی تو صفات هویی
&&
تو یکی نباشی تو هزارتویی
\\
به یکی عنایت به یکی کفایت
&&
ز غم و جنایت همه را بشویی
\\
همه یاوه گشته همه قبله هشته
&&
چه غمست کآخر همه را بجویی
\\
همه چاره جویان ز تو پای کوبان
&&
همه حمدگویان که خجسته رویی
\\
تو مرا نگویی ز کدام باغی
&&
تو مرا نگویی ز کدام کویی
\\
همه شاه دوزی همه ماه سوزی
&&
همه وای وایی همه‌های و هویی
\\
تو اگر حبیبی چه عجب حبیبی
&&
تو اگر عدویی چه عجب عدویی
\\
ز حیات بشنو که حیات بخشی
&&
ز نبات بشنو که نبات خویی
\\
تو اگر ز مستی دل ما بخستی
&&
دو سبو شکستی نه دو صد سبویی
\\
تو سماع گوشی تو نشاط هوشی
&&
نظر دو چشمی شکر گلویی
\\
نه دلت گشادم که دگر نگویی
&&
نه چو موت کردم که دگر نه مویی
\\
کدوییست سرکه کدوییست باده
&&
ترشی رها کن اگر آن کدویی
\\
تو خموش آخر که رباب گشتی
&&
که به تن چو چوبی که به دل چو مویی
\\
تو چرا بکوشی جهت خموشی
&&
که جهان نماند تو اگر نگویی
\\
\end{longtable}
\end{center}
