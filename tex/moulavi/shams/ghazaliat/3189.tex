\begin{center}
\section*{غزل شماره ۳۱۸۹: یا ساقی اسقنی براح}
\label{sec:3189}
\addcontentsline{toc}{section}{\nameref{sec:3189}}
\begin{longtable}{l p{0.5cm} r}
یا ساقی اسقنی براح
&&
عجل فقد استضا صباحی
\\
یا ساقیتی و نور عینی
&&
یا راحة مهجتی وزینی
\\
چون از رخ او نظر ربودی
&&
هر لحظه که با خودی جهودی
\\
قد جء قلندر مباحی
&&
یا ساقی اقبلی براح
\\
زان روی که جان و جان فزایی
&&
از یک نظری تو دلربایی
\\
سر دست بر آن قرار بودن
&&
با فصل خزان بهار بودن
\\
زان رو که ز هر خسیم خسته
&&
اسرار تو ای مه خجسته
\\
در عشق درآمدی بچستی
&&
وانگاه تو لوح ما بشستی
\\
زین آتش در هزار داغیم
&&
وز داغ چو صد هزار باغیم
\\
گویند که: « در جفاست، اسرار »
&&
باور کردم ز عشق آن یار
\\
ای دل تو به عشق چند جوشی؟!
&&
تا کی تو ز عاشقی خروشی؟!
\\
ای نقش خیال شهرهٔاری
&&
از دیدهٔ ما مرو تو، باری
\\
ای باغ بمانده از بهاری
&&
گل رفت و بمانده سبزه‌زاری
\\
من بند تو یار می‌گزینم
&&
لیک از تبریز شمس دینم
\\
\end{longtable}
\end{center}
