\begin{center}
\section*{غزل شماره ۴۴۲: بر عاشقان فریضه بود جست و جوی دوست}
\label{sec:0442}
\addcontentsline{toc}{section}{\nameref{sec:0442}}
\begin{longtable}{l p{0.5cm} r}
بر عاشقان فریضه بود جست و جوی دوست
&&
بر روی و سر چو سیل دوان تا بجوی دوست
\\
خود اوست جمله طالب و ما همچو سایه‌ها
&&
ای گفت و گوی ما همگی گفت و گوی دوست
\\
گاهی به جوی دوست چو آب روان خوشیم
&&
گاهی چو آب حبس شدم در سبوی دوست
\\
گه چون حویج دیگ بجوشیم و او به فکر
&&
کفگیر می‌زند که چنینست خوی دوست
\\
بر گوش ما نهاده دهان او به دمدمه
&&
تا جان ما بگیرد یک باره بوی دوست
\\
چون جان جان وی آمد از وی گزیر نیست
&&
من در جهان ندیدم یک جان عدوی دوست
\\
بگدازدت ز ناز و چو مویت کند ضعیف
&&
ندهی به هر دو عالم یکتای موی دوست
\\
با دوست ما نشسته که ای دوست دوست کو
&&
کو کو همی‌زنیم ز مستی به کوی دوست
\\
تصویرهای ناخوش و اندیشه رکیک
&&
از طبع سست باشد و این نیست سوی دوست
\\
خاموش باش تا صفت خویش خود کند
&&
کو های های سرد تو کو های هوی دوست
\\
\end{longtable}
\end{center}
