\begin{center}
\section*{غزل شماره ۲۹۶۶: هر چند بی‌گه آیی بی‌گاه خیز مایی}
\label{sec:2966}
\addcontentsline{toc}{section}{\nameref{sec:2966}}
\begin{longtable}{l p{0.5cm} r}
هر چند بی‌گه آیی بی‌گاه خیز مایی
&&
ای خواجه خانه بازآ بی‌گاه شد کجایی
\\
برگ قفس نداری جز ما هوس نداری
&&
یکتا چو کس نداری برخیز از دوتایی
\\
جان را به عشق واده دل بر وفای ما نه
&&
در ما روی تو را به کز خویشتن برآیی
\\
بگذر ز خشک و از تر بازآ به خانه زوتر
&&
از جمله باوفاتر آخر چه بی‌وفایی
\\
لطفت به کس نماند قدر تو کس نداند
&&
عشقت به ما کشاند زیرا به ما تو شایی
\\
گر چشم رفت خوابش از عاشقی و تابش
&&
بر ما بود جوابش ای جان مرتضایی
\\
گر شاه شمس تبریز پنهان شود به استیز
&&
در عشق او تو جان بیز تا جان شوی بقایی
\\
\end{longtable}
\end{center}
