\begin{center}
\section*{غزل شماره ۱۴۲۷: من از اقلیم بالایم سر عالم نمی‌دارم}
\label{sec:1427}
\addcontentsline{toc}{section}{\nameref{sec:1427}}
\begin{longtable}{l p{0.5cm} r}
من از اقلیم بالایم سر عالم نمی‌دارم
&&
نه از آبم نه از خاکم سر عالم نمی‌دارم
\\
اگر بالاست پراختر وگر دریاست پرگوهر
&&
وگر صحراست پرعبهر سر آن هم نمی‌دارم
\\
مرا گویی ظریفی کن دمی با ما حریفی کن
&&
مرا گفته‌ست لاتسکن تو را همدم نمی‌دارم
\\
مرا چون دایه فضلش به شیر لطف پرورده‌ست
&&
چو من مخمور آن شیرم سر زمزم نمی‌دارم
\\
در آن شربت که جان سازد دل مشتاق جان بازد
&&
خرد خواهد که دریازد منش محرم نمی‌دارم
\\
ز شادی‌ها چو بیزارم سر غم از کجا دارم
&&
به غیر یار دلدارم خوش و خرم نمی‌دارم
\\
پی آن خمر چون عندم شکم بر روزه می بندم
&&
که من آن سرو آزادم که برگ غم نمی‌دارم
\\
درافتادم در آب جو شدم شسته ز رنگ و بو
&&
ز عشق ذوق زخم او سر مرهم نمی‌دارم
\\
تو روز و شب دو مرکب دان یکی اشهب یکی ادهم
&&
بر اشهب بر نمی‌شینم سر ادهم نمی‌دارم
\\
جز این منهاج روز و شب بود عشاق را مذهب
&&
که بر مسلک به زیر این کهن طارم نمی‌دارم
\\
به باغ عشق مرغانند سوی بی‌سویی پران
&&
من ایشان را سلیمانم ولی خاتم نمی‌دارم
\\
منم عیسی خوش خنده که شد عالم به من زنده
&&
ولی نسبت ز حق دارم من از مریم نمی‌دارم
\\
ز عشق این حرف بشنیدم خموشی راه خود دیدم
&&
بگو عشقا که من با دوست لا و لم نمی‌دارم
\\
\end{longtable}
\end{center}
