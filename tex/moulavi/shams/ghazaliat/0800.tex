\begin{center}
\section*{غزل شماره ۸۰۰: گر نخسبی ز تواضع شبکی جان چه شود}
\label{sec:0800}
\addcontentsline{toc}{section}{\nameref{sec:0800}}
\begin{longtable}{l p{0.5cm} r}
گر نخسبی ز تواضع شبکی جان چه شود
&&
ور نکوبی به درشتی در هجران چه شود
\\
ور به یاری و کریمی شبکی روز آری
&&
از برای دل پرآتش یاران چه شود
\\
ور دو دیده به تماشای تو روشن گردد
&&
کوری دیده ناشسته شیطان چه شود
\\
ور بگیرد ز بهاران و ز نوروز رخت
&&
همه عالم گل و اشکوفه و ریحان چه شود
\\
آب حیوان که نهفته‌ست و در آن تاریکیست
&&
پر شود شهر و کهستان و بیابان چه شود
\\
ور بپوشند و بیابند یکی خلعت نو
&&
این غلامان و ضعیفان ز تو سلطان چه شود
\\
ور سواره تو برانی سوی میدان آیی
&&
تا شود گوشه هر سینه چو میدان چه شود
\\
دل ما هست پریشان تن تیره شده جمع
&&
صاف اگر جمع شود تیره پریشان چه شود
\\
به ترازو کم از آنیم که مه با ما نیست
&&
بهر ما گر برود ماه به میزان چه شود
\\
چون عزیر و خر او را به دمی جان بخشید
&&
گر خر نفس شود لایق جولان چه شود
\\
بر سر کوی غمت جان مرا صومعه ایست
&&
گر نباشد قدمش بر که لبنان چه شود
\\
هین خمش باش و بیندیش از آن جان غیور
&&
جمع شو گر نبود حرف پریشان چه شود
\\
\end{longtable}
\end{center}
