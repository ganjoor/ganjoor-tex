\begin{center}
\section*{غزل شماره ۶۷۹: چو دیوم عاشق آن یک پری شد}
\label{sec:0679}
\addcontentsline{toc}{section}{\nameref{sec:0679}}
\begin{longtable}{l p{0.5cm} r}
چو دیوم عاشق آن یک پری شد
&&
ز دیو خویشتن یک سر بری شد
\\
چو ناگاهان بدیدش همچو برقی
&&
برون پرید عقلش را سری شد
\\
در انگشت پری مهر سلیمان
&&
چو دید آن جان و دل در چاکری شد
\\
چو سر چاکری عشق دریافت
&&
فراز هفت چرخ مهتری شد
\\
چو لب تر کرد او از جام عشقش
&&
بدان خشکی لب او از تری شد
\\
چو شد او مشتری عشق جنی
&&
کمینه بندگانش مشتری شد
\\
چو گاوی بود بی‌جان و زبان دیو
&&
بداد جان و عشقش سامری شد
\\
همه جور و جفا و محنت عشق
&&
بر او شیرین چو مهر مادری شد
\\
مگر درد فراق و جور هجران
&&
که تاب آن نبودش زان بری شد
\\
ز دست هجر او تا پیش مخدوم
&&
که شمس الدینست بهر داوری شد
\\
چو دیو آمد به پیشش خاک بوسید
&&
از آتش با ملایک همپری شد
\\
از آن مستی به تبریز است گردان
&&
که از جانش هوای کافری شد
\\
\end{longtable}
\end{center}
