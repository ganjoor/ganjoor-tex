\begin{center}
\section*{غزل شماره ۱۲۶: گستاخ مکن تو ناکسان را}
\label{sec:0126}
\addcontentsline{toc}{section}{\nameref{sec:0126}}
\begin{longtable}{l p{0.5cm} r}
گستاخ مکن تو ناکسان را
&&
در چشم میار این خسان را
\\
درزی دزدی چو یافت فرصت
&&
کم آرد جامه رسان را
\\
ایشان را دار حلقه بر در
&&
هم نیز نیند لایق آن را
\\
پیشت به فسون و سخره آیند
&&
از طمع مپوش این عیان را
\\
ایشان چو ز خویش پرغمانند
&&
چون دور کنند ز تو غمان را
\\
جز خلوت عشق نیست درمان
&&
رنج باریک اندهان را
\\
یا دیدن دوست یا هوایش
&&
دیگر چه کند کسی جهان را
\\
تا دیدن دوست در خیالش
&&
می‌دار تو در سجود جان را
\\
پیشش چو چراغپایه می‌ایست
&&
چون فرصت‌هاست مر مهان را
\\
وامانده از این زمانه باشی
&&
کی بینی اصل این زمان را
\\
چون گشت گذار از مکان چشم
&&
زو بیند جان آن مکان را
\\
جان خوردی تن چو قازغانی
&&
بر آتش نه تو قازغان را
\\
تا جوش ببینی ز اندرونت
&&
زان پس نخری تو داستان را
\\
نظاره نقد حال خویشی
&&
نظاره درونست راستان را
\\
این حال بدایت طریقست
&&
با گم شدگان دهم نشان را
\\
چون صد منزل از این گذشتند
&&
این چون گویم مران کسان را
\\
مقصود از این بگو و رستی
&&
یعنی که چراغ آسمان را
\\
مخدومم شمس حق و دین را
&&
کوهست پناه انس و جان را
\\
تبریز از او چو آسمان شد
&&
دل گم مکناد نردبان را
\\
\end{longtable}
\end{center}
