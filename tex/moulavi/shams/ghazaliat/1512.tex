\begin{center}
\section*{غزل شماره ۱۵۱۲: گهی در گیرم و گه بام گیرم}
\label{sec:1512}
\addcontentsline{toc}{section}{\nameref{sec:1512}}
\begin{longtable}{l p{0.5cm} r}
گهی در گیرم و گه بام گیرم
&&
چو بینم روی تو آرام گیرم
\\
زبون خاص و عامم در فراقت
&&
بیا تا ترک خاص و عام گیرم
\\
دلم از غم گریبان می دراند
&&
که کی دامان آن خوش نام گیرم
\\
نگیرم عیش و عشرت تا نیاید
&&
وگر گیرم در آن هنگام گیرم
\\
چو زلف انداز من ساقی درآید
&&
به دستی زلف و دستی جام گیرم
\\
اگر در خرقه زاهد درآید
&&
شوم حاجی و راه شام گیرم
\\
وگر خواهد که من دیوانه باشم
&&
شوم خام و حریف خام گیرم
\\
وگر چون مرغ اندر دل بپرد
&&
شوم صیاد مرغان دام گیرم
\\
چو گویم شب نخسپم او بگوید
&&
که من خواب از نماز شام گیرم
\\
وگر گویم عنایت کن بگوید
&&
که نی من جنگیم دشنام گیرم
\\
مراد خویش بگذارم همان دم
&&
مراد دلبر خودکام گیرم
\\
\end{longtable}
\end{center}
