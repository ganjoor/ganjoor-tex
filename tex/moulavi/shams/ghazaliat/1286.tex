\begin{center}
\section*{غزل شماره ۱۲۸۶: شنو ز سینه ترنگاترنگ آوازش}
\label{sec:1286}
\addcontentsline{toc}{section}{\nameref{sec:1286}}
\begin{longtable}{l p{0.5cm} r}
شنو ز سینه ترنگاترنگ آوازش
&&
دل خراب طپیدن گرفت از آغازش
\\
به بر گرفت رباب و ز سر نهاد کله
&&
ز دست رفت دل من چو دید سر بازش
\\
دل از بریشم او چون کلابه گردانست
&&
کلابه ظاهر و پنهان ز چشم قزازش
\\
دو سه بریشم از این ارغنون فروتر گیرد
&&
که تند می‌رسد آواز عقل پردازش
\\
بدانک تن چو غبارست و جان در او چون باد
&&
ولیک فعل غبار تنست غمازش
\\
غبار جان بود و می‌رسد دگر جانی
&&
که ذره ذره به رقص آمدست از آوازش
\\
جهان تنور و در آن نان‌های رنگارنگ
&&
تنور و نان چه کند آنک دید خبازش
\\
ز سینه نیست سماع دل و ز بیرون نیست
&&
فدات جانم هر جا که هست بنوازش
\\
شبی به طنز بگفتم دلا به مه بنگر
&&
که هست مه را چیزی ز لطف پروازش
\\
چو آفتاب نهان شد به جای او بنهند
&&
چراغکی که بود شب شراراندازش
\\
به هر دو دست دل از ماه چشم خود بگرفت
&&
که دل ز غیرت شه واقفست و از نازش
\\
\end{longtable}
\end{center}
