\begin{center}
\section*{غزل شماره ۱۸۳۵: گرم درآ و دم مده ساقی بردبار من}
\label{sec:1835}
\addcontentsline{toc}{section}{\nameref{sec:1835}}
\begin{longtable}{l p{0.5cm} r}
گرم درآ و دم مده ساقی بردبار من
&&
ای دم تو ندیم من ای رخ تو بهار من
\\
هین که خروس بانگ زد بوی صبوح می دهد
&&
بر کف همچو بحر نه بلبله عقار من
\\
گریه به باده خنده کن مرده به باده زنده کن
&&
چونک چنین کنی بتا بس به نواست کار من
\\
بند من است مشتبه باز گشا گره گره
&&
تا که برهنه‌تر شود خفیه و آشکار من
\\
ترک حیا و شرم کن پشت مراد گرم کن
&&
پشت من و پناه من خویش من و تبار من
\\
نیست قبول مست تو باده ز غیر دست تو
&&
آن رخ من چو گل کند وان شکند خمار من
\\
داد هزار جان بده باده آسمان بده
&&
تا که پرد همای جان مست سوی مطار من
\\
جان برهد ز کنده‌ها زین همه تخته بندها
&&
مقعد صدق بررود صادق حق گزار من
\\
باده ده و نهان بده از ره عقل و جان بده
&&
تا نرسد به هر کسی عشرت و کار و بار من
\\
چشم عوام بسته به روح ز شهر رسته به
&&
فتنه و شر نشسته به ای شه باوقار من
\\
باده همی‌زند لمع جان هزار با طمع
&&
مست و پیاده می تپد گرد می سوار من
\\
دست بدار از این قدح گیر عوض از آن فرج
&&
تا بزند بر اندهت تابش ابتشار من
\\
هیچ نیرزد این میش نی غلیان و نی قیش
&&
این بفروش و باده بین باده بی‌کنار من
\\
دست نلرزدت از این بی‌خرد خوش رزین
&&
جام گزین و می ببین از کف شهریار من
\\
پر ز حیات جام او مشک و عبر ختام او
&&
دیو و پری غلام او چستی و انتشار من
\\
برجه ساقیا تو گو چون تو صفت کننده کو
&&
ای که ز لطف نسج او سخت درید تار من
\\
\end{longtable}
\end{center}
