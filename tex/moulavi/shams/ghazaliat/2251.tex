\begin{center}
\section*{غزل شماره ۲۲۵۱: چو از سر بگیرم بود سرور او}
\label{sec:2251}
\addcontentsline{toc}{section}{\nameref{sec:2251}}
\begin{longtable}{l p{0.5cm} r}
چو از سر بگیرم بود سرور او
&&
چو من دل بجویم بود دلبر او
\\
چو من صلح جویم شفیع او بود
&&
چو در جنگ آیم بود خنجر او
\\
چو در مجلس آیم شراب است و نقل
&&
چو در گلشن آیم بود عبهر او
\\
چو در کان روم او عقیق است و لعل
&&
چو در بحر آیم بود گوهر او
\\
چو در دشت آیم بود روضه او
&&
چو وا چرخ آیم بود اختر او
\\
چو در صبر آیم بود صدر او
&&
چو از غم بسوزم بود مجمر او
\\
چو در رزم آیم به وقت قتال
&&
بود صف نگهدار و سرلشکر او
\\
چو در بزم آیم به وقت نشاط
&&
بود ساقی و مطرب و ساغر او
\\
چو نامه نویسم سوی دوستان
&&
بود کاغذ و خامه و محبر او
\\
چون بیدار گردم بود هوش نو
&&
چو بخوابم بیاید به خواب اندر او
\\
چو جویم برای غزل قافیه
&&
به خاطر بود قافیه گستر او
\\
تو هر صورتی که مصور کنی
&&
چو نقاش و خامه بود بر سر او
\\
تو چندانک برتر نظر می‌کنی
&&
از آن برتر تو بود برتر او
\\
برو ترک گفتار و دفتر بگو
&&
که آن به که باشد تو را دفتر او
\\
خمش کن که هر شش جهت نور او است
&&
وزین شش جهت بگذری داور او
\\
رضاک رضای الذی اوثر
&&
و سرک سری فما اظهر
\\
زهی شمس تبریز خورشیدوش
&&
که خود را بود سخت اندرخور او
\\
\end{longtable}
\end{center}
