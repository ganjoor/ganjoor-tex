\begin{center}
\section*{غزل شماره ۱۲۳۰: جانم به چه آرامد ای یار به آمیزش}
\label{sec:1230}
\addcontentsline{toc}{section}{\nameref{sec:1230}}
\begin{longtable}{l p{0.5cm} r}
جانم به چه آرامد ای یار به آمیزش
&&
صحت به چه دریابد بیمار به آمیزش
\\
هر چند به بر گیری او را نبود سیری
&&
دانی به چه بنشیند این بار به آمیزش
\\
آن تشنه ده روزه کی به شود از کوزه
&&
الا که کند آبش خوش خوار به آمیزش
\\
در وصل تو می‌جوید وز شرم نمی‌گوید
&&
کامسال طرب خواهد چون پار به آمیزش
\\
کاری که کند بنده تقدیر زند خنده
&&
کای خفته بجو آخر این کار به آمیزش
\\
زیرا که به آمیزش یک خشت شود قصری
&&
زیرا که شود جامه یک تار به آمیزش
\\
اندر چمن عشقت شمس الحق تبریزی
&&
صد گلشن و گل گردد یک خار به آمیزش
\\
\end{longtable}
\end{center}
