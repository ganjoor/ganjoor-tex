\begin{center}
\section*{غزل شماره ۲۷۰۰: ز هر چیزی ملول است آن فضولی}
\label{sec:2700}
\addcontentsline{toc}{section}{\nameref{sec:2700}}
\begin{longtable}{l p{0.5cm} r}
ز هر چیزی ملول است آن فضولی
&&
ملولش کن خدایا از ملولی
\\
به قاصد تا بیاشوبد بجنگد
&&
بدو گفتم ملولی هست گولی
\\
بخورد آن بازی من خشمگین شد
&&
مرا گفتا خمش دیوانه لولی
\\
نگوید هیچ را بد مرد این راه
&&
مبین بد هیچ را ور نی تو غولی
\\
بگفتم عین انکار تو بر من
&&
نه بد دیدن بود یا بی‌حصولی
\\
مرا گفت او تناقض‌های بینا
&&
بود از مصلحت نه از بی‌اصولی
\\
محالی گر بگوید مرد کامل
&&
تو عین حال دانش ای حلولی
\\
گهی درد که داند گه بدوزد
&&
گهی شاهی کند گاهی رسولی
\\
به تأویلات تو او درنگنجد
&&
که تو هستی فصولی او اصولی
\\
ز خود منگر در او از خود برون آ
&&
که بر بی‌حد ندارد حد شمولی
\\
خمش ای نفس تازی هم بگویم
&&
دوباره لا تقولی لا تقولی
\\
\end{longtable}
\end{center}
