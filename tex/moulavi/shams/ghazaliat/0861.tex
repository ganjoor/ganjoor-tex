\begin{center}
\section*{غزل شماره ۸۶۱: لطفی نماند کان صنم خوش لقا نکرد}
\label{sec:0861}
\addcontentsline{toc}{section}{\nameref{sec:0861}}
\begin{longtable}{l p{0.5cm} r}
لطفی نماند کان صنم خوش لقا نکرد
&&
ما را چه جرم اگر کرمش با شما نکرد
\\
تشنیع می‌زنی که جفا کرد آن نگار
&&
خوبی که دید در دو جهان کو جفا نکرد
\\
عشقش شکر بس است اگر او شکر نداد
&&
حسنش همه وفاست اگر او وفا نکرد
\\
بنمای خانه‌ای که از او نیست پرچراغ
&&
بنمای صفه‌ای که رخش پرصفا نکرد
\\
این چشم و آن چراغ دو نورند هر یکی
&&
چون آن به هم رسید کسیشان جدا نکرد
\\
چون روح در نظاره فنا گشت این بگفت
&&
نظاره جمال خدا جز خدا نکرد
\\
هر یک از این مثال بیانست و مغلطه است
&&
حق جز ز رشک نام رخش والضحی نکرد
\\
خورشیدروی مفخر تبریز شمس دین
&&
بر فانیی نتافت که آن را بقا نکرد
\\
\end{longtable}
\end{center}
