\begin{center}
\section*{غزل شماره ۱۹۸۸: چه شکر داد عجب یوسف خوبی به لبان}
\label{sec:1988}
\addcontentsline{toc}{section}{\nameref{sec:1988}}
\begin{longtable}{l p{0.5cm} r}
چه شکر داد عجب یوسف خوبی به لبان
&&
که شد ادریسش قیماز و سلیمان به لبان
\\
به شکرخانه او رفته به سر لب شکران
&&
مانده اندر عجبش خیره همه بوالعجبان
\\
خبر افتاد که گرگی طمع یوسف کرد
&&
همه گرگان شده از خجلت این گرگ شبان
\\
چه خوشی‌های نهان است در آن درد و غمش
&&
که رمیدند ز دارو همه درمان طلبان
\\
بس بود هستی او مایه هر نیست شده
&&
بس بود مستی او عذر همه بی‌ادبان
\\
عارف از ورزش اسباب بدان کاهل شد
&&
که همان بی‌سببی شد سبب بی‌سببان
\\
خیز کامروز ز اقبال و سعادت باری
&&
طرب اندر طرب است از مدد بوطربان
\\
من بر آن بودم کز جان و دل تفسیده
&&
بازگویی صفت عشق به روزان و شبان
\\
شمس تبریزی مرا دوش همی‌گفت خموش
&&
چون تو را عشق لب ماست نگهدار زبان
\\
\end{longtable}
\end{center}
