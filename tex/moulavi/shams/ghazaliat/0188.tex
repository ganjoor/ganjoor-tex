\begin{center}
\section*{غزل شماره ۱۸۸: بر چشمه ضمیرت کرد آن پری وثاقی}
\label{sec:0188}
\addcontentsline{toc}{section}{\nameref{sec:0188}}
\begin{longtable}{l p{0.5cm} r}
این جا کسیست پنهان خود را مگیر تنها
&&
بس تیز گوش دارد مگشا به بد زبان را
\\
بر چشمه ضمیرت کرد آن پری وثاقی
&&
هر صورت خیالت از وی شدست پیدا
\\
هر جا که چشمه باشد باشد مقام پریان
&&
بااحتیاط باید بودن تو را در آن جا
\\
این پنج چشمه حس تا بر تنت روانست
&&
ز اشراق آن پری دان گه بسته گاه مجری
\\
وان پنج حس باطن چون وهم و چون تصور
&&
هم پنج چشمه می‌دان پویان به سوی مرعی
\\
هر چشمه را دو مشرف پنجاه میرابند
&&
صورت به تو نمایند اندر زمان اجلا
\\
زخمت رسد ز پریان گر باادب نباشی
&&
کاین گونه شهره پریان تندند و بی‌محابا
\\
تقدیر می‌فریبد تدبیر را که برجه
&&
مکرش گلیم برده از صد هزار چون ما
\\
مرغان در قفس بین در شست ماهیان بین
&&
دل‌های نوحه گر بین زان مکرساز دانا
\\
دزدیده چشم مگشا بر هر بت از خیانت
&&
تا نفکند ز چشمت آن شهریار بینا
\\
ماندست چند بیتی این چشمه گشت غایر
&&
برجوشد آن ز چشمه خون برجهیم فردا
\\
\end{longtable}
\end{center}
