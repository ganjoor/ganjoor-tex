\begin{center}
\section*{غزل شماره ۲۲۹۲: به لاله دوش نسرین گفت برخیزیم مستانه}
\label{sec:2292}
\addcontentsline{toc}{section}{\nameref{sec:2292}}
\begin{longtable}{l p{0.5cm} r}
به لاله دوش نسرین گفت برخیزیم مستانه
&&
به دامان گل تازه درآویزیم مستانه
\\
چو باده بر سر باده خوریم از گلرخ ساده
&&
بیا تا چون گل و لاله درآمیزیم مستانه
\\
چو نرگس شوخ چشم آمد سمن را رشک و خشم آمد
&&
به نسرین گفت تا ما هم براستیزیم مستانه
\\
بت گلروی چون شکر چو غنچه بسته بود آن در
&&
چو در بگشاد وقت آمد که درریزیم مستانه
\\
که جان‌ها کز الست آمد بسی بی‌خویش و مست آمد
&&
از آن در آب و گل هر دم همی‌لغزیم مستانه
\\
دلا تو اندر این شادی ز سرو آموز آزادی
&&
که تا از جرم و از توبه بپرهیزیم مستانه
\\
صلاح دیده ره بین صلاح الدین صلاح الدین
&&
برای او ز خود شاید که بگریزیم مستانه
\\
\end{longtable}
\end{center}
