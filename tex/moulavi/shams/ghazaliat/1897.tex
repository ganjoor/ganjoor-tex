\begin{center}
\section*{غزل شماره ۱۸۹۷: در این دم همدمی آمد خمش کن}
\label{sec:1897}
\addcontentsline{toc}{section}{\nameref{sec:1897}}
\begin{longtable}{l p{0.5cm} r}
در این دم همدمی آمد خمش کن
&&
که او ناگفته می داند خمش کن
\\
ز جام باده خاموش گویا
&&
تو را بی‌خویش بنشاند خمش کن
\\
مزن تشنیع بر سلطان عشقش
&&
که او کس را نرنجاند خمش کن
\\
اگر در آینه دم را بگیری
&&
تو را از گفت برهاند خمش کن
\\
ز گردش‌های تو می داند آن کس
&&
که گردون را بگرداند خمش کن
\\
هر اندیشه که در دل دفن کردی
&&
یکایک بر تو برخواند خمش کن
\\
ز هر اندیشه مرغی آفریند
&&
در آن عالم بپراند خمش کن
\\
یکی جغد و یکی باز و یکی زاغ
&&
که یک یک را نمی‌ماند خمش کن
\\
گر آن مه را نمی‌بینی ببینی
&&
چو چشمت را بپیچاند خمش کن
\\
از این عالم و زان عالم مگو زانک
&&
به یک رنگیت می راند خمش کن
\\
\end{longtable}
\end{center}
