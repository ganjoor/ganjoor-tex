\begin{center}
\section*{غزل شماره ۲۴۰۶: ایا دلی چو صبا ذوق صبح‌ها دیده}
\label{sec:2406}
\addcontentsline{toc}{section}{\nameref{sec:2406}}
\begin{longtable}{l p{0.5cm} r}
ایا دلی چو صبا ذوق صبح‌ها دیده
&&
ز دیده مست شدی یا ز ذوق نادیده
\\
گهی به بحر تحیر گهی به دامن کوه
&&
کمر ببسته و در کوه کهربا دیده
\\
ورای دیده و دل صد دریچه بگشاده
&&
برون ز چرخ و زمین رفته صد سما دیده
\\
چو جوششی و بخاری فتاد در دریا
&&
ز لذت نظرش رست در قفا دیده
\\
چو موج موج درآمیخت چشم با دریا
&&
عجب عجب که همه بحر گشت یا دیده
\\
به پیش دیده دو عالم چو دانه پیش خروس
&&
چنین بود نظر پاک کبریادیده
\\
نه طالب است و نه مطلوب آن که در توحید
&&
صفات طالب و مطلوب را جدا دیده
\\
اله را کی شناسد کسی که رست ز لا
&&
ز لا کی رست بگو عاشق بلادیده
\\
رموز لیس و فی جبتی بدانسته
&&
هزار بار من این جبه را قبا دیده
\\
دهان گشاد ضمیر و صلاح دین را گفت
&&
تویی حیات من ای دیده خدادیده
\\
\end{longtable}
\end{center}
