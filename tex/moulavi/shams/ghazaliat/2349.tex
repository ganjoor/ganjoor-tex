\begin{center}
\section*{غزل شماره ۲۳۴۹: ای دیده راست راست دیده}
\label{sec:2349}
\addcontentsline{toc}{section}{\nameref{sec:2349}}
\begin{longtable}{l p{0.5cm} r}
ای دیده راست راست دیده
&&
چون دیده تو کجاست دیده
\\
آن قطره بی‌وفا چه دیده‌ست
&&
بحر گهر وفاست دیده
\\
اجری خور توتیا چه بیند
&&
اجری ده توتیاست دیده
\\
ای آنک ز روز و شب برونی
&&
روز و شب مر تو راست دیده
\\
در پرتو آفتاب رویت
&&
در رقص چو ذره‌هاست دیده
\\
بد بی‌تو دو دیده دشمن جان
&&
اکنون ز تو جان ماست دیده
\\
ای دیده تان چو دل پریشان
&&
در عین دل شماست دیده
\\
هر دیده جدا جدا از آن است
&&
کز دیده ما جداست دیده
\\
چون دیده خدای را ببیند
&&
گویی که مگر خداست دیده
\\
چون دیده کوه بر حق افتاد
&&
از هر سنگیش خاست دیده
\\
زر شد همه کوه از تجلی
&&
یعنی همه کیمیاست دیده
\\
\end{longtable}
\end{center}
