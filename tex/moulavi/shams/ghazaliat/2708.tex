\begin{center}
\section*{غزل شماره ۲۷۰۸: تو هر روزی از آن پشته برآیی}
\label{sec:2708}
\addcontentsline{toc}{section}{\nameref{sec:2708}}
\begin{longtable}{l p{0.5cm} r}
تو هر روزی از آن پشته برآیی
&&
کنی مر تشنه جانان را سقایی
\\
تو هر صبحی جهان را نور بخشی
&&
که جان جان خورشید سمایی
\\
مباد آن روز کز تو بازماند
&&
دو دیده‌ای چراغ و روشنایی
\\
تو دریایی و می‌گویی جهان را
&&
درآ در من بیاموز آشنایی
\\
لب و لنج کفوری را دریدی
&&
بدان دریای امواج عطایی
\\
گشادی چشم و گوش خاکیان را
&&
همه حیران که چون بر می‌گشایی
\\
گلوی جان بسوزید از حلاوت
&&
چنین شیرین چنین حلوا چرایی
\\
اگر چون آسیا گردم شب و روز
&&
ز تو باشد که آب آسیایی
\\
وگر این آسیا جوید سکونت
&&
ز چرخ تو نمی‌یابد رهایی
\\
هر آن سنگی که در چرخش کشیدی
&&
بیابد کان بیابد کیمیایی
\\
به تو جنبد جهان جان جهانی
&&
اگر چه او نداند که کجایی
\\
\end{longtable}
\end{center}
