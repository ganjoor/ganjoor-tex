\begin{center}
\section*{غزل شماره ۲۸۰۸: گر چه در مستی خسی را تو مراعاتی کنی}
\label{sec:2808}
\addcontentsline{toc}{section}{\nameref{sec:2808}}
\begin{longtable}{l p{0.5cm} r}
گر چه در مستی خسی را تو مراعاتی کنی
&&
و آنک نفی محض باشد گر چه اثباتی کنی
\\
آنک او رد دل است از بددرونی‌های خویش
&&
گر نفاقی پیشش آری یا که طاماتی کنی
\\
ور تو خود را از بد او کور و کر سازی دمی
&&
مدح سر زشت او یا ترک زلاتی کنی
\\
آن تکلف چند باشد آخر آن زشتی او
&&
بر سر آید تا تو بگریزی و هیهاتی کنی
\\
او به صحبت‌ها نشاید دور دارش ای حکیم
&&
جز که در رنجش قضاگو دفع حاجاتی کنی
\\
مر مناجات تو را با او نباشد همدم او
&&
جز برای حاجتش با حق مناجاتی کنی
\\
آن مراعات تو او را در غلط‌ها افکند
&&
پس ملازم گردد او وز غصه ویلاتی کنی
\\
آن طرب بگذشت او در پیش چون قولنج ماند
&&
تا گریزی از وثاق و یا که حیلاتی کنی
\\
آن کسی را باش کو در گاه رنج و خرمی
&&
هست همچون جنت و چون حور کش هاتی کنی
\\
از هواخواهان آن مخدوم شمس الدین بود
&&
شاید او را گر پرستی یا که چون لاتی کنی
\\
ور نه بگریز از دگر کس تا به تبریز صفا
&&
تا شوی مست از جمال و ذوق و حالاتی کنی
\\
\end{longtable}
\end{center}
