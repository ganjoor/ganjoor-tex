\begin{center}
\section*{غزل شماره ۳۰۳۰: سلمک الله نیست مثل تو یاری}
\label{sec:3030}
\addcontentsline{toc}{section}{\nameref{sec:3030}}
\begin{longtable}{l p{0.5cm} r}
سلمک الله نیست مثل تو یاری
&&
نیست نکوتر ز بندگی تو کاری
\\
ای دل گفتی که یار غار منست او
&&
هیچ نگنجد چنین محیط به غاری
\\
عاشق او خرد نیست زانک نخسبد
&&
بر سر آن گنج غیب هر نره ماری
\\
ذره به ذره کنار شوق گشادست
&&
گر چه نگنجد نگار ما به کناری
\\
آن شکرستان رسید تا نگذارد
&&
سرکه فروشنده‌ای و غوره فشاری
\\
جوی فراتی روان شدست از این سو
&&
کاین همه جان‌ها ز آب اوست بخاری
\\
از سر مستی پریر گفتم او را
&&
کار مرا این زمان بده تو قراری
\\
خنده شیرین زد و ز شرم برافروخت
&&
ماه غریب از چو من غریب شماری
\\
گفت مخور غم که زرد و خشک نماند
&&
باغ تو با این چنین لطیف بهاری
\\
هفت فلک ز آتش منست چو دودی
&&
هفت زمین در ره منست غباری
\\
دام جهان را هزار قرن گذشتست
&&
درخور صیدم نیامدست شکاری
\\
هم به کنار آمد این زمانه و دورش
&&
عاشق مستی ز ما نیافت کناری
\\
این مه و خورشید چون دو گاو خراسند
&&
روز چرایی و شب اسیر شیاری
\\
جمع خرانی نگر که گاوپرستند
&&
یاوه شدستند بی‌شکال و فساری
\\
رو به خران گو که ریش گاو بریزاد
&&
توبه کنید و روید سوی مطاری
\\
تا که شود هر خری ندیم مسیحی
&&
وحی پذیرنده‌ای و روح سپاری
\\
از شش و از پنج بگذرید و ببینید
&&
شهره حریفان و مقبلانه قماری
\\
چون به خلاصه رسید تا که بگویم
&&
سوخت لبم را ز شوق دوست شراری
\\
ماند سخن در دهان و رفت دل من
&&
جانب یاران به سوی دور دیاری
\\
\end{longtable}
\end{center}
