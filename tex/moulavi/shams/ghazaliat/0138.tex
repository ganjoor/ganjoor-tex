\begin{center}
\section*{غزل شماره ۱۳۸: سکه رخسار ما جز زر مبادا بی‌شما}
\label{sec:0138}
\addcontentsline{toc}{section}{\nameref{sec:0138}}
\begin{longtable}{l p{0.5cm} r}
سکه رخسار ما جز زر مبادا بی‌شما
&&
در تک دریای دل گوهر مبادا بی‌شما
\\
شاخه‌های باغ شادی کان قوی تازه‌ست و تر
&&
خشک بادا بی‌شما و تر مبادا بی‌شما
\\
این همای دل که خو کردست در سایه شما
&&
جز میان شعله آذر مبادا بی‌شما
\\
دیدمش بیمار جان را گفتمش چونی خوشی
&&
هین بگو چون نیست میوه برمبادا بی‌شما
\\
روز من تابید جان و در خیالش بنگرید
&&
گفت رنج صعب من خوشتر مبادا بی‌شما
\\
چون شما و جمله خلقان نقش‌های آزرند
&&
نقش‌های آزر و آزر مبادا بی‌شما
\\
جرعه جرعه مر جگر را جام آتش می‌دهیم
&&
کاین جگر را شربت کوثر مبادا بی‌شما
\\
صد هزاران جان فدا شد از پی باده الست
&&
عقل گوید کان می‌ام در سر مبادا بی‌شما
\\
هر دو ده یعنی دو کون از بوی تو رونق گرفت
&&
در دو ده این چاکرت مهتر مبادا بی‌شما
\\
چشم را صد پر ز نور از بهر دیدار توست
&&
ای که هر دو چشم را یک پر مبادا بی‌شما
\\
بی شما هر موی ما گر سنجر و خسرو شوند
&&
خسرو شاهنشه و سنجر مبادا بی‌شما
\\
تا فراق شمس تبریزی همی خنجر کشد
&&
دست‌های گل به جز خنجر مبادا بی‌شما
\\
\end{longtable}
\end{center}
