\begin{center}
\section*{غزل شماره ۳۰۵۲: چه باده بود که در دور از بگه دادی}
\label{sec:3052}
\addcontentsline{toc}{section}{\nameref{sec:3052}}
\begin{longtable}{l p{0.5cm} r}
چه باده بود که در دور از بگه دادی
&&
که می‌شکافد دور زمانه از شادی
\\
نبود باده به جان تو راست گو که چه بود
&&
بهانه راست مکن کژ مگو به استادی
\\
چه راست می‌طلبی ای دل سلیم از او
&&
که راست نیست به جز قد او در این وادی
\\
تو راست باش چو تیر و حریف کژ چو کمان
&&
چو تیر زه به دهان گیر چون درافتادی
\\
ازانک راستی تو غلام آن کژی است
&&
اگر تو تیری بهر کمان کژ زادی
\\
بیار بار دگر تا ببینم آن چه میست
&&
که جان عارف مستی و خصم زهادی
\\
نکو ندیدم آن بار سخت تشنه بدم
&&
بیار بار دگر چون مطیع و منقادی
\\
نمی‌فریبمت این یک بیار و دیگر بس
&&
کی با تو حیله کند حیله را تو بنیادی
\\
فریب و عشوه تو تلقین کنی دو عالم را
&&
ولی مرا مددی ده چو خنب بگشادی
\\
چو جمع روزه گشادند خیک را بمبند
&&
که عیش را تو عروسی و هم تو دامادی
\\
اگر به خوک از آن خیک جرعه‌ای بدهی
&&
به پیش خوک کند شیر چرخ آحادی
\\
چو نام باده برم آن تویی و آتش تو
&&
وگر غریو کنم در میان فریادی
\\
چنان نه‌ای تو که با تو دگر کسی گنجد
&&
ولی ز رشک لقب‌های طرفه بنهادی
\\
گهی سبو و گهی جام و گه حلال و حرام
&&
همه تویی که گهی مهدیی و گه هادی
\\
به نور رفعت ماهی به لطف چون گلزار
&&
ولی چو سرو و چو سوسن ز هر دو آزادی
\\
ولی چو ای همه گویم نداندت اجزا
&&
که فرد جزو نداند به غیر افرادی
\\
مثل به جزو زنم تا که جزو میل کند
&&
چو میل کرد کشانیش تو به آبادی
\\
بیار مفخر تبریز شمس تبریزی
&&
مثال اصل که اصل وجود و ایجادی
\\
\end{longtable}
\end{center}
