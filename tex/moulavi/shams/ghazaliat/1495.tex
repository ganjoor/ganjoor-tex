\begin{center}
\section*{غزل شماره ۱۴۹۵: اگر تو نیستی در عاشقی خام}
\label{sec:1495}
\addcontentsline{toc}{section}{\nameref{sec:1495}}
\begin{longtable}{l p{0.5cm} r}
اگر تو نیستی در عاشقی خام
&&
بیا مگریز از یاران بدنام
\\
تو آن مرغی که میل دانه داری
&&
نباشد در جهان یک دانه بی‌دام
\\
مکن ناموس و با قلاش بنشین
&&
که پیش عاشقان چه خاص و چه عام
\\
اگر ناموس راه تو بگیرد
&&
بکش او را و خونش را بیاشام
\\
که این سودا هزاران ناز دارد
&&
مکن ناز و بکش ناز و بیارام
\\
حریفا اندر آتش صبر می کن
&&
که آتش آب می گردد به ایام
\\
نشان ده راه خمخانه که مستم
&&
که دادم من جهانی را به یک جام
\\
برادر کوی قلاشان کدام است
&&
اگر در بسته باشد رفتم از بام
\\
به پیش پیر میخانه بمیرم
&&
زهی مرگ و زهی برگ و سرانجام
\\
\end{longtable}
\end{center}
