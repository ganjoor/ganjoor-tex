\begin{center}
\section*{غزل شماره ۱۴۸۵: صبح است و صبوح است بر این بام برآییم}
\label{sec:1485}
\addcontentsline{toc}{section}{\nameref{sec:1485}}
\begin{longtable}{l p{0.5cm} r}
صبح است و صبوح است بر این بام برآییم
&&
از ثور گریزیم و به برج قمر آییم
\\
پیکار نجوییم و ز اغیار نگوییم
&&
هنگام وصال است بدان خوش صور آییم
\\
روی تو گلستان و لب تو شکرستان
&&
در سایه این هر دو همه گلشکر آییم
\\
خورشید رخ خوب تو چون تیغ کشیده‌ست
&&
شاید که به پیش تو چو مه شب سپر آییم
\\
زلف تو شب قدر و رخ تو همه نوروز
&&
ما واسطه روز و شبش چون سحر آییم
\\
این شکل ندانیم که آن شکل نمودی
&&
ور زانک دگرگونه نمایی دگر آییم
\\
خورشید جهانی تو و ما ذره پنهان
&&
درتاب در این روزن تا در نظر آییم
\\
خورشید چو از روی تو سرگشته و خیره‌ست
&&
ما ذره عجب نیست که خیره نگر آییم
\\
گفتم چو بیایید دو صد در بگشایید
&&
گفتند که این هست ولیکن اگر آییم
\\
گفتم که چو دریا به سوی جوی نیاید
&&
چون آب روان جانب او در سفر آییم
\\
ای ناطقه غیب تو برگوی که تا ما
&&
از مخبر و اخبار خوشت خوش خبر آییم
\\
\end{longtable}
\end{center}
