\begin{center}
\section*{غزل شماره ۱۴۷۹: آن خانه که صد بار در او مایده خوردیم}
\label{sec:1479}
\addcontentsline{toc}{section}{\nameref{sec:1479}}
\begin{longtable}{l p{0.5cm} r}
آن خانه که صد بار در او مایده خوردیم
&&
بر گرد حوالی گه آن خانه بگردیم
\\
ماییم و حوالی گه آن خانه دولت
&&
ما نعمت آن خانه فراموش نکردیم
\\
آن خانه مردی است و در او شیردلانند
&&
از خانه مردی بگریزیم چه مردیم
\\
آن جا همه مستی است و برون جمله خمار است
&&
آن جا همه لطفیم و دگر جا همه دردیم
\\
آن جا طرب انگیزتر از باده لعلیم
&&
وین جا بد و رخ زردتر از شیشه زردیم
\\
آن جای به گرمی همه خورشید تموزیم
&&
وین جای به سردی همه چون بهمن سردیم
\\
آن جا همه آمیخته چون شکر و شیریم
&&
وین جا همه آویخته در جنگ و نبردیم
\\
آن جا شه شطرنج بساط دو جهانیم
&&
وین جا همه سرگشته‌تر از مهره نردیم
\\
چرخی است کز آن چرخ چو یک برق بتابد
&&
بر چرخ برآییم و زمین را بنوردیم
\\
\end{longtable}
\end{center}
