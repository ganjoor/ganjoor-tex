\begin{center}
\section*{غزل شماره ۱۷۰۹: ما قحطیان تشنه و بسیارخواره‌ایم}
\label{sec:1709}
\addcontentsline{toc}{section}{\nameref{sec:1709}}
\begin{longtable}{l p{0.5cm} r}
ما قحطیان تشنه و بسیارخواره‌ایم
&&
بیچاره نیستیم که درمان و چاره‌ایم
\\
در بزم چون عقار و گه رزم ذوالفقار
&&
در شکر همچو چشمه و در صبر خاره‌ایم
\\
ما پادشاه رشوت باره نبوده‌ایم
&&
بل پاره دوز خرقه دل‌های پاره‌ایم
\\
از ما مپوش راز که در سینه توایم
&&
وز ما مدزد دل که نه ما دل فشاره‌ایم
\\
ما آب قلزمیم نهان گشته زیر کاه
&&
یا آفتاب تن زده اندر ستاره‌ایم
\\
ما را ببین تو مست چنین بر کنار بام
&&
داند کنار بام که ما بی‌کناره‌ایم
\\
مهتاب را چه ترس بود از کنار بام
&&
پس ما چه غم خوریم که بر مه سواره‌ایم
\\
گر تیردوز گشت جگرهای ما ز عشق
&&
بی‌زحمت جگر تو ببین خون چه کاره‌ایم
\\
قصاب ده اگر چه که ما را بکشت زار
&&
هم می چریم در ده و هم بر قناره‌ایم
\\
ما مهره‌ایم و هم جهت مهره حقه‌ایم
&&
هنگامه گیر دل شده و هم نظاره‌ایم
\\
خاموش باش اگر چه به بشرای احمدی
&&
همچون مسیح ناطق طفل گواره‌ایم
\\
در عشق شمس مفخر تبریز روز و شب
&&
بر چرخ دیوکش چو شهاب و شراره‌ایم
\\
\end{longtable}
\end{center}
