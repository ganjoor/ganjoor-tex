\begin{center}
\section*{غزل شماره ۱۰۱۸: رو چشم جان را برگشا در بی‌دلان اندرنگر}
\label{sec:1018}
\addcontentsline{toc}{section}{\nameref{sec:1018}}
\begin{longtable}{l p{0.5cm} r}
رو چشم جان را برگشا در بی‌دلان اندرنگر
&&
قومی چو دل زیر و زبر قومی چو جان بی‌پا و سر
\\
بی‌کسب و بی‌کوشش همه چون دیگ در جوشش همه
&&
بی‌پرده و پوشش همه دل پیش حکمش چون سپر
\\
از باغ و گل دلشادتر وز سرو هم آزادتر
&&
وز عقل و دانش رادتر وز آب حیوان پاکتر
\\
چون ذره‌ها اندر هوا خورشید ایشان را قبا
&&
بر آب و گل بنهاده پا وز عین دل برکرده سر
\\
در موج دریاهای خون بگذشته بر بالای خون
&&
وز موج وز غوغای خون دامانشان ناگشته تر
\\
در خار لیکن همچو گل در حبس ولیکن همچو مل
&&
در آب و گل لیکن چو دل در شب ولیکن چو سحر
\\
باری تو از ارواحشان وز باده و اقداحشان
&&
مستی خوشی از راحشان فارغ شده از خیر و شر
\\
بس کن که هر مرغ ای پسر خود کی خورد انجیر تر
&&
شد طعمه طوطی شکر وان زاغ را چیزی دگر
\\
\end{longtable}
\end{center}
