\begin{center}
\section*{غزل شماره ۱۱۰۶: راز را اندر میان نه وامگیر}
\label{sec:1106}
\addcontentsline{toc}{section}{\nameref{sec:1106}}
\begin{longtable}{l p{0.5cm} r}
راز را اندر میان نه وامگیر
&&
بنده را هر لحظه از بالا مگیر
\\
تو نکو دانی که هر چیز از کجاست
&&
گر خطاها رفت آن از ما مگیر
\\
روستایی گر بوم آن توام
&&
روستایی خویش را رستا مگیر
\\
چون مرا در عشق‌ست ا کرده‌ای
&&
خود مرا شاگرد گیر ستا مگیر
\\
تو مرا از ذوق می‌گیری گلو
&&
تا بنالم گویمت آن جا مگیر
\\
سوی بحرم کش که خاشاک توام
&&
تو مرا خود لایق دریا مگیر
\\
از الست آمد صلاح الدین تمام
&&
تو ورا ز امروز و از فردا مگیر
\\
\end{longtable}
\end{center}
