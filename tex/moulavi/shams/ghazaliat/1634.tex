\begin{center}
\section*{غزل شماره ۱۶۳۴: عقل گوید که من او را به زبان بفریبم}
\label{sec:1634}
\addcontentsline{toc}{section}{\nameref{sec:1634}}
\begin{longtable}{l p{0.5cm} r}
عقل گوید که من او را به زبان بفریبم
&&
عشق گوید تو خمش باش به جان بفریبم
\\
جان به دل گوید رو بر من و بر خویش مخند
&&
چیست کو را نبود تاش بدان بفریبم
\\
نیست غمگین و پراندیشه و بی‌هوشی جوی
&&
تا من او را به می و رطل گران بفریبم
\\
ناوک غمزه او را به کمان حاجت نیست
&&
تا خدنگ نظرش را به کمان بفریبم
\\
نیست محبوس جهان بسته این عالم خاک
&&
تا من او را به زر و ملک جهان بفریبم
\\
او فرشته‌ست اگر چه که به صورت بشر است
&&
شهوتی نیست که او را به زنان بفریبم
\\
خانه کاین نقش در او هست فرشته برمد
&&
پس کیش من به چنین نقش و نشان بفریبم
\\
گله اسب نگیرد چو به پر می پرد
&&
خور او نور بود چونش به نان بفریبم
\\
نیست او تاجر و سوداگر بازار جهان
&&
تا به افسونش به هر سود و زیان بفریبم
\\
نیست محجوب که رنجور کنم من خود را
&&
آه آهی کنم او را به فغان بفریبم
\\
سر ببندم بنهم سر که من از دست شدم
&&
رحمتش را به مرض یا خفقان بفریبم
\\
موی در موی ببیند کژی و فعل مرا
&&
چیست پنهان بر او کش به نهان بفریبم
\\
نیست شهرت طلب و خسرو شاعرباره
&&
کش به بیت غزل و شعر روان بفریبم
\\
عزت صورت غیبی خود از آن افزون است
&&
که من او را به جنان یا به جنان بفریبم
\\
شمس تبریز که بگزیده و محبوب وی است
&&
مگر او را به همان قطب زمان بفریبم
\\
\end{longtable}
\end{center}
