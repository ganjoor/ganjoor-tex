\begin{center}
\section*{غزل شماره ۲۲۶۶: بگردان ساقی مه روی جام}
\label{sec:2266}
\addcontentsline{toc}{section}{\nameref{sec:2266}}
\begin{longtable}{l p{0.5cm} r}
بگردان ساقی مه روی جام
&&
رهایی ده مرا از ننگ و نام
\\
گرفتارم به دامت ساقیا ز آنک
&&
نهادستی به هر گامی تو دام
\\
رها کن کاهلی دریاب ما را
&&
و لا تکسل فان القوم قاموا
\\
الیس الصحو منزل کل هم
&&
الیس العیش فی هم حرام
\\
الا صوموا فان الصوم غنم
&&
شراب الروح یشربه الصیام
\\
هر آن کو روزه دارد در حدیث است
&&
مه حق را ببیند وقت شام
\\
نکو نبود که من از در درآیم
&&
تو بگریزی ز من از راه بام
\\
تو بگریزی و من فریاد در پی
&&
که یک دم صبر کن ای تیزگام
\\
مسلمانان مسلمانان چه چاره‌ست
&&
که من سوزیدم و این کار خام
\\
نباشد چاره جز صافی شرابی
&&
باقداح یقلبها الکرام
\\
حدیث عاشقان پایان ندارد
&&
فنستکفی بهذا و السلام
\\
جواب گفته متنبی است این
&&
فؤاد ما تسلیه المدام
\\
\end{longtable}
\end{center}
