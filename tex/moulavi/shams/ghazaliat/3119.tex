\begin{center}
\section*{غزل شماره ۳۱۱۹: نشانت کی جوید که تو بی‌نشانی}
\label{sec:3119}
\addcontentsline{toc}{section}{\nameref{sec:3119}}
\begin{longtable}{l p{0.5cm} r}
نشانت کی جوید که تو بی‌نشانی
&&
مکانت کی یابد که تو بی‌مکانی
\\
چه صورت کنیمت که صورت نبندی
&&
که کفست صورت به بحر معانی
\\
از آن سوی پرده چه شهری شگرفست
&&
که عالم از آن جاست یک ارمغانی
\\
به نو نو هلالی به نو نو خیالی
&&
رسد تا نماند حقیقت نهانی
\\
گدارو مباش و مزن هر دری را
&&
که هر چیز را که بجویی تو آنی
\\
دلا خیمه خود بر این آسمان زن
&&
مگو که نتانم بلی می‌توانی
\\
مددهای جانت همه ز آسمانست
&&
از آن سو رسیدی همان سوی روانی
\\
گمان‌های ناخوش برد بر تو دل‌ها
&&
نداند که تو حاضر هر گمانی
\\
به چه عذر آید چه روپوش دارد
&&
که تو نانبشته غرض را بخوانی
\\
خنک آن زمانی که ساقی تو باشی
&&
بریزی تو بر ما قدح‌های جانی
\\
ز سر گیرد این دل عروج منازل
&&
ز سر گیرد این تن مزاج جوانی
\\
خنک آن زمانی که هر پاره ما
&&
به رقص اندرآید که ربی سقانی
\\
گرانی نماند در آن جا و غیری
&&
که گیرد سر مست از می گرانی
\\
به گفت اندرآیند اجزای خامش
&&
چنان که تو ناطق در آن خیره مانی
\\
چه‌ها می‌کند مادر نفس کلی
&&
که تا بی‌لسانی بیابد لسانی
\\
ایا نفس کلی به هر دم کیاست
&&
کیت می‌فرستد به رسم نهانی
\\
مگو عقل کلی که آن عقل کل را
&&
به هر دم کسی می‌کند مستعانی
\\
که آن عقل کلی شود عقل کلی
&&
گر آبی نیاید ز بحر عیانی
\\
\end{longtable}
\end{center}
