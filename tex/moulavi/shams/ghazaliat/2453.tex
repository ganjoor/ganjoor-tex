\begin{center}
\section*{غزل شماره ۲۴۵۳: بویی ز گردون می‌رسد با پرسش و دلداریی}
\label{sec:2453}
\addcontentsline{toc}{section}{\nameref{sec:2453}}
\begin{longtable}{l p{0.5cm} r}
بویی ز گردون می‌رسد با پرسش و دلداریی
&&
از دام تن وا می‌رهد هر خسته دل اشکاریی
\\
هر مرغ صدپر می‌شود سوی ثریا می‌پرد
&&
هر کوه و لنگر زین صلا دارد دگر رهواریی
\\
مرغان ابراهیم بین با پاره پاره گشتگی
&&
اجزای هر تن سوی سر برداشته طیاریی
\\
ای جزو چون بر می‌پری چون بی‌پری و بی‌سری
&&
گفتا شکفته می‌شوم اندر نسیم یاریی
\\
در شهر دیگر نشنوی از غیر سرنا ناله‌ای
&&
از غیر چنگی نشنوی در هیچ خانه زاریی
\\
طنبور دل برداشته لا عیش الا عیشنا
&&
زنبور جان آموخته زین انگبین معماریی
\\
امروز ساقی کرم دریاعطای محتشم
&&
آمیخته با بندگان بی‌نخوت و جباریی
\\
امروز رستیم ای خدا از غصه آنک قضا
&&
در گوش فتنه دردمد هر لحظه‌ای مکاریی
\\
راقی جان در می‌دمد چون پور مریم رقیه‌ای
&&
ساقی ما هم می‌کند چون شیر حق کراریی
\\
گر درک بت را بشکند صد بت تراشد در عوض
&&
ور بشکند دو سه سبو کم نیستش فخاریی
\\
ای بلبل ار چه یافتی از دولت گل لحن خوش
&&
زینهار فراموشت شود در انس کم گفتاریی
\\
\end{longtable}
\end{center}
