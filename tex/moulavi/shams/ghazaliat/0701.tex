\begin{center}
\section*{غزل شماره ۷۰۱: آن خواجه خوش لقا چه دارد}
\label{sec:0701}
\addcontentsline{toc}{section}{\nameref{sec:0701}}
\begin{longtable}{l p{0.5cm} r}
آن خواجه خوش لقا چه دارد
&&
بازار مرا بها چه دارد
\\
او عشوه دهد از او تو مشنو
&&
رختش بطلب که تا چه دارد
\\
نقدش برکش ببین که چندست
&&
در نقد دگر دغا چه دارد
\\
گر دست و ترازوی نداری
&&
تا برکشی کز صفا چه دارد
\\
اندر سخنش کشان و بو گیر
&&
کز بوی می بقا چه دارد
\\
شاد آن که بجست جان خود را
&&
کز حالت مرتضا چه دارد
\\
در خویش ز اولیا چه بیند
&&
وز لذت انبیا چه دارد
\\
گفتم به قلندری که بنگر
&&
کان چرخ که شد دوتا چه دارد
\\
گفتا که فراغتیست ما را
&&
کو خود چه کس است یا چه دارد
\\
مستم ز خدا و سخت مستم
&&
سبحان الله خدا چه دارد
\\
از رحمت شمس دین تبریز
&&
هر سینه جدا جدا چه دارد
\\
\end{longtable}
\end{center}
