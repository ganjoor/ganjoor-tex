\begin{center}
\section*{غزل شماره ۱۴۱۴: درخت و آتشی دیدم ندا آمد که جانانم}
\label{sec:1414}
\addcontentsline{toc}{section}{\nameref{sec:1414}}
\begin{longtable}{l p{0.5cm} r}
درخت و آتشی دیدم ندا آمد که جانانم
&&
مرا می خواند آن آتش مگر موسی عمرانم
\\
دخلت التیه بالبلوی و ذقت المن و السلوی
&&
چهل سال است چون موسی به گرد این بیابانم
\\
مپرس از کشتی و دریا بیا بنگر عجایب‌ها
&&
که چندین سال من کشتی در این خشکی همی‌رانم
\\
بیا ای جان تویی موسی وین قالب عصای تو
&&
چو برگیری عصا گردم چو افکندیم ثعبانم
\\
تویی عیسی و من مرغت تو مرغی ساختی از گل
&&
چنانک دردمی در من چنان در اوج پرانم
\\
منم استون آن مسجد که مسند ساخت پیغامبر
&&
چو او مسند دگر سازد ز درد هجر نالانم
\\
خداوند خداوندان و صورت ساز بی‌صورت
&&
چه صورت می کشی بر من تو دانی من نمی‌دانم
\\
گهی سنگم گهی آهن زمانی آتشم جمله
&&
گهی میزان بی‌سنگم گهی هم سنگ و میزانم
\\
زمانی می چرم این جا زمانی می چرند از من
&&
گهی گرگم گهی میشم گهی خود شکل چوپانم
\\
هیولایی نشان آمد نشان دایم کجا ماند
&&
نه این ماند نه آن ماند بداند آن من آنم
\\
\end{longtable}
\end{center}
