\begin{center}
\section*{غزل شماره ۶۱۳: ای خواجه بازرگان از مصر شکر آمد}
\label{sec:0613}
\addcontentsline{toc}{section}{\nameref{sec:0613}}
\begin{longtable}{l p{0.5cm} r}
ای خواجه بازرگان از مصر شکر آمد
&&
وان یوسف چون شکر ناگه ز سفر آمد
\\
روح آمد و راح آمد معجون نجاح آمد
&&
ور چیز دگر خواهی آن چیز دگر آمد
\\
آن میوه یعقوبی وان چشمه ایوبی
&&
از منظره پیدا شد هنگام نظر آمد
\\
خضر از کرم ایزد بر آب حیاتی زد
&&
نک زهره غزل گویان در برج قمر آمد
\\
آمد شه معراجی شب رست ز محتاجی
&&
گردون به نثار او با دامن زر آمد
\\
موسی نهان آمد صد چشمه روان آمد
&&
جان همچو عصا آمد تن همچو حجر آمد
\\
زین مردم کارافزا زین خانه پرغوغا
&&
عیسی نخورد حلوا کاین آخر خر آمد
\\
چون بسته نبود آن دم در شش جهت عالم
&&
در جستن او گردون بس زیر و زبر آمد
\\
آن کو مثل هدهد بی‌تاج نبد هرگز
&&
چون مور ز مادر او بربسته کمر آمد
\\
در عشق بود بالغ از تاج و کمر فارغ
&&
کز کرسی و از عرشش منشور ظفر آمد
\\
باقیش ز سلطان جو سلطان سخاوت خو
&&
زو پرس خبرها را کو کان خبر آمد
\\
\end{longtable}
\end{center}
