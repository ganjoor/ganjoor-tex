\begin{center}
\section*{غزل شماره ۳۵۴: ز میخانه دگربار این چه بویست}
\label{sec:0354}
\addcontentsline{toc}{section}{\nameref{sec:0354}}
\begin{longtable}{l p{0.5cm} r}
ز میخانه دگربار این چه بویست
&&
دگربار این چه شور و گفت و گویست
\\
جهان بگرفت ارواح مجرد
&&
زمین و آسمان پرهای و هوی‌ست
\\
بیا ای عشق این می از چه خمست
&&
اشارت کن خرابات از چه سوی‌ست
\\
چه می‌گویم اشارت چیست کاین جا
&&
نگنجد فکرتی کان همچو مویست
\\
نیاید در نظر آن سر یک تو
&&
که در فکر آنچ آید چارتویست
\\
چو ز اندیشه به گفت آید چه گویم
&&
که خانه کنده و رسوای کویست
\\
ز رسوایی به بحر دل رود باز
&&
که دل بحرست و گفتن‌ها چو جویست
\\
خزینه دار گوهر بحر بدخوست
&&
که آب جو و چه تن جامه شویست
\\
\end{longtable}
\end{center}
