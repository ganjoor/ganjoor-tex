\begin{center}
\section*{غزل شماره ۱۳۰۰: عیسی روح گرسنه‌ست چو زاغ}
\label{sec:1300}
\addcontentsline{toc}{section}{\nameref{sec:1300}}
\begin{longtable}{l p{0.5cm} r}
عیسی روح گرسنه‌ست چو زاغ
&&
خر او می‌کند ز کنجد کاغ
\\
چونک خر خورد جمله کنجد را
&&
از چه روغن کشیم بهر چراغ
\\
چونک خورشید سوی عقرب رفت
&&
شد جهان تیره رو ز میغ و ز ماغ
\\
آفتابا رجوع کن به محل
&&
بر جبین خزان و دی نه داغ
\\
آفتابا تو در حمل جانی
&&
از تو سرسبز خاک و خندان باغ
\\
آفتابا چو بشکنی دل دی
&&
از تو گردد بهار گرم دماغ
\\
آفتابا زکات نور تو است
&&
آنچ این آفتاب کرد ابلاغ
\\
صد هزار آفتاب دید احمد
&&
چون تو را دیده بود او مازاغ
\\
زان نگشت او بگرد پایه حوض
&&
کو ز بحر حیات دید اسباغ
\\
آفتابت از آن همی‌خوانم
&&
که عبارت ز تست تنگ مساغ
\\
مژده تو چو درفکند بهار
&&
باغ برداشت بزم و مجلس و لاغ
\\
کرده مستان باغ اشکوفه
&&
کرده سیران خاک استفراغ
\\
حله بافان غیب می‌بافند
&&
حله‌ها و پدید نیست پناغ
\\
کی گذارد خدا تو را فارغ
&&
چون خدا را ز کار نیست فراغ
\\
صد هزاران بنا و یک بنا
&&
رنگ جامه هزار و یک صباغ
\\
نغزها را مزاج او مایه
&&
پوست‌ها را علاج او دباغ
\\
لعل‌ها را درخش او صیقل
&&
سیم و زر را کفایتش صواغ
\\
بلبلان ضمیر خود دگرند
&&
نطق حس پیششان چو بانگ کلاغ
\\
بس که همراز بلبلان نبود
&&
آنک بیرون بود ز باغ و ز راغ
\\
\end{longtable}
\end{center}
