\begin{center}
\section*{غزل شماره ۳۱۶۵: جان و جهان! دوش کجا بودهٔ}
\label{sec:3165}
\addcontentsline{toc}{section}{\nameref{sec:3165}}
\begin{longtable}{l p{0.5cm} r}
جان و جهان! دوش کجا بوده‌ای
&&
نی غلطم، در دل ما بوده‌ای
\\
دوش ز هجر تو جفا دیده‌ام
&&
ای که تو سلطان وفا بوده‌ای
\\
آه که من دوش چه سان بوده‌ام!
&&
آه که تو دوش کرا بوده‌ای!
\\
رشک برم کاش قبا بودمی
&&
چونک در آغوش قبا بوده‌ای
\\
زهره ندارم که بگویم ترا
&&
« بی من بیچاره چرا بوده‌ای؟! »
\\
یار سبک روح! به وقت گریز
&&
تیزتر از باد صبا بوده‌ای
\\
بی‌تو مرا رنج و بلا بند کرد
&&
باش که تو بنده بلا بوده‌ای
\\
رنگ رخ خوب تو آخر گواست
&&
در حرم لطف خدا بوده‌ای
\\
رنگ تو داری، که زرنگ جهان
&&
پاکی، و همرنگ بقا بوده‌ای
\\
آینهٔ رنگ تو عکس کسیست
&&
تو ز همه رنگ جدا بوده‌ای
\\
\end{longtable}
\end{center}
