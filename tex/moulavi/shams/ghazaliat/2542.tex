\begin{center}
\section*{غزل شماره ۲۵۴۲: بتاب ای ماه بر یارم بگو یارا اغا پوسی}
\label{sec:2542}
\addcontentsline{toc}{section}{\nameref{sec:2542}}
\begin{longtable}{l p{0.5cm} r}
بتاب ای ماه بر یارم بگو یارا اغا پوسی
&&
بزن ای باد بر زلفش که ای زیبا اغا پوسی
\\
گر این جایی گر آن جایی وگر آیی وگر نایی
&&
همه قندی و حلوایی زهی حلوا اغا پوسی
\\
ملامت نشنوم هرگز نگردم در طلب عاجز
&&
نباشد عشق بازیچه بیا حقا اغا پوسی
\\
اگر در خاک بنهندم تویی دلدار و دلبندم
&&
وگر بر چرخ آرندم از آن بالا اغا پوسی
\\
اگر بالای که باشم چو رهبان عشق تو جویم
&&
وگر در قعر دریاام در آن دریا اغا پوسی
\\
ز تاب روی تو ماها ز احسان‌های تو شاها
&&
شده زندان مرا صحرا در آن صحرا اغا پوسی
\\
چو مست دیدن اویم دو دست از شرم واشویم
&&
بگیرم در رهش گویم که ای مولا اغا پوسی
\\
دلارام خوش روشن ستیزه می‌کند با من
&&
بیار ای اشک و بر وی زن بگو ایلا اغا پوسی
\\
تو را هر جان همی‌جوید که تا پای تو را بوسد
&&
ندارد زهره تا گوید بیا این جا اغا پوسی
\\
وگر از بنده سیرابی بگیری خشم و دیر آیی
&&
بماند بی‌کس و تنها تو را تنها اغا پوسی
\\
بیا ای باغ و ای گلشن بیا ای سرو و ای سوسن
&&
برای کوری دشمن بگو ما را اغا پوسی
\\
بیا پهلوی من بنشین به رسم و عادت پیشین
&&
بجنبان آن لب شیرین که مولانا اغا پوسی
\\
منم نادان تویی دانا تو باقی را بگو جانا
&&
به گویایی افیغومی به ناگویا اغا پوسی
\\
\end{longtable}
\end{center}
