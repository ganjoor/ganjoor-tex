\begin{center}
\section*{غزل شماره ۳۱۸۵: عزیزی و کریم و لطف داری}
\label{sec:3185}
\addcontentsline{toc}{section}{\nameref{sec:3185}}
\begin{longtable}{l p{0.5cm} r}
عزیزی و کریم و لطف داری
&&
ولیکن دور شو، چون هوشیاری
\\
نشاید عاشقان را یار هشیار
&&
ز هشیاران نیاید هیچ یاری
\\
مرا یکدم چو ساقی کم دهد می
&&
بگیرم دامن او را به زاری
\\
صراحی‌وار خون گریم به پیشش
&&
بجوشم همچو می در بی‌قراری
\\
که از اندیشه بیزارم، بده می
&&
مرا تا کی به اندیشه سپاری؟!
\\
چه حیله سازم ای ساقی؟! چه حیله؟!
&&
که حیله آفرین و حیله‌کاری
\\
به حجت هر دمم بیرون فرستی
&&
که بس باغیرتی و تنگ باری
\\
برون و اندرون و جام و می نیست
&&
ولیکن در سخن اینست جاری
\\
قفی یا ناقتی هذا مناخ
&&
ولا تسرین من هذاالدیار
\\
فدیت‌العشق ما احلی هواه
&&
تقطع فی هواه اختیاری
\\
فلا تشغلنی یا ساقی بلهو
&&
واسکرنی بکاسات کبار
\\
ایا بدرالتمام اطلع علینا
&&
بحق العشق اسمع، لاتمار
\\
وخلصنی من‌الدنیا واسکر
&&
فلا ادری یمینی من یساری
\\
\end{longtable}
\end{center}
