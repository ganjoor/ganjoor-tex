\begin{center}
\section*{غزل شماره ۹۵۸: کدام لب که از او بوی جان نمی‌آید}
\label{sec:0958}
\addcontentsline{toc}{section}{\nameref{sec:0958}}
\begin{longtable}{l p{0.5cm} r}
کدام لب که از او بوی جان نمی‌آید
&&
کدام دل که در او آن نشان نمی‌آید
\\
مثال اشتر هر ذره‌ای چه می‌خاید
&&
اگر نواله از آن شهره خوان نمی‌آید
\\
سگان طمع چپ و راست از چه می‌پویند
&&
چو بوی قلیه از آن دیگدان نمی‌آید
\\
چراست پنجه شیران چو برگ گل لرزان
&&
اگر ز غیب به دل‌ها سنان نمی‌آید
\\
هزار بره و گرگ از چه روی هم علفند
&&
به جان چو هیبت و بانگ شبان نمی‌آید
\\
برون گوش دو صد نعره جان همی‌شنود
&&
تو هوش دار چنین گر چنان نمی‌آید
\\
در این جهان کهن جان نو چرا روید
&&
چو هر دمی مددی زان جهان نمی‌آید
\\
به دست خویش تو در چشم می‌فشانی خاک
&&
نه آن که صورت نو نو عیان نمی‌آید
\\
شکسته قرن نگر صد هزار ذوالقرنین
&&
قرین بسیست که صاحب قران نمی‌آید
\\
دهان و دست به آب وفا کی می‌شوید
&&
که دم دمش می جان در دهان نمی‌آید
\\
دو سه قدم به سوی باغ عشق کس ننهاد
&&
که صد سلامش از آن باغبان نمی‌آید
\\
ورای عشق هزاران هزار ایوان هست
&&
ز عزت و عظمت در گمان نمی‌آید
\\
به هر دمی ز درونت ستاره‌ای تابد
&&
که هین مگو کاثری ز آسمان نمی‌آید
\\
دهان ببند و دهان آفرین کند شرحش
&&
به صورتی که تو را در زبان نمی‌آید
\\
\end{longtable}
\end{center}
