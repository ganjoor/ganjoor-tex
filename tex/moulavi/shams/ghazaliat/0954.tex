\begin{center}
\section*{غزل شماره ۹۵۴: افزود آتش من آب را خبر ببرید}
\label{sec:0954}
\addcontentsline{toc}{section}{\nameref{sec:0954}}
\begin{longtable}{l p{0.5cm} r}
فزود آتش من آب را خبر ببرید
&&
اسیر می‌بردم غم ز کافرم بخرید
\\
خدای داد شما را یکی نظر که مپرس
&&
اگر چه زان نظر این دم به سکر بی‌خبرید
\\
طراز خلعت آن خوش نظر چو دیده شود
&&
هزار جامه ز درد و دریغ و غم بدرید
\\
ز دیده موی برست از دقیقه بینی‌ها
&&
چرا به موی و به روی خوشش نمی‌نگرید
\\
ز حرص خواجگی از بندگی چه محرومید
&&
ز غورها همه پختید یا که کور و کرید
\\
در آشنا عجمی وار منگرید چنین
&&
فرشته‌اید به معنی اگر به تن بشرید
\\
هزار حاجب و جاندار منتظر دارید
&&
برای خدمتتان لیک در ره و سفرید
\\
همی‌پرد به سوی آسمان روان شما
&&
اگر چه زیر لحافید و هیچ می‌نپرید
\\
همی‌چرد همه اجزای جان به روض صفات
&&
از آن ریاض که رستید چون از آن نچرید
\\
درخت مایه از آن یافت سبز و تر زان شد
&&
زبون مایه چرایید چونک شیر نرید
\\
هزار گونه کجا خستتان به زیر سجود
&&
کجا نظر که بدانید تیغ یا سپرید
\\
هزار حرف به بیگار گفتم و مقصود
&&
به هر دمی ز چه شما خفیه تر چه بی‌هنرید
\\
هنر چو بی‌هنری آمد اندر این درگاه
&&
هنروران ز شادیت چون نه زین نفرید
\\
همه حیات در اینست کاذبحوا بقره
&&
چو عاشقان حیاتید چون پس بقرید
\\
هزار شیر تو را بنده‌اند چه بود گاو
&&
هزار تاج زر آمد چه در غم کمرید
\\
چو شب خطیب تو ماهست بر چنین منبر
&&
اگر نه فهم تباهست از چه در سمرید
\\
کجا بلاغت ماه و کجا خیال سپاه
&&
به مقنعه بمنازید چون کلاه ورید
\\
بیافت کوزه زرین و آب بی‌حد خورد
&&
خموش باش که تا ز آب هم شکم ندرید
\\
\end{longtable}
\end{center}
