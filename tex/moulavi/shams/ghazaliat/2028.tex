\begin{center}
\section*{غزل شماره ۲۰۲۸: گر چه بسی نشستم در نار تا به گردن}
\label{sec:2028}
\addcontentsline{toc}{section}{\nameref{sec:2028}}
\begin{longtable}{l p{0.5cm} r}
گر چه بسی نشستم در نار تا به گردن
&&
اکنون در آب وصلم با یار تا به گردن
\\
گفتم که تا به گردن در لطف‌هات غرقم
&&
قانع نگشت از من دلدار تا به گردن
\\
گفتا که سر قدم کن تا قعر عشق می‌رو
&&
زیرا که راست ناید این کار تا به گردن
\\
گفتم سر من ای جان نعلین توست لیکن
&&
قانع شو ای دو دیده این بار تا به گردن
\\
گفتا تو کم ز خاری کز انتظار گل‌ها
&&
در خاک بود نه مه آن خار تا به گردن
\\
گفتم که خار چه بود کز بهر گلستانت
&&
در خون چو گل نشستم بسیار تا به گردن
\\
گفتا به عشق رستی از عالم کشاکش
&&
کان جا همی‌کشیدی بیگار تا به گردن
\\
رستی ز عالم اما از خویشتن نرستی
&&
عار است هستی تو وین عار تا به گردن
\\
عیاروار کم نه تو دام و حیله کم کن
&&
در دام خویش ماند عیار تا به گردن
\\
دامی است دام دنیا کز وی شهان و شیران
&&
ماندند چون سگ اندر مردار تا به گردن
\\
دامی است طرفه‌تر زین کز وی فتاده بینی
&&
بی‌عقل تا به کعب و هشیار تا به گردن
\\
بس کن ز گفتن آخر کان دم بود بریده
&&
کز تاسه نبود آخر گفتار تا به گردن
\\
\end{longtable}
\end{center}
