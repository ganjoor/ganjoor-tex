\begin{center}
\section*{غزل شماره ۲۱۴۴: کار جهان هر چه شود کار تو کو بار تو کو}
\label{sec:2144}
\addcontentsline{toc}{section}{\nameref{sec:2144}}
\begin{longtable}{l p{0.5cm} r}
کار جهان هر چه شود کار تو کو بار تو کو
&&
گر دو جهان بتکده شد آن بت عیار تو کو
\\
گیر که قحط است جهان نیست دگر کاسه و نان
&&
ای شه پیدا و نهان کیله و انبار تو کو
\\
گیر که خار است جهان گزدم و مار است جهان
&&
ای طرب و شادی جان گلشن و گلزار تو کو
\\
گیر که خود مرد سخا کشت بخیلی همه را
&&
ای دل و ای دیده ما خلعت و ادرار تو کو
\\
گیر که خورشید و قمر هر دو فروشد به سقر
&&
ای مدد سمع و بصر شعله و انوار تو کو
\\
گیر که خود جوهریی نیست پی مشتریی
&&
چون نکنی سروریی ابر گهربار تو کو
\\
گیر دهانی نبود گفت زبانی نبود
&&
تا دم اسرار زند جوشش اسرار تو کو
\\
هین همه بگذار که ما مست وصالیم و لقا
&&
بی‌گه شد زود بیا خانه خمار تو کو
\\
تیز نگر مست مرا همدل و هم دست مرا
&&
گر نه خرابی و خرف جبه و دستار تو کو
\\
برد کلاه تو غری برد قبایت دگری
&&
روی تو زرد از قمری پشت و نگهدار تو کو
\\
بر سر مستان ابد خارجیی راه زند
&&
شحنگیی چون نکنی زخم تو کو دار تو کو
\\
خامش ای حرف فشان درخور گوش خمشان
&&
ترجمه خلق مکن حالت و گفتار تو کو
\\
\end{longtable}
\end{center}
