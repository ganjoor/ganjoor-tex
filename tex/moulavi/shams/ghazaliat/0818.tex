\begin{center}
\section*{غزل شماره ۸۱۸: ساقیان سرمست در کار آمدند}
\label{sec:0818}
\addcontentsline{toc}{section}{\nameref{sec:0818}}
\begin{longtable}{l p{0.5cm} r}
ساقیان سرمست در کار آمدند
&&
مستیان در کوی خمار آمدند
\\
حلقه حلقه عاشقان و بی‌دلان
&&
بر امید بوی دلدار آمدند
\\
بلبلان مست و مستان الست
&&
بر امید گل به گلزار آمدند
\\
هین که مخموران در این دم جوق جوق
&&
بر در ساقی به زنهار آمدند
\\
یک ندا آمد عجب از کوی دل
&&
بی دل و بی‌پا به یک بار آمدند
\\
از خوشی بوی او در کوی او
&&
بیخود و بی‌کفش و دستار آمدند
\\
بی محابا ده تو ای ساقی مدام
&&
هین که جان‌ها مست اسرار آمدند
\\
عارفان از خویش بی‌خویش آمدند
&&
زاهدان در کار هشیار آمدند
\\
ساقیا تو جمله را یک رنگ کن
&&
باده ده گر یار و اغیار آمدند
\\
\end{longtable}
\end{center}
