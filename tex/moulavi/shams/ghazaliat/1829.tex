\begin{center}
\section*{غزل شماره ۱۸۲۹: گفتم دوش عشق را ای تو قرین و یار من}
\label{sec:1829}
\addcontentsline{toc}{section}{\nameref{sec:1829}}
\begin{longtable}{l p{0.5cm} r}
گفتم دوش عشق را ای تو قرین و یار من
&&
هیچ مباش یک نفس غایب از این کنار من
\\
نور دو دیده منی دور مشو ز چشم من
&&
شعله سینه منی کم مکن از شرار من
\\
یار من و حریف من خوب من و لطیف من
&&
چست من و ظریف من باغ من و بهار من
\\
ای تن من خراب تو دیده من سحاب تو
&&
ذره آفتاب تو این دل بی‌قرار من
\\
لب بگشا و مشکلم حل کن و شاد کن دلم
&&
کآخر تا کجا رسد پنج و شش قمار من
\\
تا که چه زاید این شب حامله از برای من
&&
تا به کجا کشد بگو مستی بی‌خمار من
\\
تا چه عمل کند عجب شکر من و سپاس من
&&
تا چه اثر کند عجب ناله و زینهار من
\\
گفت خنک تو را که تو در غم ما شدی دوتو
&&
کار تو راست در جهان ای بگزیده کار من
\\
مست منی و پست من عاشق و می پرست من
&&
برخورد او ز دست من هر کی کشید بار من
\\
رو که تو راست کر و فر مجلس عیش نه ز سر
&&
زانک نظر دهد نظر عاقبت انتظار من
\\
گفتم وانما که چون زنده کنی تو مرده را
&&
زنده کن این تن مرا از پی اعتبار من
\\
مرده‌تر از تنم مجو زنده کنش به نور هو
&&
تا همه جان شود تنم این تن جان سپار من
\\
گفت ز من نه بارها دیده‌ای اعتبارها
&&
بر تو یقین نشد عجب قدرت و کاربار من
\\
گفتم دید دل ولی سیر کجا شود دلی
&&
از لطف و عجایبت ای شه و شهریار من
\\
عشق کشید در زمان گوش مرا به گوشه‌ای
&&
خواند فسون فسون او دام دل شکار من
\\
جان ز فسون او چه شد دم مزن و مگو چه شد
&&
ور بچخی تو نیستی محرم و رازدار من
\\
\end{longtable}
\end{center}
