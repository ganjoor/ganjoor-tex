\begin{center}
\section*{غزل شماره ۲۷۰۵: مرا هر لحظه منزل آسمانی}
\label{sec:2705}
\addcontentsline{toc}{section}{\nameref{sec:2705}}
\begin{longtable}{l p{0.5cm} r}
مرا هر لحظه منزل آسمانی
&&
تو را هر دم خیالی و گمانی
\\
تو گویی کو طمع کرده‌ست در من
&&
جهانی زین خیال اندر زیانی
\\
بر آن چشم دروغت طمع کردم
&&
که چون دوزخ نمودستت جنانی
\\
بر آن عقل خسیست طمع کردم
&&
که جان دادی برای خاکدانی
\\
چه نور افزاید از برق آفتابی
&&
چه بربندد ز ویرانی جهانی
\\
ز یک قطره چه خواهد خورد بحری
&&
ز یک حبه چه دزدد گنج و کانی
\\
چه رونق یا چه آرایش فزاید
&&
ز پژمرده گیایی گلستانی
\\
به حق نور چشم دلبر من
&&
که روشنتر از این نبود نشانی
\\
به حق آن دو لعل قندبارش
&&
که شرح آن نگنجد در دهانی
\\
که مقصودم گشاد سینه‌ای بود
&&
نه طمع آنک بگشایم دکانی
\\
غرض تا نانی آن جا پخته گردد
&&
نه آنک درربایم از تو نانی
\\
ز بهمان و فلان تو فارغ آیند
&&
طمع آن نی که گویندم فلانی
\\
\end{longtable}
\end{center}
