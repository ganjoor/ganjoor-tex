\begin{center}
\section*{غزل شماره ۹۸۵: هر که بهر تو انتظار کند}
\label{sec:0985}
\addcontentsline{toc}{section}{\nameref{sec:0985}}
\begin{longtable}{l p{0.5cm} r}
هر که بهر تو انتظار کند
&&
بخت و اقبال را شکار کند
\\
بهر باران چو کشت منتظر است
&&
سینه را سبز و لاله زار کند
\\
بهر خورشید کان چو منتظر است
&&
سنگ را لعل آبدار کند
\\
انتظار ادیم بهر سهیل
&&
اندر او صد هزار کار کند
\\
آهنی کانتظار صیقل کرد
&&
روی را صاف و بی‌غبار کند
\\
ز انتظار رسول تیغ علی
&&
در غزا خویش ذوالفقار کند
\\
انتظار جنین درون رحم
&&
نطفه را شاه خوش عذار کند
\\
انتظار حبوب زیر زمین
&&
هر یکی دانه را هزار کند
\\
آسیا آب را چو منتظر است
&&
سنگ را چست و بی‌قرار کند
\\
انتظار قبول وحی خدا
&&
چشم را چشم اعتبار کند
\\
انتظار نثار بحر کرم
&&
سینه را درج در چو نار کند
\\
شیره را انتظار در دل خم
&&
بهر مغز شهان عقار کند
\\
بی کنارست فضل منتظرش
&&
رانده را لایق کنار کند
\\
تا قیامت تمام هم نشود
&&
شرح آن کانتظار یار کند
\\
ز انتظارات شمس تبریزی
&&
شمس و ناهید و مه دوار کند
\\
\end{longtable}
\end{center}
