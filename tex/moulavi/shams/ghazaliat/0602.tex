\begin{center}
\section*{غزل شماره ۶۰۲: آن کس که تو را دارد از عیش چه کم دارد}
\label{sec:0602}
\addcontentsline{toc}{section}{\nameref{sec:0602}}
\begin{longtable}{l p{0.5cm} r}
آن کس که تو را دارد از عیش چه کم دارد
&&
وان کس که تو را بیند ای ماه چه غم دارد
\\
از رنگ بلور تو شیرین شده جور تو
&&
هر چند که جور تو بس تند قدم دارد
\\
ای نازش حور از تو وی تابش نور از تو
&&
ای آنک دو صد چون مه شاگرد و حشم دارد
\\
ور خود حشمش نبود خورشید بود تنها
&&
آخر حشم حسنش صد طبل و علم دارد
\\
بس عاشق آشفته آسوده و خوش خفته
&&
در سایه آن زلفی کو حلقه و خم دارد
\\
گفتم به نگار من کز جور مرا مشکن
&&
گفتا به صدف مانی کو در به شکم دارد
\\
تا نشکنی ای شیدا آن در نشود پیدا
&&
آن در بت من باشد یا شکل بتم دارد
\\
شمس الحق تبریزی بر لوح چو پیدا شد
&&
والله که بسی منت بر لوح و قلم دارد
\\
\end{longtable}
\end{center}
