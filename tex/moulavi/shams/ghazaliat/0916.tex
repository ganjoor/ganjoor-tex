\begin{center}
\section*{غزل شماره ۹۱۶: درخت و برگ برآید ز خاک این گوید}
\label{sec:0916}
\addcontentsline{toc}{section}{\nameref{sec:0916}}
\begin{longtable}{l p{0.5cm} r}
درخت و برگ برآید ز خاک این گوید
&&
که خواجه هر چه بکاری تو را همان روید
\\
تو را اگر نفسی ماند جز که عشق مکار
&&
که چیست قیمت مردم هر آنچ می‌جوید
\\
بشو دو دست ز خویش و بیا بخوان بنشین
&&
که آب بهر وی آمد که دست و رو شوید
\\
زهی سلیم که معشوق او به خانه اوست
&&
به سوی خانه نیاید گزاف می‌پوید
\\
به سوی مریم آید دوانه گر عیسیست
&&
وگر خر است بهل تا کمیز خر بوید
\\
کسی که همره ساقیست چون بود هشیار
&&
چرا نباشد لمتر چرا نیفزوید
\\
کسی که کان عسل شد ترش چرا باشد
&&
کسی که مرده ندارد بگو چرا موید
\\
تو را بگویم پنهان که گل چرا خندد
&&
که گلرخیش به کف گیرد و بینبوید
\\
بگو غزل که به صد قرن خلق این خوانند
&&
نسیج را که خدا بافت آن نفرسوید
\\
\end{longtable}
\end{center}
