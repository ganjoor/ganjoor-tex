\begin{center}
\section*{غزل شماره ۲۹۲۹: ای خیالی که به دل می‌گذری}
\label{sec:2929}
\addcontentsline{toc}{section}{\nameref{sec:2929}}
\begin{longtable}{l p{0.5cm} r}
ای خیالی که به دل می‌گذری
&&
نی خیالی نی پری نی بشری
\\
اثر پای تو را می‌جویم
&&
نه زمین و نه فلک می‌سپری
\\
گر ز تو باخبران بی‌خبرند
&&
نه تو از بی‌خبران باخبری
\\
مونس و یار دلی یا تو دلی
&&
تو مقیم نظری یا نظری
\\
ایها الخاطر فی مکرمه
&&
قف زمانا بخداء البصر
\\
لا تعجل به مرور و نوی
&&
بدل اللیل بضؤ السحر
\\
حسن تدبیرک قد صاغ لنا
&&
الهیولی به حسان الصور
\\
گر صور جان و هیولی خرد است
&&
عشق تو دیگر و تو خود دگری
\\
این هیولی پدر صورت‌هاست
&&
ای تو کرده پدران را پدری
\\
نی هیولای همه آبی بود
&&
چه کند آب چو آبش ببری
\\
گر هیولا و صور جان افزاست
&&
دگرم عشوه مده تو دگری
\\
از هیولا است صور ریگ روان
&&
ریگ را هرزه چرا می‌شمری
\\
\end{longtable}
\end{center}
