\begin{center}
\section*{غزل شماره ۱۶۲۳: هذیان که گفت دشمن به درون دل شنیدم}
\label{sec:1623}
\addcontentsline{toc}{section}{\nameref{sec:1623}}
\begin{longtable}{l p{0.5cm} r}
هذیان که گفت دشمن به درون دل شنیدم
&&
پی من تصوری را که بکرد هم بدیدم
\\
سگ او گزید پایم بنمود بس جفایم
&&
نگزم چو سگ من او را لب خویش را گزیدم
\\
چو به رازهای فردان برسیده‌ام چو مردان
&&
چه بدین تفاخر آرم که به راز او رسیدم
\\
همه عیب از من آمد که ز من چنین فن آمد
&&
که به قصد کزدمی را سوی پای خود کشیدم
\\
چو بلیس کو ز آدم بندید جز که نقشی
&&
من از این بلیس ناکس به خدا که نابدیدم
\\
برسان به همدمانم که من از چه روگرانم
&&
چو گزید مار رانم ز سیه رسن رمیدم
\\
خمشان بس خجسته لب و چشم برببسته
&&
ز رهی که کس نداند به ضمیرشان دویدم
\\
چو ز دل به جانب دل ره خفیه است و کامل
&&
ز خزینه‌های دل‌ها زر و نقره برگزیدم
\\
به ضمیر همچو گلخن سگ مرده درفکندم
&&
ز ضمیر همچو گلشن گل و یاسمن بچیدم
\\
بد و نیک دوستان را به کنایت ار بگفتم
&&
به بهینه پرده آن را چو نساج برتنیدم
\\
چو دلم رسید ناگه به دلی عظیم و آگه
&&
ز مهابت دل او به مثال دل طپیدم
\\
چو به حال خویش شادی تو به من کجا فتادی
&&
پس کار خویشتن رو که نه شیخ و نه مریدم
\\
به سوی تو ای برادر نه مسم نه زر سرخم
&&
ز در خودم برون ران که نه قفل و نه کلیدم
\\
تو بگیر آن چنانک بنگفتم این سخن هم
&&
اگرم به یاد بودی به خدا نمی‌چخیدم
\\
\end{longtable}
\end{center}
