\begin{center}
\section*{غزل شماره ۱۳۲۱: ایا هوای تو در جان‌ها سلام علیک}
\label{sec:1321}
\addcontentsline{toc}{section}{\nameref{sec:1321}}
\begin{longtable}{l p{0.5cm} r}
ایا هوای تو در جان‌ها سلام علیک
&&
غلام می‌خری ارزان بها سلام علیک
\\
ایا کسی که هزاران هزار جان و روان
&&
همی‌کشند ز هر سو تو را سلام علیک
\\
به وقت خواندن آن نامه‌های خون آلود
&&
بخوان ز جانب این آشنا سلام علیک
\\
تو می‌خرامی و خورشید و ماه در پی تو
&&
همی‌دوند که‌ای خوش لقا سلام علیک
\\
به خاک پای تو هر دم همی‌کنند پیغام
&&
هزار چشم که ای توتیا سلام علیک
\\
تو تیزگوش تری از همه که هر نفست
&&
ز غیب می‌رسد از انبیا سلام علیک
\\
سلام خشک نباشد خصوص از شاهان
&&
هزار خلعت و هدیه‌ست با سلام علیک
\\
چنانک کرد خداوند در شب معراج
&&
به نور مطلق بر مصطفی سلام علیک
\\
زهی سلام که دارد ز نور دنب دراز
&&
چنین بود چو کند کبریا سلام علیک
\\
گذشت این همه ای دوست ماجرا بشنو
&&
ولیک پیشتر از ماجرا سلام علیک
\\
\end{longtable}
\end{center}
