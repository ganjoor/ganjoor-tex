\begin{center}
\section*{غزل شماره ۳۱۵۵: در غم یار یار بایستی}
\label{sec:3155}
\addcontentsline{toc}{section}{\nameref{sec:3155}}
\begin{longtable}{l p{0.5cm} r}
در غم یار یار بایستی
&&
یا غمم را کنار بایستی
\\
به یکی غم چو جان نخواهم داد
&&
یک چه باشد هزار بایستی
\\
دشمن شادکام بسیارند
&&
دوستی غمگسار بایستی
\\
در فراقند زین سفر یاران
&&
این سفر را قرار بایستی
\\
تا بدانستیی ز دشمن و دوست
&&
زندگانی دوبار بایستی
\\
شیر بیشه میان زنجیرست
&&
شیر در مرغزار بایستی
\\
ماهیان می‌طپند اندر ریگ
&&
چشمه یا جویبار بایستی
\\
بلبل مست سخت مخمورست
&&
گلشن و سبزه زار بایستی
\\
دیده را عبرت نیست زین پرده
&&
دیده اعتبار بایستی
\\
همه گل خواره‌اند این طفلان
&&
مشفقی دایه وار بایستی
\\
ره بر آب حیات می‌نبرند
&&
خضری آبخوار بایستی
\\
دل پشیمان شده‌ست
&&
دل امسال پار بایستی
\\
اندر این شهر قحط خورشیدست
&&
سایه شهریار بایستی
\\
شهر سرگین پرست پر گشته‌ست
&&
مشک نافه تتار بایستی
\\
مشک از پشک کس نمی‌داند
&&
مشک را انتشار بایستی
\\
دولت کودکانه می‌جویند
&&
دولتی بی‌عثار بایستی
\\
چون بمیری بمیرد این هنرت
&&
زین هنرهات عار بایستی
\\
طالب کار و بار بسیارند
&&
طالب کردگار بایستی
\\
مرگ تا در پی‌است روز شبست
&&
شب ما را نهار بایستی
\\
دم معدود اندکی ماندست
&&
نفسی بی‌شمار بایستی
\\
نفس ایزدی ز سوی یمن
&&
بر خلایق نثار بایستی
\\
ملک‌ها ماند و مالکان مردند
&&
ملکت پایدار بایستی
\\
عقل بسته شد و هوا مختار
&&
عقل را اختیار بایستی
\\
هوش‌ها چون مگس در آن دوغست
&&
هوش‌ها هوشیار بایستی
\\
زین چنین دوغ زشت گندیده
&&
پوز دل را حذار بایستی
\\
معده پردوغ و گوش پر ز دروغ
&&
همت الفرار بایستی
\\
گوش‌ها بسته است لب بربند
&&
از خرد گوشوار بایستی
\\
\end{longtable}
\end{center}
