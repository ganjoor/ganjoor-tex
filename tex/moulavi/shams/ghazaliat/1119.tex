\begin{center}
\section*{غزل شماره ۱۱۱۹: کس بی‌کسی نماند می‌دان تو این قدر}
\label{sec:1119}
\addcontentsline{toc}{section}{\nameref{sec:1119}}
\begin{longtable}{l p{0.5cm} r}
کس بی‌کسی نماند می‌دان تو این قدر
&&
گر با یکی نسازی آید یکی دگر
\\
زین خانه گر روم من و خانه تهی کنم
&&
آید یکی دگر چو منی یا ز من بتر
\\
میراث مانده است جهان از هزار قرن
&&
چون شد به زیر خاک پدر شد پسر پدر
\\
تنها نه آدمی حیوان نیز همچنین
&&
ور نی ندیدی تو در آفاق جانور
\\
شب آفتاب اگر برود هم ز بام چرخ
&&
بر جای آفتاب ستاره‌ست یا قمر
\\
گر ترک یک هنر بکند مرد طبع او
&&
مشغول کار دیگر گشت و دگر هنر
\\
زیرا که بر دل همه خلقان موکلیست
&&
بی‌کارشان ندارد و بی‌یار و بی‌سفر
\\
\end{longtable}
\end{center}
