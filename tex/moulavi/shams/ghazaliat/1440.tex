\begin{center}
\section*{غزل شماره ۱۴۴۰: بنه ای سبز خنگ من فراز آسمان‌ها سم}
\label{sec:1440}
\addcontentsline{toc}{section}{\nameref{sec:1440}}
\begin{longtable}{l p{0.5cm} r}
بنه ای سبز خنگ من فراز آسمان‌ها سم
&&
که بنشست آن مه زیبا چو صد تنگ شکر پیشم
\\
روان شد سوی ما کوثر پر از شیر و پر از شکر
&&
بدران مشک سقا را بزن سنگی و بشکن خم
\\
یکی آهوی جان پرور برآمد از بیابانی
&&
که شیر نر ز بیم او زند بر ریگ سوزان دم
\\
همه مستیم ای خواجه به روز عید می ماند
&&
دهل مست و دهلزن مست و بیخود می زند لم لم
\\
درآمد عقل در میدان سر انگشت در دندان
&&
که با سرمست و با حیران چه گفتم من که الهاکم
\\
یکی عاقل میان ما به دارو هم نمی‌یابد
&&
در این زنجیر مجنونان چه مجنون می شود مردم
\\
به نزد من یکی ساغر به از صد خانه پرزر
&&
بریزم بر تن لاغر از آن باده یکی قمقم
\\
میان روزه داران خوش شراب عید در می کش
&&
نه آن مستی که شب آیی ز ترس خلق چون کزدم
\\
بخور بی‌رطل و بی‌کوزه میی کو بشکند روزه
&&
نه ز انگورست و نی شیره نی از طزغو نی از گندم
\\
شرابی نی که درریزی سحر مخمور برخیزی
&&
دروغین است آن باده از آن افتاده کوته دم
\\
دهان بربند و محرم شو به کعبه خامشان می رو
&&
پیاپی اندر این مستی نی اشتر جو و نی جم جم
\\
\end{longtable}
\end{center}
