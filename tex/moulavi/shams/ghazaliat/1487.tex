\begin{center}
\section*{غزل شماره ۱۴۸۷: امروز چنانم که خر از بار ندانم}
\label{sec:1487}
\addcontentsline{toc}{section}{\nameref{sec:1487}}
\begin{longtable}{l p{0.5cm} r}
امروز چنانم که خر از بار ندانم
&&
امروز چنانم که گل از خار ندانم
\\
امروز مرا یار بدان حال ز سر برد
&&
با یار چنانم که خود از یار ندانم
\\
دی باده مرا برد ز مستی به در یار
&&
امروز چه چاره که در از دار ندانم
\\
از خوف و رجا پار دو پر داشت دل من
&&
امروز چنان شد که پر از پار ندانم
\\
از چهره زار چو زرم بود شکایت
&&
رستم ز شکایت چو زر از زار ندانم
\\
از کار جهان کور بود مردم عاشق
&&
اما نه چو من خود که کر از کار ندانم
\\
جولاهه تردامن ما تار بدرید
&&
می گفت ز مستی که تر از تار ندانم
\\
چون چنگم از زمزمه خود خبرم نیست
&&
اسرار همی‌گویم و اسرار ندانم
\\
مانند ترازو و گزم من که به بازار
&&
بازار همی‌سازم و بازار ندانم
\\
در اصبع عشقم چو قلم بیخود و مضطر
&&
طومار نویسم من و طومار ندانم
\\
\end{longtable}
\end{center}
