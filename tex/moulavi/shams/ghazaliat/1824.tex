\begin{center}
\section*{غزل شماره ۱۸۲۴: سیر نمی‌شوم ز تو ای مه جان فزای من}
\label{sec:1824}
\addcontentsline{toc}{section}{\nameref{sec:1824}}
\begin{longtable}{l p{0.5cm} r}
سیر نمی‌شوم ز تو ای مه جان فزای من
&&
جور مکن جفا مکن نیست جفا سزای من
\\
با ستم و جفا خوشم گر چه درون آتشم
&&
چونک تو سایه افکنی بر سرم ای همای من
\\
چونک کند شکرفشان عشق برای سرخوشان
&&
نرخ نبات بشکند چاشنی بلای من
\\
عود دمد ز دود من کور شود حسود من
&&
زفت شود وجود من تنگ شود قبای من
\\
آن نفس این زمین بود چرخ زنان چو آسمان
&&
ذره به ذره رقص در نعره زنان که‌های من
\\
آمد دی خیال تو گفت مرا که غم مخور
&&
گفتم غم نمی‌خورم ای غم تو دوای من
\\
گفت که غم غلام تو هر دو جهان به کام تو
&&
لیک ز هر دو دور شو از جهت لقای من
\\
گفتم چون اجل رسد جان بجهد از این جسد
&&
گر بروم به سوی جان باد شکسته پای من
\\
گفت بلی به گل نگر چون ببرد قضا سرش
&&
خنده زنان سری نهد در قدم قضای من
\\
گفتم اگر ترش شوم از پی رشک می شوم
&&
تا نرسد به چشم بد کر و فر ولای من
\\
گفت که چشم بد بهل کو نخورد جز آب و گل
&&
چشم بدان کجا رسد جانب کبریای من
\\
گفتم روزکی دو سه مانده‌ام در آب و گل
&&
بسته خوفم و رجا تا برسد صلای من
\\
گفت در آب و گل نه‌ای سایه توست این طرف
&&
برد تو را از این جهان صنعت جان ربای من
\\
زینچ بگفت دلبرم عقل پرید از سرم
&&
باقی قصه عقل کل بو نبرد چه جای من
\\
\end{longtable}
\end{center}
