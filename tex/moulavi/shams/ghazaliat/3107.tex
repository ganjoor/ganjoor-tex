\begin{center}
\section*{غزل شماره ۳۱۰۷: میان تیرگی خواب و نور بیداری}
\label{sec:3107}
\addcontentsline{toc}{section}{\nameref{sec:3107}}
\begin{longtable}{l p{0.5cm} r}
میان تیرگی خواب و نور بیداری
&&
چنان نمود مرا دوش در شب تاری
\\
که خوب طلعتی از ساکنان حضرت قدس
&&
که جمله محض خرد بود و نور هشیاری
\\
تنش چو روی مقدس بری ز کسوت جسم
&&
چو عقل و جان گهردار، وز غرض عاری
\\
مرا ستایش بسیار کرد و گفت:« ای آن
&&
که در جحیم طبیعت چنین گرفتاری
\\
شکفته گلبن جوزا برای عشرت تست
&&
تو سر به گلخن گیتی چرا فرود آری
\\
سریر هفت فلک تخت تست اگرچه کنون
&&
ز دست طبع، گرفتار چار دیواری
\\
کمال جان چو بهایم ز خواب و خور مطلب
&&
که آفریده تو زین‌سان نه بهر این کاری
\\
بدی مکن که درین کشت زار زود زوال
&&
به داس دهر همان بدروی که می‌کاری
\\
پی مراد چه پویی به عالمی که درو
&&
چو دفع رنج کنی جمله راحت انگاری؟!
\\
حقیقت این شکم از آزپر نخواهد شد
&&
اگر به ملک همه عالمش بینباری
\\
گرفتمست که رسیدی بدانچ می‌طلبی
&&
ولی چه سود ازان، چون بجاش بگذاری؟!
\\
شب جوانیت ای دوست چون سپیده دمید
&&
تو مست، خفته و آگه نه‌ای ز بیداری
\\
\end{longtable}
\end{center}
