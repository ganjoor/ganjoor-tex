\begin{center}
\section*{غزل شماره ۲۱۳۸: ای عشق تو موزونتری یا باغ و سیبستان تو}
\label{sec:2138}
\addcontentsline{toc}{section}{\nameref{sec:2138}}
\begin{longtable}{l p{0.5cm} r}
ای عشق تو موزونتری یا باغ و سیبستان تو
&&
چرخی بزن ای ماه تو جان بخش مشتاقان تو
\\
تلخی ز تو شیرین شود کفر و ضلالت دین شود
&&
خار خسک نسرین شود صد جان فدای جان تو
\\
در آسمان درها نهی در آدمی پرها نهی
&&
صد شور در سرها نهی ای خلق سرگردان تو
\\
عشقا چه شیرین خوستی عشقا چه گلگون روستی
&&
عشقا چه عشرت دوستی ای شادی اقران تو
\\
ای بر شقایق رنگ تو جمله حقایق دنگ تو
&&
هر ذره را آهنگ تو در مطمع احسان تو
\\
بی‌تو همه بازارها پژمرده اندر کارها
&&
باغ و رز و گلزارها مستسقی باران تو
\\
رقص از تو آموزد شجر پا با تو کوبد شاخ تر
&&
مستی کند برگ و ثمر بر چشمه حیوان تو
\\
گر باغ خواهد ارمغان از نوبهار بی‌خزان
&&
تا برفشاند برگ خود بر باد گل افشان تو
\\
از اختران آسمان از ثابت و از سایره
&&
عار آید آن استاره را کو تافت بر کیوان تو
\\
ای خوش منادی‌های تو در باغ شادی‌های تو
&&
بر جای نان شادی خورد جانی که شد مهمان تو
\\
من آزمودم مدتی بی‌تو ندارم لذتی
&&
کی عمر را لذت بود بی‌ملح بی‌پایان تو
\\
رفتم سفر بازآمدم ز آخر به آغاز آمدم
&&
در خواب دید این پیل جان صحرای هندستان تو
\\
صحرای هندستان تو میدان سرمستان تو
&&
بکران آبستان تو از لذت دستان تو
\\
سودم نشد تدبیرها بسکست دل زنجیرها
&&
آورد جان را کشکشان تا پیش شادروان تو
\\
آن جا نبینم ماردی آن جا نبینم باردی
&&
هر دم حیاتی واردی از بخشش ارزان تو
\\
ای کوه از حلمت خجل وز حلم تو گستاخ دل
&&
تا درجهد دیوانه‌ای گستاخ در ایوان تو
\\
از بس که بگشادی تو در در آهن و کوه و حجر
&&
چون مور شد دل رخنه جو در طشت و در پنگان تو
\\
گر تا قیامت بشمرم در شرح رویت قاصرم
&&
پیموده کی تاند شدن ز اسکره عمان تو
\\
\end{longtable}
\end{center}
