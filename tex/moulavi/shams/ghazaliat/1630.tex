\begin{center}
\section*{غزل شماره ۱۶۳۰: گر مرا خار زند آن گل خندان بکشم}
\label{sec:1630}
\addcontentsline{toc}{section}{\nameref{sec:1630}}
\begin{longtable}{l p{0.5cm} r}
گر مرا خار زند آن گل خندان بکشم
&&
ور لبش جور کند از بن دندان بکشم
\\
ور بسوزد دل مسکین مرا همچو سپند
&&
پای کوبان شوم و سوز سپندان بکشم
\\
گر سر زلف چو چوگانش مرا دور کند
&&
همچنین سجده کنان تا بن میدان بکشم
\\
لعل در کوه بود گوهر در قلزم تلخ
&&
از پی لعل و گهر این بخورم آن بکشم
\\
این نبوده‌ست و نباشد که من از طنز و گزاف
&&
گهر از ره ببرم لعل بدخشان بکشم
\\
رخم از خون جگر صدره اطلس پوشید
&&
چه شود گر ز خطا خلعت سلطان بکشم
\\
من چو در سایه آن زلف پریشان جمعم
&&
لازمم نیست که من راه پریشان بکشم
\\
همرهانم همه رفتند سوی رهزن دل
&&
بگشایید رهم تا سوی ایشان بکشم
\\
گر کسی قصه کند بارکشی مجنونی
&&
از درون نعره زند دل که دو چندان بکشم
\\
ور به زندان بردم یوسف من بی‌گنهی
&&
همچو یوسف بروم وحشت زندان بکشم
\\
گر دلم سر کشد از درد تو جان سیر شود
&&
جان و دل تا برود بی‌دل و بی‌جان بکشم
\\
شور و شر در دو جهان افتد از عنبر و مشک
&&
چونک من دامن مشکین تو پنهان بکشم
\\
\end{longtable}
\end{center}
