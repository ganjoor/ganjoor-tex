\begin{center}
\section*{غزل شماره ۴۹۳: تو مردی و نظرت در جهان جان نگریست}
\label{sec:0493}
\addcontentsline{toc}{section}{\nameref{sec:0493}}
\begin{longtable}{l p{0.5cm} r}
تو مردی و نظرت در جهان جان نگریست
&&
چو باز زنده شدی زین سپس بدانی زیست
\\
هر آن کسی که چو ادریس مرد و بازآمد
&&
مدرس ملکوتست و بر غیوب حفیست
\\
بیا بگو به کدامین ره از جهان رفتی
&&
و زان طرف به کدامین ره آمدی که خفیست
\\
رهی که جمله جان‌ها به هر شبی بپرند
&&
که شهر شهر قفس‌ها به شب ز مرغ تهیست
\\
چو مرغ پای ببسته‌ست دور می‌نپرد
&&
به چرخ می‌نرسد وز دوار او عجمیست
\\
علاقه را چو ببرد به مرگ و بازپرد
&&
حقیقت و سر هر چیز را ببیند چیست
\\
خموش باش که پرست عالم خمشی
&&
مکوب طبل مقالت که گفت طبل تهیست
\\
\end{longtable}
\end{center}
