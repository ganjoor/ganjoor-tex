\begin{center}
\section*{غزل شماره ۸۸۷: روبهکی دنبه برد شیر مگر خفته بود}
\label{sec:0887}
\addcontentsline{toc}{section}{\nameref{sec:0887}}
\begin{longtable}{l p{0.5cm} r}
روبهکی دنبه برد شیر مگر خفته بود
&&
جان نبرد خود ز شیر روبه کور و کبود
\\
قاصد ره داد شیر ور نه کی باور کند
&&
این چه که روباه لنگ دنبه ز شیری ربود
\\
گوید گرگی بخورد یوسف یعقوب را
&&
شیر فلک هم بر او پنجه نیارد گشود
\\
هر نفس الهام حق حارس دل‌های ماست
&&
از دل ما کی برد میمنه دیو حسود
\\
دست حق آمد دراز با کف حق کژ مباز
&&
در ره حق هر کی کاشت دانه جو جو درود
\\
هر که تو را کرد خوار رو به خدایش سپار
&&
هر کی بترساندت روی به حق آر زود
\\
غصه و ترس و بلا هست کمند خدا
&&
گوش کشان آردت رنج به درگاه جود
\\
یارب و یارب کنان روی سوی آسمان
&&
آب ز دیده روان بر رخ زردت چو رود
\\
سبزه دمیده ز آب بر دل و جان خراب
&&
صبح گشاده نقاب ذلک یوم الخلود
\\
گر سر فرعون را درد بدی و بلا
&&
لاف خدایی کجا دردهدی آن عنود
\\
چون دم غرقش رسید گفت اقل العبید
&&
کفر شد ایمان و دید چونک بلا رو نمود
\\
رنج ز تن برمدار در تک نیلش درآر
&&
تا تن فرعون وار پاک شود از جحود
\\
نفس به مصرست امیر در تک نیلست اسیر
&&
باش بر او جبرئیل دود برآور ز عود
\\
عود بخیلست او بو نرساند به تو
&&
راز نخواهد گشا تا نکشد نار و دود
\\
مفخر تبریز گفت شمس حق و دین نهفت
&&
رو ترش از توست عشق سرکه نشاید فزود
\\
\end{longtable}
\end{center}
