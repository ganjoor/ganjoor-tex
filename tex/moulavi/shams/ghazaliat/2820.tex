\begin{center}
\section*{غزل شماره ۲۸۲۰: چو به شهر تو رسیدم تو ز من گوشه گزیدی}
\label{sec:2820}
\addcontentsline{toc}{section}{\nameref{sec:2820}}
\begin{longtable}{l p{0.5cm} r}
چو به شهر تو رسیدم تو ز من گوشه گزیدی
&&
چو ز شهر تو برفتم به وداعیم ندیدی
\\
تو اگر لطف گزینی و اگر بر سر کینی
&&
همه آسایش جانی همه آرایش عیدی
\\
سبب غیرت توست آنک نهانی و اگر نی
&&
همه خورشید عیانی که ز هر ذره پدیدی
\\
تو اگر گوشه بگیری تو جگرگوشه و میری
&&
و اگر پرده دری تو همه را پرده دریدی
\\
دل کفر از تو مشوش سر ایمان به میت خوش
&&
همه را هوش ربودی همه را گوش کشیدی
\\
همه گل‌ها گرو دی همه سرها گرو می
&&
تو هم این را و هم آن را ز کف مرگ خریدی
\\
چو وفا نبود در گل چو رهی نیست سوی کل
&&
همه بر توست توکل که عمادی و عمیدی
\\
اگر از چهره یوسف نفری کف ببریدند
&&
تو دو صد یوسف جان را ز دل و عقل بریدی
\\
ز پلیدی و ز خونی تو کنی صورت شخصی
&&
که گریزد به دو فرسنگ وی از بوی پلیدی
\\
کنیش طعمه خاکی که شود سبزه پاکی
&&
برهد او ز نجاست چو در او روح دمیدی
\\
هله ای دل به سما رو به چراگاه خدا رو
&&
به چراگاه ستوران چو یکی چند چریدی
\\
تو همه طمع بر آن نه که در او نیست امیدت
&&
که ز نومیدی اول تو بدین سوی رسیدی
\\
تو خمش کن که خداوند سخن بخش بگوید
&&
که همو ساخت در قفل و همو کرد کلیدی
\\
\end{longtable}
\end{center}
