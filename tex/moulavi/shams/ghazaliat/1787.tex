\begin{center}
\section*{غزل شماره ۱۷۸۷: گر آخر آمد عشق تو گردد ز اول‌ها فزون}
\label{sec:1787}
\addcontentsline{toc}{section}{\nameref{sec:1787}}
\begin{longtable}{l p{0.5cm} r}
گر آخر آمد عشق تو گردد ز اول‌ها فزون
&&
بنوشت توقیعت خدا کاخرون السابقون
\\
زرین شده طغرای او ز انا فتحناهای او
&&
سر کرده صورت‌های او از بحر جان آبگون
\\
آدم دگربار آمده بر تخت دین تکیه زده
&&
در سجده شکر آمده سرهای نحن الصافون
\\
رستم که باشد در جهان در پیش صف عاشقان
&&
شبدیز می رانند خوش هر روز در دریای خون
\\
هر سو دو صد ببریده سر در بحر خون زان کر و فر
&&
رقصان و خندان چون شکر ز انا الیه راجعون
\\
گر سایه عاشق فتد بر کوه سنگین برجهد
&&
نه چرخ صدق‌ها زند تو منکری نک آزمون
\\
بر کوه زد اشراق او بشنو تو چاقاچاق او
&&
خود کوه مسکین که بود آن جا که شد موسی زبون
\\
خود پیش موسی آسمان باشد کمینه نردبان
&&
کو آسمان کو ریسمان کو جان کو دنیای دون
\\
تن را تو مشتی کاه دان در زیر او دریای جان
&&
گر چه ز بیرون ذره‌ای صد آفتابی از درون
\\
خورشیدی و زرین طبق دیگ تو را پخته است حق
&&
مطلوب بودی در سبق طالب شدستی تو کنون
\\
او پار کشتی کاشته امسال برگ افراشته
&&
سر از زمین برداشته بر خویش می خواند فسون
\\
جان مست گشت از کاس او ای شاد کاس و طاس او
&&
طاسی که بهر سجده‌اش شد طشت گردون سرنگون
\\
ای شمس تبریز از کرم ای رشک فردوس و ارم
&&
تا چنگ اندر من زدی در عشق گشتم ارغنون
\\
\end{longtable}
\end{center}
