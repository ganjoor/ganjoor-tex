\begin{center}
\section*{غزل شماره ۳۰۴۱: اگر مرا تو ندانی بپرس از شب تاری}
\label{sec:3041}
\addcontentsline{toc}{section}{\nameref{sec:3041}}
\begin{longtable}{l p{0.5cm} r}
اگر مرا تو ندانی بپرس از شب تاری
&&
شبست محرم عاشق گواه ناله و زاری
\\
چه جای شب که هزاران نشانه دارد عاشق
&&
کمینه اشک و رخ زرد و لاغری و نزاری
\\
چو ابر ساعت گریه چو کوه وقت تحمل
&&
چو آب سجده کنان و چو خاک راه به خواری
\\
ولیک این همه محنت به گرد باغ چو خاری
&&
درون باغ گلستان و یار و چشمه جاری
\\
چو بگذری تو ز دیوار باغ و در چمن آیی
&&
زبان شکر گزاری سجود شکر بیاری
\\
که شکر و حمد خدا را که برد جور خزان را
&&
شکفته گشت زمین و بهار کرد بهاری
\\
هزار شاخ برهنه قرین حله گل شد
&&
هزار خار مغیلان رهیده گشت ز خاری
\\
حلاوت غم معشوق را چه داند عاقل
&&
چو جوله‌ست نداند طریق جنگ و سواری
\\
برادر و پدر و مادر تو عشاقند
&&
که جمله یک شده‌اند و سرشته‌اند ز یاری
\\
نمک شود چو درافتد هزار تن به نمکدان
&&
دوی نماند در تن چه مرغزی چه بخاری
\\
مکش عنان سخن را به کودنی ملولان
&&
تو تشنگان ملک بین به وقت حرف گزاری
\\
\end{longtable}
\end{center}
