\begin{center}
\section*{غزل شماره ۱۳۵۰: چشم تو با چشم من هر دم بی‌قیل و قال}
\label{sec:1350}
\addcontentsline{toc}{section}{\nameref{sec:1350}}
\begin{longtable}{l p{0.5cm} r}
چشم تو با چشم من هر دم بی‌قیل و قال
&&
دارد در درس عشق بحث و جواب و سؤال
\\
گاه کند لاغرم همچو لب ساغرم
&&
گاه کند فربهم تا نروم در جوال
\\
چون کشدم سوی طوی من بکشم گوش شیر
&&
چونک نهان کرد روی ناله کنم از شغال
\\
چون نگرم سوی نقش گوید ای بت پرست
&&
چشم نهم سوی مال او دهدم گوشمال
\\
گویمش ای آفتاب بر همه دل‌ها بتاب
&&
جمله جهان ذره‌ها نور خوشت را عیال
\\
سر بزن ای آفتاب از پس کوه سحاب
&&
هر نظری را نما بی‌سخنی شرح حال
\\
بازمگیر آب پاک از جگر شوره خاک
&&
منع مکن از جلال پرتو نور جلال
\\
جلوه چو شد نور ما آن ملک نورها
&&
نور شود جمله روح عقل شود بی‌عقال
\\
ای که میش خورده‌ای از چه تو پژمرده‌ای
&&
باغ رخش دیده‌ای باز گشا پر و بال
\\
باز سرم گشت مست هیچ مگو دست دست
&&
باقی این بایدت رو شب و فردا تعال
\\
\end{longtable}
\end{center}
