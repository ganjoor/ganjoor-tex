\begin{center}
\section*{غزل شماره ۲۶۲۶: ای دل تو در این غارت و تاراج چه دیدی}
\label{sec:2626}
\addcontentsline{toc}{section}{\nameref{sec:2626}}
\begin{longtable}{l p{0.5cm} r}
ای دل تو در این غارت و تاراج چه دیدی
&&
تا رخت گشادی و دکان بازکشیدی
\\
چون جولهه حرص در این خانه ویران
&&
از آب دهان دام مگس گیر تنیدی
\\
از لذت و از مستی این دانه دنیا
&&
پنداشت دل تو که از این دام رهیدی
\\
در سیل کسی خانه کند از گل و از خاک
&&
در دام کسی دانه خورد هیچ شنیدی
\\
ای دل ببر از دام و برون جه تو به هنگام
&&
آن سوی که در روضه ارواح دویدی
\\
ای روح چو طاووس بیفشان تو پر عقل
&&
یا یاد نداری تو که بر عرش پریدی
\\
از عرش سوی فرش فتادی و قضا بود
&&
دادی تو پر خویش و دو سه دانه خریدی
\\
چون گرسنه قحط در این لقمه فتادی
&&
گه لب بگزیدی و گهی دست خلیدی
\\
کو همت شاهانه نه زان دایه دولت
&&
زان شیر تباشیر سعادت بمزیدی
\\
آن خوی ملوکانه که با شیر فرورفت
&&
والله که نیامیزد با خون و پلیدی
\\
آن شاه گل ما به کف خویش سرشته‌ست
&&
آن همت و بختش ز کف شاه چشیدی
\\
والله که در آن زاویه کاوراد الست است
&&
آموخت تو را شاه تو شیخی و مریدی
\\
آموخت تو را که دل و دلدار یکی اند
&&
گه قفل شود گاه کند رسم کلیدی
\\
گه پند و گهی بند و گهی زهر و گهی قند
&&
گه تازه و برجسته گهی کهنه قدیدی
\\
ای سیل در این راه تو بالا و نشیب است
&&
تلوین برود از تو چو در بحر رسیدی
\\
ای خاک از این زخم پیاپی تو نژندی
&&
وی چرخ از این بار گران سنگ خمیدی
\\
ای بحر حقایق که زمین موج و کف توست
&&
پنهانی و در فعل چه پیدا و پدیدی
\\
ای چشمه خورشید که جوشیدی از آن بحر
&&
تا پرده ظلمات به انوار دریدی
\\
هر خاک که در دست گرفتی همه زر شد
&&
شد لعل و زمرد ز تو سنگی که گزیدی
\\
بس تلخ و ترش از تو چو حلوا و شکر شد
&&
بگزیده شد آن میوه که او را بگزیدی
\\
شاگرد کی بودی که تو استاد جهانی
&&
این صنعت بی‌آلت و بی‌کف ز کی دیدی
\\
چون مرکب جبریلی و از سم تو هر خاک
&&
سبزه شود آخر ز چه کهسار چریدی
\\
خامش کن و یاد آور آن را که به حضرت
&&
صد بار از این ذکر و از این فکر بریدی
\\
\end{longtable}
\end{center}
