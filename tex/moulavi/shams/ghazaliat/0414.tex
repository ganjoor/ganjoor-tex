\begin{center}
\section*{غزل شماره ۴۱۴: روز و شب خدمت تو بی‌سر و بی‌پا چه خوشست}
\label{sec:0414}
\addcontentsline{toc}{section}{\nameref{sec:0414}}
\begin{longtable}{l p{0.5cm} r}
روز و شب خدمت تو بی‌سر و بی‌پا چه خوشست
&&
در شکرخانه تو مرغ شکرخا چه خوشست
\\
بر سر غنچه بسته که نهان می‌خندد
&&
سایه سرو خوش نادره بالا چه خوشست
\\
زاغ اگر عاشق سرگین خر آمد گو باش
&&
بلبلان را به چمن با گل رعنا چه خوشست
\\
بانک سرنای چه گر مونس غمگینان‌ست
&&
از دم روح نفخنا دل سرنا چه خوشست
\\
گر چه شب بازرهد خلق ز اندیشه به خواب
&&
در رخ شمس ضحی دیده بینا چه خوشست
\\
بت پرستانه تو را پای فرورفت به گل
&&
تو چه دانی که بر این گنبد مینا چه خوشست
\\
چون تجلی بود از رحمت حق موسی را
&&
زان شکرریز لقا سینه سینا چه خوشست
\\
که صدا دارد و در کان زر صامت هم هست
&&
گه خمش بودن و گه گفت مواسا چه خوشست
\\
\end{longtable}
\end{center}
