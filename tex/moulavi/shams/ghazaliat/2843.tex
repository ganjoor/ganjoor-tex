\begin{center}
\section*{غزل شماره ۲۸۴۳: هله ای پری شب رو که ز خلق ناپدیدی}
\label{sec:2843}
\addcontentsline{toc}{section}{\nameref{sec:2843}}
\begin{longtable}{l p{0.5cm} r}
هله ای پری شب رو که ز خلق ناپدیدی
&&
به خدا به هیچ خانه تو چنین چراغ دیدی
\\
نه ز بادها بمیرد نه ز نم کمی پذیرد
&&
نه ز روزگار گیرد کهنی و یا قدیدی
\\
هله آسمان عالی ز تو خوش همه حوالی
&&
سفری دراز کردی به مسافران رسیدی
\\
تو بگو وگر نگویی به خدا که من بگویم
&&
که چرا ستارگان را سوی کهکشان کشیدی
\\
سخنی ز نسر طایر طلبیدم از ضمایر
&&
که عجب در آن چمن‌ها که ملک بود پریدی
\\
بزد آه سرد و گفتا که بر آن در است قفلی
&&
که به جز عنایت شه نکند برو کلیدی
\\
چو فغان او شنیدم سوی عشق بنگریدم
&&
که چو نیستت سر او دل او چرا خلیدی
\\
به جواب گفت عشقم که مکن تو باور او را
&&
که درونه گنج دارد تو چه مکر او خریدی
\\
چو شنیدم این بگفتم تو عجبتری و یا او
&&
که هزار جوحی این جا نکند به جز مریدی
\\
هله عشق عاشقان را و مسافران جان را
&&
خوش و نوش و شادمان کن که هزار روز عیدی
\\
تو چو یوسف جمالی که ز ناز لاابالی
&&
به درآمدی و حالی کف عاشقان گزیدی
\\
خمش ار چه داد داری طرب و گشاد داری
&&
به چنین گشاد گویی که روان بایزیدی
\\
\end{longtable}
\end{center}
