\begin{center}
\section*{غزل شماره ۱۹۹۶: چون خیال تو درآید به دلم رقص کنان}
\label{sec:1996}
\addcontentsline{toc}{section}{\nameref{sec:1996}}
\begin{longtable}{l p{0.5cm} r}
چون خیال تو درآید به دلم رقص کنان
&&
چه خیالات دگر مست درآید به میان
\\
گرد بر گرد خیالش همه در رقص شوند
&&
وان خیال چو مه تو به میان چرخ زنان
\\
هر خیالی که در آن دم به تو آسیب زند
&&
همچو آیینه ز خورشید برآید لمعان
\\
سخنم مست شود از صفتی و صد بار
&&
از زبانم به دلم آید و از دل به زبان
\\
سخنم مست و دلم مست و خیالات تو مست
&&
همه بر همدگر افتاده و در هم نگران
\\
همه بر همدگر از بس که بمالند دهن
&&
آن خیالات به هم درشکند او ز فغان
\\
همه چون دانه انگور و دلم چون چرش است
&&
همه چون برگ گلاب و دل من همچو دکان
\\
ز صلاح دل و دین زر برم و زر کوبم
&&
تا مفرح شود آن را که بود دیده جان
\\
\end{longtable}
\end{center}
