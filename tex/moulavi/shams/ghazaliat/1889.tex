\begin{center}
\section*{غزل شماره ۱۸۸۹: دل دل دل تو دل مرا مرنجان}
\label{sec:1889}
\addcontentsline{toc}{section}{\nameref{sec:1889}}
\begin{longtable}{l p{0.5cm} r}
دل دل دل تو دل مرا مرنجان
&&
چرا چرا چه معنی مرا کنی پریشان
\\
بیا بیا و بازآ به صلح سوی خانه
&&
مرو مرو ز پیشم کتف چنین مجنبان
\\
تو صد شکرستانی ترش چه کردی ابرو
&&
سبکتر از صبایی چرا شوی گران جان
\\
منم کنون ز عشق رخ چو گلشن تو
&&
فراز سرو و گلشن چو صد هزاردستان
\\
بیا بیا دمم ده که دمدمه لطیفت
&&
حیات دل فزاید مرا چو آب حیوان
\\
بیار عشوه اینک بهای عشوه صد جان
&&
هزار جان به ارزد زهی متاع ارزان
\\
تو عقل عقل مایی چرا ز ما جدایی
&&
سری که عقل از او شد نه گیج ماند و حیران
\\
ستون این سرایی ز در برون چرایی
&&
سرا که بی‌ستون شد نه پست گشت ویران
\\
تو ماه آسمانی و ما شبیم تاری
&&
شبی که مه نباشد غلس بود فراوان
\\
تو پادشاه شهری و ما کنار شهری
&&
چو شهر ماند بی‌شه چه سر بود چه سامان
\\
مها تویی سلیمان فراق و غم چو دیوان
&&
چو دور شد سلیمان نه دست یافت شیطان
\\
تویی به جای موسی و ما تو را عصایی
&&
بجز به کف موسی عصا نیافت برهان
\\
مسیح خوش دمی تو و ما ز گل چو مرغی
&&
دمی بدم تو بر ما بر اوج بین تو جولان
\\
تو نوح روزگاری و ما چو اهل کشتی
&&
چو نوح رفت کشتی کجا رهد ز طوفان
\\
تویی خلیل ای جان همه جهان پرآتش
&&
که بی‌خلیل آتش نمی‌شود گلستان
\\
تو نور مصطفایی و کعبه پربتان شد
&&
هلا بیا برون کن بتان ز بیت رحمان
\\
تو یوسف جمالی و چشم خلق بسته
&&
نظر ز تو گشاید چو چشم پیر کنعان
\\
تو گوهر صفایی و ما صدف به گردت
&&
صدف چه قیمت آرد چو رفت گوهر کان
\\
تو جان آفتابی که او است جان عالم
&&
سزد گرت بگویم که جان جان کیهان
\\
به غیب باشد ایمان تو غیب را عیانی
&&
که عین عین عینی و اصل اصل ایمان
\\
خمش که تا قیامت اگر دهی علامت
&&
جوی نموده باشی به ما ز گنج پنهان
\\
\end{longtable}
\end{center}
