\begin{center}
\section*{غزل شماره ۲۵۲۹: ایا نزدیک جان و دل چنین دوری روا داری}
\label{sec:2529}
\addcontentsline{toc}{section}{\nameref{sec:2529}}
\begin{longtable}{l p{0.5cm} r}
ایا نزدیک جان و دل چنین دوری روا داری
&&
به جانی کز وصالت زاد مهجوری روا داری
\\
گرفتم دانه تلخم نشاید کشت و خوردن را
&&
تو با آن لطف شیرین کار این شوری روا داری
\\
تو آن نوری که دوزخ را به آب خود بمیرانی
&&
مرا در دل چنین سوزی و محروری روا داری
\\
اگر در جنت وصلت چو آدم گندمی خوردم
&&
مرا بی‌حله وصلت بدین عوری روا داری
\\
مرا در معرکه هجران میان خون و زخم جان
&&
مثال لشکر خوارزم با غوری روا داری
\\
مرا گفتی تو مغفوری قبول قبله نوری
&&
چنین تعذیب بعد از عفو و مغفوری روا داری
\\
مها چشمی که او روزی بدید آن چشم پرنورت
&&
به زخم چشم بدخواهان در او کوری روا داری
\\
جهان عشق را اکنون سلیمان بن داوودی
&&
معاذالله که آزار یکی موری روا داری
\\
تو آن شمسی که نور تو محیط نورها گشته‌ست
&&
سوی تبریز واگردی و مستوری روا داری
\\
\end{longtable}
\end{center}
