\begin{center}
\section*{غزل شماره ۲۷۳۳: ای وصل تو آب زندگانی}
\label{sec:2733}
\addcontentsline{toc}{section}{\nameref{sec:2733}}
\begin{longtable}{l p{0.5cm} r}
ای وصل تو آب زندگانی
&&
تدبیر خلاص ما تو دانی
\\
از دیده برون مشو که نوری
&&
وز سینه جدا مشو که جانی
\\
آن دم که نهان شوی ز چشمم
&&
می‌نالد جان من نهانی
\\
من خود چه کسم که وصل جویم
&&
از لطف توم همی‌کشانی
\\
ای دل تو مرو سوی خرابات
&&
هر چند قلندر جهانی
\\
کان جا همه پاکباز باشند
&&
ترسم که تو کم زنی بمانی
\\
ور ز آنک روی مرو تو با خویش
&&
درپوش نشان بی‌نشانی
\\
مانند سپر مپوش سینه
&&
گر عاشق تیر آن کمانی
\\
پرسید یکی که عاشقی چیست
&&
گفتم که مپرس از این معانی
\\
آنگه که چو من شوی ببینی
&&
آنگه که بخواندت به خوانی
\\
مردانه درآ چو شیرمردی
&&
دل را چو زنان چه می‌طپانی
\\
ای از رخ گلرخان غیبت
&&
گشته رخ سرخ زعفرانی
\\
ای از هوس بهار حسنت
&&
در هر نفسم دم خزانی
\\
ای آنک تو باغ و بوستان را
&&
از جور خزان همی‌رهانی
\\
ای داده تو گوشت پاره‌ای را
&&
در گفت و شنود ترجمانی
\\
ای داده زبان انبیا را
&&
با سر قدیم همزبانی
\\
ای داده روان اولیا را
&&
در مرگ حیات جاودانی
\\
ای داده تو عقل بدگمان را
&&
بر بام دماغ پاسبانی
\\
ای آنک تو هر شبی ز خلقان
&&
این پنج چراغ می‌ستانی
\\
ای داده تو چشم گلرخان را
&&
مخموری و سحر و دلستانی
\\
ای داده دو قطره خون دل را
&&
اندیشه و فکر و خرده دانی
\\
ای داده تو عشق را به قدرت
&&
مردی و نری و پهلوانی
\\
این بود نصیحت سنایی
&&
جان باز چو طالب عیانی
\\
شمس تبریز نور محضی
&&
زیرا که چراغ آسمانی
\\
\end{longtable}
\end{center}
