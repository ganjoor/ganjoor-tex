\begin{center}
\section*{غزل شماره ۱۱۳۷: شدست نور محمد هزار شاخ هزار}
\label{sec:1137}
\addcontentsline{toc}{section}{\nameref{sec:1137}}
\begin{longtable}{l p{0.5cm} r}
شدست نور محمد هزار شاخ هزار
&&
گرفته هر دو جهان از کنار تا به کنار
\\
اگر حجاب بدرد محمد از یک شاخ
&&
هزار راهب و قسیس بردرد زنار
\\
تو را اگر سر کارست روزگار مبر
&&
شکار شو نفسی و دمی بگیر شکار
\\
تو را سعادت بادا که ما ز دست شدیم
&&
ز دست رفتن این بار نیست چون هر بار
\\
پریر یار مرا گفت کاین جهان بلاست
&&
بگفتمش که ولیکن نه چون تو بی‌زنهار
\\
جواب داد تو باری چرا زنی تشنیع
&&
که پات خار ندید و سرت نیافت خمار
\\
بگفتمش که بلی لیک هم مگیر مرا
&&
نیاحتی که کنم وفق نوحه اغیار
\\
چو میرخوان توام ترش بنهم و شیرین
&&
که هر کسی بخورد بای خود ز خوان کبار
\\
به سوزنی که دهان‌ها بدوخت در رمضان
&&
بیا بدوز دهانم که سیرم از گفتار
\\
ولی چو جمله دهانم کدام را دوزی
&&
نیم چو سوزن کو را بود یکی سوفار
\\
خیار امت محتاج شمس تبریزند
&&
شکافت خربزه زین غم چه جای خیر و خیار
\\
\end{longtable}
\end{center}
