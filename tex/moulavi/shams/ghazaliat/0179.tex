\begin{center}
\section*{غزل شماره ۱۷۹: گر تو عودی سوی این مجمر بیا}
\label{sec:0179}
\addcontentsline{toc}{section}{\nameref{sec:0179}}
\begin{longtable}{l p{0.5cm} r}
گر تو عودی سوی این مجمر بیا
&&
ور برانندت ز بام از در بیا
\\
یوسفی از چاه و زندان چاره نیست
&&
سوی زهر قهر چون شکر بیا
\\
گفتنت الله اکبر رسمی است
&&
گر تو آن اکبری اکبر بیا
\\
چون می احمر سگان هم می‌خورند
&&
گر تو شیری چون می احمر بیا
\\
زر چه جویی مس خود را زر بساز
&&
گر نباشد زر تو سیمین بر بیا
\\
اغنیا خشک و فقیران چشم تر
&&
عاشقا بی‌شکل خشک و تر بیا
\\
گر صفت‌های ملک را محرمی
&&
چون ملک بی‌ماده و بی‌نر بیا
\\
ور صفات دل گرفتی در سفر
&&
همچو دل بی‌پا بیا بی‌سر بیا
\\
چون لب لعلش صلایی می‌دهد
&&
گر نه‌ای چون خاره و مرمر بیا
\\
چون ز شمس الدین جهان پرنور شد
&&
سوی تبریز آ دلا بر سر بیا
\\
\end{longtable}
\end{center}
