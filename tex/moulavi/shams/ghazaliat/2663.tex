\begin{center}
\section*{غزل شماره ۲۶۶۳: مرا چون ناف بر مستی بریدی}
\label{sec:2663}
\addcontentsline{toc}{section}{\nameref{sec:2663}}
\begin{longtable}{l p{0.5cm} r}
مرا چون ناف بر مستی بریدی
&&
ز من چه ساقیا دامن کشیدی
\\
چنین عشقی پدید آری به هر دم
&&
پدیدآرنده چون ناپدیدی
\\
دهل پیدا دهلزن چون است پنهان
&&
زهی قفل و زهی این بی‌کلیدی
\\
جنون طرفه پیدا گشت در جان
&&
جنون را عقل‌ها کرده مریدی
\\
هزاران رنگ پیدا شد از آن خم
&&
منزه از کبودی و سپیدی
\\
دو دیده در عدم دوز و عجب بین
&&
زهی اومیدها در ناامیدی
\\
اگر دریای عمانی سراسر
&&
در آن ابری نگر کز وی چکیدی
\\
در آن دکان تو تخته تخته بودی
&&
اگر خود این زمان عرش مجیدی
\\
در اقلیم عدم ز آحاد بودی
&&
در این ده گر چه مشهور و وحیدی
\\
همان جا رو چنان ز آحاد می‌باش
&&
از آن گلشن چرا بیرون پریدی
\\
بر این سو صد گره بر پایت افتاد
&&
ز فکر وهمی و نکته عمیدی
\\
\end{longtable}
\end{center}
