\begin{center}
\section*{غزل شماره ۲۲۰۱: در گذر آمد خیالش گفت جان این است او}
\label{sec:2201}
\addcontentsline{toc}{section}{\nameref{sec:2201}}
\begin{longtable}{l p{0.5cm} r}
در گذر آمد خیالش گفت جان این است او
&&
پادشاه شهرهای لامکان این است او
\\
صد هزار انگشت‌ها اندر اشارت دیده شد
&&
سوی او از نور جان‌ها کای فلان این است او
\\
چون زمین سرسبز گشت از عکس آن گلزار او
&&
نعره‌ها آمد به گوشم ز آسمان این است او
\\
هین سبکتر دست درزن در عنان مرکبش
&&
پیش از آن کو برکشاند آن عنان این است او
\\
جمله نور حق گرفته همچو طور این جان از او
&&
همچو گوهر تافته از عین کان این است او
\\
رو به ماه آورد مریخ و بگفتش هوش دار
&&
تا نلافی تو ز خوبی هان و هان این است او
\\
شمس تبریزی شنیدستی ببین این نور را
&&
کز وی آمد کاسدی‌های بتان این است او
\\
\end{longtable}
\end{center}
