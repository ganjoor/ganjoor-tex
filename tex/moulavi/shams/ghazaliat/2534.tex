\begin{center}
\section*{غزل شماره ۲۵۳۴: مها یک دم رعیت شو مرا شه دان و سالاری}
\label{sec:2534}
\addcontentsline{toc}{section}{\nameref{sec:2534}}
\begin{longtable}{l p{0.5cm} r}
مها یک دم رعیت شو مرا شه دان و سالاری
&&
اگر مه را جفا گویم بجنبان سر بگو آری
\\
مرا بر تخت خود بنشان دوزانو پیش من بنشین
&&
مرا سلطان کن و می‌دو به پیشم چون سلحداری
\\
شها شیری تو من روبه تو من شو یک زمان من تو
&&
چو روبه شیرگیر آید جهان گوید خوش اشکاری
\\
چنان نادر خداوندی ز نادر خسروی آید
&&
که بخشد تاج و تخت خود مگر چون تو کلهداری
\\
ز بس احسان که فرمودی چنانم آرزو آمد
&&
که موسی چون سخن بشنود در می‌خواست دیداری
\\
یکی کف خاک بستان شد یکی کف خاک بستانبان
&&
که زنده می‌شود زین لطف هر خاکی و مرداری
\\
تو خود بی‌تخت سلطانی و بی‌خاتم سلیمانی
&&
تو ماهی وین فلک پیشت یکی طشت نگوساری
\\
کی باشد عقل کل پیشت یکی طفلی نوآموزی
&&
چه دارد با کمال تو به جز ریشی و دستاری
\\
گلیم موسی و هارون به از مال و زر قارون
&&
چرا شاید که بفروشی تو دیداری به دیناری
\\
مرا باری بحمدالله چه قرص مه چه برگ که
&&
ز مستی خود نمی‌دانم یکی جو را ز قنطاری
\\
سر عالم نمی‌دارم بیار آن جام خمارم
&&
ز هست خویش بیزارم چه باشد هست من باری
\\
سگ کهفی که مجنون شد ز شیر شرزه افزون شد
&&
خمش کردم که سرمستم نباید بسکلد تاری
\\
بهل ای دل چو بینایی سخن گویی و رعنایی
&&
هلا بگذار تا یابی از این اطلس کلهواری
\\
\end{longtable}
\end{center}
