\begin{center}
\section*{غزل شماره ۷۹۸: ننگ عالم شدن از بهر تو ننگی نبود}
\label{sec:0798}
\addcontentsline{toc}{section}{\nameref{sec:0798}}
\begin{longtable}{l p{0.5cm} r}
ننگ عالم شدن از بهر تو ننگی نبود
&&
با دل مرده دلان حاجت جنگی نبود
\\
عشق شیرینی جانست و همه چاشنی است
&&
چاشنی و مزه را صورت و رنگی نبود
\\
عشق شاخیست ز دریا که درآید در دل
&&
جای دریا و گهر سینه تنگی نبود
\\
ساحل نفس رها کن به تک دریا رو
&&
کاندر این بحر تو را خوف نهنگی نبود
\\
صورت هر دو جهان جمله ز آیینه عشق
&&
بنماید چو که بر آینه زنگی نبود
\\
کار روبه نبود عشق که هر روبه را
&&
حمله شیر نر و کبر پلنگی نبود
\\
\end{longtable}
\end{center}
