\begin{center}
\section*{غزل شماره ۱۳: ای باد بی‌آرام ما با گل بگو پیغام ما}
\label{sec:0013}
\addcontentsline{toc}{section}{\nameref{sec:0013}}
\begin{longtable}{l p{0.5cm} r}
ای باد بی‌آرام ما با گل بگو پیغام ما
&&
کای گل گریز اندر شکر چون گشتی از گلشن جدا
\\
ای گل ز اصل شکری تو با شکر لایقتری
&&
شکر خوش و گل هم خوش و از هر دو شیرینتر وفا
\\
رخ بر رخ شکر بنه لذت بگیر و بو بده
&&
در دولت شکر بجه از تلخی جور فنا
\\
اکنون که گشتی گلشکر قوت دلی نور نظر
&&
از گل برآ بر دل گذر آن از کجا این از کجا
\\
با خار بودی همنشین چون عقل با جانی قرین
&&
بر آسمان رو از زمین منزل به منزل تا لقا
\\
در سر خلقان می‌روی در راه پنهان می‌روی
&&
بستان به بستان می‌روی آن جا که خیزد نقش‌ها
\\
ای گل تو مرغ نادری برعکس مرغان می‌پری
&&
کامد پیامت زان سری پرها بنه بی‌پر بیا
\\
ای گل تو این‌ها دیده‌ای زان بر جهان خندیده‌ای
&&
زان جامه‌ها بدریده‌ای ای کربز لعلین قبا
\\
گل‌های پار از آسمان نعره زنان در گلستان
&&
کای هر که خواهد نردبان تا جان سپارد در بلا
\\
هین از ترشح زین طبق بگذر تو بی‌ره چون عرق
&&
از شیشه گلابگر چون روح از آن جام سما
\\
ای مقبل و میمون شما با چهره گلگون شما
&&
بودیم ما همچون شما ما روح گشتیم الصلا
\\
از گلشکر مقصود ما لطف حقست و بود ما
&&
ای بود ما آهن صفت وی لطف حق آهن ربا
\\
آهن خرد آیینه گر بر وی نهد زخم شرر
&&
ما را نمی‌خواهد مگر خواهم شما را بی‌شما
\\
هان ای دل مشکین سخن پایان ندارد این سخن
&&
با کس نیارم گفت من آن‌ها که می‌گویی مرا
\\
ای شمس تبریزی بگو سر شهان شاه خو
&&
بی حرف و صوت و رنگ و بو بی‌شمس کی تابد ضیا
\\
\end{longtable}
\end{center}
