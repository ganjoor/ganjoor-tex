\begin{center}
\section*{غزل شماره ۲۸۶۴: به شکرخنده اگر می‌ببرد دل ز کسی}
\label{sec:2864}
\addcontentsline{toc}{section}{\nameref{sec:2864}}
\begin{longtable}{l p{0.5cm} r}
به شکرخنده اگر می‌ببرد دل ز کسی
&&
می‌دهد در عوضش جان خوشی بوالهوسی
\\
گه سحر حمله برد بر دو جهان خورشیدش
&&
گه به شب گشت کند بر دل و جان چون عسسی
\\
گه بگوید که حذر کن شه شطرنج منم
&&
بیدقی گر ببری من برم از تو فرسی
\\
طوطیانند که خود را بکشند از غیرت
&&
گر به سوی شکرش راه برد خرمگسی
\\
پاره پاره کند آن طوطی مسکین خود را
&&
گر یکی پاره شکر زو ببرد مرتبسی
\\
در رخ دشمن من دوست بخندید چو برق
&&
همچو ابر این دل من پر شد و بگریست بسی
\\
در دل عارف تو هر دو جهان یاوه شود
&&
کی درآید به دو چشمی که تو را دید خسی
\\
جیب مریم ز دمش حامل معنی گردد
&&
که منم کز نفسی سازم عیسی نفسی
\\
مجمع روح تویی جان به تو خواهد آمد
&&
تو چو بحری همه سیل‌اند و فرات و ارسی
\\
ای که صالح تو و این هر دو جهان یک اشتر
&&
ما همه نعره زنان زنگله همچون جرسی
\\
نعره زنگله از جنبش اشتر باشد
&&
که شتر نقل کند از کنسی تا کنسی
\\
هر چراغی که بسوزد مطلب زو نوری
&&
نور موسی طلبی رو به چنان مقتبسی
\\
بس کن این گفت خیال است مشو وقف خیال
&&
چونک هستت به حقیقت نظر و دسترسی
\\
ای ضیاء الحق ذوالفضل حسام الدین تو
&&
عارف طب دلی بی‌رگ و نبض و مجسی
\\
\end{longtable}
\end{center}
