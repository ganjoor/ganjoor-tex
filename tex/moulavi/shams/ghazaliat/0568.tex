\begin{center}
\section*{غزل شماره ۵۶۸: چو آمد روی مه رویم چه باشد جان که جان باشد}
\label{sec:0568}
\addcontentsline{toc}{section}{\nameref{sec:0568}}
\begin{longtable}{l p{0.5cm} r}
چو آمد روی مه رویم چه باشد جان که جان باشد
&&
چو دیدی روز روشن را چه جای پاسبان باشد
\\
برای ماه و هنجارش که تا برنشکند کارش
&&
تو لطف آفتابی بین که در شب‌ها نهان باشد
\\
دلا بگریز از این خانه که دلگیرست و بیگانه
&&
به گلزاری و ایوانی که فرشش آسمان باشد
\\
از این صلح پر از کینش وز این صبح دروغینش
&&
همیشه این چنین صبحی هلاک کاروان باشد
\\
بجو آن صبح صادق را که جان بخشد خلایق را
&&
هزاران مست عاشق را صبوحی و امان باشد
\\
هر آن آتش که می‌زاید غم و اندیشه را سوزد
&&
به هر جایی که گل کاری نهالش گلستان باشد
\\
یکی یاری نکوکاری ز هر آفت نگهداری
&&
ظریفی ماه رخساری به صد جان رایگان باشد
\\
یکی خوبی شکرریزی چو باده رقص انگیزی
&&
یکی مستی خوش آمیزی که وصلش جاودان باشد
\\
اگر با نقش گرمابه شود یک لحظه همخوابه
&&
همان دم نقش گیرد جان چو من دستک زنان باشد
\\
دل آواره ما را از آن دلبر خبر آید
&&
شبی استاره ما را به ماه او قران باشد
\\
چو از بام بلند او رو نماید ناگهان ما را
&&
هوای سست بی آن دم مثال نردبان باشد
\\
کسی کو یار صبر آمد سوار ماه و ابر آمد
&&
مکن باور که ابر تر گدای ناودان باشد
\\
چو چشم چپ همی‌پرد نشان شادی دل دان
&&
چو چشم دل همی‌پرد عجب آن چه نشان باشد
\\
بسی کمپیر در چادر ز مردان برده عمر و زر
&&
مبین چادر تو آن بنگر که در چادر نهان باشد
\\
بسی ماه و بسی فتنه به زیر چادر کهنه
&&
بسی پالانیی لنگی که در برگستوان باشد
\\
بسی خرگه سیه باشد در او ترکی چو مه باشد
&&
چه غم داری تو از پیری چو اقبالت جوان باشد
\\
بریزد صورت پیرت بزاید صورت بختت
&&
ز ابر تیره زاید او که خورشید جهان باشد
\\
کسی کو خواب می‌بیند که با ماهست بر گردون
&&
چه غم گر این تن خفته میان کاهدان باشد
\\
معاذالله که مرغ جان قفس را آهنین خواهد
&&
معاذالله که سیمرغی در این تنگ آشیان باشد
\\
دهان بربند و خامش کن که نطق جاودان داری
&&
سخن با گوش و هوشی گو که او هم جاودان باشد
\\
\end{longtable}
\end{center}
