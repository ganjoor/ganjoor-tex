\begin{center}
\section*{غزل شماره ۱۸۳۷: یا رب من بدانمی‌چیست مراد یار من}
\label{sec:1837}
\addcontentsline{toc}{section}{\nameref{sec:1837}}
\begin{longtable}{l p{0.5cm} r}
یا رب من بدانمی‌چیست مراد یار من
&&
بسته ره گریز من برده دل و قرار من
\\
یا رب من بدانمی‌تا به کجام می کشد
&&
بهر چه کار می کشد هر طرفی مهار من
\\
یا رب من بدانمی‌سنگ دلی چرا کند
&&
آن شه مهربان من دلبر بردبار من
\\
یا رب من بدانمی‌هیچ به یار می رسد
&&
دود من و نفیر من یارب و زینهار من
\\
یا رب من بدانمی‌عاقبت این کجا کشد
&&
یا رب بس دراز شد این شب انتظار من
\\
یا رب چیست جوش من این همه روی پوش من
&&
چونک مرا توی توی هم یک و هم هزار من
\\
عشق تو است هر زمان در خمشی و در بیان
&&
پیش خیال چشم من روزی و روزگار من
\\
گاه شکار خوانمش گاه بهار خوانمش
&&
گاه میش لقب نهم گاه لقب خمار من
\\
کفر من است و دین من دیده نوربین من
&&
آن من است و این من نیست از او گذار من
\\
صبر نماند و خواب من اشک نماند و آب من
&&
یا رب تا کی می کند غارت هر چهار من
\\
خانه آب و گل کجا خانه جان و دل کجا
&&
یا رب آرزوم شد شهر من و دیار من
\\
این دل شهر رانده در گل تیره مانده
&&
ناله کنان که ای خدا کو حشم و تبار من
\\
یا رب اگر رسیدمی شهر خود و بدیدمی
&&
رحمت شهریار من وان همه شهر یار من
\\
رفته ره درشت من بار گران ز پشت من
&&
دلبر بردبار من آمده برده بار من
\\
آهوی شیرگیر من سیر خورد ز شیر من
&&
آن که منم شکار او گشته بود شکار من
\\
نیست شب سیاه رو جفت و حریف روز من
&&
نیست خزان سنگ دل در پی نوبهار من
\\
هیچ خمش نمی‌کنی تا به کی این دهل زنی
&&
آه که پرده در شدی ای لب پرده دار من
\\
\end{longtable}
\end{center}
