\begin{center}
\section*{غزل شماره ۳۱۱۶: دلا گر مرا تو ببینی ندانی}
\label{sec:3116}
\addcontentsline{toc}{section}{\nameref{sec:3116}}
\begin{longtable}{l p{0.5cm} r}
دلا گر مرا تو ببینی ندانی
&&
به جان آتشینم به رخ زعفرانی
\\
دل از دل بکندم که تا دل تو باشی
&&
ز جان هم بریدم که جان را تو جانی
\\
ز خون بر رخ من بدیدی نشان‌ها
&&
کنون رفت کارم گذشت از نشانی
\\
تو شاه عظیمی که در دل مقیمی
&&
تو آب حیاتی که در تن روانی
\\
تو آن نازنینی که در غیب بینی
&&
نگفتند هرگز تو را لن ترانی
\\
چه می نوش کردی چه روپوش کردی
&&
تو روپوش می‌کن که پنهان نمانی
\\
چه جنت چه دوزخ توی شاه برزخ
&&
برانی برانی بخوانی بخوانی
\\
تو آن پهلوانی که چون اسب رانی
&&
ز مشرق به مغرب به یک دم رسانی
\\
تو آن صدر و بدری که در بر و بحری
&&
هم الیاس و خضری و هم جان جانی
\\
کسی بی‌تو زنده زهی تلخ مردن
&&
چو پیش تو میرد زهی زندگانی
\\
ایا همنشینا جز این چشم بینا
&&
دو صد چشم دیگر تو داری نهانی
\\
اگر مرد دینی بسی نقش بینی
&&
مکن سجده آن را که تو جان آنی
\\
گره را تو بگشا ایا شمس تبریز
&&
گره از گمانست و تو صد عیانی
\\
\end{longtable}
\end{center}
