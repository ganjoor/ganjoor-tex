\begin{center}
\section*{غزل شماره ۱۶۴۵: گر تو مستی بر ما آی که ما مستانیم}
\label{sec:1645}
\addcontentsline{toc}{section}{\nameref{sec:1645}}
\begin{longtable}{l p{0.5cm} r}
گر تو مستی بر ما آی که ما مستانیم
&&
ور نه ما عشوه و ناموس کسی نستانیم
\\
یوسفانند که درمان دل پردردند
&&
که ز مستی بندانند که ما درمانیم
\\
ور بدانند حق و قیمت خود درشکنند
&&
چونک درمان سر خود گیرد ما درمانیم
\\
ما خرابیم و خرابات ز ما شوریده‌ست
&&
گنج عیشیم اگر چند در این ویرانیم
\\
کدخدامان به خرابات همان ساقی و بس
&&
کدخدا اوست و خدا اوست همو را دانیم
\\
مست را با غم و اندیشه و تدبیر چه کار
&&
که سزای سر صدریم و یا دربانیم
\\
هر کی از صدر خبر دارد او دربان است
&&
ما ز جان بی‌خبریم و بر آن جانانیم
\\
من نخواهم که سخن گویم الا ساقی
&&
می دمد در دل ما زانک چو نای انبانیم
\\
خوش بود سیمتنی کو بنداند که کییم
&&
بار ما می کشد و ماش همی‌رنجانیم
\\
یار ما داند کو کیست ولی برشکند
&&
خویش کاسد کند و گوید ما ارزانیم
\\
سر فرود آرد چون شاخ تر از لطف و کرم
&&
ما چو برگ از حذر فرقت او لرزانیم
\\
یک زمانم بهل ای جان که خموشانه خوش است
&&
ما سخن گوی خموشیم که چون میزانیم
\\
بس کن ار چند بیان طرق از ارکان است
&&
ما به ارکان به چه مشغول شویم ار کانیم
\\
\end{longtable}
\end{center}
