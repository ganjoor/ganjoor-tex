\begin{center}
\section*{غزل شماره ۱۹۵۰: بوی آن باغ و بهار و گلبن رعناست این}
\label{sec:1950}
\addcontentsline{toc}{section}{\nameref{sec:1950}}
\begin{longtable}{l p{0.5cm} r}
بوی آن باغ و بهار و گلبن رعناست این
&&
بوی آن یار جهان آرای جان افزاست این
\\
این چنین بویی کز او اجزای عالم مست شد
&&
از زمین نبود مگر از جانب بالا است این
\\
اختران گویند از بالا که این خورشید چیست
&&
ماهیان گویند در دریا که چه غوغاست این
\\
آفتابش روی‌ها را می کند چون آفتاب
&&
رشک جان ماه سیم افشان خوش سیماست این
\\
بعد چندین سال حسن یوسفی واپس رسید
&&
این چه حسن و خوبی است این حیرت حور است این
\\
این عجب خضری است ساقی گشته از آب حیات
&&
کوه قاف نادر است و نادره عنقاست این
\\
شعله انافتحنا مشرق و مغرب گرفت
&&
قره العین و حیات جان مولاناست این
\\
این چه می پوشی مپوشان ظاهر و مطلق بگو
&&
سنجق نصرالله و اسپاه شاه ماست این
\\
این امان هر دو عالم وین پناه هر دو کون
&&
دستگیر روز سخت و کافل فرداست این
\\
چرخ را چرخی دگر آموخت پرآشوب و شور
&&
این چه عشق است ای خداوند و عجب سوداست این
\\
ای خوش آوازی که آوازت به هر دل می رسد
&&
شرح کن این را که گوهرهای آن دریاست این
\\
\end{longtable}
\end{center}
