\begin{center}
\section*{غزل شماره ۲۳۷۳: مشنو حیلت خواجه هله ای دزد شبانه}
\label{sec:2373}
\addcontentsline{toc}{section}{\nameref{sec:2373}}
\begin{longtable}{l p{0.5cm} r}
مشنو حیلت خواجه هله ای دزد شبانه
&&
بشلولم بشلولم مجه از روزن خانه
\\
بمشو غره پرستش بمده ریش به دستش
&&
وگرت شاه کند او که تویی یار یگانه
\\
سوی صحرای عدم رو به سوی باغ ارم رو
&&
می بی‌درد نیابی تو در این دور زمانه
\\
به شه بنده نوازی تو بپر باز چو بازی
&&
به خدا لقمه بازان نخورد هیچ سمانه
\\
بخورم گر نخورم من بنهد در دهن من
&&
بروم گر نروم من کندم گوش کشانه
\\
همه میرند ولیکن همه میرند به پیشت
&&
همه تیر ای مه مه رو نپرد سوی نشانه
\\
ز چه افروخت خیالش رخ خورشیدصفت را
&&
ز کی آموخت خدایا عجب این فعل و بهانه
\\
چو تو را حسن فزون شد خردم صید جنون شد
&&
چو مرا درد فزون شد بده آن درد مغانه
\\
چو تو جمعیت جمعی تو در این جمع چو شمعی
&&
چو در این حلقه نگینی مجه ای جان زمانه
\\
تو اگر نوش حدیثی ز حدیثان خوش او
&&
تو مگو تا که بگوید لب آن قندفسانه
\\
\end{longtable}
\end{center}
