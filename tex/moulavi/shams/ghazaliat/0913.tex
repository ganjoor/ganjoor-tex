\begin{center}
\section*{غزل شماره ۹۱۳: بگو به گوش کسانی که نور چشم منند}
\label{sec:0913}
\addcontentsline{toc}{section}{\nameref{sec:0913}}
\begin{longtable}{l p{0.5cm} r}
بگو به گوش کسانی که نور چشم منند
&&
که باز نوبت آن شد که توبه‌ها شکنند
\\
هزار توبه و سوگند بشکنند آن دم
&&
که غمزه‌های دلارام طبل حسن زنند
\\
چو یار مست خرابست و روز روز طرب
&&
به غیر شنگی و مستی بیا بگو چه کنند
\\
به گوش هوش بگفتم به آب روی برو
&&
که این دم ار که قافی هم از بنت بکنند
\\
ز بس که خرقه گرو برد پیر باده فروش
&&
کنون به کوی خرابات جمله بوالحسن اند
\\
بگیر مطرب جانی قنینه کانی
&&
نواز تنتن تنتن که جمله بی‌تو تنند
\\
مقیم همچو نگین شو به حلقه عشاق
&&
که غیر حلقه عشاق جمله ممتحنند
\\
به جان جمله مردان که هر که عاشق نیست
&&
همه زنند به معنی ببین زنان چه زنند
\\
به جان جمله جان‌ها که هر کش آن جان نیست
&&
همه تنند نگه کن فروتنان چه تنند
\\
خموش باش که گفتی از این سپیتر چیست
&&
خسان سیاه گلیمند اگر چه یاسمنند
\\
\end{longtable}
\end{center}
