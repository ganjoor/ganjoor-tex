\begin{center}
\section*{غزل شماره ۱۴۳۰: نهادم پای در عشق که بر عشاق سر باشم}
\label{sec:1430}
\addcontentsline{toc}{section}{\nameref{sec:1430}}
\begin{longtable}{l p{0.5cm} r}
نهادم پای در عشق که بر عشاق سر باشم
&&
منم فرزند عشق جان ولی پیش از پدر باشم
\\
اگر چه روغن بادام از بادام می زاید
&&
همی‌گوید که جان داند که من بیش از شجر باشم
\\
به ظاهربین همی‌گوید چو مسجود ملایک شد
&&
که ای ابله روا داری که جسم مختصر باشم
\\
زمانی بر کف عشقش چو سیمابی همی‌لرزم
&&
زمانی در بر معدن همه دل همچو زر باشم
\\
منم پیدا و ناپیدا چو جان و عشق در قالب
&&
گهی اندر میان پنهان گهی شهره کمر باشم
\\
در آن زلفین آن یارم چه سوداها که من دارم
&&
گهی در حلقه می آیم گهی حلقه شمر باشم
\\
اگر عالم بقا یابد هزاران قرن و من رفته
&&
میان عاشقان هر شب سمر باشم سمر باشم
\\
مرا معشوق پنهانی چو خود پنهان همی‌خواهد
&&
وگر نی رغم شب کوران عیان همچون قمر باشم
\\
مرا گردون همی‌گوید که چون مه بر سرت دارم
&&
بگفتم نیک می گویی بپرس از من اگر باشم
\\
اگر ساحل شود جنت در او ماهی نیارامد
&&
حدیث شهد او گویم پس آنگه در شکر باشم
\\
به روز وصل اگر ما را از آن دلدار بشناسی
&&
پس آن دلبر دگر باشد من بی‌دل دگر باشم
\\
بسوزا این تنم گر من ز هر آتش برافروزم
&&
مبادم آب اگر خود من ز هر سیلاب تر باشم
\\
در آن محوی که شمس الدین تبریزیم پالاید
&&
ملک را بال می ریزد من آن جا چون بشر باشم
\\
\end{longtable}
\end{center}
