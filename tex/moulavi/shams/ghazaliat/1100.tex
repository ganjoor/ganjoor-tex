\begin{center}
\section*{غزل شماره ۱۱۰۰: بس که می‌انگیخت آن مه شور و شر}
\label{sec:1100}
\addcontentsline{toc}{section}{\nameref{sec:1100}}
\begin{longtable}{l p{0.5cm} r}
بس که می‌انگیخت آن مه شور و شر
&&
بس که می‌کرد او جهان زیر و زبر
\\
مر زبان را طاقت شرحش نماند
&&
خیره گشته همچنین می‌کرد سر
\\
ای بسا سر همچنین جنبان شده
&&
با دهان خشک و با چشمان تر
\\
در دو چشمش بین خیال یار ما
&&
رقص رقصان در سواد آن بصر
\\
من به سر گویم حدیثش بعد از این
&&
من زبان بستم ز گفتن ای پسر
\\
پیش او رو ای نسیم نرم رو
&&
پیش او بنشین به رویش درنگر
\\
تیز تیزش بنگر ای باد صبا
&&
چشم و دل را پر کن از خوبی و فر
\\
ور ببینی یار ما را روترش
&&
پرده‌ای باشد ز غیرت در نظر
\\
مو نباشد عکس مو باشد در آب
&&
صورتی باشد ترش اندر شکر
\\
توبه کردم از سخن این باز چیست
&&
توبه نبود عاشقانش را مگر
\\
توبه شیشه عشق او چون گازرست
&&
پیش گازر چیست کار شیشه گر
\\
بشکنم شیشه بریزم زیر پای
&&
تا خلد در پای مرد بی‌خبر
\\
شحنه یار ماست هر کو خسته شد
&&
گو مرا بسته به پیش شحنه بر
\\
شحنه را چاه زنخ زندان ماست
&&
تا نهم زنجیر زلفش پای بر
\\
بند و زندان خوش ای زنده دلان
&&
خوش مرا عیشیست آن جا معتبر
\\
گر چه می‌کاهم چو ماه از عشق او
&&
گر چه می‌گردم چه گردون بر قمر
\\
بعد من صد سال دیگر این غزل
&&
چون جمال یوسفی باشد سمر
\\
زانک دل هرگز نپوسد زیر خاک
&&
این ز دل گفتم نگفتم از جگر
\\
من چو داوودم شما مرغان پاک
&&
وین غزل‌ها چون زبور مستطر
\\
ای خدایا پر این مرغان مریز
&&
چون به داوودند از جان یارگر
\\
ای خدایا دست بر لب می‌نهم
&&
تا نگویم زان چه گشتم مستتر
\\
\end{longtable}
\end{center}
