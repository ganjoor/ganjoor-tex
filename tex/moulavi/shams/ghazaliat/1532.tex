\begin{center}
\section*{غزل شماره ۱۵۳۲: بیا تا عاشقی از سر بگیریم}
\label{sec:1532}
\addcontentsline{toc}{section}{\nameref{sec:1532}}
\begin{longtable}{l p{0.5cm} r}
بیا تا عاشقی از سر بگیریم
&&
جهان خاک را در زر بگیریم
\\
بیا تا نوبهار عشق باشیم
&&
نسیم از مشک و از عنبر بگیریم
\\
زمین و کوه و دشت و باغ و جان را
&&
همه در حله اخضر بگیریم
\\
دکان نعمت از باطن گشاییم
&&
چنین خو از درخت تر بگیریم
\\
ز سر خوردن درخت این برگ و بر یافت
&&
ز سر خویش برگ و بر بگیریم
\\
در دل ره برده‌اند ایشان به دلبر
&&
ز دل ما هم ره دلبر بگیریم
\\
مسلمانی بیاموزیم از وی
&&
اگر آن طره کافر بگیریم
\\
دلی دارد غمش چون سنگ مرمر
&&
از آن مرمر دو صد گوهر بگیریم
\\
چو جوشد سنگ او هفتاد چشمه
&&
سبو و کوزه و ساغر بگیریم
\\
کمینه چشمه‌اش چشمی است روشن
&&
که ما از نور او صد فر بگیریم
\\
\end{longtable}
\end{center}
