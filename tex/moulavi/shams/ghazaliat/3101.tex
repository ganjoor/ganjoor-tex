\begin{center}
\section*{غزل شماره ۳۱۰۱: کسی که باده خورد بامداد زین ساقی}
\label{sec:3101}
\addcontentsline{toc}{section}{\nameref{sec:3101}}
\begin{longtable}{l p{0.5cm} r}
کسی که باده خورد بامداد زین ساقی
&&
خمار چشم خوشش بین و فهم کن باقی
\\
به ناشتاب سعادت مرا رسید شتاب
&&
چنانک کعبه بیاید به نزد آفاقی
\\
بیا حیات همه ساقیان بپیما زود
&&
شراب لعل خدایی خاص رواقی
\\
هزار جام پر از زهر داده بود فراق
&&
رسید معدن تریاق و کرد تریاقی
\\
بیا که دولت نو یافت از تو بخت جوان
&&
بیا که خلعت نو یافت از تو مشتاقی
\\
چگونه خنده بپوشم انار خندانم
&&
نبات و قند نتاند نمود سماقی
\\
تویی که جفت کنی هر یتیم را به مراد
&&
که هیچ جفت نداری به مکرمت طاقی
\\
جهان لهو و لعب کودکانه باده دهد
&&
ز توست مستی بالغ که زفت سغراقی
\\
به گرد خانه دل مرا غم همی‌گردد
&&
بکند دیده ماران زمرد راقی
\\
برآ در آینه شو یا ز پیش چشمم دور
&&
که زنگ قیصر روم و عدو احداقی
\\
نماید آینه‌ام عکس روی و قانع نیست
&&
صور نماید و بخشد مزید براقی
\\
از این گذر کن کامروز تا به شب عیش است
&&
خراب و مست دریدیم دلق زراقی
\\
بریز بر سر و ریشش سبوی می امروز
&&
هر آنک دم زند از عقل و خوب اخلاقی
\\
چراغ قصر جهان قیصر منست امروز
&&
به برق عارض رومی و چشم قفچاقی
\\
به باد باده پراکنده گشت ابر سخن
&&
فرست باده بی‌ابر را که رزاقی
\\
\end{longtable}
\end{center}
