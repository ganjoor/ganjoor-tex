\begin{center}
\section*{غزل شماره ۱۳۴۵: تو مرا می بده و مست بخوابان و بهل}
\label{sec:1345}
\addcontentsline{toc}{section}{\nameref{sec:1345}}
\begin{longtable}{l p{0.5cm} r}
تو مرا می بده و مست بخوابان و بهل
&&
چون رسد نوبت خدمت نشوم هیچ خجل
\\
چو گه خدمت شه آید من می‌دانم
&&
گر ز آب و گلم ای دوست نیم پای به گل
\\
در نمازش چو خروسم سبک و وقت شناس
&&
نه چو زاغم که بود نعره او وصل گسل
\\
من ز راز خوش او یک دو سخن خواهم گفت
&&
دل من دار دمی ای دل تو بی‌غش و غل
\\
لذت عشق بتان را ز زحیران مطلب
&&
صبح کاذب بود این قافله را سخت مضل
\\
من بحل کردم ای جان که بریزی خونم
&&
ور نریزی تو مرا مظلمه داری نه بحل
\\
پس خمش کردم و با چشم و به ابرو گفتم
&&
سخنانی که نیاید به زبان و به سجل
\\
گر چه آن فهم نکردی تو ولی گرم شدی
&&
هله گرمی تو بیفزا چه کنی جهد مقل
\\
سردی از سایه بود شمس بود روشن و گرم
&&
فانی طلعت آن شمس شو ای سرد چو ظل
\\
تا درآمد بت خوبم ز در صومعه مست
&&
چند قندیل شکستم پی آن شمع چگل
\\
شمس تبریز مگر ماه ندانست حقت
&&
که گرفتار شدست او به چنین علت سل
\\
\end{longtable}
\end{center}
