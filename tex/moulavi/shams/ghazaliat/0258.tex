\begin{center}
\section*{غزل شماره ۲۵۸: گر بنخسبی شبی ای مه لقا}
\label{sec:0258}
\addcontentsline{toc}{section}{\nameref{sec:0258}}
\begin{longtable}{l p{0.5cm} r}
گر بنخسبی شبی ای مه لقا
&&
رو به تو بنماید گنج بقا
\\
گرم شوی شب تو به خورشید غیب
&&
چشم تو را باز کند توتیا
\\
امشب استیزه کن و سر منه
&&
تا که ببینی ز سعادت عطا
\\
جلوه گه جمله بتان در شبست
&&
نشنود آن کس که بخفت الصلا
\\
موسی عمران نه به شب دید نور
&&
سوی درختی که بگفتش بیا
\\
رفت به شب بیش ز ده ساله راه
&&
دید درختی همه غرق ضیا
\\
نی که به شب احمد معراج رفت
&&
برد براقیش به سوی سما
\\
روز پی کسب و شب از بهر عشق
&&
چشم بدی تا که نبیند تو را
\\
خلق بخفتند ولی عاشقان
&&
جمله شب قصه کنان با خدا
\\
گفت به داوود خدای کریم
&&
هر کی کند دعوی سودای ما
\\
چون همه شب خفت بود آن دروغ
&&
خواب کجا آید مر عشق را
\\
زان که بود عاشق خلوت طلب
&&
تا غم دل گوید با دلربا
\\
تشنه نخسپید مگر اندکی
&&
تشنه کجا خواب گران از کجا
\\
چونک بخسپید به خواب آب دید
&&
یا لب جو یا که سبو یا سقا
\\
جمله شب می رسد از حق خطاب
&&
خیز غنیمت شمر ای بی‌نوا
\\
ور نه پس مرگ تو حسرت خوری
&&
چونک شود جان تو از تن جدا
\\
جفت ببردند و زمین ماند خام
&&
هیچ ندارد جز خار و گیا
\\
من شدم از دست تو باقی بخوان
&&
مست شدم سر نشناسم ز پا
\\
شمس حق مفخر تبریزیان
&&
بستم لب را تو بیا برگشا
\\
\end{longtable}
\end{center}
