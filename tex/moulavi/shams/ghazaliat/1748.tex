\begin{center}
\section*{غزل شماره ۱۷۴۸: می گریزد از ما و ما قوامش داریم}
\label{sec:1748}
\addcontentsline{toc}{section}{\nameref{sec:1748}}
\begin{longtable}{l p{0.5cm} r}
می گریزد از ما و ما قوامش داریم
&&
زن زنانش آریم کش کشانش آریم
\\
می دود آن زیبا بر گل و سوسن‌ها
&&
گو بیا ما را بین ما از آن گلزاریم
\\
می کند دلداری وان همه طراری
&&
حق آن طره او که همه طراریم
\\
دام دل بگشاییم بوسه زو برباییم
&&
تا نپندارد که ما تهی گفتاریم
\\
هوش ما چون اختر یار ما خورشیدی
&&
زین سبب هر صبحی کشته آن یاریم
\\
گر بگوید فردا از غرور و سودا
&&
نقد را نگذاریم پا بر این افشاریم
\\
بحر او پرمرجان مشرب محتاجان
&&
تا بود در تن جان ما بر این اقراریم
\\
هر چه تو فرمایی عقل و دین افزایی
&&
هین بفرما که ما بنده و اشکاریم
\\
ای لبانت شکر گیسوانت عنبر
&&
وی از آن شیرینتر که همی‌پنداریم
\\
ساربان آهسته بهر هر دلخسته
&&
کن مدارا آخر کاندر این قطاریم
\\
اندر این بیشه ستان رحم کن بر مستان
&&
گر نی ما چون شیریم هم نی چون کفتاریم
\\
هین خمش کان مه رو وان مه نازک خو
&&
سر بپوشد چون ما کاشف اسراریم
\\
با همو گوید سر خالق هر مخبر
&&
ما هنوز از خامی سخت ناهمواریم
\\
\end{longtable}
\end{center}
