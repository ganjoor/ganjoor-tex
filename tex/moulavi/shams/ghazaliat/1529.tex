\begin{center}
\section*{غزل شماره ۱۵۲۹: شب دوشینه ما بیدار بودیم}
\label{sec:1529}
\addcontentsline{toc}{section}{\nameref{sec:1529}}
\begin{longtable}{l p{0.5cm} r}
شب دوشینه ما بیدار بودیم
&&
همه خفتند و ما بر کار بودیم
\\
حریف غمزه غماز گشتیم
&&
ندیم طره طرار بودیم
\\
به گرد نقطه خوبی و مستی
&&
به سر گردنده چون پرگار بودیم
\\
تو چون دی زاده‌ای با تو چه گویم
&&
که با یار قدیمی یار بودیم
\\
مثال کاسه‌های لب شکسته
&&
به دکان شه جبار بودیم
\\
چرا چون جام شه زرین نباشیم
&&
چو اندر مخزن اسرار بودیم
\\
چرا خود کف ما دریا نباشد
&&
چو اندر قعر دریابار بودیم
\\
خمش باش و دو عالم را به گفت آر
&&
کز اول گفت بی‌گفتار بودیم
\\
\end{longtable}
\end{center}
