\begin{center}
\section*{غزل شماره ۳۱۴۱: ای خجل از تو شکر و آزادی}
\label{sec:3141}
\addcontentsline{toc}{section}{\nameref{sec:3141}}
\begin{longtable}{l p{0.5cm} r}
ای خجل از تو شکر و آزادی
&&
لایق آن وصال کو شادی
\\
عشق را بین که صد دهان بگشاد
&&
چون تو چشمان عشق بگشادی
\\
ای دلا گرد حوض می‌گشتی
&&
دیدی آخر که هم درافتادی
\\
ز آب و آتش چو باد بگذشتی
&&
ای دل ار آتشی و ار بادی
\\
دل و عشق‌اند هر دو شاگردش
&&
خورد شاگرد را به استادی
\\
اولا هر چه خاک و خاکی بود
&&
پیش جاروب باد بنهادی
\\
تا همه باد گشت آبستن
&&
تا از آن باد عالمی زادی
\\
زاده باد خورد مادر را
&&
همچو آتش ز تاب بیدادی
\\
کرمکی در درخت پیدا شد
&&
تا بخوردش ز اصل و بنیادی
\\
عشق آن کرم بود در تحقیق
&&
در دل صد جنید بغدادی
\\
نی جنیدی گذاشت و نی بغداد
&&
عشق خونی به زخم جلادی
\\
چون خلیفه بکوفت طبل بقا
&&
کرد خالق اساس ایجادی
\\
یک وجودی بزرگ ظاهر شد
&&
همه شادی و عشرت و رادی
\\
شمس تبریز چهره‌ای بنما
&&
تا نمایم سخن بعبادی
\\
\end{longtable}
\end{center}
