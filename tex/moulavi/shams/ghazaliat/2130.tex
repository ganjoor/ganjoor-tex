\begin{center}
\section*{غزل شماره ۲۱۳۰: ای عاشقان ای عاشقان آن کس که بیند روی او}
\label{sec:2130}
\addcontentsline{toc}{section}{\nameref{sec:2130}}
\begin{longtable}{l p{0.5cm} r}
ای عاشقان ای عاشقان آن کس که بیند روی او
&&
شوریده گردد عقل او آشفته گردد خوی او
\\
معشوق را جویان شود دکان او ویران شود
&&
بر رو و سر پویان شود چون آب اندر جوی او
\\
در عشق چون مجنون شود سرگشته چون گردون شود
&&
آن کو چنین رنجور شد نایافت شد داروی او
\\
جان ملک سجده کند آن را که حق را خاک شد
&&
ترک فلک چاکر شود آن را که شد هندوی او
\\
عشقش دل پردرد را بر کف نهد بو می‌کند
&&
چون خوش نباشد آن دلی کو گشت دستنبوی او
\\
بس سینه‌ها را خست او بس خواب‌ها را بست او
&&
بسته‌ست دست جادوان آن غمزه جادوی او
\\
شاهان همه مسکین او خوبان قراضه چین او
&&
شیران زده دم بر زمین پیش سگان کوی او
\\
بنگر یکی بر آسمان بر قله روحانیان
&&
چندین چراغ و مشعله بر برج و بر باروی او
\\
شد قلعه دارش عقل کل آن شاه بی‌طبل و دهل
&&
بر قلعه آن کس بررود کو را نماند اوی او
\\
ای ماه رویش دیده‌ای خوبی از او دزدیده‌ای
&&
ای شب تو زلفش دیده‌ای نی نی و نی یک موی او
\\
این شب سیه پوش است از آن کز تعزیه دارد نشان
&&
چون بیوه‌ای جامه سیه در خاک رفته شوی او
\\
شب فعل و دستان می‌کند او عیش پنهان می‌کند
&&
نی چشم بندد چشم او کژ می‌نهد ابروی او
\\
ای شب من این نوحه گری از تو ندارم باوری
&&
چون پیش چوگان قدر هستی دوان چون گوی او
\\
آن کس که این چوگان خورد گوی سعادت او برد
&&
بی‌پا و بی‌سر می‌دود چون دل به گرد کوی او
\\
ای روی ما چون زعفران از عشق لاله ستان او
&&
ای دل فرورفته به سر چون شانه در گیسوی او
\\
مر عشق را خود پشت کو سر تا به سر روی است او
&&
این پشت و رو این سو بود جز رو نباشد سوی او
\\
او هست از صورت بری کارش همه صورتگری
&&
ای دل ز صورت نگذری زیرا نه‌ای یک توی او
\\
داند دل هر پاک دل آواز دل ز آواز گل
&&
غریدن شیر است این در صورت آهوی او
\\
بافیده دست احد پیدا بود پیدا بود
&&
از صنعت جولاهه‌ای وز دست وز ماکوی او
\\
ای جان‌ها ماکوی او وی قبله ما کوی او
&&
فراش این کو آسمان وین خاک کدبانوی او
\\
سوزان دلم از رشک او گشته دو چشمم مشک او
&&
کی ز آب چشم او تر شود ای بحر تا زانوی او
\\
این عشق شد مهمان من زخمی بزد بر جان من
&&
صد رحمت و صد آفرین بر دست و بر بازوی او
\\
من دست و پا انداختم وز جست و جو پرداختم
&&
ای مرده جست و جوی من در پیش جست و جوی او
\\
من چند گفتم های دل خاموش از این سودای دل
&&
سودش ندارد های من چون بشنود دل هوی او
\\
\end{longtable}
\end{center}
