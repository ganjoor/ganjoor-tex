\begin{center}
\section*{غزل شماره ۱۹۰۵: اگر تو عاشقی غم را رها کن}
\label{sec:1905}
\addcontentsline{toc}{section}{\nameref{sec:1905}}
\begin{longtable}{l p{0.5cm} r}
اگر تو عاشقی غم را رها کن
&&
عروسی بین و ماتم را رها کن
\\
تو دریا باش و کشتی را برانداز
&&
تو عالم باش و عالم را رها کن
\\
چو آدم توبه کن وارو به جنت
&&
چه و زندان آدم را رها کن
\\
برآ بر چرخ چون عیسی مریم
&&
خر عیسی مریم را رها کن
\\
وگر در عشق یوسف کف بریدی
&&
همو را گیر و مرهم را رها کن
\\
وگر بیدار کردت زلف درهم
&&
خیال و خواب درهم را رها کن
\\
نفخت فیه من روحی رسیده‌ست
&&
غم بیش و غم کم را رها کن
\\
مسلم کن دل از هستی مسلم
&&
امید نامسلم را رها کن
\\
بگیر ای شیرزاده خوی شیران
&&
سگان نامعلم را رها کن
\\
حریصان را جگرخون بین و گرگین
&&
گر و ناسور محکم را رها کن
\\
بر آن آرد تو را حرص چو آزر
&&
که ابراهیم ادهم را رها کن
\\
خمش زان نوع کوته کن سخن را
&&
که الله گو اعلم را رها کن
\\
چو طالع گشت شمس الدین تبریز
&&
جهان تنگ مظلم را رها کن
\\
\end{longtable}
\end{center}
