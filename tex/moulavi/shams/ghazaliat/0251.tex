\begin{center}
\section*{غزل شماره ۲۵۱: پیشتر آ پیشتر ای بوالوفا}
\label{sec:0251}
\addcontentsline{toc}{section}{\nameref{sec:0251}}
\begin{longtable}{l p{0.5cm} r}
پیشتر آ پیشتر ای بوالوفا
&&
از من و ما بگذر و زوتر بیا
\\
پیشتر آ درگذر از ما و من
&&
پیشتر آ تا نه تو باشی نه ما
\\
کبر و تکبر بگذار و بگیر
&&
در عوض کبر چنین کبریا
\\
گفت الست و تو بگفتی بلی
&&
شکر بلی چیست کشیدن بلا
\\
سر بلی چیست که یعنی منم
&&
حلقه زن درگه فقر و فنا
\\
هم برو از جا و هم از جا مرو
&&
جا ز کجا حضرت بی‌جا کجا
\\
پاک شو از خویش و همه خاک شو
&&
تا که ز خاک تو بروید گیا
\\
ور چو گیا خشک شوی خوش بسوز
&&
تا که ز سوز تو فروزد ضیا
\\
ور شوی از سوز چو خاکستری
&&
باشد خاکستر تو کیمیا
\\
بنگر در غیب چه سان کیمیاست
&&
کو ز کف خاک بسازد تو را
\\
از کف دریا بنگارد زمین
&&
دود سیه را بنگارد سما
\\
لقمه نان را مدد جان کند
&&
باد نفس را دهد این علم‌ها
\\
پیش چنین کار و کیا جان بده
&&
فقر به جان داند جود و سخا
\\
جان پر از علت او را دهی
&&
جان بستانی خوش و بی‌منتها
\\
بس کنم این گفتن و خامش کنم
&&
در خمشی به سخن جان فزا
\\
\end{longtable}
\end{center}
