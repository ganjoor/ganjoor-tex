\begin{center}
\section*{غزل شماره ۲۲۹۰: کجا شد عهد و پیمانی که کردی دوش با بنده}
\label{sec:2290}
\addcontentsline{toc}{section}{\nameref{sec:2290}}
\begin{longtable}{l p{0.5cm} r}
کجا شد عهد و پیمانی که کردی دوش با بنده
&&
که بادا عهد و بدعهدی و حسنت هر سه پاینده
\\
ز بدعهدی چه غم دارد شهنشاهی که برباید
&&
جهانی را به یک غمزه قرانی را به یک خنده
\\
بخواه ای دل چه می‌خواهی عطا نقد است و شه حاضر
&&
که آن مه رو نفرماید که رو تا سال آینده
\\
به جان شه که نشنیدم ز نقدش وعده فردا
&&
شنیدی نور رخ نسیه ز قرص ماه تابنده
\\
کجا شد آن عنایت‌ها کجا شد آن حکایت‌ها
&&
کجا شد آن گشایش‌ها کجا شد آن گشاینده
\\
همه با ماست چه با ما که خود ماییم سرتاسر
&&
مثل گشته‌ست در عالم که جوینده‌ست یابنده
\\
چه جای ما که ما مردیم زیر پای عشق او
&&
غلط گفتم کجا میرد کسی کو شد بدو زنده
\\
خیال شه خرامان شد کلوخ و سنگ باجان شد
&&
درخت خشک خندان شد سترون گشت زاینده
\\
خیالش چون چنین باشد جمالش بین که چون باشد
&&
جمالش می‌نماید در خیال نانماینده
\\
خیالش نور خورشیدی که اندر جان‌ها افتد
&&
جمالش قرص خورشیدی به چارم چرخ تازنده
\\
نمک را در طعام آن کس شناسد در گه خوردن
&&
که تنها خورده‌ست آن را و یا بوده‌ست ساینده
\\
عجایب غیر و لاغیری که معشوق است با عاشق
&&
وصال بوالعجب دارد زدوده با زداینده
\\
\end{longtable}
\end{center}
