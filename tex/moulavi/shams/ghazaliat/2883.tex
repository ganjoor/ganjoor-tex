\begin{center}
\section*{غزل شماره ۲۸۸۳: چه حریصی که مرا بی‌خور و بی‌خواب کنی}
\label{sec:2883}
\addcontentsline{toc}{section}{\nameref{sec:2883}}
\begin{longtable}{l p{0.5cm} r}
چه حریصی که مرا بی‌خور و بی‌خواب کنی
&&
درکشی روی و مرا روی به محراب کنی
\\
آب را در دهنم تلختر از زهر کنی
&&
زهره‌ام را ببری در غم خود آب کنی
\\
سوی حج رانی و در بادیه‌ام قطع کنی
&&
اشتر و رخت مرا قسمت اعراب کنی
\\
گه ببخشی ثمر و زرع مرا خشک کنی
&&
گه به بارانش همی سخره سیلاب کنی
\\
چون ز دام تو گریزم تو به تیرم دوزی
&&
چون سوی دام روم دست به مضراب کنی
\\
باادب باشم گویی که برو مست نه‌ای
&&
بی ادب گردم تو قصه آداب کنی
\\
گر بباری تو چو باران کرم بر بامم
&&
هر دو چشمم ز نم و قطره چو میزاب کنی
\\
گه عزلت تو بگویی که چو رهبان گشتی
&&
گه صحبت تو مرا دشمن اصحاب کنی
\\
گر قصب وار نپیچم دل خود در غم تو
&&
چون قصب پیچ مرا هالک مهتاب کنی
\\
در توکل تو بگویی که سبب سنت ماست
&&
در تسبب تو نکوهیدن اسباب کنی
\\
باز جان صید کنی چنگل او درشکنی
&&
تن شود کلب معلم تش بی‌ناب کنی
\\
زرگر رنگ رخ ما چو دکانی گیرد
&&
لقب زرگر ما را همه قلاب کنی
\\
من که باشم که به درگاه تو صبح صادق
&&
هست لرزان که مباداش که کذاب کنی
\\
همه را نفی کنی بازدهی صد چندان
&&
دی دهی و به بهارش همه ایجاب کنی
\\
بزنی گردن انجم تو به تیغ خورشید
&&
بازشان هم تو فروز رخ عناب کنی
\\
چو خمش کرد بگویی که بگو و چو بگفت
&&
گوییش پس تو چرا فتح چنین باب کنی
\\
\end{longtable}
\end{center}
