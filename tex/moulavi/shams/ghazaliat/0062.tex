\begin{center}
\section*{غزل شماره ۶۲: بهار آمد بهار آمد سلام آورد مستان را}
\label{sec:0062}
\addcontentsline{toc}{section}{\nameref{sec:0062}}
\begin{longtable}{l p{0.5cm} r}
بهار آمد بهار آمد سلام آورد مستان را
&&
از آن پیغامبر خوبان پیام آورد مستان را
\\
زبان سوسن از ساقی کرامت‌های مستان گفت
&&
شنید آن سرو از سوسن قیام آورد مستان را
\\
ز اول باغ در مجلس نثار آورد آنگه نقل
&&
چو دید از لاله کوهی که جام آورد مستان را
\\
ز گریه ابر نیسانی دم سرد زمستانی
&&
چه حیلت کرد کز پرده به دام آورد مستان را
\\
سقاهم ربهم خوردند و نام و ننگ گم کردند
&&
چو آمد نامه ساقی چه نام آورد مستان را
\\
درون مجمر دل‌ها سپند و عود می‌سوزد
&&
که سرمای فراق او زکام آورد مستان را
\\
درآ در گلشن باقی برآ بر بام کان ساقی
&&
ز پنهان خانه غیبی پیام آورد مستان را
\\
چو خوبان حله پوشیدند درآ در باغ و پس بنگر
&&
که ساقی هر چه درباید تمام آورد مستان را
\\
که جان‌ها را بهار آورد و ما را روی یار آورد
&&
ببین کز جمله دولت‌ها کدام آورد مستان را
\\
ز شمس الدین تبریزی به ناگه ساقی دولت
&&
به جام خاص سلطانی مدام آورد مستان را
\\
\end{longtable}
\end{center}
