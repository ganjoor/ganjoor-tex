\begin{center}
\section*{غزل شماره ۲۵۱۴: درآمد در میان شهر آدم زفت سیلابی}
\label{sec:2514}
\addcontentsline{toc}{section}{\nameref{sec:2514}}
\begin{longtable}{l p{0.5cm} r}
درآمد در میان شهر آدم زفت سیلابی
&&
فنا شد چرخ و گردان شد ز نور پاک دولابی
\\
نبود آن شهر جز سودا بنی آدم در او شیدا
&&
برست از دی و از فردا چو شد بیدار از خوابی
\\
چو جوشید آب بادی شد که هر که را بپراند
&&
چو کاهش پیش باد تند باسهمی و باتابی
\\
چو که‌ها را شکافانید کان‌ها را پدید آرد
&&
ببینی لعل اندر لعل می‌تابد چو مهتابی
\\
در آن تابش ببینی تو یکی مه روی چینی تو
&&
دو دست هجر او پرخون مثال دست قصابی
\\
ز بوی خون دست او همه ارواح مست او
&&
همه افلاک پست او زهی بالطف وهابی
\\
مثال کشتنش باشد چو انگوری که کوبندش
&&
که تا فانی شود باقی شود انگور دوشابی
\\
اگر چه صد هزار انگور کوبی یک بود جمله
&&
چو وا شد جانب توحید جان را این چنین بابی
\\
بیاید شمس تبریزی بگیرد دست آن جان را
&&
در انگشتش کند خاتم دهد ملکی و اسبابی
\\
\end{longtable}
\end{center}
