\begin{center}
\section*{غزل شماره ۲۹۹۸: سوگند خورده‌ای که از این پس جفا کنی}
\label{sec:2998}
\addcontentsline{toc}{section}{\nameref{sec:2998}}
\begin{longtable}{l p{0.5cm} r}
سوگند خورده‌ای که از این پس جفا کنی
&&
سوگند بشکنی و جفا را رها کنی
\\
امروز دامن تو گرفتیم و می‌کشیم
&&
تا کی بهانه گیری و تا کی دغا کنی
\\
می‌خندد آن لبت صنما مژده می‌دهد
&&
کاندیشه کرده‌ای که از این پس وفا کنی
\\
بی تو نماز ما چو روا نیست سود چیست
&&
آنگه روا شود که تو حاجت روا کنی
\\
بی بحر تو چو ماهی بر خاک می‌طپیم
&&
ماهی همین کند چو ز آبش جدا کنی
\\
ظالم جفا کند ز تو ترساندش اسیر
&&
حق با تو آن کند که تو در حق ما کنی
\\
چون تو کنی جفا ز کی ترساندت کسی
&&
جز آنک سر نهد به هر آنچ اقتضا کنی
\\
خاموش کم فروش تو در یتیم را
&&
آن کش بها نباشد چونش بها کنی
\\
\end{longtable}
\end{center}
