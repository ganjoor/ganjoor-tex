\begin{center}
\section*{غزل شماره ۱۶۹۰: اندر دو کون جانا بی‌تو طرب ندیدم}
\label{sec:1690}
\addcontentsline{toc}{section}{\nameref{sec:1690}}
\begin{longtable}{l p{0.5cm} r}
اندر دو کون جانا بی‌تو طرب ندیدم
&&
دیدم بسی عجایب چون تو عجب ندیدم
\\
گفتند سوز آتش باشد نصیب کافر
&&
محروم ز آتش تو جز بولهب ندیدم
\\
من بر دریچه دل بس گوش جان نهادم
&&
چندان سخن شنیدم اما دو لب ندیدم
\\
بر بنده ناگهانی کردی نثار رحمت
&&
جز لطف بی‌حد تو آن را سبب ندیدم
\\
ای ساقی گزیده مانندت ای دو دیده
&&
اندر عجم نیامد و اندر عرب ندیدم
\\
زان باده که عصیرش اندر چرش نیامد
&&
وان شیشه که نظیرش اندر حلب ندیدم
\\
چندان بریز باده کز خود شوم پیاده
&&
کاندر خودی و هستی غیر تعب ندیدم
\\
ای شمس و ای قمر تو ای شهد و ای شکر تو
&&
ای مادر و پدر تو جز تو نسب ندیدم
\\
ای عشق بی‌تناهی وی مظهر الهی
&&
هم پشت و هم پناهی کفوت لقب ندیدم
\\
پولادپاره‌هاییم آهن رباست عشقت
&&
اصل همه طلب تو در تو طلب ندیدم
\\
خامش کن ای برادر فضل و ادب رها کن
&&
تا تو ادب بخواندی در تو ادب ندیدم
\\
ای شمس حق تبریز ای اصل اصل جان‌ها
&&
بی‌بصره وجودت من یک رطب ندیدم
\\
\end{longtable}
\end{center}
