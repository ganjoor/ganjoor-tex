\begin{center}
\section*{غزل شماره ۵۲۷: گر جان عاشق دم زند آتش در این عالم زند}
\label{sec:0527}
\addcontentsline{toc}{section}{\nameref{sec:0527}}
\begin{longtable}{l p{0.5cm} r}
گر جان عاشق دم زند آتش در این عالم زند
&&
وین عالم بی‌اصل را چون ذره‌ها برهم زند
\\
عالم همه دریا شود دریا ز هیبت لا شود
&&
آدم نماند و آدمی گر خویش با آدم زند
\\
دودی برآید از فلک نی خلق ماند نی ملک
&&
زان دود ناگه آتشی بر گنبد اعظم زند
\\
بشکافد آن دم آسمان نی کون ماند نی مکان
&&
شوری درافتد در جهان، وین سور بر ماتم زند
\\
گه آب را آتش برد گه آب آتش را خورد
&&
گه موج دریای عدم بر اشهب و ادهم زند
\\
خورشید افتد در کمی از نور جان آدمی
&&
کم پرس از نامحرمان آن جا که محرم کم زند
\\
مریخ بگذارد نری دفتر بسوزد مشتری
&&
مه را نماند، مِهتری، شادّیِ او بر غم زند
\\
افتد عطارد در وحل آتش درافتد در زحل
&&
زهره نماند زهره را تا پرده خرم زند
\\
نی قوس ماند نی قزح نی باده ماند نی قدح
&&
نی عیش ماند نی فرح نی زخم بر مرهم زند
\\
نی آب نقاشی کند نی باد فراشی کند
&&
نی باغ خوش باشی کند نی ابر نیسان نم زند
\\
نی درد ماند نی دوا نی خصم ماند نی گوا
&&
نی نای ماند نی نوا نی چنگ زیر و بم زند
\\
اسباب در باقی شود ساقی به خود ساقی شود
&&
جان ربی الاعلی گود دل ربی الاعلم زند
\\
برجه که نقاش ازل بار دوم شد در عمل
&&
تا نقش‌های بی‌بدل بر کسوه معلم زند
\\
حق آتشی افروخته تا هر چه ناحق سوخته
&&
آتش بسوزد قلب را بر قلب آن عالم زند
\\
خورشید حق دل شرق او شرقی که هر دم برق او
&&
بر پوره ادهم جهد بر عیسی مریم زند
\\
\end{longtable}
\end{center}
