\begin{center}
\section*{غزل شماره ۴۷۶: بخند بر همه عالم که جای خنده تو راست}
\label{sec:0476}
\addcontentsline{toc}{section}{\nameref{sec:0476}}
\begin{longtable}{l p{0.5cm} r}
بخند بر همه عالم که جای خنده تو راست
&&
که بنده قد و ابروی تست هر کژ و راست
\\
فتد به پای تو دولت نهد به پیش تو سر
&&
که آدمی و پری در ره تو بی‌سر و پاست
\\
پریر جان من از عشق سوی گلشن رفت
&&
تو را ندید به گلشن دمی نشست و نخاست
\\
برون دوید ز گلشن چو آب سجده کنان
&&
که جویبار سعادت که اصل جاست کجاست
\\
چو اهل دل ز دلم قصه تو بشنیدند
&&
ز جمله نعره برآمد که مست دلبر ماست
\\
پس آدمی و پری جمع گشت بر من و گفت
&&
بده ز شرق نشان‌ها که این دمت چو صباست
\\
جفات نیز شکروار چاشنی دارد
&&
زهی جفا که در او صد هزار گنج وفاست
\\
قفا بداد و سفر کرد شمس تبریزی
&&
بگو مرا تو که خورشید را چه رو و قفاست
\\
\end{longtable}
\end{center}
