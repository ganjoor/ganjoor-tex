\begin{center}
\section*{غزل شماره ۲۸۸۶: اگر امشب بر من باشی و خانه نروی}
\label{sec:2886}
\addcontentsline{toc}{section}{\nameref{sec:2886}}
\begin{longtable}{l p{0.5cm} r}
اگر امشب بر من باشی و خانه نروی
&&
یا علی شیر خدا باشی یا خود علوی
\\
اندک اندک به جنون راه بری از دم من
&&
برهی از خرد و ناگه دیوانه شوی
\\
کهنه و پیر شدی زین خرد پیر گریز
&&
تا بهار تو نماید گل و گلزار نوی
\\
به خیالی به من آیی به خیالی بروی
&&
این چه رسوایی و ننگ است زهی بند قوی
\\
به ترازوی زر ار راه دهندت غلط است
&&
بجوی زر بنه ارزی چو همان حب جوی
\\
پیک لابد بدود کیک چو او هم بدود
&&
پس کمال تو در آن نیست که یاوه بدوی
\\
بهر بردن بدو از هیبت مردن بمدو
&&
بهر کعبه بدو ای جان نه ز خوف بدوی
\\
باش شب‌ها بر من تا به سحر تا که شبی
&&
مه برآید برهی از ره و همراه غوی
\\
همه کس بیند رخساره مه را از دور
&&
خنک آن کس که برد از بغل مه گروی
\\
مه ز آغاز چو خورشید بسی تیغ کشد
&&
که ببرم سر تو گر تو از این جا نروی
\\
چون ببیند که سر خویش نمی‌گیرد او
&&
گوید او را که حریفی و ظریفی و روی
\\
من توام ور تو نیم یار شب و روز توام
&&
پدر و مادر و خویش تو به منهاج سوی
\\
چه شود گر من و تو بی‌من و تو جمع شویم
&&
فرد باشیم و یکی کوری چشم ثنوی
\\
\end{longtable}
\end{center}
