\begin{center}
\section*{غزل شماره ۲۵۰۱: گرم سیم و درم بودی مرا مونس چه کم بودی}
\label{sec:2501}
\addcontentsline{toc}{section}{\nameref{sec:2501}}
\begin{longtable}{l p{0.5cm} r}
گرم سیم و درم بودی مرا مونس چه کم بودی
&&
وگر یارم فقیرستی ز زر فارغ چه غم بودی
\\
خدایا حرمت مردان ز دنیا فارغش گردان
&&
از آن گر فارغستی او ز پیش من چه کم بودی
\\
نگارا گر مرا خواهی وگر همدرد و همراهی
&&
مکن آه و مخور حسرت که بختم محتشم بودی
\\
بتا زیبا و نیکویی رها کن این گدارویی
&&
اگر چشم تو سیرستی فلک ما را حشم بودی
\\
ز طمع آدمی باشد که خویش از وی چو بیگانه است
&&
وگر او بی‌طمع بودی همه کس خال و عم بودی
\\
بیا چون ما شو ای مه رو نه نعمت جو نه دولت جو
&&
گر ابلیس این چنین بودی شه و صاحب علم بودی
\\
از ابلیسی جدا بودی سقط او را ثنا بودی
&&
جفا او را وفا بودی سقم او را کرم بودی
\\
زهی اقبال درویشی زهی اسرار بی‌خویشی
&&
اگر دانستیی پیشت همه هستی عدم بودی
\\
جهانی هیچ و ما هیچان خیال و خواب ما پیچان
&&
وگر خفته بدانستی که در خوابم چه غم بودی
\\
خیالی بیند این خفته در اندیشه فرورفته
&&
وگر زین خواب آشفته بجستی در نعم بودی
\\
یکی زندان غم دیده یکی باغ ارم دیده
&&
وگر بیدار گشتی او نه زندان نی ارم بودی
\\
\end{longtable}
\end{center}
