\begin{center}
\section*{غزل شماره ۱۲۱۱: ای روترش به پیشم بد گفته‌ای مرا پس}
\label{sec:1211}
\addcontentsline{toc}{section}{\nameref{sec:1211}}
\begin{longtable}{l p{0.5cm} r}
ای روترش به پیشم بد گفته‌ای مرا پس
&&
مردار بوی دارد دایم دهان کرکس
\\
آن گفته پلیدت در روی شدت پدیدت
&&
پیدا بود خبیثی در روی و رنگ ناکس
\\
ما راست یار و دلبر تو مرگ و جسک می‌خور
&&
هین کز دهان هر سگ دریا نشد منجس
\\
بیت القدس اگر شد ز افرنگ پر از خوکان
&&
بدنام کی شد آخر آن مسجد مقدس
\\
این روی آینه‌ست این یوسف در او بتابد
&&
بیگانه پشت باشد هر چند شد مقرنس
\\
خفاش اگر سگالد خورشید غم ندارد
&&
خورشید را چه نقصان گر سایه شد منکس
\\
ضحاک بود عیسی عباس بود یحیی
&&
این ز اعتماد خندان وز خوف آن معبس
\\
گفتند از این دو یا رب پیش تو کیست بهتر
&&
زین هر دو چیست بهتر در منهج مؤسس
\\
حق گفت افضل آنست کش ظن به من نکوتر
&&
که حسن ظن مجرم نگذاردش مدنس
\\
تو خود عبوس گینی نه از خوف و طمع دینی
&&
از رشک زعفرانی یا از شماتت اطلس
\\
این دو به کار ناید جز ناروا نشاید
&&
ای وای آن که در وی باشد حسد مغرس
\\
واهل ز دست او را تبت بس است او را
&&
هر کو عدوی مه شد ظلمات مر ورا بس
\\
اعدات آفتابا می‌دان یقین خفاشند
&&
هم ننگ جمله مرغان هم حبس لیل عسعس
\\
ابتر بود عدوش وان منصبش نماند
&&
در دیده کی بماند گر درفتد در او خس
\\
\end{longtable}
\end{center}
