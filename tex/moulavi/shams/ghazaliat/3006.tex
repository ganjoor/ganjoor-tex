\begin{center}
\section*{غزل شماره ۳۰۰۶: سیمرغ و کیمیا و مقام قلندری}
\label{sec:3006}
\addcontentsline{toc}{section}{\nameref{sec:3006}}
\begin{longtable}{l p{0.5cm} r}
سیمرغ و کیمیا و مقام قلندری
&&
وصف قلندرست و قلندر از او بری
\\
گویی قلندرم من و این دل پذیر نیست
&&
زیرا که آفریده نباشد قلندری
\\
دام و دم قلندر بی‌چون بود مقیم
&&
خالیست از کفایت و معنی داوری
\\
از خود به خود چه جویی چون سر به سر تویی
&&
چون آب در سبویی کلی ز کل پری
\\
از خود به خود سفر کن در راه عاشقی
&&
وین قصه مختصر کن ای دوست یک سری
\\
نی بیم و نی امید نه طاعت نه معصیت
&&
نی بنده نی خدای نه وصف مجاوری
\\
عجزست و قدرتست و خدایی و بندگی
&&
بیرون ز جمله آمد این ره چو بنگری
\\
راه قلندری ز خدایی برون بود
&&
در بندگی نیاید و نه در پیمبری
\\
زینهار تا نلافد هر عاشق از گزاف
&&
کس را نشد مسلم این راه و ره بری
\\
\end{longtable}
\end{center}
