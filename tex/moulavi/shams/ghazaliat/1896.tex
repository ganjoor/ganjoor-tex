\begin{center}
\section*{غزل شماره ۱۸۹۶: نشاید از تو چندین جور کردن}
\label{sec:1896}
\addcontentsline{toc}{section}{\nameref{sec:1896}}
\begin{longtable}{l p{0.5cm} r}
نشاید از تو چندین جور کردن
&&
نشاید خون مظلومان به گردن
\\
مرا بهر تو باید زندگانی
&&
وگر نی سهل دارم جان سپردن
\\
از آن روزی که نام تو شنیدم
&&
شدم عاجز من از شب‌ها شمردن
\\
روا باشد که از چون تو کریمی
&&
نصیب من بود افسوس خوردن
\\
خداوندا از آن خوشتر چه باشد
&&
بدیدن روی تو پیش تو مردن
\\
مثال شمع شد خونم در آتش
&&
ز دل جوشیدن و بر رخ فسردن
\\
در این زندان مرا کند است دندان
&&
از این صبر و از این دندان فشردن
\\
از این خانه شدم من سیر وقت است
&&
به بام آسمان‌ها رخت بردن
\\
\end{longtable}
\end{center}
