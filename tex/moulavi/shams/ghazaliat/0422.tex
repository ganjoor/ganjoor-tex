\begin{center}
\section*{غزل شماره ۴۲۲: ای که رویت چو گل و زلف تو چون شمشادست}
\label{sec:0422}
\addcontentsline{toc}{section}{\nameref{sec:0422}}
\begin{longtable}{l p{0.5cm} r}
ای که رویت چو گل و زلف تو چون شمشادست
&&
جانم آن لحظه که غمگین تو باشم شادست
\\
نقدهایی که نه نقد غم توست آن خاکست
&&
غیر پیمودن باد هوس تو بادست
\\
کار او دارد کموخته کار توست
&&
زانک کار تو یقین کارگه ایجادست
\\
آسمان را و زمین را خبرست و معلوم
&&
کآسمان همچو زمین امر تو را منقادست
\\
روی بنمای و خمار دو جهان را بشکن
&&
نه که امروز خماران تو را میعادست
\\
آفتاب ار چه در این دور فریدست و وحید
&&
شرقیانند که او در صفشان آحادست
\\
خسروان خاک کفش را به خدا تاج کنند
&&
هر که شیرین تو را دلشده چون فرهادست
\\
می‌نهد بر لب خود دست دل من که خموش
&&
این چه وقت سخن‌ست و چه گه فریادست
\\
\end{longtable}
\end{center}
