\begin{center}
\section*{غزل شماره ۱۸۹۰: با روی تو کفر است به معنی نگریدن}
\label{sec:1890}
\addcontentsline{toc}{section}{\nameref{sec:1890}}
\begin{longtable}{l p{0.5cm} r}
با روی تو کفر است به معنی نگریدن
&&
یا باغ صفا را به یکی تره خریدن
\\
با پر تو مرغان ضمیر دل ما را
&&
در جنت فردوس حرام است پریدن
\\
اندر فلک عشق هر آن مه که بتابد
&&
آن ابر تو است ای مه و فرض است دریدن
\\
دشتی که چراگاه شکاران تو باشد
&&
شیران بنیارند در آن دست چریدن
\\
هر عشق که از آتش حسن تو نخیزد
&&
آن عشق حرام است و صلای فسریدن
\\
در باطن من جان من از غیر تو ببرید
&&
محسوس شنیدم من آواز بریدن
\\
در خواب شود غافل از این دولت بیدار
&&
از پوست چه شیره بودت در فشریدن
\\
رنجور شقاوت چو بیفتاد به یاسین
&&
لاحول بود چاره و انگشت گزیدن
\\
جز عشق خداوندی شمس الحق تبریز
&&
آن موی بصر باشد باید ستریدن
\\
\end{longtable}
\end{center}
