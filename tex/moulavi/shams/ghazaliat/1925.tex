\begin{center}
\section*{غزل شماره ۱۹۲۵: بازآمد آستین فشانان}
\label{sec:1925}
\addcontentsline{toc}{section}{\nameref{sec:1925}}
\begin{longtable}{l p{0.5cm} r}
بازآمد آستین فشانان
&&
آن دشمن جان و عقل و ایمان
\\
غارتگر صد هزار خانه
&&
ویران کن صد هزار دکان
\\
شورنده صد هزار فتنه
&&
حیرتگه صد هزار حیران
\\
آن دایه عقل و آفت عقل
&&
آن مونس جان و دشمن جان
\\
او عقل سبک کجا رباید
&&
عقلی خواهد چو عقل لقمان
\\
او جان خسیس کی پذیرد
&&
جانی خواهد چو بحر عمان
\\
آمد که خراج ده بیاور
&&
گفتم که چه ده دهی است ویران
\\
طوفان تو شهرها شکست است
&&
یک ده چه زند میان طوفان
\\
گفتا ویران مقام گنج است
&&
ویرانه ماست ای مسلمان
\\
ویرانه به ما ده و برون رو
&&
تشنیع مزن مگو پریشان
\\
ویرانه ز توست چون تو رفتی
&&
معمور شود به عدل سلطان
\\
حیلت مکن و مگو که رفتم
&&
اندر پس در مباش پنهان
\\
چون مرده بساز خویشتن را
&&
تا زنده شوی به روح انسان
\\
گفتی که تو در میان نباشی
&&
آن گفت تو هست عین قرآن
\\
کاری که کنی تو در میان نی
&&
آن کرده حق بود یقین دان
\\
باقی غزل به سر بگوییم
&&
نتوان گفتن به پیش خامان
\\
خاموش که صد هزار فرق است
&&
از گفت زبان و نور فرقان
\\
\end{longtable}
\end{center}
