\begin{center}
\section*{غزل شماره ۱۲۲۷: رویش خوش و مویش خوش وان طره جعدینش}
\label{sec:1227}
\addcontentsline{toc}{section}{\nameref{sec:1227}}
\begin{longtable}{l p{0.5cm} r}
رویش خوش و مویش خوش وان طره جعدینش
&&
صد رحمت هر ساعت بر جانش و بر دینش
\\
هر لحظه و هر ساعت یک شیوه نو آرد
&&
شیرینتر و نادرتر زان شیوه پیشینش
\\
آن طره پرچین را چون باد بشوراند
&&
صد چین و دو صد ماچین گم گردد در چینش
\\
بر روی و قفای مه سیلی زده حسن او
&&
بر دبدبه قارون تسخر زده مسکینش
\\
آن ماه که می‌خندد در شرح نمی‌گنجد
&&
ای چشم و چراغ من دم درکش و می‌بینش
\\
صد چرخ همی‌گردد بر آب حیات او
&&
صد کوه کمر بندد در خدمت تمکینش
\\
گولی مگر ای لولی این جا به چه می‌لولی
&&
رو صید و تماشا کن در شاهی شاهینش
\\
گر اسب ندارد جان پیشش برود لنگان
&&
بنشاند آن فارس جان را سپس زینش
\\
ور پای ندارد هم سر بندد و سر بنهد
&&
مانند طبیب آید آن شاه به بالینش
\\
عشقست یکی جانی دررفته به صد صورت
&&
دیوانه شدم باری من در فن و آیینش
\\
حسن و نمک نادر در صورت عشق آمد
&&
تا حسن و سکون یابد جان از پی تسکینش
\\
بر طالع ماه خود تقویم عجب بست او
&&
تقویم طلب می‌کن در سوره والتینش
\\
خورشید به تیغ خود آن را که کشد ای جان
&&
از تابش خود سازد تجهیزش و تکفینش
\\
فرهاد هوای او رفتست به که کندن
&&
تا لعل شود مرمر از ضربت میتینش
\\
من بس کنم ای مطرب بر پرده بگو این را
&&
بشنو ز پس پرده کر و فر تحسینش
\\
خامش که به پیش آمد جوزینه و لوزینه
&&
لوزینه دعا گوید حلوا کند آمینش
\\
\end{longtable}
\end{center}
