\begin{center}
\section*{غزل شماره ۱۳۵۲: چند از این قیل و قال عشق پرست و ببال}
\label{sec:1352}
\addcontentsline{toc}{section}{\nameref{sec:1352}}
\begin{longtable}{l p{0.5cm} r}
چند از این قیل و قال عشق پرست و ببال
&&
تا تو بمانی چو عشق در دو جهان بی‌زوال
\\
چند کشی بار هجر غصه و تیمار هجر
&&
خاصه که منقار هجر کند تو را پر و بال
\\
آه ز نفس فضول آه ز ضعف عقول
&&
آه ز یار ملول چند نماید ملال
\\
آن که همی‌خوانمش عجز نمی‌دانمش
&&
تا که بترسانمش از ستم و از وبال
\\
جمله سؤال و جواب زوست منم چون رباب
&&
می‌زندم او شتاب زخمه که یعنی بنال
\\
یک دم بانگ نجات یک دم آواز مات
&&
می‌زند آن خوش صفات بر من و بر وصف حال
\\
تصلح میزاننا تحسن الحاننا
&&
تذهب احزاننا انت شدید المحال
\\
\end{longtable}
\end{center}
