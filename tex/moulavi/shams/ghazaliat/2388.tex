\begin{center}
\section*{غزل شماره ۲۳۸۸: این جا کسی است پنهان دامان من گرفته}
\label{sec:2388}
\addcontentsline{toc}{section}{\nameref{sec:2388}}
\begin{longtable}{l p{0.5cm} r}
این جا کسی است پنهان دامان من گرفته
&&
خود را سپس کشیده پیشان من گرفته
\\
این جا کسی است پنهان چون جان و خوشتر از جان
&&
باغی به من نموده ایوان من گرفته
\\
این جا کسی است پنهان همچون خیال در دل
&&
اما فروغ رویش ارکان من گرفته
\\
این جا کسی است پنهان مانند قند در نی
&&
شیرین شکرفروشی دکان من گرفته
\\
جادو و چشم بندی چشم کسش نبیند
&&
سوداگری است موزون میزان من گرفته
\\
چون گلشکر من و او در همدگر سرشته
&&
من خوی او گرفته او آن من گرفته
\\
در چشم من نیاید خوبان جمله عالم
&&
بنگر خیال خوبش مژگان من گرفته
\\
من خسته گرد عالم درمان ز کس ندیدم
&&
تا درد عشق دیدم درمان من گرفته
\\
تو نیز دل کبابی درمان ز درد یابی
&&
گر گرد درد گردی فرمان من گرفته
\\
در بحر ناامیدی از خود طمع بریدی
&&
زین بحر سر برآری مرجان من گرفته
\\
بشکن طلسم صورت بگشای چشم سیرت
&&
تا شرق و غرب بینی سلطان من گرفته
\\
ساقی غیب بینی پیدا سلام کرده
&&
پیمانه جام کرده پیمان من گرفته
\\
من دامنش کشیده کای نوح روح دیده
&&
از گریه عالمی بین طوفان من گرفته
\\
تو تاج ما وآنگه سرهای ما شکسته
&&
تو یار غار وآنگه یاران من گرفته
\\
گوید ز گریه بگذر زان سوی گریه بنگر
&&
عشاق روح گشته ریحان من گرفته
\\
یاران دل شکسته بر صدر دل نشسته
&&
مستان و می‌پرستان میدان من گرفته
\\
همچو سگان تازی می‌کن شکار خامش
&&
نی چون سگان عوعو کهدان من گرفته
\\
تبریز شمس دین را بر چرخ جان ببینی
&&
اشراق نور رویش کیهان من گرفته
\\
\end{longtable}
\end{center}
