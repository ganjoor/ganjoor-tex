\begin{center}
\section*{غزل شماره ۵۸۶: سعادت جو دگر باشد و عاشق خود دگر باشد}
\label{sec:0586}
\addcontentsline{toc}{section}{\nameref{sec:0586}}
\begin{longtable}{l p{0.5cm} r}
سعادت جو دگر باشد و عاشق خود دگر باشد
&&
ندارد پای عشق او کسی کش عشق سر باشد
\\
مراد دل کجا جوید بقای جان کجا خواهد
&&
دو چشم عشق پرآتش که در خون جگر باشد
\\
ز بدحالی نمی‌نالد دو چشم از غم نمی‌مالد
&&
که او خواهد که هر لحظه ز حال بد بتر باشد
\\
نه روز بخت می‌خواهد نه شب آرام می‌جوید
&&
میان روز و شب پنهان دلش همچون سحر باشد
\\
دو کاشانه‌ست در عالم یکی دولت یکی محنت
&&
به ذات حق که آن عاشق از این هر دو به درباشد
\\
ز دریا نیست جوش او که در بس یتیمست او
&&
از این کان نیست روی او اگر چه همچو زر باشد
\\
دل از سودای شاه جان شهنشاهی کجا جوید
&&
قبا کی جوید آن جانی که کشته آن کمر باشد
\\
اگر عالم هما گیرد نجوید سایه‌اش عاشق
&&
که او سرمست عشق آن همای نام ور باشد
\\
اگر عالم شکر گیرد دلش نالان چو نی باشد
&&
وگر معشوق نی گوید گدازان چون شکر باشد
\\
ز شمس الدین تبریزی مقیم عشق می‌گویم
&&
خداوندا چرا چندین شهی اندر سفر باشد
\\
\end{longtable}
\end{center}
