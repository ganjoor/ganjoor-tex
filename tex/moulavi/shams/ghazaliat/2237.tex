\begin{center}
\section*{غزل شماره ۲۲۳۷: این ترک ماجرا ز دو حکمت برون نبو}
\label{sec:2237}
\addcontentsline{toc}{section}{\nameref{sec:2237}}
\begin{longtable}{l p{0.5cm} r}
این ترک ماجرا ز دو حکمت برون نبو
&&
یا کینه را نهفتن یا عفو و حسن خو
\\
یا آنک ماجرا نکنی به هر فرصتی
&&
یا برکنی ز خویش تو آن کین تو به تو
\\
از یار بد چه رنجی از نقص خود برنج
&&
کان خصم عکس توست مپندارشان تو دو
\\
از کبر و بخل غیر مرنج و ز خویش رنج
&&
زیرا که از دی آمد افسردگی جو
\\
ز افسردگی غیر نرنجید گرم عشق
&&
کاندر تموز مردم تشنه‌ست برف جو
\\
آن خشم انبیا مثل خشم مادر است
&&
خشمی است پر ز حلم پی طفل خوبرو
\\
خشمی است همچو خاک و یکی خاک بر دهد
&&
نسرین و سوسن و گل صدبرگ مشک بو
\\
خاکی دگر بود که همه خار بر دهد
&&
هر چند هر دو خاک یکی رنگ بد عمو
\\
در گور مار نیست تو پرمار سله‌ای
&&
چون هست این خصال بدت یک به یک عدو
\\
در نطفه می‌نگر که به یک رنگ و یک فن است
&&
زنگی و هندو است و قریشی باعلو
\\
اعراض و جسم جمله همه خاک‌هاست بس
&&
در مرتبه نگر که سفول آمد و سمو
\\
چون کاسه گدایان هر ذره بر رهش
&&
آن را کند پر از زر و در دیگری تسو
\\
از نیک بد بزاید چون گبر ز اهل دین
&&
وز بد نکو بزاید از صانعی هو
\\
گویی فسوس باشد کز من فسوس خوار
&&
صرفه برد نه خود من صرفه برم از او
\\
این مایه می‌ندانی کاین سود هر دو کون
&&
اندر سخاوت است نه در کسب سو به سو
\\
خود را و دوستان را ایثار بخش از آنک
&&
بالادو است حرص تو بی‌پای چون کدو
\\
در جود کن لجاج نه اندر مکاس و بخل
&&
چون کف شمس دین که به تبریز کرد طو
\\
\end{longtable}
\end{center}
