\begin{center}
\section*{غزل شماره ۱۸۹۴: گر زانک ملولی ز من ای فتنه حوران}
\label{sec:1894}
\addcontentsline{toc}{section}{\nameref{sec:1894}}
\begin{longtable}{l p{0.5cm} r}
گر زانک ملولی ز من ای فتنه حوران
&&
این سلسله بگذار و کسی را بمشوران
\\
در کوچه کوران تو یکی روز گذشتی
&&
افتاد دو صد خارش در دیده کوران
\\
در خواب نمودی تو شبی قامت خود را
&&
بر سرو بیفزود ز تو قد قصوران
\\
ای آنک تو را جنبش این عشق نبوده‌ست
&&
حیران شده بر جای تو چون تازه حضوران
\\
از لحن عرابی چو شتر بادیه کوبد
&&
زین لحن چه بیگانه‌ای ای کم ز ستوران
\\
عشقا تو سلیمان و سماع است سپاهت
&&
رفتند به سوراخ خود از بیم تو موران
\\
شمس الحق تبریز چو خورشید برآید
&&
زیرا که ز خورشید بود جامه عوران
\\
\end{longtable}
\end{center}
