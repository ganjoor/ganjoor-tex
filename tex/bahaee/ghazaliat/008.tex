\begin{center}
\section*{غزل شماره ۸: یک گل ز باغ دوست، کسی بو نمی‌کند}
\label{sec:008}
\addcontentsline{toc}{section}{\nameref{sec:008}}
\begin{longtable}{l p{0.5cm} r}
یک گل ز باغ دوست، کسی بو نمی‌کند
&&
تا هرچه غیر اوست، به یک سو نمی‌کند
\\
روشن نمی‌شود ز رمد، چشم سالکی
&&
تا از غبار میکده، دارو نمی‌کند
\\
گفتم: ز شیخ صومعه، کارم شود درست
&&
گفتند: او به دردکشان خو نمی‌کند
\\
گفتم: روم به میکده، گفتند: پیر ما
&&
خوش می‌کشد پیاله و خوش بو نمی‌کند
\\
رفتم به سوی مدرسه، پیری به طنز گفت:
&&
تب را کسی علاج، به طنزو نمی‌کند
\\
آن را که پیر عشق، به ماهی کند تمام
&&
در صد هزار سال، ارسطو نمی‌کند
\\
کرد اکتفا به دنیی دون خواجه، کاین عروس
&&
هیچ اکتفا، به شوهری او نمی‌کند
\\
آن کو نوید آیهٔ «لا تقنطوا» شنید
&&
گوشی به حرف واعظ پرگو نمی‌کند
\\
زرق و ریاست زهد بهائی، وگرنه او
&&
کاری کند که کافر هندو نمی‌کند
\\
\end{longtable}
\end{center}
