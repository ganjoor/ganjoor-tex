\begin{center}
\section*{غزل شماره ۱۳: اگر کنم گله من از زمانهٔ غدار}
\label{sec:013}
\addcontentsline{toc}{section}{\nameref{sec:013}}
\begin{longtable}{l p{0.5cm} r}
اگر کنم گله من از زمانهٔ غدار
&&
به خاطرت نرسد از من شکسته غبار
\\
به گوش من، سخنی گفت دوش باد صبا
&&
من از شنیدن آن، گشته‌ام ز خود بیزار
\\
که بنده را به کسان کرده‌ای شها! نسبت
&&
که از تصور ایشان مرا بود صد عار
\\
شها! شکایت، خود نیست گرچه از آداب
&&
ولی به وقت ضرورت، روا بود اظهار
\\
رواست گر من از این غصه خون بگریم، خون
&&
سزاست گر من از این غصه، زار گریم، زار
\\
بپرس قدر مرا، گرچه خوب می‌دانی
&&
که من گلم، گل؛ خارند این جماعت، خار
\\
من آن یگانهٔ دهرم که وصف فضل مرا
&&
نوشته منشی قدرت، به هر در و دیوار
\\
به هر دیار که آیی، حکایتی شنوی
&&
به هر کجا که روی، ذکر من بود در کار
\\
تو قدر من نشناسی، مرا به کم مفروش
&&
بهائیم من و باشد بهای من بسیار
\\
\end{longtable}
\end{center}
