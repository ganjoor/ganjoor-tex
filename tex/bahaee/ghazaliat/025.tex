\begin{center}
\section*{غزل شماره ۲۵: ساقیا! بده جامی، زان شراب روحانی}
\label{sec:025}
\addcontentsline{toc}{section}{\nameref{sec:025}}
\begin{longtable}{l p{0.5cm} r}
ساقیا! بده جامی، زان شراب روحانی
&&
تا دمی برآسایم زین حجاب جسمانی
\\
بهر امتحان ای دوست، گر طلب کنی جان را
&&
آنچنان برافشانم، کز طلب خجل مانی
\\
بی‌وفا نگار من، می‌کند به کار من
&&
خنده‌های زیر لب، عشوه‌های پنهانی
\\
دین و دل به یک دیدن، باختیم و خرسندیم
&&
در قمار عشق ای دل، کی بود پشیمانی؟
\\
ما ز دوست غیر از دوست، مقصدی نمی‌خواهیم
&&
حور و جنت ای زاهد! بر تو باد ارزانی
\\
رسم و عادت رندیست، از رسوم بگذشتن
&&
آستین این ژنده، می‌کند گریبانی
\\
زاهدی به میخانه، سرخ روز می‌دیدم
&&
گفتمش: مبارک باد بر تو این مسلمانی
\\
زلف و کاکل او را چون به یاد می‌آرم
&&
می‌نهم پریشانی بر سر پریشانی
\\
خانهٔ دل ما را از کرم، عمارت کن!
&&
پیش از آنکه این خانه رو نهد به ویرانی
\\
ما سیه گلیمان را جز بلا نمی‌شاید
&&
بر دل بهائی نه هر بلا که بتوانی
\\
\end{longtable}
\end{center}
