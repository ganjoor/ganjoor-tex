\begin{center}
\section*{غزل شماره ۱۱: نگشود مرا ز یاریت کار}
\label{sec:011}
\addcontentsline{toc}{section}{\nameref{sec:011}}
\begin{longtable}{l p{0.5cm} r}
نگشود مرا ز یاریت کار
&&
دست از دلم ای رفیق! بردار
\\
گرد رخ من، ز خاک آن کوست
&&
ناشسته مرا به خاک بسپار
\\
رندیست ره سلامت ای دل!
&&
من کرده‌ام استخاره، صد بار
\\
سجادهٔ زهد من، که آمد
&&
خالی از عیب و عاری از عار
\\
پودش، همگی ز تار چنگ است
&&
تارش، همگی ز پود زنار
\\
خالی شده کوی دوست از دوست
&&
از بام و درش، چه پرسی اخبار؟
\\
کز غیر صدا جواب ناید
&&
هرچند کنی سؤال تکرار
\\
گر می‌پرسی: کجاست دلدار؟
&&
آید ز صدا: کجاست دلدار؟
\\
از بهر فریب خلق، دامی است
&&
هان! تا نشوی بدان گرفتار
\\
افسوس که تقوی بهائی
&&
شد شهره به رندی آخر کار
\\
\end{longtable}
\end{center}
