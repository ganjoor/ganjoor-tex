\begin{center}
\section*{بخش ۱ - بسم الله الرحمن الرحیم}
\label{sec:001}
\addcontentsline{toc}{section}{\nameref{sec:001}}
\begin{longtable}{l p{0.5cm} r}
ای که روز و شب زنی از علم لاف
&&
هیچ بر جهلت نداری اعتراف
\\
ادعای اتباع دین و شرع
&&
شرع و دین مقصود دانسته به فرع
\\
و آن هم استحسان و رأی از اجتهاد
&&
نه خبر از مبداء و نه از معاد
\\
بر ظواهر گشته قائل، چون عوام
&&
گاه ذم حکمت و گاهی کلام
\\
گه تنیدت بر ارسطالیس، گاه
&&
بر فلاطون طعن کردن بی‌گناه
\\
دعوی فهم علوم و فلسفه
&&
نفی یا اثباتش از روی سفه
\\
تو چه از حکمت به دست آورده‌ای
&&
حاش لله، ار تصور کرده‌ای
\\
چیست حکمت؟ طائر قدسی شدن
&&
سیر کردن در وجود خویشتن
\\
ظلمت تن طی نمودن، بعد از آن
&&
خویش را بردن سوی انوار جان
\\
پا نهادن در جهان دیگری
&&
خوشتری، زیباتری، بالاتری
\\
کشور جان و جهان تازه‌ای
&&
کش جهان تن بود دروازه‌ای
\\
خالص و صافی شوی از خاک پاک
&&
نه ز آتش خوف و نه از آب پاک
\\
هر طرف وضع رشیقی در نظر
&&
هر طرف طور انیقی جلوه‌گر
\\
هر طرف انوار فیض لایزال
&&
حسن در حسن و جمال اندر جمال
\\
حکمت آمد گنج مقصود ای حزین!
&&
لیک اگر با فقه و زهد آید قرین
\\
فقه و زهد ار مجتمع نبود به هم
&&
کی توان زد در ره حکمت قدم؟
\\
فقه چبود؟ آنچه محتاجی بر آن
&&
هر صباح و شام بل آنا فن
\\
فقه چبود؟ زاد راه سالکین
&&
آنکه شد بی‌زاد، گشت از هالکین
\\
زهد چه؟ تجرید قلب از حب غیر
&&
تا تعلق نایدت مانع ز سیر
\\
گر رسد مالی، نگردی شادمان
&&
ور رود هم، نبودت با کی از آن
\\
لطف دانی؟ آنچه آید از خدا
&&
خواه ذل و فقر، خواه عز و غنا
\\
هر که او را این صفت حالی نشد
&&
دل ز حب ماسوی خالی نشد
\\
نفی، «لاتأسوا علی ما فاتکم»
&&
یأس آوردش، شده از راه گم
\\
نیست با وجه زهادت معتبر
&&
نقد باغ و راغ و گاو و اسب و خر
\\
گرچه اینها غالبا سد رهند
&&
پای‌بند ناقصان گمرهند
\\
آنکه گشت آگاه و شد واقف ز حال
&&
داند از دنیا بود بس انفعال
\\
مال دنیا را معین خود مدان
&&
ای محدث «فاحذروا» را هم بخوان
\\
حب دنیا، گرچه رأس هر خطاست
&&
اهل دنیا را در آن، بس خیرهاست
\\
حب آن، رأس الخطیات آمدست
&&
بین حب الشیء و الشیء فرق هست
\\
سیب، طعمش قوت دل می‌دهد
&&
گه ز رنگش، طفل را دل می‌جهد
\\
عاقل آن را بهر قوت می‌خورد
&&
بهر رنگش، طفل حسرت می‌برد
\\
پس مدار کارها، عقل است، عقل
&&
گر نداری باور، اینک راه نقل
\\
\end{longtable}
\end{center}
