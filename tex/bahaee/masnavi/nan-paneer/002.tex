\begin{center}
\section*{بخش ۲ - حکایت}
\label{sec:002}
\addcontentsline{toc}{section}{\nameref{sec:002}}
\begin{longtable}{l p{0.5cm} r}
عابدی از قوم اسرائیلیان
&&
در عبادت بود روزان و شبان
\\
روی از لذات جسمی تافته
&&
لذت جان در عبادت یافته
\\
قطعه‌ای از ارض بود او را مکان
&&
کز سرای خلد می‌دادی نشان
\\
صیت عابد رفت تا چرخ کبود
&&
بس که بودی در رکوع و در سجود
\\
قدسیی از حال او شد باخبر
&&
کرد اندر لوح اجر او نظر
\\
دید اجری بس حقیر و بس قلیل
&&
سر او را خواست از رب جلیل
\\
وحی آمد کز برای امتحان
&&
وقتی از اوقات با وی بگذران
\\
پس ممثل گشت پیش او ملک
&&
تا کند ظاهر، عیارش بر محک
\\
گفت عابد: کیستی، احوال چیست؟
&&
زانکه با ناجنس، نتوان کرد زیست
\\
گفت: مردی، از علایق رسته‌ای
&&
چون تو، دل بر قید طاعت بسته‌ای
\\
حسن حالت دیدم و حسن مکان
&&
آمدم تا با تو باشم، یک زمان
\\
گفت عابد: آری این منزل خوش است
&&
لیک با وی، عیب زشتی نیز هست
\\
عیب آن باشد که آن زیبا علف
&&
خودبخود، صد حیف می‌گردد تلف
\\
از برای رب ما نبود حمار
&&
این علفها تا چرد فصل بهار
\\
گفت قدسی: چونکه بشنید این مقال
&&
نیست ربت را خری، ای بی‌کمال
\\
بود مقصود ملک، از این کلام
&&
نفی خر اندر خصوص آن مقام
\\
عابد این فهمید، یعنی نیست خر
&&
نه در اینجا و نه در جای دگر
\\
گفت: حاشا! این سخن دیوانگان
&&
این چنین بی‌ربط آمد بر زبان
\\
پیش هر سبزه، خری می‌داشتی
&&
خوش بود تا در چرا بگماشتی
\\
گر نبودی خر که اینها را چرید
&&
این علفها را چرا می‌آفرید؟
\\
گفت قدسی: هست خر، نی خلق را
&&
حق منزه از صفات خلق را
\\
پس ملک، هردم صد استغفار برد
&&
گرچه وی را ناقص و جاهل شمرد
\\
با وجود نفی اقرار وجود
&&
چون علفخوارش تصور کرده بود
\\
بی‌تجارب، از کیا را علم نیست
&&
کز علف حیوان تواند کرد زیست
\\
هان، تأمل کن در این نقل شریف
&&
که در آن پنهان بود سر لطیف
\\
عابد اول در میان خلق بود
&&
کسب آداب و عبادت می‌نمود
\\
ورنه، چون داند عبادت چون کند؟
&&
بر چه ملت طاعت بی‌چون کند
\\
در اوان خلطه را خلق جهان
&&
دیده بود او، آنچه دیده دیگران
\\
بعد از آن کرد او تجرد اختیار
&&
چون ندیده به ز طاعت، هیچ کار
\\
بود عقلش فاسد و ناقص ولی
&&
نه فساد ظاهر و نقص جلی
\\
مرد عابد، دیده بد خر را بسی
&&
هر یکی را لیک در دست کسی
\\
گفت: اینها خود همه، از مردم است
&&
هر یک از سعی خود آورده به دست
\\
مالک ملک آمده هر کس به عقل
&&
در تمسک، دست ما را نیست دخل
\\
چون شد اینها جمله ملک دیگری
&&
پس نباشد، حضرت رب را خری
\\
او ندانسته که کل از حق بود
&&
جمله را حق مالک مطلق بود
\\
هر که را ملکیست، از ابناء اوست
&&
هر که را مالیست، از اعطاء اوست
\\
نزع و ایتایش به وفق حکمت است
&&
هر که را گه عزت و گه ذلت است
\\
هر کجا باشد وجود خر به کار
&&
می‌کند ایجاد، از یک تا هزار
\\
هرچه خواهد می‌کند، پیدا بکن
&&
بی‌علاج و آلت حرف و سخن
\\
عقل عابد را چو این عرفان نبود
&&
با ملک کرد آنچنان گفت و شنود
\\
هان! مخند ای نفس بر عابد ز جهل
&&
هان، مدان رستن ز نقص عقل سهل
\\
در کمین خود نشینی، گر دمی
&&
خویش را بینی کم از عابد همی
\\
گر تو این اموال دانی مال رب
&&
بهر چه در غصب داری، روز و شب؟
\\
گر بود در عقد قلبت آنکه نیست
&&
مال، جز مال خدا، پس ظلم چیست؟
\\
آنچه داری مال حق دانی اگر
&&
پس به چشم عاریت، در وی نگر
\\
زان به هر وجهی که خواهی نفع گیر
&&
داده بهر انتفاع، او را معیر
\\
لیک نه وجهی که مالک نهی کرد
&&
تا شوی از خجلت آن، روی زرد
\\
گر نکردی این لوازم را ادا
&&
دعوی ملزوم کردن، دان خطا
\\
عابد اندر عقل، گرچه بود سست
&&
بود اخلاص و عباداتش درست
\\
کان ملک، تا آن زمان آمد پدید
&&
علت نقصان اجر وی بدید
\\
تا که آخر، در خلال گفتگو
&&
کرد استنباط ضعف عقل او
\\
هست در عقل تو نیز این اختلال
&&
نفی خر کرد او ز حق، تو نفی مال
\\
در تو آیا هست اخلاص و عمل؟
&&
پس چه خندی بر وی ای نفس دغل!
\\
\end{longtable}
\end{center}
