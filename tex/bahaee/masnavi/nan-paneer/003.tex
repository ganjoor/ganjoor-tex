\begin{center}
\section*{بخش ۳ - فی‌العقل}
\label{sec:003}
\addcontentsline{toc}{section}{\nameref{sec:003}}
\begin{longtable}{l p{0.5cm} r}
چیست دانی عقل در نزد حکیم؟
&&
مقتبس، نوری ز مشکوة قدیم
\\
از برای نفس تا سازد عیان
&&
از معانی، آنچه می‌تابد بر آن
\\
چون جمال عقل، عین ذات اوست
&&
نیستش محتاج عینی کو نکوست
\\
بلکه ذاتش هم لطیف و هم نکوست
&&
دیگران را نیز نیکویی به اوست
\\
پس اگر گویی، چرا نیکوست عقل
&&
خواهمت گفتن: نکو زان روست عقل
\\
جان و عقل آمد، بعینه، جان نور
&&
که بود از عین ذات او ظهور
\\
او بذاته، ظاهر آمد، نی به ذات
&&
فهم کن، تا وارهی از مشکلات
\\
نیر اعظم دو باشد: شمس و عقل
&&
جسم و جان باشند عقل و شرع و نقل
\\
نور عقلانی، فزون از شمس دان
&&
زانکه این تابد به جسم و آن به جان
\\
نور عقلانی کند تنویر دل
&&
نور شمسانی کند تنویر گل
\\
شمس بر ظاهر، همین تابان بود
&&
لیک باطن، از خرد ریان بود
\\
گر تو وصف عقل از من نشنوی
&&
گوش کن ابیات چند از مثنوی
\\
\end{longtable}
\end{center}
