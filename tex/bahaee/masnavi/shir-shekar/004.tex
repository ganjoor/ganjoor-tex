\begin{center}
\section*{بخش ۴ - فی ذم من صرف خلاصة عمره فی العلوم الرسمیة المجازیة}
\label{sec:004}
\addcontentsline{toc}{section}{\nameref{sec:004}}
\begin{longtable}{l p{0.5cm} r}
ای کرده به علم مجازی خوی
&&
نشنیده ز علم حقیقی بوی
\\
سرگرم به حکمت یونانی
&&
دلسرد ز حکمت ایمانی
\\
در علم رسوم گرو مانده
&&
نشکسته ز پای خود این کنده
\\
بر علم رسوم چو دل بستی
&&
بر اوجت اگر ببرد، پستی
\\
یک در نگشود ز مفتاحش
&&
اشکال افزود ز ایضاحش
\\
ز مقاصد آن، مقصد نایاب
&&
ز مطالع آن، طالع در خواب
\\
راهی ننمود اشاراتش
&&
دل شاد نشد ز بشاراتش
\\
محصول نداد محصل آن
&&
اجمال افزود مفصل آن
\\
تا کی ز شفاش، شفا طلبی
&&
وز کاسهٔ زهر، دوا طلبی؟
\\
تا چند چون نکبتیان مانی
&&
بر سفرهٔ چرکن یونانی
\\
تا کی به هزار شعف لیسی
&&
ته ماندهٔ کاسهٔ ابلیسی؟
\\
سؤرالمؤمن، فرموده نبی
&&
از سؤر ارسطو چه می‌طلبی؟
\\
سؤر آن جو که به روز نشور
&&
خواهی که شوی با او محشور
\\
سؤر آن جو که در عرصات
&&
ز شفاعت او یابی درجات
\\
در راه طریقت او رو کن
&&
با نان شریعت او خو کن
\\
کان راه نه ریب در او نه شک است
&&
و آن نان نه شور و نه بی‌نمک است
\\
تا چند ز فلسفه‌ات لافی
&&
وین یابس و رطب به هم بافی؟
\\
رسوا کردت به میان بشر
&&
برهان ثبوت «عقل عشر»
\\
در سر ننهاده، به جز بادت
&&
برهان «تناهی ابعادت»
\\
تا کی لافی ز «طبیعی دون»
&&
تا کی باشی به رهش مفتون؟
\\
و آن فکر که شد به هیولا صرف
&&
صورت نگرفت از آن یک حرف
\\
تصدیق چگونه به این بتوان
&&
کاندر ظلمت، برود الوان
\\
علمی که مسائل او این است
&&
بی‌شبهه، فریب شیاطین است
\\
تا چند دو اسبه پی‌اش تازی
&&
تا کی به مطالعه‌اش نازی؟
\\
وین علم دنی که تو را جان است
&&
فضلات فضایل یونان است
\\
خود گو تا چند چو خرمگسان
&&
نازی به سر فضلات کسان!
\\
تا چند ز غایت بی‌دینی
&&
خشت کتبش بر هم چینی؟
\\
اندر پی آن کتب افتاده
&&
پشتی به کتاب خداداده
\\
نی رو به شریعت مصطفوی
&&
نی دل به طریقت مرتضوی
\\
نه بهره ز علم فروع و اصول
&&
شرمت بادا ز خدا و رسول
\\
ساقی! ز کرم دو سه پیمانه
&&
در ده به بهائی دیوانه
\\
زان می که کند مس او اکسیر
&&
و «علیه یسهل کل عسیر»
\\
زان می که اگر ز قضا روزی
&&
یک جرعه از آن شودش روزی
\\
از صفحهٔ خاک رود اثرش
&&
وز قلهٔ عرش رسد خبرش
\\
\end{longtable}
\end{center}
