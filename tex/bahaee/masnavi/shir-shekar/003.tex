\begin{center}
\section*{بخش ۳ - فی نصیحة نفس الامارة و تحذیرها من الدنیا الغدارة}
\label{sec:003}
\addcontentsline{toc}{section}{\nameref{sec:003}}
\begin{longtable}{l p{0.5cm} r}
ای باد صبا، به پیام کسی
&&
چو به شهر خطاکاران برسی
\\
بگذر ز محلهٔ مهجوران
&&
وز نفس و هوی ز خدا دوران
\\
وانگاه بگو به بهائی زار
&&
کای نامه سیاه و خطا کردار
\\
کای عمر تباه گنه پیشه!
&&
تا چند زنی تو به پا تیشه؟
\\
یک دم به خود آی و به‌آیین چه کسی
&&
به چه بسته دل، به که همنفسی
\\
شد عمر تو شصت و همان پستی
&&
وز بادهٔ لهو و لعب مستی
\\
گفتم که مگر چو به سی برسی
&&
یابی خود را، دانی چه کسی
\\
درسی، درسی ز کتاب خدا
&&
رهبر نشدت به طریق هدا
\\
وز سی به چهل، چو شدی واصل
&&
جز جهل از چهل، نشدت حاصل
\\
اکنون، چو به شصت رسیدت سال
&&
یک دم نشدی فارغ ز وبال
\\
در راه خدا، قدمی نزدی
&&
بر لوح وفا، رقمی نزدی
\\
مستی ز علایق جسمانی
&&
رسوا شده‌ای و نمی‌دانی
\\
از اهل غرور، ببر پیوند
&&
خود را به شکسته دلان بربند
\\
شیشه چو شکست، شود ابتر
&&
جز شیشهٔ دل که شود بهتر
\\
ای ساقی بادهٔ روحانی
&&
زارم ز علایق جسمانی
\\
یک لمعه ز عالم نورم بخش
&&
یک جرعه ز جام طهورم بخش
\\
کز سرفکنم به صد آسانی
&&
این کهنه لحاف هیولانی
\\
\end{longtable}
\end{center}
