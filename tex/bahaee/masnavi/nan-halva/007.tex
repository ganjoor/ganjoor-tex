\begin{center}
\section*{بخش ۷ - «فی ذم العلماء المشبهین بالامراء المترفعین عن سیرة الفقرا»}
\label{sec:007}
\addcontentsline{toc}{section}{\nameref{sec:007}}
\begin{longtable}{l p{0.5cm} r}
علم یابد زیب از فقر، ای پسر
&&
نی ز باغ و راغ و اسب و گاو و خر
\\
مولوی را، هست دایم این گمان
&&
کان بیابد زیب ز اسباب جهان
\\
نقص علم است، ای جناب مولوی
&&
حشمت و مال و منال دنیوی
\\
قاقم و خز چند پوشی چون شهان؟
&&
مرغ و ماهی، چند سازی زیب خوان؟
\\
خود بده انصاف، ای صاحب کمال
&&
کی شود اینها میسر از حلال؟
\\
ای علم افراشته، در راه دین
&&
از چه شد مأکول و ملبوست چنین؟
\\
چند مال شبهه ناک آری به کف؟
&&
تا که باشی نرم پوش و خوش علف
\\
عاقبت سازد تو را، از دین بری
&&
این خودآرایی و این تن پروری
\\
لقمه کید از طریق مشتبه
&&
خاک خور خاک و بر آن دندان منه
\\
کان تو را در راه دین مغبون کند
&&
نور عرفان از دلت بیرون کند
\\
لقمهٔ نانی که باشد شبهه ناک
&&
در حریم کعبه، ابراهیم پاک
\\
گر، به دست خود فشاندی تخم آن
&&
ور به گاو چرخ کردی شخم آن
\\
ور، مه نو در حصادش داس کرد
&&
ور به سنگ کعبه‌اش، دست آس کرد
\\
ور به آب زمزمش کردی عجین
&&
مریم آیین پیکری از حور عین
\\
ور بخواندی بر خمیرش بی‌عدد
&&
فاتحه، با قل هوالله احد
\\
ور بود از شاخ طوبی آتشش
&&
ور شدی روح‌الامین هیزم کشش
\\
ور تو برخوانی هزاران بسمله
&&
بر سر آن لقمهٔ پر ولوله
\\
عاقبت، خاصیتش ظاهر شود
&&
نفس از آن لقمه تو را قاهر شود
\\
در ره طاعت، تو را بی‌جان کند
&&
خانهٔ دین تو را ویران کند
\\
درد دینت گر بود، ای مرد راه!
&&
چارهٔ خود کن، که دینت شد تباه
\\
از هوس بگذر! رها کن کش و فش
&&
پا ز دامان قناعت، در مکش
\\
گر نباشد جامهٔ اطلس تو را
&&
کهنه دلقی، ساتر تن، بس تو را
\\
ور مزعفر نبودت با قند و مشک
&&
خوش بود دوغ و پیاز و نان خشک
\\
ور نباشد مشربه از زر ناب
&&
با کف خود می‌توانی خورد آب
\\
ور نباشد مرکب زرین لگام
&&
می‌توانی زد به پای خویش گام
\\
ور نباشد دور باش از پیش و پس
&&
دور باش نفرت خلق، از تو بس
\\
ور نباشد خانه‌های زرنگار
&&
می‌توان بردن به سر در کنج غار
\\
ور نباشد فرش ابریشم طراز
&&
با حصیر کهنهٔ مسجد بساز
\\
ور نباشد شانه‌ای از بهر ریش
&&
شانه بتوان کرد با انگشت خویش
\\
هرچه بینی در جهان دارد عوض
&&
در عوض گردد تو را حاصل، غرض
\\
بی‌عوض، دانی چه باشد در جهان؟
&&
عمر باشد، عمر، قدر آن بدان
\\
\end{longtable}
\end{center}
