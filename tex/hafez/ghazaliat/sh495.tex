\begin{center}
\section*{غزل شماره ۴۹۵: می خواه و گل افشان کن از دهر چه می‌جویی}
\label{sec:sh495}
\addcontentsline{toc}{section}{\nameref{sec:sh495}}
\begin{longtable}{l p{0.5cm} r}
می خواه و گل افشان کن از دهر چه می‌جویی
&&
این گفت سحرگه گل بلبل تو چه می‌گویی
\\
مسند به گلستان بر تا شاهد و ساقی را
&&
لب گیری و رخ بوسی می نوشی و گل بویی
\\
شمشاد خرامان کن و آهنگ گلستان کن
&&
تا سرو بیاموزد از قد تو دلجویی
\\
تا غنچه خندانت دولت به که خواهد داد
&&
ای شاخ گل رعنا از بهر که می‌رویی
\\
امروز که بازارت پرجوش خریدار است
&&
دریاب و بنه گنجی از مایه نیکویی
\\
چون شمع نکورویی در رهگذر باد است
&&
طرف هنری بربند از شمع نکورویی
\\
آن طره که هر جعدش صد نافه چین ارزد
&&
خوش بودی اگر بودی بوییش ز خوش خویی
\\
هر مرغ به دستانی در گلشن شاه آمد
&&
بلبل به نواسازی حافظ به غزل گویی
\\
\end{longtable}
\end{center}
