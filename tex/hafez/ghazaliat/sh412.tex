\begin{center}
\section*{غزل شماره ۴۱۲: مرا چشمیست خون افشان ز دست آن کمان ابرو}
\label{sec:sh412}
\addcontentsline{toc}{section}{\nameref{sec:sh412}}
\begin{longtable}{l p{0.5cm} r}
مرا چشمیست خون افشان ز دست آن کمان ابرو
&&
جهان بس فتنه خواهد دید از آن چشم و از آن ابرو
\\
غلام چشم آن ترکم که در خواب خوش مستی
&&
نگارین گلشنش روی است و مشکین سایبان ابرو
\\
هلالی شد تنم زین غم که با طغرای ابرویش
&&
که باشد مه که بنماید ز طاق آسمان ابرو
\\
رقیبان غافل و ما را از آن چشم و جبین هر دم
&&
هزاران گونه پیغام است و حاجب در میان ابرو
\\
روان گوشه گیران را جبینش طرفه گلزاریست
&&
که بر طرف سمن زارش همی‌گردد چمان ابرو
\\
دگر حور و پری را کس نگوید با چنین حسنی
&&
که این را این چنین چشم است و آن را آن چنان ابرو
\\
تو کافردل نمی‌بندی نقاب زلف و می‌ترسم
&&
که محرابم بگرداند خم آن دلستان ابرو
\\
اگر چه مرغ زیرک بود حافظ در هواداری
&&
به تیر غمزه صیدش کرد چشم آن کمان ابرو
\\
\end{longtable}
\end{center}
