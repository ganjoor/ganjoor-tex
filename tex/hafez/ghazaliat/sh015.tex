\begin{center}
\section*{غزل شماره ۱۵: ای شاهد قدسی که کشد بند نقابت}
\label{sec:sh015}
\addcontentsline{toc}{section}{\nameref{sec:sh015}}
\begin{longtable}{l p{0.5cm} r}
ای شاهد قدسی که کشد بند نقابت
&&
و ای مرغ بهشتی که دهد دانه و آبت
\\
خوابم بشد از دیده در این فکر جگرسوز
&&
کاغوش که شد منزل آسایش و خوابت
\\
درویش نمی‌پرسی و ترسم که نباشد
&&
اندیشه آمرزش و پروای ثوابت
\\
راه دل عشاق زد آن چشم خماری
&&
پیداست از این شیوه که مست است شرابت
\\
تیری که زدی بر دلم از غمزه خطا رفت
&&
تا باز چه اندیشه کند رای صوابت
\\
هر ناله و فریاد که کردم نشنیدی
&&
پیداست نگارا که بلند است جنابت
\\
دور است سر آب از این بادیه هش دار
&&
تا غول بیابان نفریبد به سرابت
\\
تا در ره پیری به چه آیین روی ای دل
&&
باری به غلط صرف شد ایام شبابت
\\
ای قصر دل افروز که منزلگه انسی
&&
یا رب مکناد آفت ایام خرابت
\\
حافظ نه غلامیست که از خواجه گریزد
&&
صلحی کن و بازآ که خرابم ز عتابت
\\
\end{longtable}
\end{center}
