\begin{center}
\section*{غزل شماره ۹۲: میر من خوش می‌روی کاندر سر و پا میرمت}
\label{sec:sh092}
\addcontentsline{toc}{section}{\nameref{sec:sh092}}
\begin{longtable}{l p{0.5cm} r}
میر من خوش می‌روی کاندر سر و پا میرمت
&&
خوش خرامان شو که پیش قد رعنا میرمت
\\
گفته بودی کی بمیری پیش من تعجیل چیست
&&
خوش تقاضا می‌کنی پیش تقاضا میرمت
\\
عاشق و مخمور و مهجورم بت ساقی کجاست
&&
گو که بخرامد که پیش سروبالا میرمت
\\
آن که عمری شد که تا بیمارم از سودای او
&&
گو نگاهی کن که پیش چشم شهلا میرمت
\\
گفته‌ای لعل لبم هم درد بخشد هم دوا
&&
گاه پیش درد و گه پیش مداوا میرمت
\\
خوش خرامان می‌روی چشم بد از روی تو دور
&&
دارم اندر سر خیال آن که در پا میرمت
\\
گر چه جای حافظ اندر خلوت وصل تو نیست
&&
ای همه جای تو خوش پیش همه جا میرمت
\\
\end{longtable}
\end{center}
