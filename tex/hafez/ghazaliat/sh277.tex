\begin{center}
\section*{غزل شماره ۲۷۷: فکر بلبل همه آن است که گل شد یارش}
\label{sec:sh277}
\addcontentsline{toc}{section}{\nameref{sec:sh277}}
\begin{longtable}{l p{0.5cm} r}
فکر بلبل همه آن است که گل شد یارش
&&
گل در اندیشه که چون عشوه کند در کارش
\\
دلربایی همه آن نیست که عاشق بکشند
&&
خواجه آن است که باشد غم خدمتگارش
\\
جای آن است که خون موج زند در دل لعل
&&
زین تغابن که خزف می‌شکند بازارش
\\
بلبل از فیض گل آموخت سخن ور نه نبود
&&
این همه قول و غزل تعبیه در منقارش
\\
ای که در کوچه معشوقه ما می‌گذری
&&
بر حذر باش که سر می‌شکند دیوارش
\\
آن سفرکرده که صد قافله دل همره اوست
&&
هر کجا هست خدایا به سلامت دارش
\\
صحبت عافیتت گر چه خوش افتاد ای دل
&&
جانب عشق عزیز است فرومگذارش
\\
صوفی سرخوش از این دست که کج کرد کلاه
&&
به دو جام دگر آشفته شود دستارش
\\
دل حافظ که به دیدار تو خوگر شده بود
&&
نازپرورد وصال است مجو آزارش
\\
\end{longtable}
\end{center}
