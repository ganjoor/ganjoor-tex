\begin{center}
\section*{غزل شماره ۲۱۹: کنون که در چمن آمد گل از عدم به وجود}
\label{sec:sh219}
\addcontentsline{toc}{section}{\nameref{sec:sh219}}
\begin{longtable}{l p{0.5cm} r}
کنون که در چمن آمد گل از عدم به وجود
&&
بنفشه در قدم او نهاد سر به سجود
\\
بنوش جام صبوحی به ناله دف و چنگ
&&
ببوس غبغب ساقی به نغمه نی و عود
\\
به دور گل منشین بی شراب و شاهد و چنگ
&&
که همچو روز بقا هفته‌ای بود معدود
\\
شد از خروج ریاحین چو آسمان روشن
&&
زمین به اختر میمون و طالع مسعود
\\
ز دست شاهد نازک عذار عیسی دم
&&
شراب نوش و رها کن حدیث عاد و ثمود
\\
جهان چو خلد برین شد به دور سوسن و گل
&&
ولی چه سود که در وی نه ممکن است خلود
\\
چو گل سوار شود بر هوا سلیمان وار
&&
سحر که مرغ درآید به نغمه داوود
\\
به باغ تازه کن آیین دین زردشتی
&&
کنون که لاله برافروخت آتش نمرود
\\
بخواه جام صبوحی به یاد آصف عهد
&&
وزیر ملک سلیمان عماد دین محمود
\\
بود که مجلس حافظ به یمن تربیتش
&&
هر آنچه می‌طلبد جمله باشدش موجود
\\
\end{longtable}
\end{center}
