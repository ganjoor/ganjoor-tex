\begin{center}
\section*{غزل شماره ۱۸۶: گر می فروش حاجت رندان روا کند}
\label{sec:sh186}
\addcontentsline{toc}{section}{\nameref{sec:sh186}}
\begin{longtable}{l p{0.5cm} r}
گر می فروش حاجت رندان روا کند
&&
ایزد گنه ببخشد و دفع بلا کند
\\
ساقی به جام عدل بده باده تا گدا
&&
غیرت نیاورد که جهان پربلا کند
\\
حقا کز این غمان برسد مژده امان
&&
گر سالکی به عهد امانت وفا کند
\\
گر رنج پیش آید و گر راحت ای حکیم
&&
نسبت مکن به غیر که این‌ها خدا کند
\\
در کارخانه‌ای که ره عقل و فضل نیست
&&
فهم ضعیف رای فضولی چرا کند
\\
مطرب بساز پرده که کس بی اجل نمرد
&&
وان کو نه این ترانه سراید خطا کند
\\
ما را که درد عشق و بلای خمار کشت
&&
یا وصل دوست یا می صافی دوا کند
\\
جان رفت در سر می و حافظ به عشق سوخت
&&
عیسی دمی کجاست که احیای ما کند
\\
\end{longtable}
\end{center}
