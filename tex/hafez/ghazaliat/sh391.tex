\begin{center}
\section*{غزل شماره ۳۹۱: خوشتر از فکر می و جام چه خواهد بودن}
\label{sec:sh391}
\addcontentsline{toc}{section}{\nameref{sec:sh391}}
\begin{longtable}{l p{0.5cm} r}
خوشتر از فکر می و جام چه خواهد بودن
&&
تا ببینم که سرانجام چه خواهد بودن
\\
غم دل چند توان خورد که ایام نماند
&&
گو نه دل باش و نه ایام چه خواهد بودن
\\
مرغ کم حوصله را گو غم خود خور که بر او
&&
رحم آن کس که نهد دام چه خواهد بودن
\\
باده خور غم مخور و پند مقلد منیوش
&&
اعتبار سخن عام چه خواهد بودن
\\
دست رنج تو همان به که شود صرف به کام
&&
دانی آخر که به ناکام چه خواهد بودن
\\
پیر میخانه همی‌خواند معمایی دوش
&&
از خط جام که فرجام چه خواهد بودن
\\
بردم از ره دل حافظ به دف و چنگ و غزل
&&
تا جزای من بدنام چه خواهد بودن
\\
\end{longtable}
\end{center}
