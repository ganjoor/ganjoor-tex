\begin{center}
\section*{غزل شماره ۲۹۶: طالع اگر مدد دهد دامنش آورم به کف}
\label{sec:sh296}
\addcontentsline{toc}{section}{\nameref{sec:sh296}}
\begin{longtable}{l p{0.5cm} r}
طالع اگر مدد دهد دامنش آورم به کف
&&
گر بکشم زهی طرب ور بکشد زهی شرف
\\
طرف کرم ز کس نبست این دل پر امید من
&&
گر چه سخن همی‌برد قصه من به هر طرف
\\
از خم ابروی توام هیچ گشایشی نشد
&&
وه که در این خیال کج عمر عزیز شد تلف
\\
ابروی دوست کی شود دستکش خیال من
&&
کس نزده‌ست از این کمان تیر مراد بر هدف
\\
چند به ناز پرورم مهر بتان سنگدل
&&
یاد پدر نمی‌کنند این پسران ناخلف
\\
من به خیال زاهدی گوشه‌نشین و طرفه آنک
&&
مغبچه‌ای ز هر طرف می‌زندم به چنگ و دف
\\
بی خبرند زاهدان نقش بخوان ولاتقل
&&
مست ریاست محتسب باده بده ولاتخف
\\
صوفی شهر بین که چون لقمه شبهه می‌خورد
&&
پاردمش دراز باد آن حیوان خوش علف
\\
حافظ اگر قدم زنی در ره خاندان به صدق
&&
بدرقه رهت شود همت شحنه نجف
\\
\end{longtable}
\end{center}
