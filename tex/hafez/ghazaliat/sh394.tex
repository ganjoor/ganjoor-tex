\begin{center}
\section*{غزل شماره ۳۹۴: ای روی ماه منظر تو نوبهار حسن}
\label{sec:sh394}
\addcontentsline{toc}{section}{\nameref{sec:sh394}}
\begin{longtable}{l p{0.5cm} r}
ای روی ماه منظر تو نوبهار حسن
&&
خال و خط تو مرکز حسن و مدار حسن
\\
در چشم پرخمار تو پنهان فسون سحر
&&
در زلف بی‌قرار تو پیدا قرار حسن
\\
ماهی نتافت همچو تو از برج نیکویی
&&
سروی نخاست چون قدت از جویبار حسن
\\
خرم شد از ملاحت تو عهد دلبری
&&
فرخ شد از لطافت تو روزگار حسن
\\
از دام زلف و دانه خال تو در جهان
&&
یک مرغ دل نماند نگشته شکار حسن
\\
دایم به لطف دایه طبع از میان جان
&&
می‌پرورد به ناز تو را در کنار حسن
\\
گرد لبت بنفشه از آن تازه و تر است
&&
کآب حیات می‌خورد از جویبار حسن
\\
حافظ طمع برید که بیند نظیر تو
&&
دیار نیست جز رخت اندر دیار حسن
\\
\end{longtable}
\end{center}
