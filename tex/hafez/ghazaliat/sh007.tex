\begin{center}
\section*{غزل شماره ۷: صوفی بیا که آینه صافیست جام را}
\label{sec:sh007}
\addcontentsline{toc}{section}{\nameref{sec:sh007}}
\begin{longtable}{l p{0.5cm} r}
صوفی بیا که آینه صافیست جام را
&&
تا بنگری صفای می لعل فام را
\\
راز درون پرده ز رندان مست پرس
&&
کاین حال نیست زاهد عالی مقام را
\\
عنقا شکار کس نشود دام بازچین
&&
کان جا همیشه باد به دست است دام را
\\
در بزم دور یک دو قدح درکش و برو
&&
یعنی طمع مدار وصال دوام را
\\
ای دل شباب رفت و نچیدی گلی ز عیش
&&
پیرانه سر مکن هنری ننگ و نام را
\\
در عیش نقد کوش که چون آبخور نماند
&&
آدم بهشت روضه دارالسلام را
\\
ما را بر آستان تو بس حق خدمت است
&&
ای خواجه بازبین به ترحم غلام را
\\
حافظ مرید جام می است ای صبا برو
&&
وز بنده بندگی برسان شیخ جام را
\\
\end{longtable}
\end{center}
