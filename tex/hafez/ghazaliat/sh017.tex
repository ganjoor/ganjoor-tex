\begin{center}
\section*{غزل شماره ۱۷: سینه از آتش دل در غم جانانه بسوخت}
\label{sec:sh017}
\addcontentsline{toc}{section}{\nameref{sec:sh017}}
\begin{longtable}{l p{0.5cm} r}
سینه از آتش دل در غم جانانه بسوخت
&&
آتشی بود در این خانه که کاشانه بسوخت
\\
تنم از واسطه دوری دلبر بگداخت
&&
جانم از آتش مهر رخ جانانه بسوخت
\\
سوز دل بین که ز بس آتش اشکم دل شمع
&&
دوش بر من ز سر مهر چو پروانه بسوخت
\\
آشنایی نه غریب است که دلسوز من است
&&
چون من از خویش برفتم دل بیگانه بسوخت
\\
خرقه زهد مرا آب خرابات ببرد
&&
خانه عقل مرا آتش میخانه بسوخت
\\
چون پیاله دلم از توبه که کردم بشکست
&&
همچو لاله جگرم بی می و خمخانه بسوخت
\\
ماجرا کم کن و بازآ که مرا مردم چشم
&&
خرقه از سر به درآورد و به شکرانه بسوخت
\\
ترک افسانه بگو حافظ و می نوش دمی
&&
که نخفتیم شب و شمع به افسانه بسوخت
\\
\end{longtable}
\end{center}
