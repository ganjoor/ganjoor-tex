\begin{center}
\section*{غزل شماره ۲۰۹: قتل این خسته به شمشیر تو تقدیر نبود}
\label{sec:sh209}
\addcontentsline{toc}{section}{\nameref{sec:sh209}}
\begin{longtable}{l p{0.5cm} r}
قتل این خسته به شمشیر تو تقدیر نبود
&&
ور نه هیچ از دل بی‌رحم تو تقصیر نبود
\\
من دیوانه چو زلف تو رها می‌کردم
&&
هیچ لایق‌ترم از حلقه زنجیر نبود
\\
یا رب این آینه حسن چه جوهر دارد
&&
که در او آه مرا قوت تأثیر نبود
\\
سر ز حسرت به در میکده‌ها برکردم
&&
چون شناسای تو در صومعه یک پیر نبود
\\
نازنین‌تر ز قدت در چمن ناز نرست
&&
خوش‌تر از نقش تو در عالم تصویر نبود
\\
تا مگر همچو صبا باز به کوی تو رسم
&&
حاصلم دوش به جز ناله شبگیر نبود
\\
آن کشیدم ز تو ای آتش هجران که چو شمع
&&
جز فنای خودم از دست تو تدبیر نبود
\\
آیتی بود عذاب انده حافظ بی تو
&&
که بر هیچ کسش حاجت تفسیر نبود
\\
\end{longtable}
\end{center}
