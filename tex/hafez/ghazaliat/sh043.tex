\begin{center}
\section*{غزل شماره ۴۳: صحن بستان ذوق بخش و صحبت یاران خوش است}
\label{sec:sh043}
\addcontentsline{toc}{section}{\nameref{sec:sh043}}
\begin{longtable}{l p{0.5cm} r}
صحن بستان ذوق بخش و صحبت یاران خوش است
&&
وقت گل خوش باد کز وی وقت میخواران خوش است
\\
از صبا هر دم مشام جان ما خوش می‌شود
&&
آری آری طیب انفاس هواداران خوش است
\\
ناگشوده گل نقاب آهنگ رحلت ساز کرد
&&
ناله کن بلبل که گلبانگ دل افکاران خوش است
\\
مرغ خوشخوان را بشارت باد کاندر راه عشق
&&
دوست را با ناله شب‌های بیداران خوش است
\\
نیست در بازار عالم خوشدلی ور زان که هست
&&
شیوه رندی و خوش باشی عیاران خوش است
\\
از زبان سوسن آزاده‌ام آمد به گوش
&&
کاندر این دیر کهن کار سبکباران خوش است
\\
حافظا ترک جهان گفتن طریق خوشدلیست
&&
تا نپنداری که احوال جهان داران خوش است
\\
\end{longtable}
\end{center}
