\begin{center}
\section*{غزل شماره ۱۱۲: آن که رخسار تو را رنگ گل و نسرین داد}
\label{sec:sh112}
\addcontentsline{toc}{section}{\nameref{sec:sh112}}
\begin{longtable}{l p{0.5cm} r}
آن که رخسار تو را رنگ گل و نسرین داد
&&
صبر و آرام تواند به من مسکین داد
\\
وان که گیسوی تو را رسم تطاول آموخت
&&
هم تواند کرمش داد من غمگین داد
\\
من همان روز ز فرهاد طمع ببریدم
&&
که عنان دل شیدا به لب شیرین داد
\\
گنج زر گر نبود کنج قناعت باقیست
&&
آن که آن داد به شاهان به گدایان این داد
\\
خوش عروسیست جهان از ره صورت لیکن
&&
هر که پیوست بدو عمر خودش کاوین داد
\\
بعد از این دست من و دامن سرو و لب جوی
&&
خاصه اکنون که صبا مژده فروردین داد
\\
در کف غصه دوران دل حافظ خون شد
&&
از فراق رخت ای خواجه قوام الدین داد
\\
\end{longtable}
\end{center}
