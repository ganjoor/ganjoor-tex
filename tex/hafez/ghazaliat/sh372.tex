\begin{center}
\section*{غزل شماره ۳۷۲: بگذار تا ز شارع میخانه بگذریم}
\label{sec:sh372}
\addcontentsline{toc}{section}{\nameref{sec:sh372}}
\begin{longtable}{l p{0.5cm} r}
بگذار تا ز شارع میخانه بگذریم
&&
کز بهر جرعه‌ای همه محتاج این دریم
\\
روز نخست چون دم رندی زدیم و عشق
&&
شرط آن بود که جز ره آن شیوه نسپریم
\\
جایی که تخت و مسند جم می‌رود به باد
&&
گر غم خوریم خوش نبود به که می‌خوریم
\\
تا بو که دست در کمر او توان زدن
&&
در خون دل نشسته چو یاقوت احمریم
\\
واعظ مکن نصیحت شوریدگان که ما
&&
با خاک کوی دوست به فردوس ننگریم
\\
چون صوفیان به حالت و رقصند مقتدا
&&
ما نیز هم به شعبده دستی برآوریم
\\
از جرعه تو خاک زمین در و لعل یافت
&&
بیچاره ما که پیش تو از خاک کمتریم
\\
حافظ چو ره به کنگره کاخ وصل نیست
&&
با خاک آستانه این در به سر بریم
\\
\end{longtable}
\end{center}
