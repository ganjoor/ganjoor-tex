\begin{center}
\section*{غزل شماره ۴۰۲: نکته‌ای دلکش بگویم خال آن مه رو ببین}
\label{sec:sh402}
\addcontentsline{toc}{section}{\nameref{sec:sh402}}
\begin{longtable}{l p{0.5cm} r}
نکته‌ای دلکش بگویم خال آن مه رو ببین
&&
عقل و جان را بسته زنجیر آن گیسو ببین
\\
عیب دل کردم که وحشی وضع و هرجایی مباش
&&
گفت چشم شیرگیر و غنج آن آهو ببین
\\
حلقه زلفش تماشاخانه باد صباست
&&
جان صد صاحب دل آن جا بسته یک مو ببین
\\
عابدان آفتاب از دلبر ما غافلند
&&
ای ملامتگو خدا را رو مبین آن رو ببین
\\
زلف دل دزدش صبا را بند بر گردن نهاد
&&
با هواداران ره رو حیله هندو ببین
\\
این که من در جست و جوی او ز خود فارغ شدم
&&
کس ندیده‌ست و نبیند مثلش از هر سو ببین
\\
حافظ ار در گوشه محراب می‌نالد رواست
&&
ای نصیحتگو خدا را آن خم ابرو ببین
\\
از مراد شاه منصور ای فلک سر برمتاب
&&
تیزی شمشیر بنگر قوت بازو ببین
\\
\end{longtable}
\end{center}
