\begin{center}
\section*{غزل شماره ۵۸: سر ارادت ما و آستان حضرت دوست}
\label{sec:sh058}
\addcontentsline{toc}{section}{\nameref{sec:sh058}}
\begin{longtable}{l p{0.5cm} r}
سر ارادت ما و آستان حضرت دوست
&&
که هر چه بر سر ما می‌رود ارادت اوست
\\
نظیر دوست ندیدم اگر چه از مه و مهر
&&
نهادم آینه‌ها در مقابل رخ دوست
\\
صبا ز حال دل تنگ ما چه شرح دهد
&&
که چون شکنج ورق‌های غنچه تو بر توست
\\
نه من سبوکش این دیر رندسوزم و بس
&&
بسا سرا که در این کارخانه سنگ و سبوست
\\
مگر تو شانه زدی زلف عنبرافشان را
&&
که باد غالیه سا گشت و خاک عنبربوست
\\
نثار روی تو هر برگ گل که در چمن است
&&
فدای قد تو هر سروبن که بر لب جوست
\\
زبان ناطقه در وصف شوق نالان است
&&
چه جای کلک بریده زبان بیهده گوست
\\
رخ تو در دلم آمد مراد خواهم یافت
&&
چرا که حال نکو در قفای فال نکوست
\\
نه این زمان دل حافظ در آتش هوس است
&&
که داغدار ازل همچو لاله خودروست
\\
\end{longtable}
\end{center}
