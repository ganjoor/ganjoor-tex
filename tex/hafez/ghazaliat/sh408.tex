\begin{center}
\section*{غزل شماره ۴۰۸: ای آفتاب آینه دار جمال تو}
\label{sec:sh408}
\addcontentsline{toc}{section}{\nameref{sec:sh408}}
\begin{longtable}{l p{0.5cm} r}
ای آفتاب آینه دار جمال تو
&&
مشک سیاه مجمره گردان خال تو
\\
صحن سرای دیده بشستم ولی چه سود
&&
کاین گوشه نیست درخور خیل خیال تو
\\
در اوج ناز و نعمتی ای پادشاه حسن
&&
یا رب مباد تا به قیامت زوال تو
\\
مطبوعتر ز نقش تو صورت نبست باز
&&
طغرانویس ابروی مشکین مثال تو
\\
در چین زلفش ای دل مسکین چگونه‌ای
&&
کآشفته گفت باد صبا شرح حال تو
\\
برخاست بوی گل ز در آشتی درآی
&&
ای نوبهار ما رخ فرخنده فال تو
\\
تا آسمان ز حلقه به گوشان ما شود
&&
کو عشوه‌ای ز ابروی همچون هلال تو
\\
تا پیش بخت بازروم تهنیت کنان
&&
کو مژده‌ای ز مقدم عید وصال تو
\\
این نقطه سیاه که آمد مدار نور
&&
عکسیست در حدیقه بینش ز خال تو
\\
در پیش شاه عرض کدامین جفا کنم
&&
شرح نیازمندی خود یا ملال تو
\\
حافظ در این کمند سر سرکشان بسیست
&&
سودای کج مپز که نباشد مجال تو
\\
\end{longtable}
\end{center}
