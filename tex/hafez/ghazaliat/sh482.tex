\begin{center}
\section*{غزل شماره ۴۸۲: ای دل به کوی عشق گذاری نمی‌کنی}
\label{sec:sh482}
\addcontentsline{toc}{section}{\nameref{sec:sh482}}
\begin{longtable}{l p{0.5cm} r}
ای دل به کوی عشق گذاری نمی‌کنی
&&
اسباب جمع داری و کاری نمی‌کنی
\\
چوگان حکم در کف و گویی نمی‌زنی
&&
باز ظفر به دست و شکاری نمی‌کنی
\\
این خون که موج می‌زند اندر جگر تو را
&&
در کار رنگ و بوی نگاری نمی‌کنی
\\
مشکین از آن نشد دم خلقت که چون صبا
&&
بر خاک کوی دوست گذاری نمی‌کنی
\\
ترسم کز این چمن نبری آستین گل
&&
کز گلشنش تحمل خاری نمی‌کنی
\\
در آستین جان تو صد نافه مدرج است
&&
وان را فدای طره یاری نمی‌کنی
\\
ساغر لطیف و دلکش و می افکنی به خاک
&&
و اندیشه از بلای خماری نمی‌کنی
\\
حافظ برو که بندگی پادشاه وقت
&&
گر جمله می‌کنند تو باری نمی‌کنی
\\
\end{longtable}
\end{center}
