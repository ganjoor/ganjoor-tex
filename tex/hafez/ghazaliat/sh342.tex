\begin{center}
\section*{غزل شماره ۳۴۲: حجاب چهره جان می‌شود غبار تنم}
\label{sec:sh342}
\addcontentsline{toc}{section}{\nameref{sec:sh342}}
\begin{longtable}{l p{0.5cm} r}
حجاب چهره جان می‌شود غبار تنم
&&
خوشا دمی که از آن چهره پرده برفکنم
\\
چنین قفس نه سزای چو من خوش الحانیست
&&
روم به گلشن رضوان که مرغ آن چمنم
\\
عیان نشد که چرا آمدم کجا رفتم
&&
دریغ و درد که غافل ز کار خویشتنم
\\
چگونه طوف کنم در فضای عالم قدس
&&
که در سراچه ترکیب تخته بند تنم
\\
اگر ز خون دلم بوی شوق می‌آید
&&
عجب مدار که همدرد نافه ختنم
\\
طراز پیرهن زرکشم مبین چون شمع
&&
که سوزهاست نهانی درون پیرهنم
\\
بیا و هستی حافظ ز پیش او بردار
&&
که با وجود تو کس نشنود ز من که منم
\\
\end{longtable}
\end{center}
