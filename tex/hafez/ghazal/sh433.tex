\begin{center}
\section*{غزل شماره ۴۳۳: ای که بر ماه از خط مشکین نقاب انداختی}
\label{sec:sh433}
\addcontentsline{toc}{section}{\nameref{sec:sh433}}
\begin{longtable}{l p{0.5cm} r}
ای که بر ماه از خط مشکین نقاب انداختی
&&
لطف کردی سایه‌ای بر آفتاب انداختی
\\
تا چه خواهد کرد با ما آب و رنگ عارضت
&&
حالیا نیرنگ نقشی خوش بر آب انداختی
\\
گوی خوبی بردی از خوبان خلخ شاد باش
&&
جام کیخسرو طلب کافراسیاب انداختی
\\
هر کسی با شمع رخسارت به وجهی عشق باخت
&&
زان میان پروانه را در اضطراب انداختی
\\
گنج عشق خود نهادی در دل ویران ما
&&
سایه دولت بر این کنج خراب انداختی
\\
زینهار از آب آن عارض که شیران را از آن
&&
تشنه لب کردی و گردان را در آب انداختی
\\
خواب بیداران ببستی وان گه از نقش خیال
&&
تهمتی بر شب روان خیل خواب انداختی
\\
پرده از رخ برفکندی یک نظر در جلوه گاه
&&
و از حیا حور و پری را در حجاب انداختی
\\
باده نوش از جام عالم بین که بر اورنگ جم
&&
شاهد مقصود را از رخ نقاب انداختی
\\
از فریب نرگس مخمور و لعل می پرست
&&
حافظ خلوت نشین را در شراب انداختی
\\
و از برای صید دل در گردنم زنجیر زلف
&&
چون کمند خسرو مالک رقاب انداختی
\\
داور دارا شکوه‌ای آن که تاج آفتاب
&&
از سر تعظیم بر خاک جناب انداختی
\\
نصره الدین شاه یحیی آن که خصم ملک را
&&
از دم شمشیر چون آتش در آب انداختی
\\
\end{longtable}
\end{center}
