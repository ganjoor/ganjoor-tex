\begin{center}
\section*{غزل شماره ۴۷۷: دو یار زیرک و از باده کهن دومنی}
\label{sec:sh477}
\addcontentsline{toc}{section}{\nameref{sec:sh477}}
\begin{longtable}{l p{0.5cm} r}
دو یار زیرک و از باده کهن دومنی
&&
فراغتی و کتابی و گوشه چمنی
\\
من این مقام به دنیا و آخرت ندهم
&&
اگر چه در پی ام افتند هر دم انجمنی
\\
هر آن که کنج قناعت به گنج دنیا داد
&&
فروخت یوسف مصری به کمترین ثمنی
\\
بیا که رونق این کارخانه کم نشود
&&
به زهد همچو تویی یا به فسق همچو منی
\\
ز تندباد حوادث نمی‌توان دیدن
&&
در این چمن که گلی بوده است یا سمنی
\\
ببین در آینه جام نقش بندی غیب
&&
که کس به یاد ندارد چنین عجب زمنی
\\
از این سموم که بر طرف بوستان بگذشت
&&
عجب که بوی گلی هست و رنگ نسترنی
\\
به صبر کوش تو ای دل که حق رها نکند
&&
چنین عزیز نگینی به دست اهرمنی
\\
مزاج دهر تبه شد در این بلا حافظ
&&
کجاست فکر حکیمی و رای برهمنی
\\
\end{longtable}
\end{center}
