\begin{center}
\section*{غزل شماره ۵: دل می‌رود ز دستم صاحب دلان خدا را}
\label{sec:sh005}
\addcontentsline{toc}{section}{\nameref{sec:sh005}}
\begin{longtable}{l p{0.5cm} r}
دل می‌رود ز دستم صاحب دلان خدا را
&&
دردا که راز پنهان خواهد شد آشکارا
\\
کشتی شکستگانیم ای باد شرطه برخیز
&&
باشد که بازبینم دیدار آشنا را
\\
ده روزه مهر گردون افسانه است و افسون
&&
نیکی به جای یاران فرصت شمار یارا
\\
در حلقه گل و مل خوش خواند دوش بلبل
&&
هات الصبوح هبوا یا ایها السکارا
\\
ای صاحب کرامت شکرانه سلامت
&&
روزی تفقدی کن درویش بی‌نوا را
\\
آسایش دو گیتی تفسیر این دو حرف است
&&
با دوستان مروت با دشمنان مدارا
\\
در کوی نیک نامی ما را گذر ندادند
&&
گر تو نمی‌پسندی تغییر کن قضا را
\\
آن تلخ وش که صوفی ام الخبائثش خواند
&&
اشهی لنا و احلی من قبله العذارا
\\
هنگام تنگدستی در عیش کوش و مستی
&&
کاین کیمیای هستی قارون کند گدا را
\\
سرکش مشو که چون شمع از غیرتت بسوزد
&&
دلبر که در کف او موم است سنگ خارا
\\
آیینه سکندر جام می است بنگر
&&
تا بر تو عرضه دارد احوال ملک دارا
\\
خوبان پارسی گو بخشندگان عمرند
&&
ساقی بده بشارت رندان پارسا را
\\
حافظ به خود نپوشید این خرقه می آلود
&&
ای شیخ پاکدامن معذور دار ما را
\\
\end{longtable}
\end{center}
