\begin{center}
\section*{غزل شماره ۴۳۱: لبش می‌بوسم و در می‌کشم می}
\label{sec:sh431}
\addcontentsline{toc}{section}{\nameref{sec:sh431}}
\begin{longtable}{l p{0.5cm} r}
لبش می‌بوسم و در می‌کشم می
&&
به آب زندگانی برده‌ام پی
\\
نه رازش می‌توانم گفت با کس
&&
نه کس را می‌توانم دید با وی
\\
لبش می‌بوسد و خون می‌خورد جام
&&
رخش می‌بیند و گل می‌کند خوی
\\
بده جام می و از جم مکن یاد
&&
که می‌داند که جم کی بود و کی کی
\\
بزن در پرده چنگ ای ماه مطرب
&&
رگش بخراش تا بخروشم از وی
\\
گل از خلوت به باغ آورد مسند
&&
بساط زهد همچون غنچه کن طی
\\
چو چشمش مست را مخمور مگذار
&&
به یاد لعلش ای ساقی بده می
\\
نجوید جان از آن قالب جدایی
&&
که باشد خون جامش در رگ و پی
\\
زبانت درکش ای حافظ زمانی
&&
حدیث بی زبانان بشنو از نی
\\
\end{longtable}
\end{center}
