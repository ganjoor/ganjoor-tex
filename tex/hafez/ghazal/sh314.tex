\begin{center}
\section*{غزل شماره ۳۱۴: دوش بیماری چشم تو ببرد از دستم}
\label{sec:sh314}
\addcontentsline{toc}{section}{\nameref{sec:sh314}}
\begin{longtable}{l p{0.5cm} r}
دوش بیماری چشم تو ببرد از دستم
&&
لیکن از لطف لبت صورت جان می‌بستم
\\
عشق من با خط مشکین تو امروزی نیست
&&
دیرگاه است کز این جام هلالی مستم
\\
از ثبات خودم این نکته خوش آمد که به جور
&&
در سر کوی تو از پای طلب ننشستم
\\
عافیت چشم مدار از من میخانه نشین
&&
که دم از خدمت رندان زده‌ام تا هستم
\\
در ره عشق از آن سوی فنا صد خطر است
&&
تا نگویی که چو عمرم به سر آمد رستم
\\
بعد از اینم چه غم از تیر کج انداز حسود
&&
چون به محبوب کمان ابروی خود پیوستم
\\
بوسه بر درج عقیق تو حلال است مرا
&&
که به افسوس و جفا مهر وفا نشکستم
\\
صنمی لشکریم غارت دل کرد و برفت
&&
آه اگر عاطفت شاه نگیرد دستم
\\
رتبت دانش حافظ به فلک بر شده بود
&&
کرد غمخواری شمشاد بلندت پستم
\\
\end{longtable}
\end{center}
