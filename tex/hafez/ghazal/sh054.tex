\begin{center}
\section*{غزل شماره ۵۴: ز گریه مردم چشمم نشسته در خون است}
\label{sec:sh054}
\addcontentsline{toc}{section}{\nameref{sec:sh054}}
\begin{longtable}{l p{0.5cm} r}
ز گریه مردم چشمم نشسته در خون است
&&
ببین که در طلبت حال مردمان چون است
\\
به یاد لعل تو و چشم مست میگونت
&&
ز جام غم می لعلی که می‌خورم خون است
\\
ز مشرق سر کو آفتاب طلعت تو
&&
اگر طلوع کند طالعم همایون است
\\
حکایت لب شیرین کلام فرهاد است
&&
شکنج طره لیلی مقام مجنون است
\\
دلم بجو که قدت همچو سرو دلجوی است
&&
سخن بگو که کلامت لطیف و موزون است
\\
ز دور باده به جان راحتی رسان ساقی
&&
که رنج خاطرم از جور دور گردون است
\\
از آن دمی که ز چشمم برفت رود عزیز
&&
کنار دامن من همچو رود جیحون است
\\
چگونه شاد شود اندرون غمگینم
&&
به اختیار که از اختیار بیرون است
\\
ز بیخودی طلب یار می‌کند حافظ
&&
چو مفلسی که طلبکار گنج قارون است
\\
\end{longtable}
\end{center}
