\begin{center}
\section*{غزل شماره ۱۱۰: پیرانه سرم عشق جوانی به سر افتاد}
\label{sec:sh110}
\addcontentsline{toc}{section}{\nameref{sec:sh110}}
\begin{longtable}{l p{0.5cm} r}
پیرانه سرم عشق جوانی به سر افتاد
&&
وان راز که در دل بنهفتم به درافتاد
\\
از راه نظر مرغ دلم گشت هواگیر
&&
ای دیده نگه کن که به دام که درافتاد
\\
دردا که از آن آهوی مشکین سیه چشم
&&
چون نافه بسی خون دلم در جگر افتاد
\\
از رهگذر خاک سر کوی شما بود
&&
هر نافه که در دست نسیم سحر افتاد
\\
مژگان تو تا تیغ جهانگیر برآورد
&&
بس کشته دل زنده که بر یک دگر افتاد
\\
بس تجربه کردیم در این دیر مکافات
&&
با دردکشان هر که درافتاد برافتاد
\\
گر جان بدهد سنگ سیه لعل نگردد
&&
با طینت اصلی چه کند بدگهر افتاد
\\
حافظ که سر زلف بتان دست کشش بود
&&
بس طرفه حریفیست کش اکنون به سر افتاد
\\
\end{longtable}
\end{center}
