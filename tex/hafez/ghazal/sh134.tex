\begin{center}
\section*{غزل شماره ۱۳۴: بلبلی خون دلی خورد و گلی حاصل کرد}
\label{sec:sh134}
\addcontentsline{toc}{section}{\nameref{sec:sh134}}
\begin{longtable}{l p{0.5cm} r}
بلبلی خون دلی خورد و گلی حاصل کرد
&&
باد غیرت به صدش خار پریشان دل کرد
\\
طوطی ای را به خیال شکری دل خوش بود
&&
ناگهش سیل فنا نقش امل باطل کرد
\\
قرة العین من آن میوه دل یادش باد
&&
که چه آسان بشد و کار مرا مشکل کرد
\\
ساروان بار من افتاد خدا را مددی
&&
که امید کرمم همره این محمل کرد
\\
روی خاکی و نم چشم مرا خوار مدار
&&
چرخ فیروزه طربخانه از این کهگل کرد
\\
آه و فریاد که از چشم حسود مه چرخ
&&
در لحد ماه کمان ابروی من منزل کرد
\\
نزدی شاه رخ و فوت شد امکان حافظ
&&
چه کنم بازی ایام مرا غافل کرد
\\
\end{longtable}
\end{center}
