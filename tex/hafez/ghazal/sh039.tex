\begin{center}
\section*{غزل شماره ۳۹: باغ مرا چه حاجت سرو و صنوبر است}
\label{sec:sh039}
\addcontentsline{toc}{section}{\nameref{sec:sh039}}
\begin{longtable}{l p{0.5cm} r}
باغ مرا چه حاجت سرو و صنوبر است
&&
شمشاد خانه پرور ما از که کمتر است
\\
ای نازنین پسر تو چه مذهب گرفته‌ای
&&
کت خون ما حلالتر از شیر مادر است
\\
چون نقش غم ز دور ببینی شراب خواه
&&
تشخیص کرده‌ایم و مداوا مقرر است
\\
از آستان پیر مغان سر چرا کشیم
&&
دولت در آن سرا و گشایش در آن در است
\\
یک قصه بیش نیست غم عشق وین عجب
&&
کز هر زبان که می‌شنوم نامکرر است
\\
دی وعده داد وصلم و در سر شراب داشت
&&
امروز تا چه گوید و بازش چه در سر است
\\
شیراز و آب رکنی و این باد خوش نسیم
&&
عیبش مکن که خال رخ هفت کشور است
\\
فرق است از آب خضر که ظلمات جای او است
&&
تا آب ما که منبعش الله اکبر است
\\
ما آبروی فقر و قناعت نمی‌بریم
&&
با پادشه بگوی که روزی مقدر است
\\
حافظ چه طرفه شاخ نباتیست کلک تو
&&
کش میوه دلپذیرتر از شهد و شکر است
\\
\end{longtable}
\end{center}
