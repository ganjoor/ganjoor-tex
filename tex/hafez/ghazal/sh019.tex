\begin{center}
\section*{غزل شماره ۱۹: ای نسیم سحر آرامگه یار کجاست}
\label{sec:sh019}
\addcontentsline{toc}{section}{\nameref{sec:sh019}}
\begin{longtable}{l p{0.5cm} r}
ای نسیم سحر آرامگه یار کجاست
&&
منزل آن مه عاشق کش عیار کجاست
\\
شب تار است و ره وادی ایمن در پیش
&&
آتش طور کجا موعد دیدار کجاست
\\
هر که آمد به جهان نقش خرابی دارد
&&
در خرابات بگویید که هشیار کجاست
\\
آن کس است اهل بشارت که اشارت داند
&&
نکته‌ها هست بسی محرم اسرار کجاست
\\
هر سر موی مرا با تو هزاران کار است
&&
ما کجاییم و ملامت گر بی‌کار کجاست
\\
بازپرسید ز گیسوی شکن در شکنش
&&
کاین دل غمزده سرگشته گرفتار کجاست
\\
عقل دیوانه شد آن سلسله مشکین کو
&&
دل ز ما گوشه گرفت ابروی دلدار کجاست
\\
ساقی و مطرب و می جمله مهیاست ولی
&&
عیش بی یار مهیا نشود یار کجاست
\\
حافظ از باد خزان در چمن دهر مرنج
&&
فکر معقول بفرما گل بی خار کجاست
\\
\end{longtable}
\end{center}
