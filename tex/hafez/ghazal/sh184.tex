\begin{center}
\section*{غزل شماره ۱۸۴: دوش دیدم که ملایک در میخانه زدند}
\label{sec:sh184}
\addcontentsline{toc}{section}{\nameref{sec:sh184}}
\begin{longtable}{l p{0.5cm} r}
دوش دیدم که ملایک در میخانه زدند
&&
گل آدم بسرشتند و به پیمانه زدند
\\
ساکنان حرم ستر و عفاف ملکوت
&&
با من راه نشین باده مستانه زدند
\\
آسمان بار امانت نتوانست کشید
&&
قرعه کار به نام من دیوانه زدند
\\
جنگ هفتاد و دو ملت همه را عذر بنه
&&
چون ندیدند حقیقت ره افسانه زدند
\\
شکر ایزد که میان من و او صلح افتاد
&&
صوفیان رقص کنان ساغر شکرانه زدند
\\
آتش آن نیست که از شعله او خندد شمع
&&
آتش آن است که در خرمن پروانه زدند
\\
کس چو حافظ نگشاد از رخ اندیشه نقاب
&&
تا سر زلف سخن را به قلم شانه زدند
\\
\end{longtable}
\end{center}
