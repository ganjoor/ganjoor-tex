\begin{center}
\section*{غزل شماره ۲۱۵: به کوی میکده یا رب سحر چه مشغله بود}
\label{sec:sh215}
\addcontentsline{toc}{section}{\nameref{sec:sh215}}
\begin{longtable}{l p{0.5cm} r}
به کوی میکده یا رب سحر چه مشغله بود
&&
که جوش شاهد و ساقی و شمع و مشعله بود
\\
حدیث عشق که از حرف و صوت مستغنیست
&&
به ناله دف و نی در خروش و ولوله بود
\\
مباحثی که در آن مجلس جنون می‌رفت
&&
ورای مدرسه و قال و قیل مسئله بود
\\
دل از کرشمه ساقی به شکر بود ولی
&&
ز نامساعدی بختش اندکی گله بود
\\
قیاس کردم و آن چشم جادوانه مست
&&
هزار ساحر چون سامریش در گله بود
\\
بگفتمش به لبم بوسه‌ای حوالت کن
&&
به خنده گفت کی ات با من این معامله بود
\\
ز اخترم نظری سعد در ره است که دوش
&&
میان ماه و رخ یار من مقابله بود
\\
دهان یار که درمان درد حافظ داشت
&&
فغان که وقت مروت چه تنگ حوصله بود
\\
\end{longtable}
\end{center}
