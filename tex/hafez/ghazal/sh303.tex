\begin{center}
\section*{غزل شماره ۳۰۳: شممت روح وداد و شمت برق وصال}
\label{sec:sh303}
\addcontentsline{toc}{section}{\nameref{sec:sh303}}
\begin{longtable}{l p{0.5cm} r}
شَمَمتُ روحَ وِدادٍ و شِمتُ برقَ وصال
&&
بیا که بوی تو را میرم ای نسیم شمال
\\
اَحادیاً بجمالِ الحبیبِ قِف وانزِل
&&
که نیست صبر جمیلم ز اشتیاق جمال
\\
حکایت شب هجران فروگذاشته به
&&
به شکر آن که برافکند پرده روز وصال
\\
بیا که پردهٔ گلریز هفت خانه چشم
&&
کشیده‌ایم به تحریر کارگاه خیال
\\
چو یار بر سر صلح است و عذر می‌طلبد
&&
توان گذشت ز جور رقیب در همه حال
\\
به جز خیال دهان تو نیست در دل تنگ
&&
که کس مباد چو من در پی خیال محال
\\
قتیل عشق تو شد حافظ غریب ولی
&&
به خاک ما گذری کن که خون مات حلال
\\
\end{longtable}
\end{center}
