\begin{center}
\section*{غزل شماره ۳۲۶: در نهانخانه عشرت صنمی خوش دارم}
\label{sec:sh326}
\addcontentsline{toc}{section}{\nameref{sec:sh326}}
\begin{longtable}{l p{0.5cm} r}
در نهانخانه عشرت صنمی خوش دارم
&&
کز سر زلف و رخش نعل در آتش دارم
\\
عاشق و رندم و میخواره به آواز بلند
&&
وین همه منصب از آن حور پریوش دارم
\\
گر تو زین دست مرا بی سر و سامان داری
&&
من به آه سحرت زلف مشوش دارم
\\
گر چنین چهره گشاید خط زنگاری دوست
&&
من رخ زرد به خونابه منقش دارم
\\
گر به کاشانه رندان قدمی خواهی زد
&&
نقل شعر شکرین و می بی‌غش دارم
\\
ناوک غمزه بیار و رسن زلف که من
&&
جنگ‌ها با دل مجروح بلاکش دارم
\\
حافظا چون غم و شادی جهان در گذر است
&&
بهتر آن است که من خاطر خود خوش دارم
\\
\end{longtable}
\end{center}
