\begin{center}
\section*{غزل شماره ۶۰: آن پیک نامور که رسید از دیار دوست}
\label{sec:sh060}
\addcontentsline{toc}{section}{\nameref{sec:sh060}}
\begin{longtable}{l p{0.5cm} r}
آن پیک نامور که رسید از دیار دوست
&&
آورد حرز جان ز خط مشکبار دوست
\\
خوش می‌دهد نشان جلال و جمال یار
&&
خوش می‌کند حکایت عز و وقار دوست
\\
دل دادمش به مژده و خجلت همی‌برم
&&
زین نقد قلب خویش که کردم نثار دوست
\\
شکر خدا که از مدد بخت کارساز
&&
بر حسب آرزوست همه کار و بار دوست
\\
سیر سپهر و دور قمر را چه اختیار
&&
در گردشند بر حسب اختیار دوست
\\
گر باد فتنه هر دو جهان را به هم زند
&&
ما و چراغ چشم و ره انتظار دوست
\\
کحل الجواهری به من آر ای نسیم صبح
&&
زان خاک نیکبخت که شد رهگذار دوست
\\
ماییم و آستانه عشق و سر نیاز
&&
تا خواب خوش که را برد اندر کنار دوست
\\
دشمن به قصد حافظ اگر دم زند چه باک
&&
منت خدای را که نیم شرمسار دوست
\\
\end{longtable}
\end{center}
