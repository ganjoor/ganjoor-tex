\begin{center}
\section*{غزل شماره ۱۸۲: حسب حالی ننوشتی و شد ایامی چند}
\label{sec:sh182}
\addcontentsline{toc}{section}{\nameref{sec:sh182}}
\begin{longtable}{l p{0.5cm} r}
حسب حالی ننوشتی و شد ایامی چند
&&
محرمی کو که فرستم به تو پیغامی چند
\\
ما بدان مقصد عالی نتوانیم رسید
&&
هم مگر پیش نهد لطف شما گامی چند
\\
چون می از خم به سبو رفت و گل افکند نقاب
&&
فرصت عیش نگه دار و بزن جامی چند
\\
قند آمیخته با گل نه علاج دل ماست
&&
بوسه‌ای چند برآمیز به دشنامی چند
\\
زاهد از کوچه رندان به سلامت بگذر
&&
تا خرابت نکند صحبت بدنامی چند
\\
عیب می جمله چو گفتی هنرش نیز بگو
&&
نفی حکمت مکن از بهر دل عامی چند
\\
ای گدایان خرابات خدا یار شماست
&&
چشم انعام مدارید ز انعامی چند
\\
پیر میخانه چه خوش گفت به دردی کش خویش
&&
که مگو حال دل سوخته با خامی چند
\\
حافظ از شوق رخ مهر فروغ تو بسوخت
&&
کامگارا نظری کن سوی ناکامی چند
\\
\end{longtable}
\end{center}
