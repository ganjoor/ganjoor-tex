\begin{center}
\section*{غزل شماره ۳۰۶: اگر به کوی تو باشد مرا مجال وصول}
\label{sec:sh306}
\addcontentsline{toc}{section}{\nameref{sec:sh306}}
\begin{longtable}{l p{0.5cm} r}
اگر به کوی تو باشد مرا مجال وصول
&&
رسد به دولت وصل تو کار من به اصول
\\
قرار برده ز من آن دو نرگس رعنا
&&
فراغ برده ز من آن دو جادوی مکحول
\\
چو بر در تو من بینوای بی زر و زور
&&
به هیچ باب ندارم ره خروج و دخول
\\
کجا روم چه کنم چاره از کجا جویم
&&
که گشته‌ام ز غم و جور روزگار ملول
\\
من شکستهٔ بدحال زندگی یابم
&&
در آن زمان که به تیغ غمت شوم مقتول
\\
خرابتر ز دل من غم تو جای نیافت
&&
که ساخت در دل تنگم قرارگاه نزول
\\
دل از جواهر مهرت چو صیقلی دارد
&&
بود ز زنگ حوادث هر آینه مصقول
\\
چه جرم کرده‌ام ای جان و دل به حضرت تو
&&
که طاعت من بیدل نمی‌شود مقبول
\\
به درد عشق بساز و خموش کن حافظ
&&
رموز عشق مکن فاش پیش اهل عقول
\\
\end{longtable}
\end{center}
