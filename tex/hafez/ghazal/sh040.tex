\begin{center}
\section*{غزل شماره ۴۰: المنه لله که در میکده باز است}
\label{sec:sh040}
\addcontentsline{toc}{section}{\nameref{sec:sh040}}
\begin{longtable}{l p{0.5cm} r}
المنة لله که در میکده باز است
&&
زان رو که مرا بر در او روی نیاز است
\\
خم‌ها همه در جوش و خروشند ز مستی
&&
وان می که در آن جاست حقیقت نه مجاز است
\\
از وی همه مستی و غرور است و تکبر
&&
وز ما همه بیچارگی و عجز و نیاز است
\\
رازی که بر غیر نگفتیم و نگوییم
&&
با دوست بگوییم که او محرم راز است
\\
شرح شکن زلف خم اندر خم جانان
&&
کوته نتوان کرد که این قصه دراز است
\\
بار دل مجنون و خم طره لیلی
&&
رخساره محمود و کف پای ایاز است
\\
بردوخته‌ام دیده چو باز از همه عالم
&&
تا دیده من بر رخ زیبای تو باز است
\\
در کعبه کوی تو هر آن کس که بیاید
&&
از قبله ابروی تو در عین نماز است
\\
ای مجلسیان سوز دل حافظ مسکین
&&
از شمع بپرسید که در سوز و گداز است
\\
\end{longtable}
\end{center}
