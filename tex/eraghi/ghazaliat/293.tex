\begin{center}
\section*{غزل شماره ۲۹۳: دلی دارم، چه دل؟ محنت سرایی}
\label{sec:293}
\addcontentsline{toc}{section}{\nameref{sec:293}}
\begin{longtable}{l p{0.5cm} r}
دلی دارم، چه دل؟ محنت سرایی
&&
که در وی خوشدلی را نیست جایی
\\
دل مسکین چرا غمگین نباشد؟
&&
که در عالم نیابد دل‌ربایی
\\
تن مهجور چون رنجور نبود؟
&&
چه تاب کوه دارد رشته تایی؟
\\
چگونه غرق خونابه نباشم؟
&&
که دستم می‌نگیرد آشنایی
\\
بمیرد دل چو دلداری نبیند
&&
بکاهد جان چون نبود جان فزایی
\\
بنالم بلبل‌آسا چون نیابم
&&
ز باغ دلبران بوی وفایی
\\
فتادم باز در وادی خون خوار
&&
نمی‌بینم رهی را رهنمایی
\\
نه دل را در تحیر پای بندی
&&
نه جان را جز تمنی دلگشایی
\\
درین وادی فرو شد کاروان‌ها
&&
که کس نشنید آواز درایی
\\
درین ره هر نفس صد خون بریزد
&&
نیارد خواستن کس خونبهایی
\\
دل من چشم می‌دارد کزین ره
&&
بیابد بهر چشمش توتیایی
\\
روانم نیز در بسته است همت
&&
که بگشاید در راحت سرایی
\\
تنم هم گوش می‌دارد کزین در
&&
به گوش جانش آید مرحبایی
\\
تمنا می‌کند مسکین عراقی
&&
که دریابد بقا بعد از فنایی
\\
\end{longtable}
\end{center}
