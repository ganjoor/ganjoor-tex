\begin{center}
\section*{غزل شماره ۱۳۵: کار ما، بنگر، که خام افتاد باز}
\label{sec:135}
\addcontentsline{toc}{section}{\nameref{sec:135}}
\begin{longtable}{l p{0.5cm} r}
کار ما، بنگر، که خام افتاد باز
&&
کار با پیک و پیام افتاد باز
\\
من چه دانم در میان دوستان
&&
دشمن بد گو کدام افتاد باز؟
\\
این همی دانم که گفت و گوی ما
&&
در زبان خاص و عام افتاد باز
\\
عاشق دیوانه نامم کرده‌اند
&&
بر من آخر این چه نام افتاد باز؟
\\
روز بخت من چو شب تاریک شد
&&
صبح امیدم به شام افتاد باز
\\
توسن دولت، که بودی رام من
&&
آن هم‌اکنون بدلگام افتاد باز
\\
باز اقبال از کف من بر پرید
&&
زاغ ادبارم به دام افتاد باز
\\
مجلس عیش دل‌افروز مرا
&&
باطیه بشکست و جام افتاد باز
\\
در گلستان می‌گذشتم صبحدم
&&
بوی یارم در مشام افتاد باز
\\
در سر سودای زلفش شد دلم
&&
مرغ صحرایی به دام افتاد باز
\\
تا بدیدم عکس او در جام می
&&
در سرم سودای خام افتاد باز
\\
تا چشیدم جرعه‌ای از جام می
&&
در دلم مهر مدام افتاد باز
\\
من چو از سودای خوبان سوختم
&&
پس عراقی از چه خام افتاد باز؟
\\
\end{longtable}
\end{center}
