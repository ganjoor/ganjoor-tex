\begin{center}
\section*{غزل شماره ۱۳۹: تماشا می‌کند هر دم دلم در باغ رخسارش}
\label{sec:139}
\addcontentsline{toc}{section}{\nameref{sec:139}}
\begin{longtable}{l p{0.5cm} r}
تماشا می‌کند هر دم دلم در باغ رخسارش
&&
به کام دل همی نوشد می لعل شکر بارش
\\
دلی دارم، مسلمانان، چو زلف یار سودایی
&&
همه در بند آن باشد که گردد گرد رخسارش
\\
چه خوش باشد دل آن لحظه! که در باغ جمال او
&&
گهی گل چیند از رویش، گهی شکر ز گفتارش
\\
گهی در پای او غلتان چو زلف بی‌قرار او
&&
گه‌از خال لبش سرمست همچون چشم خونخوارش
\\
از آن خوشتر تماشایی تواند بود در عالم
&&
که بیند دیدهٔ عاشق به خلوت روی دلدارش؟
\\
چنان سرمست شد جانم ز جام عشق جانانم
&&
که تا روز قیامت هم نخواهی یافت هشیارش
\\
بهار و باغ و گلزار عراقی روی جانان است
&&
ز صد خلد برین خوشتر بهار و باغ و گلزارش
\\
\end{longtable}
\end{center}
