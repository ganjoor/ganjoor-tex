\begin{center}
\section*{غزل شماره ۲۳۴: ای جمالت برقع از رخ ناگهان انداخته}
\label{sec:234}
\addcontentsline{toc}{section}{\nameref{sec:234}}
\begin{longtable}{l p{0.5cm} r}
ای جمالت برقع از رخ ناگهان انداخته
&&
عالمی در شور و شوری در جهان انداخته
\\
عشق رویت رستخیزی از زمین انگیخته
&&
آرزویت غلغلی در آسمان انداخته
\\
چشم بد از تاب رویت آتشی افروخته
&&
چون سپندی جان مشتاقان در آن انداخته
\\
روی بنموده جمالت، باز پنهان کرده رخ
&&
در دل بیچارگان شور و فغان انداخته
\\
دیدن رویت، که دیرینه تمنای دل است
&&
آرزویی در دل این ناتوان انداخته
\\
چند باشد بی‌دلی در آرزوی روی تو؟
&&
بر سر کوی تو سر بر آستان انداخته
\\
بی‌تو عمرم شد، دریغا! و چه حاصل از دریغ؟
&&
چون نیاید باز تیر از کمان انداخته
\\
مانده‌ام در چاه هجران، پای در دنبال مار
&&
دست در کام نهنگ جان ستان انداخته
\\
هیچ بینم باز در حلق عراقی ناگهان
&&
جذبه‌های دلربایی ریسمان انداخته؟
\\
\end{longtable}
\end{center}
