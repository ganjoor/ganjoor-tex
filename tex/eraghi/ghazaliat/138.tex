\begin{center}
\section*{غزل شماره ۱۳۸: در بزم قلندران قلاش}
\label{sec:138}
\addcontentsline{toc}{section}{\nameref{sec:138}}
\begin{longtable}{l p{0.5cm} r}
در بزم قلندران قلاش
&&
بنشین و شراب نوش و خوش باش
\\
تا ذوق می و خمار یابی
&&
باید که شوی تو نیز قلاش
\\
در صومعه چند خود پرستی؟
&&
رو باده‌پرست شو چو اوباش
\\
در جام جهان‌نمای می بین
&&
سر دو جهان، ولی مکن فاش
\\
ور خود نظری کنی به ساقی
&&
سرمست شوی ز چشم رعناش
\\
جز نقش نگار هر چه بینی
&&
از لوح ضمیر پاک بخراش
\\
باشد که ببینی، ای عراقی،
&&
در نقش وجود خویش نقاش
\\
\end{longtable}
\end{center}
