\begin{center}
\section*{غزل شماره ۲۹۶: زهی! جمال تو رشک بتان یغمایی}
\label{sec:296}
\addcontentsline{toc}{section}{\nameref{sec:296}}
\begin{longtable}{l p{0.5cm} r}
زهی! جمال تو رشک بتان یغمایی
&&
وصال تو هوس عاشقان شیدایی
\\
عروس حسن تو را هیچ در نمی‌یابد
&&
به گاه جلوه‌گری دیدهٔ تماشایی
\\
بدین صفت که تویی بر جمال خود عاشق
&&
به غیر خود، نه همانا، که روی بنمایی
\\
حجاب روی تو هم روی توست در همه حال
&&
نهانی از همه عالم، ز بسکه پیدایی
\\
بهر چه می‌نگرم صورت تو می‌بینم
&&
ازین میان همه در چشم من تو می‌آیی
\\
همه جهان به تو می‌بینم و عجب نبود
&&
ازان سبب که تویی در دو دیده بینایی
\\
ز رشک تا نشناسد تو را کسی، هر دم
&&
جمال خود به لباس دگر بیارایی
\\
تو را چگونه توان یافت؟ در تو خود که رسد؟
&&
که هر نفس به دگر منزل و دگر جایی
\\
عراقی از پی تو دربه در همی گردد
&&
تو خود مقیم میان دلش هویدایی
\\
\end{longtable}
\end{center}
