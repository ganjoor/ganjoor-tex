\begin{center}
\section*{غزل شماره ۲۸۰: ترسا بچه‌ای، شنگی، شوخی، شکرستانی}
\label{sec:280}
\addcontentsline{toc}{section}{\nameref{sec:280}}
\begin{longtable}{l p{0.5cm} r}
ترسا بچه‌ای، شنگی، شوخی، شکرستانی
&&
در هر خم زلف او گمراه مسلمانی
\\
از حسن و جمال او حیرت زده هر عقلی
&&
وز ناز و دلال او واله شده هر جانی
\\
بر لعل شکر ریزش آشفته هزاران دل
&&
وز زلف دلاویزش آویخته هر جانی
\\
چشم خوش سرمستش اندر پی هر دینی
&&
زنار سر زلفش دربند هر ایمانی
\\
بر مائدهٔ عیسی افزوده لبش حلوا
&&
وز معجزهٔ موسی زلفش شده ثعبانی
\\
ترسا به چه‌ای رعنا، از منطق روح‌افزا
&&
صد معجزهٔ عیسی بنموده به برهانی
\\
لعلش ز شکر خنده در مرده دمیده جان
&&
چشمش ز سیه کاری برده دل کیهانی
\\
عیسی نفسی، کز لب در مرده دمد صد جان
&&
بهر چه بود دلها هر لحظه به دستانی؟
\\
تا سیر نیارد دید نظارگی رویش
&&
بگماشته از غمزه هر گوشه نگهبانی
\\
از چشم روان کرده بهر دل مشتاقان
&&
از هر نظری تیری وز هر مژه پیکانی
\\
از دیر برون آمد از خوبی خود سرمست
&&
هر کس که بدید او را واله شد و حیرانی
\\
شماس چو رویش خورشید پرستی شد
&&
زاهد هم اگر دیدی رهبان شدی آسانی
\\
ور زانکه به چشم من صوفی رخ او دیدی
&&
خورشید پرستیدی، در دیر، چو رهبانی
\\
یاد لب و دندانش بر خاطر من بگذشت
&&
چشمم گهرافشان شد، طبعم شکرستانی
\\
جان خواستم افشاندن پیش رخ او دل گفت:
&&
خاری چه محل دارد در پیش گلستانی؟
\\
گر خاک رهش گردم هم پا ننهد بر من
&&
کی پای نهد، حاشا، بر مور سلیمانی؟
\\
زین پس نرود ظلمی بر آدم ازین دیوان
&&
زیرا که سلیمان شد فرماندهٔ دیوانی
\\
نه بس که عراقی را بینی تو ز نظم تر
&&
در وصف جمال او پرداخته دیوانی
\\
\end{longtable}
\end{center}
