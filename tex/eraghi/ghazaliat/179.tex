\begin{center}
\section*{غزل شماره ۱۷۹: کجایی، ای دل و جانم، که از غم تو بجانم}
\label{sec:179}
\addcontentsline{toc}{section}{\nameref{sec:179}}
\begin{longtable}{l p{0.5cm} r}
کجایی، ای دل و جانم، که از غم تو بجانم
&&
بیا، که بی رخ خوب تو بیش می‌نتوانم
\\
بیا، ببین، نه همانا که زنده خواهم ماندن
&&
تو خود بگوی که: بی تو چگونه زنده بمانم؟
\\
چگونه باشد در دام مانده حیران صید
&&
ز جان امید بریده؟ ز دوری تو چنانم
\\
هوات تا ز من دلشده چه برد؟ چه گویم
&&
جفات تا به من غمزده چه کرد؟ چه دانم؟
\\
ببرد این دل و اندر میان بحر غم افگند
&&
سپرد آن به کف صد بلا و رنج روانم
\\
بلا به پیش خیال تو گفت دوش دل من
&&
که: پای پیشترک نه، ز خویشتن برهانم
\\
ز گوشه‌ای غم تو گفت: می‌خورم غم کارت
&&
ز جانبی ستمت گفت: غم مخور که در آنم
\\
درین غمم که: عراقی چگونه خواهد مردن؟
&&
ندیده سیر رخ تو، برای او نگرانم
\\
\end{longtable}
\end{center}
