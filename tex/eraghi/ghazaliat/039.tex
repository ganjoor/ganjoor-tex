\begin{center}
\section*{غزل شماره ۳۹: دل، که دایم عشق می‌ورزید رفت}
\label{sec:039}
\addcontentsline{toc}{section}{\nameref{sec:039}}
\begin{longtable}{l p{0.5cm} r}
دل، که دایم عشق می‌ورزید رفت
&&
گفتمش: جانا مرو، نشنید رفت
\\
هر کجا بوی دلارامی شنید
&&
یا رخ خوب نگاری دید رفت
\\
هرکجا شکرلبی دشنام داد
&&
یا نگاری زیر لب خندید رفت
\\
در سر زلف بتان شد عاقبت
&&
در کنار مهوشی غلتید رفت
\\
دل چو آرام دل خود بازیافت
&&
یک نفس با من نیارامید رفت
\\
چون لب و دندان دلدارم بدید
&&
در سر آن لعل و مروارید رفت
\\
دل ز جان و تن کنون دل برگرفت
&&
از بد و نیک جهان ببرید رفت
\\
عشق می‌ورزید دایم، لاجرم
&&
در سر چیزی که می‌ورزید رفت
\\
باز کی یابم دل گم گشته را؟
&&
دل که در زلف بتان پیچید رفت
\\
بر سر جان و جهان چندین ملرز
&&
آنکه شایستی بدو لرزید رفت
\\
ای عراقی، چند زین فریاد و سوز؟
&&
دلبرت یاری دگر بگزید رفت
\\
\end{longtable}
\end{center}
