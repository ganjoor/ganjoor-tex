\begin{center}
\section*{غزل شماره ۸۸: ناگه بت من مست به بازار برآمد}
\label{sec:088}
\addcontentsline{toc}{section}{\nameref{sec:088}}
\begin{longtable}{l p{0.5cm} r}
ناگه بت من مست به بازار برآمد
&&
شور از سر بازار به یکبار برآمد
\\
بس دل که به کوی غم او شاد فروشد
&&
بس جان که ز عشق رخ او زار برآمد
\\
در صومعه و بتکده عشقش گذری کرد
&&
مؤمن ز دل و گبر و ز زنار برآمد
\\
در کوی خرابات جمالش نظر افگند
&&
شور و شغبی از در خمار برآمد
\\
در وقت مناجات خیال رخش افروخت
&&
فریاد و فغان از دل ابرار برآمد
\\
یک جرعه ز جام لب او می‌زده‌ای یافت
&&
سرمست و خرامان به سر دار برآمد
\\
در سوخته‌ای آتش شمع رخش افتاد
&&
از سوز دلش شعلهٔ انوار برآمد
\\
باد در او سر آتش گذری کرد
&&
از آتش سوزان گل بی خوار برآمد
\\
ناگاه ز رخسار شبی پرده برانداخت
&&
صد مهر ز هر سو به شب تار برآمد
\\
باد سحر از خاک درش کرد حکایت
&&
صد نالهٔ زار از دل بیمار برآمد
\\
کی بو که فروشد لب او بوسه به جانی؟
&&
کز بوک و مگر جان خریدار برآمد
\\
\end{longtable}
\end{center}
