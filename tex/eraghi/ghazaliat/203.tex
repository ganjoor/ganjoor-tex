\begin{center}
\section*{غزل شماره ۲۰۳: ای دوست، بیا، که ما توراییم}
\label{sec:203}
\addcontentsline{toc}{section}{\nameref{sec:203}}
\begin{longtable}{l p{0.5cm} r}
ای دوست، بیا، که ما توراییم
&&
بیگانه مشو، که آشناییم
\\
رخ بازنمای، تا ببینیم
&&
در بازگشای، تا درآییم
\\
هر چند نه‌ایم در خور تو
&&
لیکن چه کنیم؟ مبتلاییم
\\
چون بی‌تو نه‌ایم زنده یک دم
&&
پیوسته چرا ز تو جداییم؟
\\
چون عکس جمال تو ندیدیم
&&
بر روی تو شیفته چراییم؟
\\
آن کس که ندیده روی خوبت
&&
در حسرت تو بمرد، ماییم
\\
ماییم کنون و نیم جانی
&&
بپذیر ز ما، که بی‌نواییم
\\
تا دور شدیم از بر تو
&&
دور از تو همیشه در بلاییم
\\
بس لایق و در خوری تو ما را
&&
هر چند که ما تو را نشاییم
\\
آنچ از تو سزد به جای ما کن
&&
نه آنچه که ما بدان سزاییم
\\
هم زان توایم، هر چه هستیم
&&
گر محتشمیم و گر گداییم
\\
از عشق رخ تو چون عراقی
&&
هر دم غزلی دگر سراییم
\\
\end{longtable}
\end{center}
