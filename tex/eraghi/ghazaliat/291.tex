\begin{center}
\section*{غزل شماره ۲۹۱: چه بود گر نقاب بگشایی}
\label{sec:291}
\addcontentsline{toc}{section}{\nameref{sec:291}}
\begin{longtable}{l p{0.5cm} r}
چه بود گر نقاب بگشایی؟
&&
بی‌دلان را جمال بنمایی؟
\\
مفلسان را نظاره‌ای بخشی؟
&&
خستگان را دمی ببخشایی؟
\\
عمر ما شد، دریغ! ناشده ما
&&
بر سر کوی تو تماشایی
\\
با وصالت نپخته سودایی
&&
از فراغت شدیم سودایی
\\
چه توان کرد؟ یار می‌نشنوی
&&
هیچ باشد که یار ما آیی؟
\\
جان را به چهره شاد کنی؟
&&
دل ما را به غمزه بربایی؟
\\
بی تومان جان و دل نمی‌باید
&&
دل ما را به جان تو می‌بایی
\\
پرده بردار، تا سر اندازیم
&&
به سر کوی تو، ز شیدایی
\\
ور بر آنی که خون ما ریزی
&&
غمزه را حکم کن، چه می‌پایی؟
\\
مفلسانیم بر درت عاجز
&&
منتظر گشته تا چه فرمایی؟
\\
چون عراقی امید در بسته
&&
تا در بسته، بو که، بگشایی
\\
\end{longtable}
\end{center}
