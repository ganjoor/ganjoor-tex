\begin{center}
\section*{غزل شماره ۲۹۷: سحرگه بر در راحت سرایی}
\label{sec:297}
\addcontentsline{toc}{section}{\nameref{sec:297}}
\begin{longtable}{l p{0.5cm} r}
سحرگه بر در راحت سرایی
&&
گذر کردم شنیدم مرحبایی
\\
درون رفتم، ندیمی چند دیدم
&&
همه سر مست عشق دلربایی
\\
همه از بیخودی خوش وقت بودند
&&
همه ز آشفتگی در هوی و هایی
\\
ز رنگ نیستی شان رنگ و بویی
&&
ز برگ بی‌نوایی‌شان نوایی
\\
ز سدره برتر ایشان را مقامی
&&
ورای عرش و کرسی متکایی
\\
نشسته بر سر خوان فتوت
&&
بهر دو کون در داده صلایی
\\
نظر کردم، ندیدم ملک ایشان
&&
درین عالم، به جز تن، رشته‌تایی
\\
ز حیرت در همه گم گشته از خود
&&
ولی در عشق هر یک رهنمایی
\\
مرا گفتند: حالی چیست؟ گفتم:
&&
چه پرسی حال مسکین گدایی؟
\\
\end{longtable}
\end{center}
