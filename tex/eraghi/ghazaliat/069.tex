\begin{center}
\section*{غزل شماره ۶۹: تا کی کشم جفای تو؟ این نیز بگذرد}
\label{sec:069}
\addcontentsline{toc}{section}{\nameref{sec:069}}
\begin{longtable}{l p{0.5cm} r}
تا کی کشم جفای تو؟ این نیز بگذرد
&&
بسیار شد بلای تو، این نیز بگذرد
\\
عمرم گذشت و یک نفسم بیشتر نماند
&&
خوش باش کز جفای تو، این نیز بگذرد
\\
آیی و بگذری به من و باز ننگری
&&
ای جان من فدای تو، این نیز بگذرد
\\
هر کس رسید از تو به مقصود و این گدا
&&
محروم از عطای تو، این نیز بگذرد
\\
ای دوست، تو مرا همه دشنام می‌دهی
&&
من می‌کنم، دعای تو، این نیز بگذرد
\\
آیم به درگهت، نگذاری که بگذرم
&&
پیرامن سرای تو، این نیز بگذرد
\\
آمدم دلم به کوی تو، نومید بازگشت
&&
نشنید مرحبای تو، این نیز بگذرد
\\
بگذشت آنکه دوست همی داشتی مرا
&&
دیگر شده است رای تو، این نیز بگذرد
\\
تا کی کشد عراقی مسکین جفای تو؟
&&
بگذشت چون جفای تو، این نیز بگذرد
\\
\end{longtable}
\end{center}
