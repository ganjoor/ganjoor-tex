\begin{center}
\section*{غزل شماره ۲۰۸: مبتلای هجر یارم، الغیاث ای دوستان}
\label{sec:208}
\addcontentsline{toc}{section}{\nameref{sec:208}}
\begin{longtable}{l p{0.5cm} r}
مبتلای هجر یارم، الغیاث ای دوستان
&&
از فراقش سخت زارم، الغیاث ای دوستان
\\
می‌تپم چون مرغ بسمل در میان خاک و خون
&&
ننگرد در من نگارم، الغیاث ای دوستان
\\
از فراق خویش همچون دشمنانم می‌کشد
&&
زانکه او را دوست دارم، الغیاث ای دوستان
\\
دیده‌اید آخر که چون بودم عزیز در گهش؟
&&
بنگرید اکنون چه خوارم؟ الغیاث ای دوستان
\\
غصه‌های نامرادی می‌کشم از دست او
&&
زهره نه کهی برآرم، الغیاث ای دوستان
\\
یاد نارد از من مسکین، نپرسد حال من
&&
هم چنین یار است یارم، الغیاث ای دوستان
\\
هم به نگذارد مرا تا با سگان کوی او
&&
روزگاری می‌گذارم، الغیاث ای دوستان
\\
قصه‌ها دارم ز جور او میان جان نهان
&&
با کسی گفتن نیارم، الغیاث ای دوستان
\\
جان فرستم تحفه نزد یار و نپذیرد ز من
&&
غم فرستد یادگارم، الغیاث ای دوستان
\\
باز پرسد از من بیچارهٔ ماتم زده
&&
کز فراقش سوکوارم؟ الغیاث ای دوستان
\\
یار من باشید، کز ننگ عراقی وارهم
&&
کز پی او شرمسارم الغیاث ای دوستان
\\
\end{longtable}
\end{center}
