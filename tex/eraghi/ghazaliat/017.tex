\begin{center}
\section*{غزل شماره ۱۷: از میکده تا چه شور برخاست}
\label{sec:017}
\addcontentsline{toc}{section}{\nameref{sec:017}}
\begin{longtable}{l p{0.5cm} r}
از میکده تا چه شور برخاست؟
&&
کاندر همه شهر شور و غوغاست
\\
باری، به نظاره‌ای برون آی
&&
کان روی تو از در تماشاست
\\
پنهان چه شوی؟ که عکس رویت
&&
در جام جهان نمای پیداست
\\
گل گر ز رخ تو رنگ ناورد
&&
رنگ رخش آخر از چه زیباست؟
\\
ور نه به جمال تو نظر کرد
&&
چشم خوش نرگس از چه بیناست؟
\\
ور سرو نه قامت تو دیده است
&&
او را کشش از چه سوی بالاست
\\
تا یافت بنفشه بوی زلفت
&&
ما را همه میل سوی صحراست
\\
ما را چه ز باغ لاله و گل؟
&&
از جام، غرض می مصفاست
\\
جز حسن و جمال تو نبیند
&&
از گلشن و لاله هر که بیناست
\\
\end{longtable}
\end{center}
