\begin{center}
\section*{غزل شماره ۱۳۴: از غم عشقت جگر خون است باز}
\label{sec:134}
\addcontentsline{toc}{section}{\nameref{sec:134}}
\begin{longtable}{l p{0.5cm} r}
از غم عشقت جگر خون است باز
&&
خود بپرس از دل که او چون است باز؟
\\
هر زمان از غمزهٔ خونریز تو
&&
بر دل من صد شبیخون است باز
\\
تا سر زلف تو را دل جای کرد
&&
از سرای عقل بیرون است باز
\\
حال دل بودی پریشان پیش ازین
&&
نی چنین درهم که اکنون است باز
\\
از فراق تو برای درد دل
&&
صد بلا و غصه معجون است باز
\\
تا جگر خون کردی، ای جان، ز انتظار
&&
روزی دل، بی‌جگر خون است باز
\\
از برای دل ببار، ای دیده خون
&&
زان که حال او دگرگون است باز
\\
گر چه می‌کاهد غم تو جان و دل
&&
لیک مهرت هر دم افزون است باز
\\
من چو شادم از غم و تیمار تو
&&
پس عراقی از چه محزون است باز؟
\\
\end{longtable}
\end{center}
