\begin{center}
\section*{غزل شماره ۱۵۳: مبند، ای دل، بجز در یار خود دل}
\label{sec:153}
\addcontentsline{toc}{section}{\nameref{sec:153}}
\begin{longtable}{l p{0.5cm} r}
مبند، ای دل، به جز در یار خود دل
&&
امید از هر که داری جمله بگسل
\\
ز منزلگاه دونان رخت بربند
&&
ورای هر دو عالم جوی منزل
\\
برون کن از درون سودای گیتی
&&
ازین سودا به جز سودا چه حاصل؟
\\
منه دل بر چنین محنت سرایی
&&
که هرگز زو نیابی راحت دل
\\
دل از جان و جهان بردار کلی
&&
نخست آنگه قدم زن در مراحل
\\
که راهی بس خطرناک است و تاریک
&&
که کاری سخت دشوار است و مشکل
\\
نمی‌بینی چو روی دوست، باری
&&
حجابی پیش روی خود فروهل
\\
ز شوق او تپان می‌باش پیوست
&&
میان خاک و خون، چون مرغ بسمل
\\
چو روی حق نبینی دیده بر دوز
&&
نباید دید، باری، روی باطل
\\
تو هم بربند بار خود از آنجا
&&
که همراهانت بربستند محمل
\\
قدم بر فرق عالم نه، عراقی،
&&
نمانی تا درینجا پای در گل
\\
\end{longtable}
\end{center}
