\begin{center}
\section*{غزل شماره ۷۷: من رنجور را یک دم نپرسد یار چتوان کرد}
\label{sec:077}
\addcontentsline{toc}{section}{\nameref{sec:077}}
\begin{longtable}{l p{0.5cm} r}
من رنجور را یک دم نپرسد یار چتوان کرد؟
&&
نگوید: چون شد آخر آن دل بیمار چتوان کرد؟
\\
تنم از رنج بگدازد، دلم از غم به جان آرد
&&
چنین است، ای مسلمانان مرا غمخوار چتوان کرد؟
\\
ز داروخانهٔ لطفش چو دارو جان نمی‌یابد
&&
بسازم با غم دردش بنالم زار چتوان کرد؟
\\
دلا، بر من همین باشد که جان در راه او بازم
&&
اگر آن ماه ننماید مرا رخسار چتوان کرد؟
\\
چو از خوان وصال او ندارم جز جگر قوتی
&&
بخایم هم از بن دندان جگر ناچار چتوان کرد؟
\\
سحرگاهان به کوی او بسی رفتم به بوی او
&&
بسی گفتم: قبولم کن، نکرد آن یار چتوان کرد؟
\\
چنان نالیدم از شوقش که شد بیدار همسایه
&&
ز خواب این دیدهٔ بختم نشد بیدار چتوان کرد؟
\\
مرا چون نیست از عشقش به جز تیمار و غم روزی
&&
ضرورت می‌خورم هر دم غم و تیمار چتوان کرد؟
\\
عراقی نیک می‌خواهد که فخر عالمی باشد
&&
ولیکن یار می‌خواهد که باشد عار چتوان کرد؟
\\
\end{longtable}
\end{center}
