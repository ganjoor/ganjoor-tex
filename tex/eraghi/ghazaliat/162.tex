\begin{center}
\section*{غزل شماره ۱۶۲: دل گم شد، ازو نشان نمی‌یابم}
\label{sec:162}
\addcontentsline{toc}{section}{\nameref{sec:162}}
\begin{longtable}{l p{0.5cm} r}
دل گم شد، ازو نشان نمی‌یابم
&&
آن گم شده در جهان نمی‌یابم
\\
زان یوسف گم شده به عالم در
&&
پیدا و نهان نشان نمی‌یابم
\\
تا گوهر شب چراغ گم کردم
&&
ره بر در دوستان نمی‌یابم
\\
تا بلبل خوش نوا ز باغم رفت
&&
بوی گل و گلستان نمی‌یابم
\\
تا آب حیات رفت از جویم
&&
عیش خوش جاودان نمی‌یابم
\\
سیر آمدم از حیات خود، زیراک
&&
بی او ز حیات آن نمی‌یابم
\\
سرمایه برفت و سود می‌جویم
&&
زان است که جز زیان نمی‌یابم
\\
آن یوسف خویش را کجا جویم
&&
چون در همه کن فکان نمی‌یابم
\\
هم بر در دوست باشد ار باشد
&&
از خود بجزین گمان نمی‌یابم
\\
بر خاک درش روم بنالم زار
&&
چاره به جز از فغان نمی‌یابم
\\
چون جانش عزیز دارم، ار یابم
&&
دل، کز غم او امان نمی‌یابم
\\
تا بر من دلشده بگرید زار
&&
یک مشفق مهربان نمی‌یابم
\\
تا یک نفسی مرا دهد یاری
&&
یک یار درین زمان نمی‌یابم
\\
یاری ده خویشتن درین ماتم
&&
جز دیدهٔ خون‌فشان نمی‌یابم
\\
بر خوان جهان چه می‌نشینم من؟
&&
چون لقمه جز استخوان نمی‌یابم
\\
برخیزم ازین جهان بی حاصل
&&
نقدی چو درین دکان نمی‌یابم
\\
خواهم که شوم به بام عالم بر
&&
چه چاره؟ که نردبان نمی‌یابم
\\
خواهم که کشم ز چه عراقی را
&&
افسوس که ریسمان نمی‌یابم
\\
\end{longtable}
\end{center}
