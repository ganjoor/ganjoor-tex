\begin{center}
\section*{غزل شماره ۱۵۹: یاران، غمم خورید، که غمخوار مانده‌ام}
\label{sec:159}
\addcontentsline{toc}{section}{\nameref{sec:159}}
\begin{longtable}{l p{0.5cm} r}
یاران، غمم خورید، که غمخوار مانده‌ام
&&
در دست هجر یار گرفتار مانده‌ام
\\
یاری دهید، کز در او دور گشته‌ام
&&
رحمی کنید، کز غم او زار مانده‌ام
\\
یاران من ز بادیه آسان گذشته‌اند
&&
من بی‌رفیق در ره دشوار مانده‌ام
\\
در راه باز مانده‌ام، ار یار دیدمی
&&
با او بگفتمی که: من از یار مانده‌ام
\\
دستم بگیر، کز غمت افتاده‌ام ز پای
&&
کارم کنون بساز، که از کار مانده‌ام
\\
وقت است اگر به لطف دمی دست گیریم
&&
کاندر چه فراق نگونسار مانده‌ام
\\
ور در خور وصال نیم مرهمی فرست
&&
از درد خویشتن، که دل‌افگار مانده‌ام
\\
دردت چو می‌دهد دل بیمار را شفا
&&
من بر امید درد تو بیمار مانده‌ام
\\
بیمار پرسش از تو نیاید، به درد گو:
&&
تا باز پرسدم، که جگرخوار مانده‌ام
\\
مانا که بر در تو عراقی عزیز نیست
&&
کز صحبتش همیشه چنین خوار مانده‌ام
\\
\end{longtable}
\end{center}
