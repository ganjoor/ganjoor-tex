\begin{center}
\section*{غزل شماره ۲۱۵: عاشقی دانی چه باشد؟ بی‌دل و جان زیستن}
\label{sec:215}
\addcontentsline{toc}{section}{\nameref{sec:215}}
\begin{longtable}{l p{0.5cm} r}
عاشقی دانی چه باشد؟ بی‌دل و جان زیستن
&&
جان و دل بر باختن، بر روی جانان زیستن
\\
سوختن در هجر و خوش بودن به امید وصال
&&
ساختن با درد و پس با بوی درمان زیستن
\\
تا کی از هجران جانان ناله و زاری کنم؟
&&
از حیات خود به جانم، چند ازین سان زیستن؟
\\
بس مرا از زندگانی، مرگ کو، تا جان دهم؟
&&
مرگ خوشتر تا چنین با درد هجران زیستن
\\
ای ز جان خوشتر، بیا، تا بر تو افشانم روان
&&
نزد تو مردن به از تو دور و حیران زیستن
\\
بر سر کویت چه خوش باشد به بوی وصل تو
&&
در میان خاک و خون افتان و خیزان زیستن؟
\\
از خودم دور افگنی، وانگاه گویی: خوش بزی
&&
بی‌دلان را مرگ باشد بی‌تو، ای جان، زیستن
\\
هان! عراقی، جان به جانان ده، گران جانی مکن
&&
بعد از این بی‌روی خوب یار نتوان زیستن
\\
\end{longtable}
\end{center}
