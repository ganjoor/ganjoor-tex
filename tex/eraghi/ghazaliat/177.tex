\begin{center}
\section*{غزل شماره ۱۷۷: ای راحت روانم، دور از تو ناتوانم}
\label{sec:177}
\addcontentsline{toc}{section}{\nameref{sec:177}}
\begin{longtable}{l p{0.5cm} r}
ای راحت روانم، دور از تو ناتوانم
&&
باری، بیا که جان را در پای تو فشانم
\\
این هم روا ندارم کایی برای جانی
&&
بگذار تا برآید در آرزوت جانم
\\
بگذار تا بمیرم در آرزوی رویت
&&
بی روی خوبت آخر تا چند زنده مانم؟
\\
دارم بسی شکایت چون نشنوی چه گویم؟
&&
بیهوده قصهٔ خود در پیش تو چه خوانم؟
\\
گیرم که من نگویم لطف تو خود نگوید:
&&
کین خسته چند نالد هر شب بر آستانم؟
\\
ای بخت خفته، برخیز، تا حال من ببینی
&&
وی عمر رفته، بازآ، تا بشنوی فغانم
\\
ای دوست گاهگاهی میکن به من نگاهی
&&
آخر چو چشم مستت من نیز ناتوانم
\\
بر من همای وصلت سایه از آن نیفکند
&&
کز محنت فراقت پوسیده استخوانم
\\
ای طرفه‌تر که دایم تو با منی و من باز
&&
چون سایه در پی تو گرد جهان دوانم
\\
کس دید تشنه‌ای را غرقه در آب حیوان
&&
جانش به لب رسیده از تشنگی؟ من آنم
\\
زان دم که دور ماندم از درگهت نگفتی:
&&
کاخر شکسته‌ای بد، روزی بر آستانم
\\
هرگز نگفتی، ای جان، کان خسته را بپرسم
&&
وز محنت فراقش یک لحظه وارهانم
\\
اکنون سزد ، نگارا، گر حال من بپرسی
&&
یادم کنی، که این دم دور از تو ناتوانم
\\
بر دست باد کویت بوی خودت فرستی
&&
تا بوی جان فزایت زنده کند روانم
\\
باری، عراقی این دم بس ناخوش است و در هم
&&
حال دلش دگر دم، تا چون شود، چه دانم؟
\\
\end{longtable}
\end{center}
