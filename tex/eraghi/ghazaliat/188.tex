\begin{center}
\section*{غزل شماره ۱۸۸: ما چو قدر وصلت، ای جان و جهان، نشناختیم}
\label{sec:188}
\addcontentsline{toc}{section}{\nameref{sec:188}}
\begin{longtable}{l p{0.5cm} r}
ما چو قدر وصلت، ای جان و جهان، نشناختیم
&&
لاجرم در بوتهٔ هجران تو بگداختیم
\\
ما که از سوز دل و درد جدایی سوختیم
&&
سوز دل را مرهم از مژگان دیده ساختیم
\\
بسکه ما خون جگر خوردیم از دست غمت
&&
جان ما خون گشت و دل در موج خون انداختیم
\\
در سماع دردمندان حاضر آ، یارا، دمی
&&
بشنو این سازی که ما از خون دل بنواختیم
\\
عمری اندر جست‌و جویت دست و پایی می‌زدیم
&&
عمر ما، افسوس، بگذشت و تو را نشناختیم
\\
زان چنین ماندیم اندر ششدر هجرت، که ما
&&
بر بساط راستی نزد وفا کژ باختیم
\\
چون عراقی با غمت دیدیم خوش، ما همچو او
&&
از طرب فارغ شدیم و با غمت پرداختیم
\\
\end{longtable}
\end{center}
