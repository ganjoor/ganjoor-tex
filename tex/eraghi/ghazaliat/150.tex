\begin{center}
\section*{غزل شماره ۱۵۰: تنگ آمدم از وجود خود، تنگ}
\label{sec:150}
\addcontentsline{toc}{section}{\nameref{sec:150}}
\begin{longtable}{l p{0.5cm} r}
تنگ آمدم از وجود خود، تنگ
&&
ای مرگ، به سوی من کن آهنگ
\\
بازم خر ازین غم فراوان
&&
فریاد رسم ازین دل تنگ
\\
تا چند آخر امید یابیم؟
&&
تا کی به امید بوی یا رنگ؟
\\
کی بود که ز خود خلاص یابم
&&
فارغ گردم ز نام و از ننگ؟
\\
افتادم در خلاب محنت
&&
افتان خیزان، چو لاشهٔ لنگ
\\
گر بر در دوست راه جویم
&&
یک گام شود هزار فرسنگ
\\
ور جانب خود کنم نگاهی
&&
در دیدهٔ من فتد دو صد سنگ
\\
ور در ره راستی روم راست
&&
چون در نگرم، روم چو خرچنگ
\\
ور زانکه به سوی گل برم دست
&&
آید همه زخم خار در چنگ
\\
دارم گله‌ها، ولی نه از دوست
&&
از دشمن پر فسون و نیرنگ
\\
با دوست مرا همیشه صلح است
&&
با خود بود، ار بود مرا جنگ
\\
این جمله شکایت از عراقی است
&&
کو بر تن خود نگشت سرهنگ
\\
\end{longtable}
\end{center}
