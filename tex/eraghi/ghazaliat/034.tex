\begin{center}
\section*{غزل شماره ۳۴: یک لحظه دیدن رخ جانانم آرزوست}
\label{sec:034}
\addcontentsline{toc}{section}{\nameref{sec:034}}
\begin{longtable}{l p{0.5cm} r}
یک لحظه دیدن رخ جانانم آرزوست
&&
یکدم وصال آن مه خوبانم آرزوست
\\
در خلوتی چنان، که نگنجد کسی در آن
&&
یکبار خلوت خوش جانانم آرزوست
\\
من رفته از میانه و او در کنار من
&&
با آن نگار عیش بدینسانم آرزوست
\\
جانا، ز آرزوی تو جانم به لب رسید
&&
بنمای رخ، که قوت دل و جانم آرزوست
\\
گر بوسه‌ای از آن لب شیرین طلب کنم
&&
طیره مشو، که چشمهٔ حیوانم آرزوست
\\
یک بار بوسه‌ای ز لب تو ربوده‌ام
&&
یک بار دیگر آن شکرستانم آرزوست
\\
ور لحظه‌ای به کوی تو ناگاه بگذرم
&&
عیبم مکن، که روضهٔ رضوانم آرزوست
\\
وز روی آن که رونق خوبان ز روی توست
&&
دایم نظارهٔ رخ خوبانم آرزوست
\\
بر بوی آن که بوی تو دارد نسیم گل
&&
پیوسته بوی باغ و گلستانم آرزوست
\\
سودای تو خوش است و وصال تو خوشتر است
&&
خوشتر ازین و آن چه بود؟ آنم آرزوست
\\
ایمان و کفر من همه رخسار و زلف توست
&&
در بند کفر مانده و ایمانم آرزوست
\\
درد دل عراقی و درمان من تویی
&&
از درد بس ملولم و درمانم آرزوست
\\
\end{longtable}
\end{center}
