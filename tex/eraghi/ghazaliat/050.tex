\begin{center}
\section*{غزل شماره ۵۰: هر شب دل پر خونم بر خاک درت افتد}
\label{sec:050}
\addcontentsline{toc}{section}{\nameref{sec:050}}
\begin{longtable}{l p{0.5cm} r}
هر شب دل پر خونم بر خاک درت افتد
&&
باشد که چو روز آید بروی گذرت افتد
\\
زیبد که ز درگاهت نومید نگردد باز
&&
آن کس که به امیدی بر خاک درت افتد
\\
آیم به درت افتم، تا جور کنی کمتر
&&
از بخت بدم گویی خود بیشترت افتد
\\
من خاک شوم، جانا، در رهگذرت افتم
&&
آخر به غلط روزی بر من گذرت افتد
\\
گفتم که: بده دادم، بیداد فزون کردی
&&
بد رفت، ندانستم، گفتم: مگرت افتد
\\
در عمر اگر یک دم خواهی که دهی دادم
&&
ناگاه چو وابینی رایی دگرت افتد
\\
کم نال، عراقی، زانک این قصهٔ درد تو
&&
گر شرح دهی عمری، هم مختصرت افتد
\\
\end{longtable}
\end{center}
