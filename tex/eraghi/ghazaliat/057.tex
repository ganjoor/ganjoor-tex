\begin{center}
\section*{غزل شماره ۵۷: امروز مرا در دل جز یار نمی‌گنجد}
\label{sec:057}
\addcontentsline{toc}{section}{\nameref{sec:057}}
\begin{longtable}{l p{0.5cm} r}
امروز مرا در دل جز یار نمی‌گنجد
&&
وز یار چنان پر شد کاغیار نمی‌گنجد
\\
در چشم پر آب من جز دوست نمی‌آید
&&
در جان خراب من جز یار نمی‌گنجد
\\
این لحظه از آن شادم کاندر دل تنگ من
&&
غم جای نمی‌گیرد، تیمار نمی‌گنجد
\\
این قطرهٔ خون تا یافت از لعل لبش رنگی
&&
از شادی آن در پوست چون نار نمی‌گنجد
\\
رو بر در او سرمست، از عشق رخش، زیراک:
&&
در بزم وصال او هشیار نمی‌گنجد
\\
شیدای جمال او در خلد نیرامد
&&
مشتاق لقای او در نار نمی‌گنجد
\\
چون پرده براندازد عالم بسر اندازد
&&
جایی که یقین آید پندار نمی‌گنجد
\\
از گفت بد دشمن آزرده نگردم، زانک:
&&
با دوست مرا در دل آزار نمی‌گنجد
\\
جانم در دل می‌زد، گفتا که: برو این دم
&&
با یار درین جلوه دیار نمی‌گنجد
\\
خواهی که درون آیی بگذار عراقی را
&&
کاندر طبق انوار اطوار نمی‌گنجد
\\
\end{longtable}
\end{center}
