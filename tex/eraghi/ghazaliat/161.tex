\begin{center}
\section*{غزل شماره ۱۶۱: دل گم شد، ازو نشان نیابم}
\label{sec:161}
\addcontentsline{toc}{section}{\nameref{sec:161}}
\begin{longtable}{l p{0.5cm} r}
دل گم شد، ازو نشان نیابم
&&
آن گم شده در جهان نیابم
\\
زان یوسف گم شده به عالم
&&
پیدا و نهان نشان نیابم
\\
تا گوهر شب چراغ گم شد
&&
ره بر در دوستان نیابم
\\
تا بلبل خوشنوای گم شد
&&
بوی گل و بوستان نیابم
\\
تا آب حیات رفت از جوی
&&
عیش خوش جاودان نیابم
\\
سرمایه برفت و سود جویم
&&
زان است که جز زیان نیابم
\\
آن یوسف خویش را چه جویم؟
&&
چون در چه کن فکان نیابم
\\
هم بر در دوست باشد آرام
&&
از خود به جز این گمان نیابم
\\
بر خاک درش چرا ننالم ؟
&&
چاره به جز از فغان نیابم
\\
چون جانش عزیز دارم، آری
&&
دل، کز غم او امان نیابم
\\
تا بر من دلشده بگرید
&&
یک مشفق مهربان نیابم
\\
تا یک نفسی مرا بود یار
&&
یک یار درین زمان نیابم
\\
یاری ده خویشتن درین حال
&&
جز دیدهٔ خون‌فشان نیابم
\\
بر خوان جهان چه می‌نشینم؟
&&
چون لقمه جز استخوان نیابم
\\
بی‌حاصل ازین دکان بخیزم
&&
نقدی چو درین دکان نیابم
\\
خواهم که شوم به بام عالم
&&
چه چاره، چو نردبان نیابم
\\
خواهم که کشم ز چه عراقی
&&
افسوس که ریسمان نیابم!
\\
\end{longtable}
\end{center}
