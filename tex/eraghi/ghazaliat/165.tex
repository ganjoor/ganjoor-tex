\begin{center}
\section*{غزل شماره ۱۶۵: من باز ره خانهٔ خمار گرفتم}
\label{sec:165}
\addcontentsline{toc}{section}{\nameref{sec:165}}
\begin{longtable}{l p{0.5cm} r}
من باز ره خانهٔ خمار گرفتم
&&
ترک ورع و زهد به یک بار گرفتم
\\
سجاده و تسبیح به یک سوی فکندم
&&
بر کف می چون رنگ رخ یار گرفتم
\\
کارم همه با جام می و شاهد و شمع است
&&
ترک دل و دین بهر چنین کار گرفتم
\\
شمعم رخ یار است و شرابم لب دلدار
&&
پیمانه همان لب که به هنجار گرفتم
\\
چشم خوش ساقی دل و دین برد ز دستم
&&
وین فایده زان نرگس بیمار گرفتم
\\
پیوسته چینین می زده و مست و خرابم
&&
تا عادت چشم خوش خونخوار گرفتم
\\
شیرین لب ساقی چو می و نقل فرو ریخت
&&
بس کام کز آن لعل شکربار گرفتم
\\
چون مست شدم خواستم از پای درآمد
&&
حالی سر زلف بت عیار گرفتم
\\
آویختم اندر سر آن زلف پریشان
&&
این شیفتگی بین که دم مار گرفتم
\\
گفتی: کم سودای سر زلف بتان گیر،
&&
چندین چه نصیحت کنی؟ انگار گرفتم
\\
با توبه و تقوی تو ره خلد برین گیر
&&
من با می و معشوقه ره نار گرفتم
\\
در نار چو رنگ رخ دلدار بدیدم
&&
آتش همه باغ و گل و گلزار گرفتم
\\
المنة لله که میان گل و گلزار
&&
دلدار در آغوش دگربار گرفتم
\\
بگرفت به دندان فلک انگشت تعجب
&&
چون من به دو انگشت لب یار گرفتم
\\
دور از لب و دندان عراقی لب دلدار
&&
هم باز به دست خوش دلدار گرفتم
\\
\end{longtable}
\end{center}
