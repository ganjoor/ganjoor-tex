\begin{center}
\section*{غزل شماره ۲۴۵: ای دوست الغیاث! که جانم بسوختی}
\label{sec:245}
\addcontentsline{toc}{section}{\nameref{sec:245}}
\begin{longtable}{l p{0.5cm} r}
ای دوست الغیاث! که جانم بسوختی
&&
فریاد! کز فراق روانم بسوختی
\\
در بوتهٔ بلا تن زارم گداختی
&&
در آتش عنا دل و جانم بسوختی
\\
دانم که: سوختی ز غم عشق خود مرا
&&
لیکن ندانم آنکه چه سانم بسوختی؟
\\
می‌سوزیم درون و تو در وی نشسته‌ای
&&
پیدا نمی‌شود، که نهانم بسوختی
\\
زاتش چگونه سوزد پروانه؟ دیده‌ای؟
&&
ز اندیشهٔ فراق چنانم بسوختی
\\
سود و زیان من، ز جهان، جز دلی نبود
&&
آتش زدی و سود و زیانم بسوختی
\\
تا کی ز حسرت تو برآرم ز سینه آه؟
&&
کز آه سوزناک زیانم بسوختی
\\
بر خاک درگه تو تپیدم بسی ز غم
&&
چو مرغ نیم کشته تپانم بسوختی
\\
تا گفتمت که: کام عراقی ز لب بده
&&
کامم گداختی و زبانم بسوختی
\\
\end{longtable}
\end{center}
