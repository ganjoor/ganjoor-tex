\begin{center}
\section*{غزل شماره ۲۹۲: در کوی تو لولیی، گدایی}
\label{sec:292}
\addcontentsline{toc}{section}{\nameref{sec:292}}
\begin{longtable}{l p{0.5cm} r}
در کوی تو لولیی، گدایی
&&
آمد به امید مرحبایی
\\
بر خاک درت گدای مسکین
&&
با آنکه نرفته بود جایی
\\
از دولت لطف تو، که عام است
&&
محروم چراست بی‌نوایی؟
\\
پیش که رود؟ کجا گریزد؟
&&
از دست غمت شکسته پایی
\\
مگذار که بی نصیب ماند
&&
از درگه پادشه گدایی
\\
چشمم ز رخ تو چشم دارد
&&
هر دم به مبارکی لقایی
\\
جانم ز لب تو می‌کند وام
&&
هر لحظه به تازگی بقایی
\\
جستم همه جای را، ندیدم
&&
جز در دل تنگ جایگایی
\\
بی روی تو هر رخی که دیدم
&&
ننمود مرا جز ابتدایی
\\
دل در سر زلف هر که بستم
&&
دادم دل خود به اژدهایی
\\
در بحر فراق غرق گشتم
&&
دستم نگرفت آشنایی
\\
در بادیهٔ بلا بماندم
&&
راهم ننمود رهنمایی
\\
در آینهٔ جهان ندیدم
&&
جز عکس رخت جهان نمایی
\\
خود هر چه به جز تو در جهان است
&&
هست آن چو سراب یا صدایی
\\
فی‌الجمله ندید دیدهٔ من
&&
از تیرگی جهان صفایی
\\
اکنون به در تو آمدم باز
&&
یابم مگر از درت عطایی؟
\\
در چشم نهاده‌ام که یابم
&&
از خاک در تو توتیایی
\\
در گلشن عشق تو عراقی
&&
مرغی است که نیستش نوایی
\\
\end{longtable}
\end{center}
