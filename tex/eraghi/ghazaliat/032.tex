\begin{center}
\section*{غزل شماره ۳۲: مشو، مشو، ز من خسته‌دل جدا ای دوست}
\label{sec:032}
\addcontentsline{toc}{section}{\nameref{sec:032}}
\begin{longtable}{l p{0.5cm} r}
مشو، مشو، ز من خسته‌دل جدا ای دوست
&&
مکن، مکن، به کف‌اند هم رها ای دوست
\\
برس، که بی‌تو مرا جان به لب رسید، برس
&&
بیا که بر تو فشانم روان، بیا ای دوست
\\
بیا، که بی‌تو مرا برگ زندگانی نیست
&&
بیا، که بی‌تو ندارم سر بقا ای دوست
\\
اگر کسی به جهان در، کسی دگر دارد
&&
من غریب ندارم مگر تو را ای دوست
\\
چه کرده‌ام که مرا مبتلای غم کردی؟
&&
چه اوفتاد که گشتی ز من جدا ای دوست؟
\\
کدام دشمن بدگو میان ما افتاد؟
&&
که اوفتاد جدایی میان ما ای دوست
\\
بگفت دشمن بدگو ز دوستان مگسل
&&
برغم دشمن شاد از درم درآ ای دوست
\\
از آن نفس که جدا گشتی از من بی‌دل
&&
فتاده‌ام به کف محنت و بلا ای دوست
\\
ز دار ضرب توام سکه بر وجود زده
&&
مرا بر آتش محنت میازما ای دوست
\\
چو از زیان منت هیچگونه سودی نیست
&&
مخواه بیش زیان من گدا ای دوست
\\
ز لطف گرد دل بی‌غمان بسی گشتی
&&
دمی به گرد دل پر غمان برآ ای دوست
\\
ز شادی همه عالم شدست بیگانه
&&
دلم که با غم تو گشت آشنا ای دوست
\\
ز روی لطف و کرم شاد کن بروی خودم
&&
که کرد بار غمت پشت من دوتا ای دوست
\\
ز همرهی عراقی ز راه واماندم
&&
ز لطف بر در خویشم رهی‌نما ای دوست
\\
\end{longtable}
\end{center}
