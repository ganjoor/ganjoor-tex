\begin{center}
\section*{غزل شماره ۱۴۱: نرسد به هر زبانی سخن دهان تنگش}
\label{sec:141}
\addcontentsline{toc}{section}{\nameref{sec:141}}
\begin{longtable}{l p{0.5cm} r}
نرسد به هر زبانی سخن دهان تنگش
&&
نه به هر کسی نماید رخ خوب لاله رنگش
\\
لب لعل او نبوسد، به مراد، جز لب او
&&
رخ خوب او نبیند به جز از دو چشم شنگش
\\
لب من رسیدی آخر ز لبش به کام روزی
&&
شدی ار پدید وقتی اثر از دهان تنگش
\\
به من ار خدنگ غمزه فگند چه باک؟ لیکن
&&
سپرش تن است، ترسم که بدور رسد خدنگش
\\
چو مرا نماند رنگی همه رنگ او گرفتم
&&
که جهان مسخرم شد چو برآمدم به رنگش
\\
منم آفتاب از دل، که ز سنگ لعل سازم
&&
منم آبگینه آخر، که کند خراب سنگش
\\
ز میان ما عراقی چو برون فتاد، حالی
&&
پس ازین نمانده ما را سرآشتی و جنگش
\\
\end{longtable}
\end{center}
