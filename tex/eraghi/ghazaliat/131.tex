\begin{center}
\section*{غزل شماره ۱۳۱: بی‌دلی را بی سبب آزرده گیر}
\label{sec:131}
\addcontentsline{toc}{section}{\nameref{sec:131}}
\begin{longtable}{l p{0.5cm} r}
بی‌دلی را بی سبب آزرده گیر
&&
خاکساری را به خاک اسپرده گیر
\\
خسته‌ای از جور عشقت کشته دان
&&
واله‌ای از عشق رویت مرده گیر
\\
گر چنین خواهی کشیدن تیغ غم
&&
جانم اندر تن چون خون افسرده گیر
\\
چند خواهی کرد ازین جور و ستم؟
&&
بی‌دلی از غم به جان آزرده گیر
\\
برده‌ای، هوش دلم، اکنون مرا
&&
نیم جانی مانده وین هم برده گیر
\\
گر بخواهی کرد تیمار دلم
&&
از غم و تیمار جانم خرده گیر
\\
ور عراقی را تو ننوازی کنون
&&
عالمی از بهر او آزرده گیر
\\
\end{longtable}
\end{center}
