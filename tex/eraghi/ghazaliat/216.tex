\begin{center}
\section*{غزل شماره ۲۱۶: سهل گفتی به ترک جان گفتن}
\label{sec:216}
\addcontentsline{toc}{section}{\nameref{sec:216}}
\begin{longtable}{l p{0.5cm} r}
سهل گفتی به ترک جان گفتن
&&
من بدیدم، نمی‌توان گفتن
\\
جان فرهاد خسته شیرین است
&&
کی تواند به ترک جان گفتن؟
\\
دوست می‌دارمت به بانگ بلند
&&
تا کی آهسته و نهان گفتن؟
\\
وصف حسن جمال خود خود گوی
&&
حیف باشد به هر زبان گفتن؟
\\
تا به حدی است تنگی دهنت
&&
که نشاید سخن در آن گفتن؟
\\
گر نبودی کمر، میانت را
&&
کی توانستمی نشان گفتن؟
\\
ز آرزوی لبت عراقی را
&&
شد مسلم حدیث جان گفتن
\\
\end{longtable}
\end{center}
