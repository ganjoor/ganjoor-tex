\begin{center}
\section*{غزل شماره ۱۵۲: ای دیده، بدار ماتم دل}
\label{sec:152}
\addcontentsline{toc}{section}{\nameref{sec:152}}
\begin{longtable}{l p{0.5cm} r}
ای دیده، بدار ماتم دل
&&
کو در خطری فتاد مشکل
\\
خون شد ز فراق یار و از یار
&&
جز خون جگر دگر چه حاصل؟
\\
عمری بتپید بر در یار
&&
آن خسته جگر، چو مرغ بسمل
\\
چون دید به عاقبت که دلدار
&&
در خانهٔ او نکرد منزل
\\
دل در پی وصل یار جان داد
&&
و آن یار نشد، دریغ، حاصل
\\
بر خاک درش فتاد و جان داد
&&
آن قطرهٔ خون، که خوانیش دل
\\
چون یاور نیست بخت با ما
&&
از بهر چه می‌سرشتمان گل؟
\\
ای کاش که بود ما نبودی!
&&
کز بودن ماست کار باطل
\\
ای یار، مبر ز من به یک بار
&&
پیوسته ازین شکسته مگسل
\\
در بحر فراق تو فتادم
&&
دریاب، مگر فتم به ساحل
\\
مگذار که هم چنین بماند
&&
بیچاره عراقی از تو غافل
\\
\end{longtable}
\end{center}
