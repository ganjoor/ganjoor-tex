\begin{center}
\section*{غزل شماره ۲۶۲: چو برقع از رخ زیبای خود براندازی}
\label{sec:262}
\addcontentsline{toc}{section}{\nameref{sec:262}}
\begin{longtable}{l p{0.5cm} r}
چو برقع از رخ زیبای خود براندازی
&&
بگو نظارگیان را صلای جانبازی
\\
ز روی خوب نقاب آنگهی براندازی
&&
که جان جمله جهان ز انتظار بگدازی
\\
نقاب روی تو، جانا، منم که چون گویم:
&&
رخ از نقاب برافگن، مرا براندازی
\\
ز رخ نقاب برانداز، گو: بسوز جهان
&&
که شمع روشنی آنگه دهد که بگدازی
\\
عجب‌تر آنکه جهان را ز تو برون انداخت
&&
به صد زبان و تو با وی هنوز دمسازی
\\
ز نقش روی تو با هیچ کس نشان ندهد
&&
زمان زمان ز رخت نقش دیگری آغاز
\\
رخ تو راز همه عالم آشکارا کرد
&&
بلی، عجب نبود ز آفتاب غمازی
\\
ز رخ نقاب برانداز و پس تماشا کن
&&
که عاشقان تو چون می‌کنند جانبازی؟
\\
به تیر غمزه چرا خسته می‌کنی دل‌ها؟
&&
چو چارهٔ دل بیچارگان نمی‌سازی
\\
دلم، که در سر زلف تو شد، طمع دارد
&&
ز پای بوس تو بر گردنان سرافرازی
\\
اگر تن است و اگر جان، فدای توست همه
&&
به هیچ وجه مرا نیست با تو انبازی
\\
بساز با من مسکین، که ساز بزم توام
&&
ز پرده‌ساز نباشد غریب دمسازی
\\
صدای صوت توام، گرچه زار می‌نالم
&&
بدان خوشم که تو با ناله‌ام هم‌آوازی
\\
از آن خوش است چو نی ناله‌ام به گوش جهان
&&
که هیچ دم نزنم تا توام به ننوازی
\\
بهر چه می‌نگرم چون رخ تو می‌بینم
&&
بگویم: از همه خوبان به حسن ممتازی
\\
کمال حسن تو را چون نهایتی نبود
&&
چگونه بر رخ زیبات برقع اندازی؟
\\
همای عشق عراقی چو بال باز کند
&&
کسی بدو نرسد از بلند پروازی
\\
\end{longtable}
\end{center}
