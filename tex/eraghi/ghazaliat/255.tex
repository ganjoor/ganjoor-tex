\begin{center}
\section*{غزل شماره ۲۵۵: آمد به درت امیدواری}
\label{sec:255}
\addcontentsline{toc}{section}{\nameref{sec:255}}
\begin{longtable}{l p{0.5cm} r}
آمد به درت امیدواری
&&
کو را به جز از تو نیست یاری
\\
محنت‌زده‌ای، نیازمندی
&&
خجلت‌زده‌ای، گناهکاری
\\
از گفتهٔ خود سیاه‌رویی
&&
وز کردهٔ خویش شرمساری
\\
از یار جدا فتاده عمری
&&
وز دوست بمانده روزگاری
\\
بوده به درت چنان عزیزی
&&
دور از تو چنین بمانده خواری
\\
خرسند ز خاک درگه تو
&&
بیچاره به بوی یا غباری
\\
شاید ز در تو باز گردد؟
&&
نومید، چنین امیدواری
\\
زیبد که شود به کام دشمن
&&
از دوستی تو دوستداری؟
\\
بخشای ز لطف بر عراقی
&&
کو ماند کنون و زینهاری
\\
\end{longtable}
\end{center}
