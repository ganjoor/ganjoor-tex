\begin{center}
\section*{غزل شماره ۵۱: بنمای به من رویت، یارات نمی‌افتد}
\label{sec:051}
\addcontentsline{toc}{section}{\nameref{sec:051}}
\begin{longtable}{l p{0.5cm} r}
بنمای به من رویت، یارات نمی‌افتد
&&
آری چه توان کردن؟ با مات نمی‌افتد
\\
گیرم که نمی‌افتد با وصل منت رایی
&&
با جور و جفا، باری، هم‌رات نمی‌افتد؟
\\
می‌افتدت این یک دم کیی براین پر غم
&&
شادم کنی و خرم، هان یات نمی‌افتد؟
\\
هر بیدل و شیدایی افتاده به سودایی
&&
وندر دل من الا سودات نمی‌افتد
\\
با عشق تو می‌بازم شطرنج وفا، لیکن
&&
از بخت بدم، باری، جز مات نمی‌افتد
\\
از غمزهٔ خونریزت هرجای شبیخون است
&&
شب نیست که این بازی صد جات نمی‌افتد
\\
افتاده دو صد شیون از جور تو هرجایی
&&
این جور و جفا با من تنهات نمی‌افتد
\\
بیچاره عراقی، هان! دم درکش و خون می‌خور
&&
چون هیچ دمی با او گیرات نمی‌افتد
\\
\end{longtable}
\end{center}
