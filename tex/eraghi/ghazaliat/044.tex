\begin{center}
\section*{غزل شماره ۴۴: هر که را جام می به دست افتاد}
\label{sec:044}
\addcontentsline{toc}{section}{\nameref{sec:044}}
\begin{longtable}{l p{0.5cm} r}
هر که را جام می به دست افتاد
&&
رند و قلاش و می‌پرست افتاد
\\
دل و دین و خرد زدست بداد
&&
هر که را جرعه‌ای به دست افتاد
\\
چشم میگون یار هر که بدید
&&
ناچشیده شراب، مست افتاد
\\
وانکه دل بست در سر زلفش
&&
ماهی‌آسا، میان شست افتاد
\\
لشکر عشق باز بیرون تاخت
&&
قلب عشاق را شکست افتاد
\\
عاشقی کز سر جهان برخاست
&&
زود با دوستش نشست افتاد
\\
هر که پا بر سر جهان ننهاد
&&
همت او عظیم پست افتاد
\\
سر جان و جهان ندارد آنک:
&&
در سرش بادهٔ الست افتاد
\\
وآنکه از دست خود خلاص نیافت
&&
در ره عشق پای‌بست افتاد
\\
هان، عراقی، ببر ز هستی خویش
&&
نیستی بهره‌ات ز هست افتاد
\\
\end{longtable}
\end{center}
