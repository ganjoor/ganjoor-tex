\begin{center}
\section*{غزل شماره ۷۹: بدین زبان صفت حسن یار نتوان کرد}
\label{sec:079}
\addcontentsline{toc}{section}{\nameref{sec:079}}
\begin{longtable}{l p{0.5cm} r}
بدین زبان صفت حسن یار نتوان کرد
&&
به طعمهٔ پشه عنقا شکار نتوان کرد
\\
به گفتگو سخن عشق دوست نتوان گفت
&&
به جست و جو طلب وصل یار نتوان کرد
\\
بدان مخسب که در خواب روی او بینی
&&
خیال او بود آن، اعتبار نتوان کرد
\\
دو چشم تو، خود اگر عاشقی، پر آب بود
&&
بر آب نقش لطیف نگار نتوان کرد
\\
به چشم او رخ او بین، به دیدهٔ خفاش
&&
به آفتاب نظر آشکار نتوان کرد
\\
به چشم نرگس کوته‌نظر به وقت بهار
&&
نظارهٔ چمن و لاله‌زار نتوان کرد
\\
شدم که بوسه زنم بر درش ادب گفتا
&&
به بوسه خاک در یار خوار نتوان کرد
\\
به نیم جان که تو داری و یک نفس که تو راست
&&
حدیث پیشکشش زینهار نتوان کرد
\\
چه به که پیش سگان درش فشانی جان
&&
که این متاع بر آن رخ نثار نتوان کرد
\\
بلا به پیش خیالش شبی همی گفتم
&&
که : دشمنی همه با دوستدار نتوان کرد
\\
بگوی تا نکند زلف تو پریشانی
&&
که بیش ازین دل ما بی‌قرار نتوان کرد
\\
به تیغ غمزهٔ خون خوار، جان مجروحم
&&
هزار بار، به روزی فگار نتوان کرد
\\
دلی که با غم عشق تو در میان آمد
&&
بهر گنه ز کنارش کنار نتوان کرد
\\
بدان که نام وصال تو می‌برم روزی
&&
به دست هجر مرا جان سپار نتوان کرد
\\
جواب داد خیالش که، با سلیمانی
&&
برای مورچه‌ای کارزار نتوان کرد
\\
میان هجر و وصالش، گر اختیار دهند
&&
ز هر دو هیچ یکی اختیار نتوان کرد
\\
رموز عشق، عراقی، مگو چنین روشن
&&
که راز خویش چنین آشکار نتوان کرد
\\
\end{longtable}
\end{center}
