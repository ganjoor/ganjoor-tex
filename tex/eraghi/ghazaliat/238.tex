\begin{center}
\section*{غزل شماره ۲۳۸: در صومعه نگنجد، رند شرابخانه}
\label{sec:238}
\addcontentsline{toc}{section}{\nameref{sec:238}}
\begin{longtable}{l p{0.5cm} r}
در صومعه نگنجد، رند شرابخانه
&&
عنقا چگونه گنجد در کنج آشیانه؟
\\
ساقی، به یک کرشمه بشکن هزار توبه
&&
بستان مرا ز من باز زان چشم جاودانه
\\
تا وارهم ز هستی وز ننگ خودپرستی
&&
بر هم زنم ز مستی نیک و بد زمانه
\\
زین زهد و پارسایی چون نیست جز ریایی
&&
ما و شراب و شاهد، کنج شرابخانه
\\
چه خوش بود خرابی! افتاده در خرابات
&&
چون چشم یار مخمور از مستی شبانه
\\
آیا بود که بختم بیند به خواب مستی
&&
او در کناره، آنگه من رفته از میانه؟
\\
ساقی شراب داده هر لحظه جام دیگر
&&
مطرب سرود گفته هر دم دگر ترانه
\\
در جام باده دیده عکس جمال ساقی
&&
و آواز او شنوده از زخمهٔ چغانه
\\
این است زندگانی، باقی همه حکایت
&&
این است کامرانی، باقی همه فسانه
\\
میخانه حسن ساقی، میخواره چشم مستش
&&
پیمانه هم لب او، باقی همه بهانه
\\
در دیدهٔ عراقی جام شراب و ساقی
&&
هر سه یکی است و احول بیند یکی دوگانه
\\
\end{longtable}
\end{center}
