\begin{center}
\section*{غزل شماره ۱۴۷: بیا، که خانهٔ دل پاک کردم از خاشاک}
\label{sec:147}
\addcontentsline{toc}{section}{\nameref{sec:147}}
\begin{longtable}{l p{0.5cm} r}
بیا، که خانهٔ دل پاک کردم از خاشاک
&&
درین خرابه تو خود کی قدم نهی؟ حاشاک
\\
هزار دل کنی از غم خراب و نندیشی
&&
هزار جان به لب آری، ز کس نداری باک
\\
کدام دل که ز جور تو دست بر سر نیست؟
&&
کدام جان که نکرد از جفات بر سر خاک؟
\\
دلم، که خون جگر می‌خورد ز دست غمت،
&&
در انتظار تو صد زهر خورده بی تریاک
\\
کنون که جان به لب آمد مپیچ در کارم
&&
مکن، که کار من از تو بماند در پیچاک
\\
نه هیچ کیسه‌بری همچو طره‌ات طرار
&&
نه هیچ راهزنی همچو غمزه‌ات چالاک
\\
به طره صید کنی صدهزار دل هر دم
&&
به غمزه بیش کشی هر نفس دو صد غمناک
\\
دل عراقی مسکین، که صید لاغر توست
&&
چو می کشیش میفگن، ببند بر فتراک
\\
\end{longtable}
\end{center}
