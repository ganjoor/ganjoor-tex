\begin{center}
\section*{غزل شماره ۱۸۷: من آن قلاش و رند بی‌نوایم}
\label{sec:187}
\addcontentsline{toc}{section}{\nameref{sec:187}}
\begin{longtable}{l p{0.5cm} r}
من آن قلاش و رند بی‌نوایم
&&
که در رندی مغان را پیشوایم
\\
گدای درد نوش می پرستم
&&
حریف پاکباز کم دغایم
\\
ز بند زهد و قرابی برستم
&&
نه مرد زرق و سالوس و ریایم
\\
ردا و طیلسان یکسو نهادم
&&
همه زنار شد بند قبایم
\\
مگر خاکم ز میخانه سرشتند
&&
که هر دم سوی میخانه گرایم؟
\\
کجایی، ساقیا، جامی به من ده
&&
که یک دم با حریفان خوش برآیم
\\
مرا برهان زخود، کز جان به جانم
&&
درین وحشت سرا تا چند پایم؟
\\
زمانی شادمان و خوش نبودم
&&
از آنم کاندرین وحشت سرایم
\\
مرا از درگه پاکان براندند
&&
به صد خواری، که رند ناسزایم
\\
برون کردندم از کعبه به خواری
&&
درون بتکده کردند جایم
\\
درین ره خواستم زد دست و پایی
&&
بریدند، ای دریغا، دست و پایم
\\
بماندم در بیابان تحیر
&&
نه ره پیدا کنون، نه رهنمایم
\\
امید از هر که هست اکنون بریدم
&&
فتاده بر در لطف خدایم
\\
از آن است این همه بیداد بر من
&&
که پیوسته ز یار خود جدایم
\\
ز بیداد زمانه وارهم من
&&
عراقی گر کند از کف رهایم
\\
\end{longtable}
\end{center}
