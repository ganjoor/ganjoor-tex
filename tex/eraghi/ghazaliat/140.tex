\begin{center}
\section*{غزل شماره ۱۴۰: بکشم به ناز روزی سر زلف مشک رنگش}
\label{sec:140}
\addcontentsline{toc}{section}{\nameref{sec:140}}
\begin{longtable}{l p{0.5cm} r}
بکشم به ناز روزی سر زلف مشک رنگش
&&
ندهم ز دست این بار، اگر آورم به چنگش
\\
سر زلف او بگیرم، لب لعل او ببوسم
&&
به مراد، اگر نترسم ز دو چشم شوخ شنگش
\\
سخن دهان تنگش بود ار چه خوش، ولیکن
&&
نرسد به هر زبانی سخن دهان تنگش
\\
چون نبات می‌گدازم، همه شب، در آب دیده
&&
به امید آنکه یابم شکر از دهان تنگش
\\
بروم، ز چشم مستش نظری تمام گیرم
&&
که بدان نظر ببینم رخ خوب لاله رنگش
\\
چو کمان ابروانش فکند خدنگ غمزه
&&
چه کنم که جان نسازم سپر از پی خدنگش؟
\\
زلبش عناب، یارب، چه خوش است!صلح اوخود
&&
بنگر چگونه باشد؟ چو چنین خوش است جنگش
\\
دلم آینه است و در وی رخ او نمی‌نماید
&&
نفسی بزن، عراقی، بزدا به ناله زنگش
\\
\end{longtable}
\end{center}
