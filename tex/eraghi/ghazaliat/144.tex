\begin{center}
\section*{غزل شماره ۱۴۴: باز غم بگرفت دامانم، دریغ}
\label{sec:144}
\addcontentsline{toc}{section}{\nameref{sec:144}}
\begin{longtable}{l p{0.5cm} r}
باز غم بگرفت دامانم، دریغ
&&
سر برآورد از گریبانم دریغ
\\
غصه دم‌دم می‌کشم از جام غم
&&
نیست جز غصه گوارانم، دریغ
\\
ابر محنت خیمه زد بر بام دل
&&
صاعقه افتاد در جانم، دریغ
\\
مبتلا گشتم به درد یار خود
&&
کس نداند کرد درمانم، دریغ
\\
در چنین جان کندنی کافتاده‌ام
&&
چاره جز مردن نمی‌دانم، دریغ
\\
الغیاث! ای دوستان، رحمی کنید
&&
کز فراق یار قربانم، دریغ
\\
جور دلدار و جفای روزگار
&&
می‌کشد هر یک دگرسانم، دریغ
\\
گر چه خندم گاه گاهی همچو شمع
&&
در میان خنده گریانم، دریغ
\\
صبح وصل او نشد روشن هنوز
&&
در شب تاریک هجرانم، دریغ
\\
کار من ناید فراهم، تا بود
&&
در هم این حال پریشانم، دریغ
\\
نیست امید بهی از بخت من
&&
تا کی از دست تو درمانم؟ دریغ
\\
لاجرم خون خور، عراقی، دم به دم
&&
چون نکردی هیچ فرمانم، دریغ
\\
\end{longtable}
\end{center}
