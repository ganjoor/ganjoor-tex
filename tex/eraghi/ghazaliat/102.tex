\begin{center}
\section*{غزل شماره ۱۰۲: باز دلم عیش و طرب می‌کند}
\label{sec:102}
\addcontentsline{toc}{section}{\nameref{sec:102}}
\begin{longtable}{l p{0.5cm} r}
باز دلم عیش و طرب می‌کند
&&
هیچ ندانم چه سبب می‌کند؟
\\
از می عشق تو مگر مست شد
&&
کین همه شادی و طرب می‌کند؟
\\
تا سر زلف تو پریشان بدید
&&
شیفته شد ، شور و شغب می کند
\\
تا دل من در سر زلف تو شد
&&
عیش همه در دل شب می‌کند
\\
برد به بازی دل جمله جهان
&&
زلف تو بازی چه عجب می‌کند؟
\\
طرهٔ طرار تو کرد آن چه کرد
&&
فتنه نگر باز که لب می‌کند
\\
می‌برد از من دل و گوید به طنز:
&&
باز فلانی چه طلب می‌کند؟
\\
از لب لعلش چه عجب گر مرا
&&
آرزوی قند و طرب می‌کند
\\
گر طلبد بوسه، عراقی مرنج،
&&
گرچه همه ترک ادب می‌کند
\\
\end{longtable}
\end{center}
