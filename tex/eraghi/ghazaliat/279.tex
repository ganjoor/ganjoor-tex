\begin{center}
\section*{غزل شماره ۲۷۹: ای که از لطف سراسر جانی}
\label{sec:279}
\addcontentsline{toc}{section}{\nameref{sec:279}}
\begin{longtable}{l p{0.5cm} r}
ای که از لطف سراسر جانی
&&
جان چه باشد؟ که تو صد چندانی
\\
تو چه چیزی؟ چه بلایی؟ چه کسی؟
&&
فتنه‌ای؟ شنقصه‌ای؟ فتانی؟
\\
حکمت از چیست روان بر همه کس؟
&&
کیقبادی؟ ملکی؟ خاقانی؟
\\
به دمی زنده کنی صد مرده
&&
عیسیی؟ آب حیاتی؟ جانی؟
\\
به تماشای تو آید همه کس
&&
لاله‌زاری؟ چمنی؟ بستانی؟
\\
روی در روی تو آرند همه
&&
قبله‌ای؟ آینه‌ای؟ جانانی؟
\\
در مذاق همه کس شیرینی
&&
انگبینی؟ شکری؟ سیلانی؟
\\
گر چه خردی، همه را در خوردی
&&
نمکی؟ آب روانی؟ نانی؟
\\
آرزوی دل بیمار منی
&&
صحتی؟ عافیتی؟ درمانی؟
\\
گه خمارم شکنی، گه توبه
&&
می نابی؟ فقعی؟ رمانی؟
\\
دیدهٔ من به تو بیند عالم
&&
آفتابی؟ قمری؟ اجفانی؟
\\
همه خوبان به تو آراسته‌اند
&&
کهربایی؟ گهری؟ مرجانی؟
\\
مهر هر روز دمی در بنده‌ات
&&
سحری؟ صبح‌دمی؟ خندانی؟
\\
همه در بزم ملوکت خوانند
&&
قصه‌ای؟ مثنویی؟ دیوانی؟
\\
\end{longtable}
\end{center}
