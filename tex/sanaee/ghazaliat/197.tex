\begin{center}
\section*{غزل شماره ۱۹۷: برخیز و برو باده بیار ای پسر خوش}
\label{sec:197}
\addcontentsline{toc}{section}{\nameref{sec:197}}
\begin{longtable}{l p{0.5cm} r}
برخیز و برو باده بیار ای پسر خوش
&&
وین گفت مرا خوار مدار ای پسر خوش
\\
باده خور و مستی کن و دلداری و عشرت
&&
و اندوه جهان باد شمار ای پسر خوش
\\
رنج و غم بیهوده منه بر دل و بر جان
&&
و آن چت بنخارد بمخار ای پسر خوش
\\
خواهی که بود خاک درت افسر عشاق
&&
در باده فزون کن تو خمار ای پسر خوش
\\
ناموس خرد بشکن و سالوس طریقت
&&
وز هر دو برآور تو دمار ای پسر خوش
\\
زهد و گنه و کفر و هدی را همه در هم
&&
در باز به یک داو قمار ای پسر خوش
\\
تا زنده شود مجلس ما از رخ و زلفت
&&
در مجلس ما مشک و گل آر ای پسر خوش
\\
از جان و جوانی نبود شاد سنایی
&&
تا دل نکند بر تو نثار ای پسر خوش
\\
صد سجدهٔ شکر از دل و از جان به تو آرد
&&
او را ز چه داری تو فگار ای پسر خوش
\\
\end{longtable}
\end{center}
