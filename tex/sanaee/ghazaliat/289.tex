\begin{center}
\section*{غزل شماره ۲۸۹: او چنان داند که ما در عشق او کمتر زنیم}
\label{sec:289}
\addcontentsline{toc}{section}{\nameref{sec:289}}
\begin{longtable}{l p{0.5cm} r}
او چنان داند که ما در عشق او کمتر زنیم
&&
یا دو چنگ از جور او در دامن دیگر زنیم
\\
هر زمان ما را دلی کی باشد و جانی دگر
&&
تا به عشق بی‌وفایی دیگر آتش در زنیم
\\
تا کی از نادیدنش ما دیده‌ها پر خون کنیم
&&
تا کی از هجران او ما دستها بر سر زنیم
\\
گاه آن آمد که بر ما باد سلوت برجهد
&&
گاه آن آمد که ما با رود و رامشگر زنیم
\\
گر فلک در عهد او با ما نسازد گو مساز
&&
ما به یک دم آتش اندر چرخ و بر چنبر زنیم
\\
گه ز رخسار بتان بر لاله و گل می‌خوریم
&&
گه ز زلف دلبران با مشک و با عنبر زنیم
\\
پشتمان از غم کمان شد از قدش تیری کنیم
&&
باده پیماییم از خم بر خم دیگر زنیم
\\
\end{longtable}
\end{center}
