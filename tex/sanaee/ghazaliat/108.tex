\begin{center}
\section*{غزل شماره ۱۰۸: ناز را رویی بباید همچو ورد}
\label{sec:108}
\addcontentsline{toc}{section}{\nameref{sec:108}}
\begin{longtable}{l p{0.5cm} r}
ناز را رویی بباید همچو ورد
&&
چون نداری گرد بدخویی مگرد
\\
یا بگستر فرش زیبایی و حسن
&&
یا بساط کبر و ناز اندر نورد
\\
نیکویی و لطف گو با تاج و کبر
&&
کعبتین و مهره گو با تخته نرد
\\
در سرت بادست و بر رو آب نیست
&&
پس میان ما دو تن زین‌ست گرد
\\
زشت باشد روی نازیبا و ناز
&&
صعب باشد چشم نا بینا و درد
\\
جوهرت ز اول نبودست این چنین
&&
با تو ناز و کبر کرد این کار کرد
\\
زر ز معدن سرخ روی آید برون
&&
صحبت ناجنس کردش روی زرد
\\
کی کند ناخوب را بیداد خوب
&&
چون کند نامرد را کافور مرد
\\
تو همه بادی و ما را با تو صلح
&&
ما ترا خاک و ترا با ما نبرد
\\
لیکن از یاد تو ما را چاره نیست
&&
تا دین خاکست ما را آب خورد
\\
ناز با ما کن که درباید همی
&&
این نیاز گرم را آن ناز سرد
\\
ور ثنا خواهی که باشد جفت تو
&&
با سنایی چون سنایی باش فرد
\\
در جهان امروز بردار برد اوست
&&
باردی باشد بدو گفتن که برد
\\
\end{longtable}
\end{center}
