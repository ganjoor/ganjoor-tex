\begin{center}
\section*{غزل شماره ۱۴۴: روزی بت من مست به بازار برآمد}
\label{sec:144}
\addcontentsline{toc}{section}{\nameref{sec:144}}
\begin{longtable}{l p{0.5cm} r}
روزی بت من مست به بازار برآمد
&&
گرد از دل عشاق به یک بار بر آمد
\\
صد دلشده را از غم او روز فرو شد
&&
صد شیفته را از غم او کار برآمد
\\
رخسار و خطش بود چو دیبا و چو عنبر
&&
باز آن دو بهم کرد و خریدار برآمد
\\
در حسرت آن عنبر و دیبای نو آیین
&&
فریاد ز بزاز و ز عطار برآمد
\\
رشک ست بتان را ز بناگوش و خط او
&&
گویند که بر برگ گلش خار برآمد
\\
آن مایه بدانید که ایزد نظری کرد
&&
تا سوسن و شمشاد ز گلزار برآمد
\\
و آن شب که مرا بود به خلوت بر او بار
&&
پیش از شب من صبح ز کهسار برآمد
\\
\end{longtable}
\end{center}
