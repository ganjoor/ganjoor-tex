\begin{center}
\section*{غزل شماره ۱۴۸: هر که در بند خویشتن نبود}
\label{sec:148}
\addcontentsline{toc}{section}{\nameref{sec:148}}
\begin{longtable}{l p{0.5cm} r}
هر که در بند خویشتن نبود
&&
وثن خویش را شمن نبود
\\
آنکه خالی شود ز خویشی خویش
&&
خویشی خویش را وطن نبود
\\
من مگوی ار ز خویش بی خبری
&&
زان که از خویش مرده من نبود
\\
در خرابات هر که مرد از خویش
&&
تن او را ز من کفن نبود
\\
ارنه‌ای مرده هر چه خواهی گوی
&&
از همه جز منت سخن نبود
\\
با سنایی ازین خصومت نیست
&&
زین خصومت ورا حزن نبود
\\
مست باش ای پسر که مستان را
&&
دل به تیمار ممتحن نبود
\\
راستی را همی چو خواهی کرد
&&
نیستی جز هلاک تن نبود
\\
\end{longtable}
\end{center}
