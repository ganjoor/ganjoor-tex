\begin{center}
\section*{غزل شماره ۱۶: نیست بی دیدار تو در دل شکیبایی مرا}
\label{sec:016}
\addcontentsline{toc}{section}{\nameref{sec:016}}
\begin{longtable}{l p{0.5cm} r}
نیست بی دیدار تو در دل شکیبایی مرا
&&
نیست بی‌گفتار تو در دل توانایی مرا
\\
در وصالت بودم از صفرا و از سودا تهی
&&
کرد هجران تو صفرایی و سودایی مرا
\\
عشق تو هر شب برانگیزد ز جانم رستخیز
&&
چون تو بگریزی و بگذاری به تنهایی مرا
\\
چشمهٔ خورشید را از ذره نشناسم همی
&&
نیست گویی ذره‌ای دردیده بینایی مرا
\\
از تو هر جایی ننالم تو هر جایی شدی
&&
نیست جای ناله از معشوق هر جایی مرا
\\
گاه پیری آمد از عشق تو بر رویم پدید
&&
آنچه پنهان بود در دل گاه برنایی مرا
\\
کرد معزولم زمانه گاه دانایی و عقل
&&
با بلای تو چه سود از عقل و دانایی مرا
\\
\end{longtable}
\end{center}
