\begin{center}
\section*{غزل شماره ۴۲: توبهٔ من جزع و لعل و زلف و رخسارت شکست}
\label{sec:042}
\addcontentsline{toc}{section}{\nameref{sec:042}}
\begin{longtable}{l p{0.5cm} r}
توبهٔ من جزع و لعل و زلف و رخسارت شکست
&&
دی که بودم روزه‌دار امروز هستم بت‌پرست
\\
از ترانهٔ عشق تو نور نبی موقوف گشت
&&
وز مغابهٔ جام تو قندیلها بر هم شکست
\\
رمزهای لعل تو دست جوانمردان گشاد
&&
حلقه‌های زلف تو پای خردمندان ببست
\\
ابروی مقرونت ای دلبر کمان اندر کشید
&&
ناوک مژگانت ای جانان دل و جانم بخست
\\
با چنان مژگان و ابرو با چنان رخسار و لب
&&
بود نتوان جز صبور و عاشق و مخمور و مست
\\
پارسایی را بود در عشق تو بازار سست
&&
پادشاهی را بود در وصل تو مقدار پست
\\
جز برای تو نسازم من ز فرق خویش پای
&&
جز به یاد تو نیارم سوی رطل و جام دست
\\
شادی و آرام نبود هر کرا وصل تو نیست
&&
هر کرا وصل تو باشد هر چه باید جمله هست
\\
\end{longtable}
\end{center}
