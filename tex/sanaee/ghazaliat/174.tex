\begin{center}
\section*{غزل شماره ۱۷۴: چون سخنگویی از آن لب لطف باری ای پسر}
\label{sec:174}
\addcontentsline{toc}{section}{\nameref{sec:174}}
\begin{longtable}{l p{0.5cm} r}
چون سخنگویی از آن لب لطف باری ای پسر
&&
پس به شوخی لب چرا خاموش داری ای پسر
\\
در ره عشق تو ما را یار و مونس گفت تست
&&
زان بگفتی از تو می‌خواهم یاری ای پسر
\\
دیر زی در شادکامی کز اثرهای لطیف
&&
مونس عقلی و جان را غمگساری ای پسر
\\
تلخ گردد عیش شیرین بر بتان قندهار
&&
چون به گاه بذله زان لب لطف باری ای پسر
\\
بامداد از رشک دامن را کند خورشید چاک
&&
روی چون ماه از گریبان چون برآری ای پسر
\\
سر بسان سایه زان بر خاک دارم پیش تو
&&
کز رخ و زلف آفتاب و سایه داری ای پسر
\\
سرکشان سر بر خط فرمان من بنهند باش
&&
تا به گرد مه خط مشکین برآری ای پسر
\\
ار نبودی ماه رخسار تو تابان زیر زلف
&&
با سر زلف تو بودی دهر تاری ای پسر
\\
کودکی کان را به معنی در خم چوگان زلف
&&
همچو گویی روز و شب گردان نداری ای پسر
\\
شد گرفتار سر زلف کمند آسای تو
&&
روز دعوی کردن مردان کاری ای پسر
\\
شد شکار چشم روبه باز پر دستان تو
&&
صدهزاران جان شیران شکاری ای پسر
\\
ماه روی تو چو برگ گل به باغ دلبری
&&
شد شکفته بر نهال کامگاری ای پسر
\\
بس دلا کز خرمی بی برگ شد زان برگ گل
&&
آه اگر بر برگ گل شمشاد کاری ای پسر
\\
کی شدندی عالمی در عشق تو یعقوب‌وار
&&
گر نه از یوسف جهان را یادگاری ای پسر
\\
چون سنایی را به عالم نام فخر از عشق تست
&&
ننگ و عار از وصلت او می چه داری ای پسر
\\
\end{longtable}
\end{center}
