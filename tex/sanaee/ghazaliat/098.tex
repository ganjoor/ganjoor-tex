\begin{center}
\section*{غزل شماره ۹۸: آنرا که خدا از قلم لطف نگارد}
\label{sec:098}
\addcontentsline{toc}{section}{\nameref{sec:098}}
\begin{longtable}{l p{0.5cm} r}
آنرا که خدا از قلم لطف نگارد
&&
شاید که به خود زحمت مشاطه نیارد
\\
مشاطه چه حاجت بود آن را که همی حسن
&&
هر ساعت ماهی ز گریبانش برآرد
\\
انگشت نمای همه دلها شود ار چه
&&
ناخنش نباشد که سر خویش بخارد
\\
با زحمت شانه چکند چنبر زلفی
&&
کاندر شب او عقل همی روز گذارد
\\
مشاطه نه خام آید جایی که بدانجای
&&
نقاش ازل بر صفتش خامه گذارد
\\
کی زشت شود روی نکو ار بنشویند
&&
کی خشک شود طوبی اگر ابر نبارد
\\
ای آنکه همه برزگر دیو در اسلام
&&
در مزرعهٔ جان تو جز لاف نکارد
\\
مشاطهٔ تو چون تو بوی دیو تو لابد
&&
هم نقش ترا بر دل و جان تو نگارد
\\
کانکس که مر او را نبود جلوه‌گر از عشق
&&
شهد از لب او جان و خرد زهر شمارد
\\
وانرا که قبولش نکند عالم اقبال
&&
گر گلشکری گردد کس را نگوارد
\\
حقا که به مردم سقر نقد ببینی
&&
گر هیچ ترا حسن به خوی تو سپارد
\\
هر روز دگر لام کشی از پی خوبی
&&
زین لام چه فایده کالف هیچ ندارد
\\
آنجا که چنو جان طلبی یافت سنایی
&&
جان را بگذارد چو تویی را نگذارد
\\
\end{longtable}
\end{center}
