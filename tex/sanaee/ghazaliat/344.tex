\begin{center}
\section*{غزل شماره ۳۴۴: چون سخن زان زلف و رخ گویی مگو از کفر و دین}
\label{sec:344}
\addcontentsline{toc}{section}{\nameref{sec:344}}
\begin{longtable}{l p{0.5cm} r}
چون سخن زان زلف و رخ گویی مگو از کفر و دین
&&
زان که هر جای این دو رنگ آمد نه آن ماند نه این
\\
نیست با زلفین او پیکار دارالضرب کفر
&&
نیست با رخسان او بی‌شاه دارالملک دین
\\
خود ز رنگ زلف و نور روی او برساختند
&&
کفر خالی از گمان و دین جمالی از یقین
\\
خاکپای و خار راهش دیده را و دست را
&&
توده توده سنبلست و دسته دسته یاسمین
\\
چون به کوی اندر خرامد آن چنان باشد ز لطف
&&
پای آن بت ز آستان چون دست موسی ز آستین
\\
چون نقاب از رخ براندازد ز خاتونان خلد
&&
بانگ برخیزد که: هین ای آفرینش آفرین
\\
لعبت چین خواندم او را و بد خواندم نه نیک
&&
لاجرم زین شرم شد رویم چو زلفش پر ز چین
\\
لعبت چین چون توان خواند آن نگاری را که هست
&&
زیر یک چین از دو زلفش صدهزار ار تنگ چین
\\
خود حدیث عاشقی بگذار و انصافم بده
&&
کافری نبود چنانی را صفت کردن چنین
\\
خط او را گر تو خط خوانی خطا باشد که نیست
&&
آن مگر دولت گیای خطهٔ روح‌الامین
\\
آسمان آن خط بر آن عارض نه بهر آن نوشت
&&
تا من و تو رنجه دل گردیم و آن بت شرمگین
\\
لیک چون دید آسمان کز حسن او چون آفتاب
&&
رامش و آرامش و آرایشست اندر زمین
\\
حسن را بر چهرهٔ او بنده کرد و بر نوشت
&&
آسمان از مشک بر گردش صلاح‌المسلمین
\\
از دو یاقوتش دو چیز طرفه یابم در دو حال
&&
چون بگوید حلقه باشد چون خمش گردد نگین
\\
دل چو ز آن لب دور ماند گر بسوزد گو بسوز
&&
موم را ز آتش چه چاره چون جدا شد ز انگبین
\\
هر زمان گویی سنایی کیست خیز اندر نگر
&&
هم سنا و هم سنایی را در آن صورت ببین
\\
خود سنایی او بود چون بنگری زیرا بر اوست
&&
لب چو باقامت الف ابرو چو نون دندان چو سین
\\
\end{longtable}
\end{center}
