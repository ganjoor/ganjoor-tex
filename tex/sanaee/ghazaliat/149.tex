\begin{center}
\section*{غزل شماره ۱۴۹: هر کو به راه عاشقی اندر فنا شود}
\label{sec:149}
\addcontentsline{toc}{section}{\nameref{sec:149}}
\begin{longtable}{l p{0.5cm} r}
هر کو به راه عاشقی اندر فنا شود
&&
تا رنج وقت او همه اندر بلا شود
\\
آری بدین مقام نیارد کسی رسید
&&
تا همتش بریده ز هر دو سرا شود
\\
راهیست بلعجب که درو چون قدم زنی
&&
کمتر منازلش دهن اژدها شود
\\
بی چون و بی چگونه رهی کاندر و قدم
&&
گاهی زمین تیره و گاهی سما شود
\\
در منزل نخستین مردم ز نام و ننگ
&&
از روزگار مذهب و آیین جدا شود
\\
هر کس نشان نیافت از این راه بر کران
&&
آن مرد غرقه گشته به دریا کجا شود
\\
در کوی آدمی نتوان جست راه دین
&&
کاندر نسب عقیدهٔ مردم دو تا شود
\\
زاندر که آمدی به همان بایدت شدن
&&
پس جز به نیستی نسب تو خطا شود
\\
صحرا مشو که عیب نهانست در جهان
&&
ور عیب غیب گردد عاشق فنا شود
\\
\end{longtable}
\end{center}
