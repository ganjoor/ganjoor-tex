\begin{center}
\section*{غزل شماره ۳۷۳: ساقیا مستان خواب‌آلوده را آواز ده}
\label{sec:373}
\addcontentsline{toc}{section}{\nameref{sec:373}}
\begin{longtable}{l p{0.5cm} r}
ساقیا مستان خواب‌آلوده را آواز ده
&&
روز را از روی خویش و سوز ایشان ساز ده
\\
غمزه‌ها سر تیز دار و طره‌ها سر پست کن
&&
رمزها سرگرم گوی و بوسها سرباز ده
\\
سرخ روی ناز را چون گل اسیر خار کن
&&
زرد روی آز را چون زر به دست گاز ده
\\
حربه و شل در بر بهرام خربط سوز نه
&&
زخمه و مل در کف ناهید بر بط ساز ده
\\
هم بخور هم صوفیان عقل را سرمست کن
&&
هم برو هم صافیان روح را ره باز ده
\\
در هوای شمع عشق و شمع می پروانه‌وار
&&
پیشوای خلد و صدر سدره را پرواز ده
\\
چنگل گیراست اینک باز و باشهٔ عشق را
&&
صعوه پیش باشه و آن کبک رازی باز ده
\\
پیش کان پیر منافق بانگ قامت در دهد
&&
غارت عقل و دل و جان را هلا آواز ده
\\
پیش کز بالا درآید ارسلان سلطان روز
&&
پیش من بکتاش سرمست مرابه گماز ده
\\
ور همی چون عشق خواهی عقل خود را پاکباز
&&
نصفیی پر کن بدان پیر دوالک باز ده
\\
گر همی سرمست خواهی صبح را چون چشم خود
&&
جرعه‌ای زان می به صبح منهی غماز ده
\\
روزه چون پیوسته خواهد بود ما را زیر خاک
&&
باده ما را زین سپس بر رسم سنگ‌انداز ده
\\
جبرییل اینجا اگر زحمت کند خونش بریز
&&
خونبهای جبرییل از گنج رحمت باز ده
\\
بادبان راز اگر مجروح گردد ز آه ما
&&
درپه‌ای از خامشی در بادبان راز ده
\\
وارهان یک دم سنایی را ز بند عافیت
&&
تا دهی او را شراب عافیت پرداز ده
\\
\end{longtable}
\end{center}
