\begin{center}
\section*{غزل شماره ۸۰: در کوی ما که مسکن خوبان سعتریست}
\label{sec:080}
\addcontentsline{toc}{section}{\nameref{sec:080}}
\begin{longtable}{l p{0.5cm} r}
در کوی ما که مسکن خوبان سعتریست
&&
از باقیات مردان پیری قنلدریست
\\
پیری که از مقام منیت تنش جداست
&&
پیری که از بقای بقیت دلش بریست
\\
تا روز دوش مست و خرابات اوفتاده بود
&&
بر صورتی که خلق برو بر همی گریست
\\
گفتم و را بمیر که این سخت منکرست
&&
گفتا که حال منکری از شرط منکریست
\\
گفتم گر این حدیث درست ست پس چراست
&&
کاندر وجود معنی و با خلق داوریست
\\
گفت آن وجود فعل بود کاندرو ترا
&&
با غیر داوری ز پی فضل و برتریست
\\
آن کس که دیو بود چو آمد درین طریق
&&
بنگر به راستی که کنون خاصه چون پریست
\\
از دست خود نهاد کله بر سر خرد
&&
هر نکته از کلامش دینار جعفریست
\\
گفتم دل سنایی از کفر آگهست
&&
گفت این نه از شما ز سخنهای سر سریست
\\
در حق اتحاد حقیقت به حق حق
&&
چون تو نه‌ای حقیقت اسلام کافریست
\\
\end{longtable}
\end{center}
