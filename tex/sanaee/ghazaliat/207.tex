\begin{center}
\section*{غزل شماره ۲۰۷: آن کژدم زلف تو که زد بر دل من نیش}
\label{sec:207}
\addcontentsline{toc}{section}{\nameref{sec:207}}
\begin{longtable}{l p{0.5cm} r}
آن کژدم زلف تو که زد بر دل من نیش
&&
از ضربت آن زخم دل نازک من ریش
\\
آنجا که بود انجمن لشگر خوبان
&&
نام تو بود اول و پای تو بود پیش
\\
بنگر که همی با من و با تو چکند چرخ
&&
بر هر دو همی چون شمرد مکر و فن خویش
\\
هر شب که کند عشق شکیبایی من کم
&&
هم در گذرد خوبی و زیبایی تو بیش
\\
ای روی تو قارون شده از حسن و ملاحت
&&
از هجر تو قارونم و از وصل تو درویش
\\
خود چون بود آخر به غم هجر گرفتار
&&
آن کس که به اول نبود عافیت اندیش
\\
\end{longtable}
\end{center}
