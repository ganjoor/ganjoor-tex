\begin{center}
\section*{غزل شماره ۱۵۵: هر که او معشوق دارد گو چو من عیار دار}
\label{sec:155}
\addcontentsline{toc}{section}{\nameref{sec:155}}
\begin{longtable}{l p{0.5cm} r}
هر که او معشوق دارد گو چو من عیار دار
&&
خوش لب و شیرین زبان خوش عیش و خوش گفتار دار
\\
یار معنی دار باید خاصه اندر دوستی
&&
تا توانی دوستی با یار معنی دار دار
\\
از عزیزی گر نخواهی تا به خواری اوفتی
&&
روی نیکو را عزیز و مال و نعمت خوار دار
\\
ماه ترکستان بسی از ماه گردون خوبتر
&&
مه ز ترکستان گزین و ز ماه گردون دون عار دار
\\
زلف عنبر بار گیر و جام مالامال کش
&&
دوستی با جام و با زلفین عنبر بار دار
\\
ور همی خواهی که گردد کار تو همچون نگار
&&
چون سنایی خویشتن در عشق او بر کار دار
\\
\end{longtable}
\end{center}
