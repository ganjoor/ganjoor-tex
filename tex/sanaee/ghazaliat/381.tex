\begin{center}
\section*{غزل شماره ۳۸۱: جان جز پیش خود چمانه منه}
\label{sec:381}
\addcontentsline{toc}{section}{\nameref{sec:381}}
\begin{longtable}{l p{0.5cm} r}
جان جز پیش خود چمانه منه
&&
طبع جز بر می مغانه منه
\\
باده را تا به باغ شاید برد
&&
آنچنان در شرابخانه منه
\\
گر چه همرنگ نار دانه بود
&&
نام او آب نار دانه منه
\\
در هر آن خانه‌ای که می نبود
&&
پای اندر چنان ستانه منه
\\
تا بود باغ آسمان گردان
&&
چشم بر روی آسمانه منه
\\
روی جز بر جناح چنگ ممال
&&
دست جو بر بر چغانه منه
\\
گر نخواهی که در تو پیچد غم
&&
رنج بر طبع شادمانه منه
\\
بد و نیک زمانه گردانست
&&
بر بد و نیک او بهانه منه
\\
بخردان بر زمانه دل ننهند
&&
پس تو دل نیز بر زمانه منه
\\
\end{longtable}
\end{center}
