\begin{center}
\section*{غزل شماره ۳۰۰: ساقیا مستان خواب آلوده را بیدار کن}
\label{sec:300}
\addcontentsline{toc}{section}{\nameref{sec:300}}
\begin{longtable}{l p{0.5cm} r}
ساقیا مستان خواب آلوده را بیدار کن
&&
از فروغ باده رنگ رویشان گلنار کن
\\
لاابالی پیشه‌گیر و عاشقی بر طاق نه
&&
عشق را در کار گیر و عقل را بیکار کن
\\
گر ز چرخ چنبری از غم همی خواهی نجات
&&
دور باده پیش گیر و قصد زلف یار کن
\\
پنج حس و چار طبع از پنج باده برفروز
&&
وز دو گیتی دل به یکبار از خوشی بیزار کن
\\
دانشت بسیار باشد چونکه اندک می خوری
&&
دانشی کو غم فزاید از میش بردار کن
\\
ور ز راه پنج حس خواهی که یار آید ترا
&&
پنج باده نوش کن هر پنج در مسمار کن
\\
دوستار عشق گشتی دشمن جانان مشو
&&
چاکری می چون گرفتی بندگی خمار کن
\\
ور به عمر اندر به نادانی نشسته بوده‌ای
&&
از زبان عاجزی یکدم یک استغفار کن
\\
\end{longtable}
\end{center}
