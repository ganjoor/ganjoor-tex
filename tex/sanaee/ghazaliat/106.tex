\begin{center}
\section*{غزل شماره ۱۰۶: سوال کرد دل من که دوست با تو چه کرد}
\label{sec:106}
\addcontentsline{toc}{section}{\nameref{sec:106}}
\begin{longtable}{l p{0.5cm} r}
سوال کرد دل من که دوست با تو چه کرد
&&
چرات بینم با اشک سرخ و با رخ زرد
\\
دراز قصه نگویم حدیث جمله کنم
&&
هر آنچه گفت نکرد و هر آنچه کشت نخورد
\\
جفا نمود و نبخشود و دل ربود و نداد
&&
وفا بگفت و نکرد و جفا نگفت و بکرد
\\
چو پیشم آمد کردم سلام روی بتافت
&&
چو آستینش گرفتم گفت بردا برد
\\
نه چاره‌ای که دل از دوستیش برگیرم
&&
نه حیله‌ای که توانمش باز راه آورد
\\
بر انتظار میان دو حال ماندستم
&&
کشید باید رنج و چشید باید درد
\\
ایا سنایی لولو ز دیدگانت مبار
&&
که در عقیلهٔ هجران صبور باید مرد
\\
\end{longtable}
\end{center}
