\begin{center}
\section*{غزل شماره ۲۵: ای لعبت صافی صفات ای خوشتر از آب حیات}
\label{sec:025}
\addcontentsline{toc}{section}{\nameref{sec:025}}
\begin{longtable}{l p{0.5cm} r}
ای لعبت صافی صفات ای خوشتر از آب حیات
&&
هستی درین آخر زمان این منکران را معجزات
\\
هم دیده داری هم قدم هم نور داری هم ظلم
&&
در هزل وجد ای محتشم هم کعبه گردی هم منات
\\
حسن ترا بینم فزون خلق ترا بینم زبون
&&
چون آمد از جنت برون چون تو نگاری بی برات
\\
در نارم از گلزار تو بیزارم از آزار تو
&&
یک دیدن از دیدار تو خوشتر ز کل کاینات
\\
هر گه که بگشایی دهن گردد جهان پر نسترن
&&
بر تو ثنا گوید چو من ریگ و مطر سنگ و نبات
\\
عالی چو کعبه کوی تو نه خاکپای روی تو
&&
بر دو لب خوشبوی تو جان را به دل دارد حیات
\\
برهان آن نوشین لبت چون روز گرداند شبت
&&
وان خالها بر غبغبت تابان چو از گردون بنات
\\
بر ما لبت دعوت کنی بر ما سخن حجت کنی
&&
وقتی که جان غارت کنی چون صوفیان در ده صلات
\\
باز ار بکشتی عاجزی بنمای از لب معجزی
&&
چون از عزی نبود عزی لا را بزن بر روی لات
\\
غمهات بر ما جمله شد بغداد همچون حله شد
&&
یک دیده اینجا دجله شد یک دیده آنجا شد فرات
\\
جان سنایی مر ترا از وی حذر کردن چرا
&&
از تو گذر نبود ورا هم در حیات و هم ممات
\\
ای چون ملک گه سامری وی چون فلک گه ساحری
&&
تا بر تو خوانم یک سری «الباقیات الصالحات»
\\
\end{longtable}
\end{center}
