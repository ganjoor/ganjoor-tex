\begin{center}
\section*{غزل شماره ۱۲۶: مردمان دوستی چنین نکنند}
\label{sec:126}
\addcontentsline{toc}{section}{\nameref{sec:126}}
\begin{longtable}{l p{0.5cm} r}
مردمان دوستی چنین نکنند
&&
هر زمان اسب هجر زین نکنند
\\
جنگ و آزار و خشم یکباره
&&
مذهب و اعتقاد و دین نکنند
\\
چون کسی را به مهر بگزینند
&&
دیگری را بر او گزین نکنند
\\
در رخ دوستان کمان نکشند
&&
بر دل عاشقان کمین نکنند
\\
چون منی را به چاره‌ها کردن
&&
دل بیگانه را رهین نکنند
\\
روز و شب اختیار مهر کنند
&&
سال و مه آرزوی کین نکنند
\\
چون وفا خوبتر بود که جفا
&&
آن کنند اختیار و این نکنند
\\
بر سماع حزین خورند شراب
&&
لیک عاشق را حزین نکنند
\\
زلف پر چین ز بهر فتنهٔ خلق
&&
همچو زلف بتان چین نکنند
\\
اینهمه می‌کنی و پنداری
&&
که ترا خلق پوستین نکنند
\\
مکن ای لعبت پری‌زاده
&&
که پری‌زادگان چنین نکنند
\\
همه شاه و گدا و میر و وزیر
&&
بهر دنیا به ترک دین نکنند
\\
\end{longtable}
\end{center}
