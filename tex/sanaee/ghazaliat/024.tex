\begin{center}
\section*{غزل شماره ۲۴: از آن می خوردن عشقست دایم کار من هر شب}
\label{sec:024}
\addcontentsline{toc}{section}{\nameref{sec:024}}
\begin{longtable}{l p{0.5cm} r}
از آن می خوردن عشقست دایم کار من هر شب
&&
که بی من در خراباتست دایم یار من هر شب
\\
بتم را عیش و قلاشیست بی من کار هر روزی
&&
خروش و ناله و زاریست بی او کار من هر شب
\\
من آن رهبان خود نامم من آن قلاش خود کامم
&&
که دستوری بود ابلیس را کردار من هر شب
\\
برهنه پا و سر زانم که دایم در خراباتم
&&
همی باشد گرو هم کفش و هم دستار من هر شب
\\
همه شب مست و مخمورم به عشق آن بت کافر
&&
مغان دایم برند آتش ز بیت‌النار من هر شب
\\
مرا گوید به عشق اندر چرا چندین همی نالی
&&
نگار من چو بیند چشم گوهر بار من هر شب
\\
دو صد زنار دارم بر میان بسته به روم اندر
&&
همی بافند رهبانان مگر زنار من هر شب
\\
\end{longtable}
\end{center}
