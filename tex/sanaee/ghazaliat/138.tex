\begin{center}
\section*{غزل شماره ۱۳۸: ای یار بی تکلف ما را نبید باید}
\label{sec:138}
\addcontentsline{toc}{section}{\nameref{sec:138}}
\begin{longtable}{l p{0.5cm} r}
ای یار بی تکلف ما را نبید باید
&&
وین قفل رنج ما را امشب کلید باید
\\
جام و سماع و شاهد حاضر شدند باری
&&
وین خرقه‌های دعوی بر هم درید باید
\\
ایمان و زاهدی را بر هم شکست باید
&&
زنار جاحدی را از جان خرید باید
\\
از روی آن صنوبر ما را چراغ باید
&&
وز زلف آن ستمگر ما را گزید باید
\\
جامی بهای جانی بستان ز دست دلبر
&&
آمد مراد حاصل اکنون مرید باید
\\
چون مطربان خوشدل گشتند جمله حاضر
&&
پایی بکوفت باید بیتی شنید باید
\\
ای ساقی سمنبر در ده تو بادهٔ تر
&&
زیرا صبوح ما را «هل من مزید» باید
\\
از بادهٔ تو مستند ای دوست این عزیزان
&&
رنج و عنای مستان اکنون کشید باید
\\
سالی برفت ناگه روزی دو عید دیدم
&&
این هر دو عید امروز خوشتر ز عید باید
\\
از بوستان رحمت حالی کرانه جویید
&&
چون در سرای همت می‌آرمید باید
\\
از گفتن عبارت گر عبرتی نگیری
&&
در گردن اشارت معنی گزید باید
\\
تا در مکان امنی خر پشته زن فرود آی
&&
چون وقت کوچ آمد نایی دمید باید
\\
گر بایدت که بویی آنجا گل عنایت
&&
اینجا گل ریاست می‌پژمرید باید
\\
ای شکر شگرفی در گفتگوی معنی
&&
گر لب شفات آرد آخر بدید باید
\\
هر چند دیر مانی آخر برفت باید
&&
چون شکری بخوردی زهری چشید باید
\\
بفروخته خریدی آورده را ببردی
&&
یاری چه دیده‌ای تو زین پس چه دید باید
\\
چون لاله گر بخندی عمرت کرانه جوید
&&
چون شمع اگر بگریی حلقت برید باید
\\
\end{longtable}
\end{center}
