\begin{center}
\section*{غزل شماره ۱۱۴: زهی چابک زهی شیرین بنامیزد بنامیزد}
\label{sec:114}
\addcontentsline{toc}{section}{\nameref{sec:114}}
\begin{longtable}{l p{0.5cm} r}
زهی چابک زهی شیرین بنامیزد بنامیزد
&&
زهی خسرو زهی شیرین بنامیزد بنامیزد
\\
میان مجلس عشرت ز گم گویی و خوشخویی
&&
زهی سوسن زهی نسرین بنامیزد بنامیزد
\\
میان مردمان اندر ز خوش خویی و دلجویی
&&
زهی زهره زهی پروین بنامیزد بنامیزد
\\
دو قبضه جان همی باشد به غمزه ناوک مژگانت
&&
زهی ناوک زهی زوبین بنامیزد بنامیزد
\\
خرد زان صورت و سیرت همی عاجز فروماند
&&
زهی آیین زهی آذین بنامیزد بنامیزد
\\
مرا گفتی تویی عاشق بدین ره جان و دل در باز
&&
زهی فرمان زهی تلقین بنامیزد بنامیزد
\\
ز درد عشق خود رستم ز درد خویشتن بینی
&&
زهی شربت زهی تسکین بنامیزد بنامیزد
\\
چو چشم و شکل دندانت ببینم هر زمان گویم
&&
زهی طاها زهی یاسین بنامیزد بنامیزد
\\
گل افشان شد همی چشمم ز نعل سم یک رانت
&&
زهی امکان زهی تمکین بنامیزد بنامیزد
\\
سگی خواندی سنایی را وانگه گفتی آن من
&&
زهی احسان زهی تحسین بنامیزد بنامیزد
\\
\end{longtable}
\end{center}
