\begin{center}
\section*{غزل شماره ۱۵۲: میر خوبان را کنون منشور خوبی در رسید}
\label{sec:152}
\addcontentsline{toc}{section}{\nameref{sec:152}}
\begin{longtable}{l p{0.5cm} r}
میر خوبان را کنون منشور خوبی در رسید
&&
مملکت بر وی سهی شد ملک بر وی آرمید
\\
نامه آن نامه‌ست کاکنون عاشقی خواهد نوشت
&&
پرده آن پرده‌ست کاکنون عاشقی خواهد درید
\\
دلبران را جان همی بر روی او باید فشاند
&&
نوخطان را می همی بر یاد او باید چشید
\\
آفت جانهای ما شد خط دلبندش ولیک
&&
آفت جان را ز بت رویان به جان باید خرید
\\
گویی اکنون راست شد «والشمس» اندر آسمان
&&
آیت «واللیل» کرد و «الضحاش» اندر کشید
\\
گر ز مرد گرد بیجاده‌ش پدید آمد چه شد
&&
خرمی باید که اندر سبزه زیباتر نبید
\\
هر چه عمرش بیش گردد بیش گرداند زمان
&&
چون غزلهای سنایی تری اندر وی پدید
\\
کی تبه گرداندش هرگز به دست روزگار
&&
صورتی کایزد برای عشقبازی آفرید
\\
\end{longtable}
\end{center}
