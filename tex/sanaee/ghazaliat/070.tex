\begin{center}
\section*{غزل شماره ۷۰: تا سوی خرابات شد آن شاه خرابات}
\label{sec:070}
\addcontentsline{toc}{section}{\nameref{sec:070}}
\begin{longtable}{l p{0.5cm} r}
تا سوی خرابات شد آن شاه خرابات
&&
همواره منم معتکف راه خرابات
\\
کردند همه خلق همی خطبهٔ شاهی
&&
چون خیل خرابات بر آن شاه خرابات
\\
من خود چه خطر دارم تا بنده نباشم
&&
چون شاه خرابات بود ماه خرابات
\\
گر صومعهٔ شیخ خبر یابد ازین حرف
&&
حقا که شود بندهٔ خرگاه خرابات
\\
بشنو که سنایی سخن صدق به تحقیق
&&
آن کس که چنو نیست هواخواه خرابات
\\
او نیست به جز صورت بی هیئت بی روح
&&
افگنده به میدان شهنشاه خرابات
\\
آن روز مبادم من و آن روز مبادا
&&
بینند ز من خالی درگاه خرابات
\\
شیر نر اگر سوی خرابات خرامد
&&
روباه کند او را روباه خرابات
\\
آنکو «لمن الملک» زند هم حسد آید
&&
او را ز خرابات و علی‌الاه خرابات
\\
\end{longtable}
\end{center}
