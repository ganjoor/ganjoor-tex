\begin{center}
\section*{غزل شماره ۱۳۲: روی او ماهست اگر بر ماه مشک افشان بود}
\label{sec:132}
\addcontentsline{toc}{section}{\nameref{sec:132}}
\begin{longtable}{l p{0.5cm} r}
روی او ماهست اگر بر ماه مشک افشان بود
&&
قد او سروست اگر بر سرو لالستان بود
\\
گر روا باشد که لالستان بود بالای سرو
&&
بر مه روشن روا باشد که مشک افشان بود
\\
دل چو گوی و پشت چون چوگان بود عشاق را
&&
تا زنخدانش چو گوی و زلف چون چوگان بود
\\
گر ز دو هاروت او دلها نژند آید همی
&&
درد دلها را ز دو یاقوت او درمان بود
\\
من به جان مرجان و لولو را خریداری کنم
&&
گر چو دندان و لب او لولو و مرجان بود
\\
راز او در عشق او پنهان نماند تا مرا
&&
روی زرد و آه سرد و دیدهٔ گریان بود
\\
زان که غمازان من هستند هر سه پیش خلق
&&
هر کجا غماز باشد راز کی پنهان بود
\\
بر کنار خویش رضوان پرورد او را به ناز
&&
حور باشد هر که او پروردهٔ رضوان بود
\\
هر زمان گویم به شیرینی و پاکی در جهان
&&
چون لب و دندان او یارب لب و دندان بود
\\
\end{longtable}
\end{center}
