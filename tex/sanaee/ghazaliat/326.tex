\begin{center}
\section*{غزل شماره ۳۲۶: بند ترکش یک زمان ای ترک زیبا باز کن}
\label{sec:326}
\addcontentsline{toc}{section}{\nameref{sec:326}}
\begin{longtable}{l p{0.5cm} r}
بند ترکش یک زمان ای ترک زیبا باز کن
&&
با رهی یک دم بساز و خرمی را ساز کن
\\
جامهٔ جنگ از سر خود برکش و خوش طبع باش
&&
خانهٔ لهو و طرب را یک زمان در باز کن
\\
چند گه در رزم شه پرواز کردی گرد خصم
&&
گرد جام می کنون در بزم ما پرواز کن
\\
یک زمان با عشق خود می خور و دلشاد زی
&&
ترکی و مستی مکن چندان که خواهی ناز کن
\\
ناز ترکان خوش بود چندان که در مستی شود
&&
چون شوی مست و خراب آنگاه ناز آغاز کن
\\
ناز و مستی دلبران بر عاشقان زیبا بود
&&
ناز را با مستی اندر دلبری دمساز کن
\\
گر شکار خویش خواهی کرد جملهٔ خلق را
&&
زلف را گه چون کمند و گه چو چنگ باز کن
\\
مهر تو گردنکشان را صید تو کرد آنگهی
&&
پادشه امروز گشتی در جهان آواز کن
\\
\end{longtable}
\end{center}
