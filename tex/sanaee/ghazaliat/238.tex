\begin{center}
\section*{غزل شماره ۲۳۸: من نصیب خویش دوش از عمر خود برداشتم}
\label{sec:238}
\addcontentsline{toc}{section}{\nameref{sec:238}}
\begin{longtable}{l p{0.5cm} r}
من نصیب خویش دوش از عمر خود برداشتم
&&
کز سمن بالین و از شمشاد بستر داشتم
\\
داشتم در بر نگاری را که از دیدار او
&&
پایهٔ تخت خود از خورشید برتر داشتم
\\
نرگس و شمشاد و سوسن مشک و سیم و ماه و گل
&&
تا به هنگام سحر هر هفت در بر داشتم
\\
بر نهاده بر بر چون سیم و سوسن داشتم
&&
لب نهاده بر لب چون شیر و شکر داشتم
\\
دست او بر گردن من همچو چنبر بود و من
&&
دست خود در گردن او همچو چنبر داشتم
\\
بامدادان چون نگه کردم بسی فرقی نبود
&&
چنیر از زر داشت او سوسن ز عنبر داشتم
\\
چون موذن گفت یک الله اکبر کافرم
&&
گر امید آن دگر الله اکبر داشتم
\\
\end{longtable}
\end{center}
