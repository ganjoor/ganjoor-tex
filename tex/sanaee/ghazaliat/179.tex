\begin{center}
\section*{غزل شماره ۱۷۹: ساقیا می ده و نمی کم گیر}
\label{sec:179}
\addcontentsline{toc}{section}{\nameref{sec:179}}
\begin{longtable}{l p{0.5cm} r}
ساقیا می ده و نمی کم گیر
&&
وز سر زلف خود خمی کم گیر
\\
گر به یک دم بمانده‌ای در دام
&&
جستی از دام پس دمی کم گیر
\\
رو که عیسی دلیل و همره تست
&&
ره همی رو تو مریمی کم گیر
\\
عالمی علم بر تو جمع شدست
&&
علم باقیست عالمی کم گیر
\\
ز کما بیش بر تو نقصان نیست
&&
چون تو بیشی ز کم کمی کم گیر
\\
بم گسسته ست زیر و زار خوشست
&&
زحمت زخمه را به می کم گیر
\\
گر سنایی غمی‌ست بر دل تو
&&
یا غمی باش یا غمی کم گیر
\\
\end{longtable}
\end{center}
