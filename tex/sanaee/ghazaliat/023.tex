\begin{center}
\section*{غزل شماره ۲۳: فریاد از آن دو چشمک جادوی دلفریب}
\label{sec:023}
\addcontentsline{toc}{section}{\nameref{sec:023}}
\begin{longtable}{l p{0.5cm} r}
فریاد از آن دو چشمک جادوی دلفریب
&&
فریاد از آن دو کافر غازی با نهیب
\\
این همبر دو ترکش دلگیر جان ستان
&&
وان پیش دو شمامهٔ کافور یا دو سیب
\\
بردوش غایه کش او زهره می‌رود
&&
چون کیقباد و قیصر پانصدش در رکیب
\\
یوسف نبود هرگز چون او به نیکویی
&&
چون سامری هزارش چاکر گه فریب
\\
آسیب عاشقی و غم عشق و گمرهی
&&
تا روی او بدید پس آن طرفه‌ها و زیب
\\
غمخانه برگزید و ره عشق و گمرهی
&&
هر روز می برآرد نوعی دگر ز جیب
\\
بسترد و گفت چون که سنایی همه ز جهل
&&
بنبشت در هوای غم عشق صد کتیب
\\
\end{longtable}
\end{center}
