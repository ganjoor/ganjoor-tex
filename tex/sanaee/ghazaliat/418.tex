\begin{center}
\section*{غزل شماره ۴۱۸: ای سنایی چو تو در بند دل و جان باشی}
\label{sec:418}
\addcontentsline{toc}{section}{\nameref{sec:418}}
\begin{longtable}{l p{0.5cm} r}
ای سنایی چو تو در بند دل و جان باشی
&&
کی سزاوار هوای رخ جانان باشی
\\
در دریا تو چگونه به کف آری که همی
&&
به لب جوی چو اطفال هراسان باشی
\\
چون به ترک دل و جان گفت نیاری آن به
&&
که شوی دور ازین کوی و تن آسان باشی
\\
تا تو فرمانبر چوگان سواران نشوی
&&
نیست ممکن که تو اندر خور میدان باشی
\\
کار بر بردن چوگان نبود صنعت تو
&&
تو همان به که اسیر خم چوگان باشی
\\
به عصایی و گلیمی که تو داری پسرا
&&
تو همی خواهی چون موسی عمران باشی
\\
خواجهٔ ما غلطی کردست این راه مگر
&&
خود نه بس آنکه نمیری و مسلمان باشی
\\
\end{longtable}
\end{center}
