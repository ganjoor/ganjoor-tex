\begin{center}
\section*{غزل شماره ۲۰۰: چه رسمست آن نهادن زلف بر دوش}
\label{sec:200}
\addcontentsline{toc}{section}{\nameref{sec:200}}
\begin{longtable}{l p{0.5cm} r}
چه رسمست آن نهادن زلف بر دوش
&&
نمودن روز را در زیر شب پوش
\\
گه از بادام کردن جعبهٔ نیش
&&
گه از یاقوت کردن چشمهٔ نوش
\\
برآوردن برای فتنهٔ خلق
&&
هزاران صبحدم از یک بناگوش
\\
تو خورشیدی از آن پیش تو آرند
&&
فلک را از مه نو حلقه در گوش
\\
پری و سرو و خورشیدی ولیکن
&&
قدح گیر و کمربند و قباپوش
\\
گل و مه پیش تو بر منبر حسن
&&
همه آموخته کرده فراموش
\\
سنایی را خریدستی دل و جان
&&
اگر صد جان دهندت باز مفروش
\\
\end{longtable}
\end{center}
