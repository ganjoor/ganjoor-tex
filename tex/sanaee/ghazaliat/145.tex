\begin{center}
\section*{غزل شماره ۱۴۵: هر که در کوی خرابات مرا بار دهد}
\label{sec:145}
\addcontentsline{toc}{section}{\nameref{sec:145}}
\begin{longtable}{l p{0.5cm} r}
هر که در کوی خرابات مرا بار دهد
&&
به کمال و کرمش جان من اقرار دهد
\\
بار در کوی خرابات مرا هیچ کسی
&&
ندهد ور دهد آن یار وفادار دهد
\\
در خرابات بود یار من و من شب و روز
&&
به سر کوی همی گردم تا بار دهد
\\
ای خوشا کوی خرابات که پیوسته در او
&&
مر مرا دوست همی وعدهٔ دیدار دهد
\\
هر که او حال خرابات بداند به درست
&&
هر چه دارد همه در حال به بازار دهد
\\
در خرابات نبینی که ز مستی همه سال
&&
راهب دیر ترا کشتی و زنار دهد
\\
آنکه چون باشد هشیار به فرزند عزیز
&&
در می سیم به صد زاری دشخوار دهد
\\
هر دو عالم را چون مست شود از دل و جان
&&
به بهای قدح می دهد و خوار دهد
\\
آنکه بیرون خرابات به قطمیر و نقیر
&&
چون در آید به خرابات به قنطار دهد
\\
آنکه نانی همه آفاق بود در چشمش
&&
در خرابات به می جبه و دستار دهد
\\
آنکه او کیسه ز طرار نگهدارد چون
&&
به خرابات شود کیسه به طرار دهد
\\
ای تو کز کوی خرابات نداری گذری
&&
زان سناییت همی پند به مقدار دهد
\\
تو برو زاویهٔ زهد نگهدار و مترس
&&
که خداوند سزا را به سزاوار دهد
\\
\end{longtable}
\end{center}
