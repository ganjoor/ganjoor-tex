\begin{center}
\section*{غزل شماره ۲۴۷: فراق آمد کنون از وصل برخوردار چون باشم}
\label{sec:247}
\addcontentsline{toc}{section}{\nameref{sec:247}}
\begin{longtable}{l p{0.5cm} r}
فراق آمد کنون از وصل برخوردار چون باشم
&&
جدا گردید یار از من جدا از یار چون باشم
\\
به چشم ار نیستم گنج عقیق و لولو و گوهر
&&
عقیق‌افشان و گوهربیز و لولوبار چون باشم
\\
کسی کوبست خواب من در آب افگند پنداری
&&
چو خوابم شد تبه در آب جز بیدار چون باشم
\\
بت من هست دلداری و زود آزار و من دایم
&&
دل آزرده ز عشق یار زود آزار چون باشم
\\
دهانش نیم دینارست و دینارست روی من
&&
چو از دینار بی‌بهرم به رخ دینار چون باشم
\\
ز بی‌خوابی همی خوانم به عمدا این غزل هردم
&&
«همه شب مردمان در خواب و من بیدار چون باشم»
\\
\end{longtable}
\end{center}
