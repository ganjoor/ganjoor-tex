\begin{center}
\section*{غزل شماره ۴۱۲: انصاف بده که نیک یاری}
\label{sec:412}
\addcontentsline{toc}{section}{\nameref{sec:412}}
\begin{longtable}{l p{0.5cm} r}
انصاف بده که نیک یاری
&&
زو هیچ مگو که خوش نگاری
\\
در رود زدن شکر سماعی
&&
در گوی زدن شکر سواری
\\
مه جبهت و آفتاب رویی
&&
زهره دل و مشتری عذاری
\\
بنوشت زمانه گویی آنجا
&&
در جانت کتاب بردباری
\\
بنگاشت خدای گویی اینجا
&&
در دیده‌ت نقش حقگزاری
\\
از لعل تو هست عاقلان را
&&
یک نوش و هزار گونه خاری
\\
در جزع تو هست عاشقان را
&&
یک غمزه و صد هزار خاری
\\
جز غمزهٔ تو که دید هرگز
&&
یک ناوک و صد جهان حصاری
\\
جز خندهٔ تو که داشت در دهر
&&
یک شکر و نه فلک شکاری
\\
در رزم تو هیچ دل نپوشد
&&
بر تن زره ستیزه‌کاری
\\
در بزم تو هیچ شه ندارد
&&
بر سر کله بزرگواری
\\
ای شوخ سیه‌گری که از تو
&&
کم دید کسی سپیدکاری
\\
از ابجد برتری ازیراک
&&
نی یک نه دو نه سه نه چهاری
\\
سرمازدگان آب و گل را
&&
در جمله، بهار در بهاری
\\
جان و دل و دین بنده با تست
&&
تا اینهمه را چگونه داری
\\
چون بازسپید دلفریبی
&&
چون شیرسیاه جانشکاری
\\
تا پای من اندرین میانست
&&
دستی به سرم فرو نیاری
\\
من پای فرو نهادم ایراک
&&
دانم سر پای من نداری
\\
دشنام دهی که ای سنایی
&&
بس خوش سخن و بزرگواری
\\
هر چند جواب شرط من نیست
&&
با این همه صد هزار باری
\\
\end{longtable}
\end{center}
