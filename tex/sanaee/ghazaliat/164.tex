\begin{center}
\section*{غزل شماره ۱۶۴: چون رخ به سراب آری ای مه به شراب اندر}
\label{sec:164}
\addcontentsline{toc}{section}{\nameref{sec:164}}
\begin{longtable}{l p{0.5cm} r}
چون رخ به سراب آری ای مه به شراب اندر
&&
اقبال گیا روید در عین سراب اندر
\\
ور رای شکار آری او شکر شکارت را
&&
الحمد کنان آید جانش به کباب اندر
\\
جلاب خرد باشد هر گه که تو در مجلس
&&
از شرم برآمیزی شکر به گلاب اندر
\\
راز «ارنی ربی» در سینه پدید آید
&&
گر زخم زند ما را چشم تو به خواب اندر
\\
جانها به شتاب آرد لعلت به درنگ اندر
&&
دلها به درنگ آرد لعلت به شتاب اندر
\\
هر لحظه یکی عیسی از پرده برون آری
&&
مریم کده‌ها داری گویی به حجاب اندر
\\
مهر تو برآمیزد پاکی به گناه اندر
&&
قهر تو درانگیزد دیوی به شهاب اندر
\\
ما و تو و قلاشی چه باک همی با تو
&&
راند پسر مریم خر را به خلاب اندر
\\
هر روز بهشتی نو ما را بدهی زان لب
&&
دندان نزنی هرگز با ما و ثواب اندر
\\
دانی که خراباتیم از زلزلهٔ عشقت
&&
کم رای خراج آید شه را به خراب اندر
\\
ما را ز میان ما چون کرد برون عشقت
&&
اکنون همه خود خوان خود ما را به خطاب اندر
\\
ما گر تو شدیم ای جان نشگفت که از قوت
&&
دراج عقابی شد چون شد به عقاب اندر
\\
ای جوهر روح ما در هم شده با عشقت
&&
چون بوی به باد اندر چون رنگ به آب اندر
\\
یارب چه لبی داری کز بهر صلاح ما
&&
جز آب نمی‌باشد با ما به شراب اندر
\\
از دل چکنی وقتی در عشق سوال او را
&&
در گوش طلب جان را چون شد به جواب اندر
\\
شعری به سجود آید اشعار سنایی را
&&
هر گه که تو بسرایی شعرش به رباب اندر
\\
\end{longtable}
\end{center}
