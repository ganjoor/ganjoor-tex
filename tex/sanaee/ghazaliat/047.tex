\begin{center}
\section*{غزل شماره ۴۷: ای پیک عاشقان گذری کن به بام دوست}
\label{sec:047}
\addcontentsline{toc}{section}{\nameref{sec:047}}
\begin{longtable}{l p{0.5cm} r}
ای پیک عاشقان گذری کن به بام دوست
&&
بر گرد بنده‌وار به گرد مقام دوست
\\
گرد سرای دوست طوافی کن و ببین
&&
آن بار و بارنامه و آن احتشام دوست
\\
خواهی که نرخ مشک شکسته شود به چین
&&
بر زن به زلف پر شکن مشکفام دوست
\\
برخاست اختیار و تصرف ز فعل ما
&&
چون کم ز دیم خویشتن از بهر کام دوست
\\
خواهی که بار عنبر بندی تو از سرخس
&&
زآنجا میار هیچ خبر جز پیام دوست
\\
خواهی که کاروان سلامت بود ترا
&&
همراه خویش کن به سوی ما سلام دوست
\\
بر دانه‌های گوهر او عاشقی مباز
&&
تا همچو من نژند نمانی به دام دوست
\\
با خود بیار خاک سر کوی او به من
&&
تا بر سرش نهم به عزیزی چو نام دوست
\\
بینا مباد چشم من ار سوی چشم من
&&
بهتر ز توتیا نبود گرد گام دوست
\\
گر دوست را به غربت من خوش بود همی
&&
ای من رهی غربت و ای من غلام دوست
\\
از مال و جان و دین مرا ار کام جوید او
&&
بی کام بادم ار کنم آن جز به کام دوست
\\
\end{longtable}
\end{center}
