\begin{center}
\section*{غزل شماره ۲۹۱: سنایی را یکی برهان ز ننگ و نام جان ای جان}
\label{sec:291}
\addcontentsline{toc}{section}{\nameref{sec:291}}
\begin{longtable}{l p{0.5cm} r}
سنایی را یکی برهان ز ننگ و نام جان ای جان
&&
ز عشق دانهٔ دو جهان میان دام جان ای جان
\\
مکن در قبهٔ زنگار اوصاف حروف او را
&&
چو عشق عافیت پخته چو کارم خام جان ای جان
\\
به قهر از دست او بستان حروف کلک صورت را
&&
به لطف از لوح او بستر تمامی نام جان ای جان
\\
چو روی خویش خرم کن یکی بستان طبع ای بت
&&
چو زلف خویش در هم زن همه ایام جان ای جان
\\
ببین در کوی کفر و دین به مهر و درد دل بنشست
&&
هزاران آه خون آلود زیر کام جان ای جان
\\
مرا گویی قناعت کن ز جوش یک جهان رعنا
&&
به بوی نون شهوانی به رنگ لام جان ای جان
\\
کسی کو عاشق تو بود بگو آخر که تا چکند
&&
سماع وحی و نقل عقل و خمر خام جان ای جان
\\
مگر تو زینهمه خوبان که پیدایند و ناپیدا
&&
درین مردودهٔ ویران نیابم کام جان ای جان
\\
\end{longtable}
\end{center}
