\begin{center}
\section*{غزل شماره ۲۶۸: ای دو زلفت دراز و بالا هم}
\label{sec:268}
\addcontentsline{toc}{section}{\nameref{sec:268}}
\begin{longtable}{l p{0.5cm} r}
ای دو زلفت دراز و بالا هم
&&
وی دو لعلت نهان و پیدا هم
\\
شوخ تنها که خواند چشم ترا
&&
چشم تو شوخ هست و رعنا هم
\\
بستهٔ تو هزار نادان هست
&&
چه عجب صدهزار دانا هم
\\
بستهٔ تست طبع ناگویا
&&
من چه گویم زبان گویا هم
\\
در دریا غلام خندهٔ تست
&&
ای شکر لب چه در ثریا هم
\\
کوه آتش همیشه همره تست
&&
کوه آتش مگو که دریا هم
\\
از قرینان نکوتری چون ماه
&&
نه که چون آفتاب تنها هم
\\
چند گویی سنایی آن منست
&&
با همه کس پلاس و با ما هم؟
\\
\end{longtable}
\end{center}
