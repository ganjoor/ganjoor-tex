\begin{center}
\section*{غزل شماره ۷۵: رازی ز ازل در دل عشاق نهانست}
\label{sec:075}
\addcontentsline{toc}{section}{\nameref{sec:075}}
\begin{longtable}{l p{0.5cm} r}
رازی ز ازل در دل عشاق نهانست
&&
زان راز خبر یافت کسی را که عیانست
\\
او را ز پس پردهٔ اغیار دوم نیست
&&
زان مثل ندارد که شهنشاه جهانست
\\
گویند ازین میدان آن را که درآمد
&&
کی خواجه دل و روح و روانت ز روانست
\\
گر ماه هلال آید در نعت کسوفست
&&
ور تیر وصال آید بر بسته کمانست
\\
کاین کوی دو صد بار هزار از سر معنی
&&
گشتست کز ایشان تف انگشت نشانست
\\
آنکس که ردایی ز ریا بر کتف افگند
&&
آن نیست ردا آن به صف دان طلسانست
\\
گر چند نگونست درین پرده دل ما
&&
میدان به حقیقت که ز اقبال ستانست
\\
قاف از خبر هیبت این خوف به تحقیق
&&
چون سین سلامت ز پی خواجه روانست
\\
گویی که مگر سینهٔ پر آتش دارد
&&
یا دیدهٔ او بر صفت بحر عمانست
\\
این چیست چنین باید اندر ره معنی
&&
آن کس که چنین نیست یقین دان که چنانست
\\
نظم گهر معنی در دیدهٔ دعوی
&&
چون مردمک دیده درین مقله نهانست
\\
در راه فنا باید جانهای عزیزان
&&
کاین شعر سنایی سبب قوت جانست
\\
\end{longtable}
\end{center}
