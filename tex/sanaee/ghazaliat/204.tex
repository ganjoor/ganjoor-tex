\begin{center}
\section*{غزل شماره ۲۰۴: ز جزع و لعلت ای سیمین بناگوش}
\label{sec:204}
\addcontentsline{toc}{section}{\nameref{sec:204}}
\begin{longtable}{l p{0.5cm} r}
ز جزع و لعلت ای سیمین بناگوش
&&
دلم پر نیش گشت و طبع پر نوش
\\
دو جادوی کمین ساز کمان کش
&&
دو نقاش شکر پاش گهر نوش
\\
که پیش این و آن جان را و دل را
&&
هزاران غاشیه ست امروز بر دوش
\\
چو بینمت آن دو تا لعل پر از کبر
&&
چو بینمت آن دو تا جزع پر از جوش
\\
بدین گویم زهی خاموش گویا
&&
بدان گویم زهی گویای خاموش
\\
بسا زهاد گیتی را که بردی
&&
بدان لبهای چون می مایهٔ هوش
\\
بسا شیران عالم را که دادی
&&
ز چشم آهوانه خواب خرگوش
\\
زنی گل را و مل را خاک در چشم
&&
چو اندر مجلس آیی زلف بر دوش
\\
ز مستی باز کرده بند کرته
&&
ز شوخی کج نهاده طرف شب پوش
\\
ز جزعت خانه خانه دل شود خون
&&
ز لعلت چشمه چشمه خون شود نوش
\\
گریزد در عدم هر روز و هم شب
&&
ز شرم روی و مویت چون دی و دوش
\\
تو جانی گر نه‌ای د ربر عجب نیست
&&
که جان در جان در آید نه در آغوش
\\
نگارا از سر آزاد مردی
&&
حدیث دردناک بنده بنیوش
\\
مرا چون از ولی بخریده‌ای دی
&&
کنونم بر عدو امروز مفروش
\\
مرا گفتی فراموشم مکن نیز
&&
تو روی از بهر این مخراش و مخروش
\\
که گشت از بهر یاد جزع و لعلت
&&
سنایی را فراموشی فراموش
\\
\end{longtable}
\end{center}
