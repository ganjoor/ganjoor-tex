\begin{center}
\section*{غزل شماره ۴۲۶: تو آفت عقل و جان و دینی}
\label{sec:426}
\addcontentsline{toc}{section}{\nameref{sec:426}}
\begin{longtable}{l p{0.5cm} r}
تو آفت عقل و جان و دینی
&&
تو رشک پری و حور عینی
\\
تا چشم تو روی تو نبیند
&&
تو نیز چو خویشتن نبینی
\\
ای در دل و جان من نشسته
&&
یک جال دو جای چون نشینی
\\
سروی و مهی عجایب تو
&&
نه بر فلک و نه بر زمینی
\\
بی روی تو عقل من نه خوبست
&&
در خاتم عقل من نگینی
\\
بر مهر تو دل نهاد نتوان
&&
تو اسب فراق کرده زینی
\\
گه یار قدیم را برانی
&&
گه یار نوآمده گزینی
\\
این جور و جفات نه کنونست
&&
دیریست بتا که تو چنینی
\\
ای بوقلمون کیش و دینم
&&
گه کفر منی و گاه دینی
\\
\end{longtable}
\end{center}
