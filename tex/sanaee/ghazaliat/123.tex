\begin{center}
\section*{غزل شماره ۱۲۳: بر مه از عنبر معشوق من چنبر کند}
\label{sec:123}
\addcontentsline{toc}{section}{\nameref{sec:123}}
\begin{longtable}{l p{0.5cm} r}
بر مه از عنبر معشوق من چنبر کند
&&
هیچ کس دیدی که بر مه چنبر از عنبر کند
\\
گه ز مشک سوده نقش آرد همی بر آفتاب
&&
گه عبیر بیخته بر لالهٔ احمر کند
\\
گرد زنگارش پدید آمد ز روی برگ گل
&&
ترسم امسالش بنفشه از سمن سر بر کند
\\
ای دریغا آن پریرو از نهیب چشم بد
&&
سوسن آزاده را در زیر سیسنبر کند
\\
هر که دید آن خط نورسته بدان یاقوت سرخ
&&
عاجز آید گر صفات رنگ نیلوفر کند
\\
خیز تا یک چند بر دیدار او باده خوریم
&&
پیش از آن کش روزگار بی وفا ساغر کند
\\
مهره بازی دارد اندر لب که همچون بلعجب
&&
گه عقیق کانی و گه در و گه شکر کند
\\
چشم جان آهنج دل الفنج جادو بند او
&&
جادویی داند مگر کز جزع من عبهر کند
\\
آفرین بادا بر آن رویی که گر بیند پری
&&
بی گمان از رشک رویش خاک را بر سر کند
\\
این چنین دلبر که گفتم در صفات عشق من
&&
گه دو چشمم پر ز آب و گه رخم پر زر کند
\\
گاه چون عودم بسوزد گه گدازد چون شکر
&&
گه چو زیر چنگم اندر چنگ رامشگر کند
\\
گه کند بر من جهان همچون دهان خویش تنگ
&&
گه تنم چون موی خویش آن لاله رخ لاغر کند
\\
گاه چون ذره نشاند مر مرا اندر هوا
&&
گه رخم از اشک چشمم زعفران پر زر کند
\\
ای مسلمانان فغان زان دلربای مستحیل
&&
کو جهان بر جان من چون سد اسکندر کند
\\
\end{longtable}
\end{center}
