\begin{center}
\section*{غزل شماره ۴۲۵: ای زبدهٔ راز آسمانی}
\label{sec:425}
\addcontentsline{toc}{section}{\nameref{sec:425}}
\begin{longtable}{l p{0.5cm} r}
ای زبدهٔ راز آسمانی
&&
وی حلهٔ عقل پر معانی
\\
ای در دو جهان ز تو رسیده
&&
آوازهٔ کوس «لن ترانی»
\\
ای یوسف عصر همچو یوسف
&&
افتاده به دست کاروانی
\\
لعل تو به غمزه کفر و دین را
&&
پرداخته مخزن امانی
\\
لعل تو به بوسه عقل و جان را
&&
برساخته عقل جاودانی
\\
با آفت زلف تو که بیند
&&
یک لحظه زعمر شادمانی
\\
با آتش عشق تو که یابد
&&
یک قطره ز آب زندگانی
\\
موسی چکند که بی‌جمالت
&&
نکشد غم غربت شبانی
\\
فرعون که بود که با کمالت
&&
کوبد در ملک جاودانی
\\
«آن» گویم «آن» چو صوفیانت
&&
نی نی که تو پادشاه آنی
\\
جان خوانم جان چو عاشقانت
&&
نی نی که تو کدخدای جانی
\\
از جملهٔ عاشقان تو نیست
&&
یکتن چو سنایی و تو دانی
\\
زیبد که سبک نداری او را
&&
گر گه گهکی کند گرانی
\\
\end{longtable}
\end{center}
