\begin{center}
\section*{غزل شماره ۱۵۹: زهی حسن و زهی عشق و زهی نور و زهی نار}
\label{sec:159}
\addcontentsline{toc}{section}{\nameref{sec:159}}
\begin{longtable}{l p{0.5cm} r}
زهی حسن و زهی عشق و زهی نور و زهی نار
&&
زهی خط و زهی زلف و زهی مور و زهی مار
\\
به نزدیک من از شق زهی شور و زهی شر
&&
به درگاه تو از حسن زهی کار و زهی بار
\\
به بالا و کمرگاه به زلفین و به مژگان
&&
زهی تیر و زهی تار و زهی قیر و زهی قار
\\
یکی گلبنی از روح گلت عقل و گلت عشق
&&
زهی بیخ و زهی شاخ و زهی برگ و زهی بار
\\
بهشت از تو و گردون حواس از تو و ارکان
&&
زهی هشت و زهی هفت زهی پنج و زهی چار
\\
برین فرق و برین دست برین روی و برین دل
&&
زهی خاک و زهی باد زهی آب و زهی نار
\\
میان خرد و روح دو زلفین و دو چشمت
&&
زهی حل و زهی عقد زهی گیر و زهی دار
\\
همه دل سوختگان را از سر زلف و زنخدانت
&&
زهی جاه و زهی چاه زهی بند و زهی بار
\\
به نزدیک سناییست ز عشق تو و غیرت
&&
زهی نام و زهی ننگ زهی فخر و زهی عار
\\
\end{longtable}
\end{center}
