\begin{center}
\section*{غزل شماره ۱۹۴: ای سنایی جان ده و در بند کام دل مباش}
\label{sec:194}
\addcontentsline{toc}{section}{\nameref{sec:194}}
\begin{longtable}{l p{0.5cm} r}
ای سنایی جان ده و در بند کام دل مباش
&&
راه رو چون زندگان چون مرده بر منزل مباش
\\
چون نپاشی آب رحمت نار زحمت کم فروز
&&
ور نباشی خاک معنی آب بی حاصل مباش
\\
رافت یاران نباشی آفت ایشان مشو
&&
سیرت حق چون نباشی صورت باطل مباش
\\
در میان عارفان جز نکتهٔ روشن مگوی
&&
در کتاب عاشقان جز آیت مشکل مباش
\\
در منای قرب یاران جان اگر قربان کنی
&&
جز به تیغ مهر او در پیش او بسمل مباش
\\
گر همی خواهی که با معشوق در هودج بوی
&&
با عدو و خصم او همواره در محمل مباش
\\
گر شوی جان جز هوای دوست رامسکن مشو
&&
ور شوی دل جز نگار عشق را قابل مباش
\\
روی چون زی کعبه کردی رای بتخانه مکن
&&
دشمنان دوست را جز حنظل قاتل مباش
\\
در نهاد تست با تو دشمن معشوق تو
&&
مانع او گر نه‌ای باری بدو مایل مباش
\\
\end{longtable}
\end{center}
