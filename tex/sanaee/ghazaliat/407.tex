\begin{center}
\section*{غزل شماره ۴۰۷: زان خط که تو بر عارض گلنار کشیدی}
\label{sec:407}
\addcontentsline{toc}{section}{\nameref{sec:407}}
\begin{longtable}{l p{0.5cm} r}
زان خط که تو بر عارض گلنار کشیدی
&&
ابدال جهان را همه در کار کشیدی
\\
بر ماه به پرگار کشیدی خط مشکین
&&
دلها همه در نقطهٔ پرگار کشیدی
\\
هر دل که ترا جست چو دیوانهٔ مستی
&&
در سلسلهٔ زلف زره‌دار کشیدی
\\
زنار پرستی مکن ای بت که جهانی
&&
در سلسلهٔ زلف چو زنار کشیدی
\\
بس زاهد و عابد که بر آن طرهٔ طرار
&&
از صومعه در خانهٔ خمار کشیدی
\\
هر دل که سرافراشت به دعوی صبوری
&&
او را به سوی خویش نگونسار کشیدی
\\
\end{longtable}
\end{center}
