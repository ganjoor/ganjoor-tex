\begin{center}
\section*{غزل شماره ۱۳۴: نور تا کیست که آن پردهٔ روی تو بود}
\label{sec:134}
\addcontentsline{toc}{section}{\nameref{sec:134}}
\begin{longtable}{l p{0.5cm} r}
نور تا کیست که آن پردهٔ روی تو بود
&&
مشک خود کیست که آن بندهٔ موی تو بود
\\
ز آفتابم عجب آید که کند دعوی نور
&&
در سرایی که درو تابش روی تو بود
\\
در ترازوی قیامت ز پی سختن نور
&&
صد من عرش کم از نیم تسوی تو بود
\\
راه پر جان شود آن جای که گام تو بود
&&
گوش پر در شود آنجا که گلوی تو بود
\\
هر که او روی تو بیند ز پی خدمت تو
&&
هم به روی تو که پشتش چو به روی تو بود
\\
از تو با رنگ گل و بوی گلابیم از آنک
&&
خوی احمد بود آنجا که خوی تو بود
\\
دیدهٔ حور بر آن خاک همی رشک برد
&&
که بر آن نقش ز لعل سر کوی تو بود
\\
کافهٔ خلق همه پیش رخت سجده برد
&&
حور یا روح که باشد که کفوی تو بود
\\
قبلهٔ جایست همه سوی تو چون کعبه از آن
&&
قبلهٔ جان سنایی همه سوی تو بود
\\
\end{longtable}
\end{center}
