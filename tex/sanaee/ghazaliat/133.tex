\begin{center}
\section*{غزل شماره ۱۳۳: از هر چه گمان بر دلم یار نه آن بود}
\label{sec:133}
\addcontentsline{toc}{section}{\nameref{sec:133}}
\begin{longtable}{l p{0.5cm} r}
از هر چه گمان بر دلم یار نه آن بود
&&
پندار بد آن عشق و یقین جمله گمان بود
\\
آن ناز تکلف بد و آن مهر فسون بود
&&
وان عشق مجازی بد و آن سود و زیان بود
\\
بر روی رقم شد شرری کز دل و جان تافت
&&
و ز دیده برون آمد دردی که نهان بود
\\
توحید من آن زلف بشولیدهٔ او بود
&&
ایمان من آن روی چو خورشید جهان بود
\\
رویی که رقم بود برو دولت اسلام
&&
زلفی که درو مرتدی و کفر نشان بود
\\
بنمود رخ و روم به یک بار بشورید
&&
آیین بت بتگری از دیدن آن بود
\\
پس زلف برافشاند و جهان کفر پراکند
&&
الحق ز چنان زلف مسلمان نتوان بود
\\
گویی که درو پای عزیزان همه سر بود
&&
راهی که در وصل نکویان همه جان بود
\\
از خون جگر سیل وز دل پاره درو خاک
&&
منزلگهش از آتش سوزان دمان بود
\\
بس جان عزیزان که در آن راه فنا شد
&&
گور و لحد آنجا دهن شیر ژیان بود
\\
چون کعبهٔ آمال پدید آمد از دور
&&
گفتند رسیدیم سر ره بر آن بود
\\
بر درگه تو خوار و ز دیدار تو نومید
&&
بر خاک نشستند که افلاس بیان بود
\\
بیرون ز خیالی نبد آنجا که نظر بود
&&
افزون ز حدیثی نبد آنجا که گمان بود
\\
\end{longtable}
\end{center}
