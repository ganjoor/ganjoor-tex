\begin{center}
\section*{غزل شماره ۳۲۸: ای شوخ دیده اسب جفا بیش زین مکن}
\label{sec:328}
\addcontentsline{toc}{section}{\nameref{sec:328}}
\begin{longtable}{l p{0.5cm} r}
ای شوخ دیده اسب جفا بیش زین مکن
&&
ما را چو چشم خویش نژند و حزین مکن
\\
ای ماه روی بر سر ما هر زمان ز جور
&&
چون دور آسمان دگری به گزین مکن
\\
مهری که خود نهاده‌ای آن مهر بر مدار
&&
مهری که خود نموده‌ای آن مهر کین مکن
\\
گه چون خدای حاجت ما ز آستان مساز
&&
گه چون خلیفه نایب ما ز آستین مکن
\\
در خال و لب نگر سمر عز و دل مگوی
&&
در زلف و رخ نگر سخن کفر و دین مکن
\\
از زلف تابدار نشان گمان مجوی
&&
نوز روی شرم دار حدیث یقین مکن
\\
زلفت چو طوق گردن دیو لعین شدست
&&
رخ چون چراغ حجرهٔ روح‌الامین مکن
\\
ای ما به روح تیر تو با ما سنان مباش
&&
ای ما به تن کمان تو با ما کمین مکن
\\
خواهی که تا چو حلقه بمانیم بر درت
&&
با ما چو حلقه‌دار لبان چون نگین مکن
\\
خواهی که لاله پاش نگردد دو چشم من
&&
از روی خویش چشم خسان لاله چین مکن
\\
بنشانمان بر آتش و بر تیغ و زینهار
&&
با هجر خویشمان نفسی همنشین مکن
\\
تو هم میی و هم شکری هان و هان بتا
&&
از خود بترس و دیدهٔ ما را چو هین مکن
\\
ای از کمال و لطف و بزرگی بر آسمان
&&
عهد و وفا و خدمت ما در زمین مکن
\\
مردی نه کودکی که زنی هر دم از تری
&&
خود را چو کودکان و زنان نازنین مکن
\\
با تو وفا کنیم و تو با ما جفا کنی
&&
با ما همی چو آن نکنی باری این مکن
\\
آخر ترا که گفت که در جام بی‌دلان
&&
وقت علاج سرکه کن و انگبین مکن
\\
آخر ترا که گفت که با عاشقان خویش
&&
نان گندمین بدار و سخن گندمین مکن
\\
آنان فسرده‌اند کشان پوستین کنی
&&
ما را ز غم چو سوخته‌ای پوستین مکن
\\
گر چه غریب و بی کس و درویش و عاجزم
&&
ما را بپرس گه گهی آخر چنین مکن
\\
ای پیش تو سنایی گه یا و گه الف
&&
او را به تیغ هجر چو نون و چو سین مکن
\\
\end{longtable}
\end{center}
