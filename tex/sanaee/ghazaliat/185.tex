\begin{center}
\section*{غزل شماره ۱۸۵: چون تو نمودی جمال عشق بتان شد هوس}
\label{sec:185}
\addcontentsline{toc}{section}{\nameref{sec:185}}
\begin{longtable}{l p{0.5cm} r}
چون تو نمودی جمال عشق بتان شد هوس
&&
رو که ازین دلبران کار تو داری و بس
\\
با رخ تو کیست عقل جز که یکی بلفضول
&&
با لب تو کیست جان جز که یکی بلهوس
\\
کفر معطل نمود زلفت و دین حکیم
&&
نان موذن ببرد رویت و آب عسس
\\
با رخ و با زلف تو در سر بازار عشق
&&
فتنه به میدان درست عافیت اندر حرس
\\
روی تو از دل ببرد منزلت و قدر ناز
&&
موی تو از جان ببرد توش و توان و هوس
\\
جزع تو بر هم گسست بر همه مردان زره
&&
لعل تو در هم شکست بر همه مرغان قفس
\\
در بر تو با سماع بی خطران چون نجیب
&&
بر در تو با خروش بی خبران چون جرس
\\
دایهٔ تو حسن نست میبردت چپ و راست
&&
سایهٔ تو عشق ماست میدودت پیش و پس
\\
هستی دریای حسن از پی او همچنان
&&
نعل پی تست در تاج سر تست خس
\\
کرد مرا همچو صبح روی چو خورشید تو
&&
تا همه بی جان زنم در ره عشقت نفس
\\
تا به هم آورد سر آن خط چون مورچه
&&
بر همه چیزی نشست عشق تو همچون مگس
\\
جان همه عاشقان بر لب تو تعبیه‌ست
&&
ای همه با تو همه بی‌لب تو هیچکس
\\
انس سنایی بسست خاک سر کوی تو
&&
نور رخ مصطفا بس بود انس انس
\\
\end{longtable}
\end{center}
