\begin{center}
\section*{غزل شماره ۱۴۲: ای چو چشم سوزن عیسی دهانت}
\label{sec:142}
\addcontentsline{toc}{section}{\nameref{sec:142}}
\begin{longtable}{l p{0.5cm} r}
ای چو چشم سوزن عیسی دهانت
&&
هست گویی رشتهٔ مریم میانت
\\
چون دم عیسی‌زنی از چشم سوزن
&&
چشمهٔ خورشید گردد جان فشانت
\\
آنچه بر مریم ز راه آستین زد
&&
می‌توان یافت از هوای آستانت
\\
ماه کو از آسمان سازد زمینی
&&
بر زمین سر می‌نهد از آسمانت
\\
نقد صد دل بایدم در هر زمانی
&&
بر امید صید زلف دلستانت
\\
گرچه غلطان است در پای تو زلفت
&&
هم سری جز زلف نبود یک زمانت
\\
گر سخن چون زهر گویی باک نبود
&&
کان شکر دایم بماند در دهانت
\\
ور سخن خوش گویی ای جان و جهانم
&&
بنده گردد بی سخن جان و جهانت
\\
من روا دارم که کام من برآید
&&
ور فرو خواهد شدن جانم به جانت
\\
نیست جز دستان چو زلفت هیچ کارم
&&
زانکه دیدم روی همچون گلستانت
\\
گر به دستانی به دست آرد فریدت
&&
در فشاند در سخن همچون زبانت
\\
\end{longtable}
\end{center}
