\begin{center}
\section*{غزل شماره ۳۵۷: چو نقاب برگشائی مه آن جهان برآید}
\label{sec:357}
\addcontentsline{toc}{section}{\nameref{sec:357}}
\begin{longtable}{l p{0.5cm} r}
چو نقاب برگشائی مه آن جهان برآید
&&
ز فروغ نور رویت ز جهان فغان برآید
\\
هم دورهای عالم بگذشت و کس ندانست
&&
که رخ چو آفتابت ز چه آسمان برآید
\\
ز دو لعل جان‌فزایت دو جهان پر از گهر شد
&&
چو تو گوهری ندانم ز کدام کان برآید
\\
دل و جان عاشقانت ز غمت به جوش آید
&&
چو ز سر سینه نامت به سر زبان برآید
\\
ره عشق چون تویی را که سزد، کسی که بیخود
&&
چو فرو شود به کویت ز همه جهان برآید
\\
چه ره است این که هرکس که دمی بدو فروشد
&&
نه ازو خبر بماند نه ازو نشان برآید
\\
همه عمر عاشق تو شب و روز آن نکوتر
&&
که ز کفر و دین بیفتد که ز خان و مان برآید
\\
ز حجاب اگر برآیی برسند خلق در تو
&&
پس از آن دم اناالحق ز جهانیان برآید
\\
منم و غم تو دایم که کسی که در غم تو
&&
به تو در گریخت غمگین، ز تو شادمان برآید
\\
چو غم تو هست جانا چه غمم بود که دل را
&&
غم تو به غمگساری ز میان جان برآید
\\
ز پی تو جان عطار اگرش قبول باشد
&&
ز مکان خلاص یابد چو به لامکان برآید
\\
\end{longtable}
\end{center}
