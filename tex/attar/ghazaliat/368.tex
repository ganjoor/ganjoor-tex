\begin{center}
\section*{غزل شماره ۳۶۸: سر زلف تو پر خون می‌نماید}
\label{sec:368}
\addcontentsline{toc}{section}{\nameref{sec:368}}
\begin{longtable}{l p{0.5cm} r}
سر زلف تو پر خون می‌نماید
&&
رجوع از صیدش اکنون می‌نماید
\\
کمند زلف تو در صید یارب
&&
چگونه چست و موزون می‌نماید
\\
شب زلف تو خوش باد از پی آنک
&&
همه کارش شبیخون می‌نماید
\\
که می‌داند که آن زنجیر زلفت
&&
چگونه عقل مجنون می‌نماید
\\
چو زلف تو بشوریده است عالم
&&
رخت از پرده بیرون می‌نماید
\\
ز حسن روی تو چون روی تابم
&&
که هر ساعت در افزون می‌نماید
\\
عجب خاصیتی دارد رخ تو
&&
که از شبرنگ گلگون می‌نماید
\\
چو دریا چشم من زان گشت در عشق
&&
که درجت در مکنون می‌نماید
\\
دهانت ای عجب سی در مکنون
&&
ز چشم سوزنی چون می‌نماید
\\
مرا گفتی دلت یکرنگ گردان
&&
که صد رنگ او چو گردون می‌نماید
\\
مرا کو دل ندارم هیچ دل من
&&
وگر دارم دلی خون می‌نماید
\\
دل عطار با خاک در تو
&&
چو خونی کرده معجون می‌نماید
\\
\end{longtable}
\end{center}
