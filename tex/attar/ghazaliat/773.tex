\begin{center}
\section*{غزل شماره ۷۷۳: هم تن مویم از آن میان که نداری}
\label{sec:773}
\addcontentsline{toc}{section}{\nameref{sec:773}}
\begin{longtable}{l p{0.5cm} r}
هم تن مویم از آن میان که نداری
&&
تنگ دلم مانده زان دهان که نداری
\\
ننگری از ناز در زمین که دمی نیست
&&
سر ز تکبر بر آسمان که نداری
\\
من چه بلایی است هر نفس که ندارم
&&
تو چه نکویی است هر زمان که نداری
\\
هرچه بباید ز نیکویی همه هستت
&&
مثل بماند است در جهان که نداری
\\
نام وفا می‌بری و هیچ وفایی
&&
از تو نیاید بدان نشان که نداری
\\
گفته بدی عاقبت وفای تو آرم
&&
این ننیوشم از این زبان که نداری
\\
یک شکرم ده که سود بنده در آن است
&&
زانکه بسی افتد این زیان که نداری
\\
گرچه شکر داری و قیاس ندارد
&&
هست چو ندهی به کس چنان که نداری
\\
گفته بدی خون تو به درد بریزم
&&
تا برهی تو ز نیم جان که نداری
\\
تو نتوانی به خون من کمری بست
&&
خاصه کمر بر چنان میان که نداری
\\
بر تن عطار کز غمت چو کمانی است
&&
چند کشی آخر این کمان که نداری
\\
\end{longtable}
\end{center}
