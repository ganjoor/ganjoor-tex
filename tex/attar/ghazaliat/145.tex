\begin{center}
\section*{غزل شماره ۱۴۵: ای پرتو وجودت در عقل بی نهایت}
\label{sec:145}
\addcontentsline{toc}{section}{\nameref{sec:145}}
\begin{longtable}{l p{0.5cm} r}
ای پرتو وجودت در عقل بی نهایت
&&
هستی کاملت را نه ابتدا نه غایت
\\
هستی هر دو عالم در هستی تو گمشد
&&
ای هستی تو کامل باری زهی ولایت
\\
ای صد هزار تشنه، لب‌خشک و جان پرآتش
&&
افتاده پست گشته موقوف یک عنایت
\\
غیر تو در حقیقت یک ذره می‌نبینم
&&
ای غیر تو خیالی کرده ز تو سرایت
\\
چندان که سالکانت ره بیش پیش بردند
&&
ره پیش بیش دیدند بودند در بدایت
\\
چون این ره عجایب بس بی نهایت افتاد
&&
آخر که یابد آخر این راه را نهایت
\\
عطار در دل و جان اسرار دارد از تو
&&
چون مستمع نیابد پس چون کند روایت
\\
\end{longtable}
\end{center}
