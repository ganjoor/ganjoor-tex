\begin{center}
\section*{غزل شماره ۳۸۳: الا ای زاهدان دین دلی بیدار بنمایید}
\label{sec:383}
\addcontentsline{toc}{section}{\nameref{sec:383}}
\begin{longtable}{l p{0.5cm} r}
الا ای زاهدان دین دلی بیدار بنمایید
&&
همه مستند در پندار یک هشیار بنمایید
\\
ز دعوی هیچ نگشاید اگر مردید اندر دین
&&
چنان کز اندرون هستید در بازار بنمایید
\\
هزاران مرد دعوی دار بنماییم از مسجد
&&
شما یک مرد معنی‌دار از خمار بنمایید
\\
من اندر یک زمان صد مست از خمار بنمودم
&&
شما مستی اگر دارید از اسرار بنمایید
\\
خرابی را که دعوی اناالحق کرد از مستی
&&
به هر آدینه صد خونی به زیر دار بنمایید
\\
اگر صد خون بود ما را نخواهیم آن ز کس هرگز
&&
اگر این را جوابی هست بی انکار بنمایید
\\
خراباتی است پر رندان دعوی دار دردی کش
&&
میان خود چنین یک رند دعوی‌دار بنمایید
\\
من این رندان مفلس را همه عاشق همی بینم
&&
شما یک عاشق صادق چنین بیدار بنمایید
\\
به زیر خرقهٔ تزویر زنار مغان تا کی
&&
ز زیر خرقه گر مردید آن زنار بنمایید
\\
چو عیاران بی جامه میان جمع درویشان
&&
درین وادی بی پایان یکی عیار بنمایید
\\
ز نام و ننگ و زرق و فن نخیزد جز نگونساری
&&
یکن بی زرق و فن خود را قلندروار بنمایید
\\
کنون چون توبه کردم من ز بد نامی و بد کاری
&&
مرا گر دست آن دارید روی کار بنمایید
\\
مرا در وادی حیرت چرا دارید سرگردان
&&
مرا یک تن ز چندین خلق گو یکبار بنمایید
\\
شما عمری درین وادی به تک رفتید روز و شب
&&
ز گرد کوی او آخر مرا آثار بنمایید
\\
چه گویم جمله را در پیش راهی بس خطرناک است
&&
دلی از هیبت این راه بی‌تیمار بنمایید
\\
چنین بی آلت و بی دل قدم نتوان زدن در ره
&&
اگر مردان این راهید دست‌افزار بنمایید
\\
به رنج آید چنان گنجی به دست و خود که یابد آن
&&
وگر هستید از یابندگان دیار بنمایید
\\
درین ره با دلی پر خون به صد حیرت فروماندم
&&
درین اندیشه یک سرگشته چون عطار بنمایید
\\
\end{longtable}
\end{center}
