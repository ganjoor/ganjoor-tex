\begin{center}
\section*{غزل شماره ۵۹۱: در چه طلسم است که ما مانده‌ایم}
\label{sec:591}
\addcontentsline{toc}{section}{\nameref{sec:591}}
\begin{longtable}{l p{0.5cm} r}
در چه طلسم است که ما مانده‌ایم
&&
با تو به هم وز تو جدا مانده‌ایم
\\
نی که تویی جمله و ما هیچ نه
&&
مانده تویی ما ز کجا مانده‌ایم
\\
از همه معنی چو تویی هرچه هست
&&
پس به چه معنی من و ما مانده‌ایم
\\
رشته چو یکتاست در اصلی که هست
&&
پس ز برای چه دوتا مانده‌ایم
\\
چون تو سزاوار وجودی و بس
&&
ما نه به حق نه به سزا مانده‌ایم
\\
چون همه تو ما همه هیچ آمدیم
&&
ای همه تو هیچ چرا مانده‌ایم
\\
چون همه نه با تو و نه بی توییم
&&
نه به بقا نه به فنا مانده‌ایم
\\
در خم چوگان سر زلف تو
&&
گوی صفت بی سر و پا مانده‌ایم
\\
پاک کن از ما دل ما زانکه ما
&&
سوختهٔ خوف و رجا مانده‌ایم
\\
ما چو فریدیم نه نیک و نه بد
&&
کز دو جهان فرد تو را مانده‌ایم
\\
\end{longtable}
\end{center}
