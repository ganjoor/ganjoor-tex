\begin{center}
\section*{غزل شماره ۴۰۰: گر ز سر عشق او داری خبر}
\label{sec:400}
\addcontentsline{toc}{section}{\nameref{sec:400}}
\begin{longtable}{l p{0.5cm} r}
گر ز سر عشق او داری خبر
&&
جان بده در عشق و در جانان نگر
\\
چون کسی از عشق هرگز جان نبرد
&&
گر تو هم از عاشقانی جان مبر
\\
گر ز جان خویش سیری الصلا
&&
ور همی ترسی تو از جان الحذر
\\
عشق دریایی است قعرش ناپدید
&&
آب دریا آتش و موجش گهر
\\
گوهرش اسرار و هر سری ازو
&&
سالکی را سوی معنی راهبر
\\
سرکشی از هر دو عالم همچو موی
&&
گر سر مویی درین یابی خبر
\\
دوش مست و خفته بودم نیمشب
&&
کوفتاد آن ماه را بر من گذر
\\
دید روی زرد ما در ماهتاب
&&
کرد روی زرد ما از اشک تر
\\
رحمش آمد شربت وصلم بداد
&&
یافت یک یک موی من جانی دگر
\\
گرچه مست افتاده بودم زان شراب
&&
گشت یک یک موی بر من دیده‌ور
\\
در رخ آن آفتاب هر دو کون
&&
مست و لایعقل همی کردم نظر
\\
گرچه بود از عشق جانم پر سخن
&&
یک نفس نامد زبانم کارگر
\\
خفته و مستم گرفت آن ماه روی
&&
لاجرم ماندم چنین بی خواب و خور
\\
گاه می‌مردم گهی می‌زیستم
&&
در میان سوز چون شمع سحر
\\
عاقبت بانگی برآمد از دلم
&&
موج‌ها برخاست از خون جگر
\\
چون از آن حالت گشادم چشم باز
&&
نه ز جانان نام دیدم نه اثر
\\
من ز درد و حسرت و شوق و طلب
&&
می‌زدم چون مرغ بسمل بال و پر
\\
هاتفی آواز داد از گوشه‌ای
&&
کای ز دستت رفته مرغی معتبر
\\
خاک بر دنبال او بایست کرد
&&
تا نرفتی او ازین گلخن به در
\\
تن فرو ده آب در هاون مکوب
&&
در قفس تا کی کنی باد ای پسر
\\
بی نیازی بین که اندر اصل هست
&&
خواه مطرب باش و خواهی نوحه‌گر
\\
این کمان هرگز به بازوی تو نیست
&&
جان خود می‌سوز و حیران می‌نگر
\\
ماندی ای عطار در اول قدم
&&
کی توانی برد این وادی به سر
\\
\end{longtable}
\end{center}
