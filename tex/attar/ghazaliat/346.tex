\begin{center}
\section*{غزل شماره ۳۴۶: گر دلبرم به یک شکر از لب زبان دهد}
\label{sec:346}
\addcontentsline{toc}{section}{\nameref{sec:346}}
\begin{longtable}{l p{0.5cm} r}
گر دلبرم به یک شکر از لب زبان دهد
&&
مرغ دلم ز شوق به شکرانه جان دهد
\\
می‌ندهد او به جان گرانمایه بوسه‌ای
&&
پنداشتی که بوسه چنین رایگان دهد
\\
چون کس نیافت از دهن تنگ او خبر
&&
هر بی خبر چگونه خبر زان دهان دهد
\\
معدوم شیء گوید اگر نقطهٔ دلم
&&
جز نام از خیال دهانش نشان دهد
\\
مردی محال گوی بود آنکه بی خبر
&&
یک موی فی‌المثل خبر از آن میان دهد
\\
چون دید آفتاب که آن ماه هشت خلد
&&
از روی خود زکات به هفت آسمان دهد
\\
افتاد در غروب و فروشد خجل زده
&&
تا نوبت طلوع بدان دلستان دهد
\\
در آفتاب صد شکن آرم چو زلف او
&&
گر زلف او مرا سر مویی امان دهد
\\
ابروی چون کمانش که آن غمزه تیر اوست
&&
هر ساعتی چو تیر سرم در جهان دهد
\\
گویی که جور هندوی زلفش تمام نیست
&&
آخر به ترک مست که تیر و کمان دهد
\\
از عشق او چگونه کنم توبه چون دلم
&&
صد توبهٔ درست به یک پاره نان دهد
\\
آن دارد آن نگار ز عطار چون گذشت
&&
امکان ندارد آنکه کسی شرح آن دهد
\\
\end{longtable}
\end{center}
