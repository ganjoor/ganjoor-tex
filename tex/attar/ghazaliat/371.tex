\begin{center}
\section*{غزل شماره ۳۷۱: قطره گم گردان چو دریا شد پدید}
\label{sec:371}
\addcontentsline{toc}{section}{\nameref{sec:371}}
\begin{longtable}{l p{0.5cm} r}
قطره گم گردان چو دریا شد پدید
&&
خانه ویران کن چو صحرا شد پدید
\\
گم نیارد گشت در دریا دمی
&&
هر که در قطره هویدا شد پدید
\\
گر کسی در قطره بودن بازماند
&&
قطره ماند گرچه دریا شد پدید
\\
گم شو اینجا از وجود خویش پاک
&&
کان که اینجا گم شد آنجا شد پدید
\\
ناپدید امروز شو از هرچه هست
&&
کین چنین شد هر که فردا شد پدید
\\
روی‌های زشت فانی محو به
&&
خاصه دایم روی زیبا شد پدید
\\
دوشم از پیشان خطاب آمد به جان
&&
کان که پنهان گشت پیدا شد پدید
\\
ناپدید از خویش شو یکبارگی
&&
کان که از خود محو، از ما شد پدید
\\
بستهٔ پستی مباش ای مرغ عرش
&&
پر برآور هین که بالا شد پدید
\\
گم شدن فرض است هر دو کون را
&&
لا چه وزن آرد چو الا شد پدید
\\
خرد مشمر لا که از لا بود و بس
&&
کز ثری تا بر ثریا شد پدید
\\
در احد چون اسم ما یک جلوه کرد
&&
در عدد بنگر چه اسما شد پدید
\\
ترک اسما کن که هر کو ترک کرد
&&
در مسما رفت و تنها شد پدید
\\
از هزاران درد دایم باز رست
&&
تا ابد در یک تماشا شد پدید
\\
در چنین بازار چون عطار را
&&
سود وافر بود سودا شد پدید
\\
\end{longtable}
\end{center}
