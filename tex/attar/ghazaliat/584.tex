\begin{center}
\section*{غزل شماره ۵۸۴: هر شبی وقت سحر در کوی جانان می‌روم}
\label{sec:584}
\addcontentsline{toc}{section}{\nameref{sec:584}}
\begin{longtable}{l p{0.5cm} r}
هر شبی وقت سحر در کوی جانان می‌روم
&&
چون ز خود نامحرمم از خویش پنهان می‌روم
\\
چون حجابی مشکل آمد عقل و جان در راه او
&&
لاجرم در کوی او بی عقل و بی جان می‌روم
\\
همچو لیلی مستمندم در فراقش روز و شب
&&
همچو مجنون گرد عالم دوست جویان می‌روم
\\
هر سحر عنبر فشاند زلف عنبر بار او
&&
من بدان آموختم وقت سحر زان می‌روم
\\
تا بدیدم زلف چون چوگان او بر روی ماه
&&
در خم چوگان او چون گوی گردان می‌روم
\\
ماه رویا در من مسکین نگر کز عشق تو
&&
با دلی پر خون به زیر خاک حیران می‌روم
\\
ذره ذره زان شدم تا پیش خورشید رخش
&&
همچو ذره بی سر و تن پای کوبان می‌روم
\\
چون بیابانی نهد هر ساعتی در پیش من
&&
من چنین شوریده دل سر در بیابان می‌روم
\\
تا کی ای عطار از ننگ وجود تو مرا
&&
کین زمان از ننگ تو با خاک یکسان می‌روم
\\
\end{longtable}
\end{center}
