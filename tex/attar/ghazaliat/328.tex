\begin{center}
\section*{غزل شماره ۳۲۸: زلف تو که فتنهٔ جهان بود}
\label{sec:328}
\addcontentsline{toc}{section}{\nameref{sec:328}}
\begin{longtable}{l p{0.5cm} r}
زلف تو که فتنهٔ جهان بود
&&
جانم بربود و جای آن بود
\\
هر دل که زعشق تو خبر یافت
&&
صد جانش به رایگان گران بود
\\
مرده‌دل آن کسی که او را
&&
در عشق تو زندگی به جان بود
\\
گفتم دل خویش خون کنم من
&&
کز دست دلم بسی زیان بود
\\
ناگاه کشیده داشت دستم
&&
چون پای غم تو در میان بود
\\
گر من دادم امان دلم را
&&
دل را ز غم تو کی امان بود
\\
گفتم که دهان تو ببینم
&&
خود از دهنت که را نشان بود
\\
هرگز نرسید هیچ جایی
&&
آن را که غم چنان دهان بود
\\
گفتی که چگونه‌ای تو بی‌من
&&
دانی تو که بی‌تو چون توان بود
\\
ز آنروز که یک زمانت دیدم
&&
صد ساله غمم به یک زمان بود
\\
بر خاک درت نشسته عطار
&&
تا بود ز عشق جان فشان بود
\\
\end{longtable}
\end{center}
