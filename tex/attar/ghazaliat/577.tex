\begin{center}
\section*{غزل شماره ۵۷۷: محلم نیست که خورشید جمالت بینم}
\label{sec:577}
\addcontentsline{toc}{section}{\nameref{sec:577}}
\begin{longtable}{l p{0.5cm} r}
محلم نیست که خورشید جمالت بینم
&&
بو که باری اثر عکس خیالت بینم
\\
کاشکی خاک رهت سرمهٔ چشمم بودی
&&
که ندانم که دمی گرد وصالت بینم
\\
صد هزاران دل کامل شده در کوی امید
&&
خاک بوس در و درگاه جلالت بینم
\\
همچو پروانه پر و بال زنم در غم تو
&&
گر شبی پرتو آن شمع جمالت بینم
\\
جگرم خون شد از اندیشهٔ آن تا پس ازین
&&
جان و دل خون شود و من به چه حالت بینم
\\
تو مرا دم به دم اندر غم خود می‌بینی
&&
من زهی دولت اگر سال به سالت بینم
\\
خاک پای تو شدم خون دلم پاک مریز
&&
نی بخور خون دل من که حلالت بینم
\\
گر دهد شرح غمت خاطر عطار بسی
&&
نشوم هیچ ملول و نه ملالت بینم
\\
\end{longtable}
\end{center}
