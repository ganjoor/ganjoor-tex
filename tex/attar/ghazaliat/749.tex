\begin{center}
\section*{غزل شماره ۷۴۹: ای از شکنج زلفت هرجا که انقلابی}
\label{sec:749}
\addcontentsline{toc}{section}{\nameref{sec:749}}
\begin{longtable}{l p{0.5cm} r}
ای از شکنج زلفت هرجا که انقلابی
&&
هرگز نتافت بر کس چون رویت آفتابی
\\
در پیش عکس رویت شمس و قمر خیالی
&&
در جنب طاق چشمت نیل فلک سرابی
\\
بی تنگی دهانت جان مانده در مضیقی
&&
بی آتش رخ تو دل گشته چون کبابی
\\
چون چشم نیم خوابت بیدار کرد فتنه
&&
ناموس شوخ چشمان آنجا نمود خوابی
\\
آن چشمه‌ای که لعلت سیراب شد از آنجا
&&
در خلد هیچ حوری آنجا نیافت آبی
\\
من تاب می نیارم تابی ز زلف کم کن
&&
تا کی بود ز زلفت در دل فتاده تابی
\\
ای گنج آفرینش دلها خراب از تو
&&
آرام گیر با من چون گنج در خرابی
\\
در شش جهات عالم از هشت خلد خوشتر
&&
در تو نگاه کردن در نور ماهتابی
\\
خواهم که مست باشی در ماهتاب خفته
&&
من بر رخت فشانده از چشم تر گلابی
\\
گه کرده بر رخ تو از برگ گل نثاری
&&
گه خورده با لب تو از جام جم شرابی
\\
این آرزوست اکنون عطار را ز عالم
&&
این آرزوی او را هین باز ده جوابی
\\
\end{longtable}
\end{center}
