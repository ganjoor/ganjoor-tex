\begin{center}
\section*{غزل شماره ۲۳۷: بوی زلف یار آمد یارم اینک می‌رسد}
\label{sec:237}
\addcontentsline{toc}{section}{\nameref{sec:237}}
\begin{longtable}{l p{0.5cm} r}
بوی زلف یار آمد یارم اینک می‌رسد
&&
جان همی آساید و دلدارم اینک می‌رسد
\\
اولین شب صبحدم با یارم اینک می‌دمد
&&
وآخرین اندیشه و تیمارم اینک می‌رسد
\\
در کنار جویباران قامت و رخسار او
&&
سرو سیمین آن گل بی خارم اینک می‌رسد
\\
ای بسا غم کو مرا خورد و غمم کس می نخورد
&&
چون نباشم شاد چون غمخوارم اینک می‌رسد
\\
مدتی تا بودم اندر آرزوی یک نظر
&&
لاجرم چندین نظر در کارم اینک می‌رسد
\\
دین و دنیا و دل و جان و جهان و مال و ملک
&&
آنچه هست از اندک و بسیارم اینک می‌رسد
\\
روی تو ماه است و مه اندر سفر گردد مدام
&&
همچو ماه از مشرق ره یارم اینک می‌رسد
\\
بزم شادی از برای نقل سرمستان عشق
&&
پسته و عناب شکر بارم اینک می‌رسد
\\
من به استقبال او جان بر کف از بهر نثار
&&
یار می‌گوید کنون عطارم اینک می‌رسد
\\
\end{longtable}
\end{center}
