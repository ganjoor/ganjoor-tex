\begin{center}
\section*{غزل شماره ۴۴۴: هر روز که جلوه می‌کند رویش}
\label{sec:444}
\addcontentsline{toc}{section}{\nameref{sec:444}}
\begin{longtable}{l p{0.5cm} r}
هر روز که جلوه می‌کند رویش
&&
بر می‌خیزد قیامت ز کویش
\\
می‌نتوان دید روی او لیکن
&&
می‌بتوان دید روی در رویش
\\
می‌نتوان یافت سوی او راهی
&&
ای بس که برآمدم ز هر سویش
\\
تا فال گرفته‌ام جمال او
&&
چون قرعه بگشته‌ام به پهلویش
\\
در هر نفسم هزار جان باید
&&
تا صید کنند کمند گیسویش
\\
هر روز به نو خراج می‌آرند
&&
از هندستان به هندوی مویش
\\
جان بر کف دست می‌رسد هر شب
&&
از ترکستان هزار هندویش
\\
شد حلقه به گوش لؤلؤ لالا
&&
در لالایی درج لولویش
\\
خورشید که تیغ می‌زند در میغ
&&
افکند سپر ز جزع جادویش
\\
دل را به دهان شیر می‌خواند
&&
رو به بازی چشم آهویش
\\
خواهم که ببیند ابرویش رستم
&&
تا هست خود این کمان به بازویش
\\
رستم به هزار سال چون زالی
&&
بر زه نکند کمان ابرویش
\\
عطار که طاق از ابروی او شد
&&
دردی دارد که نیست دارویش
\\
\end{longtable}
\end{center}
