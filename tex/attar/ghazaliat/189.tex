\begin{center}
\section*{غزل شماره ۱۸۹: اگر درمان کنم امکان ندارد}
\label{sec:189}
\addcontentsline{toc}{section}{\nameref{sec:189}}
\begin{longtable}{l p{0.5cm} r}
اگر درمان کنم امکان ندارد
&&
که درد عشق تو درمان ندارد
\\
ز بحر عشق تو موجی نخیزد
&&
که در هر قطره صد طوفان ندارد
\\
غمت را پاک‌بازی می‌بباید
&&
که صد جان بخشد و یک جان ندارد
\\
به حسن رای خویش اندیشه کردم
&&
به حسن روی تو امکان ندارد
\\
فروگیرد جهان خورشید رویت
&&
اگر زلف تواش پنهان ندارد
\\
فلک گر صوفییی پیروزه‌پوش است
&&
ولی این هست او را کان ندارد
\\
اگرچه در جهان خورشید رویش
&&
به زیبایی خود تاوان ندارد
\\
چو نتواند که چون روی تو باشد
&&
بگو تا خویش سرگردان ندارد
\\
چو طوطی خط تو بر دهانت
&&
کسی بر نقطه صد برهان ندارد
\\
سر زلف تو چون گیرم که بی تو
&&
غمم چون زلف تو پایان ندارد
\\
لبت خونم چرا ریزد به دندان
&&
اگر بر من به خون دندان ندارد
\\
فرید امروز خوش خوان‌تر ز خطت
&&
خطی سرسبز در دیوان ندارد
\\
\end{longtable}
\end{center}
