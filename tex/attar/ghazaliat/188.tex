\begin{center}
\section*{غزل شماره ۱۸۸: دلی کز عشق جانان جان ندارد}
\label{sec:188}
\addcontentsline{toc}{section}{\nameref{sec:188}}
\begin{longtable}{l p{0.5cm} r}
دلی کز عشق جانان جان ندارد
&&
توان گفتن که او ایمان ندارد
\\
درین میدان که یارد گشت یکدم
&&
که کس مردی یک جولان ندارد
\\
شگرفی باید از گنج دو عالم
&&
که جان یک لحظه بی‌جانان ندارد
\\
به آسانی منه در کوی او پای
&&
که رهرو راه را آسان ندارد
\\
چه عشق است این که خود نقصان نگیرد
&&
چه درد است این که خود درمان ندارد
\\
دلم در درد عشق او چنان است
&&
که دل بی درد عشقش جان ندارد
\\
مرو در راه او گر ناتوانی
&&
که دور است این ره و پایان ندارد
\\
اگر قوت نداری دور ازین راه
&&
که کوی عاشقان پیشان ندارد
\\
برو عطار دم درکش که جانان
&&
همه عمرت چنین حیران ندارد
\\
\end{longtable}
\end{center}
