\begin{center}
\section*{غزل شماره ۳۹۰: ای عشق تو کیمیای اسرار}
\label{sec:390}
\addcontentsline{toc}{section}{\nameref{sec:390}}
\begin{longtable}{l p{0.5cm} r}
ای عشق تو کیمیای اسرار
&&
سیمرغ هوای تو جگرخوار
\\
سودای تو بحر آتشین موج
&&
اندوه تو ابر تند خون‌بار
\\
در پرتو آفتاب رویت
&&
خورشید سپهر ذره کردار
\\
یک موی ز زلف کافر تو
&&
غارتگر صد هزار دین‌دار
\\
چون زلف به ناز برفشانی
&&
صد خرقه بدل شود به زنار
\\
آنجا که سخن رود ز زلفت
&&
چه کفر و چه دین چه تخت و چه دار
\\
تا بنشستی به دلربایی
&&
برخاست قیامتی به یکبار
\\
آن شد که ز وصل تو زدم لاف
&&
اکنون من و پشت دست و دیوار
\\
در عشق تو کار خویش هر روز
&&
از سر گیرم زهی سر و کار
\\
دستی بر نه که دور از تو
&&
چون باد ز دست رفت عطار
\\
\end{longtable}
\end{center}
