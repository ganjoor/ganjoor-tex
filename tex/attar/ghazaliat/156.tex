\begin{center}
\section*{غزل شماره ۱۵۶: هرچه دارم در میان خواهم نهاد}
\label{sec:156}
\addcontentsline{toc}{section}{\nameref{sec:156}}
\begin{longtable}{l p{0.5cm} r}
هرچه دارم در میان خواهم نهاد
&&
بی خبر سر در جهان خواهم نهاد
\\
آب حیوان چون به تاریکی در است
&&
جام جم در جنب جان خواهم نهاد
\\
زین همت در ره سودای عشق
&&
بر براق لامکان خواهم نهاد
\\
گر بجنبد کاروان عاشقان
&&
پای پیش کاروان خواهم نهاد
\\
جان چو صبحی بر جهان خواهم فشاند
&&
سر چو شمعی در میان خواهم نهاد
\\
سود ممکن نیست در بازار عشق
&&
پس اساسی بر زیان خواهم نهاد
\\
گر قدم از خویش برخواهم گرفت
&&
از زمین بر آسمان خواهم نهاد
\\
مرغ عرشم سیر گشتم از قفس
&&
روی سوی آشیان خواهم نهاد
\\
تا نیاید سر جانم بر زبان
&&
مهر مطلق بر زبان خواهم نهاد
\\
زهر خواهد شد ز عیش تلخ من
&&
صد شکر گر در دهان خواهم نهاد
\\
آستین پر خون به امید وصال
&&
سر بسی بر آستان خواهم نهاد
\\
دست چون می نرسدم در زلف دوست
&&
سر به زیر پای از آن خواهم نهاد
\\
در زبان گوهرافشان فرید
&&
طرفه گنجی جاودان خواهم نهاد
\\
\end{longtable}
\end{center}
