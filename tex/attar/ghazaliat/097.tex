\begin{center}
\section*{غزل شماره ۹۷: شادی به روزگار شناسندگان مست}
\label{sec:097}
\addcontentsline{toc}{section}{\nameref{sec:097}}
\begin{longtable}{l p{0.5cm} r}
شادی به روزگار شناسندگان مست
&&
جانها فدای مرتبهٔ نیستان هست
\\
از ناز برکشیده کله گوشهٔ بلی
&&
در گوش کرده حلقه معشوقهٔ الست
\\
گاهی ز فخر تاج سر عالمی بلند
&&
گاهی ز فقر خاک ره این جهان پست
\\
دستار عقلشان کف طرار عشق برد
&&
بازار توبه‌شان شکن زلف لا شکست
\\
برخاستند از سر اسرار هر دو کون
&&
چون شاه عشق در دل ایشان فرو نشست
\\
زنجیر در میان و نمد دربرند از آنک
&&
مردی که راه فقر به سر برد حیدر است
\\
آنجا که پای جای ندارد فشرده پای
&&
وانجا که دست جای ندارد فشانده دست
\\
در قعر بحر نور فرو خورده غوطها
&&
وز شوق ذوق ملک عدم نیستی به هست
\\
عطار جام دولت ایشان به کف گرفت
&&
جاوید از آن شراب معطر بماند مست
\\
\end{longtable}
\end{center}
