\begin{center}
\section*{غزل شماره ۵۰: ندانم تا چه کارم اوفتادست}
\label{sec:050}
\addcontentsline{toc}{section}{\nameref{sec:050}}
\begin{longtable}{l p{0.5cm} r}
ندانم تا چه کارم اوفتادست
&&
که جانی بی قرارم اوفتادست
\\
چنان کاری که آن کس را نیفتاد
&&
به یک ساعت هزارم اوفتادست
\\
همان آتش که در حلاج افتاد
&&
همان در روزگارم اوفتادست
\\
دلم را اختیاری می‌نبینم
&&
خلل در اختیارم اوفتادست
\\
مگر با حلقه‌های زلف معشوق
&&
شماری بی‌شمارم اوفتادست
\\
مگر در عشق او نادیده رویش
&&
دلی پر انتظارم اوفتادست
\\
شبی بوی می او ناشنوده
&&
نصیب از وی خمارم اوفتادست
\\
هزاران شب چو شمعی غرقه در اشک
&&
سر خود در کنارم اوفتادست
\\
هزاران روز بس تنها و بی کس
&&
مصیبت‌های زارم اوفتادست
\\
اگر تر دامن افتادم عجب نیست
&&
که چشمی اشکبارم اوفتادست
\\
کجا مردی است در عالم که او را
&&
نظر بر کار و بارم اوفتادست
\\
نیفتاد آنچه از عطار افتاد
&&
که تا او هست کارم اوفتادست
\\
\end{longtable}
\end{center}
