\begin{center}
\section*{غزل شماره ۷۴۵: گر کسی یابد درین کو خانه‌ای}
\label{sec:745}
\addcontentsline{toc}{section}{\nameref{sec:745}}
\begin{longtable}{l p{0.5cm} r}
گر کسی یابد درین کو خانه‌ای
&&
هر دمش واجب بود شکرانه‌ای
\\
هر که او بویی ندارد زین حدیث
&&
هر بن مویش بود بتخانه‌ای
\\
هر که در عقل لجوج خویش ماند
&&
زین سخن خواند مرا دیوانه‌ای
\\
هر که اینجا آشنای او نشد
&&
باز ماند تا ابد بیگانه‌ای
\\
گر چنین خوابت نبردی از غرور
&&
این سخن نشنودیی افسانه‌ای
\\
زن‌صفت را نیست با این راز کار
&&
پر دلی می‌باید و مردانه‌ای
\\
مرغ این اسرار را در حوصله
&&
از دو عالم می‌بباید دانه‌ای
\\
گر ازین مویی چو شانه ره بری
&&
شاخ شاخ آید دلت چون شانه‌ای
\\
گر برانند از دو عالم باک نیست
&&
هست زین هر دو برون ویرانه‌ای
\\
زان شرابی کان شراب عاشقانست
&&
نیست در هر دو جهان پیمانه‌ای
\\
گر جهان آتش بگیرد پیش و پس
&&
نیستم آخر کم از پروانه‌ای
\\
خویش بر آتش زنم پروانه‌وار
&&
یا بسوزم یا شوم فرزانه‌ای
\\
شمع جمعم من که هر دم غیب پاک
&&
می‌دهد عطار را پروانه‌ای
\\
\end{longtable}
\end{center}
