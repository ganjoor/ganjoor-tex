\begin{center}
\section*{غزل شماره ۳۴: راه عشق او که اکسیر بلاست}
\label{sec:034}
\addcontentsline{toc}{section}{\nameref{sec:034}}
\begin{longtable}{l p{0.5cm} r}
راه عشق او که اکسیر بلاست
&&
محو در محو و فنا اندر فناست
\\
فانی مطلق شود از خویشتن
&&
هر دلی که کو طالب این کیمیاست
\\
گر بقا خواهی فنا شو کز فنا
&&
کمترین چیزی که می‌زاید بقاست
\\
گم شود در نقطهٔ فای فنا
&&
هر چه در هر دو جهان شد از تو راست
\\
در چنین دریا که عالم ذره‌ای است
&&
ذره‌ای هست آمدن یارا کراست
\\
گر ازین دریا بگیری قطره‌ای
&&
زیر او پوشیده صد دریا بلاست
\\
برنیاری جان و ایمان گم کنی
&&
گر درین دریا بری یک ذره خواست
\\
گرد این دریا مگرد و لب بدوز
&&
کین نه کار ما و نه کار شماست
\\
گر گدایی را رسد بویی ازین
&&
تا ابد بر هرچه باشد پادشاست
\\
از خودی خود قدم برگیر زود
&&
تا ز پیشان بانگت آید کان ماست
\\
دم نیارد زد ازین سیر شگرف
&&
هر که را یک‌دم سر این ماجراست
\\
زهد و علم و زیرکی بسیار هست
&&
آن نمی‌خواهند درویشی جداست
\\
آنچه من گفتم زبور پارسی است
&&
فهم آن نه کار مرد پارساست
\\
سلطنت باید که گردد آشکار
&&
تا بدانی تو که این معنی کجاست
\\
در دل عشاق از تعظیم او
&&
کبریایی خالی از کبر و ریاست
\\
محو کن عطار را زین جایگاه
&&
کین نه کسب اوست بل عین عطاست
\\
\end{longtable}
\end{center}
