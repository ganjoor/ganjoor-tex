\begin{center}
\section*{غزل شماره ۶۹۴: گر چنین سنگدل بمانی تو}
\label{sec:694}
\addcontentsline{toc}{section}{\nameref{sec:694}}
\begin{longtable}{l p{0.5cm} r}
گر چنین سنگدل بمانی تو
&&
وه که بس خون‌ها برانی تو
\\
چه بلایی بر اهل روی زمین
&&
از بلاهای آسمانی تو
\\
از تو صد فتنه در جهان افتاد
&&
فتنهٔ جملهٔ جهانی تو
\\
فتنه برخیزد آن زمان که سحر
&&
فرق مشکین فرو فشانی تو
\\
دهن عقل باز ماند باز
&&
چون درآیی به خوش زبانی تو
\\
همه اهل زمانه دل بنهند
&&
بر امیدی که دلستانی تو
\\
خط نویسی به خون ما چو قلم
&&
سرکشان را به سر دوانی تو
\\
سرگرانی و سرکشی چه کنی
&&
که سبک روح‌تر از آنی تو
\\
باده ناخورده از من بیدل
&&
با من آخر چه سر گرانی تو
\\
چشم من ظاهرت همی بیند
&&
گرچه از چشم بد نهانی تو
\\
اگر از من کنار خواهی کرد
&&
روز و شب در میان جانی تو
\\
گلی از گلستانت باز کنم
&&
که به رخ همچو گلستانی تو
\\
شکری از لب تو بربایم
&&
که به لب چون شکرستانی تو
\\
خون فشانند عاشقان بر خاک
&&
چون ز یاقوت درفشانی تو
\\
چند آخر به خون نویسی خط
&&
هیچ خط نیز می ندانی تو
\\
دل عطار در غمت ریش است
&&
مرهمی کن اگر توانی تو
\\
\end{longtable}
\end{center}
