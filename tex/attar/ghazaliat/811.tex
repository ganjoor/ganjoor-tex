\begin{center}
\section*{غزل شماره ۸۱۱: ای هجر تو وصل جاودانی}
\label{sec:811}
\addcontentsline{toc}{section}{\nameref{sec:811}}
\begin{longtable}{l p{0.5cm} r}
ای هجر تو وصل جاودانی
&&
واندوه تو عین شادمانی
\\
در عشق تو نیم ذره حسرت
&&
خوشتر ز حیات جاودانی
\\
بی یاد حضور تو زمانی
&&
کفر است حدیث زندگانی
\\
صد جان و هزار جان نثارت
&&
آن لحظه که از درم برانی
\\
کار دو جهان من برآید
&&
گر یک نفسم به خویش خوانی
\\
با خوندان و راندنم چه کار است
&&
خواه این کن و خواه آن تو دانی
\\
گر قهر کنی سزای آنم
&&
ور لطف کنی برای آنی
\\
صد دل باید به هر زمانم
&&
تا تو ببری به دلستانی
\\
گر بر فکنی نقاب از روی
&&
جبریل سزد به جان‌فشانی
\\
کس نتواند جمال تو دید
&&
زیرا که ز دیده بس نهانی
\\
نی نی که به جز تو کس نبیند
&&
چون جمله تویی بدین عیانی
\\
در عشق تو گر بمرد عطار
&&
شد زندهٔ دایم از معانی
\\
\end{longtable}
\end{center}
