\begin{center}
\section*{غزل شماره ۵۳: دلی کز عشق جانان دردمند است}
\label{sec:053}
\addcontentsline{toc}{section}{\nameref{sec:053}}
\begin{longtable}{l p{0.5cm} r}
دلی کز عشق جانان دردمند است
&&
همو داند که قدر عشق چند است
\\
دلا گر عاشقی از عشق بگذر
&&
که تا مشغول عشقی عشق بند است
\\
وگر در عشق از عشقت خبر نیست
&&
تو را این عشق عشقی سودمند است
\\
هر آن مستی که بشناسد سر از پای
&&
ازو دعوی مستی ناپسند است
\\
ز شاخ عشق برخوردار گردی
&&
اگر عشق از بن و بیخت بکند است
\\
سرافرازی مجوی و پست شو پست
&&
که تاج پاک‌بازان تخته بند است
\\
چو تو در غایت پستی فتادی
&&
ز پستی در گذر کارت بلند است
\\
بخند ای زاهد خشک ارنه ای سنگ
&&
چه وقت گریه و چه جای پند است
\\
نگارا روز روز ماست امروز
&&
که در کف باده و در کام قند است
\\
می و معشوق و وصل جاودان هست
&&
کنون تدبیر ما لختی سپند است
\\
یقین می‌دان که اینجا مذهب عشق
&&
ورای مذهب هفتاد و اند است
\\
خرابی دیده‌ای در هیچ گلخن
&&
که خود را از خرابات اوفگند است
\\
مرا نزدیک او بر خاک بنشان
&&
که میل من به مشتی مستمند است
\\
مرا با عاشقان مست بنشان
&&
چه جای زاهدان پر گزند است
\\
بیا گو یک نفس در حلقهٔ ما
&&
کسی کز عشق در حلقش کمند است
\\
حریفی نیست ای عطار امروز
&&
وگر هست از وجود خود نژند است
\\
\end{longtable}
\end{center}
