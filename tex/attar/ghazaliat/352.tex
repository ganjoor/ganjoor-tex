\begin{center}
\section*{غزل شماره ۳۵۲: عشق تو به جان دریغم آید}
\label{sec:352}
\addcontentsline{toc}{section}{\nameref{sec:352}}
\begin{longtable}{l p{0.5cm} r}
عشق تو به جان دریغم آید
&&
نامت به زبان دریغم آید
\\
وصف سر زلف پر طلسمت
&&
از شرح و بیان دریغم آید
\\
از زلف تو سرکشان ره را
&&
یک موی نشان دریغم آید
\\
من موی‌میان نگویمت زانک
&&
این وصف بدان دریغم آید
\\
هر چند میان تو چو مویی است
&&
مویی به میان دریغم آید
\\
دل می‌خواهی و من نیم آنک
&&
هرگز ز تو جان دریغم آید
\\
یک ذره خیال چهرهٔ تو
&&
از هر دو جهان دریغم آید
\\
نی نی که ز رخ نقاب بردار
&&
کان روی نهان دریغم آید
\\
عطار چون از تو شد سبک دل
&&
در بند گران دریغم آید
\\
\end{longtable}
\end{center}
