\begin{center}
\section*{غزل شماره ۴۲۱: در عشق تو من توام تو من باش}
\label{sec:421}
\addcontentsline{toc}{section}{\nameref{sec:421}}
\begin{longtable}{l p{0.5cm} r}
در عشق تو من توام تو من باش
&&
یک پیرهن است گو دو تن باش
\\
چون یک تن را هزار جان هست
&&
گو یک جان را هزار تن باش
\\
نی نی که نه یک تن و نه یک جانست
&&
هیچند همه تو خویشتن باش
\\
چون جمله یکی است در حقیقت
&&
گو یک تن را دو پیرهن باش
\\
جانا همه آن تو شدم من
&&
من آن توام تو آن من باش
\\
ای دل به میان این سخن در
&&
مانندهٔ مرده در کفن باش
\\
چون سوسن ده زبان درین سر
&&
می‌دار زبان و بی سخن باش
\\
یک رمز مگوی لیک چون گل
&&
می‌خند خوش و همه دهن باش
\\
گر گویندت که کافری چیست
&&
گو عاشق زلف پر شکن باش
\\
ور پرسندت که چیست ایمان
&&
گو روی ببین و نعره‌زن باش
\\
گر روی بدین حدیث داری
&&
چون ابراهیم بت‌شکن باش
\\
ور گویندت ببایدت سوخت
&&
تو خود ز برای سوختن باش
\\
ور کشتن تو دهند فتوی
&&
در کشتن خود به تاختن باش
\\
مانند حسین بر سر دار
&&
در کشتن و سوختن حسن باش
\\
انگشت‌زن فنای خود شو
&&
وانگشت نمای مرد و زن باش
\\
گه ماده و گاه نر چه باشی
&&
گر مرغی ویی نه چون زغن باش
\\
انجام ره تو گفت عطار
&&
رسوای هزار انجمن باش
\\
\end{longtable}
\end{center}
