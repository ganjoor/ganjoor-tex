\begin{center}
\section*{غزل شماره ۲۰۰: بس نظر تیز که تقدیر کرد}
\label{sec:200}
\addcontentsline{toc}{section}{\nameref{sec:200}}
\begin{longtable}{l p{0.5cm} r}
بس نظر تیز که تقدیر کرد
&&
تا رخ زیبای تو تصویر کرد
\\
روی تو عقلم صدف عشق ساخت
&&
چشم تو جانم هدف تیر کرد
\\
نرگس جادوت دل از من ربود
&&
گفت که این جادوی کشمیر کرد
\\
جادوی کشمیر نیارد همی
&&
پیش تو یک مسئله تقریر کرد
\\
زلف تو باز این دل دیوانه را
&&
حلقه درافکند و به زنجیر کرد
\\
هر که سر زلف تو در خواب دید
&&
کافریش عشق تو تعبیر کرد
\\
با سر زلف تو همه هیچ بود
&&
هرچه دلم حیله و تدبیر کرد
\\
کفر از آن خاست که در کاینات
&&
کوکبهٔ زلف تو تأثثیر کرد
\\
زلف تو اسلام برافکنده بود
&&
لیک نکو کرد که تاخیر کرد
\\
مرغ دلم تا که زبون تو شد
&&
قصد بدو عشق زبون گیر کرد
\\
در ره عشق تو دلم جان بداد
&&
تا جگر سوخته توفیر کرد
\\
نالهٔ شبگیر من از حد گذشت
&&
چند توان نالهٔ شبگیر کرد
\\
کس بنداند که دل عاشقم
&&
در ره عشق تو چه تقصیر کرد
\\
لاجرم اکنون چو به دام اوفتاد
&&
دانهٔ جان در سر تشویر کرد
\\
بر دل عطار ببخشای از آنک
&&
روز جوانیش غمت پیر کرد
\\
\end{longtable}
\end{center}
