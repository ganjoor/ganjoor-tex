\begin{center}
\section*{غزل شماره ۱۸۴: هر که بر روی او نظر دارد}
\label{sec:184}
\addcontentsline{toc}{section}{\nameref{sec:184}}
\begin{longtable}{l p{0.5cm} r}
هر که بر روی او نظر دارد
&&
از بسی نیکوی خبر دارد
\\
تو نکوتر ز نیکوان دو کون
&&
که دو کون از تو یک اثر دارد
\\
هرچه اندر دو کون می‌بینم
&&
از جمال تو یک نظر دارد
\\
در جمالت مدام بیخبر است
&&
هر که او ذره‌ای بصر دارد
\\
دیده‌جان که در تو حیران است
&&
هرچه جز توست مختصر دارد
\\
هر که روی چو آفتاب تو دید
&&
نتواند که دیده بردارد
\\
هر که بویی بیافت از ره تو
&&
خاک راه تو تاج سر دارد
\\
عاشق از خویشتن نیندیشد
&&
گرچه راهت بسی خطر دارد
\\
خویش را مست وار درفکند
&&
هر که او جان دیده‌ور دارد
\\
در ره عشق تو دل عطار
&&
آتشی سخت در جگر دارد
\\
\end{longtable}
\end{center}
