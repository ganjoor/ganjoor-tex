\begin{center}
\section*{غزل شماره ۱۹۷: از کمان ابروش چون تیر مژگان بگذرد}
\label{sec:197}
\addcontentsline{toc}{section}{\nameref{sec:197}}
\begin{longtable}{l p{0.5cm} r}
از کمان ابروش چون تیر مژگان بگذرد
&&
بر دل آید چون ز دل بگذشت از جان بگذرد
\\
راست اندازی چشمش بین که گر خواهد به حکم
&&
ناوک مژگان او بر موی مژگان بگذرد
\\
باد وقتی آب را همچون زره داند نمود
&&
کز نخست آید بر آن زلف زره‌سان بگذرد
\\
در زمان آزاد گردد سرو از بالای خویش
&&
گر به پیش قد آن سرو خرامان بگذرد
\\
ماه‌رویا آفتاب از شرم تو پنهان شود
&&
گر ز رویت سایه بر خورشید رخشان بگذرد
\\
با توام خون نیزه گردان نیست، دور از روی تو
&&
نیزه بالا خون ز بالای سرم زان بگذرد
\\
تو ز آه من چو گردون فارغ و از هجر تو
&&
آه خون آلودم از گردون گردان بگذرد
\\
در دل عطار از عشقت چنان آتش فتاد
&&
کز تف او آتش از بالای کیوان بگذرد
\\
\end{longtable}
\end{center}
