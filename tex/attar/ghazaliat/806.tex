\begin{center}
\section*{غزل شماره ۸۰۶: ای ساقی از آن قدح که دانی}
\label{sec:806}
\addcontentsline{toc}{section}{\nameref{sec:806}}
\begin{longtable}{l p{0.5cm} r}
ای ساقی از آن قدح که دانی
&&
پیش آر سبک مکن گرانی
\\
یک قطره شراب در صبوحی
&&
باشد که به حلق ما چکانی
\\
زان پیش خمار در سر آید
&&
یک باده به دست ما رسانی
\\
بگذر تو ز خویش و از قرابات
&&
پیش آر قرابهٔ مغانی
\\
در عقل مغیش تا نبینی
&&
وز علم مجوس تا نخوانی
\\
کین جای نه جای قیل و قال است
&&
کافسانه کنی و قصه خوانی
\\
این جای مقام کم زنان است
&&
تو مرد ردا و طیلسانی
\\
ساقی تو بیا و بر کفم نه
&&
یک کوزهٔ آب زندگانی
\\
یک قطرهٔ درد اگر بنوشی
&&
یابی تو حیات جاودانی
\\
ساقی شو و راوقی در انداز
&&
زان لعل چو در که می‌چکانی
\\
عطار بیا ز پرده بیرون
&&
تا چند سخن ز پرده رانی
\\
\end{longtable}
\end{center}
