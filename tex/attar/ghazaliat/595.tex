\begin{center}
\section*{غزل شماره ۵۹۵: هرچه همه عمر همی ساختیم}
\label{sec:595}
\addcontentsline{toc}{section}{\nameref{sec:595}}
\begin{longtable}{l p{0.5cm} r}
هرچه همه عمر همی ساختیم
&&
در ره ترسابچه درباختیم
\\
راهب دیرش چو سپه عرضه داد
&&
صد علم عشق برافراختیم
\\
رقص‌کنان بر سر میدان شدیم
&&
نعره‌زنان بر دو جهان تاختیم
\\
ترک فلک غاشیهٔ ما کشد
&&
زانکه نه با اسب و نه با ساختیم
\\
عشق رخش چون به سر ما رسید
&&
سر به دل خرقه برانداختیم
\\
سینه به شکرانهٔ او سوختیم
&&
قبله ز بتخانهٔ او ساختیم
\\
گرچه فشاندیم بر او دین و دل
&&
قیمت ترسابچه نشناختیم
\\
درد ده ای ساقی مجلس که ما
&&
پردهٔ درد است که بنواختیم
\\
نه که نه ما بابت درد توییم
&&
زانکه ز درد تو بنگداختیم
\\
با تو که پردازد اگر راستی است
&&
چون همه از خویش نپرداختیم
\\
جز سخنی بهرهٔ عطار نیست
&&
زان به سخن تیغ زبان آختیم
\\
\end{longtable}
\end{center}
