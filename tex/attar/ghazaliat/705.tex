\begin{center}
\section*{غزل شماره ۷۰۵: دوش درآمد ز درم صبحگاه}
\label{sec:705}
\addcontentsline{toc}{section}{\nameref{sec:705}}
\begin{longtable}{l p{0.5cm} r}
دوش درآمد ز درم صبحگاه
&&
حلقهٔ زلفش زده صف گرد ماه
\\
زلف پریشانش شکن کرده باز
&&
کرده پریشان شکنش صد سپاه
\\
از سر زلفش به دل عاشقان
&&
مژده رسان باد صبا صبحگاه
\\
مست برم آمد و دردیم داد
&&
تا دلم از درد برآورد آه
\\
گفت رخم بین که گر از عشق من
&&
توبه کنی توبه بتر از گناه
\\
گفتمش ای جان چکنم تا مرا
&&
زین می نوشین بدهی گاه گاه
\\
گفت ز خود فانی مطلق بباش
&&
تا برسی زود بدین دستگاه
\\
گر بخورندت به مترس از وجود
&&
گرچه بگردی تو نگردی تباه
\\
آهو چینی چو گیاهی خورد
&&
در شکمش مشک شود آن گیاه
\\
مات شو ار شاه همه عالمی
&&
تا برهی از ضرر آب و جاه
\\
از شدن و آمدن و از گریز
&&
کی برهد تا نشود مات شاه
\\
گفتمش از علم مرا کوه‌هاست
&&
کس نتواند که کند کوه کاه
\\
گفت که هرچیز که دانسته‌ای
&&
جمله فرو شوی به آب سیاه
\\
چون همه چیزیت فراموش شد
&&
بر دل و جانت بگشایند راه
\\
یوسف قدسی تو و ملک تو مصر
&&
جهد بر آن کن که برآیی ز چاه
\\
تا سر عطار نگردد چو گوی
&&
از مه و خورشید نیابد کلاه
\\
هرکه درین واقعه آزاد نیست
&&
گو برو و خرقه ز عطار خواه
\\
\end{longtable}
\end{center}
