\begin{center}
\section*{غزل شماره ۵۰۳: تو می‌دانی که در کار تو چون مضطر فرو ماندم}
\label{sec:503}
\addcontentsline{toc}{section}{\nameref{sec:503}}
\begin{longtable}{l p{0.5cm} r}
تو می‌دانی که در کار تو چون مضطر فرو ماندم
&&
به خاک و خون فرو رفتم ز خواب و خور فرو ماندم
\\
ز حیرانی عشق تو خلاصم کی بود هرگز
&&
که از عشقت به نو هر روز حیران تر فرو ماندم
\\
عجایب نامهٔ عشقت به پایان چون برم آخر
&&
که اندر اولین حرفی به سر دفتر فرو ماندم
\\
چو دست من به یک بازی فرو بستی چه بازم من
&&
مکن داویم ده آخر که در ششدر فرو ماندم
\\
همه شب بی تو چون شمعی میان آتش و آبم
&&
نگه کن در من مسکین که بس مضطر فرو ماندم
\\
چگونه چشمهٔ حیوان درین وادی به دست آرم
&&
که اندر قعر تاریکی چو اسکندر فرو ماندم
\\
از آن شد کشتیم غرقاب و من با پاره‌ای تخته
&&
که در گرداب این دریای موج‌آور فرو ماندم
\\
چو از شوق گهر رفتم بدین دریا و گم گشتم
&&
هم از خشکی هم از دریا هم از گوهر فرو ماندم
\\
ز بس کاندر خم چوگان محنت گوی گشتم من
&&
چو گویی اندرین میدان ز پای و سر فرو ماندم
\\
ندانم تا تو ای عطار گنج عشق کی یابی
&&
که از سودای گنج ایدر به رنج اندر فرو ماندم
\\
\end{longtable}
\end{center}
