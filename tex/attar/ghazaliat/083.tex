\begin{center}
\section*{غزل شماره ۸۳: عشق تو قلاوز جهان است}
\label{sec:083}
\addcontentsline{toc}{section}{\nameref{sec:083}}
\begin{longtable}{l p{0.5cm} r}
عشق تو قلاوز جهان است
&&
سودای تو رهنمای جان است
\\
وصل تو خلاصهٔ وجود است
&&
درد تو دریچهٔ عیان است
\\
هاروت تو چاره ساز سحر است
&&
یاقوت تو مایه‌بخش جان است
\\
کس را ز دهان تو سخن نیست
&&
زان روی که نقطه گمان است
\\
تا بر دهنت نهاده‌ام دل
&&
این تنگ‌دلی من از آن است
\\
لعلت شکری است تنگ بر تنگ
&&
یعنی دل من بر آن دهان است
\\
کس بر کمرت میان ندیدست
&&
گرچه کمر تو را میان است
\\
تا ابروی چون کمانت دیدم
&&
صد گونه ز هم از آن کمان است
\\
چون ابروی توست چون کمانی
&&
چندین ز هم از چه در زبان است
\\
دندان تو مغز پستهٔ توست
&&
مغزی دیدی که استخوان است
\\
گفتی که دلت بسوز در عشق
&&
یعنی که سپند عاشقان است
\\
از دست تو دل چگونه سوزم
&&
چون پای غم تو در میان است
\\
یک ذره غم تو خوشتر آید
&&
از هر شادی که در جهان است
\\
آن درد که در دل من از توست
&&
هر وصف که گویمش نه آن است
\\
در روی من شکسته دل خند
&&
گر موجب خنده زعفران است
\\
در کار عقوبت تو عطار
&&
چون ممتحنی در امتحان است
\\
\end{longtable}
\end{center}
