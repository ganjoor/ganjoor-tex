\begin{center}
\section*{غزل شماره ۱۲۸: تا دل من راه جانان بازیافت}
\label{sec:128}
\addcontentsline{toc}{section}{\nameref{sec:128}}
\begin{longtable}{l p{0.5cm} r}
تا دل من راه جانان بازیافت
&&
گوهری در پردهٔ جان بازیافت
\\
دل که ره می‌جست در وادی عشق
&&
خویش را گم کرد ره زان بازیافت
\\
هر که از دشورای هستی برست
&&
آنچه مقصود است آسان بازیافت
\\
یک شبی درتاخت دل مست و خراب
&&
راه آن زلف پریشان بازیافت
\\
چون به تاریکی زلفش راه برد
&&
زنده گشت و آب حیوان بازیافت
\\
آفتاب هر دو عالم آشکار
&&
زیر زلف دوست پنهان بازیافت
\\
آنچه خلق از دامن آفاق جست
&&
او نهان سر در گریبان بازیافت
\\
می‌ندانم تا ز جان برخورد نیز
&&
آنکه روی و زلف جانان بازیافت
\\
هر که زلفش دید کافر شد به حکم
&&
وانکه درویش دید ایمان بازیافت
\\
طالب درد است عطار این زمان
&&
کز میان درد درمان بازیافت
\\
\end{longtable}
\end{center}
