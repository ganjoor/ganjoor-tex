\begin{center}
\section*{غزل شماره ۲۱۴: چو طوطی خط او پر بر آورد}
\label{sec:214}
\addcontentsline{toc}{section}{\nameref{sec:214}}
\begin{longtable}{l p{0.5cm} r}
چو طوطی خط او پر بر آورد
&&
جهان حسن در زیر پر آورد
\\
به خوش رنگی رخش عالم برافروخت
&&
ز سرسبزی خطش رنگی بر آورد
\\
لب چون لعلش از چشمم گهر ریخت
&&
بر چون سیمش از رویم زر آورد
\\
گل از شرم رخ او خشک لب گشت
&&
ز خشکی ای عجب دامن تر آورد
\\
دهان تنگ او یارب چه چشمه است
&&
که از خنده به دریا گوهر آورد
\\
سر زلفش شکار دلبری را
&&
هزاران حلقه در یکدیگر آورد
\\
فلک زان چنبری آمد که زلفش
&&
فلک را نیز سر در چنبر آورد
\\
فلک در پای او چون گوی می‌گشت
&&
چو چوگانش به خدمت بر سر آورد
\\
چو شد عطار لالای در او
&&
ز زلفش خادمی را عنبر آورد
\\
\end{longtable}
\end{center}
