\begin{center}
\section*{غزل شماره ۷۷: تا در تو خیال خاص و عام است}
\label{sec:077}
\addcontentsline{toc}{section}{\nameref{sec:077}}
\begin{longtable}{l p{0.5cm} r}
تا در تو خیال خاص و عام است
&&
از عشق نفس زدن حرام است
\\
تا هیچ و همه یکی نگردد
&&
دعوی یگانگیت عام است
\\
تا پاک نگردی از وجودت
&&
هر پختگیی که هست خام است
\\
چون اصل همه به قطع هیچ است
&&
این از همه، هیچ ناتمام است
\\
تو اصل طلب ز فرع بگذر
&&
کین یک گذرنده و آن مدام است
\\
چون او همه را ندید می‌گفت
&&
اکنون جز ازین همه کدام است
\\
هر مرد که مرد هیچ آمد
&&
او را همه چیز یک مقام است
\\
تا تو به وجود مانده‌ای باز
&&
در گردن تو هزار دام است
\\
کانجا که وجود دم به دم نیست
&&
اصلت عدم علی‌الدوام است
\\
شرمت نامد از آن وجودی
&&
کان را به نفس نفس قیام است
\\
بگذر ز وجود و با عدم ساز
&&
زیرا که عدم، عدم به نام است
\\
می‌دان به یقین که با عدم خاست
&&
هرجا که وجود را نظام است
\\
آری چو عدم وجود بخش است
&&
موجوداتش به جان غلام است
\\
چون فقر عدم برای خاص است
&&
کفر است کزو نصیب عام است
\\
گر تو سر هیچ هیچ داری
&&
در هر گامت هزار کام است
\\
وامانده به ذره‌ای تو کم باز
&&
هرگز نه تو را جم و نه جام است
\\
عطار ز هیچ هیچ دل یافت
&&
آن دل که برون دال و لام است
\\
\end{longtable}
\end{center}
