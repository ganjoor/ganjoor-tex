\begin{center}
\section*{غزل شماره ۵۹۶: بس که جان در خاک این در سوختیم}
\label{sec:596}
\addcontentsline{toc}{section}{\nameref{sec:596}}
\begin{longtable}{l p{0.5cm} r}
بس که جان در خاک این در سوختیم
&&
دل چو خون کردیم و در بر سوختیم
\\
در رهش با نیک و بد در ساختیم
&&
در غمش هم خشک و هم تر سوختیم
\\
سوز ما با عشق او قوت نداشت
&&
گرچه ما هر دم قوی‌تر سوختیم
\\
چون بدو ره نی و بی او صبر نی
&&
مضطرب گشتیم و مضطر سوختیم
\\
چون ز جانان آتشی در جان فتاد
&&
جان خود چون عود مجمر سوختیم
\\
چون ز دلبر طعم شکر یافتیم
&&
دل چو عود از طعم شکر سوختیم
\\
چون دل و جان پردهٔ این راه بود
&&
جان ز جانان دل ز دلبر سوختیم
\\
مدت سی سال سودا پخته‌ایم
&&
مدت سی سال دیگر سوختیم
\\
عاقبت چون شمع رویش شعله زد
&&
راست چون پروانه‌ای پر سوختیم
\\
پر چو سوخت آنگه درافکندیم خویش
&&
تا به‌کلی پای تا سر سوختیم
\\
خواه گو بنمای روی و خواه نه
&&
ما سپند روی او بر سوختیم
\\
چون به یک چو می‌نیرزیدیم ما
&&
خرمن پندار یکسر سوختیم
\\
چون شکست اینجا قلم عطار را
&&
اعجمی گشتیم و دفتر سوختیم
\\
\end{longtable}
\end{center}
