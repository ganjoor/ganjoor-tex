\begin{center}
\section*{غزل شماره ۴۲۵: اگر دلم ببرد یار دلبری رسدش}
\label{sec:425}
\addcontentsline{toc}{section}{\nameref{sec:425}}
\begin{longtable}{l p{0.5cm} r}
اگر دلم ببرد یار دلبری رسدش
&&
وگر بپروردم بنده‌پروری رسدش
\\
ز بس که من سر او دارم از قدم تا فرق
&&
گرم چو شمع بسوزد به سرسری رسدش
\\
سفید کاری صبح رخش جهان بگرفت
&&
چو شب به طره طلسم سیه‌گری رسدش
\\
چو آفتاب رخش نور بخش اسلام است
&&
اگر ز زلف نهد رسم کافری رسدش
\\
چو پشت لشکر حسن است روی صف شکنش
&&
اگر به عمد کند قصد لشکری رسدش
\\
بدید بیخبری روی او و گفت امروز
&&
به حکم با مه گردون برابری رسدش
\\
صد آفتاب مرا روشن است کین ساعت
&&
نطاق بسته چو جوزا به چاکری رسدش
\\
چو هست چشمهٔ حیوان زکات‌خواه لبش
&&
اگر قیام کند در سکندری رسدش
\\
سکندری چه بود با لب چو آب حیات
&&
که گر چو خضر رود در پیمبری رسدش
\\
فرید چون ز لب لعل او سخن گوید
&&
نثار در و گهر در سخن‌وری رسدش
\\
\end{longtable}
\end{center}
