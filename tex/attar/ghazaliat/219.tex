\begin{center}
\section*{غزل شماره ۲۱۹: زندهٔ عشق تو آب زندگانی کی خورد}
\label{sec:219}
\addcontentsline{toc}{section}{\nameref{sec:219}}
\begin{longtable}{l p{0.5cm} r}
زندهٔ عشق تو آب زندگانی کی خورد
&&
عاشق رویت غم جان و جوانی کی خورد
\\
هر که خورد از جام دولت درد دردت قطره‌ای
&&
تا که جان دارد شراب شادمانی کی خورد
\\
جان چو باقی شد ز خورشید جمالت تا ابد
&&
ذره‌ای اندوه این زندان فانی کی خورد
\\
گر فصیح عالمی باشد به پیش عشق تو
&&
تا نه لال آید زلال جاودانی کی خورد
\\
دل که عشقت یافت بیرون آمد از بار دو کون
&&
هر که سلطان شد قفای پاسبانی کی خورد
\\
هر کسی گوید شرابی خورده‌ام از دست دوست
&&
پادشه با هر گدایی دوستگانی کی خورد
\\
جان ما چون نوش‌داروی یقین عشق خورد
&&
با یقین عشق ز هر بد گمانی کی خورد
\\
چون دل عطار در عشقت غم صد جان نخورد
&&
پس غم این تنگ جای استخوانی کی خورد
\\
\end{longtable}
\end{center}
