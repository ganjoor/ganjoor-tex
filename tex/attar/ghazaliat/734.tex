\begin{center}
\section*{غزل شماره ۷۳۴: چون کشته شدم هزار باره}
\label{sec:734}
\addcontentsline{toc}{section}{\nameref{sec:734}}
\begin{longtable}{l p{0.5cm} r}
چون کشته شدم هزار باره
&&
بر من به چه می‌کشی کناره
\\
از کشتن کشته‌ای چه خیزد
&&
کشته که کشد هزار باره
\\
حاجت نبود به تیغ کشتن
&&
در پیش رخ تو ماه‌پاره
\\
خود خلق دو کون کشته گردند
&&
هر گه که شوی تو آشکاره
\\
زیرا که ز تیغ غمزهٔ تو
&&
خونی گردد چو لعل خاره
\\
گر بر گیری نقاب از روی
&&
مه شق شود آفتاب پاره
\\
ذرات دو کون دیده گردند
&&
وایند چو ذره در نظاره
\\
از پرتو رویت آخرالامر
&&
هر ذره شود چو صد ستاره
\\
از پرده چو آفتاب رویت
&&
بر مرکب حسن شد سواره
\\
خورشید که شاه پیشگاه است
&&
شد پیش رخ تو پیشکاره
\\
چون شیر عنایتت درآید
&&
هر ذره شوند شیرخواره
\\
طفلان زمانهٔ خرف را
&&
لطف تو بس است گاهواره
\\
کاجزای دو کون را تمام است
&&
لطف تو چو بحر بی کناره
\\
بیچارهٔ خود فرید را خوان
&&
زیرا که ندارد از تو چاره
\\
\end{longtable}
\end{center}
