\begin{center}
\section*{غزل شماره ۵۶۵: چون ندارم سر یک موی خبر زانچه منم}
\label{sec:565}
\addcontentsline{toc}{section}{\nameref{sec:565}}
\begin{longtable}{l p{0.5cm} r}
چون ندارم سر یک موی خبر زانچه منم
&&
بی خبر عمر به سر می‌برم و دم نزنم
\\
نا پدیدار شود در بر من هر دو جهان
&&
گر پدیدار شود یک سر مو زانچه منم
\\
مشکل این است که از خویشتنم نیست خبر
&&
مگر این مشکل از آن است که بی خویشتنم
\\
قرب سی سال ز خود خاک همی دادم باد
&&
تا به جان راه برم راه ببردم به تنم
\\
ای گل باغ دلم، پرده برانداز از روی
&&
ورنه چون گل ز تو صد پاره کنم پیرهنم
\\
چون تویی جمله چرا از تو خبر نیست مرا
&&
که به جان آمد ازین غصه تن ممتحنم
\\
من تو را دارم و بس، در دو جهان وین عجب است
&&
که ز تو در دو جهان بوی ندارم چکنم
\\
تو فکندی ز وطن دور مرا دستم گیر
&&
که چنین بی دل و بی صبر ز حب الوطنم
\\
تا که هستم سخنم از تو و از شیوهٔ توست
&&
چه غمم بودی اگر بشنویی یک سخنم
\\
گر چو شمعم بکشی زار همه روز رواست
&&
ور بسوزیم به شب عاشق آن سوختنم
\\
ور شدم خسته و کشته کفنی نیست مرا
&&
بی گل روی تو چون لاله بس از خون کفنم
\\
ور شوم سوخته و آب ندارم بر لب
&&
صف کشم از مژه و آنگه صف دریا شکنم
\\
چون فرید از غم تو سوخته شد نیست عجب
&&
که چو شمع آتش سوزنده دمد از دهنم
\\
\end{longtable}
\end{center}
