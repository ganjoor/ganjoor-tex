\begin{center}
\section*{غزل شماره ۸۳۴: گر تو خلوتخانهٔ توحید را محرم شوی}
\label{sec:834}
\addcontentsline{toc}{section}{\nameref{sec:834}}
\begin{longtable}{l p{0.5cm} r}
گر تو خلوتخانهٔ توحید را محرم شوی
&&
تاج عالم گردی و فخر بنی آدم شوی
\\
سایه‌ای شو تا اگر خورشید گردد آشکار
&&
تو چو سایه محو خورشید آیی و محرم شوی
\\
جانت در توحید دایم معتکف بنشسته است
&&
تو چرا در تفرقه هر دم به صد عالم شوی
\\
بوده‌ای همرنگ او از پیش و خواهی بد ز پس
&&
این زمان همرنگ او شو نیز تا همدم شوی
\\
چون نداری ز اول و آخر خبر جز بیخودی
&&
گر بکوشی در میانه بیخود اکنون هم شوی
\\
رنگ دریا گیر چون شبنم ز خود بیخود شده
&&
تا شوی همرنگ دریا گرچه یک شبنم شوی
\\
چیست شبنم یک نم از دریاست ناآمیخته
&&
گر بیامیزی تو هم در بحر کل بی غم شوی
\\
ور در آمیزی ز غفلت با هزاران تفرقه
&&
چون نتابد بحر صحبت بو که تو محرم شوی
\\
دل پراکنده روی از جام جم در آینه
&&
جز پراکنده نبینی از پی ماتم شوی
\\
هیچ نبودی هیچ خواهی شد کنون هم هیچ باش
&&
زانکه گر تو هیچ گردی تو ز هیچی کم شوی
\\
گر تو ای عطار هیچ آیی همه گردی مدام
&&
ور همه خواهی چو مردان هیچ در یک دم شوی
\\
\end{longtable}
\end{center}
