\begin{center}
\section*{غزل شماره ۷۷۴: تورا گر نیست با من هیچ کاری}
\label{sec:774}
\addcontentsline{toc}{section}{\nameref{sec:774}}
\begin{longtable}{l p{0.5cm} r}
تورا گر نیست با من هیچ کاری
&&
مرا با تو بسی کار است باری
\\
منت پیوسته خواهم بود غمخوار
&&
توم گرچه نباشی غمگساری
\\
ز حل و عقد عشق ملک رویت
&&
ندارم حاصلی جز انتظاری
\\
بر امید رخ چون آفتابت
&&
چو سایه می‌گذارم روزگاری
\\
دلم را تا تو خواهی بود باقی
&&
نخواهد بود یک ساعت قراری
\\
دلا گر سر عشقت اختیار است
&&
شوی در راه او بی اختیاری
\\
اگر خود را سر مویی شماری
&&
سر مویی نیایی در شماری
\\
اگر خود را ز فرعونی ندانی
&&
ز فرعونی تمامت خاکساری
\\
جهان پر آفتاب است و تو سایه
&&
نیابی جز فنا اینجا حصاری
\\
که اگر در آفتاب آیی تو یکدم
&&
برآرد از تو آن یک دم دماری
\\
چه گردی گرد این دریای اعظم
&&
که جایی غرقه گردی زار زاری
\\
اگر موجی ازین دریا برآید
&&
نماند صورت و صورت نگاری
\\
ز دریا چند گویی چون ندیدی
&&
ازین دریا به جز پر خون کناری
\\
تو معذوری که پشمین دیده‌ای شیر
&&
ندیدی هیچ شیر مرغزاری
\\
اگر روزی ببینی جنگ شیران
&&
ز فای فخر سازی عین عاری
\\
برو چندین چه گردی گرد این راه
&&
که چشمت کور گردد از غباری
\\
به چشم خود برو پیری طلب کن
&&
که تو ننگی شوی بی نامداری
\\
چو نتوانی که سلطان باشی ای دوست
&&
ز خدمتکار سلطان باش باری
\\
اگر نرسد تو را تخت و وزارت
&&
به سگبانی او بر ساز کاری
\\
به هر نوعی که باشی آن او باش
&&
چو بودی آن او چه گل چه خاری
\\
اگر تو یاد گیری حرف عطار
&&
بست این باد دایم یادگاری
\\
\end{longtable}
\end{center}
