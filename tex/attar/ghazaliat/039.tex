\begin{center}
\section*{غزل شماره ۳۹: عزیزا هر دو عالم سایهٔ توست}
\label{sec:039}
\addcontentsline{toc}{section}{\nameref{sec:039}}
\begin{longtable}{l p{0.5cm} r}
عزیزا هر دو عالم سایهٔ توست
&&
بهشت و دوزخ از پیرایهٔ توست
\\
تویی از روی ذات آئینهٔ شاه
&&
شه از روی صفاتی آیهٔ توست
\\
که داند تا تو اندر پردهٔ غیب
&&
چه چیزی و چه اصلی مایهٔ توست
\\
تو طفلی وانکه در گهوارهٔ تو
&&
تو را کج می‌کند هم دایهٔ توست
\\
اگر بالغ شوی ظاهر ببینی
&&
که صد عالم فزون‌تر پایهٔ توست
\\
تو اندر پردهٔ غیبی و آن چیز
&&
که می‌بینی تو آن خود سایهٔ توست
\\
برآی از پرده و بیع و شرا کن
&&
که هر دو کون یک سرمایهٔ توست
\\
تو از عطار بشنو کانچه اصل است
&&
برون نی از تو و همسایهٔ توست
\\
\end{longtable}
\end{center}
