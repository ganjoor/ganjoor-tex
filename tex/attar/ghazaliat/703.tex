\begin{center}
\section*{غزل شماره ۷۰۳: ای دل به میان جان فرو شو}
\label{sec:703}
\addcontentsline{toc}{section}{\nameref{sec:703}}
\begin{longtable}{l p{0.5cm} r}
ای دل به میان جان فرو شو
&&
در حضرت بی‌نشان فرو شو
\\
تا کی گردی به گرد عالم
&&
یک بار به قعر جان فرو شو
\\
گر می‌خواهی که کل شود دل
&&
کلی به دل جهان فرو شو
\\
دریا که تو را به خویشتن خواند
&&
نعره‌زن و جان فشان فرو شو
\\
چون نیست به جز فرو شدن روی
&&
صد سال به یک زمان فرو شو
\\
چون جمله فرو شدند اینجا
&&
تو نیز درین میان فرو شو
\\
گر بر تو فشاند آستین یار
&&
سر بر سر آستان فرو شو
\\
گر هیچ در امتحان کشیدت
&&
مردانه در امتحان فرو شو
\\
تا کی گردی به گرد هرکس
&&
در هرچه دری در آن فرو شو
\\
گر در روش تو نیست سودی
&&
دل خوش کن و در زیان فرو شو
\\
چون نیست یقین که محض جانی
&&
دم درکش و در گمان فرو شو
\\
گر پنهانی برآی پیدا
&&
ور پیدایی نهان فرو شو
\\
گر نیست به عز قرب راهت
&&
در بعد به رایگان فرو شو
\\
گر نتوانی چنین فرو شد
&&
باری برو و چنان فرو شو
\\
عطار چه در مکان نشستی
&&
برخیز و به لامکان فرو شو
\\
\end{longtable}
\end{center}
