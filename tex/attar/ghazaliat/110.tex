\begin{center}
\section*{غزل شماره ۱۱۰: ای دل ز جان در آی که جانان پدید نیست}
\label{sec:110}
\addcontentsline{toc}{section}{\nameref{sec:110}}
\begin{longtable}{l p{0.5cm} r}
ای دل ز جان در آی که جانان پدید نیست
&&
با درد او بساز که درمان پدید نیست
\\
حد تو صبرکردن و خون‌خوردن است و بس
&&
زیرا که حد وادی هجران پدید نیست
\\
در زیر خاک چون دگران ناپدید شو
&&
این است چارهٔ تو چو جانان پدید نیست
\\
ای مرد کندرو چه روی بیش ازین ز پیش
&&
چندین مرو ز پیش که پیشان پدید نیست
\\
با پاسبان درگه او های و هوی زن
&&
چون طمطراق دولت سلطان پدید نیست
\\
ای دل یقین شناس که یک ذره سر عشق
&&
در ضیق کفر و وسعت ایمان پدید نیست
\\
فانی شو از وجود و امید از عدم ببر
&&
کان چیز کان همی طلبی آن پدید نیست
\\
از اصل کار ، جان تو کی با خبر شود
&&
کانجا که اصل کار بود جان پدید نیست
\\
جان ناپدید آمد و در آرزوی جان
&&
از بس که سوخت این دل حیران پدید نیست
\\
عطار را اگر دل و جان ناپدید شد
&&
نبود عجب که چشمهٔ حیوان پدید نیست
\\
\end{longtable}
\end{center}
