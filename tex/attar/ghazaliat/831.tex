\begin{center}
\section*{غزل شماره ۸۳۱: چون لبت به پسته اندر صفت گهر نبینی}
\label{sec:831}
\addcontentsline{toc}{section}{\nameref{sec:831}}
\begin{longtable}{l p{0.5cm} r}
چو لبت به پسته اندر صفت گهر نبینی
&&
چو رخت به پرده اندر تتق قمر نبینی
\\
ز فراق چون منی را چه کشی به درد و خواری
&&
گه اگر بسی بجویی چو منی دگر نبینی
\\
چه نکوییت فزاید که بد آید از تو بر من
&&
چه بود اگر به هر دم به دم از بتر نبینی
\\
مکن ای صنم که گر من نفسی ز دل برآرم
&&
ز تف دلم به عالم پس از آن اثر نبینی
\\
ز غم تو جان عطار اگر از جهان برآید
&&
تو ز بخت و دولت خود پس از آن نظر نبینی
\\
\end{longtable}
\end{center}
