\begin{center}
\section*{غزل شماره ۳۷۷: هنگام صبوح آمد ای هم نفسان خیزید}
\label{sec:377}
\addcontentsline{toc}{section}{\nameref{sec:377}}
\begin{longtable}{l p{0.5cm} r}
هنگام صبوح آمد ای هم نفسان خیزید
&&
یاران موافق را از خواب برانگیزید
\\
یاران همه مشتاقند در آرزوی یک دم
&&
می در فکن ای ساقی از مست نپرهیزید
\\
جامی که تهی گردد از خون دلم پر کن
&&
وانگه می صافی را با درد میامیزید
\\
چون روح حقیقی را افتاد می اندر سر
&&
این نفس بهیمی را از دار در آویزید
\\
خاکی که نصیب آمد از جور فلک ما را
&&
آن خاک به چنگ آرید بر فرق فلک ریزید
\\
یاران قدیم ما در موسم گل رفتند
&&
خون جگر خود را از دیده فرو ریزید
\\
عطار گریزان است از صحبت نا اهلان
&&
گر عین عیان خواهید از خلق بپرهیزید
\\
\end{longtable}
\end{center}
