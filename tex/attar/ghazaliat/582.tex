\begin{center}
\section*{غزل شماره ۵۸۲: ای برده به زلف کفر و دینم}
\label{sec:582}
\addcontentsline{toc}{section}{\nameref{sec:582}}
\begin{longtable}{l p{0.5cm} r}
ای برده به زلف کفر و دینم
&&
وز غمزه نشسته در کمینم
\\
سرگشته و سوکوار از آنم
&&
شوریده و خسته دل ازینم
\\
تا دایره وار کرد زلفت
&&
بر نقطهٔ خون نگر چنینم
\\
از بس که زنم دو دست بر سر
&&
آید به فغان دو آستینم
\\
گه دست گشاده به آسمانم
&&
گه روی نهاده بر زمینم
\\
با این همه جور کز تو دارم
&&
بی نور رخت جهان نبینم
\\
بر باد مده مرا که ناگه
&&
در تو رسد آه آتشینم
\\
عطار شدم ز بوی زلفت
&&
ای زلف تو مشک راستینم
\\
\end{longtable}
\end{center}
