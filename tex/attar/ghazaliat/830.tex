\begin{center}
\section*{غزل شماره ۸۳۰: هر روز ز دلتنگی جایی دگرم بینی}
\label{sec:830}
\addcontentsline{toc}{section}{\nameref{sec:830}}
\begin{longtable}{l p{0.5cm} r}
هر روز ز دلتنگی جایی دگرم بینی
&&
هر لحظه ز بی صبری شوریده ترم بینی
\\
در عشق چنان دلبر جان بر لب و لب برهم
&&
گه نعره‌زنم یابی گه جامه‌درم بینی
\\
در دایرهٔ گردون گر در نگری در من
&&
چون دایره‌ای گردان بی پای و سرم بینی
\\
چندان که درین دریا می‌جویم و می‌پویم
&&
از آتش دل هر دم لب‌خشک‌ترم بینی
\\
از بس‌که به سرگشتم چون چرخ فلک بر سر
&&
چون چرخ فلک دایم زیر و زبرم بینی
\\
در ره گذرت جانا با خاک شدم یکسان
&&
تا بو که برون آیی بر رهگذرم بینی
\\
بر خاک درت زانم تا گر ز سر خشمی
&&
بر بنده بدر آیی بر خاک درم بینی
\\
نی نی که نمی‌خوام کز من اثری ماند
&&
آن به که درین وادی رفته اثرم بینی
\\
تا در ره تو مویی هستیم بود باقی
&&
صد پرده از آن مویی پیش نظرم بینی
\\
چون شمع سحرگاهی می‌سوزم و می‌گریم
&&
چون صبح برآ آخر تا یک سحرم بینی
\\
در ماتم هجر تو از بس که کنم نوحه
&&
زیر بن هر مویی صد نوحه گرم بینی
\\
گر آب خورم روزی صد کوزه بگریم خون
&&
گر قوت خورم یک شب خون جگرم بینی
\\
خاک است مرا بستر خشت است مرا بالین
&&
ور هیچ نخفتم من خوابی دگرم بینی
\\
خون جگر عطار خورد این تن و خفت ای جان
&&
برخیز و بیا آخر تا خواب و خورم بینی
\\
\end{longtable}
\end{center}
