\begin{center}
\section*{غزل شماره ۳۰۹: زلف شبرنگش شبیخون می‌کند}
\label{sec:309}
\addcontentsline{toc}{section}{\nameref{sec:309}}
\begin{longtable}{l p{0.5cm} r}
زلف شبرنگش شبیخون می‌کند
&&
وز سر هر موی صد خون می‌کند
\\
نیست در کافرستان مویی روا
&&
آنچه او زان موی شبگون می‌کند
\\
زلف او کافتاده بینم بر زمین
&&
صید در صحرای گردون می‌کند
\\
زلف او چون از درازی بر زمین است
&&
تاختن بر آسمان چون می‌کند
\\
زلف او لیلی است و خلقی از نهار
&&
از سر زنجیر مجنون می‌کند
\\
آنچه رستم را سزد بر پشت رخش
&&
زلف او بر روی گلگون می‌کند
\\
این چه باشد کرد و خواهد کرد نیز
&&
تا نپنداری که اکنون می‌کند
\\
روی او کافاق یکسر عکس اوست
&&
هر زمانی رونق افزون می‌کند
\\
گر کند یک جلوه خورشید رخش
&&
عرش را با خاک هامون می‌کند
\\
ذره‌ای عکس رخش دعوی حسن
&&
از سر خورشید بیرون می‌کند
\\
از سر یک مژه چشم ساحرش
&&
چرخ را در سینه افسون می‌کند
\\
یارب ابروی کژش بر جان من
&&
راست اندازی چه موزون می‌کند
\\
عقل کل در حسن او مدهوش شد
&&
کز لبش در باده افیون می‌کند
\\
گر سخن گوید چو موسی هر که هست
&&
دایمش از شوق هارون می‌کند
\\
ور بخندد جملهٔ ذرات را
&&
با زلال خضر معجون می‌کند
\\
گر بگویم قطره‌های اشک من
&&
خندهٔ او در مکنون می‌کند
\\
هر زمان زیباتر است او تا فرید
&&
وصف او هر دم دگرگون می‌کند
\\
\end{longtable}
\end{center}
