\begin{center}
\section*{غزل شماره ۵۳۳: گر بوی یک شکن ز سر زلف دلبرم}
\label{sec:533}
\addcontentsline{toc}{section}{\nameref{sec:533}}
\begin{longtable}{l p{0.5cm} r}
گر بوی یک شکن ز سر زلف دلبرم
&&
کفار بشنوند نگروند کافرم
\\
وز زلف او اگر سر مویی به من رسد
&&
در دل نهم چو دیده و در جان بپرورم
\\
درهم ز دست دست سر زلفش از شکن
&&
دستم نمی‌دهد که شکن‌هاش بشمرم
\\
تا برد دل ز من سر زلف معنبرش
&&
از بوی دل شده است دماغی معنبرم
\\
جان من است گرچه نمی‌بینمش چو جان
&&
بی جان چگونه عمر گرامی به سر برم
\\
از پای می درآیم و آگاه نیست کس
&&
تا عشق آن نگار چه سر داشت در سرم
\\
غم می‌رسد به روی من از سوی آن نگار
&&
شادی به روی غم که غم اوست رهبرم
\\
در عشق او دلی است مرا بی خبر ز خویش
&&
وز هر چه زین گذشت خبر نیست دیگرم
\\
تا بو که پای باز نگیرد ز خاک خود
&&
با خاک راه رهگذر او برابرم
\\
زان آمده است با من بیدل به در برون
&&
کز دیرگاه خاک در آن سمن برم
\\
بر خاک خویش می‌گذرد همچو باد و من
&&
بادی به دست مانده و بر خاک آن درم
\\
گفتم بیا و خانه فروشی بزن مرا
&&
گفتا برو که من ز چنین ها نمی‌خرم
\\
گفتم که گوش دار ز عطار یک سخن
&&
گفتا خمش که سر به سخن در نیاورم
\\
\end{longtable}
\end{center}
