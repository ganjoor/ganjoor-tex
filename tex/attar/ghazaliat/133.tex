\begin{center}
\section*{غزل شماره ۱۳۳: خاک کویت هر دو عالم در نیافت}
\label{sec:133}
\addcontentsline{toc}{section}{\nameref{sec:133}}
\begin{longtable}{l p{0.5cm} r}
خاک کویت هر دو عالم در نیافت
&&
گرد راهت فرق آدم در نیافت
\\
ای به بالا برشده چندان که عرش
&&
ذره‌ای شد گرد تو هم در نیافت
\\
دولت تو هیچ بی دولت ندید
&&
شادی تو لشکر غم در نیافت
\\
گنج عشقت در جهان جد و جهد
&&
هم مؤخر هم مقدم در نیافت
\\
زانکه هرگز هفت دریای عظیم
&&
از سر خود نیم شبنم در نیافت
\\
آن چنان جامی که نتوان داد شرح
&&
آن به جد و جهد خود جم در نیافت
\\
آمد و شد صد هزاران پادشاه
&&
ملک تو جز ابن ادهم در نیافت
\\
صد هزاران راهزن در ره فتاد
&&
جز فضیل‌ابن‌عهد محکم در نیافت
\\
صد هزاران زن به نامردی بمرد
&&
این سخن جز جان مریم در نیافت
\\
وی عجب تا مرد ره جهدی نکرد
&&
آنچنان گنجی معظم در نیافت
\\
هر که او ساکن نشد در کوی تو
&&
جنه الفردوس خرم در نیافت
\\
وانکه او مجروح گشت از عشق تو
&&
تا ابد بویی ز مرهم در نیافت
\\
بیش و کم درباخت دل در راه تو
&&
لیک از تو بیش یا کم در نیافت
\\
بس بزرگان را که در گرداب درد
&&
سر فرو شد نیز همدم در نیافت
\\
من چگونه از تو دریابم به حکم
&&
آنچه از تو هر دو عالم در نیافت
\\
چند جویی ای دل برخاسته
&&
آنچه هرگز خلق یکدم در نیافت
\\
تو نیابی این که بس نامحرمی
&&
خاصه هرگز هیچ محرم در نیافت
\\
نیست غم گر چون سلیمان ای فرید
&&
هر گدا ملکی به خاتم در نیافت
\\
\end{longtable}
\end{center}
