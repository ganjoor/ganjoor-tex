\begin{center}
\section*{غزل شماره ۳۵۴: هر که را دانهٔ نار تو به دندان آید}
\label{sec:354}
\addcontentsline{toc}{section}{\nameref{sec:354}}
\begin{longtable}{l p{0.5cm} r}
هر که را دانهٔ نار تو به دندان آید
&&
هر دم از چشمهٔ خضرش مدد جان آید
\\
کو سکندر که لب چشمهٔ حیوان دیدم
&&
تا به عهد تو سوی چشمهٔ حیوان آید
\\
عقل سرکش چو ببیند لب و دندان تو را
&&
پیش لعل لب تو از بن دندان آید
\\
هر که در حال شد از زلف پریشانت دمی
&&
حال او چون سر زلف تو پریشان آید
\\
وانکه بر طرهٔ زیر و زبرت دست گشاد
&&
از پس و پیش برو ناوک مژگان آید
\\
چون سر زلف تو از مشک شود چوگان ساز
&&
همچو گویی سر مردانش به چوگان آید
\\
سر مردان جهان در سر چوگان تو شد
&&
مرد کو در ره عشقت که به میدان آید
\\
در ره عشق تو سرگشته بماندیم و هنوز
&&
نیست امید که این راه به پایان آید
\\
ماند عطار کنون چشم به ره گوش به در
&&
تا ز نزدیک تو ای ماه چه فرمان آید
\\
\end{longtable}
\end{center}
