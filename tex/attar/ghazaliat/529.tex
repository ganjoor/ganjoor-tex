\begin{center}
\section*{غزل شماره ۵۲۹: ترسا بچه‌ای کشید در کارم}
\label{sec:529}
\addcontentsline{toc}{section}{\nameref{sec:529}}
\begin{longtable}{l p{0.5cm} r}
ترسا بچه‌ای کشید در کارم
&&
بربست به زلف خویش زنارم
\\
پس حلقهٔ زلف کرد در گوشم
&&
یعنی که به بندگی ده اقرارم
\\
در بندگیش نه هندوم بدخوی
&&
هستم حبشی که داغ او دارم
\\
پروانهٔ او شدم که هر ساعت
&&
در جمع چو شمع می‌کشد زارم
\\
شاید که کشد چو هست عیسی دم
&&
کز معجزه زنده کرد صد بارم
\\
او یوسف عالم است در خوبی
&&
من دست و ترنج پیش او دارم
\\
هرگز نایم ز بار او بیرون
&&
کز عشق نهاد صاع در بارم
\\
زان روز که درد عشق او خوردم
&&
مانده است گرو به درد دستارم
\\
دی ساکن کنج صومعه بودم
&&
وامروز ز ساکنان خمارم
\\
چون دانم داد شرح حال خود
&&
فی‌الجمله نه کافرم نه دین دارم
\\
کو در عالم کسی که برهاند
&&
یکباره ز ناکسی عطارم
\\
\end{longtable}
\end{center}
