\begin{center}
\section*{غزل شماره ۵۹۳: ما درد فروش هر خراباتیم}
\label{sec:593}
\addcontentsline{toc}{section}{\nameref{sec:593}}
\begin{longtable}{l p{0.5cm} r}
ما درد فروش هر خراباتیم
&&
نه عشوه فروش هر کراماتیم
\\
انگشت‌زنان کوی معشوقیم
&&
وانگشت‌نمای اهل طاماتیم
\\
حیلت‌گر و مهره دزد و اوباشیم
&&
دردی‌کش و کم‌زن خراباتیم
\\
در شیوهٔ کفر پیر و استادیم
&&
در شیوهٔ دین خر خرافاتیم
\\
گه مرد کلیسیای و ناقوسیم
&&
گه صومعه‌دار عزی و لاتیم
\\
گه معتکفان کوی لاهوتیم
&&
گه مستمعان التحیاتیم
\\
گه مست خراب دردی دردیم
&&
گه مست شراب عالم الذاتیم
\\
با عادت و رسم نیست ما را کار
&&
ما کی ز مقام رسم و عاداتیم
\\
ما را ز عبادت و ز مسجد چه
&&
چه مرد مساجد و عباداتیم
\\
با این همه مفسدی و زراقی
&&
چه بابت قربت و مناجاتیم
\\
برخاست ز ما حدیث ما و من
&&
زیرا که نه مرد این مقاماتیم
\\
در حالت بیخودی چو عطاریم
&&
پروانهٔ شمع نور مشکاتیم
\\
\end{longtable}
\end{center}
