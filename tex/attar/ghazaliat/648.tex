\begin{center}
\section*{غزل شماره ۶۴۸: عشق را بی‌خویشتن باید شدن}
\label{sec:648}
\addcontentsline{toc}{section}{\nameref{sec:648}}
\begin{longtable}{l p{0.5cm} r}
عشق را بی‌خویشتن باید شدن
&&
نفس خود را راهزن باید شدن
\\
بت بود در راه او هرچه آن نه اوست
&&
در ره او بت‌شکن باید شدن
\\
زلف جانان را شکن بیش از حد است
&&
کافر یک یک شکن باید شدن
\\
تو بدو نزدیک نزدیکی ولیک
&&
دور دور از خویشتن باید شدن
\\
در نگنجد ما و من در راه او
&&
در رهش بی ما و من باید شدن
\\
دوست چون هرگز نیاید در وطن
&&
عاشقان را بی وطن باید شدن
\\
در ره او بر امید وصل او
&&
خاک راه تن به تن باید شدن
\\
همچو لاله غرقه در خون جگر
&&
زنده در زیر کفن باید شدن
\\
در ره او چون دویی را راه نیست
&&
با یکی در پیرهن باید شدن
\\
پس چو عطار اندر آفاق جهان
&&
پاکباز انجمن باید شدن
\\
\end{longtable}
\end{center}
