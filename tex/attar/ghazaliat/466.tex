\begin{center}
\section*{غزل شماره ۴۶۶: تا دیده‌ام رخ تو کم جان گرفته‌ام}
\label{sec:466}
\addcontentsline{toc}{section}{\nameref{sec:466}}
\begin{longtable}{l p{0.5cm} r}
تا دیده‌ام رخ تو کم جان گرفته‌ام
&&
اما هزار جان عوض آن گرفته‌ام
\\
چون ز لبت نبود مرا روی یک شکر
&&
ای بس که پشت دست به دندان گرفته‌ام
\\
تا آب زندگانی تو دیده‌ام ز دور
&&
دور از رخ تو مرگ خود آسان گرفته‌ام
\\
چون توشهٔ وصال توام دست می نداد
&&
در پا فتاده گوشهٔ هجران گرفته‌ام
\\
چون بر کمان ابروی تو تیر دیده‌ام
&&
گر خواست وگرنه کم جان گرفته‌ام
\\
آوازهٔ لب تو ز خلقی شنیده‌ام
&&
زان تشنه راه چشمهٔ حیوان گرفته‌ام
\\
آن راه چشمه در ظلمات دو زلف توست
&&
یارب رهی چه دور و پریشان گرفته‌ام
\\
چون خشک‌سال وصل تو در کون دیده‌ام
&&
از ابر چشم عادت طوفان گرفته‌ام
\\
گرچه ز چشم خاست مرا عشق تو چو اشک
&&
این جرم نیز بر دل بریان گرفته‌ام
\\
برهم دریده پرده ز تر دامنی چشم
&&
کو را به دست ابر گریبان گرفته‌ام
\\
گفتی که من به کار تو سر تیز می‌کنم
&&
کین پر دلی ز زلف زره‌سان گرفته‌ام
\\
خونی گشاد از همه سر تیزی توام
&&
وین تجربه ز ناوک مژگان گرفته‌ام
\\
چون تو ز ناز و کبر نگنجی به شهر در
&&
من شهر ترک گفته بیابان گرفته‌ام
\\
عطار تا که از تو چو یوسف جدا افتاد
&&
یعقوب‌وار کلبهٔ احزان گرفته‌ام
\\
\end{longtable}
\end{center}
