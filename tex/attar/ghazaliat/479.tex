\begin{center}
\section*{غزل شماره ۴۷۹: ساقیا توبه شکستم، جرعه‌ای می ده به دستم}
\label{sec:479}
\addcontentsline{toc}{section}{\nameref{sec:479}}
\begin{longtable}{l p{0.5cm} r}
ساقیا توبه شکستم، جرعه‌ای می ده به دستم
&&
من ز می ننگی ندارم، می‌پرستم می‌پرستم
\\
سوختم از خوی خامان، بر شدم زین ناتمامان
&&
ننگم است از ننگ نامان، توبه پیش بت شکستم
\\
رفتم و توبه شکستم، وز همه عیبی برستم
&&
با حریفان خوش نشستم، با رفیقان عهد بستم
\\
من نه مرد ننگ و نامم، فارغ از انکار عامم
&&
می فروشان را غلامم، چون کنم، چون می‌پرستم
\\
دین و دل بر باد دادم، رخت جان بر در نهادم
&&
از جهان بیرون فتادم، از خودی خود برستم
\\
خرقه از تن برکشیدم، جام صافی در کشیدم
&&
عقل را بر سر کشیدم، در صف رندان نشستم
\\
خرقه را زنار کردم، خانه را خمار کردم
&&
گوشهٔ در باز کردم، زان میان مردانه جستم
\\
ساقیا باده فزون کن، تا منت گویم که چون کن
&&
خیزم از مسجد برون کن، کز می دوشینه مستم
\\
گر چو عطارم که آبم می‌برد از دیده خوابم
&&
بس که از باده خرابم، نیستم واقف که هستم
\\
\end{longtable}
\end{center}
