\begin{center}
\section*{غزل شماره ۱۰۳: در عشق قرار بی‌قراری است}
\label{sec:103}
\addcontentsline{toc}{section}{\nameref{sec:103}}
\begin{longtable}{l p{0.5cm} r}
در عشق قرار بی‌قراری است
&&
بدنامی عشق نام‌داری است
\\
چون نیست شمار عشق پیدا
&&
مشمر که شمار بی‌شماری است
\\
در عشق ز اختیار بگذار
&&
عاشق بودن نه اختیاری است
\\
گر دل داری تو را سزد عشق
&&
ورنه همه زهد و سوگواری است
\\
زاری می‌کن چو دل ندادی
&&
تا دل ندهند کارزاری است
\\
دل کیست شکار خاص شاه است
&&
شاه از پی او به دوستداری است
\\
شاهی که همه جهانش ملک است
&&
در دشت ز بهر یک شکاری است
\\
جانا بر تو قرار آن راست
&&
کز عشق تو عین بی‌قراری است
\\
آن را که گرفت عشق تو نیست
&&
در معرض صد گرفتکاری است
\\
وآن است عزیز در دو عالم
&&
کز عشق تو در هزار خواری است
\\
هر بی‌خبری که قدر عشقت
&&
می‌نشناسد ز خاکساری است
\\
وانکس که شناخت خردهٔ عشق
&&
هر خردهٔ او بزرگواری است
\\
پروانهٔ توست جان عطار
&&
زان است که غرق جان سپاری است
\\
\end{longtable}
\end{center}
