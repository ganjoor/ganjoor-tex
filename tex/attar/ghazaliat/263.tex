\begin{center}
\section*{غزل شماره ۲۶۳: کسی کز حقیقت خبردار باشد}
\label{sec:263}
\addcontentsline{toc}{section}{\nameref{sec:263}}
\begin{longtable}{l p{0.5cm} r}
کسی کز حقیقت خبردار باشد
&&
جهان را بر او چه مقدار باشد
\\
جهان وزن جایی پدیدار آرد
&&
که در دیده او را پدیدار باشد
\\
بلی دیده‌ای کز حقیقت گشاید
&&
جهان پیش او ذره کردار باشد
\\
غلط گفتم آن ذره‌ای گر بود هم
&&
چو زان چشم بینی تو بسیار باشد
\\
کسی را که دو کون یک قطره گردد
&&
ببین تا درونش چه بر کار باشد
\\
اگر سایهٔ باطن او نباشد
&&
کجا گردش چرخ دوار باشد
\\
نباشد خبر یک سر مویش از خود
&&
بقای ابد را سزاوار باشد
\\
کسی را که تیمار دادش بقا شد
&&
فنا گشتن از خود چه تیمار باشد
\\
غم خود مخور تا تو را ذره ذره
&&
به صد وجه پیوسته غمخوار باشد
\\
به جای تو چون اصل کار است باقی
&&
اگر تو نباشی بسی کار باشد
\\
درین راه اگر تا ابد فکر برود
&&
مپندار سری که پندار باشد
\\
اگر جان عطار این بوی یابد
&&
یقین دان که آن دم نه عطار باشد
\\
\end{longtable}
\end{center}
