\begin{center}
\section*{غزل شماره ۴۱۲: جان ز مشک زلف دلم چون جگر مسوز}
\label{sec:412}
\addcontentsline{toc}{section}{\nameref{sec:412}}
\begin{longtable}{l p{0.5cm} r}
جان ز مشک زلف دلم چون جگر مسوز
&&
با من بساز و جانم ازین بیشتر مسوز
\\
هر روز تا به شب چو ز عشق تو سوختم
&&
هر شب چو شمع زار مرا تا سحر مسوز
\\
مرغ توام به دست خودم دانه‌ای فرست
&&
زین بیش در هوای خودم بال و پر مسوز
\\
چون آرزوی وصل توام خشک و تر بسوخت
&&
در آتش فراق، خودم خشک و تر مسوز
\\
چون دل ببردی و جگر من بسوختی
&&
با دل بساز و بیش ازینم جگر مسوز
\\
یکبارگی چو می‌بنسوزی مرا تمام
&&
هر روزم از فراق به نوعی دگر مسوز
\\
جانم که زآرزوی لبت همچو شمع سوخت
&&
چون عود بی‌مشاهدهٔ آن شکر مسوز
\\
عطار را اگر نظری بر تو اوفتد
&&
این نیست ور بود نظرش در بصر مسوز
\\
\end{longtable}
\end{center}
