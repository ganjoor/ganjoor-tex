\begin{center}
\section*{غزل شماره ۳۷۲: برکناری شو ز هر نقشی که آن آید پدید}
\label{sec:372}
\addcontentsline{toc}{section}{\nameref{sec:372}}
\begin{longtable}{l p{0.5cm} r}
برکناری شو ز هر نقشی که آن آید پدید
&&
تا تو را نقاش مطلق زان میان آید پدید
\\
بگذر از نقش دو عالم خواه نیک و خواه بد
&&
تا ز بی نقشیت نقشی جاودان آید پدید
\\
تو ز چشم خویش پنهانی اگر پیدا شوی
&&
در میان جان تو گنجی نهان آید پدید
\\
تو طلسم گنج جانی گر طلسمت بشکنی
&&
ز اژدها هرگز نترسی گنج جان آید پدید
\\
ای دل از تن گر برفتی رفته باشی زآسمان
&&
در خیال آسمان کی آسمان آید پدید
\\
جز خیالی چشم تو هرگز نبیند از جهان
&&
از خیال جمله بگذر تا جهان آید پدید
\\
ناپدید از فرع شو، در هرچه پیوستی ببر
&&
تا پدید آرندهٔ اصل عیان آید پدید
\\
چون تفاوت نیست در پیشان معنی ذره‌ای
&&
کس نگشت آگاه تا چون این و آن آید پدید
\\
چون در اصل کار راه و رهبر و رهرو یکی است
&&
اختلاف از بهر چه در کاروان آید پدید
\\
خار و گل چون مختلف افتاد حیران مانده‌ام
&&
تا چرا خار و گل از یک گلستان آید پدید
\\
باز کن چشم و ببین کز بی نشانی چشم را
&&
نور با آب سیه در یک مکان آید پدید
\\
بود دریای دو عالم قطره نا افشانده‌ای
&&
چون چنین می‌خواست آمد تا چنان آید پدید
\\
گر تو نشنودی ز من بشنو که شاهی ای عجب
&&
میزبانی کرده عمری میهمان آید پدید
\\
ای عجب چون گاو گردون می‌کشد باری که هست
&&
دایم از گردون چرا بانگ و فغان آید پدید
\\
چون توانم کرد شرح این داستان را ذره‌ای
&&
زانکه اینجا هر نفس صد داستان آید پدید
\\
این زمان باری فروشد صد جهان جان بی‌نشان
&&
تا ازین پس از کدامین جان نشان آید پدید
\\
چون بزرگان را درین ره آنچه باید حل نشد
&&
حل این کی از فرید خرده‌دان آید پدید
\\
\end{longtable}
\end{center}
