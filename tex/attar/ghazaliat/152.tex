\begin{center}
\section*{غزل شماره ۱۵۲: چون لعل توام هزار جان داد}
\label{sec:152}
\addcontentsline{toc}{section}{\nameref{sec:152}}
\begin{longtable}{l p{0.5cm} r}
چون لعل توام هزار جان داد
&&
بر لعل تو نیم جان توان داد
\\
جان در غم عشق تو میان بست
&&
دل در غمت از میان جان داد
\\
جانم که فلک ز دست او بود
&&
از دست تو تن در امتحان داد
\\
پر نام تو شد جهان و از تو
&&
می‌نتواند کسی نشان داد
\\
ای بس که رخ چو آتش تو
&&
دل سوخته سر درین جهان داد
\\
پنهان ز رقیب غمزه دوشم
&&
لعل تو به یک شکر زبان داد
\\
امروز چو غمزه‌ات بدانست
&&
تاب از سر زلف تو در آن داد
\\
از غمزهٔ تو کنون نترسم
&&
چون لعل توام به جان امان داد
\\
دندان تو گرچه آب دندانست
&&
هر لقمه که دادم استخوان داد
\\
ابروی تو پشت من کمان کرد
&&
ای ترک تو را که این کمان داد
\\
عطار چو مرغ توست او را
&&
سر نتوانی ز آشیان داد
\\
\end{longtable}
\end{center}
