\begin{center}
\section*{غزل شماره ۷۵۶: جانا دلم ببردی در قعر جان نشستی}
\label{sec:756}
\addcontentsline{toc}{section}{\nameref{sec:756}}
\begin{longtable}{l p{0.5cm} r}
جانا دلم ببردی در قعر جان نشستی
&&
من بر کنار رفتم تو در میان نشستی
\\
گر جان ز من ربودی الحمدلله ای جان
&&
چون تو بجای جانم بر جای جان نشستی
\\
گرچه تو را نبینم بی تو جهان نبینم
&&
یعنی تو نور چشمی در چشم از آن نشستی
\\
چون خواستی که عاشق در خون دل بگردد
&&
در خون دل نشاندی در لامکان نشستی
\\
من چون به خون نگردم از شوق تو چو تنها
&&
در زیر خدر عزت چندان نهان نشستی
\\
گفتی مرا چو جویی در جان خویش یابی
&&
چون جویمت که در جان بس بی نشان نشستی
\\
برخاست ز امتحانت یکبارگی دل من
&&
من خود کیم که با من در امتحان نشستی
\\
تا من تو را بدیدم دیگر جهان ندیدم
&&
گم شد جهان ز چشمم تا در جهان نشستی
\\
عطار عاشق از تو زین بیش صبر نکند
&&
نبود روا که چندین بی عاشقان نشستی
\\
\end{longtable}
\end{center}
