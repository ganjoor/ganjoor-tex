\begin{center}
\section*{غزل شماره ۱۲۹: تا گل از ابر آب حیوان یافت}
\label{sec:129}
\addcontentsline{toc}{section}{\nameref{sec:129}}
\begin{longtable}{l p{0.5cm} r}
تا گل از ابر آب حیوان یافت
&&
گرد خود صد هزار دستان یافت
\\
زره ابر گشت پیکان باز
&&
جوشن آب زخم پیکان یافت
\\
گل خندان چو برفکند نقاب
&&
ابر را زار زار گریان یافت
\\
چون صبا چاک کرد دامن گل
&&
نافهٔ مشک در گریبان یافت
\\
ای نگاری که هر که دید رخت
&&
از رخ جانفزای تو جان یافت
\\
به دل و جان تو را که جان و دلی
&&
هر که فرمان ببرد فرمان یافت
\\
می گلرنگ خور به موسم گل
&&
که گل تازه‌روی باران یافت
\\
می‌خور و شاد زی که خوشتر ازین
&&
یک نفس در دو کون نتوان یافت
\\
می به عطار ده به سرخی لعل
&&
که زمی جان چو در درخشان یافت
\\
\end{longtable}
\end{center}
