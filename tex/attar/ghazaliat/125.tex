\begin{center}
\section*{غزل شماره ۱۲۵: درد دل من از حد و اندازه درگذشت}
\label{sec:125}
\addcontentsline{toc}{section}{\nameref{sec:125}}
\begin{longtable}{l p{0.5cm} r}
درد دل من از حد و اندازه درگذشت
&&
از بس که اشک ریختم آبم ز سر گذشت
\\
پایم ز دست واقعه در قیر غم گرفت
&&
کارم ز جور حادثه از دست درگذشت
\\
بر روی من چو بر جگر من نماند آب
&&
بس سیل‌های خون که ز خون جگر گذشت
\\
هر شب ز جور چرخ بلایی دگر رسید
&&
هر دم ز روز عمر به دردی دگر گذشت
\\
خواب و خورم نماند و گر قصه گویمت
&&
زان غصه‌ها که بر من بی خواب و خور گذشت
\\
اشکم به قعر سینهٔ ماهی فرو رسید
&&
آهم از روی آینهٔ ماه درگذشت
\\
در بر گرفت جان مرا تیر غم چنانک
&&
پیکان به جان رسید وز جان تا به بر گذشت
\\
بر جان من که رنج و بلایی ندیده بود
&&
چندین بلا و رنج ز دردم بدر گذشت
\\
بر عمر من اجل چو سحرگاه شام خورد
&&
زان شام آفتاب من اندر سحر گذشت
\\
عطار چون که سایهٔ عزت بر او نماند
&&
چون سایه‌ای ز خواری خود در به در گذشت
\\
\end{longtable}
\end{center}
