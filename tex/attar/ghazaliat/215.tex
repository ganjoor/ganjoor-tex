\begin{center}
\section*{غزل شماره ۲۱۵: لوح چو سیمت خطی چو قیر بر آورد}
\label{sec:215}
\addcontentsline{toc}{section}{\nameref{sec:215}}
\begin{longtable}{l p{0.5cm} r}
لوح چو سیمت خطی چو قیر بر آورد
&&
تا دلم از خط تو نفیر بر آورد
\\
لعل تو می‌خورد خون سوختهٔ من
&&
تا خطت آن خون کنون ز شیر بر آورد
\\
گرچه دلم در کشید روی چه مقصود
&&
خط تو چون مویش از خمیر بر آورد
\\
چشم تو یارب ز هر که روی تو خواهد
&&
آنچه هلاکت به زخم تیر بر آورد
\\
دشمن آیینه‌ام اگرچه بود راست
&&
کو به دروغی تو را نظیر بر آورد
\\
در صفتت رفت و روب کرد بسی دل
&&
لاجرم آن گرد از ضمیر بر آورد
\\
تا که سر رزمهٔ جمال گشادی
&&
رشک دمار از مه منیر بر آورد
\\
اطلس روی تو عکس بر فلک انداخت
&&
چهرهٔ خورشید چون ز زیر برآورد
\\
صبح رخت تا ز جیب حسن برآمد
&&
تا به ابد پای شب ز قیر بر آورد
\\
عقل مگر سر کشید از سر زلفت
&&
سر به فسون‌های دلپذیر بر آورد
\\
زلف تو خود عقل را ببست به مویی
&&
گرد همه عالمش اسیر بر آورد
\\
عقل بسی گرد وصف لعل تو می‌گشت
&&
تا که سخن‌های جای‌گیر بر آورد
\\
بخت جوان لب تو در دهنش کرد
&&
هر نفسی را که عقل پیر بر آورد
\\
بی لب تو دل نداشت صبر زمانی
&&
جان به لب از حلق ناگزیر بر آورد
\\
چون ننوازی مرا چو چنگ که عطار
&&
هر نفسی ناله‌ای چو زیر بر آورد
\\
\end{longtable}
\end{center}
