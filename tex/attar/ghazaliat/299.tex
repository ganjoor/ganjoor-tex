\begin{center}
\section*{غزل شماره ۲۹۹: چون سیمبران روی به گلزار نهادند}
\label{sec:299}
\addcontentsline{toc}{section}{\nameref{sec:299}}
\begin{longtable}{l p{0.5cm} r}
چون سیمبران روی به گلزار نهادند
&&
گل را ز رخ چون گل خود خار نهادند
\\
تا با رخ چون گل بگذشتند به گلزار
&&
نار از رخ گل در دل گلنار نهادند
\\
در کار شدند و می چون زنگ کشیدند
&&
پس عاشق دلسوخته را کار نهادند
\\
تلخی ز می لعل ببردند که می را
&&
تنگی ز لب لعل شکربار نهادند
\\
ای ساقی گلرنگ درافکن می گلبوی
&&
کز گل کلهی بر سر گلزار نهادند
\\
می‌نوش چو شنگرف به سرخی که گل تر
&&
طفلی است که در مهد چو زنگار نهادند
\\
بوی جگر سوخته بشنو که چمن را
&&
گلهای جگر سوخته در بار نهادند
\\
زان غرقهٔ خون گشت تن لاله که او را
&&
آن داغ سیه بر دل خون خوار نهادند
\\
سوسن چو زبان داشت فروشد به خموشی
&&
در سینهٔ او گوهر اسرار نهادند
\\
از بر بنیارد کس و از بحر نزاید
&&
آن در که درین خاطر عطار نهادند
\\
\end{longtable}
\end{center}
