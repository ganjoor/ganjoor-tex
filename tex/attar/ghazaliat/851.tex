\begin{center}
\section*{غزل شماره ۸۵۱: چون روی بود بدان نکویی}
\label{sec:851}
\addcontentsline{toc}{section}{\nameref{sec:851}}
\begin{longtable}{l p{0.5cm} r}
چون روی بود بدان نکویی
&&
نازش برود به هرچه گویی
\\
رویی که ز شرم او درافتاد
&&
خورشید فلک به زرد رویی
\\
چون در خور او نمی‌توان شد
&&
بر بوی وصال او چه پویی
\\
خون می‌خور و پشت دست می‌خای
&&
گر در ره درد مرد اویی
\\
جانان به تو باز ننگرد راست
&&
تا دست ز جان و دل نشویی
\\
تو ره نبری تو تا تویی تو
&&
تا کی تو تویی تویی و تویی
\\
چیزی که ازو خبر نداری
&&
گم ناشده از تو چند جویی
\\
گر گویندت چه گم شد از تو
&&
ای غره به خویشتن چه گویی
\\
باری بنشین گزاف کم‌گوی
&&
بندیش که در چه آرزویی
\\
عطار کجا رسی به سلطان
&&
زیرا که کم از سگان کویی
\\
\end{longtable}
\end{center}
