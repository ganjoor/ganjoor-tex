\begin{center}
\section*{غزل شماره ۳۲۱: آن را که ز وصل او خبر بود}
\label{sec:321}
\addcontentsline{toc}{section}{\nameref{sec:321}}
\begin{longtable}{l p{0.5cm} r}
آن را که ز وصل او خبر بود
&&
هر روز قیامتی دگر بود
\\
چه جای قیامت است کاینجا
&&
این شور از آن عظیم‌تر بود
\\
زیرا که قیامت قوی را
&&
در حد وجود پا و سر بود
\\
وین شور چو پا و سر ندارد
&&
هرگز نتواندش گذر بود
\\
چون نیست نهایت ره عشق
&&
زین ره نه نشان و نه اثر بود
\\
هر کس که ازین رهت خبر داد
&&
می‌دان به یقین که بی خبر بود
\\
زین راه چو یک قدم نشان نیست
&&
چه لایق هر قدم شمر بود
\\
راهی است که هر که یک قدم زد
&&
شد محو اگر چه نامور بود
\\
چندان که به غور ره نگه کرد
&&
نه راهرو و نه راهبر بود
\\
القصه کسی که پیشتر رفت
&&
سرگشتهٔ راه بیشتر بود
\\
بر گام نخست بود مانده
&&
آنکو همه عمر در سفر بود
\\
وانکس که بیافت سر این راه
&&
شد کور اگرچه دیده‌ور بود
\\
کین راز کسی شنید و دانست
&&
کز دیده و گوش کور و کر بود
\\
مانند فرید اندرین راه
&&
پر دل شد اگرچه بی جگر بود
\\
عطار که بود مرد این راه
&&
زان جملهٔ عمر نوحه‌گر بود
\\
\end{longtable}
\end{center}
