\begin{center}
\section*{غزل شماره ۴۶۴: فتنهٔ زلف دلربای توام}
\label{sec:464}
\addcontentsline{toc}{section}{\nameref{sec:464}}
\begin{longtable}{l p{0.5cm} r}
فتنهٔ زلف دلربای توام
&&
تشنهٔ جام جانفزای توام
\\
نیست چون زلف تو سر خویشم
&&
گرچه چون زلف در قفای توام
\\
جز هوای توام نمی‌سازد
&&
زانکه پروردهٔ هوای توام
\\
گر غباری است از منت زآن است
&&
که من خسته خاک پای توام
\\
تا کنارم ز اشک دریا شد
&&
نیست کاری جز آشنای توام
\\
چون به صد وجه تو بلای منی
&&
من به صد درد مبتلای توام
\\
از همه فارغم که در دو جهان
&&
می نیاید به جز رضای توام
\\
بس بود از دو عالم این ملکم
&&
که تو آنی که من گدای توام
\\
از وجود فرید سیر شدم
&&
گمشده در عدم برای توام
\\
\end{longtable}
\end{center}
