\begin{center}
\section*{غزل شماره ۶۵۵: خیز و از می آتشی در ما فکن}
\label{sec:655}
\addcontentsline{toc}{section}{\nameref{sec:655}}
\begin{longtable}{l p{0.5cm} r}
خیز و از می آتشی در ما فکن
&&
نعرهٔ مستانه در بالا فکن
\\
چون نظیرت نیست در دریا کسی
&&
خویش را خوش در بن دریا فکن
\\
خون رز بر چهرهٔ گل نوش کن
&&
پس ز راه دیده بر صحرا فکن
\\
تا کیم خاری نهی می خور چو گل
&&
دیده بر روی گل رعنا فکن
\\
چون هزار آوا نمی‌خفتد ز عشق
&&
خرقهٔ جان بر هزار آوا فکن
\\
گر تو را مستی و عشق بلبل است
&&
شب مخسب و شورشی در ما فکن
\\
شیر گیران جمله غوغا کرده‌اند
&&
خویش را در پیش سر غوغا فکن
\\
عمر امشب رفت اگر دستیت هست
&&
عمر مستان را پی فردا فکن
\\
تا کی ای عطار از خارا دلی
&&
شیشهٔ می خواه و بر خارا فکن
\\
\end{longtable}
\end{center}
