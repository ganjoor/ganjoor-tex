\begin{center}
\section*{غزل شماره ۱۹: بعدجوی از نفس سگ گر قرب جان می‌بایدت}
\label{sec:019}
\addcontentsline{toc}{section}{\nameref{sec:019}}
\begin{longtable}{l p{0.5cm} r}
بعدجوی از نفس سگ گر قرب جان می‌بایدت
&&
ترک کن این چاه و زندان گر جهان می‌بایدت
\\
باز عرشی گر سر جبریل داری پر برآر
&&
ورنه در گلخن نشین گر استخوان می‌بایدت
\\
نفس را چون جعفر طیار برکن بال و پر
&&
گر به بالا پر و بال مرغ جان می‌بایدت
\\
در جهان قدس اگر داری سبک روحی طمع
&&
بر جهان جسم دایم سر گران می‌بایدت
\\
عمر در سود و زیان بردی به آخر بی خبر
&&
می ندارد سود با تو پس زیان می‌بایدت
\\
چند گردی در زمین بی پا و سر چون آسمان
&&
از زمین بگسل اگر بر آسمان می‌بایدت
\\
روز و شب مشغول کار و بار دنیا مانده‌ای
&&
دین به سرباری دنیا رایگان می‌بایدت
\\
هرچه گوئی چون ترازو زین زبان گر یک جو است
&&
گنگ شو از ما سوی الله گر زبان می‌بایدت
\\
جو کشی و نیم جو همچون ترازوی دو سر
&&
از خری جو می مکش گر کهکشان می‌بایدت
\\
ای عجب نمرود نفس و وانگهی همچون خلیل
&&
زحمت جبریل رفته از میان می‌بایدت
\\
در هوا استاده و از منجنیق انداخته
&&
بر سر آتش به خلوت همچنان می‌بایدت
\\
چون تو از آذر مزاجی دوستی با زر چرا
&&
پس چو ابراهیم آتش گلستان می‌بایدت
\\
ای خر مرده سگ نفست به گلخن در کشید
&&
پس چو عیس بر فلک دامن کشان می‌بایدت
\\
در جهان خوفناک ایمن نشینی ای فرید
&&
امن تو از چیست چون خط امان می‌بایدت
\\
\end{longtable}
\end{center}
