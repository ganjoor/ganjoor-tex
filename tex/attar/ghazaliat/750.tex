\begin{center}
\section*{غزل شماره ۷۵۰: درآمد از در دل چون خرابی}
\label{sec:750}
\addcontentsline{toc}{section}{\nameref{sec:750}}
\begin{longtable}{l p{0.5cm} r}
درآمد از در دل چون خرابی
&&
ز می بر آتش جانم زد آبی
\\
شرابم داد و گفتا نوش و خاموش
&&
کزین خوشتر نخوردستی شرابی
\\
چو جان نوشید جام جان فزایش
&&
میان جان برآمد آفتابی
\\
اگرچه خامشی فرمود لیکن
&&
دلم با خامشی ناورد تابی
\\
فغان دربست تا آن شمع جان‌ها
&&
برافکند از جمال خود نقابی
\\
چو جانم روی یار خوش‌نمک دید
&&
ز دل خوش بر نمک می‌زد کبابی
\\
همی ناگاه در جان من افتاد
&&
عجب شوری عجایب اضطرابی
\\
جهان از خود همی پر دید و خود نه
&&
من این ناخوانده‌ام در هیچ بابی
\\
درین منزل فروماندیم جمله
&&
که دارد مشکل ما را جوابی
\\
برو عطار و دم درکش کزین سوز
&&
چو آتش در دلم افتاد تابی
\\
\end{longtable}
\end{center}
