\begin{center}
\section*{غزل شماره ۷۷۱: الصلا ای دل اگر در عشق او اقرار داری}
\label{sec:771}
\addcontentsline{toc}{section}{\nameref{sec:771}}
\begin{longtable}{l p{0.5cm} r}
الصلا ای دل اگر در عشق او اقرار داری
&&
الحذر گر ذره‌ای در عشق او انکار داری
\\
کی توانی دید روی گل که همچون خار گشتی
&&
گر زمانی خلوتی داری میان خار داری
\\
تا تو از توی و توی خود برون آیی به‌کلی
&&
عمر بگذشت و تو در هر توی عمری کار داری
\\
همچو پروانه سر افشان گر وصال شمع خواهی
&&
همچو خرقه سر درافکن گر سر اسرار داری
\\
در گذر از کعبه و خمار گر تو مرد عشقی
&&
زانکه تو ره ماورای کعبه و خمار داری
\\
در درون صومعه معیار داری هیچ نبود
&&
در خرابات آی تا حاصل کنی معیار داری
\\
گرچه اندر صومعه از رهبران خرقه پوشی
&&
لیکن اندر میکده زین گمرهی زنار داری
\\
تا قدم در زهد داری احولی در غیر بینی
&&
غیر بینی می‌کنی اکنون سر اغیار داری
\\
دل همی بیند که در هر ذره‌ای رویی است او را
&&
در نگر ای کوردل گر دیدهٔ دیدار داری
\\
ماه‌رویا من ندارم در دو عالم جز تو کس را
&&
تو چو من در هر حوالی عاشق بسیار داری
\\
عاشقان چون ذره بسیارند و تو چون آفتابی
&&
می‌توانی گر به لطفی جمله را تیمار داری
\\
دل به نسیه دادم از دست و ز پای افتادم از غم
&&
نقد صد جان یابم اگر یک دم سر عطار داری
\\
\end{longtable}
\end{center}
