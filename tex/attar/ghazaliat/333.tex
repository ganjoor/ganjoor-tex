\begin{center}
\section*{غزل شماره ۳۳۳: کسی کو خویش بیند بنده نبود}
\label{sec:333}
\addcontentsline{toc}{section}{\nameref{sec:333}}
\begin{longtable}{l p{0.5cm} r}
کسی کو خویش بیند بنده نبود
&&
وگر بنده بود بیننده نبود
\\
به خود زنده مباش ای بنده آخر
&&
چرا شبنم به دریا زنده نبود
\\
تو هستی شبنمی دریاب دریا
&&
که جز دریا تو را دارنده نبود
\\
درین دریا چو شبنم پاک گم شو
&&
که هر کو گم نشد داننده نبود
\\
اگر در خود بمانی ناشده گم
&&
تو را جاوید کس جوینده نبود
\\
تو می‌ترسی که در دنیا مدامت
&&
بسازی از بقا افکنده نبود
\\
وجود جاودان خواهی، ندانی
&&
که گل چون گل بسی پاینده نبود
\\
وجود گل به بالای گل آمد
&&
که سلطانی مقام بنده نبود
\\
تورا در نو شدن جامه که آرد
&&
اگر بر قد تو زیبنده نبود
\\
چه می‌گویم چو تو هستی نداری
&&
تورا جز نیستی یابنده نبود
\\
اگر خواهی که دایم هست گردی
&&
که در هستی تورا ماننده نبود
\\
فرو شو در ره معشوق جاوید
&&
که هرگز رفته‌ای آینده نبود
\\
در آتش کی رسد شمع فسرده
&&
اگر شب تا سحر سوزنده نبود
\\
فلک هرگز نگردد محرم عشق
&&
اگر سر تا قدم گردنده نبود
\\
هر آن کبکی که قوت باز گردد
&&
ورای او کسی پرنده نبود
\\
چه می‌گویی تو ای عطار آخر
&&
به عالم در چو تو گوینده نبود
\\
\end{longtable}
\end{center}
