\begin{center}
\section*{غزل شماره ۷۸۴: تا تو خود را خوارتر از جملهٔ عالم نباشی}
\label{sec:784}
\addcontentsline{toc}{section}{\nameref{sec:784}}
\begin{longtable}{l p{0.5cm} r}
تا تو خود را خوارتر از جملهٔ عالم نباشی
&&
در حریم وصل جانان یک نفس محرم نباشی
\\
عشق جانان عالمی آمد که مویی در نگنجد
&&
تا طلاق خود نگویی مرد آن عالم نباشی
\\
گر همه جایی رسیدی کی رسی هرگز به جایی
&&
تا تو اندر هرچه هستی اندر آن محکم نباشی
\\
گر نشان راه می‌خواهی نشان راه اینک
&&
کاندرین ره تا ابد در بند موج و دم نباشی
\\
گر تو مرد راه عشقی ذره‌ای باشی به صورت
&&
لیکن از راه صفت از هر دو عالم کم نباشی
\\
گر برانندت به خواری زین سبب غمگین نگردی
&&
ور بخوانندت به خواهش زین قبل خرم نباشی
\\
گر بهشت عدن بفروشی به یک گندم چون آدم
&&
هم تو از جو کمتر ارزی هم تو از آدم نباشی
\\
یک‌دم است آن دم که آن دم آدم آمد از حقیقت
&&
مرتد دین باشی ار تو محرم آن دم نباشی
\\
ذره در سایه نباشد تا نباشی تو در آن دم
&&
هم بمانی هم نمانی هم تو باشی هم نباشی
\\
کی نوازی پردهٔ عشاق چون عطار عاشق
&&
تا تو زیر پردهٔ این غم چو زیر و بم نباشی
\\
\end{longtable}
\end{center}
