\begin{center}
\section*{غزل شماره ۳۴۲: یک حاجتم ز وصل میسر نمی‌شود}
\label{sec:342}
\addcontentsline{toc}{section}{\nameref{sec:342}}
\begin{longtable}{l p{0.5cm} r}
یک حاجتم ز وصل میسر نمی‌شود
&&
یک حجتم ز عشق مقرر نمی‌شود
\\
کارم درافتاد ولیکن به یل برون
&&
کاری چنین به پهلوی لاغر نمی‌شود
\\
زین شیوه آتشی که مرا در دل اوفتاد
&&
اشکم عجب بود اگر اخگر نمی‌شود
\\
یا اشک گرمم از دم سردم فسرده شد
&&
زان خشک گشت ای عجب و تر نمی‌شود
\\
پا و سرم ز دست شد و خون دل هنوز
&&
از پای می درآیم و با سر نمی‌شود
\\
نی نی که خون دل به سر آمد ز روی من
&&
از سیل اشک سرخ مزعفر نمی‌شود
\\
چون بحر خوف موت نهنگ فلک فتاد
&&
بحری که سالکیش شناور نمی‌شود
\\
تن دردهم به قهر چو دانم که با فلک
&&
یک کارم از هزار میسر نمی‌شود
\\
صافی چه خواهم از کف ساقی چرخ از آنک
&&
صافی نمی‌دهد که مکدر نمی‌شود
\\
از جای می‌برد همه کس را فلک ولی
&&
هرگز ز جای خویش فراتر نمی‌شود
\\
گر پی کند معاینه اختر هزار را
&&
عطار یکدم از پی اختر نمی‌شود
\\
\end{longtable}
\end{center}
