\begin{center}
\section*{غزل شماره ۲۸۸: عاشقان از خویشتن بیگانه‌اند}
\label{sec:288}
\addcontentsline{toc}{section}{\nameref{sec:288}}
\begin{longtable}{l p{0.5cm} r}
عاشقان از خویشتن بیگانه‌اند
&&
وز شراب بیخودی دیوانه‌اند
\\
شاه بازان مطار قدسیند
&&
ایمن از تیمار دام و دانه‌اند
\\
فارغند از خانقاه و صومعه
&&
روز و شب در گوشهٔ میخانه‌اند
\\
گرچه مستند از شراب بیخودی
&&
بی می و بی ساقی و پیمانه‌اند
\\
در ازل بودند با روحانیان
&&
تا ابد با قدسیان هم‌خانه‌اند
\\
راه جسم و جان به یک تک می‌برند
&&
در طریقت این چنین مردانه‌اند
\\
گنج‌های مخفی‌اند این طایفه
&&
لاجرم در گلخن و ویرانه‌اند
\\
هر دو عالم پیش‌شان افسانه‌ای است
&&
در دو عالم زین قبل افسانه‌اند
\\
هر دوعالم یک صدف دان وین گروه
&&
در میان آن صدف دردانه‌اند
\\
آشنایان خودند از بیخودی
&&
وز خودی خویشتن بیگانه‌اند
\\
فارغ از کون و فساد عالمند
&&
زین جهت دیوانه و فرزانه‌اند
\\
در جهان جان چو عطارند فرد
&&
بی نیاز از خانه و کاشانه‌اند
\\
\end{longtable}
\end{center}
