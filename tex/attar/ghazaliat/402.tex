\begin{center}
\section*{غزل شماره ۴۰۲: ساقیا گه جام ده گه جام خور}
\label{sec:402}
\addcontentsline{toc}{section}{\nameref{sec:402}}
\begin{longtable}{l p{0.5cm} r}
ساقیا گه جام ده گه جام خور
&&
گر به معنی پخته‌ای می خام خور
\\
زر بده بستان می تلخ آنگهی
&&
با بت شیرین سیم‌اندام خور
\\
گردن محکم نداری پس که گفت
&&
کز زبونی سیلی ایام خور
\\
ترک نام و ننگ و صلح و جنگ گیر
&&
توبه بشکن می‌ستان و جام خور
\\
با فلک تندی مکن عطاروار
&&
باده بستان لیک با آرام خور
\\
\end{longtable}
\end{center}
