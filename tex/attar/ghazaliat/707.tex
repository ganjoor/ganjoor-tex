\begin{center}
\section*{غزل شماره ۷۰۷: صد قلزم سیماب بین بر طارم زر ریخته}
\label{sec:707}
\addcontentsline{toc}{section}{\nameref{sec:707}}
\begin{longtable}{l p{0.5cm} r}
صد قلزم سیماب بین بر طارم زر ریخته
&&
صد صحن مروارید بین بر بحر اخضر ریخته
\\
مه رخ نموده از سمک ماهی شده مه را شبک
&&
هر دم شترمرغ فلک از سینه اخگر ریخته
\\
نقش از میان اختران بگریخته چون دلبران
&&
بگسسته عقد دختران وز عقد گوهر ریخته
\\
صبح آمده با جام جم چون شیر با زرین علم
&&
در حلق صبح مشک دم صد بیضه عنبر ریخته
\\
مطرب ز بانگ ارغنون کرده حریفان را زبون
&&
ساقی ز جام لاله‌گون خون معطر ریخته
\\
چون گل بتان سیم‌بر بر کف نهاده جام زر
&&
هر دم ز لعل چون شکر صد نقل دیگر ریخته
\\
سیمین‌بران بسته میان می کرده در جام کیان
&&
پسته گشاده ساقیان وز پسته شکر ریخته
\\
هر سیم‌تن از تف می، رقاص گشته زیر خوی
&&
می مرغ جان را زیر پی، هم بال و هم پر ریخته
\\
عطار با مستان خوش صافی دل است و دردکش
&&
وز خاطر خورشید وش آب زر تر ریخته
\\
\end{longtable}
\end{center}
