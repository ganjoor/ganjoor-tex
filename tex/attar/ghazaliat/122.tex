\begin{center}
\section*{غزل شماره ۱۲۲: زهی زیبا جمالی این چه روی است}
\label{sec:122}
\addcontentsline{toc}{section}{\nameref{sec:122}}
\begin{longtable}{l p{0.5cm} r}
زهی زیبا جمالی این چه روی است
&&
زهی مشکین کمندی این چه موی است
\\
ز عشق روی و موی تو به یکبار
&&
همه کون مکان پر گفت و گوی است
\\
از آن بر خاک کویت سر نهادم
&&
که زلفت را سری بر خاک کوی است
\\
چو زلفت گر نشینم بر سر خاک
&&
نمیرم نیز و اینم آرزوی است
\\
چه جای زلف چون چوگانت آنجا
&&
که آنجا صد هزاران سر چو گوی است
\\
برو ای عاشق دستار بگریز
&&
که اینجا رستخیز از چار سوی است
\\
تو مرد نازکی آگه نه کاینجا
&&
هزارن مرد را زه در گلوی است
\\
نبینی روی او یک ذره هرگز
&&
تو را یک ذره گر در خلق روی است
\\
دلا، کی آید او در جست و جویت
&&
که او دایم ورای جست و جوی است
\\
اگرچه ذره هم جوینده باشد
&&
نه چون خورشید رنگش بر رکوی است
\\
گرت او در کشد کاری بود این
&&
که گر کار تو کار شست و شوی است
\\
بسی گر تو به جویی آب ندهد
&&
که هرچه آن از تو آید آب جوی است
\\
ز کار تو چه آید یا چه خیزد
&&
که اینجا بی نیازی سد اوی است
\\
تو کار خویش می‌کن لیک می‌دان
&&
که کار او برون از رنگ و بوی است
\\
به خود هرگز کجا داند رسیدن
&&
اگر عطار را عزم علوی است
\\
\end{longtable}
\end{center}
