\begin{center}
\section*{غزل شماره ۲۹۲: دلی کز عشق تو جان برفشاند}
\label{sec:292}
\addcontentsline{toc}{section}{\nameref{sec:292}}
\begin{longtable}{l p{0.5cm} r}
دلی کز عشق تو جان برفشاند
&&
ز کفر زلف ایمان برفشاند
\\
دلی باید که گر صد جان دهندش
&&
صد و یک جان به جانان برفشاند
\\
وگر یک ذره درد عشق یابد
&&
هزاران ساله درمان برفشاند
\\
نیارد کار خود یک لحظه پیدا
&&
ولی صد جان پنهان برفشاند
\\
اگر جان هیچ دامن گیرش آید
&&
به یک دم دامن از جان برفشاند
\\
چه می‌گویم که از یک جان چه خیزد
&&
که خواهد تا هزاران برفشاند
\\
چو دوزخ گرم گردد سوز عشقش
&&
بهشت از پیش رضوان برفشاند
\\
اگر صد گنج دارد در دل و جان
&&
ز راه چشم گریان برفشاند
\\
نه این عالم نه آن عالم گذارد
&&
که این برپا شد و آن برفشاند
\\
چو جز یک چیز مقصودش نباشد
&&
دو کون از پیش آسان برفشاند
\\
چو آن یک را بیابد گم شود پاک
&&
نماند هیچ تا آن برفشاند
\\
بغرد همچو رعدی بر سر جمع
&&
همه نقدش چو باران برفشاند
\\
چو سایه خویش را عطار اینجا
&&
بر آن خورشید رخشان برفشاند
\\
\end{longtable}
\end{center}
