\begin{center}
\section*{غزل شماره ۱۵۴: پیر ما بار دگر روی به خمار نهاد}
\label{sec:154}
\addcontentsline{toc}{section}{\nameref{sec:154}}
\begin{longtable}{l p{0.5cm} r}
پیر ما بار دگر روی به خمار نهاد
&&
خط به دین برزد و سر بر خط کفار نهاد
\\
خرقه آتش زد و در حلقهٔ دین بر سر جمع
&&
خرقهٔ سوخته در حلقهٔ زنار نهاد
\\
در بن دیر مغان در بر مشتی اوباش
&&
سر فرو برد و سر اندر پی این کار نهاد
\\
درد خمار بنوشید و دل از دست بداد
&&
می‌خوران نعره‌زنان روی به بازار نهاد
\\
گفتم ای پیر چه بود این که تو کردی آخر
&&
گفت کین داغ مرا بر دل و جان یار نهاد
\\
من چه کردم چو چنین خواست چنین باید بود
&&
گلم آن است که او در ره من خار نهاد
\\
باز گفتم که اناالحق زده‌ای سر در باز
&&
گفت آری زده‌ام روی سوی دار نهاد
\\
دل چو بشناخت که عطار درین راه بسوخت
&&
از پی پیر قدم در پی عطار نهاد
\\
\end{longtable}
\end{center}
