\begin{center}
\section*{غزل شماره ۴۳۲: درکش سر زلف دلستانش}
\label{sec:432}
\addcontentsline{toc}{section}{\nameref{sec:432}}
\begin{longtable}{l p{0.5cm} r}
درکش سر زلف دلستانش
&&
بشکن در درج درفشانش
\\
جان را به لب آر و بوسه‌ای خواه
&&
تا جانت فرو شود به جانش
\\
جانت چو به جان او فروشد
&&
بنشین به نظاره جاودانش
\\
از دیدهٔ او بدو نظر کن
&&
گر خواهی دید بس عیانش
\\
زیرا که به چشم او توان دید
&&
در آینهٔ همه جهانش
\\
زلفش که فتاده بر زمین است
&&
سرگشته نگر چو آسمانش
\\
آویخته صد هزار دل هست
&&
از یک یک موی هر زمانش
\\
گر میل تو را به سوی کفر است
&&
ره جوی به زلف دلستانش
\\
ور رغبت توست سوی ایمان
&&
بنگر رخ همچو گلستانش
\\
ور کار ز کفر و دین برون است
&&
گم گرد نه این طلب نه آنش
\\
هرگه که فرید این چنین شد
&&
هم نام مجوی و هم نشانش
\\
\end{longtable}
\end{center}
