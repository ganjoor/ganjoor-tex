\begin{center}
\section*{غزل شماره ۱: چون نیست هیچ مردی در عشق یار ما را}
\label{sec:001}
\addcontentsline{toc}{section}{\nameref{sec:001}}
\begin{longtable}{l p{0.5cm} r}
چون نیست هیچ مردی در عشق یار ما را
&&
سجاده زاهدان را درد و قمار ما را
\\
جایی که جان مردان باشد چو گوی گردان
&&
آن نیست جای رندان با آن چکار ما را
\\
گر ساقیان معنی با زاهدان نشینند
&&
می زاهدان ره را درد و خمار ما را
\\
درمانش مخلصان را دردش شکستگان را
&&
شادیش مصلحان را غم یادگار ما را
\\
ای مدعی کجایی تا ملک ما ببینی
&&
کز هرچه بود در ما برداشت یار ما را
\\
آمد خطاب ذوقی از هاتف حقیقت
&&
کای خسته چون بیابی اندوه زار ما را
\\
عطار اندرین ره اندوهگین فروشد
&&
زیرا که او تمام است انده گسار ما را
\\
\end{longtable}
\end{center}
