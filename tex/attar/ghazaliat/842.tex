\begin{center}
\section*{غزل شماره ۸۴۲: منم و گوشه‌ای و سودایی}
\label{sec:842}
\addcontentsline{toc}{section}{\nameref{sec:842}}
\begin{longtable}{l p{0.5cm} r}
منم و گوشه‌ای و سودایی
&&
تن من جایی و دلم جایی
\\
هر زمانم به عالمی میلی
&&
هر دمم سوی شیوه‌ای رایی
\\
مانده در انقلاب چون گردون
&&
گاه شیبی و گاه بالایی
\\
ساکن گوشهٔ جهان ز جهان
&&
همچو من نیست هیچ تنهایی
\\
ای عجب گرچه مانده‌ام تنها
&&
مانده‌ام در میان غوغایی
\\
رهزن من بسی شدند که من
&&
راه گم کرده‌ام به صحرایی
\\
کارم اکنون ز دست من بگذشت
&&
که در افتاده‌ام به دریایی
\\
نیست غرقه شدن درین دریا
&&
کار هر نازکی و رعنایی
\\
من سرگشته عمر خام طمع
&&
می‌پزم بر کناره سودایی
\\
مانده امروز با دلی پر خون
&&
منتظر بر امید فردایی
\\
الغیاث الغیاث زانکه ندید
&&
کس چو عطار هیچ شیدایی
\\
\end{longtable}
\end{center}
