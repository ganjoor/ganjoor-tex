\begin{center}
\section*{غزل شماره ۵۴۲: هرگاه که مست آن لقا باشم}
\label{sec:542}
\addcontentsline{toc}{section}{\nameref{sec:542}}
\begin{longtable}{l p{0.5cm} r}
هرگاه که مست آن لقا باشم
&&
هشیار جهان کبریا باشم
\\
مستغرق خویش کن مرا دایم
&&
کافسوس بود که من مرا باشم
\\
کان دم که صواب کار خود جویم
&&
آن دم بتر از بت خطا باشم
\\
گه گه گویی که دیگری را باش
&&
چون نیست به جز تو من که را باشم
\\
تا چند کنی ز پیش خود دورم
&&
تا کی ز جمال تو جدا باشم
\\
از هر سویم همی فکن هر دم
&&
مگذار که یک نفس مرا باشم
\\
گر تو بکشی چو شمع صد بارم
&&
چون آن تو کنی بدان سزا باشم
\\
صد خون دارم اگر به خون خویش
&&
در بند هزار خون بها باشم
\\
گفتم به بر من آی تا یکدم
&&
در پیش تو ذرهٔ هوا باشم
\\
گر قصد کنی به خون جان من
&&
بر کشتن خویشتن گوا باشم
\\
گفتی که چو باد و دم رسد کارت
&&
من با تو در آن دم آشنا باشم
\\
گر آن نفس آشنا شوی با من
&&
آنگاه من آن نفس کجا باشم
\\
نی نی که تو باش در بقا جمله
&&
کان اولیتر که من فنا باشم
\\
عطار اگر فنا شوم در تو
&&
گر باشم و گر نه پادشا باشم
\\
\end{longtable}
\end{center}
