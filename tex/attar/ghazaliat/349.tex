\begin{center}
\section*{غزل شماره ۳۴۹: یک شکر زان لب به صد جان می‌دهد}
\label{sec:349}
\addcontentsline{toc}{section}{\nameref{sec:349}}
\begin{longtable}{l p{0.5cm} r}
یک شکر زان لب به صد جان می‌دهد
&&
الحق ارزد زانکه ارزان می‌دهد
\\
عاشق شوریده را جان است و بس
&&
لعل او می‌بیند و جان می‌دهد
\\
قوت جان آن را که خواهد در نهان
&&
زان دو یاقوت درافشان می‌دهد
\\
شیوه‌ای دارد عجب در دلبری
&&
عشوه پیدا بوسه پنهان می‌دهد
\\
عاشق گریان خود را می‌کشد
&&
خونبها زان لعل خندان می‌دهد
\\
چشم بد را چشم او بر خاک راه
&&
می‌کشد چون باد و قربان می‌دهد
\\
گر دو چشمش می‌کشد زان باک نیست
&&
چون دو لعلش آب حیوان می‌دهد
\\
عاشقان را هر پریشانی که هست
&&
زان سر زلف پریشان می‌دهد
\\
هر زمانی عالمی سرگشته را
&&
سر سوی وادی هجران می‌دهد
\\
می‌بباید شست دست از جان خویش
&&
هین که وصلش دست آسان می‌دهد
\\
از کمال نیکویی آن تندخوی
&&
بر سپهر تند فرمان می‌دهد
\\
جان ستاند هر که از وی داد خواست
&&
داد مظلومان ازین سان می‌دهد
\\
یک سخن گفته است با عطار تلخ
&&
جان شیرین بی سخن زان می‌دهد
\\
\end{longtable}
\end{center}
