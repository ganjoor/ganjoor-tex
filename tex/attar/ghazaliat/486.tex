\begin{center}
\section*{غزل شماره ۴۸۶: دوش، چون گردون کنار خویش پر خون یافتم}
\label{sec:486}
\addcontentsline{toc}{section}{\nameref{sec:486}}
\begin{longtable}{l p{0.5cm} r}
دوش، چون گردون کنار خویش پر خون یافتم
&&
مرکز دل از محیط چرخ بیرون یافتم
\\
دیدهٔ اخترشمار من ز تیزی نظر
&&
سفت هر گوهر که در دریای گردون یافتم
\\
مردم چشمم که شبرنگش طبق می‌آورد
&&
گرم می‌تازد از آتش غرقه در خون یافتم
\\
گر طبق آورد شبرنگش بقا باد اشک را
&&
زانکه یک شبرنگ را پنجاه گلگون یافتم
\\
نیز دریا را کنار خشک نتوان یافتن
&&
زانکه چون دریا کنار از در مکنون یافتم
\\
چون برابر کردم اشک خود به دریا در شمار
&&
کژ شمردن اشک خود افزون در افزون یافتم
\\
چون هم از دل می‌کشم اشک و هم از خون جگر
&&
لاجرم این اشک دلکش را جگرگون یافتم
\\
چون بهار عمر را لیلی به کام دل نبود
&&
هر بهاری در غم لیلیش مجنون یافتم
\\
در همه عمر از فلک معجون دردی خواستم
&&
خون دل با خاک ره بنگر که معجون یافتم
\\
چون زمین پستم ز دوران بلند آسمان
&&
برج من خاکی از آن آمد که هامون یافتم
\\
چون نبود از فرق من تا خاک فرقی بیشتر
&&
خاک بر سر ریختم زین فرق کاکنون یافتم
\\
هندوی خود گیردم گردون اگر من خویش را
&&
یک نفس مقبل شدم یک لحظه میمون یافتم
\\
هندوم، زان شادکامم، بنده‌ام زان مقبلم
&&
مقبلی و شاد کامی بین کزو چون یافتم
\\
سیرم از خلقی که خون یکدگر را تشنه‌اند
&&
گر به رفعت خلق را گردان گردون یافتم
\\
تا که ساقی جهان عطار را یک درد داد
&&
صد هزاران درد با یک درد مقرون یافتم
\\
\end{longtable}
\end{center}
