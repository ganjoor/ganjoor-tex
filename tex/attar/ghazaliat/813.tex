\begin{center}
\section*{غزل شماره ۸۱۳: چارهٔ کار من آن زمان که توانی}
\label{sec:813}
\addcontentsline{toc}{section}{\nameref{sec:813}}
\begin{longtable}{l p{0.5cm} r}
چارهٔ کار من آن زمان که توانی
&&
گر بکنی راضیم چنان که توانی
\\
داد طلب کردم از تو داد ندادی
&&
گر ندهی داد می‌ستان که توانی
\\
گفته بدی من ندانم و نتوانم
&&
داد تو دادن یقین بدان که توانی
\\
گر به سر زلف دل ز من بربودی
&&
باز ده از لب هزار جان که توانی
\\
دل چه بود خود که جان اگر طلبی تو
&&
حکم کنی بر همه جهان که توانی
\\
ماه رخا پرده ز آفتاب برانداز
&&
وین همه فتنه فرو نشان که توانی
\\
جملهٔ آزادگان روی زمین را
&&
بنده کن از چشم دلستان که توانی
\\
جملهٔ دل مردگان منزل غم را
&&
زنده کن از لعل درفشان که توانی
\\
یک شکر از لعل تو اگر بربایم
&&
عذر بخواهی به هر زبان که توانی
\\
گر ز تو عطار خواست بوس و کناری
&&
هیچ منه داو در میان که توانی
\\
\end{longtable}
\end{center}
