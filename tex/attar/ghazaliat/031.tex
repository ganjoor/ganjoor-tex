\begin{center}
\section*{غزل شماره ۳۱: چون مرا مجروح کردی گر کنی مرهم رواست}
\label{sec:031}
\addcontentsline{toc}{section}{\nameref{sec:031}}
\begin{longtable}{l p{0.5cm} r}
چون مرا مجروح کردی گر کنی مرهم رواست
&&
چون بمردم ز اشتیاقت مرده را ماتم رواست
\\
من کیم یک شبنم از دریای بی‌پایان تو
&&
گر رسد بویی از آن دریا به یک شبنم رواست
\\
گر رسانی ذره‌ای شادی به جانم بی جگر
&&
هم روا باشد چو بر دل بی تو چندین غم رواست
\\
چون نیایی در میان حلقه با من چون نگین
&&
حلقه‌ای بر در زن و گر در نیایی هم رواست
\\
تا درون عالمم دم با تو نتوانم زدن
&&
چون برون آیم ز عالم با توام آن دم رواست
\\
چون در اصل کار عالم هیچکس آن برنتافت
&&
آنچنان دم کی توان گفتن که در عالم رواست
\\
در صفت رو تا بدان دم بوک یکدم پی بری
&&
کان دمی پاک است و پاک از صورت آدم رواست
\\
گر سر مویی جنب را تر نشد نامحرم است
&&
ظن مبر کاینجا سر یک موی نامحرم رواست
\\
موی چون در می‌نگنجد کرده‌ای سررشته گم
&&
گر تو گویی سوزنی با عیسی مریم رواست
\\
اره چون بر فرق خواهد داشت جم پایان کار
&&
گر فرو خواهد فتاد از دست جام جم رواست
\\
چون تواند دیو بر تخت سلیمانی نشست
&&
گر سلیمان گم کند در ملک خود خاتم رواست
\\
فقر دارد اصل محکم هرچه دیگر هیچ نیست
&&
گر قدم در فقر چون مردان کنی محکم رواست
\\
بیش از زنبیل‌بافی سلیمان نیست ملک
&&
هر که این زنبیل بفروشد به چیزی کم رواست
\\
مذهب عطار اینجا چیست از خود گم شدن
&&
زانکه اینجا نه جراحت هیچ و نه مرهم رواست
\\
\end{longtable}
\end{center}
