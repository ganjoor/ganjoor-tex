\begin{center}
\section*{غزل شماره ۶۸۳: برخاست شوری در جهان از زلف شورانگیز تو}
\label{sec:683}
\addcontentsline{toc}{section}{\nameref{sec:683}}
\begin{longtable}{l p{0.5cm} r}
برخاست شوری در جهان از زلف شورانگیز تو
&&
بس خون که از دلها بریخت آن غمزهٔ خون‌ریز تو
\\
ای زلفت از نیرنگ و فن کرده مرا بی خویشتن
&&
شد خون چشمم چشمه زن از چشم رنگ آمیز تو
\\
در راه تو از سرکشان نی یاد مانده نی نشان
&&
چون کس نماند اندر جهان تا کی بود خون‌ریز تو
\\
شد بی تو ای شمع چگل دیوانگی بر من سجل
&&
از حد گذشت ای جان و دل درد من و پرهیز تو
\\
آنها که مردان رهند از شوق تو جان می‌دهند
&&
شیران همه گردن نهند از بیم دست آویز تو
\\
از شوق روی چون مهت گردن کشان درگهت
&&
چون مرغ بسمل در رهت مست از خط نوخیز تو
\\
بی روی تو ای دل گسل درماندهٔ پایی به گل
&&
عطار شد شوریده دل از چشم شورانگیز تو
\\
\end{longtable}
\end{center}
