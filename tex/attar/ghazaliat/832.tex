\begin{center}
\section*{غزل شماره ۸۳۲: هرچه هست اوست و هرچه اوست توی}
\label{sec:832}
\addcontentsline{toc}{section}{\nameref{sec:832}}
\begin{longtable}{l p{0.5cm} r}
هرچه هست اوست و هرچه اوست توی
&&
او تویی و تو اوست نیست دوی
\\
در حقیقت چو اوست جمله تو هیچ
&&
تو مجازی دو بینی و شنوی
\\
کی رسی در وصال خود هرگز
&&
که تو پیوسته در فراق توی
\\
زان خبر نیست از توی خودت
&&
که تو تا فوق عرش تو به توی
\\
تا وجود تو کل کل نشود
&&
جزو باشی به کل کجا گروی
\\
نقطه‌ای از تو بر تو ظاهر گشت
&&
تو بدان نقطه دایما گروی
\\
نقطهٔ تو اگر به دایره رفت
&&
رو که کونین را تو پیش روی
\\
ور درین نقطه باز ماندی تو
&&
اینت سجین صعب وضیق قوی
\\
چون تو در نقطه کشته باشی تخم
&&
نه همانا که دایره دروی
\\
نتوان رست از چنان ضیقی
&&
جز به خورشید نور مصطفوی
\\
کرد عطار در علو پرواز
&&
تا بدو تافت اختر نبوی
\\
\end{longtable}
\end{center}
