\begin{center}
\section*{غزل شماره ۲۰۹: روی در زیر زلف پنهان کرد}
\label{sec:209}
\addcontentsline{toc}{section}{\nameref{sec:209}}
\begin{longtable}{l p{0.5cm} r}
روی در زیر زلف پنهان کرد
&&
تا در اسلام کافرستان کرد
\\
باز چون زلف برگرفت از روی
&&
همه کفار را مسلمان کرد
\\
دوش آمد برم سحرگاهی
&&
تا دل من به زلف پیمان کرد
\\
چون سحرگاه باد صبح بخاست
&&
حلقهٔ زلف او پریشان کرد
\\
گفتم آخر چرا چنین کردی
&&
گفت این باد کرد چتوان کرد
\\
گفتمش عهد کن به چشم این بار
&&
چشم برهم نهاد و فرمان کرد
\\
چون که پیمان ما به باد بداد
&&
باز عهدم شکست و تاوان کرد
\\
چون برفتم ز چشم، او حالی
&&
دل من برد و تیرباران کرد
\\
گفتم آخر شکست چشمت عهد
&&
گفت چشمم نکرد مژگان کرد
\\
گفتمش با لب تو عهد کنم
&&
گفت کن زانکه بوسه ارزان کرد
\\
چون ببستیم عهد لب بر لب
&&
بر لبم لعل او درافشان کرد
\\
من چو بی‌خویشتن شدم ز خوشی
&&
پاره از من بکند و پنهان کرد
\\
گفتم آخر لب تو عهد شکست
&&
گفت آن لب نکرد دندان کرد
\\
درد عطار را که درمان نیست
&&
می‌ندانم که هیچ درمان کرد
\\
\end{longtable}
\end{center}
