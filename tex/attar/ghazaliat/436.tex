\begin{center}
\section*{غزل شماره ۴۳۶: آخر ای صوفی مرقع پوش}
\label{sec:436}
\addcontentsline{toc}{section}{\nameref{sec:436}}
\begin{longtable}{l p{0.5cm} r}
آخر ای صوفی مرقع پوش
&&
لاف تقوی مزن ورع مفروش
\\
خرقهٔ مخرقه ز تن برکن
&&
دلق ازرق مرائیانه مپوش
\\
از کف ساقیان روحانی
&&
صبحدم بادهٔ صبوح بنوش
\\
صورت خویش را مکن صافی
&&
یک زمان در صفای معنی کوش
\\
سعی کن در عمارت دل و جان
&&
که نیاید به کارت این تن و توش
\\
درگذر از مزابل حیوان
&&
برگذر تا به منزلات سروش
\\
سخن عقل بر عقیله مگوی
&&
سبق عشق یک زمان کن گوش
\\
اهل قالی چو سالکان می‌گوی
&&
اهل حالی چو واصلان خاموش
\\
مرد عشقی خموش باش و خراب
&&
مرد عقلی فضول باش و به هوش
\\
روشنی بایدت چو شمع بسوز
&&
پختگی بایدت چو دیگ بجوش
\\
چون نه‌ای اهل وجد، ساکن باش
&&
از تواجد چرا شدی مدهوش
\\
راه غیر خدا مده در دل
&&
بار نفس و هوا منه بر دوش
\\
عاشقی یک دم از طلب منشین
&&
تا نگیری حریف در آغوش
\\
سخن سر به گوش دل بشنو
&&
قول عطار را به جان بنیوش
\\
پند گیرند بر تو بعد از تو
&&
گر نداری نصیحت من گوش
\\
\end{longtable}
\end{center}
