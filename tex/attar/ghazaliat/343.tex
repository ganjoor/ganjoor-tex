\begin{center}
\section*{غزل شماره ۳۴۳: ای کوی توام مقصد و ای روی تو مقصود}
\label{sec:343}
\addcontentsline{toc}{section}{\nameref{sec:343}}
\begin{longtable}{l p{0.5cm} r}
ای کوی توام مقصد و ای روی تو مقصود
&&
وی آتش عشق تو دلم سوخته چون عود
\\
چه باک اگرم عقل و دل و جان بنماند
&&
گو هیچ ممان زانکه تویی زین همه مقصود
\\
در عشق تو جانم که وجود و عدمش نیست
&&
دانی تو که چون است نه معدوم و نه موجود
\\
هر آدمیی را که کفی خاک سیاه است
&&
بی واسطه دادی تو وجودی ز سر جود
\\
چون ژنده قبایی است که آن خاص ایاز است
&&
تا چند کند سرکشی از خلعت محمود
\\
مردانه در این راه درآ ای دل غافل
&&
کز عشق نه مقبول بود مرد نه مردود
\\
چون خضر برون آی ازین سد نهادت
&&
تا باز گشایند تو را این ره مسدود
\\
هرچیز که در هر دو جهان بستهٔ آنی
&&
آن است تورا در دو جهان مونس و معبود
\\
عطار اگر سایه صفت گم شود از خود
&&
خورشید بقا تابدش از طالع مسعود
\\
\end{longtable}
\end{center}
