\begin{center}
\section*{غزل شماره ۸۹: کم شدن در کم شدن دین من است}
\label{sec:089}
\addcontentsline{toc}{section}{\nameref{sec:089}}
\begin{longtable}{l p{0.5cm} r}
کم شدن در کم شدن دین من است
&&
نیستی در هستی آیین من است
\\
حال من خود در نمی‌آید به نطق
&&
شرح حالم اشک خونین من است
\\
کار من با خلق آمد پشت و روی
&&
کافرین خلق نفرین من است
\\
تا پیاده می‌روم در کوی دوست
&&
سبز خنگ چرخ در زین من است
\\
از درش گردی که آرد باد صبح
&&
سرمهٔ چشم جهان‌بین من است
\\
چون به یک دم صد جهان از پس کنم
&&
بنگرم گام نخستین من است
\\
من چرا گرد جهان گردم چو دوست
&&
در میان جان شیرین من است
\\
ماه‌رویا عشق تو گر کافری است
&&
این چنین صد کافری دین من است
\\
گر بسوزم زآتش عشقت رواست
&&
کآتش عشق تو تسکین من است
\\
تا دل عطار خونین شد ز عشق
&&
خاک بستر خشت بالین من است
\\
\end{longtable}
\end{center}
