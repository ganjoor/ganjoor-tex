\begin{center}
\section*{غزل شماره ۲۹۸: چون تتق از روی آن شمع جهان برداشتند}
\label{sec:298}
\addcontentsline{toc}{section}{\nameref{sec:298}}
\begin{longtable}{l p{0.5cm} r}
چون تتق از روی آن شمع جهان برداشتند
&&
همچو پروانه جهانی دل ز جان برداشتند
\\
چهره‌ای دیدند جانبازان که جان درباختند
&&
بهره‌ای گویی ز عمر جاودان برداشتند
\\
چون سبک‌روحی او دیدند مخموران عشق
&&
سر به سر بر روی او رطل گران برداشتند
\\
جمله رویا روی و پشتا پشت و همدرد آمدند
&&
نعره و فریاد از هفت آسمان برداشتند
\\
چون دهان او بقدر ذره‌ای شد آشکار
&&
هر نفس صد گنج پر گوهر از آن برداشتند
\\
زلف او چون پردهٔ عشاق آمد زان خوش است
&&
گر ز زلف او نوایی هر زمان برداشتند
\\
جملهٔ ترکان ز شوق ابروی و مژگان او
&&
نیک پی بردند اگر تیر و کمان برداشتند
\\
در تعجب مانده‌ام تا عاشقان بی خبر
&&
چون نشان نیست از میانش چون نشان برداشتند
\\
وصف یک یک عضو او کردم ولیکن برکنار
&&
چون رسیدم با میانش از میان برداشتند
\\
چون ز لعلش زندگی و آب حیوان یافتند
&&
مردگان در خاک گورستان فغان برداشتند
\\
خازنان هشت جنت عاشق رویش شدند
&&
در ثنای او چو سوسن ده زبان برداشتند
\\
چون تخلص را درآمد وقت جشنی ساختند
&&
جام بر یاد خداوند جهان برداشتند
\\
چون خداوند جهان عطار خود را بنده خواند
&&
خازنان خلد دست درفشان برداشتند
\\
\end{longtable}
\end{center}
