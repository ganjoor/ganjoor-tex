\begin{center}
\section*{غزل شماره ۴۹۲: ای عشق تو پیشوای دردم}
\label{sec:492}
\addcontentsline{toc}{section}{\nameref{sec:492}}
\begin{longtable}{l p{0.5cm} r}
ای عشق تو پیشوای دردم
&&
وی درد تو هر زمان و هر دم
\\
آیینهٔ عارضت سیه شد
&&
کز حد بگذشت آه سردم
\\
یک لحظه بر من آی آخر
&&
تا کی داری ز خویش فردم
\\
تا من خط سبز تو ببینم
&&
تو درنگری به روی زردم
\\
گر کار دلم ز دست بگذشت
&&
تا در خطر هزار دردم
\\
گو بگذر از آنکه شست زلفت
&&
دست آویز است و پایمردم
\\
گفتی بگریز و ترک من گیر
&&
کاورد ز خاکی تو گردم
\\
گویی من مستمند مسکین
&&
خونی کردم که آن نکردم
\\
خونم به مریز از آنکه بس زود
&&
من بی تو بسی به خون بگردم
\\
خونم بخوری و نیست یک شب
&&
تا از تو هزار خون نخوردم
\\
کو سوخته‌تر کسی ز عطار
&&
یک سوخته نیست هم نبردم
\\
\end{longtable}
\end{center}
