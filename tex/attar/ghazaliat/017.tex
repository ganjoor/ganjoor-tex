\begin{center}
\section*{غزل شماره ۱۷: زهی ماه در مهر سرو بلندت}
\label{sec:017}
\addcontentsline{toc}{section}{\nameref{sec:017}}
\begin{longtable}{l p{0.5cm} r}
زهی ماه در مهر سرو بلندت
&&
شکر در گدازش ز تشویر قندت
\\
جهان فتنه بگرفت و پر مشک شد هم
&&
چو بگذشت بادی به مشکین کمندت
\\
سر زلف پر بند تو تا بدیدم
&&
به یک دم شدم عاشق بند بندت
\\
گزند تو را قدر و قیمت که داند
&&
بیا تا به جانم رسانی گزندت
\\
برآر از سر کبر گردی ز عالم
&&
که گوگرد سرخ است گرد سمندت
\\
به چه آلتی عشق روی تو بازم
&&
چو جان مست توست و خرد مستمندت
\\
چنان ماه رویی که آئینهٔ تو
&&
به رخ با قمر در غلط او فکندت
\\
چو وجه سپندی ندارم چه سازم
&&
جگر به که سوزم به جای سپندت
\\
مزن بانگ بر من که این است جرمم
&&
که خورشید خواندم به بانگ بلندت
\\
غلط گفتم این زانکه خورشید دایم
&&
رخی همچو زر، می‌رود مستمندت
\\
چه سازم که عطار اگر جان به زاری
&&
بسوزد ز عشقت نیاید پسندت
\\
\end{longtable}
\end{center}
