\begin{center}
\section*{غزل شماره ۱۶۳: نه به کویم گذرت می‌افتد}
\label{sec:163}
\addcontentsline{toc}{section}{\nameref{sec:163}}
\begin{longtable}{l p{0.5cm} r}
نه به کویم گذرت می‌افتد
&&
نه به رویم نظرت می‌افتد
\\
آفتابی که جهان روشن ازوست
&&
ذرهٔ خاک درت می‌افتد
\\
در طلسمات عجب موی شکاف
&&
زلف زیر و زبرت می‌افتد
\\
در جگردوزی و جان سوزی سخت
&&
چشم پر شور و شرت می‌افتد
\\
در غمت بسته کمر بر هیچی
&&
دل من چون کمرت می‌افتد
\\
آب گرمم به دهن می‌آید
&&
چشم چون بر شکرت می‌افتد
\\
شکری از تو طمع می‌دارم
&&
به بیندیش اگرت می‌افتد
\\
شکرت بی‌خطری نی و دلم
&&
به خطا در خطرت می‌افتد
\\
بیشتر میل تو جانا به جفاست
&&
یا جفا بیشترت می‌افتد
\\
گر جفایی کنی و گر نکنی
&&
نه به قصد است درت می‌افتد
\\
دل عطار ازین بیش مسوز
&&
که ازین بد بترت می‌افتد
\\
\end{longtable}
\end{center}
