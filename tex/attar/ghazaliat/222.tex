\begin{center}
\section*{غزل شماره ۲۲۲: چو به خنده لب گشایی دو جهان شکر بگیرد}
\label{sec:222}
\addcontentsline{toc}{section}{\nameref{sec:222}}
\begin{longtable}{l p{0.5cm} r}
چو به خنده لب گشایی دو جهان شکر بگیرد
&&
به نظارهٔ جمالت همه تن شکر بگیرد
\\
قدری ز نور رویت به دو عالم ار در افتد
&&
همه عرصه‌های عالم به همان قدر بگیرد
\\
چو در آرزوی رویت نفسی ز دل برآرم
&&
ز دم فسردهٔ من نفس سحر بگیرد
\\
چه غم ره است این خود که دلم دمی درین ره
&&
نه غمی دگر گزیند نه رهی دگر بگیرد
\\
اگر از عتاب غیرت ره عاشقان بگیری
&&
ز سرشک عاشقانت همه رهگذر بگیرد
\\
ز پی تو جان عطار اگر امتحان کنندش
&&
به مدیح تو دو عالم به در و گهر بگیرد
\\
\end{longtable}
\end{center}
