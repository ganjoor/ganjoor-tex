\begin{center}
\section*{غزل شماره ۸۳۸: آفتاب رویت ای سرو سهی}
\label{sec:838}
\addcontentsline{toc}{section}{\nameref{sec:838}}
\begin{longtable}{l p{0.5cm} r}
آفتاب رویت ای سرو سهی
&&
بر همه می‌تابد الا بر رهی
\\
نی خطا گفتم که می‌تابد بسی
&&
بر من و من می‌نبینم ز ابلهی
\\
گرچه عالم پر جمال یوسف است
&&
نیست چشم کور را از وی بهی
\\
چون بود کز بحر پر گوهر بسی
&&
باز گردد خشک لب دستی تهی
\\
باز گردیدند ازین بحر عجب
&&
خشک لب هم مبتدی هم منتهی
\\
قعر این دریا جزین دریا نیافت
&&
دیگران هستند از مشتی کهی
\\
حلقه بر در می‌زنند و می‌روند
&&
نیست از ایشان کسی را آگهی
\\
جمله را جز عجز آنجا کار نیست
&&
نه مهی است آنجایگاه و نه کهی
\\
می فرو افتد درین حیرت زهم
&&
گر تو اینجا دو جهان برهم نهی
\\
ای فرید اینجا که هستی محو گرد
&&
چند گویی کوتهی بر کوتهی
\\
\end{longtable}
\end{center}
