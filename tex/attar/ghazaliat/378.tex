\begin{center}
\section*{غزل شماره ۳۷۸: دل چه خواهی کرد چون دلبر رسید}
\label{sec:378}
\addcontentsline{toc}{section}{\nameref{sec:378}}
\begin{longtable}{l p{0.5cm} r}
دل چه خواهی کرد چون دلبر رسید
&&
جان برافشان هین که جان پرور رسید
\\
شربت اسرار را فردا منه
&&
زانکه تا این درکشی دیگر رسید
\\
گر سفالی یافتی در راه عشق
&&
خوش بشو انگار صد گوهر رسید
\\
خود تو آتش بر سفالی می‌نهی
&&
هین که آنجا قسم تو کمتر رسید
\\
صد هزاران موج گوناگون بخاست
&&
دانی از چه موج بحر اندر رسید
\\
چون یکی است این موج بحر مختلف
&&
از چه خاست و از خشک و تر رسید
\\
بحر کل یک جوش زد در سلطنت
&&
به یکدم صد جهان لشکر رسید
\\
چون نمی‌آید به سر زان بحر هیچ
&&
پس چرا صد چشمه چون کوثر رسید
\\
قطره چون دریاست دریا قطره هم
&&
پس چرا این کامل آن ابتر رسید
\\
قرب و بعد موج چون بسیار گشت
&&
هر زمانی اختلافی در رسید
\\
سلطنت از بحر می‌ماند به سر
&&
بحر قسم قطرهٔ مضطر رسید
\\
بی نهایت بود بحر، این اختلاف
&&
از بصر آمد نه از مبصر رسید
\\
بحر چون محوست، موجش در خطر
&&
بحر را در دیده پا و سر رسید
\\
کی بیاید بی نهایت در بصر
&&
در خطر صد با خطر مبصر رسید
\\
چون عدد در بحر رنگ بحر داشت
&&
گر رسید انگشت از اخگر رسید
\\
خوش برآمد صبح توحید از افق
&&
زانکه خورشید آمد و اختر رسید
\\
این همه اختر که شب بر آسمانست
&&
لقمه‌ای گردد چو قرص خور رسید
\\
پس یقین می‌دان که یک چیز است و بس
&&
گر هزاران مختلف هم بررسید
\\
در میان این سخن عطار را
&&
هم قلم بشکست و هم دفتر رسید
\\
\end{longtable}
\end{center}
