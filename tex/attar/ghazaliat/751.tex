\begin{center}
\section*{غزل شماره ۷۵۱: گر تو نسیمی ز زلف یار نیابی}
\label{sec:751}
\addcontentsline{toc}{section}{\nameref{sec:751}}
\begin{longtable}{l p{0.5cm} r}
گر تو نسیمی ز زلف یار نیابی
&&
تا به ابد رد شوی و بار نیابی
\\
یک دم اگر بوی زلف او به تو آید
&&
گنج حقیقت کم از هزار نیابی
\\
لیک اگر بنگری به حلقهٔ زلفش
&&
تا ابد آن حلقه را شمار نیابی
\\
هر دو جهان پرده‌ای است پیش رخ تو
&&
لیک درین پرده پود و تار نیابی
\\
حجله سرایی است پیش روی تو پرده
&&
پرده بدر گرچه پرده‌دار نیابی
\\
هرچه وجودی گرفت جمله غبار است
&&
ره به عدم بر تو تا غبار نیابی
\\
یافتن یار چیست گم شدن تو
&&
تا نشوی گم ز خویش یار نیابی
\\
غار غرور است در نهاد تو پنهان
&&
غور چنین غار آشکار نیابی
\\
گر نشوی آشنای او تو درین غار
&&
غرقه شوی بوی یار غار نیابی
\\
گر شودت ملک هر دو کون میسر
&&
بگذری از هر دو و قرار نیابی
\\
ملک غمش بهتر است از دو جهان زانک
&&
جز غم او ملک پایدار نیابی
\\
گر غم او هست ذره‌ایت مخور غم
&&
زانکه ازین به تو غمگسار نیابی
\\
هرچه که فرمود عشق رو تو به جان کن
&&
ورنه به جان هیچ زینهار نیابی
\\
می فکنی کار عشق جمله به فردا
&&
می به نترسی که روزگار نیابی
\\
پای به ره در نه و ز کار مکش سر
&&
زانکه چو شد عمر وقت کار نیابی
\\
بی‌ادب آنجا مرو وگرنه کشندت
&&
در همه عالم چو خواستگار نیابی
\\
سر چه فرازی پیاده شو ز وجودت
&&
زانکه درین راه یک سوار نیابی
\\
یک قدم این جایگاه بر نتوان داشت
&&
تا سر صد صد بزرگوار نیابی
\\
تو نتوانی که راه عشق کنی قطع
&&
کین ره جانسوز را کنار نیابی
\\
چند روی ای فرید در پی آن گل
&&
خاصه تو زان سالکی که خار نیابی
\\
\end{longtable}
\end{center}
