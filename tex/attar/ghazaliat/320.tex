\begin{center}
\section*{غزل شماره ۳۲۰: شبی کز زلف تو عالم چو شب بود}
\label{sec:320}
\addcontentsline{toc}{section}{\nameref{sec:320}}
\begin{longtable}{l p{0.5cm} r}
شبی کز زلف تو عالم چو شب بود
&&
سر مویی نه طالب نه طلب بود
\\
جهانی بود در عین عدم غرق
&&
نه اسم حزن و نه اسم طرب بود
\\
چنان در هیچ پنهان بود عالم
&&
که نه زین نام و نه زان یک لقب بود
\\
بتافت از زلف آن روی چو خورشید
&&
که گفت آن جایگه هرگز که شب بود
\\
نگارستان رویت جلوه‌ای کرد
&&
جهان گفتی که دایم بر عجب بود
\\
همی تا لعل سیرابت نمودی
&&
جهانی خلق تشنه خشک‌لب بود
\\
بتا تا چشم چون نرگس گشادی
&&
همه آفاق پر شور و شغب بود
\\
همی تا حلقه‌ای در زلف دادی
&&
سر مردان کامل در کنب بود
\\
چو از حد می‌بشد گستاخی خلق
&&
مگر اینجایگه جای ادب بود
\\
خیال نار و نور افتاده در راه
&&
حجاب و کشف جان‌ها زین سبب بود
\\
درین وادی دل عطار را هیچ
&&
نه نامی بود هرگز نه نسب بود
\\
\end{longtable}
\end{center}
