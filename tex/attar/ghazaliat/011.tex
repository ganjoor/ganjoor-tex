\begin{center}
\section*{غزل شماره ۱۱: ای عجب دردی است دل را بس عجب}
\label{sec:011}
\addcontentsline{toc}{section}{\nameref{sec:011}}
\begin{longtable}{l p{0.5cm} r}
ای عجب دردی است دل را بس عجب
&&
مانده در اندیشهٔ آن روز و شب
\\
اوفتاده در رهی بی پای و سر
&&
همچو مرغی نیم بسمل زین سبب
\\
چند باشم آخر اندر راه عشق
&&
در میان خاک و خون در تاب و تب
\\
پرده برگیرند از پیشان کار
&&
هر که دارند از نسیم او نسب
\\
ای دل شوریده عهدی کرده‌ای
&&
تازه گردان چند داری در تعب
\\
برگشادی بر دلم اسرار عشق
&&
گر نبودی در میان ترک ادب
\\
پر سخن دارم دلی لیکن چه سود
&&
چون زبانم کارگر نی ای عجب
\\
آشکارایی و پنهانی نگر
&&
دوست با ما، ما فتاده در طلب
\\
زین عجب تر کار نبود در جهان
&&
بر لب دریا بمانده خشک لب
\\
اینت کاری مشکل و راهی دراز
&&
اینت رنجی سخت و دردی بوالعجب
\\
دایم ای عطار با اندوه ساز
&&
تا ز حضرت امرت آید کالطرب
\\
\end{longtable}
\end{center}
