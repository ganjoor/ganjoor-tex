\begin{center}
\section*{غزل شماره ۲۷۹: عشق تو ز سقسین و ز بلغار برآمد}
\label{sec:279}
\addcontentsline{toc}{section}{\nameref{sec:279}}
\begin{longtable}{l p{0.5cm} r}
عشق تو ز سقسین و ز بلغار برآمد
&&
فریاد ز کفار به یک بار برآمد
\\
در صومعه‌ها نیم شبان ذکر تو می‌رفت
&&
وز لات و عزی نعرهٔ اقرار برآمد
\\
گفتم که کنم توبه در عشق ببندم
&&
تا چشم زدم عشق ز دیوار برآمد
\\
یک لحظه نقاب از رخ زیبات براندند
&&
صد دلشده را زان رخ تو کار برآمد
\\
یک زمزمه از عشق تو با چنگ بگفتم
&&
صد نالهٔ زار از دل هر تار برآمد
\\
آراسته حسن تو به بازار فروشد
&&
در حال هیاهوی ز بازار برآمد
\\
عیسی به مناجات به تسبیح خجل گشت
&&
ترسا ز چلیپا و ز زنار برآمد
\\
یوسف ز می وصل تو در چاه فروشد
&&
منصور ز شوقت به سر دار برآمد
\\
ای جان جهان هر که درین ره قدمی زد
&&
کار دو جهانیش چو عطار برآمد
\\
\end{longtable}
\end{center}
