\begin{center}
\section*{غزل شماره ۶۱۴: ای صدف لعل تو حقهٔ در یتیم}
\label{sec:614}
\addcontentsline{toc}{section}{\nameref{sec:614}}
\begin{longtable}{l p{0.5cm} r}
ای صدف لعل تو حقهٔ در یتیم
&&
عارض تو بی قلم خط زده بر لوح سیم
\\
روح دهن مانده باز در سر زلفت مدام
&&
عقل میان بسته چست بر سر کویت مقیم
\\
در یتیم توام تا که درآمد به چشم
&&
چشمهٔ چشمم بماند غرقهٔ در یتیم
\\
زین سر زلفت که هست مملکت جم توراست
&&
زانکه سر زلف توست بر صفت جیم و میم
\\
چون سر زلف تو را باد پریشان کند
&&
جیم در افتد به میم، میم درافتد به جیم
\\
تیره گلیم توام رشتهٔ صبرم متاب
&&
چند زنی بیش ازین طبل به زیر گلیم
\\
برد لب لعل تو از بر عطار دل
&&
تا دل عطار ماند چون لب تو از دو نیم
\\
\end{longtable}
\end{center}
