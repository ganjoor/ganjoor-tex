\begin{center}
\section*{غزل شماره ۷۹: درج لعلت دلگشای مردم است}
\label{sec:079}
\addcontentsline{toc}{section}{\nameref{sec:079}}
\begin{longtable}{l p{0.5cm} r}
درج لعلت دلگشای مردم است
&&
عکس ماهت رهنمای انجم است
\\
مردم چشم تو با من کژ چو باخت
&&
راستی نه مردمی نه مردم است
\\
روی تو در زلف همچون عقربت
&&
تا بدیدم چون قمر در کژدم است
\\
برنیارد خورد کس از روی تو
&&
زانکه زلفت همچو عقرب کژدم است
\\
روی چون ماهت بهشتی دیگر است
&&
لیک زلف تو درخت گندم است
\\
ایدل آنکس را که می‌جویی به جان
&&
از تو دور و با تو هم در طارم است
\\
پر ز خورشید است آفاق جهان
&&
لیک او بر آسمان چارم است
\\
جملهٔ جان‌ها مثال قطره‌هاست
&&
عالم عشقش مثال قلزم است
\\
قطره را در بحر ریزی بحر از آن
&&
نه نشان نعل و نه نقش سم است
\\
هیچ کس اندر دو عالم جان ندید
&&
زانکه جاویدان درو جان‌ها گم است
\\
گم شود در ذره‌ای اندوه عشق
&&
گر ز مشرق تا به مغرب جم جم است
\\
همچو مستان غلغلی دربسته‌ای
&&
مست گشتی می هنوز اندر خم است
\\
گم شو از خود دست از مستی بدار
&&
زانکه ره باریکتر زابریشم است
\\
این ره آنجا مر کسی را می‌دهند
&&
کز تواضع خارپشتش قاقم است
\\
هیزم عطار عود است از سخن
&&
وز عمل در بند چوبی هیزم است
\\
\end{longtable}
\end{center}
