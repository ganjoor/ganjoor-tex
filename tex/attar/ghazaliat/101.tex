\begin{center}
\section*{غزل شماره ۱۰۱: گر جمله تویی همه جهان چیست}
\label{sec:101}
\addcontentsline{toc}{section}{\nameref{sec:101}}
\begin{longtable}{l p{0.5cm} r}
گر جمله تویی همه جهان چیست
&&
ور هیچ نیم من این فغان چیست
\\
هم جمله تویی و هم همه تو
&&
و آن چیست که غیر توست آن چیست
\\
چون هست یقین که نیست جز تو
&&
آوازهٔ این همه گمان چیست
\\
چون نیست غلط کننده پیدا
&&
چندین غلط یکان یکان چیست
\\
چون کار جهان فنای محض است
&&
چندین تک و پوی در جهان چیست
\\
بر ما چو وجود نیست ما را
&&
چندین غم و درد بی کران چیست
\\
چون زنده به جان نیم به عشقم
&&
پس زحمت جان درین میان چیست
\\
جان در تو ز خویشتن فنا شد
&&
زان بی خبر است جان که جان چیست
\\
عطار ضعیف را ازین سر
&&
جز گفت میان تهی نشان چیست
\\
\end{longtable}
\end{center}
