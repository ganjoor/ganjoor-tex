\begin{center}
\section*{غزل شماره ۶۵: هر که را ذره‌ای ازین سوز است}
\label{sec:065}
\addcontentsline{toc}{section}{\nameref{sec:065}}
\begin{longtable}{l p{0.5cm} r}
هر که را ذره‌ای ازین سوز است
&&
دی و فرداش نقد امروز است
\\
هست مرد حقیقت ابن‌الوقت
&&
لاجرم بر دو کون پیروز است
\\
چون همه چیز نیست جز یک چیز
&&
پس بسی سال و ماه یک روز است
\\
صد هزاران هزار قرن گذشت
&&
لیک در اصل جمله یک سوز است
\\
چون پی یار شد چنان سوزی
&&
شب و روزش چو عید و نوروز است
\\
ذره‌ای سوز اصل می‌بینم
&&
که همه کون را جگر دوز است
\\
نیست آن سوز از کسی دیگر
&&
بل همان سوز آتش‌افروز است
\\
سوز معشوق در پس پرده
&&
عاشقان را دلیل‌آموز است
\\
هرکه او شاه‌باز این سر نیست
&&
زین طریقت جهنده چون یوز است
\\
تو اگر مردی این سخن پی بر
&&
که فرید آنچه گفت مرموز است
\\
\end{longtable}
\end{center}
