\begin{center}
\section*{غزل شماره ۶۷۱: ای روی تو آفتاب کونین}
\label{sec:671}
\addcontentsline{toc}{section}{\nameref{sec:671}}
\begin{longtable}{l p{0.5cm} r}
ای روی تو آفتاب کونین
&&
ابروی تو طاق قاب قوسین
\\
بر روی جهان ندیده چشمی
&&
نقدی روشن چو چشم تو عین
\\
جز چشمهٔ کوثر لب تو
&&
یک چشمه ندید چشم بحرین
\\
دیدم کمر تو را ز هر سوی
&&
مویی آمد میانش مابین
\\
چون تو گهری ز کان جانی
&&
جان به که کنم نه کان به میتین
\\
می‌رفت دلم به غرق تا بوک
&&
از لعل تو یک شکر کند دین
\\
زلفت چو عقاب در عقب بود
&&
بربود و کشید در عقابین
\\
گر دیدهٔ من سپید کردی
&&
خال تو بس است قرةالعین
\\
در غار غم تو جان ما را
&&
درد تو بسی است ثانی‌اثنین
\\
افکندهٔ تو شدم که شرط است
&&
القای عصا و خلع نعلین
\\
چون روی تو می‌دهد به خورشید
&&
نوری که ازوست این همه زین
\\
تا چند بر آفتاب بندی
&&
کز پرتو توست نور کونین
\\
گر جمله فروغ تو ببینیم
&&
در عین عیان ما بود شین
\\
گر در غلط اوفتاد در علم
&&
کی در غلط اوفتیم در عین
\\
عطار درین سخن برون است
&&
از مطلب کیف و مطلب این
\\
\end{longtable}
\end{center}
