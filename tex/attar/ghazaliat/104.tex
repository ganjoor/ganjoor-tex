\begin{center}
\section*{غزل شماره ۱۰۴: طریق عشق جانا بی بلا نیست}
\label{sec:104}
\addcontentsline{toc}{section}{\nameref{sec:104}}
\begin{longtable}{l p{0.5cm} r}
طریق عشق جانا بی بلا نیست
&&
زمانی بی بلا بودن روا نیست
\\
اگر صد تیر بر جان تو آید
&&
چو تیر از شست او باشد خطا نیست
\\
از آنجا هرچه آید راست آید
&&
تو کژمنگر که کژ دیدن روا نیست
\\
سر مویی نمی‌دانی ازین سر
&&
تو را گر در سر مویی رضا نیست
\\
بلاکش، تا لقای دوست بینی
&&
که مرد بی بلا مرد لقا نیست
\\
میان صد بلا خوش باش با او
&&
خود آنجا کو بود هرگز بلا نیست
\\
کسی کو روز و شب خوش نیست با او
&&
شبش خوش باد کانکس مرد ما نیست
\\
که باشی تو که او خون تو ریزد
&&
وگر ریزد جز اینت خون‌بها نیست
\\
دوای جان مجوی و تن فرو ده
&&
که درد عشق را هرگز دوا نیست
\\
درین دریای بی پایان کسی را
&&
سر مویی امید آشنا نیست
\\
تو از دریا جدایی و عجب این
&&
که این دریا ز تو یکدم جدا نیست
\\
تو او را حاصلی و او تورا گم
&&
تو او را هستی اما او تورا نیست
\\
خیال کژ مبر اینجا و بشناس
&&
که هر کو در خدا گم شد خدا نیست
\\
ولی روی بقا هرگز نبینی
&&
که تا ز اول نگردی از فنا نیست
\\
چو تو در وی فنا گردی به کلی
&&
تو را دایم ورای این بقا نیست
\\
ز حیرت چون دل عطار امروز
&&
درین گرداب خون یک مبتلا نیست
\\
\end{longtable}
\end{center}
