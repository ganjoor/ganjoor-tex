\begin{center}
\section*{غزل شماره ۶۶۲: لعل تو داغی نهاد بر دل بریان من}
\label{sec:662}
\addcontentsline{toc}{section}{\nameref{sec:662}}
\begin{longtable}{l p{0.5cm} r}
لعل تو داغی نهاد بر دل بریان من
&&
زلف تو درهم شکست توبه و پیمان من
\\
بی تو دل و جان من سیر شد از جان و دل
&&
جان و دل من تویی ای دل و ای جان من
\\
چون گهر اشک من راه نظر چست بست
&&
چون نگرد در رخت دیدهٔ گریان من
\\
هر در عشقت که دل داشت نهان از جهان
&&
بر رخ زردم فشاند اشک درافشان من
\\
شد دل بیچاره خون، چارهٔ دل هم تو ساز
&&
زانکه تو دانی که چیست بر دل بریان من
\\
گر تو نگیریم دست کار من از دست شد
&&
زانکه ندارد کران، وادی هجران من
\\
هم نظری کن ز لطف تا دل درمانده را
&&
بو که به پایان رسد راه بیابان من
\\
هست دل عاشقت منتظر یک نظر
&&
تا که برآید ز تو حاجت دو جهان من
\\
تو دل عطار را سوختهٔ خویش‌دار
&&
زانکه دل سنگ سوخت از دل سوزان من
\\
\end{longtable}
\end{center}
