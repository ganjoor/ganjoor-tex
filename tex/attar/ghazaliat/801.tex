\begin{center}
\section*{غزل شماره ۸۰۱: ای هرشکنی از سر زلف تو جهانی}
\label{sec:801}
\addcontentsline{toc}{section}{\nameref{sec:801}}
\begin{longtable}{l p{0.5cm} r}
ای هرشکنی از سر زلف تو جهانی
&&
وی هر سخنی از لب جان‌بخش تو جانی
\\
نه هیچ فلک دید چو تو بدر منیری
&&
نه هیچ چمن یافت چو تو سرو روانی
\\
خورشید که بسیار بگشت از همه سویی
&&
یک ذره ندیده است ز وصل تو نشانی
\\
یک ذره اگر شمع وصال تو بتابد
&&
جان بر تو فشانند چو پروانه جهانی
\\
زابروی هلالیت که طاق است چو گردون
&&
با پشت دو تا مانده هرجا که کمانی
\\
چون دایره بی پا و سرم زانکه تو داری
&&
از دایرهٔ ماه رخ از نقطه دهانی
\\
ارباب یقین ده یک‌یک ذره گرفتند
&&
شکل دهن تنگ تو از روی گمانی
\\
حرف کمرت همچو الف هیچ ندارد
&&
زیرا که تو را چون الف افتاد میانی
\\
مویی ز میان تو کسی می بنداند
&&
گرچه بود آن کس به حقیقت همه دانی
\\
در عشق تو کار همه عشاق برآمد
&&
زیرا که خریدند به صد سود و زیانی
\\
چون لاله دلم سوخته‌تن غرقهٔ خون است
&&
تا یافته‌ام گرد رخت لاله ستانی
\\
چون حال من سوخته دل تنگ درآمد
&&
از جان رمقی مانده مرا باش زمانی
\\
عطار جگر سوخته را بود دل تنگ
&&
دل در سر کار تو شد او مانده زمانی
\\
\end{longtable}
\end{center}
