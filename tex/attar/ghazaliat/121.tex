\begin{center}
\section*{غزل شماره ۱۲۱: آیینهٔ تو سیاه رویی است}
\label{sec:121}
\addcontentsline{toc}{section}{\nameref{sec:121}}
\begin{longtable}{l p{0.5cm} r}
آیینهٔ تو سیاه رویی است
&&
او را چه خبر که ماه‌روی است
\\
آن آینه می‌زدای پیوست
&&
کورا گه پشت و گاه روی است
\\
آن پشت ز عشق روی گردان
&&
گر کرده تو را به راه روی است
\\
کز عشق چو آفتاب گردد
&&
هر ذره اگر سیاه‌روی است
\\
نه چرخ کلاه فرق عشق است
&&
پس در خور آن کلاه‌روی است
\\
تا این رویش نگردد آن روی
&&
او را همه در گناه روی است
\\
هر ذره که هست در دو عالم
&&
او را سوی پیشگاه روی است
\\
نتواند یافت هرگز این روی
&&
آن را که به عز و جاه روی است
\\
هرگز نرسد به ذروهٔ عرش
&&
آن را که به قعر چاه روی است
\\
روی از همه شیوه بست باید
&&
آن را که به پادشاه روی است
\\
زین شوق فرید را همه عمر
&&
آورده به بارگاه روی است
\\
\end{longtable}
\end{center}
