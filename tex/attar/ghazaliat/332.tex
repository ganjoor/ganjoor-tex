\begin{center}
\section*{غزل شماره ۳۳۲: چو در غم تو جز جان چیزی دگرم نبود}
\label{sec:332}
\addcontentsline{toc}{section}{\nameref{sec:332}}
\begin{longtable}{l p{0.5cm} r}
چو در غم تو جز جان چیزی دگرم نبود
&&
پیش تو کشم کز تو غمخوارترم نبود
\\
پروانه تو گشتم تا بر تو سرافشانم
&&
خود چون رخ تو بینم پروای سرم نبود
\\
پیش نظرم عالم چون روز قیامت باد
&&
آن روز که بر راهت دایم نظرم نبود
\\
گفتم خبری گویم با تو ز دل زارم
&&
اما چو تو را بینم از خود خبرم نبود
\\
گفتم که ز تیرت تیز از چشم تو بگریزم
&&
چون تیر بپیوندد کنج گذرم نبود
\\
در عشق تو صد همدم تیمار برم باید
&&
تنها چکنم چون کس تیمار برم نبود
\\
گفتی که به زر گردد کار تو چو آب زر
&&
جانی بکنم آخر گر آن قدرم نبود
\\
تو چاره کارم کن تا از رخ همچون زر
&&
تدبیر کنم وجهی گر هیچ زرم نبود
\\
بوسی ندهی جانا تا جان نستانی تو
&&
هر دم ز پی بوسی جانی دگرم نبود
\\
عطار ستمکش را دل بود به تو رهبر
&&
دردا که چو دل خون شد کس راهبرم نبود
\\
\end{longtable}
\end{center}
