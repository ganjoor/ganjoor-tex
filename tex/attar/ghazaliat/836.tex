\begin{center}
\section*{غزل شماره ۸۳۶: نگاری مست لایعقل چو ماهی}
\label{sec:836}
\addcontentsline{toc}{section}{\nameref{sec:836}}
\begin{longtable}{l p{0.5cm} r}
نگاری مست لایعقل چو ماهی
&&
درآمد از در مسجد پگاهی
\\
سیه زلف و سیه چشم و سیه دل
&&
سیه گر بود و پوشیده سیاهی
\\
ز هر مویی که اندر زلف او بود
&&
فرو می‌ریخت کفری و گناهی
\\
درآمد پیش پیر ما به زانو
&&
بدو گفت ای اسیر آب و جاهی
\\
فسردی همچو یخ از زهد کردن
&&
بسوز آخر چو آتش گاهگاهی
\\
چو پیر ما بدید او را برآورد
&&
ز جان آتشین چون آتش آهی
\\
ز راه افتاد و روی آورد در کفر
&&
نه رویی ماند در دین و نه راهی
\\
به تاریکی زلف او فرو ریخت
&&
به دست آورد از آب خضر چاهی
\\
دگر هرگز نشان او ندیدم
&&
که شد در بی نشانی پادشاهی
\\
اگر عطار با او هم برفتی
&&
نیرزیدش عالم برگ کاهی
\\
\end{longtable}
\end{center}
