\begin{center}
\section*{غزل شماره ۳۱۸: قومی که در فنا به دل یکدگر زیند}
\label{sec:318}
\addcontentsline{toc}{section}{\nameref{sec:318}}
\begin{longtable}{l p{0.5cm} r}
قومی که در فنا به دل یکدگر زیند
&&
روزی هزار بار بمیرند و بر زیند
\\
هر لحظه‌شان ز هجر به دردی دگر کشند
&&
تا هر نفس ز وصل به جانی دگر زیند
\\
در راه نه به بال و پر خویشتن پرند
&&
در عشق نه به جان و دل مختصر زیند
\\
مانند گوی در خم چوگان حکم او
&&
در خاک راه مانده و بی پا و سر زیند
\\
از زندگی خویش بمیرند همچو شمع
&&
پس همچو شمع زندهٔ بی خواب و خور زیند
\\
عود و شکر چگونه بسوزند وقت سوز
&&
ایشان درین طریق چو عود و شکر زیند
\\
چون ذرهٔ هوا سر و پا جمله گم کنند
&&
گر در هوای او نفسی بی خطر زیند
\\
فانی شوند و باقی مطلق شوند باز
&&
وانگه ازین دو پرده برون پرده‌در زیند
\\
چون زندگی ز مردگی خویش یافتند
&&
چون مرده‌تر شوند بسی زنده‌تر زیند
\\
خورشید وحدتند ولی در مقام فقر
&&
در پیش ذره‌ای همه دریوزه‌گر زیند
\\
چون آفتاب اگرچه بلندند در صفت
&&
چون سایهٔ فتادهٔ از در بدر زیند
\\
چون با خبر شوند ز یک موی زلف دوست
&&
چون موی از وجود و عدم بی خبر زیند
\\
ذرات جمله‌شان همه چشم است و گوش هم
&&
ویشان بر آستان ادب کور و کر زیند
\\
عطار چون ز سایهٔ ایشان برد حیات
&&
ایشان ز لطف بر سر او سایه‌ور زیند
\\
\end{longtable}
\end{center}
