\begin{center}
\section*{غزل شماره ۶۵۷: چو دریا شور در جانم میفکن}
\label{sec:657}
\addcontentsline{toc}{section}{\nameref{sec:657}}
\begin{longtable}{l p{0.5cm} r}
چو دریا شور در جانم میفکن
&&
ز سودا در بیابانم میفکن
\\
چو پر پشهٔ وصلت ندیدم
&&
به پای پیل هجرانم میفکن
\\
به دست خویش در پای خودم کش
&&
به دست و پای دورانم میفکن
\\
به دشواری به دست آید چو من کس
&&
چنین از دست آسانم میفکن
\\
اگر از تشنگی چون شمع مردم
&&
به سیرابی طوفانم میفکن
\\
به چشم او کز ابروی کمان کش
&&
به دل در تیر مژگانم میفکن
\\
زره چون در نمی‌پوشیم از زلف
&&
میان تیربارانم میفکن
\\
چو پیچ و تاب در زلف تو زیباست
&&
به جان تو که در جانم میفکن
\\
چو پایم نیست با چوگان زلفت
&&
چو گویی پیش چوگانم میفکن
\\
چو من جمعیت از زلف تو دارم
&&
چو زلف خود پریشانم میفکن
\\
خط آوردی و جان می‌خواهی از من
&&
ز خط خود به دیوانم میفکن
\\
چو شد خاک رهت عطار حیران
&&
به خاک راه حیرانم میفکن
\\
\end{longtable}
\end{center}
