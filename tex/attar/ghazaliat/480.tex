\begin{center}
\section*{غزل شماره ۴۸۰: دی در صف اوباش زمانی بنشستم}
\label{sec:480}
\addcontentsline{toc}{section}{\nameref{sec:480}}
\begin{longtable}{l p{0.5cm} r}
دی در صف اوباش زمانی بنشستم
&&
قلاش و قلندر شدم و توبه شکستم
\\
جاروب خرابات شد این خرقهٔ سالوس
&&
از دلق برون آمدم از زرق برستم
\\
از صومعه با میکده افتاد مرا کار
&&
می‌دادم و می‌خوردم و بی می ننشستم
\\
چون صومعه و میکده را اصل یکی بود
&&
تسبیح بیفکندم و زنار ببستم
\\
در صومعه صوفی چه شوی منکر حالم
&&
معذور بدار ار غلطی رفت که مستم
\\
سرمست چنانم که سر از پای ندانم
&&
از باده که خوردم خبرم نیست که هستم
\\
یک جرعه از آن باده اگر نوش کنی تو
&&
عیبم نکنی باز اگر باده پرستم
\\
اکنون که مرا کار شد از دست، چه تدبیر
&&
تقدیر چنین بود و قضا نیست به دستم
\\
عطار درین راه قدم زن چه زنی دم
&&
تا چند زنی لاف که من مست الستم
\\
\end{longtable}
\end{center}
