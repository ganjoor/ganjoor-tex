\begin{center}
\section*{غزل شماره ۵۶۸: قصهٔ عشق تو از بر چون کنم}
\label{sec:568}
\addcontentsline{toc}{section}{\nameref{sec:568}}
\begin{longtable}{l p{0.5cm} r}
قصهٔ عشق تو از بر چون کنم
&&
وصل را از وعده باور چون کنم
\\
جان ندارم، بار جانان چون کشم
&&
دل ندارم، قصد دلبر چون کنم
\\
حلقهٔ زلف توام چون بند کرد
&&
مانده‌ام چون حلقه بر در چون کنم
\\
چون تو خورشیدی و من چون سایه‌ام
&&
خویش را با تو برابر چون کنم
\\
گفته‌ای تو پای سر کن در رهم
&&
می ندانم پای از سر چون کنم
\\
گفته بودی عزم من کن مردوار
&&
برده‌ام صد بار کیفر چون کنم
\\
عزم کردم وصل تو جانم بسوخت
&&
مانده‌ام بی عزم مضطر چون کنم
\\
چون ندارد ذره‌ای وصل تو روی
&&
وصل روی تو میسر چون کنم
\\
کشتی عمرم به غرقاب اوفتاد
&&
مفلسم از صبر لنگر چون کنم
\\
چشم بگشادم که بینم روی تو
&&
گشت چشمم غرق گوهر چون کنم
\\
لب گشادم تا کنم وصف تو شرح
&&
نیست آن کار سخنور چون کنم
\\
گفته‌ای بردوز چشم و لب ببند
&&
چون نه خشکم ماند و نه تر چون کنم
\\
روح می‌خواهی برای یک شکر
&&
آن عوض با این محقر چون کنم
\\
گفته‌ام صد باره ترک روح خویش
&&
چون تو هستی روح پرور چون کنم
\\
چون به یک دستم همی داری نگاه
&&
می‌زیم از دست دیگر چون کنم
\\
هرگز از عطار حرفی نشنوی
&&
قصه‌ای با تو مقرر چون کنم
\\
\end{longtable}
\end{center}
