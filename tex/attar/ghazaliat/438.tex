\begin{center}
\section*{غزل شماره ۴۳۸: مست شدم تا به خرابات دوش}
\label{sec:438}
\addcontentsline{toc}{section}{\nameref{sec:438}}
\begin{longtable}{l p{0.5cm} r}
مست شدم تا به خرابات دوش
&&
نعره‌زنان رقص‌کنان دردنوش
\\
جوش دلم چون به سر خم رسید
&&
زآتش جوش دلم آمد به جوش
\\
پیر خرابات چو بانگم شنید
&&
گفت درآی ای پسر خرقه‌پوش
\\
گفتمش ای پیر چه دانی مرا
&&
گفت ز خود هیچ مگو شو خموش
\\
مذهب رندان خرابات گیر
&&
خرقه و سجاده بیفکن ز دوش
\\
کم زن و قلاش و قلندر بباش
&&
در صف اوباش برآور خروش
\\
صافی زهاد به خواری بریز
&&
دردی عشاق به شادی بنوش
\\
صورت تشبیه برون بر ز چشم
&&
پنبهٔ پندار برآور ز گوش
\\
تو تو نه‌ای چند نشینی به خود
&&
پردهٔ تو بردر و با خود بکوش
\\
قعر دلت عالم بی‌منتهاست
&&
رخت سوی عالم دل بر بهوش
\\
گوهر عطار به صد جان بخر
&&
چند بود پیش تو گوهر فروش
\\
\end{longtable}
\end{center}
