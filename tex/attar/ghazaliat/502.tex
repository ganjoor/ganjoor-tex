\begin{center}
\section*{غزل شماره ۵۰۲: رفتم به زیر پرده و بیرون نیامدم}
\label{sec:502}
\addcontentsline{toc}{section}{\nameref{sec:502}}
\begin{longtable}{l p{0.5cm} r}
رفتم به زیر پرده و بیرون نیامدم
&&
تا صید پرده‌بازی گردون نیامدم
\\
چون قطب ساکن آمدم اندر مقام فقر
&&
هر لحظه همچو چرخ دگرگون نیامدم
\\
بنهاده‌ام قدم به حرمگاه فقر در
&&
تا هرچه بود از همه بیرون نیامدم
\\
زر همچو گل ز صره از آن ریختم به خاک
&&
تا همچو غنچه با دل پر خون نیامدم
\\
از اهل روزگار به معیار امتحان
&&
کم نیستم به هیچ، گر افزون نیامدم
\\
همچون مگس به ریزهٔ کس ننگریستم
&&
هر چند چون همای همایون نیامدم
\\
منت خدای را که اگر بود و گر نبود
&&
در زیر بار منت هر دون نیامدم
\\
هر بی خبر برون درست از وجود من
&&
آخر من از عدم به شبیخون نیامدم
\\
عطار پر به سوی فلک همچو جبرئیل
&&
راه زمین مرو که چو قارون نیامدم
\\
\end{longtable}
\end{center}
