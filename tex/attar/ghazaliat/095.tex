\begin{center}
\section*{غزل شماره ۹۵: ای به وصفت گمشده هرجان که هست}
\label{sec:095}
\addcontentsline{toc}{section}{\nameref{sec:095}}
\begin{longtable}{l p{0.5cm} r}
ای به وصفت گمشده هرجان که هست
&&
جان تنها نه خرد چندان که هست
\\
وی کمال آفتاب روی تو
&&
تا ابد فارغ ز هر نقصان که هست
\\
گر سکندر چشمهٔ حیوان نیافت
&&
نیست عیب چشمهٔ حیوان که هست
\\
کور مادرزاد آید کل خلق
&&
در بر آن حسن جاویدان که هست
\\
صد هزاران قرن چرخ تیزرو
&&
بود هم زین شیوه سرگردان که هست
\\
از شفق در خون بسی گشت و نیافت
&&
چون تو خورشیدی درین دوران که هست
\\
آفتاب از شرم رویت هر شبی
&&
در سیاهی شد چنین پنهان که هست
\\
باز چون زلفت کمند او شود
&&
بی سر و بن می‌رود زین سان که هست
\\
نی چه می‌گویم فلک گویی است بس
&&
در خم آن زلف چو چوگان که هست
\\
هیچ سر بر تن نخواهد ماند از انک
&&
گوی خواهد شد درین میدان که هست
\\
زاشتیاق روی چون خورشید توست
&&
ابر را هر دیدهٔ گریان که هست
\\
وی عجب در جنب عشق عاشقانت
&&
شبنمی است این جملهٔ باران که هست
\\
ابر چبود زانکه صد دریای خون
&&
از دل هر یک درین طوفان که هست
\\
هرچه از ما می‌رود آن هیچ نیست
&&
کار تا چون رفت از آن پیشان که هست
\\
کار تنها نه مرا افتاد و بس
&&
همچو من بس بی سر و سامان که هست
\\
تو چنین در پرده و از شور توست
&&
در دو عالم این همه حیران که هست
\\
جملهٔ ذرات عالم گوش شد
&&
تا بفرمایی تو هر فرمان که هست
\\
گرد نعلین گدای کوی تو
&&
بیشتر از ملک هر سلطان که هست
\\
دوست‌تر دارم من آشفته دل
&&
ذره‌ای دردت ز هر درمان که هست
\\
همدم عیسی شود بی شک فرید
&&
گر دمی برهد ازین زندان که هست
\\
\end{longtable}
\end{center}
