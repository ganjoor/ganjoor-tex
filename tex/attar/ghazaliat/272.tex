\begin{center}
\section*{غزل شماره ۲۷۲: در قعر جان مستم دردی پدید آمد}
\label{sec:272}
\addcontentsline{toc}{section}{\nameref{sec:272}}
\begin{longtable}{l p{0.5cm} r}
در قعر جان مستم دردی پدید آمد
&&
کان درد بندیان را دایم کلید آمد
\\
چندان درین بیابان رفتم که گم شدستم
&&
هرگز کسی ندیدم کانجا پدید آمد
\\
مردان این سفر را گم‌بودگی است حاصل
&&
وین منکران ره را گفت و شنید آمد
\\
گر مست این حدیثی ایمان تو راست لایق
&&
زیرا که کافر اینجا مست نبید آمد
\\
شد مست مغز جانم از بوی باده زیرا
&&
جام محبت او با بوسعید آمد
\\
تا داده‌اند بویی عطار را ازین می
&&
عمرش درازتر شد عیشش لذیذ آمد
\\
\end{longtable}
\end{center}
