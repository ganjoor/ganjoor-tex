\begin{center}
\section*{غزل شماره ۷۲۶: در راه تو مردانند از خویش نهان مانده}
\label{sec:726}
\addcontentsline{toc}{section}{\nameref{sec:726}}
\begin{longtable}{l p{0.5cm} r}
در راه تو مردانند از خویش نهان مانده
&&
بی جسم و جهت گشته بی نام و نشان مانده
\\
در قبهٔ متواری لایعرفهم غیری
&&
محبوب ازل بوده محجوب جهان مانده
\\
در کسوت کادالفقر از کفر زده خیمه
&&
در زیر سوادالوجه از خلق نهان مانده
\\
قومی نه نکو نه بد نه با خود و نه بیخود
&&
نه بوده و نه نابوده نی مانده عیان مانده
\\
در عالم ما و من نی ما شده و نی من
&&
در کون و مکان با تو بی کون و مکان مانده
\\
جانشان به حقیقت کل تنشان به شریعت هم
&&
هم جان همه و هم تن نی این و نه آن مانده
\\
چون دایره سرگردان چون نقطه قدم محکم
&&
صد دایره عرش آسا در نقطهٔ جان مانده
\\
چون عین بقا دیده از خویش فنا گشته
&&
در بحر یقین غرقه در تیه گمان مانده
\\
فارش از سر هر مویی صد گونه سخن گفته
&&
اما همه از گنگی بی کام و زبان مانده
\\
جمله ز گران عقلی در سیر سبک بوده
&&
وآنگه ز سبک روحی در بار گران مانده
\\
صد عالم بی پایان از خوف و رجا بیرون
&&
از خوف شده مویی در خط امان مانده
\\
بشکسته دلیران را از چست سواری پشت
&&
مرکب شده ناپیدا در دست عنان مانده
\\
بفروخته از همت دو کون به یک نان خوش
&&
وز ناخوشی عالم وقوف دو نان مانده
\\
آن کس که نزاد است او از مادر خود هرگز
&&
ایشان همه هم با تو از فقر چنان مانده
\\
تا راه چنین قومی عطار بیان کرده
&&
جانش به لب افتاده دل در خفقان مانده
\\
\end{longtable}
\end{center}
