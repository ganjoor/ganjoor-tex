\begin{center}
\section*{غزل شماره ۳۹۸: نیست مرا به هیچ رو، بی تو قرار ای پسر}
\label{sec:398}
\addcontentsline{toc}{section}{\nameref{sec:398}}
\begin{longtable}{l p{0.5cm} r}
نیست مرا به هیچ رو، بی تو قرار ای پسر
&&
بی تو به سر نمی‌شود، زین همه کار ای پسر
\\
صبح دمید و گل شکفت، از پی عیش دم به دم
&&
چنگ بساز ای صنم، باده بیار ای پسر
\\
تا که ازین خمار غم، خون جگر بود مرا
&&
هین بشکن ز خون خم، رنج خمار ای پسر
\\
چند غم جهان خورم، چون نیم اهل این جهان
&&
باده بیار تا کنم، زود گذار ای پسر
\\
من چو به ترک نام و ننگ، از دل جان بگفته‌ام
&&
چند به زهد خوانیم، دست بدار ای پسر
\\
چون به شمار کس نیم، سر به هوا برآورم
&&
تا نکنندم از جهان، هیچ شمار ای پسر
\\
نیست مرا ز هیچکس، هیبت نیم جو ز من
&&
هست مرا یکی شده، منبر و دار ای پسر
\\
جان فرید از نفاق، ننگ به نام خلق شد
&&
پس تو ز شرح حال خود، ننگ مدار ای پسر
\\
\end{longtable}
\end{center}
