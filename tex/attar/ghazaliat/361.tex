\begin{center}
\section*{غزل شماره ۳۶۱: آن ماه برای کس نمی‌آید}
\label{sec:361}
\addcontentsline{toc}{section}{\nameref{sec:361}}
\begin{longtable}{l p{0.5cm} r}
آن ماه برای کس نمی‌آید
&&
کو با غم خویش بس نمی‌آید
\\
در آینه روی خویش می‌بیند
&&
در دام هوای کس نمی‌آید
\\
گر تو به هوس جمال او خواهی
&&
او در طلب و هوس نمی‌آید
\\
جانا ره عشق چون تو معشوقی
&&
در زیر تک فرس نمی‌آید
\\
در وادی بی‌نهایت عشقش
&&
سیمرغ به یک مگس نمی‌آید
\\
هرگز نشوی تو هم نفس کس را
&&
کانجا که تویی نفس نمی‌آید
\\
خورشید بلند را چه کم بیشی
&&
کش سایه ز پیش و پس نمی‌آید
\\
چون در قعر است در وصل تو
&&
جز بر سر آب خس نمی‌آید
\\
در پای فراق تو شوم پامال
&&
چون وصل تو دسترس نمی‌آید
\\
عطار که چینهٔ تو می‌چیند
&&
مرغی است که در قفس نمی‌آید
\\
\end{longtable}
\end{center}
