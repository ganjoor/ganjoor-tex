\begin{center}
\section*{غزل شماره ۲۴: دولت عاشقان هوای تو است}
\label{sec:024}
\addcontentsline{toc}{section}{\nameref{sec:024}}
\begin{longtable}{l p{0.5cm} r}
دولت عاشقان هوای تو است
&&
راحت طالبان بلای تو است
\\
کیمیای سعادت دو جهان
&&
گرد خاک در سرای تو است
\\
ناف آهو شود دهان کسی
&&
که درو وصف کبریای تو است
\\
سرمهٔ دیده‌ها بود خاکی
&&
که گذرگاه آشنای تو است
\\
ملک عالم به هیچ نشمارد
&&
آنکه در کوی تو گدای تو است
\\
به سحر ناز عاشقان با تو
&&
از سر لطف دلگشای تو است
\\
آنچه از ملک جاودان بیش است
&&
عاشقان را در سرای تو است
\\
آنچه از سیرت ملوک به است
&&
خاک کوی فلک‌نمای تو است
\\
از بلا هر کسی گریزان است
&&
این رهی طالب بلای تو است
\\
گر رضای تو در بلای من است
&&
جان من بستهٔ رضای تو است
\\
من ندانم ثنای تو به سزا
&&
وصف تو لایق ثنای تو است
\\
این تکاپوی و گفت و گوی فرید
&&
همه در جستن عطای تو است
\\
\end{longtable}
\end{center}
