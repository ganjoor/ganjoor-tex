\begin{center}
\section*{غزل شماره ۴۵۱: ای عقل گرفته از رخت فال}
\label{sec:451}
\addcontentsline{toc}{section}{\nameref{sec:451}}
\begin{longtable}{l p{0.5cm} r}
ای عقل گرفته از رخت فال
&&
بر زلف تو وقف جان ابدال
\\
از زلف تو حل نمی‌توان کرد
&&
یک شکل ز صد هزار اشکال
\\
شرح سر زلف تو دهم من
&&
هرگه که شوم به صد زبان لال
\\
ای در ره حل و عقد عشقت
&&
پیران هزار ساله اطفال
\\
در معرکهٔ تو شیرمردان
&&
بر ریگ همی زنند دنبال
\\
کردی ظلمات و آب حیوان
&&
معروف هم از لب و هم از خال
\\
در یوسف مصر کس ندیده است
&&
آن لطف که در تو بینم امسال
\\
سربسته از آن بگفتم این حرف
&&
تا بو که حلولیی کند حال
\\
اینجا که منم حلول نبود
&&
استغراق است و کشف احوال
\\
دل خون شد و زاد ره ندارم
&&
وقت است که جان دهم به دلال
\\
از هر مژه هر زمان ز شوقت
&&
می‌بگشایم هزار قیفال
\\
بگشای به نیستیم راهی
&&
تا در زنم آتشی به اعمال
\\
مرغ تو منم که تا که هستم
&&
در عشق تو می‌زنم پر و بال
\\
صد کوه به یک زمان ببخشی
&&
وانگاه بگیریم به مثقال
\\
از خرقهٔ هستیم برون آر
&&
تا خرقه درافکنم به قوال
\\
چون برهنگان بی سر و پای
&&
بگریزم ازین جهان محتال
\\
چند از متکلمان بارد
&&
وز فلسفیان عقل فعال
\\
هم فلسفه هم کلام بگذار
&&
از بهر فضولیان دخال
\\
با عیسی روح هم نفس شو
&&
بگذار جدل برای دجال
\\
در عشق گریز همچو عطار
&&
تا باز رهی ز جاه و از مال
\\
\end{longtable}
\end{center}
