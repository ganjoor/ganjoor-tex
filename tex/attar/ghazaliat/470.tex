\begin{center}
\section*{غزل شماره ۴۷۰: بی دل و بی قراری مانده‌ام}
\label{sec:470}
\addcontentsline{toc}{section}{\nameref{sec:470}}
\begin{longtable}{l p{0.5cm} r}
بی دل و بی قراری مانده‌ام
&&
زانکه در بند نگاری مانده‌ام
\\
دلخوشی با دلگشایی بوده‌ام
&&
غم کشی بی غمگساری مانده‌ام
\\
زیر بار عشق او کارم فتاد
&&
لاجرم بی کار و باری مانده‌ام
\\
در میانم با غم عشقش چو شمع
&&
گرچه چون اشک از کناری مانده‌ام
\\
گرچه وصل او محالی واجب است
&&
من مدام امیدواری مانده‌ام
\\
بی گل رویش در ایام بهار
&&
چون بنفشه سوکواری مانده‌ام
\\
همچو لاله غرقهٔ خون بی رخش
&&
داغ بر دل ز انتظاری مانده‌ام
\\
دیده‌ام میگون لب آن سنگدل
&&
سنگ بر دل در خماری مانده‌ام
\\
چون دهان او نهان شد آشکار
&&
در نهان و آشکاری مانده‌ام
\\
زنگبار زلف او مویی بتافت
&&
زان چو مویش تابداری مانده‌ام
\\
گه به دربند رهی دور و دراز
&&
گه به چین در اضطراری مانده‌ام
\\
چون سر یک موی او بارم نداد
&&
زیر بار مشکباری مانده‌ام
\\
صد جهان ناز از سر مویی که دید
&&
من که دیدم بیقراری مانده‌ام
\\
زلف چون دربند روم روی اوست
&&
من چرا در زنگباری مانده‌ام
\\
می‌شمارم حلقه‌های زلف او
&&
در شمار بی شماری مانده‌ام
\\
چون سری نیست ای عجب این کار را
&&
من مشوش بر کناری مانده‌ام
\\
روزگاری می‌برم در زلف او
&&
بس پریشان روزگاری مانده‌ام
\\
شد فرید از چین زلفش مشک بیز
&&
زان سبب زیر غباری مانده‌ام
\\
\end{longtable}
\end{center}
