\begin{center}
\section*{غزل شماره ۱۶۸: جانا شعاع رویت در جسم و جان نگنجد}
\label{sec:168}
\addcontentsline{toc}{section}{\nameref{sec:168}}
\begin{longtable}{l p{0.5cm} r}
جانا شعاع رویت در جسم و جان نگنجد
&&
وآوازهٔ جمالت اندر جهان نگنجد
\\
وصلت چگونه جویم کاندر طلب نیاید
&&
وصفت چگونه گویم کاندر زبان نگنجد
\\
هرگز نشان ندادند از کوی تو کسی را
&&
زیرا که راه کویت اندر نشان نگنجد
\\
آهی که عاشقانت از حلق جان برآرند
&&
هم در زمان نیاید هم در مکان نگنجد
\\
آنجا که عاشقانت یک دم حضور یابند
&&
دل در حساب ناید جان در میان نگنجد
\\
اندر ضمیر دلها گنجی نهان نهادی
&&
از دل اگر برآید در آسمان نگنجد
\\
عطار وصف عشقت چون در عبارت آرد
&&
زیرا که وصف عشقت اندر بیان نگنجد
\\
\end{longtable}
\end{center}
