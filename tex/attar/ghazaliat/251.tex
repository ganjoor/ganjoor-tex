\begin{center}
\section*{غزل شماره ۲۵۱: برقع از خورشید رویش دور شد}
\label{sec:251}
\addcontentsline{toc}{section}{\nameref{sec:251}}
\begin{longtable}{l p{0.5cm} r}
برقع از خورشید رویش دور شد
&&
ای عجب هر ذره‌ای صد حور شد
\\
همچو خورشید از فروغ طلعتش
&&
ذره ذره پای تا سر نور شد
\\
جملهٔ روی زمین موسی گرفت
&&
جملهٔ آفاق کوه طور شد
\\
چون تجلی‌اش به فرق که فتاد
&&
طور با موسی بهم مهجور شد
\\
فوت خورشید نبود سایه را
&&
لاجرم آن آمد این مقهور شد
\\
قطره‌ای آوازهٔ دریا شنید
&&
از طمع شوریده و مغرور شد
\\
مدتی می‌رفت چون دریا بدید
&&
محو گشت و تا ابد مستور شد
\\
چون در آن دریا نه بد دید و نه نیک
&&
نیک و بد آنجایگه معذور شد
\\
هر دوعالم انگبین صرف بود
&&
لاجرم چون خانهٔ زنبور شد
\\
زانگبین چون آن همه زنبور خاست
&&
هر یکی هم زانگبین مخمور شد
\\
قسم هر یک زانگبین چندان رسید
&&
کز خود و از هر دو عالم دور شد
\\
سایه چون از ظلمت هستی برست
&&
در بر خورشید نورالنور شد
\\
همچو این عطار بس مشهور گشت
&&
همچو آن حلاج بس منصور شد
\\
\end{longtable}
\end{center}
