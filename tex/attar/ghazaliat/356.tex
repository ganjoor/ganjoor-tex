\begin{center}
\section*{غزل شماره ۳۵۶: چو از جیبش مه تابان برآید}
\label{sec:356}
\addcontentsline{toc}{section}{\nameref{sec:356}}
\begin{longtable}{l p{0.5cm} r}
چو از جیبش مه تابان برآید
&&
خروش از گنبد گردان برآید
\\
بسی گل دیده‌ام اما ز رویش
&&
به وقت شرم صد چندان برآید
\\
اگر اندیشهٔ یک روزهٔ او
&&
بگویم با تو صد دیوان برآید
\\
بدو گفتم که ای گلچهره مگذار
&&
که از گلنار تو ریحان برآید
\\
مرا گفتا که خوش باشد که سبزه
&&
ز گرد چشمهٔ حیوان برآید
\\
خط سبزم به چستی سرخییی جست
&&
سزد گر از گل خندان برآید
\\
خطم گر می‌نخواهی نیز مگری
&&
که بی شک سبزه از باران برآید
\\
جهان‌سوزا ز پرده گر برآیی
&&
دمار از خلق سرگردان برآید
\\
فرو شد روز من یک شب برم آی
&&
که تا کار من حیران برآید
\\
مرا با شیر شد مهر تو در دل
&&
عجب نبود اگر با جان برآید
\\
ز من جان خواستی و نیست دشوار
&&
بده یک بوسه تا آسان برآید
\\
زهی زلفت گرفته گرد عالم
&&
ز بیم زلف مه پنهان برآید
\\
چو زلف کافرت در کار آید
&&
بسا مؤمن که از ایمان برآید
\\
دلم در چاه زندان فراق است
&&
ندانم تا کی از زندان برآید
\\
ز یک موی سر زلفت رسن ساز
&&
که تا زین چاه بی‌پایان برآید
\\
اگر عطار بویی یابد از تو
&&
دلش زین وادی هجران برآید
\\
\end{longtable}
\end{center}
