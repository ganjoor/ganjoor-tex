\begin{center}
\section*{غزل شماره ۱۸۳: فرو رفتم به دریایی که نه پای و نه سر دارد}
\label{sec:183}
\addcontentsline{toc}{section}{\nameref{sec:183}}
\begin{longtable}{l p{0.5cm} r}
فرو رفتم به دریایی که نه پای و نه سر دارد
&&
ولی هر قطره‌ای از وی به صد دریا اثر دارد
\\
ز عقل و جان و دین و دل به کلی بی خبر گردد
&&
کسی کز سر این دریا سر مویی خبر دارد
\\
چه گردی گرد این دریا که هر کو مردتر افتد
&&
ازین دریا به هر ساعت تحیر بیشتر دارد
\\
تورا با جان مادرزاد ره نبود درین دریا
&&
کسی این بحر را شاید که او جانی دگر دارد
\\
تو هستی مرد صحرایی نه دریابی نه بشناسی
&&
که با هر یک ازین دریا دل مردان چه سر دارد
\\
ببین تا مرد صاحب دل درین دریا چسان جنبد
&&
که بر راه همه عمری به یک ساعت گذر دارد
\\
تو آن گوهر که در دریا همه اصل اوست کی یابی
&&
چو می‌بینی که این دریا جهانی پر گهر دارد
\\
اگر خواهی که آن گوهر ببینی تو چنان باید
&&
که چون خورشید سر تا پای تو دایم نظر دارد
\\
عجب آن است کین دریا اگرچه جمله آب آمد
&&
ولی از شوق یک قطره زمین لب خشک‌تر دارد
\\
چو شوقش بود بسیاری به آبی نیز غیر خود
&&
ز تو بر ساخت غیر خود تویی غیری اگر دارد
\\
سلامت از چه می‌جویی ملامت به درین دریا
&&
که آن وقت است مرد ایمن که راهی پرخطر دارد
\\
چو از تر دامنی عطار در کنجی است متواری
&&
ندانم کین سخن گفتن ازو کس معتبر دارد
\\
\end{longtable}
\end{center}
