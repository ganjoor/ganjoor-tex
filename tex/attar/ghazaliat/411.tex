\begin{center}
\section*{غزل شماره ۴۱۱: ذره‌ای دوستی آن دمساز}
\label{sec:411}
\addcontentsline{toc}{section}{\nameref{sec:411}}
\begin{longtable}{l p{0.5cm} r}
ذره‌ای دوستی آن دمساز
&&
بهتر از صد هزار ساله نماز
\\
ذره‌ای دوستی بتافت از غیب
&&
آسمان را فکند در تک و تاز
\\
باز خورشید را که سلطانی است
&&
ذره‌ای عشق می‌دهد پرواز
\\
عشق اگر نیستی سر مویی
&&
نه حقیقت بیافتی نه مجاز
\\
ذره‌ای عشق زیر پردهٔ دل
&&
برگشاید هزار پردهٔ راز
\\
زیر هر پرده نقد تو گردد
&&
هر زمان صد جهان پر از اعزاز
\\
وی عجب زیر هر جهان که بود
&&
صد جهان عشق افتدت ز آغاز
\\
باز در هر جهان هزار جهان
&&
می‌شود کشف در نشیب و فراز
\\
گرچه هر لحظه صد جهان یابی
&&
خویش را ذره‌ای نیابی باز
\\
چون به یکدم تو گم شدی با خویش
&&
چون توانی شد آگه از دمساز
\\
تا تو هستی تو را به قطع او نیست
&&
ور نه‌ای فارغی ز ناز و نیاز
\\
او تو را نیست تا تو آن خودی
&&
با تو او نیست، اینت کار دراز
\\
گر درین راه مرد کل طلبی
&&
هرچه داری همه بکل درباز
\\
می‌شنو از فرید حرف بلند
&&
وز بد و نیک خانه می‌پرداز
\\
\end{longtable}
\end{center}
