\begin{center}
\section*{غزل شماره ۷۵۵: اگر از نسیم زلفت اثری به جان فرستی}
\label{sec:755}
\addcontentsline{toc}{section}{\nameref{sec:755}}
\begin{longtable}{l p{0.5cm} r}
اگر از نسیم زلفت اثری به جان فرستی
&&
به امید وصل جان را، خط جاودان فرستی
\\
ز پی تو پاک‌بازان به جهان در اوفتادند
&&
چه اگر ز زلف بویی به همه جهان فرستی
\\
ز تعجب و ز حیرت دل و جان به سر درآید
&&
چو تو بوی زلف مشکین به میان جان فرستی
\\
همه خلق تا قیامت به تحیر اندر افتد
&&
اگر از رخت فروغی به جهانیان فرستی
\\
عجب اینکه سر عشقت دو جهان چگونه پر شد
&&
که تو سر عشق خود را به جهان نهان فرستی
\\
چه عجب بود ز عطار اگر آیدش تفاخر
&&
به جواهری که از دل به سر زبان فرستی
\\
\end{longtable}
\end{center}
