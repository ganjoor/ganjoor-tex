\begin{center}
\section*{غزل شماره ۱۸۲: سر زلف تو بوی گلزار دارد}
\label{sec:182}
\addcontentsline{toc}{section}{\nameref{sec:182}}
\begin{longtable}{l p{0.5cm} r}
سر زلف تو بوی گلزار دارد
&&
لب لعل تو رنگ گلنار دارد
\\
از آن غم که یکدم سر گل نبودت
&&
ببین گل که چون پای بر خار دارد
\\
اگر روی تو نیست خورشید عالم
&&
چرا خلق را ذره کردار دارد
\\
وگر نقطهٔ عاشقان نیست خالت
&&
چرا عاشقان را چو پرگار دارد
\\
وگر زلف تو نیست هندوی ترسا
&&
چرا پس چلیپا و زنار دارد
\\
دهانت چو با پسته‌ای تنگ ماند
&&
شکر تنگ بسته به خروار دارد
\\
خط سبز زنگار رنگ تو یارب
&&
چو گوگرد سرخی چه مقدار دارد
\\
چرا روی کردی ترش تا ز خطت
&&
نگین مسین تو زنگار دارد
\\
ندارم به روی تو چشم تعهد
&&
که روی تو خود چشم بیمار دارد
\\
چو تیمار چشم خودش می نبینم
&&
مرا چشم زخمی چه تیمار دارد
\\
مکن بیقرارم چو گردون که گردون
&&
به صاحب قرانیم اقرار دارد
\\
به یک بوسه جان مرا زنده گردان
&&
که جانم به عالم همین کار دارد
\\
فرید از لب تو سخن چون نگوید
&&
که شعر از لب تو شکربار دارد
\\
\end{longtable}
\end{center}
