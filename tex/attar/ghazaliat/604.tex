\begin{center}
\section*{غزل شماره ۶۰۴: تا دردی درد او چشیدیم}
\label{sec:604}
\addcontentsline{toc}{section}{\nameref{sec:604}}
\begin{longtable}{l p{0.5cm} r}
تا دردی درد او چشیدیم
&&
دامن ز دو کون در کشیدیم
\\
با هم نفسی ز درد عشقش
&&
در کنج فنا بیارمیدیم
\\
بر بوی یقین که بو که بینیم
&&
زهری به گمان دل چشیدیم
\\
گه در طلبش ز دست رفتیم
&&
گه در هوسش به سر دویدیم
\\
در عالم پر عجایب عشق
&&
آوازهٔ او بسی شنیدیم
\\
درمان چه‌کنیم درد او را
&&
کین درد به جان و دل خریدیم
\\
عشقش چو به ما نمود ما را
&&
صد پرده به یک زمان دریدیم
\\
نور رخ او چو شعله‌ای زد
&&
خود را ز فروغ آن بدیدیم
\\
دیدیم که ما نه ز آب و خاکیم
&&
از هر دو برون رهی گزیدیم
\\
چه خاک و چه آب کانچه ماییم
&&
در پردهٔ غیب ناپدیدیم
\\
چون پرده ز روی کار برخاست
&&
از خود نه ازو بدو رسیدیم
\\
پیوستگیی چو یافت عطار
&&
از ننگ وجود او بریدیم
\\
\end{longtable}
\end{center}
