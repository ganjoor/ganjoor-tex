\begin{center}
\section*{غزل شماره ۳۱۲: دل نظر بر روی آن شمع جهان می‌افکند}
\label{sec:312}
\addcontentsline{toc}{section}{\nameref{sec:312}}
\begin{longtable}{l p{0.5cm} r}
دل نظر بر روی آن شمع جهان می‌افکند
&&
تن به جای خرقه چون پروانه جان می‌افکند
\\
گر بود غوغای عشقش بر کنار عالمی
&&
دل ز شوقش خویشتن را در میان می‌افکند
\\
زلف او صد توبه را در یک نفس می‌بشکند
&&
چشم او صد صید را در یک زمان می‌افکند
\\
طرهٔ مشکینش تابی در فلک می‌آورد
&&
پستهٔ شیرینش شوری در جهان می‌افکند
\\
سبز پوشان فلک ماه زمینش خوانده‌اند
&&
زانکه رویش غلغلی در آسمان می‌افکند
\\
تا ابد کامش ز شیرینی نگردد تلخ و تیز
&&
هر که نام آن شکر لب بر زبان می‌افکند
\\
ترکم آن دارد سر آن چون ندارد چون کنم
&&
هندوی خود را چنین در پا از آن می‌افکند
\\
همچو دف حلقه به گوش او شدم با این همه
&&
بر تنم چون چنگ هر رگ در فغان می‌افکند
\\
گاهگاهی گویدم هستم یقین من زان تو
&&
لاجرم عطار را اندر گمان می‌افکند
\\
\end{longtable}
\end{center}
