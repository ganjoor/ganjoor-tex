\begin{center}
\section*{غزل شماره ۱۵۳: شرح لب لعلت به زبان می‌نتوان داد}
\label{sec:153}
\addcontentsline{toc}{section}{\nameref{sec:153}}
\begin{longtable}{l p{0.5cm} r}
شرح لب لعلت به زبان می‌نتوان داد
&&
وز میم دهان تو نشان می‌نتوان داد
\\
میم است دهان تو و مویی است میانت
&&
کی را خبر موی میان می‌نتوان داد
\\
دل خواسته‌ای و رقم کفر کشم من
&&
بر هر که گمان برد که جان می‌نتوان داد
\\
گر پیش رخت جان ندهم آن نه ز بخل است
&&
در خورد رخت نیست از آن می‌نتوان داد
\\
یک جان چه بود کافرم ار پیش تو صد جان
&&
انگشت زنان رقص کنان می‌نتوان داد
\\
سگ به بود از من اگر از بهر سگت جان
&&
آزاد به یک پارهٔ نان می‌نتوان داد
\\
داد ره عشق تو چنان کرزویم هست
&&
عمرم شد و یک لحظه چنان می‌نتوان داد
\\
جانا چو بلای تو به‌ارزد به جهانی
&&
خود را ز بلای تو امان می‌نتوان داد
\\
گفتم که ز من جان بستان یک شکرم ده
&&
گفتی شکر من به زبان می‌نتوان داد
\\
چون نیست دهانم که شکر زو به در آید
&&
کس را به شکر هیچ دهان می‌نتوان داد
\\
خود طالع عطار چه چیز است که او را
&&
یک بوسه نه پیدا و نه نهان می‌نتوان داد
\\
\end{longtable}
\end{center}
