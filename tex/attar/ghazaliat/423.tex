\begin{center}
\section*{غزل شماره ۴۲۳: دستم نرسد به زلف چون شستش}
\label{sec:423}
\addcontentsline{toc}{section}{\nameref{sec:423}}
\begin{longtable}{l p{0.5cm} r}
دستم نرسد به زلف چون شستش
&&
در پای از آن فتادم از دستش
\\
گر مرغ هوای او شوم شاید
&&
صد دام معنبر است در شستش
\\
از لب ندهد میی و می‌داند
&&
مخموری من ز نرگس مستش
\\
بیچاره دلم که چشم مست او
&&
صد توبه به یک کرشمه بشکستش
\\
بشکفت گل رخش به زیبایی
&&
غنچه ز میان جان کمر بستش
\\
از بس که بریخت مشک از زلفش
&&
چون خاک به زیر پای شد پستش
\\
چون بود بتی چنان که در عالم
&&
بپرستندش که جای آن هستش
\\
یک یک سر موی من همی گوید
&&
رویش بنگر که گفت مپرستش
\\
نی نی که نقاب بر نمی‌دارد
&&
تا سجده نمی‌کنند پیوستش
\\
عطار دلی که داشت در عشقش
&&
برخاست اومید و نیست بنشستش
\\
\end{longtable}
\end{center}
