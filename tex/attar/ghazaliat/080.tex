\begin{center}
\section*{غزل شماره ۸۰: خاصیت عشقت که برون از دو جهان است}
\label{sec:080}
\addcontentsline{toc}{section}{\nameref{sec:080}}
\begin{longtable}{l p{0.5cm} r}
خاصیت عشقت که برون از دو جهان است
&&
آن است که هرچیز که گویند نه آن است
\\
برتر ز صفات خرد و دانش و عقل است
&&
بیرون ز ضمیر دل و اندیشهٔ جان است
\\
بینندهٔ انوار تو بس دوخته چشم است
&&
گویندهٔ اسرار تو بس گنگ زبان است
\\
از وصف تو هر شرح که دادند محال است
&&
وز عشق تو هر سود که کردند زیان است
\\
در پردهٔ پندار چو بازی و خیال است
&&
جز عشق تو هر چیز که در هر دو جهان است
\\
گر عقل نشان است ز خورشید جمالت
&&
یک ذره ز خورشید، فلک مژده‌رسان است
\\
یک ذرهٔ حیران شده را عقل چو داند
&&
کز جملهٔ خورشید فلک چند نشان است
\\
چو عقل یقین است که در عشق عقیله است
&&
بی شک به تو دانست تو را هر که بدان است
\\
در راه تو هرکس به گمانی قدمی زد
&&
وین شیوه کمانی نه به بازوی گمان است
\\
چه سود که نقاش کشد صورت سیمرغ
&&
چون در نفس باز پس انگشت گزان است
\\
گرچه بود آن صورت سیمرغ ولیکن
&&
چون جوهر سیمرغ به عینه نه همان است
\\
فی‌الجمله چه زارم، چکنم، قصه چه گویم
&&
کان اصل که جان است هم از خویش نهان است
\\
عطار که پی برد بسی دانش و بینش
&&
اندر پی آن است که بالای عیان است
\\
\end{longtable}
\end{center}
