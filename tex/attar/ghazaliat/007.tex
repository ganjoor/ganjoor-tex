\begin{center}
\section*{غزل شماره ۷: بار دگر شور آورید این پیر درد آشام را}
\label{sec:007}
\addcontentsline{toc}{section}{\nameref{sec:007}}
\begin{longtable}{l p{0.5cm} r}
بار دگر شور آورید این پیر درد آشام را
&&
صد جام برهم نوش کرد از خون دل پر جام ما
\\
چون راست کاندر کار شد وز کعبه در خمار شد
&&
در کفر خود دین دار شد بیزار شد ز اسلام ما
\\
پس گفت تا کی زین هوس ماییم و درد یک نفس
&&
دایم یکی گوییم وبس تا شد دو عالم رام ما
\\
بس کم زنی استاد شد بی خانه و بنیاد شد
&&
از نام و ننگ آزاد شد نیک است این بدنام را
\\
پس شد چون مردان مرد او وز هر دو عالم فرد او
&&
وز درد درد درد او شد مست هفت اندام ما
\\
دل گشت چون دلداده‌ای جان شد ز کار افتاده‌ای
&&
تا ریخت پر هر باده‌ای از جام دل در جام ما
\\
جان را چون آن می نوش شد از بی‌خودی بیهوش شد
&&
عقل از جهان خاموش شد و از دل برفت آرام ما
\\
عطار در دیر مغان خون می‌کشید اندر نهان
&&
فریاد برخاست از جهان کای رند درد آشام ما
\\
\end{longtable}
\end{center}
