\begin{center}
\section*{غزل شماره ۱۴۴: ای آفتاب سرکش یک ذره خاک پایت}
\label{sec:144}
\addcontentsline{toc}{section}{\nameref{sec:144}}
\begin{longtable}{l p{0.5cm} r}
ای آفتاب سرکش یک ذره خاک پایت
&&
آب حیات رشحی از جام جانفزایت
\\
هم خواجه تاش گردون دل بر وفا غلامت
&&
هم پادشاه گیتی جان بر میان گدایت
\\
هم چرخ خرقه‌پوشی در خانقاه عشقت
&&
هم جبرئیل مرغی در دام دل ربایت
\\
در سر گرفته عالم اندیشهٔ وصالت
&&
در چشم کرده کوثر خاک در سرایت
\\
کوثر که آب حیوان یک شبنم است از وی
&&
دربسته تا به جان دل در لعل دلگشایت
\\
سری که هر دو عالم یک ذره می‌نیابند
&&
جاوید کف گرفته جام جهان نمایت
\\
نوباوهٔ جمالت ماه نو است و هر مه
&&
بنهد کله ز خجلت در دامن قبایت
\\
تو ابرش نکویی می‌تازی و مه و مهر
&&
چون سایه در رکابت چون ذره در هوایت
\\
تا بوی مشک زلفت پر مشک کرد جانم
&&
عطار مشک ریزم از زلف مشک سایت
\\
\end{longtable}
\end{center}
