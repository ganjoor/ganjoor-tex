\begin{center}
\section*{غزل شماره ۴۷۳: از بس که روز و شب غم بر غم کشیده‌ام}
\label{sec:473}
\addcontentsline{toc}{section}{\nameref{sec:473}}
\begin{longtable}{l p{0.5cm} r}
از بس که روز و شب غم بر غم کشیده‌ام
&&
شادی فکنده‌ام غم بر غم گزیده‌ام
\\
شادی به روی غم که غمم غمگسار گشت
&&
کم غم چو روی شادی عالم بدیده‌ام
\\
گر نیز شادی است درین آشیان غم
&&
من شادیی ندیده‌ام اما شنیده‌ام
\\
کس را مباد با من و با درد من رجوع
&&
زیرا که درد عشق مسلم خریده‌ام
\\
تا کی ز درد عشق زنم لاف چون ز نفس
&&
دایم به دل رمیده به تن آرمیده‌ام
\\
هرگز دمی نیافته‌ام هیچ فرصتی
&&
چندانکه با سگان طبیعت چخیده‌ام
\\
گرچه قدم نداشته‌ام در مقام عدل
&&
باری ز اهل ظلم قدم در کشیده‌ام
\\
در گوشه‌ای نشسته بسی خون بخورده‌ام
&&
بر جایگه فسرده بسی ره بریده‌ام
\\
عمرم گذشت در بچه طبعی و من هنوز
&&
از حرص و آز چون بچهٔ نا رسیده‌ام
\\
هر روز در خزانهٔ عطار کمتر است
&&
دری که از سفینهٔ دانش گزیده‌ام
\\
\end{longtable}
\end{center}
