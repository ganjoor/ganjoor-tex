\begin{center}
\section*{غزل شماره ۱۳۱: دل کمال از لعل میگون تو یافت}
\label{sec:131}
\addcontentsline{toc}{section}{\nameref{sec:131}}
\begin{longtable}{l p{0.5cm} r}
دل کمال از لعل میگون تو یافت
&&
جان حیات از نطق موزون تو یافت
\\
گر ز چشمت خسته‌ای آمد به تیر
&&
زنده شد چون در مکنون تو یافت
\\
تا فسونت کرد چشم ساحرت
&&
جامه پر کژدم ز افسون تو یافت
\\
سخت‌تر از سنگ نتوان آمدن
&&
لعل بین یعنی دلش خون تو یافت
\\
تا فشاندی زلف و بگشادی دهن
&&
عقل خود را مست و مجنون تو یافت
\\
ملک کسری در سر زلف تو دید
&&
جام جم در لعل گلگون تو یافت
\\
قاف تا قاف جهان یکسر بگشت
&&
کاف کفر از زلف چون نون تو یافت
\\
جمله را صدباره فی‌الجمله بدید
&&
هیچش آمد هرچه بیرون تو یافت
\\
تا دل عطار عالم کم گرفت
&&
رونق از حسن در افزون تو یافت
\\
\end{longtable}
\end{center}
