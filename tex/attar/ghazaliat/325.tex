\begin{center}
\section*{غزل شماره ۳۲۵: آنرا که ز وصل او نشان بود}
\label{sec:325}
\addcontentsline{toc}{section}{\nameref{sec:325}}
\begin{longtable}{l p{0.5cm} r}
آنرا که ز وصل او نشان بود
&&
دل گم شدگیش جاودان بود
\\
آری چو بتافت شمع خورشید
&&
گر بود ستاره‌ای نهان بود
\\
نتواند رفت قطره در بحر
&&
چون بحر به جای او روان بود
\\
بحری که اگرچه موج‌ها زد
&&
اما همه عمر همچنان بود
\\
هر دم بنمود صد جهان لیک
&&
نتوان گفتن که یک جهان بود
\\
زیرا که شد آمدی که افتاد
&&
پندار خیال یا گمان بود
\\
گر بود نمود فرع غیری
&&
لاغیری دان که بس عیان بود
\\
زانجا که حیات لعب و لهوست
&&
بازی خیال در میان بود
\\
هرگاه که این خیال برخاست
&&
هر عیب که بود عیب‌دان بود
\\
چون هست حقیقت همه بحر
&&
پس قطره و بحر هم‌عنان بود
\\
خورشید رخش بتافت ناگاه
&&
هر ذره که بود دیده‌بان بود
\\
در هر دل ذره‌ای محقر
&&
گویی تو که صد هزار جان بود
\\
هر ذره اگرچه صد نشان داشت
&&
چون در نگریست بی‌نشان بود
\\
چون پرتو ذره‌ای چنین است
&&
چه جای زمین و آسمان بود
\\
طاوس رخش چو جلوه‌ای کرد
&&
ذرات جهان هم آشیان بود
\\
در پیش چنان جمال یکدم
&&
در هر دو جهان که را امان بود
\\
جانا برهان مرا ز من زانک
&&
از خویش مرا بسی زیان بود
\\
جان کاستن است بی تو بودن
&&
خود بی تو چگونه می‌توان بود
\\
عطار دمی اگر ز خود رست
&&
گویی شب و روز کامران بود
\\
\end{longtable}
\end{center}
