\begin{center}
\section*{غزل شماره ۳۸۷: از پس پردهٔ دل دوش بدیدم رخ یار}
\label{sec:387}
\addcontentsline{toc}{section}{\nameref{sec:387}}
\begin{longtable}{l p{0.5cm} r}
از پس پردهٔ دل دوش بدیدم رخ یار
&&
شدم از دست و برفت از دل من صبر و قرار
\\
کار من شد چو سر زلف سیاهش درهم
&&
حال من گشت چو خال رخ او تیره و تار
\\
گفتم ای جان شدم از نرگس مست تو خراب
&&
گفت در شهر کسی نیست ز دستم هشیار
\\
گفتم این جان به لب آمد ز فراقت گفتا
&&
چون تو در هر طرفی هست مرا کشته هزار
\\
گفتم اندر حرم وصل توام مأوی بود
&&
گفت اندر حرم شاه که را باشد بار
\\
گفتم از درد تو دل نیک شود، گفتا نی
&&
گفتم از رنج تو دل باز رهد، گفتا دشوار!
\\
گفتم از دست ستم‌های تو تا کی نالم
&&
گفت تا داغ محبت بودت بر رخسار
\\
گفتم ای جان جهان چون که مرا خواهی سوخت
&&
بکشم زود وزین بیش مرا رنجه مدار
\\
در پس پرده شد و گفت مرا از سر خشم
&&
هرزه زین بیش مگو کار به من بازگذار
\\
گر کشم زار و اگر زنده کنم من دانم
&&
در ره عشق تو را با من و با خویش چه کار
\\
حاصلت نیست ز من جز غم و سرگردانی
&&
خون خور و جان کن ازین هستی خود دل بردار
\\
چون که عطار ازین شیوه حکایات شنود
&&
دردش افزون شد ازین غصه و رنجش بسیار
\\
با رخ زرد و دم سرد و سر پر سودا
&&
بر سر کوی غمش منتظر یک دیدار
\\
\end{longtable}
\end{center}
