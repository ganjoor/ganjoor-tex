\begin{center}
\section*{غزل شماره ۱۹۶: دم عیسی است که با باد سحر می‌گذرد}
\label{sec:196}
\addcontentsline{toc}{section}{\nameref{sec:196}}
\begin{longtable}{l p{0.5cm} r}
دم عیسی است که با باد سحر می‌گذرد
&&
وآب خضر است که بر روی خضر می‌گذرد
\\
عمر اگرچه گذران است عجب می‌دارم
&&
با چنان باد و چنین آب اگر می‌گذرد
\\
می‌ندانم که ز فردوس صبا بهر چه کار
&&
می‌رسد حالی و چون مرغ به پر می‌گذرد
\\
یاسمین را که اگر هست بقایی نفسی است
&&
هر نفس جلوه‌گر از دست دگر می‌گذرد
\\
لاله بس گرم مزاج است که با سردی کوه
&&
با دلی سوخته در خون جگر می‌گذرد
\\
گوییا عمر گل تازه صبای سحر است
&&
کز پس پرده برون نامده بر می‌گذرد
\\
گل سیراب که از آتش دل تشنه لب است
&&
آب خواهی است که با جام بزر می‌گذرد
\\
ابر پر آب کند جامش و از ابر او را
&&
جام نابرده به لب آب ز سر می‌گذرد
\\
در عجب مانده‌ام تا گل‌تر را به دریغ
&&
این چه عمر است که ناآمده در می‌گذرد
\\
ابر از خجلت و تشویر درافشانی شاه
&&
می‌دمد آتش و با دامن تر می‌گذرد
\\
طربی در همه دلهاست درین فصل امروز
&&
گوییا بر لب عطار شکر می‌گذرد
\\
\end{longtable}
\end{center}
