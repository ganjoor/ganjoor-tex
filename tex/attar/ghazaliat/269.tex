\begin{center}
\section*{غزل شماره ۲۶۹: قوت بار عشق تو مرکب جان نمی‌کشد}
\label{sec:269}
\addcontentsline{toc}{section}{\nameref{sec:269}}
\begin{longtable}{l p{0.5cm} r}
قوت بار عشق تو مرکب جان نمی‌کشد
&&
روشنی جمال تو هر دو جهان نمی‌کشد
\\
بار تو چون کشد دلم گرچه چو تیر راست شد
&&
زانکه کمان چون تویی بازوی جان نمی‌کشد
\\
کون و مکان چه می‌کند عاشق تو که در رهت
&&
نعرهٔ عاشقان تو کون و مکان نمی‌کشد
\\
نام تو و نشان تو چون به زبان برآورم
&&
زانکه نشان و نام تو نام و نشان نمی‌کشد
\\
راه تو چون به سرکشم زانکه ز دوری رهت
&&
راه تو از روندگان کس به کران نمی‌کشد
\\
در ره تو به قرن‌ها چرخ دوید و دم نزد
&&
تا ره تو به سر نشد خود به میان نمی‌کشد
\\
گشت فرید در رهت سوخته همچو پشه‌ای
&&
زانکه ز نور شمع تو ره به عیان نمی‌کشد
\\
\end{longtable}
\end{center}
