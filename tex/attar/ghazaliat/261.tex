\begin{center}
\section*{غزل شماره ۲۶۱: پیر ما از صومعه بگریخت در میخانه شد}
\label{sec:261}
\addcontentsline{toc}{section}{\nameref{sec:261}}
\begin{longtable}{l p{0.5cm} r}
پیر ما از صومعه بگریخت در میخانه شد
&&
در صف دردی کشان دردی کش و مردانه شد
\\
بر بساط نیستی با کم‌زنان پاک‌باز
&&
عقل اندر باخت وز لایعقلی دیوانه شد
\\
در میان بیخودان مست دردی نوش کرد
&&
در زبان زاهدان بی‌خبر افسانه شد
\\
آشنایی یافت با چیزی که نتوان داد شرح
&&
وز همه کارث جهان یکبارگی بیگانه شد
\\
راست کان خورشید جانها برقع از رخ بر گرفت
&&
عقل چون خفاش گشت و روح چون پروانه شد
\\
چون نشان گم کرد دل از سر او افتاد نیست
&&
جان و دل در بی نشانی با فنا هم‌خانه شد
\\
عشق آمد گفت خون تو بخواهم ریختن
&&
دل که این بشنود حالی از پی شکرانه شد
\\
چون دل عطار پر جوش آمد از سودای عشق
&&
خون به سر بالا گرفت و چشم او پیمانه شد
\\
\end{longtable}
\end{center}
