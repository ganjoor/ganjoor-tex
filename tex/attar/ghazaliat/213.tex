\begin{center}
\section*{غزل شماره ۲۱۳: زین دم عیسی که هر ساعت سحر می‌آورد}
\label{sec:213}
\addcontentsline{toc}{section}{\nameref{sec:213}}
\begin{longtable}{l p{0.5cm} r}
زین دم عیسی که هر ساعت سحر می‌آورد
&&
عالمی بر خفته سر از خاک بر می‌آورد
\\
هر زمان ابر از هوا نزلی دگر می‌افکند
&&
هر نفس باغ از صبا زیبی دگر می‌آورد
\\
ابر تر دامن برای خشک مغزان چمن
&&
از بهشت عدن مروارید تر می‌آورد
\\
هر کجا در زیر خاک تیره گنجی روشن است
&&
دست ابرش پای کوبان باز بر می‌آورد
\\
طعم شیر و شکر آید از لب طفلان باغ
&&
زانکه آب از ابر شیر چون شکر می‌آورد
\\
با نسیم صبح گویی راز غیبی در میان است
&&
کز ضمیر آهوان چین خبر می‌آورد
\\
غنچه چو زرق خود از بالا طلب دارد چو ابر
&&
از برای آن دهان بالای سر می‌آورد
\\
گر ز بی برگی درون غنچه خون می‌خورد گل
&&
هر دم از پرده برون برگی دگر می‌آورد
\\
مشک را چون بوی نقصان می‌پذیرد از جگر
&&
گل چگونه بوی مشکین از جگر می‌آورد
\\
گل چو می‌داند که عمری سرسری دارد چو برق
&&
زندگانی بر سر آتش به سر می‌آورد
\\
نرگس سیمین چو پر می جام زرین می‌کشد
&&
سر گرانی هر دمش از پای در می‌آورد
\\
لاجرم از بس که می‌خورده است آن مخمور چشم
&&
چشم خواب آلود پر خواب سحر می‌آورد
\\
یا صبای تند گویی سیم و زر را می‌زند
&&
زین قبل در دست سیمین جام زر می‌آورد
\\
تا که در باغ سخن عطار شد طاوس عشق
&&
در سخن خورشید را در زیر پر می‌آورد
\\
\end{longtable}
\end{center}
