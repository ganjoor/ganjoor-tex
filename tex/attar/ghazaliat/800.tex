\begin{center}
\section*{غزل شماره ۸۰۰: ای روی تو فتنهٔ جهانی}
\label{sec:800}
\addcontentsline{toc}{section}{\nameref{sec:800}}
\begin{longtable}{l p{0.5cm} r}
ای روی تو فتنهٔ جهانی
&&
مبهوت تو هر کجا که جانی
\\
کرده سر زلف پر فریبت
&&
از هر سر مویم امتحانی
\\
در چشم زدی ز دست بر هم
&&
چشمت به کرشمه‌ای جهانی
\\
ابروی تو رستها چو تیراست
&&
بر زه که کند چنان کمانی
\\
طراری را طراوتی نیست
&&
با طرهٔ چون تو دلستانی
\\
ندهد مه و مهر نور هرگز
&&
بی عارض چون تو مهربانی
\\
در دل بردن به خوبی تو
&&
هرگز ندهد کسی نشانی
\\
خورشید رخ تو را کند ذکر
&&
هر ذره اگر شود زبانی
\\
تا من سگ تو شدم نماندست
&&
از قالب من جز استخوانی
\\
من خاک توام مرا چنین خوار
&&
در خون مفکن به هر زمانی
\\
در عشق تو چست‌تر ز عطار
&&
مرغی نپرد ز آشیانی
\\
\end{longtable}
\end{center}
