\begin{center}
\section*{غزل شماره ۷۶۸: گر یار چنین سرکش و عیار نبودی}
\label{sec:768}
\addcontentsline{toc}{section}{\nameref{sec:768}}
\begin{longtable}{l p{0.5cm} r}
گر یار چنین سرکش و عیار نبودی
&&
حال من بیچاره چنین زار نبودی
\\
گر عشق بتان خنجر هجران نکشیدی
&&
در روی زمین خوشتر ازین کار نبودی
\\
از شادی من خلق جهان شاد شدندی
&&
گر بر دل من بار غم یار نبودی
\\
از بادهٔ من خلق جهان مست بدندی
&&
در روی زمین یک تن هشیار نبودی
\\
گر یار گذر بر سر بازار نکردی
&&
هنگامهٔ ما بر سر بازار نبودی
\\
هر زاهد خشکی نفس از عشق زدندی
&&
گر یار چنین سرکش و خونخوار نبودی
\\
زلف تو اگر دعوت کفار نکردی
&&
امروز کس لایق زنار نبودی
\\
گر یار نمودی رخ خود را به همه خلق
&&
اندر دو جهان همدم عطار نبودی
\\
\end{longtable}
\end{center}
