\begin{center}
\section*{غزل شماره ۴۵۷: گشت جهان همچو نگار ای غلام}
\label{sec:457}
\addcontentsline{toc}{section}{\nameref{sec:457}}
\begin{longtable}{l p{0.5cm} r}
گشت جهان همچو نگار ای غلام
&&
بادهٔ گلرنگ بیار ای غلام
\\
با گل و با بلبل و با مل بهم
&&
وصل‌طلب فصل بهار ای غلام
\\
بلبل عاشق به صبوحی درست
&&
می‌شنوی نالهٔ زار ای غلام
\\
نرگس سرمست نگر کاو فکند
&&
سر ز گرانی به کنار ای غلام
\\
پیش نشین تازه بکن کار آب
&&
بیش مبر آب ز کار ای غلام
\\
آب بده زانکه جهان هر نفس
&&
خاک کند چون تو هزار ای غلام
\\
زخم خمارم چو به زاری بکشت
&&
نوش خمارم ز خم آر ای غلام
\\
روز چو شد باز نیاید دگر
&&
چند کنی روز گذار ای غلام
\\
چند شمار زر و زینت کنی
&&
فکر کن از روز شمار ای غلام
\\
نیستی آگه که دم واپسین
&&
از تو برآرند دمار ای غلام
\\
قصهٔ مرگم جگر و دل بسوخت
&&
دست ازین قصه بدار ای غلام
\\
واقعهٔ مشکل دارالغرور
&&
برد ز عطار قرار ای غلام
\\
\end{longtable}
\end{center}
