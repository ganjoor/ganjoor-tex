\begin{center}
\section*{رباعی شماره ۶۶۶: مزار دلی را که تو جانش باشی}
\label{sec:sh666}
\addcontentsline{toc}{section}{\nameref{sec:sh666}}
\begin{longtable}{l p{0.5cm} r}
مآزار دلی را که تو جانش باشی
&&
معشوقهٔ پیدا و نهانش باشی
\\
زان می‌ترسم که از دلازاری تو
&&
دل خون شود و تو در میانش باشی
\\
\end{longtable}
\end{center}
