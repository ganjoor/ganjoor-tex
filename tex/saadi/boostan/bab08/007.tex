\begin{center}
\section*{بخش ۷ - حکایت سلطان طغرل و هندوی پاسبان: شنیدم که طغرل شبی در خزان}
\label{sec:007}
\addcontentsline{toc}{section}{\nameref{sec:007}}
\begin{longtable}{l p{0.5cm} r}
شنیدم که طغرل شبی در خزان
&&
گذر کرد بر هندوی پاسبان
\\
ز باریدن برف و باران و سیل
&&
به لرزش در افتاده همچون سهیل
\\
دلش بر وی از رحمت آورد جوش
&&
که اینک قبا پوستینم بپوش
\\
دمی منتظر باش بر طرف بام
&&
که بیرون فرستم به دست غلام
\\
در این بود و باد صبا بروزید
&&
شهنشه در ایوان شاهی خزید
\\
وشاقی پری چهره در خیل داشت
&&
که طبعش بدو اندکی میل داشت
\\
تماشای ترکش چنان خوش فتاد
&&
که هندوی مسکین برفتش ز یاد
\\
قبا پوستینی گذشتش به گوش
&&
ز بدبختیش در نیامد به دوش
\\
مگر رنج سرما بر او بس نبود
&&
که جور سپهر انتظارش فزود
\\
نگه کن چو سلطان به غفلت بخفت
&&
که چوبک زنش بامدادان چه گفت
\\
مگر نیکبختت فراموش شد
&&
چو دستت در آغوش آغوش شد؟
\\
تو را شب به عیش و طرب می‌رود
&&
چه دانی که بر ما چه شب می‌رود؟
\\
فرو برده سر کاروانی به دیگ
&&
چه از پا فرو رفتگانش به ریگ
\\
بدار ای خداوند زورق بر آب
&&
که بیچارگان را گذشت از سر آب
\\
توقف کنید ای جوانان چست
&&
که در کاروانند پیران سست
\\
تو خوش خفته در هودج کاروان
&&
مهار شتر در کف ساروان
\\
چه هامون و کوهت، چه سنگ و رمال
&&
ز ره باز پس ماندگان پرس حال
\\
تو را کوه پیکر هیون می‌برد
&&
پیاده چه دانی که خون می‌خورد؟
\\
به آرام دل خفتگان در بنه
&&
چه دانند حال کم گرسنه؟
\\
\end{longtable}
\end{center}
