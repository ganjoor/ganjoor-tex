\begin{center}
\section*{بخش ۱۴ - در سابقهٔ حکم ازل و توفیق خیر: نخست او ارادت به دل در نهاد}
\label{sec:014}
\addcontentsline{toc}{section}{\nameref{sec:014}}
\begin{longtable}{l p{0.5cm} r}
نخست او ارادت به دل در نهاد
&&
پس این بنده بر آستان سر نهاد
\\
گر از حق نه توفیق خیری رسد
&&
کی از بنده چیزی به غیری رسد؟
\\
زبان را چه بینی که اقرار داد
&&
ببین تا زبان را که گفتار داد
\\
در معرفت دیدهٔ آدمی است
&&
که بگشوده بر آسمان و زمی است
\\
کیت فهم بودی نشیب و فراز
&&
گر این در نکردی به روی تو باز؟
\\
سر آورد و دست از عدم در وجود
&&
در این جود بنهاد و در وی سجود
\\
وگرنه کی از دست جود آمدی؟
&&
محال است کز سر سجود آمدی
\\
به حکمت زبان داد و گوش آفرید
&&
که باشند صندوق دل را کلید
\\
اگر نه زبان قصه برداشتی
&&
کس از سر دل کی خبر داشتی؟
\\
وگر نیستی سعی جاسوس گوش
&&
خبر کی رسیدی به سلطان هوش
\\
مرا لفظ شیرین خواننده داد
&&
تو را سمع و ادراک داننده داد
\\
مدام این دو چون حاجبان بر درند
&&
ز سلطان به سلطان خبر می‌برند
\\
چه اندیشی از خود که فعلم نکوست؟
&&
از آن در نگه کن که توفیق اوست
\\
برد بوستانبان به ایوان شاه
&&
به نوباوه گل هم ز بستان شاه
\\
\end{longtable}
\end{center}
