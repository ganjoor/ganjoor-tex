\begin{center}
\section*{بخش ۱ - سر آغاز: شنیدم که در وقت نزع روان}
\label{sec:001}
\addcontentsline{toc}{section}{\nameref{sec:001}}
\begin{longtable}{l p{0.5cm} r}
شنیدم که در وقت نزع روان
&&
به هرمز چنین گفت نوشیروان
\\
که خاطر نگهدار درویش باش
&&
نه در بند آسایش خویش باش
\\
نیاساید اندر دیار تو کس
&&
چو آسایش خویش جویی و بس
\\
نیاید به نزدیک دانا پسند
&&
شبان خفته و گرگ در گوسفند
\\
برو پاس درویش محتاج دار
&&
که شاه از رعیت بود تاجدار
\\
رعیت چو بیخند و سلطان درخت
&&
درخت، ای پسر، باشد از بیخ سخت
\\
مکن تا توانی دل خلق ریش
&&
وگر می‌کنی می‌کنی بیخ خویش
\\
اگر جاده‌ای بایدت مستقیم
&&
ره پارسایان امید است و بیم
\\
طبیعت شود مرد را بخردی
&&
به امید نیکی و بیم بدی
\\
گر این هر دو در پادشه یافتی
&&
در اقلیم و ملکش پنه یافتی
\\
که بخشایش آرد بر امیدوار
&&
به امید بخشایش کردگار
\\
گزند کسانش نیاید پسند
&&
که ترسد که در ملکش آید گزند
\\
وگر در سرشت وی این خوی نیست
&&
در آن کشور آسودگی بوی نیست
\\
اگر پای بندی رضا پیش گیر
&&
وگر یک سواری سر خویش گیر
\\
فراخی در آن مرز و کشور مخواه
&&
که دلتنگ بینی رعیت ز شاه
\\
ز مستکبران دلاور بترس
&&
از آن کاو نترسد ز داور بترس
\\
دگر کشور آباد بیند به خواب
&&
که دارد دل اهل کشور خراب
\\
خرابی و بدنامی آید ز جور
&&
رسد پیش بین این سخن را به غور
\\
رعیت نشاید به بیداد کشت
&&
که مر سلطنت را پناهند و پشت
\\
مراعات دهقان کن از بهر خویش
&&
که مزدور خوشدل کند کار بیش
\\
مروت نباشد بدی با کسی
&&
کز او نیکویی دیده باشی بسی
\\
شنیدم که خسرو به شیرویه گفت
&&
در آن دم که چشمش زدیدن بخفت
\\
بر آن باش تا هرچه نیت کنی
&&
نظر در صلاح رعیت کنی
\\
الا تا نپیچی سر از عدل و رای
&&
که مردم ز دستت نپیچند پای
\\
گریزد رعیت ز بیدادگر
&&
کند نام زشتش به گیتی سمر
\\
بسی بر نیاید که بنیاد خود
&&
بکند آن که بنهاد بنیاد بد
\\
خرابی کند مرد شمشیر زن
&&
نه چندان که دود دل طفل و زن
\\
چراغی که بیوه زنی برفروخت
&&
بسی دیده باشی که شهری بسوخت
\\
از آن بهره‌ورتر در آفاق کیست
&&
که در ملکرانی به انصاف زیست
\\
چو نوبت رسد زین جهان غربتش
&&
ترحم فرستند بر تربتش
\\
بد و نیک مردم چو می‌بگذرند
&&
همان به که نامت به نیکی برند
\\
خداترس را بر رعیت گمار
&&
که معمار ملک است پرهیزگار
\\
بد اندیش توست آن و خونخوار خلق
&&
که نفع تو جوید در آزار خلق
\\
ریاست به دست کسانی خطاست
&&
که از دستشان دستها برخداست
\\
نکوکارپرور نبیند بدی
&&
چو بد پروری خصم خون خودی
\\
مکافات موذی به مالش مکن
&&
که بیخش برآورد باید ز بن
\\
مکن صبر بر عامل ظلم دوست
&&
که از فربهی بایدش کند پوست
\\
سر گرگ باید هم اول برید
&&
نه چون گوسفندان مردم درید
\\
چه خوش گفت بازارگانی اسیر
&&
چو گردش گرفتند دزدان به تیر
\\
چو مردانگی آید از رهزنان
&&
چه مردان لشکر، چه خیل زنان
\\
شهنشه که بازارگان را بخست
&&
در خیر بر شهر و لشکر ببست
\\
کی آن جا دگر هوشمندان روند
&&
چو آوازهٔ رسم بد بشنوند؟
\\
نکو بایدت نام و نیکی قبول
&&
نکو دار بازارگان و رسول
\\
بزرگان مسافر به جان پرورند
&&
که نام نکویی به عالم برند
\\
تبه گردد آن مملکت عن قریب
&&
کز او خاطر آزرده آید غریب
\\
غریب آشنا باش و سیاح دوست
&&
که سیاح جلاب نام نکوست
\\
نکو دار ضیف و مسافر عزیز
&&
وز آسیبشان بر حذر باش نیز
\\
ز بیگانه پرهیز کردن نکوست
&&
که دشمن توان بود در زی دوست
\\
غریبی که پر فتنه باشد سرش
&&
میازار و بیرون کن از کشورش
\\
تو گر خشم بر وی نگیری رواست
&&
که خود خوی بد دشمنش در قفاست
\\
وگر پارسی باشدش زاد و بوم
&&
به صنعاش مفرست و سقلاب و روم
\\
هم آن جا امانش مده تا به چاشت
&&
نشاید بلا بر دگر کس گماشت
\\
که گویند برگشته باد آن زمین
&&
کز او مردم آیند بیرون چنین
\\
قدیمان خود را بیفزای قدر
&&
که هرگز نیاید ز پرورده غدر
\\
چو خدمتگزاریت گردد کهن
&&
حق سالیانش فرامش مکن
\\
گر او را هرم دست خدمت ببست
&&
تو را بر کرم همچنان دست هست
\\
شنیدم که شاپور دم در کشید
&&
چو خسرو به رسمش قلم در کشید
\\
چو شد حالش از بینوایی تباه
&&
نبشت این حکایت به نزدیک شاه
\\
چو بذل تو کردم جوانی خویش
&&
به هنگام پیری مرانم ز پیش
\\
عمل گر دهی مرد منعم شناس
&&
که مفلس ندارد ز سلطان هراس
\\
چو مفلس فرو برد گردن به دوش
&&
از او بر نیاید دگر جز خروش
\\
چو مشرف دو دست از امانت بداشت
&&
بباید بر او ناظری بر گماشت
\\
ور او نیز در ساخت با خاطرش
&&
ز مشرف عمل بر کن و ناظرش
\\
خدا ترس باید امانت گزار
&&
امین کز تو ترسد امینش مدار
\\
امین باید از داور اندیشناک
&&
نه از رفع دیوان و زجر و هلاک
\\
بیفشان و بشمار و فارغ نشین
&&
که از صد یکی را نبینی امین
\\
دو همجنس دیرینه را هم‌قلم
&&
نباید فرستاد یک جا به هم
\\
چه دانی که همدست گردند و یار
&&
یکی دزد باشد، یکی پرده‌دار
\\
چو دزدان ز هم باک دارند و بیم
&&
رود در میان کاروانی سلیم
\\
یکی را که معزول کردی ز جاه
&&
چو چندی برآید ببخشش گناه
\\
بر آوردن کام امیدوار
&&
به از قید بندی شکستن هزار
\\
نویسنده را گر ستون عمل
&&
بیفتد، نبرد طناب امل
\\
به فرمانبران بر شه دادگر
&&
پدروار خشم آورد بر پسر
\\
گهش می‌زند تا شود دردناک
&&
گهی می‌کند آبش از دیده پاک
\\
چو نرمی کنی خصم گردد دلیر
&&
وگر خشم گیری شوند از تو سیر
\\
درشتی و نرمی به هم در به است
&&
چو رگ‌زن که جراح و مرهم نه است
\\
جوانمرد و خوش خوی و بخشنده باش
&&
چو حق بر تو پاشد تو بر خلق پاش
\\
نیامد کس اندر جهان کاو بماند
&&
مگر آن کز او نام نیکو بماند
\\
نمرد آن که ماند پس از وی به جای
&&
پل و خانی و خان و مهمان سرای
\\
هر آن کاو نماند از پسش یادگار
&&
درخت وجودش نیاورد بار
\\
وگر رفت و آثار خیرش نماند
&&
نشاید پس مرگش الحمد خواند
\\
چو خواهی که نامت بود جاودان
&&
مکن نام نیک بزرگان نهان
\\
همین نقش بر خوان پس از عهد خویش
&&
که دیدی پس از عهد شاهان پیش
\\
همین کام و ناز و طرب داشتند
&&
به آخر برفتند و بگذاشتند
\\
یکی نام نیکو ببرد از جهان
&&
یکی رسم بد ماند از او جاودان
\\
به سمع رضا مشنو ایذای کس
&&
وگر گفته آید به غورش برس
\\
گنهکار را عذر نسیان بنه
&&
چو زنهار خواهند زنهار ده
\\
گر آید گنهکاری اندر پناه
&&
نه شرط است کشتن به اول گناه
\\
چو باری بگفتند و نشنید پند
&&
بده گوشمالش به زندان و بند
\\
وگر پند و بندش نیاید بکار
&&
درختی خبیث است بیخش برآر
\\
چو خشم آیدت بر گناه کسی
&&
تأمل کنش در عقوبت بسی
\\
که سهل است لعل بدخشان شکست
&&
شکسته نشاید دگرباره بست
\\
\end{longtable}
\end{center}
