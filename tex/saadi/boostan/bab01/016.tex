\begin{center}
\section*{بخش ۱۶ - حکایت برادران ظالم و عادل و عاقبت ایشان: شنیدم که در مرزی از باختر}
\label{sec:016}
\addcontentsline{toc}{section}{\nameref{sec:016}}
\begin{longtable}{l p{0.5cm} r}
شنیدم که در مرزی از باختر
&&
برادر دو بودند از یک پدر
\\
سپهدار و گردن کش و پیلتن
&&
نکو روی و دانا و شمشیر زن
\\
پدر هر دو را سهمگین مرد یافت
&&
طلبکار جولان و ناورد یافت
\\
برفت آن زمین را دو قسمت نهاد
&&
به هر یک پسر، زآن نصیبی بداد
\\
مبادا که بر یکدگر سر کشند
&&
به پیکار شمشیر کین برکشند
\\
پدر بعد از آن، روزگاری شمرد
&&
به جان آفرین جان شیرین سپرد
\\
اجل بگسلاندش طناب امل
&&
وفاتش فرو بست دست عمل
\\
مقرر شد آن مملکت بر دو شاه
&&
که بی حد و مر بود گنج و سپاه
\\
به حکم نظر در به افتاد خویش
&&
گرفتند هر یک، یکی راه پیش
\\
یکی عدل تا نام نیکو برد
&&
یکی ظلم تا مال گرد آورد
\\
یکی عاطفت سیرت خویش کرد
&&
درم داد و تیمار درویش خورد
\\
بنا کرد و نان داد و لشکر نواخت
&&
شب از بهر درویش، شبخانه ساخت
\\
خزاین تهی کرد و پر کرد جیش
&&
چنان کز خلایق به هنگام عیش
\\
برآمد همی بانگ شادی چو رعد
&&
چو شیراز در عهد بوبکر سعد
\\
خدیو خردمند فرخ نهاد
&&
که شاخ امیدش برومند باد
\\
حکایت شنو کان گو نامجوی
&&
پسندیده پی بود و فرخنده خوی
\\
ملازم به دلداری خاص و عام
&&
ثناگوی حق بامدادان و شام
\\
در آن ملک قارون برفتی دلیر
&&
که شه دادگر بود و درویش سیر
\\
نیامد در ایام او بر دلی
&&
نگویم که خاری که برگ گلی
\\
سرآمد به تأیید ملک از سران
&&
نهادند سر بر خطش سروران
\\
دگر خواست کافزون کند تخت و تاج
&&
بیافزود بر مرد دهقان خراج
\\
طمع کرد در مال بازارگان
&&
بلا ریخت بر جان بیچارگان
\\
به امید بیشی نداد و نخورد
&&
خردمند داند که ناخوب کرد
\\
که تا جمع کرد آن زر از گربزی
&&
پراکنده شد لشکر از عاجزی
\\
شنیدند بازارگانان خبر
&&
که ظلم است در بوم آن بی‌هنر
\\
بریدند از آنجا خرید و فروخت
&&
زراعت نیامد، رعیت بسوخت
\\
چو اقبالش از دوستی سر بتافت
&&
به ناکام دشمن بر او دست یافت
\\
ستیز فلک بیخ و بارش بکند
&&
سم اسب دشمن دیارش بکند
\\
وفا در که جوید چو پیمان گسیخت؟
&&
خراج از که خواهد چو دهقان گریخت؟
\\
چه نیکی طمع دارد آن بی‌صفا
&&
که باشد دعای بدش در قفا؟
\\
چو بختش نگون بود در کاف کن
&&
نکرد آنچه نیکانش گفتند کن
\\
چه گفتند نیکان بدان نیکمرد؟
&&
تو برخور که بیدادگر بر نخورد
\\
گمانش خطا بود و تدبیر سست
&&
که در عدل بود آنچه در ظلم جست
\\
یکی بر سر شاخ، بن می‌برید
&&
خداوند بستان نگه کرد و دید
\\
بگفتا گر این مرد بد می‌کند
&&
نه با من که با نفس خود می‌کند
\\
نصیحت بجای است اگر بشنوی
&&
ضعیفان میفکن به کتف قوی
\\
که فردا به داور برد خسروی
&&
گدایی که پیشت نیرزد جوی
\\
چو خواهی که فردا به وی مهتری
&&
مکن دشمن خویشتن، کهتری
\\
که چون بگذرد بر تو این سلطنت
&&
بگیرد به قهر آن گدا دامنت
\\
مکن، پنجه از ناتوانان بدار
&&
که گر بفکنندت شوی شرمسار
\\
که زشت است در چشم آزادگان
&&
بیافتادن از دست افتادگان
\\
بزرگان روشندل نیکبخت
&&
به فرزانگی تاج بردند و تخت
\\
به دنباله راستان کج مرو
&&
وگر راست خواهی ز سعدی شنو
\\
\end{longtable}
\end{center}
