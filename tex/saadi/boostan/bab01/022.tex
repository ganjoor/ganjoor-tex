\begin{center}
\section*{بخش ۲۲ - در نواخت رعیت و رحمت بر افتادگان: مدر پرده کس به هنگام جنگ}
\label{sec:022}
\addcontentsline{toc}{section}{\nameref{sec:022}}
\begin{longtable}{l p{0.5cm} r}
مدر پرده کس به هنگام جنگ
&&
که باشد تو را نیز در پرده ننگ
\\
مزن بانگ بر شیرمردان درشت
&&
چو با کودکان برنیایی به مشت
\\
یکی پند می‌داد فرزند را
&&
نگه دار پند خردمند را
\\
مکن جور بر خردکان ای پسر
&&
که یک روزت افتد بزرگی به سر
\\
نمی‌ترسی ای گرگگ کم خرد
&&
که روزی پلنگیت بر هم درد؟
\\
به خردی درم زور سرپنجه بود
&&
دل زیردستان ز من رنجه بود
\\
بخوردم یکی مشت زورآوران
&&
نکردم دگر زور بر لاغران
\\
الا تا به غفلت نخفتی که نوم
&&
حرام است بر چشم سالار قوم
\\
غم زیردستان بخور زینهار
&&
بترس از زبردستی روزگار
\\
نصیحت که خالی بود از غرض
&&
چو داروی تلخ است، دفع مرض
\\
\end{longtable}
\end{center}
