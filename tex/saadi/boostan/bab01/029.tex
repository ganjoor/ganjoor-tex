\begin{center}
\section*{بخش ۲۹ - حکایت مأمون با کنیزک: چو دور خلافت به مأمون رسید}
\label{sec:029}
\addcontentsline{toc}{section}{\nameref{sec:029}}
\begin{longtable}{l p{0.5cm} r}
چو دور خلافت به مأمون رسید
&&
یکی ماه پیکر کنیزک خرید
\\
به چهر آفتابی، به تن گلبنی
&&
به عقل خردمند بازی کنی
\\
به خون عزیزان فرو برده چنگ
&&
سر انگشتها کرده عناب رنگ
\\
بر ابروی عابد فریبش خضاب
&&
چو قوس قزح بود بر آفتاب
\\
شب خلوت آن لعبت حور زاد
&&
مگر تن در آغوش مأمون نداد
\\
گرفت آتش خشم در وی عظیم
&&
سرش خواست کردن چو جوزا دو نیم
\\
بگفتا سر اینک به شمشیر تیز
&&
بینداز و با من مکن خفت و خیز
\\
بگفت از چه بر دل گزند آمدت؟
&&
چه خصلت ز من ناپسند آمدت؟
\\
بگفت ار کشی ور شکافی سرم
&&
ز بوی دهانت به رنج اندرم
\\
کشد تیر پیکار و تیغ ستم
&&
به یک بار و بوی دهن دم به دم
\\
شنید این سخن سرور نیکبخت
&&
برآشفت تند و برنجید سخت
\\
همه شب در این فکر بود و نخفت
&&
دگر روز با هوشمندان بگفت
\\
طبیعت شناسان هر کشوری
&&
سخن گفت با هر یک از هر دری
\\
دلش گر چه در حال از او رنجه شد
&&
دوا کرد و خوشبوی چون غنچه شد
\\
پری چهره را همنشین کرد و دوست
&&
که این عیب من گفت، یار من اوست
\\
به نزد من آن کس نکوخواه توست
&&
که گوید فلان خار در راه توست
\\
به گمراه گفتن نکو می‌روی
&&
جفایی تمام است و جوری قوی
\\
هر آن گه که عیبت نگویند پیش
&&
هنر دانی از جاهلی عیب خویش
\\
مگو شهد شیرین شکر فایق است
&&
کسی را که سقمونیا لایق است
\\
چه خوش گفت یک روز دارو فروش:
&&
شفا بایدت داروی تلخ نوش
\\
اگر شربتی بایدت سودمند
&&
ز سعدی ستان تلخ داروی پند
\\
به پرویزن معرفت بیخته
&&
به شهد ظرافت برآمیخته
\\
\end{longtable}
\end{center}
