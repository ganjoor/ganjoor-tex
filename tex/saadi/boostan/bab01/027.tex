\begin{center}
\section*{بخش ۲۷ - حکایت: چو الب ارسلان جان به جان‌بخش داد}
\label{sec:027}
\addcontentsline{toc}{section}{\nameref{sec:027}}
\begin{longtable}{l p{0.5cm} r}
چو الب ارسلان جان به جان‌بخش داد
&&
پسر تاج شاهی به سر برنهاد
\\
به تربت سپردندش از تاجگاه
&&
نه جای نشستن بد آماجگاه
\\
چنین گفت دیوانه‌ای هوشیار
&&
چو دیدش پسر روز دیگر سوار
\\
زهی ملک و دوران سر در نشیب
&&
پدر رفت و پای پسر در رکیب
\\
چنین است گردیدن روزگار
&&
سبک سیر و بدعهد و ناپایدار
\\
چو دیرینه روزی سرآورد عهد
&&
جوان دولتی سر برآرد ز مهد
\\
منه بر جهان دل که بیگانه‌ای است
&&
چو مطرب که هر روز در خانه‌ای است
\\
نه لایق بود عیش با دلبری
&&
که هر بامدادش بود شوهری
\\
نکویی کن امسال چون ده تو راست
&&
که سال دگر دیگری دهخداست
\\
\end{longtable}
\end{center}
