\begin{center}
\section*{بخش ۱۲ - گفتار اندر نگه داشتن خاطر درویشان: مها زورمندی مکن با کهان}
\label{sec:012}
\addcontentsline{toc}{section}{\nameref{sec:012}}
\begin{longtable}{l p{0.5cm} r}
مها زورمندی مکن با کهان
&&
که بر یک نمط می‌نماند جهان
\\
سر پنجهٔ ناتوان بر مپیچ
&&
که گر دست یابد برآیی به هیچ
\\
عدو را به کوچک نباید شمرد
&&
که کوه کلان دیدم از سنگ خرد
\\
نبینی که چون با هم آیند مور
&&
ز شیران جنگی برآرند شور
\\
نه موری که مویی کز آن کمتر است
&&
چو پر شد ز زنجیر محکمتر است
\\
مبر گفتمت پای مردم ز جای
&&
که عاجز شوی گر درآ یی ز پای
\\
دل دوستان جمع بهتر که گنج
&&
خزینه تهی به که مردم به رنج
\\
مینداز در پای کار کسی
&&
که افتد که در پایش افتی بسی
\\
تحمل کن ای ناتوان از قوی
&&
که روزی تواناتر از وی شوی
\\
به همت برآر از ستیهنده شور
&&
که بازوی همت به از دست زور
\\
لب خشک مظلوم را گو بخند
&&
که دندان ظالم بخواهند کند
\\
به بانگ دهل خواجه بیدار گشت
&&
چه داند شب پاسبان چون گذشت؟
\\
خورد کاروانی غم بار خویش
&&
نسوزد دلش بر خر پشت ریش
\\
گرفتم کز افتادگان نیستی
&&
چو افتاده بینی چرا نیستی؟
\\
بر اینت بگویم یکی سرگذشت
&&
که سستی بود زین سخن درگذشت
\\
\end{longtable}
\end{center}
