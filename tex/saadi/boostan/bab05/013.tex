\begin{center}
\section*{بخش ۱۳ - گفتار اندر اخلاص و برکت آن و ریا و آفت آن: عبادت به اخلاص نیت نکوست}
\label{sec:013}
\addcontentsline{toc}{section}{\nameref{sec:013}}
\begin{longtable}{l p{0.5cm} r}
عبادت به اخلاص نیت نکوست
&&
وگر نه چه آید ز بی مغز پوست؟
\\
چه زنار مغ در میانت چه دلق
&&
که در پوشی از بهر پندار خلق
\\
مکن گفتمت مردی خویش فاش
&&
چو مردی نمودی مخنث مباش
\\
به اندازهٔ بود باید نمود
&&
خجالت نبرد آن که ننمود و بود
\\
که چون عاریت بر کنند از سرش
&&
نماید کهن جامه‌ای در برش
\\
اگر کوتهی پای چوبین مبند
&&
که در چشم طفلان نمایی بلند
\\
وگر نقره اندوده باشد نحاس
&&
توان خرج کردن بر ناشناس
\\
منه جان من آب زر بر پشیز
&&
که صراف دانا نگیرد به چیز
\\
زر اندودگان را به آتش برند
&&
پدید آید آنگه که مس یا زرند
\\
ندانی که بابای کوهی چه گفت
&&
به مردی که ناموس را شب نخفت؟
\\
برو جان بابا در اخلاص پیچ
&&
که نتوانی از خلق رستن به هیچ
\\
کسانی که فعلت پسندیده‌اند
&&
هنوز از تو نقش برون دیده‌اند
\\
چه قدر آورد بنده حوردیس
&&
که زیر قبا دارد اندام پیس؟
\\
نشاید به دستان شدن در بهشت
&&
که بازت رود چادر از روی زشت
\\
\end{longtable}
\end{center}
