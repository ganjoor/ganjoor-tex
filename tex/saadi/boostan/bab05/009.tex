\begin{center}
\section*{بخش ۹ - حکایت: یکی مرد درویش در خاک کیش}
\label{sec:009}
\addcontentsline{toc}{section}{\nameref{sec:009}}
\begin{longtable}{l p{0.5cm} r}
یکی پیر درویش در خاک کیش
&&
چه خوش گفت با همسر زشت خویش
\\
چو دست قضا زشت رویت سرشت
&&
میندای گلگونه بر روی زشت
\\
که حاصل کند نیکبختی به زور؟
&&
به سرمه که بینا کند چشم کور؟
\\
نیاید نکوکار از بد رگان
&&
محال است دوزندگی از سگان
\\
همه فیلسوفان یونان و روم
&&
ندانند کرد انگبین از زقوم
\\
ز وحشی نیاید که مردم شود
&&
به سعی اندر او تربیت گم شود
\\
توان پاک کردن ز زنگ آینه
&&
ولیکن نیاید ز سنگ آینه
\\
به کوشش نه روید گل از شاخ بید
&&
نه زنگی به گرمابه گردد سپید
\\
چو رد می‌نگردد خدنگ قضا
&&
سپر نیست مر بنده را جز رضا
\\
\end{longtable}
\end{center}
