\begin{center}
\section*{بخش ۱۰ - حکایت کرکس با زغن: چنین گفت پیش زغن کرکسی}
\label{sec:010}
\addcontentsline{toc}{section}{\nameref{sec:010}}
\begin{longtable}{l p{0.5cm} r}
چنین گفت پیش زغن کرکسی
&&
که نبود ز من دوربین‌تر کسی
\\
زغن گفت از این در نشاید گذشت
&&
بیا تا چه بینی بر اطراف دشت
\\
شنیدم که مقدار یک روزه راه
&&
بکرد از بلندی به پستی نگاه
\\
چنین گفت دیدم گرت باور است
&&
که یک دانه گندم به هامون بر است
\\
زغن را نماند از تعجب شکیب
&&
ز بالا نهادند سر در نشیب
\\
چو کرکس بر دانه آمد فراز
&&
گره شد بر او پایبندی دراز
\\
ندانست از آن دانه‌ای خوردنش
&&
که دهر افکند دام در گردنش
\\
نه آبستن در بود هر صدف
&&
نه هر بار شاطر زند بر هدف
\\
زغن گفت از آن دانه دیدن چه سود
&&
چو بینایی دام خصمت نبود؟
\\
شنیدم که می‌گفت و گردن به بند
&&
نباشد حذر با قدر سودمند
\\
اجل چون به خونش بر آورد دست
&&
قضا چشم باریک بینش ببست
\\
در آبی که پیدا نگردد کنار
&&
غرور شناور نیاید به کار
\\
\end{longtable}
\end{center}
