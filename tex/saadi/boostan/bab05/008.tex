\begin{center}
\section*{بخش ۸ - حکایت مرد درویش و همسایهٔ توانگر: بلند اختری نام او بختیار}
\label{sec:008}
\addcontentsline{toc}{section}{\nameref{sec:008}}
\begin{longtable}{l p{0.5cm} r}
بلند اختری نام او بختیار
&&
قوی دستگه بود و سرمایه‌دار
\\
به کوی گدایان درش خانه بود
&&
زرش همچو گندم به پیمانه بود
\\
چو درویش بیند توانگر به ناز
&&
دلش بیش سوزد به داغ نیاز
\\
زنی جنگ پیوست با شوی خویش
&&
شبانگه چو رفتش تهیدست، پیش
\\
که کس چون تو بدبخت، درویش نیست
&&
چو زنبور سرخت جز این نیش نیست
\\
بیاموز مردی ز همسایگان
&&
که آخر نیم قحبهٔ رایگان
\\
کسان را زر و سیم و ملک است و رخت
&&
چرا همچو ایشان نه‌ای نیکبخت؟
\\
بر آورد صافی دل صوف پوش
&&
چو طبل از تهیگاه خالی خروش
\\
که من دست قدرت ندارم به هیچ
&&
به سرپنجه دست قضا بر مپیچ
\\
نکردند در دست من اختیار
&&
که من خویشتن را کنم بختیار
\\
\end{longtable}
\end{center}
