\begin{center}
\section*{بخش ۱ - سر آغاز: خدا را ندانست و طاعت نکرد}
\label{sec:001}
\addcontentsline{toc}{section}{\nameref{sec:001}}
\begin{longtable}{l p{0.5cm} r}
خدا را ندانست و طاعت نکرد
&&
که بر بخت و روزی قناعت نکرد
\\
قناعت توانگر کند مرد را
&&
خبر کن حریص جهانگرد را
\\
سکونی به دست آور ای بی ثبات
&&
که بر سنگ گردان نروید نبات
\\
مپرور تن ار مرد رای و هشی
&&
که او را چو می‌پروری می‌کشی
\\
خردمند مردم هنر پرورند
&&
که تن پروران از هنر لاغرند
\\
کسی سیرت آدمی گوش کرد
&&
که اول سگ نفس خاموش کرد
\\
خور و خواب تنها طریق دد است
&&
بر این بودن آیین نابخرد است
\\
خنک نیکبختی که در گوشه‌ای
&&
به دست آرد از معرفت توشه‌ای
\\
بر آنان که شد سر حق آشکار
&&
نکردند باطل بر او اختیار
\\
ولیکن چو ظلمت نداند ز نور
&&
چه دیدار دیوش چه رخسار حور
\\
تو خود را از آن در چه انداختی
&&
که چه را ز ره باز نشناختی
\\
بر اوج فلک چون پرد جره باز
&&
که در شهپرش بسته‌ای سنگ آز؟
\\
گرش دامن از چنگ شهوت رها
&&
کنی، رفت تا سدرةالمنتهی
\\
به کم کردن از عادت خویش خورد
&&
توان خویشتن را ملک خوی کرد
\\
کجا سیر وحشی رسد در ملک
&&
نشاید پرید از ثری بر فلک
\\
نخست آدمی سیرتی پیشه کن
&&
پس آن گه ملک خویی اندیشه کن
\\
تو بر کرهٔ توسنی بر کمر
&&
نگر تا نپیچد ز حکم تو سر
\\
که گر پالهنگ از کفت در گسیخت
&&
تن خویشتن کشت و خون تو ریخت
\\
به اندازه خور زاد اگر مردمی
&&
چنین پر شکم، آدمی یا خمی؟
\\
درون جای قوت است و ذکر و نفس
&&
تو پنداری از بهر نان است و بس
\\
کجا ذکر گنجد در انبان آز؟
&&
به سختی نفس می‌کند پا دراز
\\
ندارند تن پروران آگهی
&&
که پر معده باشد ز حکمت تهی
\\
دو چشم و شکم پر نگردد به هیچ
&&
تهی بهتر این رودهٔ پیچ پیچ
\\
چو دوزخ که سیرش کنند از وقید
&&
دگر بانگ دارد که هل من مزید؟
\\
همی میردت عیسی از لاغری
&&
تو در بند آنی که خر پروی
\\
به دین، ای فرومایه، دنیا مخر
&&
تو خر را به انجیل عیسی مخر
\\
مگر می‌نبینی که دد را و دام
&&
نینداخت جز حرص خوردن به دام؟
\\
پلنگی که گردن کشد بر وحوش
&&
به دام افتد از بهر خوردن چو موش
\\
چو موش آن که نان و پنیرش خوری
&&
به دامش در افتی و تیرش خوری
\\
\end{longtable}
\end{center}
