\begin{center}
\section*{بخش ۴ - حکایت در معنی سلامت جاهل در خاموشی: یکی خوب خلق خلق پوش بود}
\label{sec:004}
\addcontentsline{toc}{section}{\nameref{sec:004}}
\begin{longtable}{l p{0.5cm} r}
یکی خوب خلق خلق پوش بود
&&
که در مصر یک چند خاموش بود
\\
خردمند مردم ز نزدیک و دور
&&
به گردش چو پروانه جویان نور
\\
تفکر شبی با دل خویش کرد
&&
که پوشیده زیر زبان است مرد
\\
اگر همچنین سر به خود در برم
&&
چه دانند مردم که دانشورم؟
\\
سخن گفت و دشمن بدانست و دوست
&&
که در مصر نادان تر از وی هم اوست
\\
حضورش پریشان شد و کار زشت
&&
سفر کرد و بر طاق مسجد نبشت
\\
در آیینه گر خویشتن دیدمی
&&
به بی دانشی پرده ندریدمی
\\
چنین زشت از آن پرده برداشتم
&&
که خود را نکو روی پنداشتم
\\
کم آواز را باشد آوازه تیز
&&
چو گفتی و رونق نماندت گریز
\\
تو را خامشی ای خداوند هوش
&&
وقار است و، نا اهل را پرده پوش
\\
اگر عالمی هیبت خود مبر
&&
وگر جاهلی پردهٔ خود مدر
\\
ضمیر دل خویش منمای زود
&&
که هر گه که خواهی توانی نمود
\\
ولیکن چو پیدا شود راز مرد
&&
به کوشش نشاید نهان باز کرد
\\
قلم سر سلطان چه نیکو نهفت
&&
که تا کارد بر سر نبودش نگفت
\\
بهایم خموشند و گویا بشر
&&
زبان بسته بهتر که گویا به شر
\\
چو مردم سخن گفت باید به هوش
&&
وگر نه شدن چون بهایم خموش
\\
به نطق است و عقل آدمی‌زاده فاش
&&
چو طوطی سخنگوی نادان مباش
\\
به نطق آدمی بهتر است از دواب
&&
دواب از تو به گر نگویی ثواب
\\
\end{longtable}
\end{center}
