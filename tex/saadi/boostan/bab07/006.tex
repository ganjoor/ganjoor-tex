\begin{center}
\section*{بخش ۶ - حکایت عضد و مرغان خوش آواز: عضد را پسر سخت رنجور بود}
\label{sec:006}
\addcontentsline{toc}{section}{\nameref{sec:006}}
\begin{longtable}{l p{0.5cm} r}
عضد را پسر سخت رنجور بود
&&
شکیب از نهاد پدر دور بود
\\
یکی پارسا گفتش از روی پند
&&
که بگذار مرغان وحشی ز بند
\\
قفسهای مرغ سحر خوان شکست
&&
که در بند ماند چو زندان شکست؟
\\
نگه داشت بر طاق بستان سرای
&&
یکی نامور بلبل خوش‌سرای
\\
پسر صبحدم سوی بستان شتافت
&&
جز آن مرغ بر طاق ایوان نیافت
\\
بخندید کای بلبل خوش نفس
&&
تو از گفت خود مانده‌ای در قفس
\\
ندارد کسی با تو ناگفته کار
&&
ولیکن چو گفتی دلیلش بیار
\\
چو سعدی که چندی زبان بسته بود
&&
ز طعن زبان آوران رسته بود
\\
کسی گیرد آرام دل در کنار
&&
که از صحبت خلق گیرد کنار
\\
مکن عیب خلق، ای خردمند، فاش
&&
به عیب خود از خلق مشغول باش
\\
چو باطل سرایند مگمار گوش
&&
چو بی‌ستر بینی بصیرت بپوش
\\
\end{longtable}
\end{center}
