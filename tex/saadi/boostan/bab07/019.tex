\begin{center}
\section*{بخش ۱۹ - حکایت اندر نکوهش غمازی و مذلت غمازان: یکی گفت با صوفیی در صفا}
\label{sec:019}
\addcontentsline{toc}{section}{\nameref{sec:019}}
\begin{longtable}{l p{0.5cm} r}
یکی گفت با صوفیی در صفا
&&
ندانی فلانت چه گفت از قفا
\\
بگفتا خموش، ای برادر، بخفت
&&
ندانسته بهتر که دشمن چه گفت
\\
کسانی که پیغام دشمن برند
&&
ز دشمن همانا که دشمن ترند
\\
کسی قول دشمن نیارد به دوست
&&
جز آن کس که در دشمنی یار اوست
\\
نیارست دشمن جفا گفتنم
&&
چنان کز شنیدن بلرزد تنم
\\
تو دشمن‌تری کآوری بر دهان
&&
که دشمن چنین گفت اندر نهان
\\
سخن چین کند تازه جنگ قدیم
&&
به خشم آورد نیکمرد سلیم
\\
از آن همنشین تا توانی گریز
&&
که مر فتنهٔ خفته را گفت خیز
\\
سیه چال و مرد اندر او بسته پای
&&
به از فتنه از جای بردن به جای
\\
میان دو تن جنگ چون آتش است
&&
سخن‌چین بدبخت هیزم کش است
\\
\end{longtable}
\end{center}
