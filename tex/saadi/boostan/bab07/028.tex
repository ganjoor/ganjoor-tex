\begin{center}
\section*{بخش ۲۸ - گفتار اندر سلامت گوشه‌نشینی و صبر  بر ایذاء خلق: اگر در جهان از جهان رسته‌ای است،}
\label{sec:028}
\addcontentsline{toc}{section}{\nameref{sec:028}}
\begin{longtable}{l p{0.5cm} r}
اگر در جهان از جهان رسته‌ای است،
&&
در از خلق بر خویشتن بسته‌ای است
\\
کس از دست جور زبانها نرست
&&
اگر خودنمای است و گر حق پرست
\\
اگر برپری چون ملک ز آسمان
&&
به دامن در آویزدت بدگمان
\\
به کوشش توان دجله را پیش بست
&&
نشاید زبان بداندیش بست
\\
فرا هم نشینند تردامنان
&&
که این زهد خشک است و آن دام نان
\\
تو روی از پرستیدن حق مپیچ
&&
بهل تا نگیرند خلقت به هیچ
\\
چو راضی شد از بنده یزدان پاک
&&
گر اینها نگردند راضی چه باک؟
\\
بد اندیش خلق از حق آگاه نیست
&&
ز غوغای خلقش به حق راه نیست
\\
از آن ره به جایی نیاورده‌اند
&&
که اول قدم پی غلط کرده‌اند
\\
دو کس بر حدیثی گمارند گوش
&&
از این تا بدان، ز اهرمن تا سروش
\\
یکی پند گیرد دگر ناپسند
&&
نپردازد از حرفگیری به پند
\\
فرو مانده در کنج تاریک جای
&&
چه دریابد از جام گیتی نمای؟
\\
مپندار اگر شیر و گر روبهی
&&
کز اینان به مردی و حیلت رهی
\\
اگر کنج خلوت گزیند کسی
&&
که پروای صحبت ندارد بسی
\\
مذمت کنندش که زرق است و ریو
&&
ز مردم چنان می گریزد که دیو
\\
وگر خنده روی است و آمیزگار
&&
عفیفش ندانند و پرهیزگار
\\
غنی را به غیبت بکاوند پوست
&&
که فرعون اگر هست در عالم اوست
\\
وگر بینوایی بگرید به سوز
&&
نگون بخت خوانندش و تیره‌روز
\\
وگر کامرانی در آید ز پای
&&
غنیمت شمارند و فضل خدای
\\
که تا چند از این جاه و گردن کشی؟
&&
خوشی را بود در قفا ناخوشی
\\
و گر تنگدستی تنک مایه‌ای
&&
سعادت بلندش کند پایه‌ای
\\
بخایندش از کینه دندان به زهر
&&
که دون پرور است این فرومایه دهر
\\
چو بینند کاری به دستت در است
&&
حریصت شمارند و دنیا پرست
\\
وگر دست همت بداری ز کار
&&
گدا پیشه خوانندت و پخته خوار
\\
اگر ناطقی طبل پر یاوه‌ای
&&
وگر خامشی نقش گرماوه‌ای
\\
تحمل کنان را نخوانند مرد
&&
که بیچاره از بیم سر برنکرد
\\
وگر در سرش هول و مردانگی است
&&
گریزند از او کاین چه دیوانگی است؟!
\\
تعنت کنندش گر اندک خوری است
&&
که مالش مگر روزی دیگری است
\\
وگر نغز و پاکیزه باشد خورش
&&
شکم بنده خوانند و تن پرورش
\\
وگر بی تکلف زید مالدار
&&
که زینت بر اهل تمیز است عار
\\
زبان در نهندش به ایذا چو تیغ
&&
که بدبخت زر دارد از خود دریغ
\\
و گر کاخ و ایوان منقش کند
&&
تن خویش را کسوتی خوش کند
\\
به جان آید از طعنه بر وی زنان
&&
که خود را بیاراست همچون زنان
\\
اگر پارسایی سیاحت نکرد
&&
سفر کردگانش نخوانند مرد
\\
که نارفته بیرون ز آغوش زن
&&
کدامش هنر باشد و رای و فن؟
\\
جهاندیده را هم بدرند پوست
&&
که سرگشتهٔ بخت برگشته اوست
\\
گرش حظ از اقبال بودی و بهر
&&
زمانه نراندی ز شهرش به شهر
\\
عزب را نکوهش کند خرده بین
&&
که می‌رنجد از خفت و خیزش زمین
\\
وگر زن کند گوید از دست دل
&&
به گردن در افتاد چون خر به گل
\\
نه از جور مردم رهد زشت روی
&&
نه شاهد ز نامردم زشت گوی
\\
غلامی به مصر اندرم بنده بود
&&
که چشم از حیا در بر افکنده بود
\\
کسی گفت: «هیچ این پسر عقل و هوش
&&
ندارد، بمالش به تعلیم گوش»
\\
شبی بر زدم بانگ بر وی درشت
&&
هم او گفت: «مسکین به جورش بکشت!»
\\
گرت برکند خشم روزی ز جای
&&
سراسیمه خوانندت و تیره رای
\\
وگر بردباری کنی از کسی
&&
بگویند غیرت ندارد بسی
\\
سخی را به اندرز گویند: «بس!
&&
که فردا دو دستت بود پیش و پس»
\\
وگر قانع و خویشتن‌دار گشت
&&
به تشنیع خلقی گرفتار گشت
\\
که همچون پدر خواهد این سفله مرد
&&
که نعمت رها کرد و حسرت ببرد
\\
که یارد به کنج سلامت نشست؟
&&
که پیغمبر از خبث ایشان نرست
\\
خدا را که مانند و انباز و جفت
&&
ندارد، شنیدی که ترسا چه گفت؟
\\
رهایی نیابد کس از دست کس
&&
گرفتار را چاره صبر است و بس
\\
\end{longtable}
\end{center}
