\begin{center}
\section*{بخش ۲۶ - حکایت: در این شهرباری به سمعم رسید}
\label{sec:026}
\addcontentsline{toc}{section}{\nameref{sec:026}}
\begin{longtable}{l p{0.5cm} r}
در این شهر باری به سمعم رسید
&&
که بازارگانی غلامی خرید
\\
شبانگه مگر دست بردش به سیب
&&
که سیمین زنخ بود و خاطر فریب
\\
پریچهره هرچ اوفتادش به دست
&&
یکی در سر و مغز خواجه شکست
\\
نه هر جا که بینی خطی دل فریب
&&
توانی طمع کردنش در کتیب
\\
گوا کرد بر خود خدای و رسول
&&
که دیگر نگردم به گرد فضول
\\
رحیل آمدش هم در آن هفته پیش
&&
دل افگار و سر بسته و روی ریش
\\
چو بیرون شد از کازرون یک دو میل
&&
به پیش آمدش سنگلاخی مهیل
\\
بپرسید کاین قله را نام چیست؟
&&
که بسیار بیند عجب هر که زیست
\\
چنین گفتش از کاروان همدمی
&&
مگر تنگ ترکان ندانی همی
\\
برنجید چون تنگ ترکان شنید
&&
تو گفتی که دیدار دشمن بدید
\\
سیه را یکی بانگ برداشت سخت
&&
که دیگر مران خر بینداز رخت
\\
نه عقل است و نه معرفت یک جوم
&&
اگر من دگر تنگ ترکان روم
\\
در شهوت نفس کافر ببند
&&
وگر عاشقی لت خور و سر ببند
\\
چو مر بنده‌ای را همی پروری
&&
به هیبت بر آرش کز او برخوری
\\
وگر سیدش لب به دندان گزد
&&
دماغ خداوندگاری پزد
\\
غلام آبکش باید و خشت زن
&&
بود بندهٔ نازنین مشت زن
\\
گروهی نشینند با خوش پسر
&&
که ما پاکبازیم و صاحب نظر
\\
ز من پرس فرسودهٔ روزگار
&&
که بر سفره حسرت خورد روزه‌دار
\\
از آن تخم خرما خورد گوسپند
&&
که قفل است بر تنگ خرما و بند
\\
سر گاو عصار از آن در که است
&&
که از کنجدش ریسمان کوته است
\\
\end{longtable}
\end{center}
