\begin{center}
\section*{بخش ۳ - حکایت بت پرست نیازمند: مغی در به روی از جهان بسته بود}
\label{sec:003}
\addcontentsline{toc}{section}{\nameref{sec:003}}
\begin{longtable}{l p{0.5cm} r}
مغی در به روی از جهان بسته بود
&&
بتی را به خدمت میان بسته بود
\\
پس از چند سال آن نکوهیده کیش
&&
قضا حالتی صعبش آورد پیش
\\
به پای بت اندر به امید خیر
&&
بغلطید بیچاره بر خاک دیر
\\
که درمانده‌ام دست گیر ای صنم
&&
به جان آمدم رحم کن بر تنم
\\
بزارید در خدمتش بارها
&&
که هیچش به سامان نشد کارها
\\
بتی چون برآرد مهمات کس
&&
که نتواند از خود براندن مگس؟
\\
برآشفت کای پای بند ضلال
&&
به باطل پرستیدمت چند سال
\\
مهمی که در پیش دارم برآر
&&
و گر نه بخواهم ز پروردگار
\\
هنوز از بت آلوده رویش به خاک
&&
که کامش برآورد یزدان پاک
\\
حقایق شناسی در این خیره شد
&&
سر وقت صافی بر او تیره شد
\\
که سرگشته‌ای دون یزدان پرست
&&
هنوزش سر از خمر بتخانه مست
\\
دل از کفر و دست از خیانت نشست
&&
خدایش برآورد کامی که جست
\\
فرو رفت خاطر در این مشکلش
&&
که پیغامی آمد به گوش دلش
\\
که پیش صنم پیر ناقص عقول
&&
بسی گفت و قولش نیامد قبول
\\
گر از درگه ما شود نیز رد
&&
پس آنگه چه فرق از صنم تا صمد؟
\\
دل اندر صمد باید ای دوست بست
&&
که عاجزترند از صنم هر که هست
\\
محال است اگر سر بر این در نهی
&&
که باز آیدت دست حاجت تهی
\\
خدایا مقصر به کار آمدیم
&&
تهیدست و امیدوار آمدیم
\\
\end{longtable}
\end{center}
