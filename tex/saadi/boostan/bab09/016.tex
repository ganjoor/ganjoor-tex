\begin{center}
\section*{بخش ۱۶ - حکایت مست خرمن سوز: یکی غله مرداد مه توده کرد}
\label{sec:016}
\addcontentsline{toc}{section}{\nameref{sec:016}}
\begin{longtable}{l p{0.5cm} r}
یکی غله مرداد مه توده کرد
&&
ز تیمار دی خاطر آسوده کرد
\\
شبی مست شد و آتشی برفروخت
&&
نگون بخت کالیوه، خرمن بسوخت
\\
دگر روز در خوشه چینی نشست
&&
که یک جو ز خرمن نماندش به دست
\\
چو سرگشته دیدند درویش را
&&
یکی گفت پروردهٔ خویش را
\\
نخواهی که باشی چنین تیره روز
&&
به دیوانگی خرمن خود مسوز
\\
گر از دست شد عمرت اندر بدی
&&
تو آنی که در خرمن آتش زدی
\\
فضیحت بود خوشه اندوختن
&&
پس از خرمن خویشتن سوختن
\\
مکن جان من، تخم دین ورز و داد
&&
مده خرمن نیکنامی به باد
\\
چو برگشته بختی در افتد به بند
&&
از او نیک‌بختان بگیرند پند
\\
تو پیش از عقوبت در عفو کوب
&&
که سودی ندارد فغان زیر چوب
\\
بر آر از گریبان غفلت سرت
&&
که فردا نماند خجل در برت
\\
\end{longtable}
\end{center}
