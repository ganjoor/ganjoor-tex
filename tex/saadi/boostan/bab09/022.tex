\begin{center}
\section*{بخش ۲۲ - حکایت: به صنعا درم طفلی اندر گذشت}
\label{sec:022}
\addcontentsline{toc}{section}{\nameref{sec:022}}
\begin{longtable}{l p{0.5cm} r}
به صنعا درم طفلی اندر گذشت
&&
چه گویم کز آنم چه بر سر گذشت
\\
قضا نقش یوسف جمالی نکرد
&&
که ماهی گورش چو یونس نخورد
\\
در این باغ سروی نیامد بلند
&&
که باد اجل بیخش از بن نکند
\\
نهالی به سی سال گردد درخت
&&
ز بیخش بر آرد یکی باد سخت
\\
عجب نیست بر خاک اگر گل شکفت
&&
که چندین گل‌اندام در خاک خفت
\\
به دل گفتم ای ننگ مردان بمیر
&&
که کودک رود پاک و آلوده پیر
\\
ز سودا و آشفتگی بر قدش
&&
برانداختم سنگی از مرقدش
\\
ز هولم در آن جای تاریک و تنگ
&&
بشورید حال و بگردید رنگ
\\
چو باز آمدم زآن تغیر به هوش
&&
ز فرزند دلبندم آمد به گوش:
\\
گرت وحشت آمد ز تاریک جای
&&
به هش باش و با روشنایی در آی
\\
شب گور خواهی منور چو روز
&&
از اینجا چراغ عمل برفروز
\\
تن کارکن می‌بلرزد ز تب
&&
مبادا که نخلش نیارد رطب
\\
گروهی فراوان طمع ظن برند
&&
که گندم نیفشانده خرمن برند
\\
بر آن خورد سعدی که بیخی نشاند
&&
کسی برد خرمن که تخمی فشاند
\\
\end{longtable}
\end{center}
