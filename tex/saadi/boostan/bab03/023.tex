\begin{center}
\section*{بخش ۲۳ - حکایت: شکر لب جوانی نی آموختی}
\label{sec:023}
\addcontentsline{toc}{section}{\nameref{sec:023}}
\begin{longtable}{l p{0.5cm} r}
شکر لب جوانی نی آموختی
&&
که دلها در آتش چو نی سوختی
\\
پدر بارها بانگ بر وی زدی
&&
به تندی و آتش در آن نی زدی
\\
شبی بر ادای پسر گوش کرد
&&
سماعش پریشان و مدهوش کرد
\\
همی گفت و بر چهره افکنده خوی
&&
که آتش به من در زد این بار نی
\\
ندانی که شوریده حالان مست
&&
چرا بر فشانند در رقص دست؟
\\
گشاید دری بر دل از واردات
&&
فشاند سر دست بر کاینات
\\
حلالش بود رقص بر یاد دوست
&&
که هر آستینیش جانی در اوست
\\
گرفتم که مردانه‌ای در شنا
&&
برهنه توانی زدن دست و پا
\\
بکن خرقه نام و ناموس و زرق
&&
که عاجز بود مرد با جامه غرق
\\
تعلق حجاب است و بی حاصلی
&&
چو پیوندها بگسلی واصلی
\\
\end{longtable}
\end{center}
