\begin{center}
\section*{بخش ۱۸ - گفتار در معنی فنای موجودات در معرض وجود باری: ره عقل جز پیچ بر پیچ نیست}
\label{sec:018}
\addcontentsline{toc}{section}{\nameref{sec:018}}
\begin{longtable}{l p{0.5cm} r}
ره عقل جز پیچ بر پیچ نیست
&&
بر عارفان جز خدا هیچ نیست
\\
توان گفتن این با حقایق شناس
&&
ولی خرده گیرند اهل قیاس
\\
که پس آسمان و زمین چیستند؟
&&
بنی آدم و دام و دد کیستند؟
\\
پسندیده پرسیدی ای هوشمند
&&
بگویم گر آید جوابت پسند
\\
که هامون و دریا و کوه و فلک
&&
پری و آدمی‌زاد و دیو و ملک
\\
همه هرچه هستند از آن کمترند
&&
که با هستیش نام هستی برند
\\
عظیم است پیش تو دریا به موج
&&
بلند است خورشید تابان به اوج
\\
ولی اهل صورت کجا پی برند
&&
که ارباب معنی به ملکی درند
\\
که گر آفتاب است یک ذره نیست
&&
وگر هفت دریاست یک قطره نیست
\\
چو سلطان عزت علم بر کشد
&&
جهان سر به جیب عدم در کشد
\\
\end{longtable}
\end{center}
