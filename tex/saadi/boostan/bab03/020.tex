\begin{center}
\section*{بخش ۲۰ - حکایت: ثنا گفت بر سعد زنگی کسی}
\label{sec:020}
\addcontentsline{toc}{section}{\nameref{sec:020}}
\begin{longtable}{l p{0.5cm} r}
ثنا گفت بر سعد زنگی کسی
&&
که بر تربتش باد رحمت بسی
\\
درم داد و تشریف و بنواختش
&&
به مقدار خود منزلت ساختش
\\
چو الله و بس دید بر نقش زر
&&
بشورید و برکند خلعت ز بر
\\
ز سوزش چنان شعله در جان گرفت
&&
که برجست و راه بیابان گرفت
\\
یکی گفتش از همنشینان دشت
&&
چه دیدی که حالت دگرگونه گشت
\\
تو اول زمین بوسه دادی به جای
&&
نبایستی آخر زدن پشت پای
\\
بخندید کاول ز بیم و امید
&&
همی لرزه بر تن فتادم چو بید
\\
به آخر ز تمکین الله و بس
&&
نه چیزم به چشم اندر آمد نه کس
\\
به شهری در از شام غوغا فتاد
&&
گرفتند پیری مبارک نهاد
\\
هنوز آن حدیثم به گوش اندر است
&&
چو قیدش نهادند بر پای و دست
\\
که گفت ار نه سلطان اشارت کند
&&
که را زهره باشد که غارت کند؟
\\
بباید چنین دشمنی دوست داشت
&&
که می‌دانمش دوست بر من گماشت
\\
اگر عز و جاه است و گر ذل و قید
&&
من از حق شناسم، نه از عمرو و زید
\\
ز علت مدار، ای خردمند، بیم
&&
چو داروی تلخت فرستد حکیم
\\
بخور هرچه آید ز دست حبیب
&&
نه بیمار داناتر است از طبیب
\\
\end{longtable}
\end{center}
