\begin{center}
\section*{بخش ۱۵ - حکایت مجنون و صدق محبت او: به مجنون کسی گفت کای نیک پی}
\label{sec:015}
\addcontentsline{toc}{section}{\nameref{sec:015}}
\begin{longtable}{l p{0.5cm} r}
به مجنون کسی گفت کای نیک پی
&&
چه بودت که دیگر نیایی به حی؟
\\
مگر در سرت شور لیلی نماند
&&
خیالت دگر گشت و میلی نماند؟
\\
چو بشنید بیچاره بگریست زار
&&
که ای خواجه دستم ز دامن بدار
\\
مرا خود دلی دردمند است ریش
&&
تو نیزم نمک بر جراحت مریش
\\
نه دوری دلیل صبوری بود
&&
که بسیار دوری ضروری بود
\\
بگفت ای وفادار فرخنده خوی
&&
پیامی که داری به لیلی بگوی
\\
بگفتا مبر نام من پیش دوست
&&
که حیف است نام من آنجا که اوست
\\
\end{longtable}
\end{center}
