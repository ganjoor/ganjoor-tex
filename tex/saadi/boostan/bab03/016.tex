\begin{center}
\section*{بخش ۱۶ - حکایت سلطان محمود و سیرت ایاز: یکی خرده بر شاه غزنین گرفت}
\label{sec:016}
\addcontentsline{toc}{section}{\nameref{sec:016}}
\begin{longtable}{l p{0.5cm} r}
یکی خرده بر شاه غزنین گرفت
&&
که حسنی ندارد ایاز ای شگفت
\\
گلی را که نه رنگ باشد نه بوی
&&
غریب است سودای بلبل بر اوی!
\\
به محمود گفت این حکایت کسی
&&
بپیچید از اندیشه بر خود بسی
\\
که عشق من ای خواجه بر خوی اوست
&&
نه بر قد و بالای نیکوی اوست
\\
شنیدم که در تنگنایی شتر
&&
بیفتاد و بشکست صندوق در
\\
به یغما ملک آستین برفشاند
&&
وز آنجا به تعجیل مرکب براند
\\
سواران پی در و مرجان شدند
&&
ز سلطان به یغما پریشان شدند
\\
نماند از وشاقان گردن فراز
&&
کسی در قفای ملک جز ایاز
\\
نگه کرد کای دلبر پیچ پیچ
&&
ز یغما چه آورده‌ای؟ گفت هیچ
\\
من اندر قفای تو می‌تاختم
&&
ز خدمت به نعمت نپرداختم
\\
گرت قربتی هست در بارگاه
&&
به خلعت مشو غافل از پادشاه
\\
خلاف طریقت بود کاولیا
&&
تمنا کنند از خدا جز خدا
\\
گر از دوست چشمت بر احسان اوست
&&
تو در بند خویشی نه در بند دوست
\\
تو را تا دهن باشد از حرص باز
&&
نیاید به گوش دل از غیب راز
\\
حقیقت سرایی است آراسته
&&
هوی و هوس گرد برخاسته
\\
نبینی که جایی که برخاست گرد
&&
نبیند نظر گرچه بیناست مرد
\\
\end{longtable}
\end{center}
