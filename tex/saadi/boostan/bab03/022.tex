\begin{center}
\section*{بخش ۲۲ - گفتار اندر سماع اهل دل و تقریر حق و باطل آن: اگر مرد عشقی کم خویش گیر}
\label{sec:022}
\addcontentsline{toc}{section}{\nameref{sec:022}}
\begin{longtable}{l p{0.5cm} r}
اگر مرد عشقی کم خویش گیر
&&
و گر نه ره عافیت پیش گیر
\\
مترس از محبت که خاکت کند
&&
که باقی شوی گر هلاکت کند
\\
نروید نبات از حبوب درست
&&
مگر حال بر وی بگردد نخست
\\
تو را با حق آن آشنایی دهد
&&
که از دست خویشت رهایی دهد
\\
که تا با خودی در خودت راه نیست
&&
وز این نکته جز بی خود آگاه نیست
\\
نه مطرب که آواز پای ستور
&&
سماع است اگر عشق داری و شور
\\
مگس پیش شوریده دل پر نزد
&&
که او چون مگس دست بر سر نزد
\\
نه بم داند آشفته سامان نه زیر
&&
به آواز مرغی بنالد فقیر
\\
سراینده خود می‌نگردد خموش
&&
ولیکن نه هر وقت باز است گوش
\\
چو شوریدگان می پرستی کنند
&&
به آواز دولاب مستی کنند
\\
به چرخ اندر آیند دولاب وار
&&
چو دولاب بر خود بگریند زار
\\
به تسلیم سر در گریبان برند
&&
چو طاقت نماند گریبان درند
\\
مکن عیب درویش مدهوش مست
&&
که غرق است از آن می‌زند پا و دست
\\
نگویم سماع ای برادر که چیست
&&
مگر مستمع را بدانم که کیست
\\
گر از برج معنی پرد طیر او
&&
فرشته فرو ماند از سیر او
\\
وگر مرد لهو است و بازی و لاغ
&&
قوی تر شود دیوش اندر دماغ
\\
چو مرد سماع است شهوت پرست
&&
به آواز خوش خفته خیزد، نه مست
\\
پریشان شود گل به باد سحر
&&
نه هیزم که نشکافدش جز تبر
\\
جهان پر سماع است و مستی و شور
&&
ولیکن چه بیند در آیینه کور؟
\\
نبینی شتر بر نوای عرب
&&
که چونش به رقص اندر آرد طرب؟
\\
شتر را چو شور طرب در سر است
&&
اگر آدمی را نباشد خر است
\\
\end{longtable}
\end{center}
