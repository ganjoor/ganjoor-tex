\begin{center}
\section*{بخش ۷ - حکایت در فدا شدن اهل محبت و غنیمت شمردن: یکی تشنه می‌گفت و جان می‌سپرد}
\label{sec:007}
\addcontentsline{toc}{section}{\nameref{sec:007}}
\begin{longtable}{l p{0.5cm} r}
یکی تشنه می‌گفت و جان می‌سپرد
&&
خنک نیکبختی که در آب مرد
\\
بدو گفت نابالغی کای عجب
&&
چو مردی چه سیراب و چه خشک لب
\\
بگفتا نه آخر دهان تر کنم
&&
که تا جان شیرینش در سر کنم؟
\\
فتد تشنه در آبدان عمیق
&&
که داند که سیراب میرد غریق
\\
اگر عاشقی دامن او بگیر
&&
وگر گویدت جان بده، گو بگیر
\\
بهشت تن آسانی آنگه خوری
&&
که بر دوزخ نیستی بگذری
\\
دل تخم کاران بود رنج کش
&&
چو خرمن برآید بخسبند خوش
\\
در این مجلس آن کس به کامی رسید
&&
که در دور آخر به جامی رسید
\\
\end{longtable}
\end{center}
