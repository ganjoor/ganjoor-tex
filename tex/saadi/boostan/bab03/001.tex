\begin{center}
\section*{بخش ۱ - سرآغاز: خوشا وقت شوریدگان غمش}
\label{sec:001}
\addcontentsline{toc}{section}{\nameref{sec:001}}
\begin{longtable}{l p{0.5cm} r}
خوشا وقت شوریدگان غمش
&&
اگر زخم بینند و گر مرهمش
\\
گدایانی از پادشاهی نفور
&&
به امیدش اندر گدایی صبور
\\
دمادم شراب الم در کشند
&&
وگر تلخ بینند دم در کشند
\\
بلای خمار است در عیش مل
&&
سلحدار خار است با شاه گل
\\
نه تلخ است صبری که بر یاد اوست
&&
که تلخی شکر باشد از دست دوست
\\
ملامت کشانند مستان یار
&&
سبک تر برد اشتر مست بار
\\
اسیرش نخواهد رهایی ز بند
&&
شکارش نجوید خلاص از کمند
\\
سلاطین عزلت، گدایان حی
&&
منازل شناسان گم کرده پی
\\
به سر وقتشان خلق ره کی برند
&&
که چون آب حیوان به ظلمت درند
\\
چو بیت‌المقدس درون پر قباب
&&
رها کرده دیوار بیرون خراب
\\
چو پروانه آتش به خود در زنند
&&
نه چون کرم پیله به خود برتنند
\\
دلارام در بر، دلارام جوی
&&
لب از تشنگی خشک، بر طرف جوی
\\
نگویم که بر آب قادر نیند
&&
که بر شاطی نیل مستسقیند
\\
\end{longtable}
\end{center}
