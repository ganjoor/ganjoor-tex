\begin{center}
\section*{بخش ۵ - حکایت در معنی اهل محبت: شنیدم که بر لحن خنیاگری}
\label{sec:005}
\addcontentsline{toc}{section}{\nameref{sec:005}}
\begin{longtable}{l p{0.5cm} r}
شنیدم که بر لحن خنیاگری
&&
به رقص اندر آمد پری پیکری
\\
ز دلهای شوریده پیرامنش
&&
گرفت آتش شمع در دامنش
\\
پراکنده خاطر شد و خشمناک
&&
یکی گفتش از دوستداران، چه باک؟
\\
تو را آتش ای دوست دامن بسوخت
&&
مرا خود به یک بار خرمن بسوخت
\\
اگر یاری از خویشتن دم مزن
&&
که شرک است با یار و با خویشتن
\\
چنین دارم از پیر داننده یاد
&&
که شوریده‌ای سر به صحرا نهاد
\\
پدر در فراقش نخورد و نخفت
&&
پسر را ملامت بکردند و گفت
\\
از انگه که یارم کس خویش خواند
&&
دگر با کسم آشنایی نماند
\\
به حقش که تا حق جمالم نمود
&&
دگر هر چه دیدم خیالم نمود
\\
نشد گم که روی از خلایق بتافت
&&
که گم کرده خویش را باز یافت
\\
پراکندگانند زیر فلک
&&
که هم دد توان خواندشان هم ملک
\\
ز یاد ملک چون ملک نارمند
&&
شب و روز چون دد ز مردم رمند
\\
قوی بازوانند کوتاه دست
&&
خردمند شیدا و هشیار مست
\\
گه آسوده در گوشه‌ای خرقه دوز
&&
گه آشفته در مجلسی خرقه سوز
\\
نه سودای خودشان، نه پروای کس
&&
نه در کنج توحیدشان جای کس
\\
پریشیده عقل و پراکنده هوش
&&
ز قول نصیحتگر آکنده گوش
\\
به دریا نخواهد شدن بط غریق
&&
سمندر چه داند عذاب حریق؟
\\
تهیدست مردان پر حوصله
&&
بیابان نوردان پی قافله
\\
عزیزان پوشیده از چشم خلق
&&
نه زنار داران پوشیده دلق
\\
ندارند چشم از خلایق پسند
&&
که ایشان پسندیده حق بسند
\\
پر از میوه و سایه ور چون رزند
&&
نه چون ما سیهکار و ازرق رزند
\\
به خود سر فرو برده همچون صدف
&&
نه مانند دریا بر آورده کف
\\
نه مردم همین استخوانند و پوست
&&
نه هر صورتی جان معنی در اوست
\\
نه سلطان خریدار هر بنده‌ای است
&&
نه در زیر هر ژنده‌ای زنده‌ای است
\\
اگر ژاله هر قطره‌ای در شدی
&&
چو خرمهره بازار از او پر شدی
\\
چو غازی به خود بر نبندند پای
&&
که محکم رود پای چوبین ز جای
\\
حریفان خلوت سرای الست
&&
به یک جرعه تا نفخهٔ صور مست
\\
به تیغ از غرض بر نگیرند چنگ
&&
که پرهیز و عشق آبگینه‌ست و سنگ
\\
\end{longtable}
\end{center}
