\begin{center}
\section*{بخش ۲۹ - حکایت در معنی ثمرات نکوکاری در آخرت: کسی دید صحرای محشر به خواب}
\label{sec:029}
\addcontentsline{toc}{section}{\nameref{sec:029}}
\begin{longtable}{l p{0.5cm} r}
کسی دید صحرای محشر به خواب
&&
مس تفته روی زمین ز آفتاب
\\
همی بر فلک شد ز مردم خروش
&&
دماغ از تبش می‌برآمد به جوش
\\
یکی شخص از این جمله در سایه‌ای
&&
به گردن بر از خلد پیرایه‌ای
\\
بپرسید کای مجلس آرای مرد
&&
که بود اندر این مجلست پایمرد؟
\\
رزی داشتم بر در خانه، گفت
&&
به سایه درش نیکمردی بخفت
\\
در این وقت نومیدی آن مرد راست
&&
گناهم ز دادار داور بخواست
\\
که یارب بر این بنده بخشایشی
&&
کز او دیده‌ام وقتی آسایشی
\\
چه گفتم چو حل کردم این راز را؟
&&
بشارت خداوند شیراز را
\\
که جمهور در سایهٔ همتش
&&
مقیمند و بر سفرهٔ نعمتش
\\
درختی است مرد کرم، باردار
&&
وز او بگذری هیزم کوهسار
\\
حطب را اگر تیشه بر پی زنند
&&
درخت برومند را کی زنند؟
\\
بسی پای دار، ای درخت هنر
&&
که هم میوه‌داری و هم سایه‌ور
\\
بگفتیم در باب احسان بسی
&&
ولیکن نه شرط است با هر کسی
\\
بخور مردم آزار را خون و مال
&&
که از مرغ بد کنده به پر و بال
\\
یکی را که با خواجهٔ توست جنگ
&&
به دستش چرا می‌دهی چوب و سنگ؟
\\
بر انداز بیخی که خار آورد
&&
درختی بپرور که بار آورد
\\
کسی را بده پایهٔ مهتران
&&
که بر کهتران سر ندارد گران
\\
مبخشای بر هر کجا ظالمی است
&&
که رحمت بر او جور بر عالمی است
\\
جهان‌سوز را کشته بهتر چراغ
&&
یکی به در آتش که خلقی به داغ
\\
هر آن کس که بر دزد رحمت کند
&&
به بازوی خود کاروان می‌زند
\\
جفاپیشگان را بده سر بباد
&&
ستم بر ستم پیشه عدل است و داد
\\
\end{longtable}
\end{center}
