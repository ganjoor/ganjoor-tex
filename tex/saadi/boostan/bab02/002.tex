\begin{center}
\section*{بخش ۲ - گفتار اندر نواخت ضعیفان: پدرمرده را سایه بر سر فکن}
\label{sec:002}
\addcontentsline{toc}{section}{\nameref{sec:002}}
\begin{longtable}{l p{0.5cm} r}
پدرمرده را سایه بر سر فکن
&&
غبارش بیفشان و خارش بکن
\\
ندانی چه بودش فرو مانده سخت؟
&&
بود تازه بی بیخ هرگز درخت؟
\\
چو بینی یتیمی سر افکنده پیش
&&
مده بوسه بر روی فرزند خویش
\\
یتیم ار بگرید که نازش خرد؟
&&
وگر خشم گیرد که بارش برد؟
\\
الا تا نگرید که عرش عظیم
&&
بلرزد همی چون بگرید یتیم
\\
به رحمت بکن آبش از دیده پاک
&&
به شفقت بیفشانش از چهره خاک
\\
اگر سایه خود برفت از سرش
&&
تو در سایه خویشتن پرورش
\\
من آنگه سر تاجور داشتم
&&
که سر بر کنار پدر داشتم
\\
اگر بر وجودم نشستی مگس
&&
پریشان شدی خاطر چند کس
\\
کنون دشمنان گر برندم اسیر
&&
نباشد کس از دوستانم نصیر
\\
مرا باشد از درد طفلان خبر
&&
که در طفلی از سر برفتم پدر
\\
یکی خار پای یتیمی بکند
&&
به خواب اندرش دید صدر خجند
\\
همی گفت و در روضه‌ها می‌چمید
&&
کز آن خار بر من چه گلها دمید
\\
مشو تا توانی ز رحمت بری
&&
که رحمت برندت چو رحمت بری
\\
چو انعام کردی مشو خودپرست
&&
که من سرورم دیگران زیردست
\\
اگر تیغ دورانش انداخته‌ست
&&
نه شمشیر دوران هنوز آخته‌ست؟
\\
چو بینی دعاگوی دولت هزار
&&
خداوند را شکر نعمت گزار
\\
که چشم از تو دارند مردم بسی
&&
نه تو چشم داری به دست کسی
\\
«کرم» خوانده‌ام سیرت سروران
&&
غلط گفتم، اخلاق پیغمبران
\\
\end{longtable}
\end{center}
