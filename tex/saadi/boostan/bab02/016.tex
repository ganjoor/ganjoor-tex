\begin{center}
\section*{بخش ۱۶ - حکایت در معنی صید کردن دلها به احسان: به ره بر یکی پیشم آمد جوان}
\label{sec:016}
\addcontentsline{toc}{section}{\nameref{sec:016}}
\begin{longtable}{l p{0.5cm} r}
به ره بر یکی پیشم آمد جوان
&&
به تک در پیش گوسفندی دوان
\\
بدو گفتم این ریسمان است و بند
&&
که می‌آرد اندر پیت گوسفند
\\
سبک طوق و زنجیر از او باز کرد
&&
چپ و راست پوییدن آغاز کرد
\\
هنوز از پیش تازیان می‌دوید
&&
که جو خورده بود از کف مرد و خوید
\\
چو باز آمد از عیش و شادی به جای
&&
مرا دید و گفت ای خداوند رای
\\
نه این ریسمان می‌برد با منش
&&
که احسان کمندی است در گردنش
\\
به لطفی که دیده‌ست پیل دمان
&&
نیارد همی حمله بر پیلبان
\\
بدان را نوازش کن ای نیکمرد
&&
که سگ پاس دارد چو نان تو خورد
\\
بر آن مرد کند است دندان یوز
&&
که مالد زبان بر پنیرش دو روز
\\
\end{longtable}
\end{center}
