\begin{center}
\section*{بخش ۵ - حکایت عابد با شوخ دیده: زباندانی آمد به صاحبدلی}
\label{sec:005}
\addcontentsline{toc}{section}{\nameref{sec:005}}
\begin{longtable}{l p{0.5cm} r}
زباندانی آمد به صاحبدلی
&&
که محکم فرومانده‌ام در گلی
\\
یکی سفله را ده درم بر من است
&&
که دانگی از او بر دلم ده من است
\\
همه شب پریشان از او حال من
&&
همه روز چون سایه دنبال من
\\
بکرد از سخنهای خاطر پریش
&&
درون دلم چون در خانه ریش
\\
خدایش مگر تا ز مادر بزاد
&&
جز این ده درم چیز دیگر نداد
\\
ندانسته از دفتر دین الف
&&
نخوانده به جز باب لاینصرف
\\
خور از کوه یک روز سر بر نزد
&&
که آن قلتبان حلقه بر در نزد
\\
در اندیشه‌ام تا کدامم کریم
&&
از آن سنگدل دست گیرد به سیم
\\
شنید این سخن پیر فرخ نهاد
&&
درستی دو، در آستینش نهاد
\\
زر افتاد در دست افسانه گوی
&&
برون رفت از آنجا چو زر تازه روی
\\
یکی گفت: شیخ! این ندانی که کیست؟
&&
بر او گر بمیرد نباید گریست
\\
گدایی که بر شیر نر زین نهد
&&
ابو زید را اسب و فرزین نهد
\\
بر آشفت عابد که خاموش باش
&&
تو مرد زبان نیستی، گوش باش
\\
اگر راست بود آنچه پنداشتم
&&
ز خلق آبرویش نگه داشتم
\\
وگر شوخ چشمی و سالوس کرد
&&
الا تا نپنداری افسوس کرد
\\
که خود را نگه داشتم آبروی
&&
ز دست چنان گربزی یاوه گوی
\\
بد و نیک را بذل کن سیم و زر
&&
که این کسب خیر است و آن دفع شر
\\
خنک آن که در صحبت عاقلان
&&
بیاموزد اخلاق صاحبدلان
\\
گرت عقل و رای است و تدبیر و هوش
&&
به عزت کنی پند سعدی به گوش
\\
که اغلب در این شیوه دارد مقال
&&
نه در چشم و زلف و بناگوش و خال
\\
\end{longtable}
\end{center}
