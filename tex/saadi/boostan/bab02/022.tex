\begin{center}
\section*{بخش ۲۲ - حکایت حاتم طائی: ز بنگاه حاتم یکی پیرمرد}
\label{sec:022}
\addcontentsline{toc}{section}{\nameref{sec:022}}
\begin{longtable}{l p{0.5cm} r}
ز بنگاه حاتم یکی پیرمرد
&&
طلب ده درم سنگ فانید کرد
\\
ز راوی چنان یاد دارم خبر
&&
که پیشش فرستاد تنگی شکر
\\
زن از خیمه گفت این چه تدبیر بود؟
&&
همان ده درم حاجت پیر بود
\\
شنید این سخن نامبردار طی
&&
بخندید و گفت ای دلارام حی
\\
گر او در خور حاجت خویش خواست
&&
جوانمردی آل حاتم کجاست؟
\\
چو حاتم به آزادمردی دگر
&&
ز دوران گیتی نیامد مگر
\\
ابوبکر سعد آن که دست نوال
&&
نهد همتش بر دهان سؤال
\\
رعیت پناها دلت شاد باد
&&
به سعیت مسلمانی آباد باد
\\
سرافرازد این خاک فرخنده بوم
&&
ز عدلت بر اقلیم یونان و روم
\\
چو حاتم، اگر نیستی کام وی
&&
نبردی کس اندر جهان نام طی
\\
ثنا ماند از آن نامور در کتاب
&&
تو را هم ثنا ماند و هم ثواب
\\
که حاتم بدان نام و آوازه خواست
&&
تو را سعی و جهد از برای خداست
\\
تکلف بر مرد درویش نیست
&&
وصیت همین یک سخن بیش نیست
\\
که چندان که جهدت بود خیر کن
&&
ز تو خیر ماند ز سعدی سخن
\\
\end{longtable}
\end{center}
