\begin{center}
\section*{بخش ۲۲ - حکایت جنید و سیرت او در تواضع: شنیدم که در دشت صنعا جنید}
\label{sec:022}
\addcontentsline{toc}{section}{\nameref{sec:022}}
\begin{longtable}{l p{0.5cm} r}
شنیدم که در دشت صنعا جنید
&&
سگی دید برکنده دندان صید
\\
ز نیروی سر پنجهٔ شیر گیر
&&
فرومانده عاجز چو روباه پیر
\\
پس از غرم و آهو گرفتن به پی
&&
لگد خوردی از گوسفندان حی
\\
چو مسکین و بی طاقتش دید و ریش
&&
بدو داد یک نیمه از زاد خویش
\\
شنیدم که می‌گفت و خوش می‌گریست
&&
که داند که بهتر ز ما هر دو کیست؟
\\
به ظاهر من امروز از این بهترم
&&
دگر تا چه راند قضا بر سرم
\\
گرم پای ایمان نلغزد ز جای
&&
به سر بر نهم تاج عفو خدای
\\
وگر کسوت معرفت در برم
&&
نماند، به بسیار از این کمترم
\\
که سگ با همه زشت نامی چو مرد
&&
مر او را به دوزخ نخواهند برد
\\
ره این است سعدی که مردان راه
&&
به عزت نکردند در خود نگاه
\\
از‍‍‍‍‍‍‍‍ آن بر ملائک شرف داشتند
&&
که خود را به از سگ نپنداشتند
\\
\end{longtable}
\end{center}
