\begin{center}
\section*{بخش ۱۹ - حکایت زاهد تبریزی: عزیزی در اقصای تبریز بود}
\label{sec:019}
\addcontentsline{toc}{section}{\nameref{sec:019}}
\begin{longtable}{l p{0.5cm} r}
عزیزی در اقصای تبریز بود
&&
که همواره بیدار و شب خیز بود
\\
شبی دید جایی که دزدی کمند
&&
بپیچید و بر طرف بامی فکند
\\
کسان را خبر کرد و آشوب خاست
&&
ز هر جانبی مرد با چوب خاست
\\
چو نامردم آواز مردم شنید
&&
میان خطر جای بودن ندید
\\
نهیبی از آن گیر و دار آمدش
&&
گریز به وقت اختیار آمدش
\\
ز رحمت دل پارسا موم شد
&&
که شب دزد بیچاره محروم شد
\\
به تاریکی از پی فراز آمدش
&&
به راهی دگر پیشباز آمدش
\\
که یارا مرو کآشنای توام
&&
به مردانگی خاک پای توام
\\
ندیدم به مردانگی چون تو کس
&&
که جنگاوری بر دو نوع است و بس
\\
یکی پیش خصم آمدن مردوار
&&
دوم جان به در بردن از کارزار
\\
بر این هر دو خصلت غلام توام
&&
چه نامی که مولای نام توام؟
\\
گرت رای باشد به حکم کرم
&&
به جایی که می‌دانمت ره برم
\\
سرایی است کوتاه و در بسته سخت
&&
نپندارم آنجا خداوند رخت
\\
کلوخی دو بالای هم بر نهیم
&&
یکی پای بر دوش دیگر نهیم
\\
به چندان که در دستت افتد بساز
&&
از آن به که گردی تهیدست باز
\\
به دلداری و چاپلوسی و فن
&&
کشیدش سوی خانهٔ خویشتن
\\
جوانمرد شبرو فرو داشت دوش
&&
به کتفش برآمد خداوند هوش
\\
به غلطاق و دستار و رختی که داشت
&&
ز بالا به دامان او در گذاشت
\\
وز آنجا برآورد غوغا که دزد
&&
ثواب ای جوانان و یاری و مزد
\\
به در جست از آشوب دزد دغل
&&
دوان، جامهٔ پارسا در بغل
\\
دل آسوده شد مرد نیک اعتقاد
&&
که سرگشته‌ای را برآمد مراد
\\
خبیثی که بر کس ترحم نکرد
&&
ببخشود بر وی دل نیکمرد
\\
عجب ناید از سیرت بخردان
&&
که نیکی کنند از کرم با بدان
\\
در اقبال نیکان بدان می‌زیند
&&
وگر چه بدان اهل نیکی نیند
\\
\end{longtable}
\end{center}
