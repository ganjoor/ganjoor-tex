\begin{center}
\section*{بخش ۲۴ - حکایت صبر مردان بر جفا: شنیدم که در خاک وخش از مهان}
\label{sec:024}
\addcontentsline{toc}{section}{\nameref{sec:024}}
\begin{longtable}{l p{0.5cm} r}
شنیدم که در خاک وخش از مهان
&&
یکی بود در کنج خلوت نهان
\\
مجرد به معنی نه عارف به دلق
&&
که بیرون کند دست حاجت به خلق
\\
سعادت گشاده دری سوی او
&&
در از دیگران بسته بر روی او
\\
زبان آوری بی‌خرد سعی کرد
&&
ز شوخی به بد گفتن نیکمرد
\\
که زنهار از این مکر و دستان و ریو
&&
بجای سلیمان نشستن چو دیو
\\
دمادم بشویند چون گربه روی
&&
طمع کرده در صید موشان کوی
\\
ریاضت کش از بهر نام و غرور
&&
که طبل تهی را رود بانگ دور
\\
همی گفت و خلقی بر او انجمن
&&
بر ایشان تفرج کنان مرد و زن
\\
شنیدم که بگریست دانای وخش
&&
که یارب مر این بنده را توبه بخش
\\
وگر راست گفت ای خداوند پاک
&&
مرا توبه ده تا نگردم هلاک
\\
پسند آمد از عیب جوی خودم
&&
که معلوم من کرد خوی بدم
\\
گر آنی که دشمنت گوید، مرنج
&&
وگر نیستی، گو برو بادسنج
\\
اگر ابلهی مشک را گنده گفت
&&
تو مجموع باش او پراکنده گفت
\\
وگر می‌رود در پیاز این سخن
&&
چنین است گو گنده مغزی مکن
\\
نگیرد خردمند روشن ضمیر
&&
زبان بند دشمن ز هنگامه گیر
\\
نه آیین عقل است و رای و خرد
&&
که دانا فریب مشعبد خورد
\\
پس کار خویش آنکه عاقل نشست
&&
زبان بداندیش بر خود ببست
\\
تو نیکو روش باش تا بد سگال
&&
نیابد به نقص تو گفتن مجال
\\
چو دشوارت آمد ز دشمن سخن
&&
نگر تا چه عیبت گرفت آن مکن
\\
جز آن کس ندانم نکو گوی من
&&
که روشن کند بر من آهوی من
\\
\end{longtable}
\end{center}
