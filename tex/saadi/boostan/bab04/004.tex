\begin{center}
\section*{بخش ۴ - حکایت بایزید بسطامی: شنیدم که وقتی سحرگاه عید}
\label{sec:004}
\addcontentsline{toc}{section}{\nameref{sec:004}}
\begin{longtable}{l p{0.5cm} r}
شنیدم که وقتی سحرگاه عید
&&
ز گرمابه آمد برون بایزید
\\
یکی طشت خاکسترش بی‌خبر
&&
فرو ریختند از سرایی به سر
\\
همی گفت شولیده دستار و موی
&&
کف دست شکرانه مالان به روی
\\
که ای نفس من در خور آتشم
&&
به خاکستری روی در هم کشم؟
\\
بزرگان نکردند در خود نگاه
&&
خدابینی از خویشتن بین مخواه
\\
بزرگی به ناموس و گفتار نیست
&&
بلندی به دعوی و پندار نیست
\\
تواضع سر رفعت افرازدت
&&
تکبر به خاک اندر اندازدت
\\
به گردن فتد سرکش تند خوی
&&
بلندیت باید بلندی مجوی
\\
ز مغرور دنیا ره دین مجوی
&&
خدابینی از خویشتن بین مجوی
\\
گرت جاه باید مکن چون خسان
&&
به چشم حقارت نگه در کسان
\\
گمان کی برد مردم هوشمند
&&
که در سرگرانی است قدر بلند؟
\\
از این نامورتر محلی مجوی
&&
که خوانند خلقت پسندیده خوی
\\
نه گر چون تویی بر تو کبر آورد
&&
بزرگش نبینی به چشم خرد؟
\\
تو نیز ار تکبر کنی همچنان
&&
نمایی، که پیشت تکبر کنان
\\
چو استاده‌ای بر مقامی بلند
&&
بر افتاده گر هوشمندی مخند
\\
بسا ایستاده در آمد ز پای
&&
که افتادگانش گرفتند جای
\\
گرفتم که خود هستی از عیب پاک
&&
تعنت مکن بر من عیب‌ناک
\\
یکی حلقهٔ کعبه دارد به دست
&&
یکی در خراباتی افتاده مست
\\
گر آن را بخواند، که نگذاردش؟
&&
ور این را براند، که باز آردش؟
\\
نه مستظهر است آن به اعمال خویش
&&
نه این را در توبه بسته‌ست پیش
\\
\end{longtable}
\end{center}
