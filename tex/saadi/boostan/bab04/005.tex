\begin{center}
\section*{بخش ۵ - حکایت عیسی (ع) و عابد و ناپارسا: شنیدستم از راویان کلام}
\label{sec:005}
\addcontentsline{toc}{section}{\nameref{sec:005}}
\begin{longtable}{l p{0.5cm} r}
شنیدستم از راویان کلام
&&
که در عهد عیسی علیه‌السلام
\\
یکی زندگانی تلف کرده بود
&&
به جهل و ضلالت سر آورده بود
\\
دلیری سیه نامه‌ای سخت دل
&&
ز ناپاکی ابلیس در وی خجل
\\
به سر برده ایام، بی حاصلی
&&
نیاسوده تا بوده از وی دلی
\\
سرش خالی از عقل و از احتشام
&&
شکم فربه از لقمه‌های حرام
\\
به ناراستی دامن آلوده‌ای
&&
به ناداشتی دوده اندوده‌ای
\\
نه چشمی چو بینندگان راست رو
&&
نه گوشی چو مردم نصیحت شنو
\\
چو سال بد از وی خلایق نفور
&&
نمایان به هم چون مه نو ز دور
\\
هوی و هوس خرمنش سوخته
&&
جوی نیکنامی نیندوخته
\\
سیه نامه چندان تنعم براند
&&
که در نامه جای نبشتن نماند
\\
گنهکار و خودرای و شهوت پرست
&&
به غفلت شب و روز مخمور و مست
\\
شنیدم که عیسی در آمد ز دشت
&&
به مقصوره عابدی برگذشت
\\
به زیر آمد از غرفه خلوت نشین
&&
به پایش در افتاد سر بر زمین
\\
گنهکار برگشته اختر ز دور
&&
چو پروانه حیران در ایشان ز نور
\\
تأمل به حسرت کنان شرمسار
&&
چو درویش در دست سرمایه‌دار
\\
خجل زیر لب عذرخواهان به سوز
&&
ز شبهای در غفلت آورده روز
\\
سرشک غم از دیده باران چو میغ
&&
که عمرم به غفلت گذشت ای دریغ!
\\
بر انداختم نقد عمر عزیز
&&
به دست از نکویی نیاورده چیز
\\
چو من زنده هرگز مبادا کسی
&&
که مرگش به از زندگانی بسی
\\
برست آن که در عهد طفلی بمرد
&&
که پیرانه سر شرمساری نبرد
\\
گناهم ببخش ای جهان آفرین
&&
که گر با من آید فبئس القرین
\\
نگون مانده از شرمساری سرش
&&
روان آب حسرت به شیب و برش
\\
در این گوشه نالان گنهکار پیر
&&
که فریاد حالم رس ای دستگیر
\\
وز آن نیمه عابد سری پر غرور
&&
ترش کرده بر فاسق ابرو ز دور
\\
که این مدبر اندر پی ما چراست؟
&&
نگون بخت جاهل چه در خورد ماست؟
\\
به گردن در آتش در افتاده‌ای
&&
به باد هوی عمر بر داده‌ای
\\
چه خیر آمد از نفس تر دامنش
&&
که صحبت بود با مسیح و منش؟
\\
چه بودی که زحمت ببردی ز پیش
&&
به دوزخ برفتی پس کار خویش
\\
همی رنجم از طلعت ناخوشش
&&
مبادا که در من فتد آتشش
\\
به محشر که حاضر شوند انجمن
&&
خدایا تو با او مکن حشر من
\\
در این بود و وحی از جلیل الصفات
&&
در آمد به عیسی علیه الصلوة
\\
که گر عالم است این و گر وی جهول
&&
مرا دعوت هر دو آمد قبول
\\
تبه کرده ایام برگشته روز
&&
بنالید بر من به زاری و سوز
\\
به بیچارگی هر که آمد برم
&&
نیندازمش ز آستان کرم
\\
عفو کردم از وی عملهای زشت
&&
به انعام خویش آرمش در بهشت
\\
و گر عار دارد عبادت پرست
&&
که در خلد با وی بود هم نشست
\\
بگو ننگ از او در قیامت مدار
&&
که آن را به جنت برند این به نار
\\
که آن را جگر خون شد از سوز و درد
&&
گر این تکیه بر طاعت خویش کرد
\\
ندانست در بارگاه غنی
&&
که بیچارگی به ز کبر و منی
\\
کرا جامه پاک است و سیرت پلید
&&
در دوزخش را نباید کلید
\\
بر این آستان عجز و مسکینیت
&&
به از طاعت و خویشتن بینیت
\\
چو خود را ز نیکان شمردی بدی
&&
نمی‌گنجد اندر خدایی خودی
\\
اگر مردی از مردی خود مگوی
&&
نه هر شهسواری به در برد گوی
\\
پیاز آمد آن بی هنر جمله پوست
&&
که پنداشت چون پسته مغزی در اوست
\\
از این نوع طاعت نیاید به کار
&&
برو عذر تقصیر طاعت بیار
\\
چه رند پریشان شوریده بخت
&&
چه زاهد که بر خود کند کار سخت
\\
به زهد و ورع کوش و صدق و صفا
&&
ولیکن میفزای بر مصطفی
\\
نخورد از عبادت بر آن بی خرد
&&
که با حق نکو بود و با خلق بد
\\
سخن ماند از عاقلان یادگار
&&
ز سعدی همین یک سخن یاد دار
\\
گنهکار اندیشناک از خدای
&&
به از پارسای عبادت نمای
\\
\end{longtable}
\end{center}
