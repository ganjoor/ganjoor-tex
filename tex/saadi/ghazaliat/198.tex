\begin{center}
\section*{غزل ۱۹۸: با کاروان مصری چندین شکر نباشد}
\label{sec:198}
\addcontentsline{toc}{section}{\nameref{sec:198}}
\begin{longtable}{l p{0.5cm} r}
با کاروان مصری چندین شکر نباشد
&&
در لعبتان چینی زین خوبتر نباشد
\\
این دلبری و شوخی از سرو و گل نیاید
&&
وین شاهدی و شنگی در ماه و خور نباشد
\\
گفتم به شیرمردی چشم از نظر بدوزم
&&
با تیر چشم خوبان تقوا سپر نباشد
\\
ما را نظر به خیرست از حسن ماه رویان
&&
هر کو به شر کند میل او خود بشر نباشد
\\
هر آدمی که بینی از سر عشق خالی
&&
در پایه جمادست او جانور نباشد
\\
الا گذر نباشد پیش تو اهل دل را
&&
ور نه به هیچ تدبیر از تو گذر نباشد
\\
هوشم نماند با کس اندیشه‌ام تویی بس
&&
جایی که حیرت آمد سمع و بصر نباشد
\\
بر عندلیب عاشق گر بشکنی قفس را
&&
از ذوق اندرونش پروای در نباشد
\\
تو مست خواب نوشین تا بامداد و بر من
&&
شب‌ها رود که گویی هرگز سحر نباشد
\\
دل می‌برد به دعوی فریاد شوق سعدی
&&
الا بهیمه‌ای را کز دل خبر نباشد
\\
تا آتشی نباشد در خرمنی نگیرد
&&
طامات مدعی را چندین اثر نباشد
\\
\end{longtable}
\end{center}
