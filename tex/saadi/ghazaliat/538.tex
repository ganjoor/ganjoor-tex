\begin{center}
\section*{غزل ۵۳۸: گفتم آهن دلی کنم چندی}
\label{sec:538}
\addcontentsline{toc}{section}{\nameref{sec:538}}
\begin{longtable}{l p{0.5cm} r}
گفتم آهن دلی کنم چندی
&&
ندهم دل به هیچ دلبندی
\\
وان که را دیده در دهان تو رفت
&&
هرگزش گوش نشنود پندی
\\
خاصه ما را که در ازل بوده‌ست
&&
با تو آمیزشی و پیوندی
\\
به دلت کز دلت به در نکنم
&&
سختتر زین مخواه سوگندی
\\
یک دم آخر حجاب یک سو نه
&&
تا برآساید آرزومندی
\\
همچنان پیر نیست مادر دهر
&&
که بیاورد چون تو فرزندی
\\
ریش فرهاد بهترک می‌بود
&&
گر نه شیرین نمک پراکندی
\\
کاشکی خاک بودمی در راه
&&
تا مگر سایه بر من افکندی
\\
چه کند بنده‌ای که از دل و جان
&&
نکند خدمت خداوندی
\\
سعدیا دور نیک نامی رفت
&&
نوبت عاشقیست یک چندی
\\
\end{longtable}
\end{center}
