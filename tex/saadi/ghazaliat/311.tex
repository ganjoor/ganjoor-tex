\begin{center}
\section*{غزل ۳۱۱: متقلب درون جامه ناز}
\label{sec:311}
\addcontentsline{toc}{section}{\nameref{sec:311}}
\begin{longtable}{l p{0.5cm} r}
متقلب درون جامه ناز
&&
چه خبر دارد از شبان دراز
\\
عاقل انجام عشق می‌بیند
&&
تا هم اول نمی‌کند آغاز
\\
جهد کردم که دل به کس ندهم
&&
چه توان کرد با دو دیده باز
\\
زینهار از بلای تیر نظر
&&
که چو رفت از کمان نیاید باز
\\
مگر از شوخی تذروان بود
&&
که فرودوختند دیده باز
\\
محتسب در قفای رندانست
&&
غافل از صوفیان شاهدباز
\\
پارسایی که خمر عشق چشید
&&
خانه گو با معاشران پرداز
\\
هر که را با گل آشنایی بود
&&
گو برو با جفای خار بساز
\\
سپرت می‌بباید افکندن
&&
ای که دل می‌دهی به تیرانداز
\\
هر چه بینی ز دوستان کرمست
&&
گر اهانت کنند و گر اعزاز
\\
دست مجنون و دامن لیلی
&&
روی محمود و خاک پای ایاز
\\
هیچ بلبل نداند این دستان
&&
هیچ مطرب ندارد این آواز
\\
هر متاعی ز معدنی خیزد
&&
شکر از مصر و سعدی از شیراز
\\
\end{longtable}
\end{center}
