\begin{center}
\section*{غزل ۵۵۴: کس درنیامدست بدین خوبی از دری}
\label{sec:554}
\addcontentsline{toc}{section}{\nameref{sec:554}}
\begin{longtable}{l p{0.5cm} r}
کس درنیامده‌ست بدین خوبی از دری
&&
دیگر نیاورد چو تو فرزند مادری
\\
خورشید اگر تو روی نپوشی فرو رود
&&
گوید دو آفتاب نباشد به کشوری
\\
اول منم که در همه عالم نیامده‌ست
&&
زیباتر از تو در نظرم هیچ منظری
\\
هرگز نبرده‌ام به خرابات عشق راه
&&
امروزم آرزوی تو در داد ساغری
\\
یا خود به حسن روی تو کس نیست در جهان
&&
یا هست و نیستم ز تو پروای دیگری
\\
بر سرو قامتت گل و بادام روی و چشم
&&
نشنیده‌ام که سرو چنین آورد بری
\\
رویی که روز روشن اگر برکشد نقاب
&&
پرتو دهد چنان که شب تیره اختری
\\
همراه من مباش که غیرت برند خلق
&&
در دست مفلسی چو ببینند گوهری
\\
من کم نمی‌کنم سر مویی ز مهر دوست
&&
ور می‌زند به هر بن موییم نشتری
\\
روزی مگر به دیده سعدی قدم نهی
&&
تا در رهت به هر قدمت می‌نهد سری
\\
\end{longtable}
\end{center}
