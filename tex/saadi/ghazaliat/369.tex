\begin{center}
\section*{غزل ۳۶۹: چو تو آمدی مرا بس که حدیث خویش گفتم}
\label{sec:369}
\addcontentsline{toc}{section}{\nameref{sec:369}}
\begin{longtable}{l p{0.5cm} r}
چو تو آمدی مرا بس که حدیث خویش گفتم
&&
چو تو ایستاده باشی ادب آن که من بیفتم
\\
تو اگر چنین لطیف از در بوستان درآیی
&&
گل سرخ شرم دارد که چرا همی‌شکفتم
\\
چو به منتها رسد گل برود قرار بلبل
&&
همه خلق را خبر شد غم دل که می‌نهفتم
\\
به امید آن که جایی قدمی نهاده باشی
&&
همه خاک‌های شیراز به دیدگان برفتم
\\
دو سه بامداد دیگر که نسیم گل برآید
&&
بتر از هزاردستان بکشد فراق جفتم
\\
نشنیده‌ای که فرهاد چگونه سنگ سفتی
&&
نه چو سنگ آستانت که به آب دیده سفتم
\\
نه عجب شب درازم که دو دیده باز باشد
&&
به خیالت ای ستمگر عجب است اگر بخفتم
\\
ز هزار خون سعدی بحلند بندگانت
&&
تو بگوی تا بریزند و بگو که من نگفتم
\\
\end{longtable}
\end{center}
