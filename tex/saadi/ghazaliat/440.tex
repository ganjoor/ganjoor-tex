\begin{center}
\section*{غزل ۴۴۰: کاش کان دلبر عیار که من کشته اویم}
\label{sec:440}
\addcontentsline{toc}{section}{\nameref{sec:440}}
\begin{longtable}{l p{0.5cm} r}
کاش کان دلبر عیار که من کشته اویم
&&
بار دیگر بگذشتی که کند زنده به بویم
\\
ترک من گفت و به ترکش نتوانم که بگویم
&&
چه کنم نیست دلی چون دل او ز آهن و رویم
\\
تا قدم باشدم اندر قدمش افتم و خیزم
&&
تا نفس ماندم اندر عقبش پرسم و پویم
\\
دشمن خویشتنم هر نفس از دوستی او
&&
تا چه دید از من مسکین که ملول است ز خویم
\\
لب او بر لب من این چه خیال است و تمنا
&&
مگر آن گه که کند کوزه‌گر از خاک سبویم
\\
همه بر من چه زنی زخم فراق ای مه خوبان
&&
نه منم تنها کاندر خم چوگان تو گویم
\\
هر کجا صاحب حسنیست ثنا گفتم و وصفش
&&
تو چنان صاحب حسنی که ندانم که چه گویم
\\
دوش می‌گفت که سعدی غم ما هیچ ندارد
&&
می‌نداند که گرم سر برود دست نشویم
\\
\end{longtable}
\end{center}
