\begin{center}
\section*{غزل ۶۱۷: نه طریق دوستانست و نه شرط مهربانی}
\label{sec:617}
\addcontentsline{toc}{section}{\nameref{sec:617}}
\begin{longtable}{l p{0.5cm} r}
نه طریق دوستان است و نه شرط مهربانی
&&
که به دوستان یک دل سر دست برفشانی
\\
دلم از تو چون برنجد که به وهم در نگنجد
&&
که جواب تلخ گویی تو بدین شکردهانی
\\
نفسی بیا و بنشین سخنی بگو و بشنو
&&
که به تشنگی بمردم بر آب زندگانی
\\
غم دل به کس نگویم که بگفت رنگ رویم
&&
تو به صورتم نگه کن که سرایرم بدانی
\\
عجبت نیاید از من سخنان سوزناکم
&&
عجب است اگر بسوزم چو بر آتشم نشانی؟
\\
دل عارفان ببردند و قرار پارسایان
&&
همه شاهدان به صورت تو به صورت و معانی
\\
نه خلاف عهد کردم که حدیث جز تو گفتم
&&
همه بر سر زبانند و تو در میان جانی
\\
اگرت به هر که دنیا بدهند حیف باشد
&&
و گرت به هر چه عقبی بخرند رایگانی
\\
تو نظیر من ببینی و بدیل من بگیری
&&
عوض تو من نیابم که به هیچ کس نمانی
\\
نه عجب کمال حسنت که به صد زبان بگویم
&&
که هنوز پیش ذکرت خجلم ز بی زبانی
\\
مده ای رفیق پندم که نظر بر او فکندم
&&
تو میان ما ندانی که چه می‌رود نهانی
\\
مزن ای عدو به تیرم که بدین قدر نمیرم
&&
خبرش بگو که جانت بدهم به مژدگانی
\\
بت من چه جای لیلی که بریخت خون مجنون
&&
اگر این قمر ببینی دگر آن سمر نخوانی
\\
دل دردمند سعدی ز محبت تو خون شد
&&
نه به وصل می‌رسانی نه به قتل می‌رهانی
\\
\end{longtable}
\end{center}
