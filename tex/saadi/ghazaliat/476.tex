\begin{center}
\section*{غزل ۴۷۶: صبحم از مشرق برآمد باد نوروز از یمین}
\label{sec:476}
\addcontentsline{toc}{section}{\nameref{sec:476}}
\begin{longtable}{l p{0.5cm} r}
صبحم از مشرق برآمد باد نوروز از یمین
&&
عقل و طبعم خیره گشت از صنع رب العالمین
\\
با جوانان راه صحرا برگرفتم بامداد
&&
کودکی گفتا تو پیری با خردمندان نشین
\\
گفتم ای غافل نبینی کوه با چندین وقار
&&
همچو طفلان دامنش پرارغوان و یاسمین
\\
آستین بر دست پوشید از بهار برگ شاخ
&&
میوه پنهان کرده از خورشید و مه در آستین
\\
باد گل‌ها را پریشان می‌کند هر صبحدم
&&
زان پریشانی مگر در روی آب افتاده چین
\\
نوبهار از غنچه بیرون شد به یک تو پیرهن
&&
بیدمشک انداخت تا دیگر زمستان پوستین
\\
این نسیم خاک شیراز است یا مشک ختن
&&
یا نگار من پریشان کرده زلف عنبرین
\\
بامدادش بین که چشم از خواب نوشین بر کند
&&
گر ندیدی سحر بابل در نگارستان چین
\\
گر سرش داری چو سعدی سر بنه مردانه وار
&&
با چنین معشوق نتوان باخت عشق الا چنین
\\
\end{longtable}
\end{center}
