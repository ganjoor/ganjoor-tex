\begin{center}
\section*{غزل ۱۳۱: دوشم آن سنگ دل پریشان داشت}
\label{sec:131}
\addcontentsline{toc}{section}{\nameref{sec:131}}
\begin{longtable}{l p{0.5cm} r}
دوشم آن سنگ دل پریشان داشت
&&
یار دل برده دست بر جان داشت
\\
دیده در می‌فشاند در دامن
&&
گوییا آستین مرجان داشت
\\
اندرونم ز شوق می‌سوزد
&&
ور ننالیدمی چه درمان داشت
\\
می‌نپنداشتم که روز شود
&&
تا بدیدم سحر که پایان داشت
\\
در باغ بهشت بگشودند
&&
باد گویی کلید رضوان داشت
\\
غنچه دیدم که از نسیم صبا
&&
همچو من دست در گریبان داشت
\\
که نه تنها منم ربوده عشق
&&
هر گلی بلبلی غزل خوان داشت
\\
رازم از پرده برملا افتاد
&&
چند شاید به صبر پنهان داشت
\\
سعدیا ترک جان بباید گفت
&&
که به یک دل دو دوست نتوان داشت
\\
\end{longtable}
\end{center}
