\begin{center}
\section*{غزل ۷۸: امشب به راستی شب ما روز روشنست}
\label{sec:078}
\addcontentsline{toc}{section}{\nameref{sec:078}}
\begin{longtable}{l p{0.5cm} r}
امشب به راستی شب ما روز روشن است
&&
عید وصال دوست علی رغم دشمن است
\\
باد بهشت می‌گذرد یا نسیم باغ
&&
یا نکهت دهان تو یا بوی لادن است
\\
هرگز نباشد از تن و جانت عزیزتر
&&
چشمم که در سرست و روانم که در تن است
\\
گردن نهم به خدمت و گوشت کنم به قول
&&
تا خاطرم معلق آن گوش و گردن است
\\
ای پادشاه سایه ز درویش وامگیر
&&
ناچار خوشه چین بود آن جا که خرمن است
\\
دور از تو در جهان فراخم مجال نیست
&&
عالم به چشم تنگ دلان چشم سوزن است
\\
عاشق گریختن نتواند که دست شوق
&&
هر جا که می‌رود متعلق به دامن است
\\
شیرین به در نمی‌رود از خانه بی رقیب
&&
داند شکر که دفع مگس بادبیزن است
\\
جور رقیب و سرزنش اهل روزگار
&&
با من همان حکایت گاو دهلزن است
\\
بازان شاه را حسد آید بدین شکار
&&
کان شاهباز را دل سعدی نشیمن است
\\
قلب رقیق چند بپوشد حدیث عشق
&&
هرچ آن به آبگینه بپوشی مبین است
\\
\end{longtable}
\end{center}
