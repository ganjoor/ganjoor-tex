\begin{center}
\section*{غزل ۲۹۵: خفتن عاشق یکیست بر سر دیبا و خار}
\label{sec:295}
\addcontentsline{toc}{section}{\nameref{sec:295}}
\begin{longtable}{l p{0.5cm} r}
خفتن عاشق یکیست بر سر دیبا و خار
&&
چون نتواند کشید دست در آغوش یار
\\
گر دگری را شکیب هست ز دیدار دوست
&&
من نتوانم گرفت بر سر آتش قرار
\\
آتش آه است و دود می‌رودش تا به سقف
&&
چشمه چشمست و موج می‌زندش بر کنار
\\
گر تو ز ما فارغی ما به تو مستظهریم
&&
ور تو ز ما بی نیاز ما به تو امیدوار
\\
ای که به یاران غار مشتغلی دوستکام
&&
غمزده‌ای بر درست چون سگ اصحاب غار
\\
این همه بار احتمال می‌کنم و می‌روم
&&
اشتر مست از نشاط گرم رود زیر بار
\\
ما سپر انداختیم گردن تسلیم پیش
&&
گر بکشی حاکمی ور بدهی زینهار
\\
تیغ جفا گر زنی ضرب تو آسایشست
&&
روی ترش گر کنی تلخ تو شیرین گوار
\\
سعدی اگر داغ عشق در تو مؤثر شود
&&
فخر بود بنده را داغ خداوندگار
\\
\end{longtable}
\end{center}
