\begin{center}
\section*{غزل ۵۴۷: خانه صاحب نظران می‌بری}
\label{sec:547}
\addcontentsline{toc}{section}{\nameref{sec:547}}
\begin{longtable}{l p{0.5cm} r}
خانه صاحب نظران می‌بری
&&
پرده پرهیزکنان می‌دری
\\
گر تو پری چهره نپوشی نقاب
&&
توبه صوفی به زیان آوری
\\
این چه وجود است نمی‌دانمت
&&
آدمیی یا ملکی یا پری
\\
گر همه سرمایه زیان می‌کند
&&
سود بود دیدن آن مشتری
\\
نسخه این روی به نقاش بر
&&
تا بکند توبه ز صورتگری
\\
با تترت حاجت شمشیر نیست
&&
حمله همی‌آری و دل می‌بری
\\
گر تو در آیینه تأمل کنی
&&
صورت خود باز به ما ننگری
\\
خسرو اگر عهد تو دریافتی
&&
دل به تو دادی که تو شیرین‌تری
\\
گر دری از خلق ببندم به روی
&&
بر تو نبندم که به خاطر دری
\\
سعدی اگر کشته شود در فراق
&&
زنده شود چون به سرش بگذری
\\
\end{longtable}
\end{center}
