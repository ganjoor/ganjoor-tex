\begin{center}
\section*{غزل ۲۶۱: یا رب شب دوشین چه مبارک سحری بود}
\label{sec:261}
\addcontentsline{toc}{section}{\nameref{sec:261}}
\begin{longtable}{l p{0.5cm} r}
یا رب شب دوشین چه مبارک سحری بود
&&
کاو را به سر کشته هجران گذری بود
\\
آن دوست که ما را به ارادت نظری هست
&&
با او مگر او را به عنایت نظری بود
\\
من بعد حکایت نکنم تلخی هجران
&&
کان میوه که از صبر برآمد شکری بود
\\
رویی نتوان گفت که حسنش به چه ماند
&&
گویی که در آن نیم شب از روز دری بود
\\
گویم قمری بود کس از من نپسندد
&&
باغی که به هر شاخ درختش قمری بود
\\
آن دم که خبر بودم از او تا تو نگویی
&&
کز خویشتن و هر که جهانم خبری بود
\\
در عالم وصفش به جهانی برسیدم
&&
کاندر نظرم هر دو جهان مختصری بود
\\
من بودم و او نی قلم اندر سر من کش
&&
با او نتوان گفت وجود دگری بود
\\
با غمزه خوبان که چو شمشیر کشیده‌ست
&&
در صبر بدیدم که نه محکم سپری بود
\\
سعدی نتوانی که دگر دیده بدوزی
&&
کان دل بربودند که صبرش قدری بود
\\
\end{longtable}
\end{center}
