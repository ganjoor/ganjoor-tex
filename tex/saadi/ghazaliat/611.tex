\begin{center}
\section*{غزل ۶۱۱: بهار آمد که هر ساعت رود خاطر به بستانی}
\label{sec:611}
\addcontentsline{toc}{section}{\nameref{sec:611}}
\begin{longtable}{l p{0.5cm} r}
بهار آمد که هر ساعت رود خاطر به بستانی
&&
به غلغل در سماع آیند هر مرغی به دستانی
\\
دم عیسیست پنداری نسیم باد نوروزی
&&
که خاک مرده باز آید در او روحی و ریحانی
\\
به جولان و خرامیدن در آمد سرو بستانی
&&
تو نیز ای سرو روحانی بکن یک بار جولانی
\\
به هر کویی پری رویی به چوگان می‌زند گویی
&&
تو خود گوی زنخ داری بساز از زلف چوگانی
\\
به چندین حیلت و حکمت که گوی از همگنان بردم
&&
به چوگانم نمی‌افتد چنین گوی زنخدانی
\\
بیار ای باغبان سروی به بالای دلارامم
&&
که باری من ندیدستم چنین گل در گلستانی
\\
تو آهو چشم نگذاری مرا از دست تا آن گه
&&
که همچون آهو از دستت نهم سر در بیابانی
\\
کمال حسن رویت را صفت کردن نمی‌دانم
&&
که حیران باز می‌مانم چه داند گفت حیرانی
\\
وصال توست اگر دل را مرادی هست و مطلوبی
&&
کنار توست اگر غم را کناری هست و پایانی
\\
طبیب از من به جان آمد که سعدی قصه کوته کن
&&
که دردت را نمی‌دانم برون از صبر درمانی
\\
\end{longtable}
\end{center}
