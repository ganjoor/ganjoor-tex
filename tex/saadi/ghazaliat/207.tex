\begin{center}
\section*{غزل ۲۰۷: تو را خود یک زمان با ما سر صحرا نمی‌باشد}
\label{sec:207}
\addcontentsline{toc}{section}{\nameref{sec:207}}
\begin{longtable}{l p{0.5cm} r}
تو را خود یک زمان با ما سر صحرا نمی‌باشد
&&
چو شمست خاطر رفتن به جز تنها نمی‌باشد
\\
دو چشم از ناز در پیشت فراغ از حال درویشت
&&
مگر کز خوبی خویشت نگه در ما نمی‌باشد
\\
ملک یا چشمه نوری پری یا لعبت حوری
&&
که بر گلبن گل سوری چنین زیبا نمی‌باشد
\\
پری رویی و مه پیکر سمن بویی و سیمین بر
&&
عجب کز حسن رویت در جهان غوغا نمی‌باشد
\\
چو نتوان ساخت بی رویت بباید ساخت با خویت
&&
که ما را از سر کویت سر دروا نمی‌باشد
\\
مرو هر سوی و هر جاگه که مسکینان نیند آگه
&&
نمی‌بیند کست ناگه که او شیدا نمی‌باشد
\\
جهانی در پیت مفتون به جای آب گریان خون
&&
عجب می‌دارم از هامون که چون دریا نمی‌باشد
\\
همه شب می‌پزم سودا به بوی وعده فردا
&&
شب سودای سعدی را مگر فردا نمی‌باشد
\\
چرا بر خاک این منزل نگریم تا بگیرد گل
&&
ولیکن با تو آهن دل دمم گیرا نمی‌باشد
\\
\end{longtable}
\end{center}
