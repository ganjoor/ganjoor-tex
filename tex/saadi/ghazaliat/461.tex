\begin{center}
\section*{غزل ۴۶۱: سهل باشد به ترک جان گفتن}
\label{sec:461}
\addcontentsline{toc}{section}{\nameref{sec:461}}
\begin{longtable}{l p{0.5cm} r}
سهل باشد به ترک جان گفتن
&&
ترک جانان نمی‌توان گفتن
\\
هر چه زان تلخ‌تر بخواهی گفت
&&
شکرین است از آن دهان گفتن
\\
توبه کردیم پیش بالایت
&&
سخن سرو بوستان گفتن
\\
آن چنان وهم در تو حیران است
&&
که نمی‌داندت نشان گفتن
\\
به کمندی درم که ممکن نیست
&&
رستگاری به الامان گفتن
\\
دفتری در تو وضع می‌کردم
&&
متردد شدم در آن گفتن
\\
که تو شیرین‌تری از آن شیرین
&&
که بشاید به داستان گفتن
\\
بلبلان نیک زهره می‌دارند
&&
با گل از دست باغبان گفتن
\\
من نمی‌یارم از جفای رقیب
&&
درد با یار مهربان گفتن
\\
وان که با یار هودجش نظر است
&&
نتواند به ساربان گفتن
\\
سخن سر به مهر دوست به دوست
&&
حیف باشد به ترجمان گفتن
\\
این حکایت که می‌کند سعدی
&&
بس بخواهند در جهان گفتن
\\
\end{longtable}
\end{center}
