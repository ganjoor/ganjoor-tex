\begin{center}
\section*{غزل ۹۲: کس به چشم در نمی‌آید که گویم مثل اوست}
\label{sec:092}
\addcontentsline{toc}{section}{\nameref{sec:092}}
\begin{longtable}{l p{0.5cm} r}
کس به چشمم در نمی‌آید که گویم مثل اوست
&&
خود به چشم عاشقان صورت نبندد مثل دوست
\\
هر که با مستان نشیند ترک مستوری کند
&&
آبروی نیکنامان در خرابات آب جوست
\\
جز خداوندان معنی را نغلطاند سماع
&&
اولت مغزی بباید تا برون آیی ز پوست
\\
بنده‌ام گو تاج خواهی بر سرم نه یا تبر
&&
هر چه پیش عاشقان آید ز معشوقان نکوست
\\
عقل باری خسروی می‌کرد بر ملک وجود
&&
باز چون فرهاد عاشق بر لب شیرین اوست
\\
عنبرین چوگان زلفش را گر استقصا کنی
&&
زیر هر مویی دلی بینی که سرگردان چو گوست
\\
سعدیا چندان که خواهی گفت وصف روی یار
&&
حسن گل بیش از قیاس بلبل بسیارگوست
\\
\end{longtable}
\end{center}
