\begin{center}
\section*{غزل ۶۵: عیب یاران و دوستان هنرست}
\label{sec:065}
\addcontentsline{toc}{section}{\nameref{sec:065}}
\begin{longtable}{l p{0.5cm} r}
عیب یاران و دوستان هنر است
&&
سخن دشمنان نه معتبر است
\\
مهر مهر از درون ما نرود
&&
ای برادر که نقش بر حجر است
\\
چه توان گفت در لطافت دوست
&&
هر چه گویم از آن لطیف‌تر است
\\
آن که منظور دیده و دل ماست
&&
نتوان گفت شمس یا قمر است
\\
هر کسی گو به حال خود باشید
&&
ای برادر که حال ما دگر است
\\
تو که در خواب بوده‌ای همه شب
&&
چه نصیبت ز بلبل سحر است
\\
آدمی را که جان معنی نیست
&&
در حقیقت درخت بی‌ثمر است
\\
ما پراکندگان مجموعیم
&&
یار ما غایب است و در نظر است
\\
برگ تر خشک می‌شود به زمان
&&
برگ چشمان ما همیشه تر است
\\
جان شیرین فدای صحبت یار
&&
شرم دارم که نیک مختصر است
\\
این قدر دون قدر اوست ولیک
&&
حد امکان ما همین قدر است
\\
پرده بر خود نمی‌توان پوشید
&&
ای برادر که عشق پرده در است
\\
سعدی از بارگاه قربت دوست
&&
تا خبر یافته‌ست بی‌خبر است
\\
ما سر اینک نهاده‌ایم به طوع
&&
تا خداوندگار را چه سر است
\\
\end{longtable}
\end{center}
