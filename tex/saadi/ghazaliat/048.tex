\begin{center}
\section*{غزل ۴۸: صبر کن ای دل که صبر سیرت اهل صفاست}
\label{sec:048}
\addcontentsline{toc}{section}{\nameref{sec:048}}
\begin{longtable}{l p{0.5cm} r}
صبر کن ای دل که صبر سیرت اهل صفاست
&&
چارهٔ عشق احتمال شرط محبت وفاست
\\
مالک رد و قبول هر چه کند پادشاست
&&
گر بزند حاکم است ور بنوازد رواست
\\
گر چه بخواند هنوز دست جزع بر دعاست
&&
ور چه براند هنوز روی امید از قفاست
\\
برق یمانی بجست باد بهاری بخاست
&&
طاقت مجنون برفت خیمهٔ لیلی کجاست
\\
غفلت از ایام عشق پیش محقق خطاست
&&
اول صبح است خیز کآخر دنیا فناست
\\
صحبت یار عزیز حاصل دور بقاست
&&
یک دمه دیدار دوست هر دو جهانش بهاست
\\
درد دل دوستان گر تو پسندی رواست
&&
هر چه مراد شماست غایت مقصود ماست
\\
بنده چه دعوی کند حکم خداوند راست
&&
گر تو قدم می‌نهی تا بنهم چشم راست
\\
از در خویشم مران کاین نه طریق وفاست
&&
در همه شهری غریب در همه ملکی گداست
\\
با همه جرمم امید با همه خوفم رجاست
&&
گر درم ما مس است لطف شما کیمیاست
\\
سعدی اگر عاشقی میل وصالت چراست
&&
هر که دل دوست جست مصلحت خود نخواست
\\
\end{longtable}
\end{center}
