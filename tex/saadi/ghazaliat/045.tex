\begin{center}
\section*{غزل ۴۵: خوش می‌رود این پسر که برخاست}
\label{sec:045}
\addcontentsline{toc}{section}{\nameref{sec:045}}
\begin{longtable}{l p{0.5cm} r}
خوش می‌رود این پسر که برخاست
&&
سرویست چنین که می‌رود راست
\\
ابروش کمان قتل عاشق
&&
گیسوش کمند عقل داناست
\\
بالای چنین اگر در اسلام
&&
گویند که هست زیر و بالاست
\\
ای آتش خرمن عزیزان
&&
بنشین که هزار فتنه برخاست
\\
بی جرم بکش که بنده مملوک
&&
بی شرع ببر که خانه یغماست
\\
دردت بکشم که درد داروست
&&
خارت بخورم که خار خرماست
\\
انگشت نمای خلق بودن
&&
زشت است ولیک با تو زیباست
\\
باید که سلامت تو باشد
&&
سهل است ملامتی که بر ماست
\\
جان در قدم تو ریخت سعدی
&&
وین منزلت از خدای می‌خواست
\\
خواهی که دگر حیات یابد
&&
یک بار بگو که کشتهٔ ماست
\\
\end{longtable}
\end{center}
