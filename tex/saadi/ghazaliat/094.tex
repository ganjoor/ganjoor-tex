\begin{center}
\section*{غزل ۹۴: خورشید زیر سایه زلف چو شام اوست}
\label{sec:094}
\addcontentsline{toc}{section}{\nameref{sec:094}}
\begin{longtable}{l p{0.5cm} r}
خورشید زیر سایه زلف چو شام اوست
&&
طوبی غلام قد صنوبرخرام اوست
\\
آن قامتست نی به حقیقت قیامتست
&&
زیرا که رستخیز من اندر قیام اوست
\\
بر مرگ دل خوشست در این واقعه مرا
&&
کآب حیات در لب یاقوت فام اوست
\\
بوی بهار می‌دمدم یا نسیم صبح
&&
باد بهشت می‌گذرد یا پیام اوست
\\
دل عشوه می‌فروخت که من مرغ زیرکم
&&
اینک فتاده در سر زلف چو دام اوست
\\
بیچاره مانده‌ام همه روزی به دام او
&&
و اینک فتاده‌ام به غریبی که کام اوست
\\
هر لحظه در برم دل از اندیشه خون شود
&&
تا خود غلام کیست که سعدی غلام اوست
\\
\end{longtable}
\end{center}
