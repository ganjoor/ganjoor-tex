\begin{center}
\section*{غزل ۱۱۸: جان ندارد هر که جانانیش نیست}
\label{sec:118}
\addcontentsline{toc}{section}{\nameref{sec:118}}
\begin{longtable}{l p{0.5cm} r}
جان ندارد هر که جانانیش نیست
&&
تنگ عیشست آن که بستانیش نیست
\\
هر که را صورت نبندد سر عشق
&&
صورتی دارد ولی جانیش نیست
\\
گر دلی داری به دلبندی بده
&&
ضایع آن کشور که سلطانیش نیست
\\
کامران آن دل که محبوبیش هست
&&
نیکبخت آن سر که سامانیش نیست
\\
چشم نابینا زمین و آسمان
&&
زان نمی‌بیند که انسانیش نیست
\\
عارفان درویش صاحب درد را
&&
پادشا خوانند گر نانیش نیست
\\
ماجرای عقل پرسیدم ز عشق
&&
گفت معزولست و فرمانیش نیست
\\
درد عشق از تندرستی خوشترست
&&
گر چه بیش از صبر درمانیش نیست
\\
هر که را با ماه رویی سرخوشست
&&
دولتی دارد که پایانیش نیست
\\
خانه زندانست و تنهایی ضلال
&&
هر که چون سعدی گلستانیش نیست
\\
\end{longtable}
\end{center}
