\begin{center}
\section*{غزل ۵۷۷: امیدوارم اگر صد رهم بیندازی}
\label{sec:577}
\addcontentsline{toc}{section}{\nameref{sec:577}}
\begin{longtable}{l p{0.5cm} r}
امیدوارم اگر صد رهم بیندازی
&&
که بار دیگرم از روی لطف بنوازی
\\
چو روزگار نسازد ستیزه نتوان برد
&&
ضرورت است که با روزگار در سازی
\\
جفای عشق تو بر عقل من همان مثل است
&&
که سرگزیت به کافر همی‌دهد غازی
\\
دریغ بازوی تقوا که دست رنگینت
&&
به عقل من به سرانگشت می‌کند بازی
\\
بسی مطالعه کردیم نقش عالم را
&&
ز هر که در نظر آید به حسن ممتازی
\\
هزار چون من اگر محنت و بلا بیند
&&
تو را از آن چه که در نعمتی و در نازی
\\
حدیث عشق تو پیدا نکردمی بر خلق
&&
گر آب دیده نکردی به گریه غمازی
\\
زهی سوار که صد دل به غمزه‌ای ببری
&&
هزار صید به یک تاختن بیندازی
\\
تو را چو سعدی اگر بنده‌ای بود چه شود
&&
که در رکاب تو باشد غلام شیرازی
\\
گرش به قهر برانی به لطف بازآید
&&
که زر همان بود ار چند بار بگدازی
\\
چو آب می‌رود این پارسی به قوت طبع
&&
نه مرکبیست که از وی سبق برد تازی
\\
\end{longtable}
\end{center}
