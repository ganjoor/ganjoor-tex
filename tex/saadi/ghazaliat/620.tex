\begin{center}
\section*{غزل ۶۲۰: فرخ صباح آن که تو بر وی نظر کنی}
\label{sec:620}
\addcontentsline{toc}{section}{\nameref{sec:620}}
\begin{longtable}{l p{0.5cm} r}
فرخ صباح آن که تو بر وی نظر کنی
&&
فیروز روز آن که تو بر وی گذر کنی
\\
آزاد بنده‌ای که بود در رکاب تو
&&
خرم ولایتی که تو آنجا سفر کنی
\\
دیگر نبات را نخرد مشتری به هیچ
&&
یک بار اگر تبسم همچون شکر کنی
\\
ای آفتاب روشن و ای سایه همای
&&
ما را نگاهی از تو تمام است اگر کنی
\\
من با تو دوستی و وفا کم نمی‌کنم
&&
چندان که دشمنی و جفا بیشتر کنی
\\
مقدور من سریست که در پایت افکنم
&&
گر زآن که التفات بدین مختصر کنی
\\
عمریست تا به یاد تو شب روز می‌کنم
&&
تو خفته‌ای که گوش به آه سحر کنی
\\
دانی که رویم از همه عالم به روی توست
&&
زنهار اگر تو روی به رویی دگر کنی
\\
گفتی که دیر و زود به حالت نظر کنم
&&
آری کنی چو بر سر خاکم گذر کنی
\\
شرط است سعدیا که به میدان عشق دوست
&&
خود را به پیش تیر ملامت سپر کنی
\\
وز عقل بهترت سپری باید ای حکیم
&&
تا از خدنگ غمزه خوبان حذر کنی
\\
\end{longtable}
\end{center}
