\begin{center}
\section*{غزل ۴۰: چنان به موی تو آشفته‌ام به بوی تو مست}
\label{sec:040}
\addcontentsline{toc}{section}{\nameref{sec:040}}
\begin{longtable}{l p{0.5cm} r}
چنان به موی تو آشفته‌ام به بوی تو مست
&&
که نیستم خبر از هر چه در دو عالم هست
\\
دگر به روی کسم دیده بر نمی‌باشد
&&
خلیل من همه بت‌های آزری بشکست
\\
مجال خواب نمی‌باشدم ز دست خیال
&&
در سرای نشاید بر آشنایان بست
\\
در قفس طلبد هر کجا گرفتاریست
&&
من از کمند تو تا زنده‌ام نخواهم جست
\\
غلام دولت آنم که پای بند یکیست
&&
به جانبی متعلق شد از هزار برست
\\
مطیع امر توام گر دلم بخواهی سوخت
&&
اسیر حکم توام گر تنم بخواهی خست
\\
نماز شام قیامت به هوش بازآید
&&
کسی که خورده بود می ز بامداد الست
\\
نگاه من به تو و دیگران به خود مشغول
&&
معاشران ز می و عارفان ز ساقی مست
\\
اگر تو سرو خرامان ز پای ننشینی
&&
چه فتنه‌ها که بخیزد میان اهل نشست
\\
برادران و بزرگان نصیحتم مکنید
&&
که اختیار من از دست رفت و تیر از شست
\\
حذر کنید ز باران دیده سعدی
&&
که قطره سیل شود چون به یک دگر پیوست
\\
خوش است نام تو بردن ولی دریغ بود
&&
در این سخن که بخواهند برد دست به دست
\\
\end{longtable}
\end{center}
