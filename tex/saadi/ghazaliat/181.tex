\begin{center}
\section*{غزل ۱۸۱: باد آمد و بوی عنبر آورد}
\label{sec:181}
\addcontentsline{toc}{section}{\nameref{sec:181}}
\begin{longtable}{l p{0.5cm} r}
باد آمد و بوی عنبر آورد
&&
بادام شکوفه بر سر آورد
\\
شاخ گل از اضطراب بلبل
&&
با آن همه خار سر درآورد
\\
تا پای مبارکش ببوسم
&&
قاصد که پیام دلبر آورد
\\
ما نامه بدو سپرده بودیم
&&
او نافه مشک اذفر آورد
\\
هرگز نشنیده‌ام که بادی
&&
بوی گلی از تو خوشتر آورد
\\
کس مثل تو خوبروی فرزند
&&
نشنید که هیچ مادر آورد
\\
بیچاره کسی که در فراقت
&&
روزی به نماز دیگر آورد
\\
سعدی دل روشنت صدف وار
&&
هر قطره که خورد گوهر آورد
\\
شیرینی دختران طبعت
&&
شور از متمیزان برآورد
\\
شاید که کند به زنده در گور
&&
در عهد تو هر که دختر آورد
\\
\end{longtable}
\end{center}
