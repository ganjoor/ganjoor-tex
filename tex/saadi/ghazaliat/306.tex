\begin{center}
\section*{غزل ۳۰۶: ای پسر دلربا وی قمر دلپذیر}
\label{sec:306}
\addcontentsline{toc}{section}{\nameref{sec:306}}
\begin{longtable}{l p{0.5cm} r}
ای پسر دلربا وی قمر دلپذیر
&&
از همه باشد گریز وز تو نباشد گزیر
\\
تا تو مصور شدی در دل یکتای من
&&
جای تصور نماند دیگرم اندر ضمیر
\\
عیب کنندم که چند در پی خوبان روی
&&
چون نرود بنده‌وار هر که برندش اسیر
\\
بسته زنجیر زلف زود نیابد خلاص
&&
دیر برآید به جهد هر که فرو شد به قیر
\\
چون تو بتی بگذرد سروقد سیم ساق
&&
هر که در او ننگرد مرده بود یا ضریر
\\
گر نبرم ناز دوست کیست که مانند اوست
&&
کبر کند بی خلاف هر که بود بی‌نظیر
\\
قامت زیبای سرو کاین همه وصفش کنند
&&
هست به صورت بلند لیک به معنی قصیر
\\
هر که طلبکار توست روی نتابد ز تیغ
&&
وان که هوادار توست بازنگردد به تیر
\\
بوسه دهم بنده‌وار بر قدمت ور سرم
&&
در سر این می‌رود بی سر و پایی مگیر
\\
سعدی اگر خون و مال صرف شود در وصال
&&
آنت مقامی بزرگ اینت بهایی حقیر
\\
گر تو ز ما فارغی وز همه کس بی نیاز
&&
ما به تو مستظهریم وز همه عالم فقیر
\\
\end{longtable}
\end{center}
