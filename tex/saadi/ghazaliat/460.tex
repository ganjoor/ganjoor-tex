\begin{center}
\section*{غزل ۴۶۰: خلاف دوستی کردن به ترک دوستان گفتن}
\label{sec:460}
\addcontentsline{toc}{section}{\nameref{sec:460}}
\begin{longtable}{l p{0.5cm} r}
خلاف دوستی کردن به ترک دوستان گفتن
&&
نبایستی نمود این روی و دیگر باز بنهفتن
\\
گدایی پادشاهی را به شوخی دوست می‌دارد
&&
نه بی او می‌توان بودن نه با او می‌توان گفتن
\\
هزارم درد می‌باشد که می‌گویم نهان دارم
&&
لبم با هم نمی‌آید چو غنچه روز بشکفتن
\\
ز دستم بر نمی‌خیزد که انصاف از تو بستانم
&&
روا داری گناه خویش و آنگه بر من آشفتن
\\
که می‌گوید به بالای تو ماند سرو بستانی
&&
بیاور در چمن سروی که بتواند چنین رفتن
\\
چنانت دوست می‌دارم که وصلم دل نمی‌خواهد
&&
کمال دوستی باشد مراد از دوست نگرفتن
\\
مراد خسرو از شیرین کناری بود و آغوشی
&&
محبت کار فرهاد است و کوه بیستون سفتن
\\
نصیحت گفتن آسان است سرگردان عاشق را
&&
ولیکن با که می‌گویی که نتواند پذیرفتن
\\
شکایت پیش از این حالت به نزدیکان و غمخواران
&&
ز دست خواب می‌کردم کنون از دست ناخفتن
\\
گر از شمشیر برگردی نه عالی همتی سعدی
&&
تو کز نیشی بیازردی نخواهی انگبین رفتن
\\
\end{longtable}
\end{center}
