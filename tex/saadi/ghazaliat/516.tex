\begin{center}
\section*{غزل ۵۱۶: کدام کس به تو ماند که گویمت که چنویی}
\label{sec:516}
\addcontentsline{toc}{section}{\nameref{sec:516}}
\begin{longtable}{l p{0.5cm} r}
کدام کس به تو ماند که گویمت که چنویی
&&
ز هر که در نظر آید گذشته‌ای به نکویی
\\
لطیف جوهر و جانی غریب قامت و شکلی
&&
نظیف جامه و جسمی بدیع صورت و خویی
\\
هزار دیده چو پروانه بر جمال تو عاشق
&&
غلام مجلس آنم که شمع مجلس اویی
\\
ندیدم آبی و خاکی بدین لطافت و پاکی
&&
تو آب چشمه حیوان و خاک غالیه بویی
\\
تو را که درد نباشد ز درد ما چه تفاوت
&&
تو حال تشنه ندانی که بر کناره جویی
\\
صبای روضه رضوان ندانمت که چه بادی
&&
نسیم وعده جانان ندانمت که چه بویی
\\
اگر من از دل یک تو برآورم دم عشقی
&&
عجب مدار که آتش درافتدم به دوتویی
\\
به کس مگوی که پایم به سنگ عشق برآمد
&&
که عیب گیرد و گوید چرا به فرق نپویی
\\
دلی دو دوست نگیرد دو مهر دل نپذیرد
&&
اگر موافق اویی به ترک خویش بگویی
\\
کنونم آب حیاتی به حلق تشنه فروکن
&&
نه آنگهی که بمیرم به آب دیده بشویی
\\
به اختیار تو سعدی چه التماس برآید
&&
گر او مراد نبخشد تو کیستی که بجویی
\\
\end{longtable}
\end{center}
