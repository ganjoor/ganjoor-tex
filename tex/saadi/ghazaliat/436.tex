\begin{center}
\section*{غزل ۴۳۶: عمرها در پی مقصود به جان گردیدیم}
\label{sec:436}
\addcontentsline{toc}{section}{\nameref{sec:436}}
\begin{longtable}{l p{0.5cm} r}
عمرها در پی مقصود به جان گردیدیم
&&
دوست در خانه و ما گرد جهان گردیدیم
\\
خود سراپرده قدرش ز مکان بیرون بود
&&
آن که ما در طلبش جمله مکان گردیدیم
\\
همچو بلبل همه شب نعره زنان تا خورشید
&&
روی بنمود چو خفاش نهان گردیدیم
\\
گفته بودیم به خوبان که نباید نگریست
&&
دل ببردند و ضرورت نگران گردیدیم
\\
صفت یوسف نادیده بیان می‌کردند
&&
با میان آمد و بی نام و نشان گردیدیم
\\
رفته بودیم به خلوت که دگر می نخوریم
&&
ساقیا باده بده کز سر آن گردیدیم
\\
تا همه شهر بیایند و ببینند که ما
&&
پیر بودیم و دگرباره جوان گردیدیم
\\
سعدیا لشکر خوبان به شکار دل ما
&&
گو میایید که ما صید فلان گردیدیم
\\
\end{longtable}
\end{center}
