\begin{center}
\section*{غزل ۱۰: با جوانی سرخوشست این پیر بی تدبیر را}
\label{sec:010}
\addcontentsline{toc}{section}{\nameref{sec:010}}
\begin{longtable}{l p{0.5cm} r}
با جوانی سر خوش است این پیر بی تدبیر را
&&
جهل باشد با جوانان پنجه کردن پیر را
\\
من که با مویی به قوت برنیایم ای عجب
&&
با یکی افتاده‌ام کاو بگسلد زنجیر را
\\
چون کمان در بازو آرد سروقد سیمتن
&&
آرزویم می‌کند کآماج باشم تیر را
\\
می‌رود تا در کمند افتد به پای خویشتن
&&
گر بر آن دست و کمان چشم اوفتد نخجیر را
\\
کس ندیدست آدمیزاد از تو شیرین‌تر سخن
&&
شکر از پستان مادر خورده‌ای یا شیر را
\\
روز بازار جوانی پنج روزی بیش نیست
&&
نقد را باش ای پسر کآفت بود تأخیر را
\\
ای که گفتی دیده از دیدار بت رویان بدوز
&&
هر چه گویی چاره دانم کرد جز تقدیر را
\\
زهد پیدا کفر پنهان بود چندین روزگار
&&
پرده از سر برگرفتیم آن همه تزویر را
\\
سعدیا در پای جانان گر به خدمت سر نهی
&&
همچنان عذرت بباید خواستن تقصیر را
\\
\end{longtable}
\end{center}
