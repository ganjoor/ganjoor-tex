\begin{center}
\section*{غزل ۶۱۰: بنده‌ام گر به لطف می‌خوانی}
\label{sec:610}
\addcontentsline{toc}{section}{\nameref{sec:610}}
\begin{longtable}{l p{0.5cm} r}
بنده‌ام گر به لطف می‌خوانی
&&
حاکمی گر به قهر می‌رانی
\\
کس نشاید که بر تو بگزینند
&&
که تو صورت به کس نمی‌مانی
\\
ندهیمت به هر که در عالم
&&
ور تو ما را به هیچ نستانی
\\
گفتم این درد عشق پنهان را
&&
به تو گویم که هم تو درمانی
\\
بازگفتم چه حاجت است به قول
&&
که تو خود در دلی و می‌دانی
\\
نفس را عقل تربیت می‌کرد
&&
کز طبیعت عنان بگردانی
\\
عشق دانی چه گفت تقوا را
&&
پنجه با ما مکن که نتوانی
\\
چه خبر دارد از حقیقت عشق
&&
پای بند هوای نفسانی
\\
خودپرستان نظر به شخص کنند
&&
پاک بینان به صنع ربانی
\\
شب قدری بود که دست دهد
&&
عارفان را سماع روحانی
\\
رقص وقتی مسلمت باشد
&&
کآستین بر دو عالم افشانی
\\
قصه عشق را نهایت نیست
&&
صبر پیدا و درد پنهانی
\\
سعدیا دیگر این حدیث مگوی
&&
تا نگویند قصه می‌خوانی
\\
\end{longtable}
\end{center}
