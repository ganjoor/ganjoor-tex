\begin{center}
\section*{غزل ۲۱۱: امروز در فراق تو دیگر به شام شد}
\label{sec:211}
\addcontentsline{toc}{section}{\nameref{sec:211}}
\begin{longtable}{l p{0.5cm} r}
امروز در فراق تو دیگر به شام شد
&&
ای دیده پاس دار که خفتن حرام شد
\\
بیش احتمال سنگ جفا خوردنم نماند
&&
کز رقت اندرون ضعیفم چو جام شد
\\
افسوس خلق می‌شنوم در قفای خویش
&&
کاین پخته بین که در سر سودای خام شد
\\
تنها نه من به دانه خالت مقیدم
&&
این دانه هر که دید گرفتار دام شد
\\
گفتم یکی به گوشه چشمت نظر کنم
&&
چشمم در او بماند و زیادت مقام شد
\\
ای دل نگفتمت که عنان نظر بتاب
&&
اکنونت افکند که ز دستت لگام شد
\\
نامم به عاشقی شد و گویند توبه کن
&&
توبت کنون چه فایده دارد که نام شد
\\
از من به عشق روی تو می‌زاید این سخن
&&
طوطی شکر شکست که شیرین کلام شد
\\
ابنای روزگار غلامان به زر خرند
&&
سعدی تو را به طوع و ارادت غلام شد
\\
آن مدعی که دست ندادی به بند کس
&&
این بار در کمند تو افتاد و رام شد
\\
شرح غمت به وصف نخواهد شدن تمام
&&
جهدم به آخر آمد و دفتر تمام شد
\\
\end{longtable}
\end{center}
