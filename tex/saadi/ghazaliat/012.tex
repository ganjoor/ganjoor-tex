\begin{center}
\section*{غزل ۱۲: دوست می‌دارم من این نالیدن دلسوز را}
\label{sec:012}
\addcontentsline{toc}{section}{\nameref{sec:012}}
\begin{longtable}{l p{0.5cm} r}
دوست می‌دارم من این نالیدن دلسوز را
&&
تا به هر نوعی که باشد بگذرانم روز را
\\
شب همه شب انتظار صبح رویی می‌رود
&&
کان صباحت نیست این صبح جهان افروز را
\\
وه که گر من بازبینم چهر مهرافزای او
&&
تا قیامت شکر گویم طالع پیروز را
\\
گر من از سنگ ملامت روی برپیچم زنم
&&
جان سپر کردند مردان ناوک دلدوز را
\\
کامجویان را ز ناکامی چشیدن چاره نیست
&&
بر زمستان صبر باید طالب نوروز را
\\
عاقلان خوشه چین از سر لیلی غافلند
&&
این کرامت نیست جز مجنون خرمن سوز را
\\
عاشقان دین و دنیاباز را خاصیتیست
&&
کان نباشد زاهدان مال و جاه اندوز را
\\
دیگری را در کمند آور که ما خود بنده‌ایم
&&
ریسمان در پای حاجت نیست دست آموز را
\\
سعدیا دی رفت و فردا همچنان موجود نیست
&&
در میان این و آن فرصت شمار امروز را
\\
\end{longtable}
\end{center}
