\begin{center}
\section*{غزل ۱۲۲: در من این هست که صبرم ز نکورویان نیست}
\label{sec:122}
\addcontentsline{toc}{section}{\nameref{sec:122}}
\begin{longtable}{l p{0.5cm} r}
در من این هست که صبرم ز نکورویان نیست
&&
زرق نفروشم و زهدی ننمایم کان نیست
\\
ای که منظور ببینی و تأمل نکنی
&&
گر تو را قوت این هست مرا امکان نیست
\\
ترک خوبان خطا عین صوابست ولیک
&&
چه کند بنده که بر نفس خودش فرمان نیست
\\
من دگر میل به صحرا و تماشا نکنم
&&
که گلی همچو رخ تو به همه بستان نیست
\\
ای پری روی ملک صورت زیباسیرت
&&
هر که با مثل تو انسش نبود انسان نیست
\\
چشم برکرده بسی خلق که نابینااند
&&
مثل صورت دیوار که در وی جان نیست
\\
درد دل با تو همان به که نگوید درویش
&&
ای برادر که تو را درد دلی پنهان نیست
\\
آن که من در قلم قدرت او حیرانم
&&
هیچ مخلوق ندانم که در او حیران نیست
\\
سعدیا عمر گران مایه به پایان آمد
&&
همچنان قصه سودای تو را پایان نیست
\\
\end{longtable}
\end{center}
