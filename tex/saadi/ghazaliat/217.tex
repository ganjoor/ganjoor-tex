\begin{center}
\section*{غزل ۲۱۷: آن را که غمی چون غم من نیست چه داند}
\label{sec:217}
\addcontentsline{toc}{section}{\nameref{sec:217}}
\begin{longtable}{l p{0.5cm} r}
آن را که غمی چون غم من نیست چه داند
&&
کز شوق توام دیده چه شب می‌گذراند
\\
وقت است اگر از پای درآیم که همه عمر
&&
باری نکشیدم که به هجران تو ماند
\\
سوز دل یعقوب ستمدیده ز من پرس
&&
کاندوه دل سوختگان سوخته داند
\\
دیوانه گرش پند دهی کار نبندد
&&
ور بند نهی سلسله در هم گسلاند
\\
ما بی تو به دل برنزدیم آب صبوری
&&
در آتش سوزنده صبوری که تواند
\\
هر گه که بسوزد جگرم دیده بگرید
&&
وین گریه نه آبیست که آتش بنشاند
\\
سلطان خیالت شبی آرام نگیرد
&&
تا بر سر صبر من مسکین ندواند
\\
شیرین ننماید به دهانش شکر وصل
&&
آن را که فلک زهر جدایی نچشاند
\\
گر بار دگر دامن کامی به کف آرم
&&
تا زنده‌ام از چنگ منش کس نرهاند
\\
ترسم که نمانم من از این رنج دریغا
&&
کاندر دل من حسرت روی تو بماند
\\
قاصد رود از پارس به کشتی به خراسان
&&
گر چشم من اندر عقبش سیل براند
\\
فریاد که گر جور فراق تو نویسم
&&
فریاد برآید ز دل هر که بخواند
\\
شرح غم هجران تو هم با تو توان گفت
&&
پیداست که قاصد چه به سمع تو رساند
\\
زنهار که خون می‌چکد از گفته سعدی
&&
هرک این همه نشتر بخورد خون بچکاند
\\
\end{longtable}
\end{center}
