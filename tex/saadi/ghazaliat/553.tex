\begin{center}
\section*{غزل ۵۵۳: سرو بستانی تو یا مه یا پری}
\label{sec:553}
\addcontentsline{toc}{section}{\nameref{sec:553}}
\begin{longtable}{l p{0.5cm} r}
سرو بستانی تو یا مه یا پری
&&
یا ملک یا دفتر صورتگری
\\
رفتنی داری و سحری می‌کنی
&&
کاندر آن عاجز بماند سامری
\\
هر که یک بارش گذشتی در نظر
&&
در دلش صد بار دیگر بگذری
\\
می‌روی و اندر پیت دل می‌رود
&&
باز می‌آیی و جان می‌پروری
\\
گر تو شاهد با میان آیی چو شمع
&&
مبلغی پروانه‌ها گرد آوری
\\
چند خواهی روی پنهان داشتن
&&
پرده می‌پوشی و بر ما می‌دری
\\
روزی آخر در میان مردم آی
&&
تا ببیند هر که می‌بیند پری
\\
آفتاب از منظر افتد در رواق
&&
چون تو را بیند بدین خوش منظری
\\
جان و خاطر با تو دارم روز و شب
&&
نقش بر دل نام بر انگشتری
\\
سعدی از گرمی بخواهد سوختن
&&
بس که تو شیرینی از حد می‌بری
\\
\end{longtable}
\end{center}
