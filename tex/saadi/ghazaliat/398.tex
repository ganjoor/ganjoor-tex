\begin{center}
\section*{غزل ۳۹۸: نظر از مدعیان بر تو نمی‌اندازم}
\label{sec:398}
\addcontentsline{toc}{section}{\nameref{sec:398}}
\begin{longtable}{l p{0.5cm} r}
نظر از مدعیان بر تو نمی‌اندازم
&&
تا نگویند که من با تو نظر می‌بازم
\\
آرزو می‌کندم در همه عالم صیدی
&&
که نباشند رفیقان حسود انبازم
\\
درد پنهان فراقم ز تحمل بگذشت
&&
ور نه از دل نرسیدی به زبان آوازم
\\
چون کبوتر بگرفتیم به دام سر زلف
&&
دیده بردوختی از خلق جهان چون بازم
\\
به سرانگشت بخواهی دل مسکینان برد
&&
دست واپوش که من پنجه نمی‌اندازم
\\
مطرب آهنگ بگردان که دگر هیچ نماند
&&
که از این پرده که گفتی به درافتد رازم
\\
کس ننالید در این عهد چو من در غم دوست
&&
که به آفاق نظر می‌رود از شیرازم
\\
چند گفتند که سعدی نفسی باز خود آی
&&
گفتم از دوست نشاید که به خود پردازم
\\
\end{longtable}
\end{center}
