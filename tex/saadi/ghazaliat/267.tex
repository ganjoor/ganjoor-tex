\begin{center}
\section*{غزل ۲۶۷: سروبالایی به صحرا می‌رود}
\label{sec:267}
\addcontentsline{toc}{section}{\nameref{sec:267}}
\begin{longtable}{l p{0.5cm} r}
سروبالایی به صحرا می‌رود
&&
رفتنش بین تا چه زیبا می‌رود
\\
تا کدامین باغ از او خرمترست
&&
کاو به رامش کردن آنجا می‌رود
\\
می‌رود در راه و در اجزای خاک
&&
مرده می‌گوید مسیحا می‌رود
\\
این چنین بیخود نرفتی سنگدل
&&
گر بدانستی چه بر ما می‌رود
\\
اهل دل را گو نگه دارید چشم
&&
کان پری پیکر به یغما می‌رود
\\
هر که را در شهر دید از مرد و زن
&&
دل ربود اکنون به صحرا می‌رود
\\
آفتاب و سرو غیرت می‌برند
&&
کآفتابی سروبالا می‌رود
\\
باغ را چندان بساط افکنده‌اند
&&
کآدمی بر فرش دیبا می‌رود
\\
عقل را با عشق زور پنجه نیست
&&
کار مسکین از مدارا می‌رود
\\
سعدیا دل در سرش کردی و رفت
&&
بلکه جانش نیز در پا می‌رود
\\
\end{longtable}
\end{center}
