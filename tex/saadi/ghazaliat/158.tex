\begin{center}
\section*{غزل ۱۵۸: مویت رها مکن که چنین بر هم اوفتد}
\label{sec:158}
\addcontentsline{toc}{section}{\nameref{sec:158}}
\begin{longtable}{l p{0.5cm} r}
مویت رها مکن که چنین بر هم اوفتد
&&
کآشوب حسن روی تو در عالم اوفتد
\\
گر در خیال خلق پری وار بگذری
&&
فریاد در نهاد بنی آدم اوفتد
\\
افتاده تو شد دلم ای دوست دست گیر
&&
در پای مفکنش که چنین دل کم اوفتد
\\
در رویت آن که تیغ نظر می‌کشد به جهل
&&
مانند من به تیر بلا محکم اوفتد
\\
مشکن دلم که حقه راز نهان توست
&&
ترسم که راز در کف نامحرم اوفتد
\\
وقتست اگر بیایی و لب بر لبم نهی
&&
چندم به جست و جوی تو دم بر دم اوفتد
\\
سعدی صبور باش بر این ریش دردناک
&&
باشد که اتفاق یکی مرهم اوفتد
\\
\end{longtable}
\end{center}
