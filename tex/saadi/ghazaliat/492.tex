\begin{center}
\section*{غزل ۴۹۲: ای باغ حسن چون تو نهالی نیافته}
\label{sec:492}
\addcontentsline{toc}{section}{\nameref{sec:492}}
\begin{longtable}{l p{0.5cm} r}
ای باغ حسن چون تو نهالی نیافته
&&
رخساره زمین چو تو خالی نیافته
\\
تابنده‌تر ز روی تو ماهی ندیده چرخ
&&
خوشتر ز ابروی تو هلالی نیافته
\\
بر دور عارض تو نظر کرده آفتاب
&&
خود را لطافتی و جمالی نیافته
\\
چرخ مشعبد از رخ تو دلفریبتر
&&
در زیر هفت پرده خیالی نیافته
\\
خود را به زیر چنگل شاهین عشق تو
&&
عنقای صبر من پر و بالی نیافته
\\
تا کی ز درد عشق تو نالد روان من
&&
روزی به لطف از تو مثالی نیافته
\\
افتاده در زبان خلایق حدیث من
&&
با تو به یک حدیث مجالی نیافته
\\
زایل شود هر آن چه به کلی کمال یافت
&&
عمرم زوال یافت کمالی نیافته
\\
گلبرگ عیش من به چه امید بشکفد
&&
از بوستان وصل شمالی نیافته
\\
سعدی هزار جامه به روزی قبا کند
&&
یک مهربانی از تو به سالی نیافته
\\
\end{longtable}
\end{center}
