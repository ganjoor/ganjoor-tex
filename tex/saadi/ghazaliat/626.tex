\begin{center}
\section*{غزل ۶۲۶: امروز چنانی ای پری روی}
\label{sec:626}
\addcontentsline{toc}{section}{\nameref{sec:626}}
\begin{longtable}{l p{0.5cm} r}
امروز چنانی ای پری روی
&&
کز ماه به حسن می‌بری گوی
\\
می‌آیی و در پی تو عشاق
&&
دیوانه شده دوان به هر سوی
\\
اینک من و زنگیان کافر
&&
وان ملعب لعبتان جادوی
\\
آورده ز غمزه سحر در چشم
&&
در داده ز فتنه تاب در موی
\\
وز بهر شکار دل نهاده
&&
تیر مژه در کمان ابروی
\\
نرخ گل و گلشکر شکسته
&&
زآن چهره خوب و لعل دلجوی
\\
چاکر شده شاه اخترانت
&&
شیر فلکت شده سگ کوی
\\
بر بام سراچه جمالت
&&
کیوان شده پاسبان هندوی
\\
عارض به مثل چو برگ نسرین
&&
بالا به صفت چو سرو خودروی
\\
گویی به چه شانه کرده‌ای زلف
&&
یا خود به چه آب شسته‌ای روی
\\
کز روی به لاله می‌دهی رنگ
&&
وز زلف به مشک می‌دهی بوی
\\
چون سعدی صد هزار بلبل
&&
گلزار رخ تو را غزل گوی
\\
\end{longtable}
\end{center}
