\begin{center}
\section*{غزل ۶۳۵: اگرم حیات بخشی و گرم هلاک خواهی}
\label{sec:635}
\addcontentsline{toc}{section}{\nameref{sec:635}}
\begin{longtable}{l p{0.5cm} r}
اگرم حیات بخشی وگرم هلاک خواهی
&&
سر بندگی به حکمت بنهم که پادشاهی
\\
من اگر هزار خدمت بکنم گناهکارم
&&
تو هزار خون ناحق بکنی و بی گناهی
\\
به کسی نمی‌توانم که شکایت از تو خوانم
&&
همه جانب تو خواهند و تو آن کنی که خواهی
\\
تو به آفتاب مانی ز کمال حسن طلعت
&&
که نظر نمی‌تواند که ببیندت کماهی
\\
من اگر چنان که نهی است نظر به دوست کردن
&&
همه عمر توبه کردم که نگردم از مناهی
\\
به خدای اگر به دردم بکشی که برنگردم
&&
کسی از تو چون گریزد که تواش گریزگاهی
\\
منم ای نگار و چشمی که در انتظار رویت
&&
همه شب نخفت مسکین و بخفت مرغ و ماهی
\\
و گر این شب درازم بکشد در آرزویت
&&
نه عجب که زنده گردم به نسیم صبحگاهی
\\
غم عشق اگر بکوشم که ز دوستان بپوشم
&&
سخنان سوزناکم بدهد بر آن گواهی
\\
خضری چو کلک سعدی همه روز در سیاحت
&&
نه عجب گر آب حیوان به درآید از سیاهی
\\
\end{longtable}
\end{center}
