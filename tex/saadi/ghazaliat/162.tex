\begin{center}
\section*{غزل ۱۶۲: طرفه می‌دارند یاران صبر من بر داغ و درد}
\label{sec:162}
\addcontentsline{toc}{section}{\nameref{sec:162}}
\begin{longtable}{l p{0.5cm} r}
طرفه می‌دارند یاران صبر من بر داغ و درد
&&
داغ و دردی کز تو باشد خوشترست از باغ ورد
\\
دوستانت را که داغ مهربانی دل بسوخت
&&
گر به دوزخ بگذرانی آتشی بینند سرد
\\
حاکمی گر عدل خواهی کرد با ما یا ستم
&&
بنده‌ایم ار صلح خواهی جست با ما یا نبرد
\\
عقل را با عشق خوبان طاقت سرپنجه نیست
&&
با قضای آسمانی برنتابد جهد مرد
\\
عافیت می‌بایدت چشم از نکورویان بدوز
&&
عشق می‌ورزی بساط نیک نامی درنورد
\\
زهره مردان نداری چون زنان در خانه باش
&&
ور به میدان می‌روی از تیرباران برمگرد
\\
حمل رعنایی مکن بر گریه صاحب سماع
&&
اهل دل داند که تا زخمی نخورد آهی نکرد
\\
هیچ کس را بر من از یاران مجلس دل نسوخت
&&
شمع می‌بینم که اشکش می‌رود بر روی زرد
\\
با شکایت‌ها که دارم از زمستان فراق
&&
گر بهاری باز باشد لیس بعد الورد برد
\\
هر که را دردی چو سعدی می‌گدازد گو منال
&&
چون دلارامش طبیبی می‌کند داروست درد
\\
\end{longtable}
\end{center}
