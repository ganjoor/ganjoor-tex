\begin{center}
\section*{غزل ۱۸۸: به حدیث درنیایی که لبت شکر نریزد}
\label{sec:188}
\addcontentsline{toc}{section}{\nameref{sec:188}}
\begin{longtable}{l p{0.5cm} r}
به حدیث درنیایی که لبت شکر نریزد
&&
نچمی که شاخ طوبی به ستیزه برنریزد
\\
هوس تو هیچ طبعی نپزد که سر نبازد
&&
ز پی تو هیچ مرغی نپرد که پر نریزد
\\
دلم از غمت زمانی نتواند ار ننالد
&&
مژه یک دم آب حسرت نشکیبد ار نریزد
\\
که نه من ز دست خوبان نبرم به عاقبت جان
&&
تو مرا بکش که خونم ز تو خوبتر نریزد
\\
دررست لفظ سعدی ز فراز بحر معنی
&&
چه کند به دامنی در که به دوست برنریزد
\\
\end{longtable}
\end{center}
