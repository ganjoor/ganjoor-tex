\begin{center}
\section*{غزل ۵۹۹: چون تنگ نباشد دل مسکین حمامی}
\label{sec:599}
\addcontentsline{toc}{section}{\nameref{sec:599}}
\begin{longtable}{l p{0.5cm} r}
چون تنگ نباشد دل مسکین حمامی
&&
که‌ش یار هم آواز بگیرند به دامی
\\
دیشب همه شب دست در آغوش سلامت
&&
و امروز همه روز تمنای سلامی
\\
آن بوی گل و سنبل و نالیدن بلبل
&&
خوش بود دریغا که نکردند دوامی
\\
از من مطلب صبر جدایی که ندارم
&&
سنگیست فراق و دل محنت زده جامی
\\
در هیچ مقامی دل مسکین نشکیبد
&&
خو کرده صحبت که برافتد ز مقامی
\\
بی دوست حرام است جهان دیدن مشتاق
&&
قندیل بکش تا بنشینم به ظلامی
\\
چندان بنشینم که برآید نفس صبح
&&
کان وقت به دل می‌رسد از دوست پیامی
\\
آنجا که تویی رفتن ما سود ندارد
&&
الا به کرم پیش نهد لطف تو گامی
\\
زان عین که دیدی اثری بیش نمانده‌ست
&&
جانی به دهان آمده در حسرت کامی
\\
سعدی سخن یار نگوید بر اغیار
&&
هرگز نبرد سوخته‌ای قصه به خامی
\\
\end{longtable}
\end{center}
