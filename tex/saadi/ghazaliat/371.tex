\begin{center}
\section*{غزل ۳۷۱: من از آن روز که دربند توام آزادم}
\label{sec:371}
\addcontentsline{toc}{section}{\nameref{sec:371}}
\begin{longtable}{l p{0.5cm} r}
من از آن روز که در بند توام آزادم
&&
پادشاهم که به دست تو اسیر افتادم
\\
همه غم‌های جهان هیچ اثر می‌نکند
&&
در من از بس که به دیدار عزیزت شادم
\\
خرم آن روز که جان می‌رود اندر طلبت
&&
تا بیایند عزیزان به مبارک بادم
\\
من که در هیچ مقامی نزدم خیمه انس
&&
پیش تو رخت بیفکندم و دل بنهادم
\\
دانی از دولت وصلت چه طلب دارم هیچ
&&
یاد تو مصلحت خویش ببرد از یادم
\\
به وفای تو کز آن روز که دلبند منی
&&
دل نبستم به وفای کس و در نگشادم
\\
تا خیال قد و بالای تو در فکر من است
&&
گر خلایق همه سروند چو سرو آزادم
\\
به سخن راست نیاید که چه شیرین سخنی
&&
وین عجبتر که تو شیرینی و من فرهادم
\\
دستگاهی نه که در پای تو ریزم چون خاک
&&
حاصل آن است که چون طبل تهی پربادم
\\
می‌نماید که جفای فلک از دامن من
&&
دست کوته نکند تا نکند بنیادم
\\
ظاهر آن است که با سابقه حکم ازل
&&
جهد سودی نکند تن به قضا در دادم
\\
ور تحمل نکنم جور زمان را چه کنم
&&
داوری نیست که از وی بستاند دادم
\\
دلم از صحبت شیراز به کلی بگرفت
&&
وقت آن است که پرسی خبر از بغدادم
\\
هیچ شک نیست که فریاد من آنجا برسد
&&
عجب ار صاحب دیوان نرسد فریادم
\\
سعدیا حب وطن گر چه حدیثیست صحیح
&&
نتوان مرد به سختی که من این جا زادم
\\
\end{longtable}
\end{center}
