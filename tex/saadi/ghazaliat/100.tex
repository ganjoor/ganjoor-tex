\begin{center}
\section*{غزل ۱۰۰: این مطرب از کجاست که برگفت نام دوست}
\label{sec:100}
\addcontentsline{toc}{section}{\nameref{sec:100}}
\begin{longtable}{l p{0.5cm} r}
این مطرب از کجاست که برگفت نام دوست
&&
تا جان و جامه بذل کنم بر پیام دوست
\\
دل زنده می‌شود به امید وفای یار
&&
جان رقص می‌کند به سماع کلام دوست
\\
تا نفخ صور بازنیاید به خویشتن
&&
هرک اوفتاد مست محبت ز جام دوست
\\
من بعد از این اگر به دیاری سفر کنم
&&
هیچ ارمغانیی نبرم جز سلام دوست
\\
رنجور عشق به نشود جز به بوی یار
&&
ور رفتنیست جان ندهد جز به نام دوست
\\
وقتی امیر مملکت خویش بودمی
&&
اکنون به اختیار و ارادت غلام دوست
\\
گر دوست را به دیگری از من فراغتست
&&
من دیگری ندارم قایم مقام دوست
\\
بالای بام دوست چو نتوان نهاد پای
&&
هم چاره آن که سر بنهی زیر بام دوست
\\
درویش را که نام برد پیش پادشاه
&&
هیهات از افتقار من و احتشام دوست
\\
گر کام دوست کشتن سعدیست باک نیست
&&
اینم حیات بس که بمیرم به کام دوست
\\
\end{longtable}
\end{center}
