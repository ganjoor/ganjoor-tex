\begin{center}
\section*{غزل ۲۵۶: نفسی وقت بهارم هوس صحرا بود}
\label{sec:256}
\addcontentsline{toc}{section}{\nameref{sec:256}}
\begin{longtable}{l p{0.5cm} r}
نفسی وقت بهارم هوس صحرا بود
&&
با رفیقی دو که دایم نتوان تنها بود
\\
خاک شیراز چو دیبای منقش دیدم
&&
وان همه صورت شاهد که بر آن دیبا بود
\\
پارس در سایه اقبال اتابک ایمن
&&
لیکن از ناله مرغان چمن غوغا بود
\\
شکرین پسته دهانی به تفرج بگذشت
&&
که چه گویم نتوان گفت که چون زیبا بود
\\
یعلم الله که شقایق نه بدان لطف و سمن
&&
نه بدان بوی و صنوبر نه بدان بالا بود
\\
فتنه سامریش در نظر شورانگیز
&&
نفس عیسویش در لب شکرخا بود
\\
من در اندیشه که بت یا مه نو یا ملکست
&&
یار بت پیکر مه روی ملک سیما بود
\\
دل سعدی و جهانی به دمی غارت کرد
&&
همچو نوروز که بر خوان ملک یغما بود
\\
\end{longtable}
\end{center}
