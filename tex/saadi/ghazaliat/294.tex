\begin{center}
\section*{غزل ۲۹۴: آمد گه آن که بوی گلزار}
\label{sec:294}
\addcontentsline{toc}{section}{\nameref{sec:294}}
\begin{longtable}{l p{0.5cm} r}
آمد گه آن که بوی گلزار
&&
منسوخ کند گلاب عطار
\\
خواب از سر خفتگان به دربرد
&&
بیداری بلبلان اسحار
\\
ما کلبه زهد برگرفتیم
&&
سجاده که می‌برد به خمار
\\
یک رنگ شویم تا نباشد
&&
این خرقه سترپوش زنار
\\
برخیز که چشم‌های مستت
&&
خفتست و هزار فتنه بیدار
\\
وقتی صنمی دلی ربودی
&&
تو خلق ربوده‌ای به یک بار
\\
یا خاطر خویشتن به ما ده
&&
یا خاطر ما ز دست بگذار
\\
نه راه شدن نه روی بودن
&&
معشوقه ملول و ما گرفتار
\\
هم زخم تو به چو می‌خورم زخم
&&
هم بار تو به چو می‌کشم بار
\\
من پیش نهاده‌ام که در خون
&&
برگردم و برنگردم از یار
\\
گر دنیی و آخرت بیاری
&&
کاین هر دو بگیر و دوست بگذار
\\
ما یوسف خود نمی‌فروشیم
&&
تو سیم سیاه خود نگه دار
\\
\end{longtable}
\end{center}
