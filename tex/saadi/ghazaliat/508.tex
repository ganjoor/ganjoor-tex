\begin{center}
\section*{غزل ۵۰۸: مشتاق توام با همه جوری و جفایی}
\label{sec:508}
\addcontentsline{toc}{section}{\nameref{sec:508}}
\begin{longtable}{l p{0.5cm} r}
مشتاق توام با همه جوری و جفایی
&&
محبوب منی با همه جرمی و خطایی
\\
من خود به چه ارزم که تمنای تو ورزم
&&
در حضرت سلطان که برد نام گدایی
\\
صاحب نظران لاف محبت نپسندند
&&
وان گه سپر انداختن از تیر بلایی
\\
باید که سری در نظرش هیچ نیرزد
&&
آن کس که نهد در طلب وصل تو پایی
\\
بیداد تو عدلست و جفای تو کرامت
&&
دشنام تو خوشتر که ز بیگانه دعایی
\\
جز عهد و وفای تو که محلول نگردد
&&
هر عهد که بستم هوسی بود و هوایی
\\
گر دست دهد دولت آنم که سر خویش
&&
در پای سمند تو کنم نعل بهایی
\\
شاید که به خون بر سر خاکم بنویسند
&&
این بود که با دوست به سر برد وفایی
\\
خون در دل آزرده نهان چند بماند
&&
شک نیست که سر برکند این درد به جایی
\\
شرط کرم آنست که با درد بمیری
&&
سعدی و نخواهی ز در خلق دوایی
\\
\end{longtable}
\end{center}
