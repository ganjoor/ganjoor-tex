\begin{center}
\section*{غزل ۲۲۸: کاروان می‌رود و بار سفر می‌بندند}
\label{sec:228}
\addcontentsline{toc}{section}{\nameref{sec:228}}
\begin{longtable}{l p{0.5cm} r}
کاروان می‌رود و بار سفر می‌بندند
&&
تا دگربار که بیند که به ما پیوندند
\\
خیلتاشان جفاکار و محبان ملول
&&
خیمه را همچو دل از صحبت ما برکندند
\\
آن همه عشوه که در پیش نهادند و غرور
&&
عاقبت روز جدایی پس پشت افکندند
\\
طمع از دوست نه این بود و توقع نه چنین
&&
مکن ای دوست که از دوست جفا نپسندند
\\
ما همانیم که بودیم و محبت باقیست
&&
ترک صحبت نکند دل که به مهر آکندند
\\
عیب شیرین دهنان نیست که خون می‌ریزند
&&
جرم صاحب نظرانست که دل می‌بندند
\\
مرض عشق نه دردیست که می‌شاید گفت
&&
با طبیبان که در این باب نه دانشمندند
\\
ساربان رخت منه بر شتر و بار مبند
&&
که در این مرحله بیچاره اسیری چندند
\\
طبع خرسند نمی‌باشد و بس می‌نکند
&&
مهر آنان که به نادیدن ما خرسندند
\\
مجلس یاران بی ناله سعدی خوش نیست
&&
شمع می‌گرید و نظارگیان می‌خندند
\\
\end{longtable}
\end{center}
