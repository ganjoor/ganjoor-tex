\begin{center}
\section*{غزل ۲۶۸: ای ساربان آهسته رو کآرام جانم می‌رود}
\label{sec:268}
\addcontentsline{toc}{section}{\nameref{sec:268}}
\begin{longtable}{l p{0.5cm} r}
ای ساربان آهسته رو کآرام جانم می‌رود
&&
وآن دل که با خود داشتم با دلستانم می‌رود
\\
من مانده‌ام مهجور از او بیچاره و رنجور از او
&&
گویی که نیشی دور از او در استخوانم می‌رود
\\
گفتم به نیرنگ و فسون پنهان کنم ریش درون
&&
پنهان نمی‌ماند که خون بر آستانم می‌رود
\\
محمل بدار ای ساروان تندی مکن با کاروان
&&
کز عشق آن سرو روان گویی روانم می‌رود
\\
او می‌رود دامن کشان من زهر تنهایی چشان
&&
دیگر مپرس از من نشان کز دل نشانم می‌رود
\\
برگشت یار سرکشم بگذاشت عیش ناخوشم
&&
چون مجمری پرآتشم کز سر دخانم می‌رود
\\
با آن همه بیداد او وین عهد بی‌بنیاد او
&&
در سینه دارم یاد او یا بر زبانم می‌رود
\\
بازآی و بر چشمم نشین ای دلستان نازنین
&&
کآشوب و فریاد از زمین بر آسمانم می‌رود
\\
شب تا سحر می‌نغنوم و اندرز کس می‌نشنوم
&&
وین ره نه قاصد می‌روم کز کف عنانم می‌رود
\\
گفتم بگریم تا ابل چون خر فروماند به گل
&&
وین نیز نتوانم که دل با کاروانم می‌رود
\\
صبر از وصال یار من برگشتن از دلدار من
&&
گر چه نباشد کار من هم کار از آنم می‌رود
\\
در رفتن جان از بدن گویند هر نوعی سخن
&&
من خود به چشم خویشتن دیدم که جانم می‌رود
\\
سعدی فغان از دست ما لایق نبود ای بی‌وفا
&&
طاقت نمی‌آرم جفا کار از فغانم می‌رود
\\
\end{longtable}
\end{center}
