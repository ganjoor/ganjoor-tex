\begin{center}
\section*{غزل ۲۶۰: من چه در پای تو ریزم که خورای تو بود}
\label{sec:260}
\addcontentsline{toc}{section}{\nameref{sec:260}}
\begin{longtable}{l p{0.5cm} r}
من چه در پای تو ریزم که خورای تو بود
&&
سر نه چیزیست که شایسته پای تو بود
\\
خرم آن روی که در روی تو باشد همه عمر
&&
وین نباشد مگر آن وقت که رای تو بود
\\
ذره‌ای در همه اجزای من مسکین نیست
&&
که نه آن ذره معلق به هوای تو بود
\\
تا تو را جای شد ای سرو روان در دل من
&&
هیچ کس می‌نپسندم که به جای تو بود
\\
به وفای تو که گر خشت زنند از گل من
&&
همچنان در دل من مهر و وفای تو بود
\\
غایت آنست که ما در سر کار تو رویم
&&
مرگ ما باک نباشد چو بقای تو بود
\\
من پروانه صفت پیش تو ای شمع چگل
&&
گر بسوزم گنه من نه خطای تو بود
\\
عجبست آن که تو را دید و حدیث تو شنید
&&
که همه عمر نه مشتاق لقای تو بود
\\
خوش بود ناله دلسوختگان از سر درد
&&
خاصه دردی که به امید دوای تو بود
\\
ملک دنیا همه با همت سعدی هیچست
&&
پادشاهیش همین بس که گدای تو بود
\\
\end{longtable}
\end{center}
