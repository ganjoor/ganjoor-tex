\begin{center}
\section*{غزل ۴۱۶: بس که در منظر تو حیرانم}
\label{sec:416}
\addcontentsline{toc}{section}{\nameref{sec:416}}
\begin{longtable}{l p{0.5cm} r}
بس که در منظر تو حیرانم
&&
صورتت را صفت نمی‌دانم
\\
پارسایان ملامتم مکنید
&&
که من از عشق توبه نتوانم
\\
هر که بینی به جسم و جان زنده‌ست
&&
من به امید وصل جانانم
\\
به چه کار آید این بقیت جان
&&
که به معشوق برنیفشانم
\\
گر تو از من عنان بگردانی
&&
من به شمشیر برنگردانم
\\
گر بخوانی مقیم درگاهم
&&
ور برانی مطیع فرمانم
\\
من نه آنم که سست بازآیم
&&
ور ز سختی به لب رسد جانم
\\
گر اجابت کنی و گر نکنی
&&
چاره من دعاست می‌خوانم
\\
سهل باشد صعوبت ظلمات
&&
گر به دست آید آب حیوانم
\\
تا کی آخر جفا بری سعدی
&&
چه کنم پای بند احسانم
\\
کار مردان تحمل است و سکون
&&
من کیم خاک پای مردانم
\\
\end{longtable}
\end{center}
