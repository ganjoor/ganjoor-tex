\begin{center}
\section*{غزل ۱۵۹: نه آن شبست که کس در میان ما گنجد}
\label{sec:159}
\addcontentsline{toc}{section}{\nameref{sec:159}}
\begin{longtable}{l p{0.5cm} r}
نه آن شبست که کس در میان ما گنجد
&&
به خاک پایت اگر ذره در هوا گنجد
\\
کلاه ناز و تکبر بنه کمر بگشای
&&
که چون تو سرو ندیدم که در قبا گنجد
\\
ز من حکایت هجران مپرس در شب وصل
&&
عتاب کیست که در خلوت رضا گنجد
\\
مرا شکر منه و گل مریز در مجلس
&&
میان خسرو و شیرین شکر کجا گنجد
\\
چو شور عشق درآمد قرار عقل نماند
&&
درون مملکتی چون دو پادشا گنجد
\\
نماند در سر سعدی ز بانگ رود و سرود
&&
مجال آن که دگر پند پارسا گنجد
\\
\end{longtable}
\end{center}
