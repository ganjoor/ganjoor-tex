\begin{center}
\section*{غزل ۱۰۵: آب حیات منست خاک سر کوی دوست}
\label{sec:105}
\addcontentsline{toc}{section}{\nameref{sec:105}}
\begin{longtable}{l p{0.5cm} r}
آب حیات من است خاک سر کوی دوست
&&
گر دو جهان خرمیست ما و غم روی دوست
\\
ولوله در شهر نیست جز شکن زلف یار
&&
فتنه در آفاق نیست جز خم ابروی دوست
\\
داروی مشتاق چیست زهر ز دست نگار
&&
مرهم عشاق چیست زخم ز بازوی دوست
\\
دوست به هندوی خود گر بپذیرد مرا
&&
گوش من و تا به حشر حلقه هندوی دوست
\\
گر متفرق شود خاک من اندر جهان
&&
باد نیارد ربود گرد من از کوی دوست
\\
گر شب هجران مرا تاختن آرد اجل
&&
روز قیامت زنم خیمه به پهلوی دوست
\\
هر غزلم نامه‌ایست صورت حالی در او
&&
نامه نوشتن چه سود چون نرسد سوی دوست
\\
لاف مزن سعدیا شعر تو خود سحر گیر
&&
سحر نخواهد خرید غمزه جادوی دوست
\\
\end{longtable}
\end{center}
