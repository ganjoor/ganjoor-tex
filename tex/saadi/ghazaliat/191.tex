\begin{center}
\section*{غزل ۱۹۱: کی برست این گل خندان و چنین زیبا شد}
\label{sec:191}
\addcontentsline{toc}{section}{\nameref{sec:191}}
\begin{longtable}{l p{0.5cm} r}
کی برست این گل خندان و چنین زیبا شد
&&
آخر این غوره نوخاسته چون حلوا شد
\\
دیگر این مرغ کی از بیضه برآمد که چنین
&&
بلبل خوش سخن و طوطی شکرخا شد
\\
که درآموختش این لطف و بلاغت کان روز
&&
مردم از عقل به دربرد که او دانا شد
\\
شاخکی تازه برآورد صبا بر لب جوی
&&
چشم بر هم نزدی سرو سهی بالا شد
\\
عالم طفلی و جهل حیوانی بگذاشت
&&
آدمی طبع و ملک خوی و پری سیما شد
\\
عقل را گفتم از این پس به سلامت بنشین
&&
گفت خاموش که این فتنه دگر پیدا شد
\\
پر نشد چون صدف از لؤلؤ لالا دهنی
&&
که نه از حسرت او دیده ما دریا شد
\\
سعدیا غنچه سیراب نگنجد در پوست
&&
وقت خوش دید و بخندید و گلی رعنا شد
\\
\end{longtable}
\end{center}
