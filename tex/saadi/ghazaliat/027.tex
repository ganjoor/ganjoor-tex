\begin{center}
\section*{غزل ۲۷: ماه رویا روی خوب از من متاب}
\label{sec:027}
\addcontentsline{toc}{section}{\nameref{sec:027}}
\begin{longtable}{l p{0.5cm} r}
ماهرویا! روی خوب از من متاب
&&
بی خطا کشتن چه می‌بینی صواب
\\
دوش در خوابم در آغوش آمدی
&&
وین نپندارم که بینم جز به خواب
\\
از درون سوزناک و چشم تر
&&
نیمه‌ای در آتشم نیمی در آب
\\
هر که بازآید ز در پندارم اوست
&&
تشنه مسکین آب پندارد سراب
\\
ناوکش را جان درویشان هدف
&&
ناخنش را خون مسکینان خضاب
\\
او سخن می‌گوید و دل می‌برد
&&
و او نمک می‌ریزد و مردم کباب
\\
حیف باشد بر چنان تن پیرهن
&&
ظلم باشد بر چنان صورت نقاب
\\
خوی به دامان از بناگوشش بگیر
&&
تا بگیرد جامه‌ات بوی گلاب
\\
فتنه باشد شاهدی شمعی به دست
&&
سرگران از خواب و سرمست از شراب
\\
بامدادی تا به شب رویت مپوش
&&
تا بپوشانی جمال آفتاب
\\
سعدیا گر در برش خواهی چو چنگ
&&
گوشمالت خورد باید چون رباب
\\
\end{longtable}
\end{center}
