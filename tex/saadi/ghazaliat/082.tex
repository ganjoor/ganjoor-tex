\begin{center}
\section*{غزل ۸۲: هزار سختی اگر بر من آید آسانست}
\label{sec:082}
\addcontentsline{toc}{section}{\nameref{sec:082}}
\begin{longtable}{l p{0.5cm} r}
هزار سختی اگر بر من آید آسان است
&&
که دوستی و ارادت هزار چندان است
\\
سفر دراز نباشد به پای طالب دوست
&&
که خار دشت محبت گل است و ریحان است
\\
اگر تو جور کنی جور نیست تربیتست
&&
و گر تو داغ نهی داغ نیست درمان است
\\
نه آبروی که گر خون دل بخواهی ریخت
&&
مخالفت نکنم آن کنم که فرمان است
\\
ز عقل من عجب آید صوابگویان را
&&
که دل به دست تو دادن خلاف در جان است
\\
من از کنار تو دور اوفتاده‌ام نه عجب
&&
گرم قرار نباشد که داغ هجران است
\\
عجب در آن سر زلف معنبر مفتول
&&
که در کنار تو خسبد چرا پریشان است
\\
جماعتی که ندانند حظ روحانی
&&
تفاوتی که میان دواب و انسان است
\\
گمان برند که در باغ عشق سعدی را
&&
نظر به سیب زنخدان و نار پستان است
\\
مرا هرآینه خاموش بودن اولی‌تر
&&
که جهل پیش خردمند عذر نادان است
\\
و ما ابرئ نفسی و لا ازکیها
&&
که هر چه نقل کنند از بشر در امکان است
\\
\end{longtable}
\end{center}
