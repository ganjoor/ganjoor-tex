\begin{center}
\section*{غزل ۵۷۱: تو اگر به حسن دعوی بکنی گواه داری}
\label{sec:571}
\addcontentsline{toc}{section}{\nameref{sec:571}}
\begin{longtable}{l p{0.5cm} r}
تو اگر به حسن دعوی بکنی گواه داری
&&
که جمال سرو بستان و کمال ماه داری
\\
در کس نمی‌گشایم که به خاطرم درآید
&&
تو به اندرون جان آی که جایگاه داری
\\
ملکی مهی ندانم به چه کنیتت بخوانم
&&
به کدام جنس گویم که تو اشتباه داری
\\
بر کس نمی‌توانم به شکایت از تو رفتن
&&
که قبول و قوتت هست و جمال و جاه داری
\\
گل بوستان رویت چو شقایق است لیکن
&&
چه کنم به سرخ رویی که دلی سیاه داری
\\
چه خطای بنده دیدی که خلاف عهد کردی
&&
مگر آن که ما ضعیفیم و تو دستگاه داری
\\
نه کمال حسن باشد ترشی و روی شیرین
&&
همه بد مکن که مردم همه نیکخواه داری
\\
تو جفا کنی و صولت دگران دعای دولت
&&
چه کنند از این لطافت که تو پادشاه داری
\\
به یکی لطیفه گفتی ببرم هزار دل را
&&
نه چنان لطیف باشد که دلی نگاه داری
\\
به خدای اگر چو سعدی برود دلت به راهی
&&
همه شب چنو نخسبی و نظر به راه داری
\\
\end{longtable}
\end{center}
