\begin{center}
\section*{غزل ۲۵۵: تو را سماع نباشد که سوز عشق نبود}
\label{sec:255}
\addcontentsline{toc}{section}{\nameref{sec:255}}
\begin{longtable}{l p{0.5cm} r}
تو را سماع نباشد که سوز عشق نبود
&&
گمان مبر که برآید ز خام هرگز دود
\\
چو هر چه می‌رسد از دست اوست فرقی نیست
&&
میان شربت نوشین و تیغ زهرآلود
\\
نسیم باد صبا بوی یار من دارد
&&
چو باد خواهم از این پس به بوی او پیمود
\\
همی‌گذشت و نظر کردمش به گوشه چشم
&&
که یک نظر بربایم مرا ز من بربود
\\
به صبر خواستم احوال عشق پوشیدن
&&
دگر به گل نتوانستم آفتاب اندود
\\
سوار عقل که باشد که پشت ننماید
&&
در آن مقام که سلطان عشق روی نمود
\\
پیام ما که رساند به خدمتش که رضا
&&
رضای توست گرم خسته داری ار خشنود
\\
شبی نرفت که سعدی به داغ عشق نگفت
&&
دگر شب آمد و کی بی تو روز خواهد بود
\\
\end{longtable}
\end{center}
