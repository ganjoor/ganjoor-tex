\begin{center}
\section*{غزل ۵۳۲: چون خراباتی نباشد زاهدی}
\label{sec:532}
\addcontentsline{toc}{section}{\nameref{sec:532}}
\begin{longtable}{l p{0.5cm} r}
چون خراباتی نباشد زاهدی
&&
کش به شب از در درآید شاهدی
\\
محتسب گو تا ببیند روی دوست
&&
همچو محرابی و من چون عابدی
\\
چون من آب زندگانی یافتم
&&
غم نباشد گر بمیرد حاسدی
\\
آنچه ما را در دل است از سوز عشق
&&
می‌نشاید گفت با هر باردی
\\
دوستان گیرند و دلداران ولیک
&&
مهربان نشناسد الا واحدی
\\
از تو روحانی‌ترم در پیش دل
&&
نگذرد شب‌های خلوت واردی
\\
خانه‌ای در کوی درویشان بگیر
&&
تا نماند در محلت زاهدی
\\
گر دلی داری و دلبندیت نیست
&&
پس چه فرق از ناطقی تا جامدی
\\
گر به خدمت قائمی خواهی منم
&&
ور نمی‌خواهی به حسرت قاعدی
\\
سعدیا گر روزگارت می‌کشد
&&
گو بکش بر دست سیمین ساعدی
\\
\end{longtable}
\end{center}
