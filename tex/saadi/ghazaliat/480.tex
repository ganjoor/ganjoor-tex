\begin{center}
\section*{غزل ۴۸۰: گفتم به عقل پای برآرم ز بند او}
\label{sec:480}
\addcontentsline{toc}{section}{\nameref{sec:480}}
\begin{longtable}{l p{0.5cm} r}
گفتم به عقل پای برآرم ز بند او
&&
روی خلاص نیست به جهد از کمند او
\\
مستوجب ملامتی ای دل که چند بار
&&
عقلت بگفت و گوش نکردی به پند او
\\
آن بوستان میوه شیرین که دست جهد
&&
دشوار می‌رسد به درخت بلند او
\\
گفتم عنان مرکب تازی بگیرمش
&&
لیکن وصول نیست به گرد سمند او
\\
سر در جهان نهادمی از دست او ولیک
&&
از شهر او چگونه رود شهربند او
\\
چشمم بدوخت از همه عالم به اتفاق
&&
تا جز در او نظر نکند مستمند او
\\
گر خود به جای مروحه شمشیر می‌زند
&&
مسکین مگس کجا رود از پیش قند او
\\
نومید نیستم که هم او مرهمی نهد
&&
ور نه به هیچ به نشود دردمند او
\\
او خود مگر به لطف خداوندیی کند
&&
ور نه ز ما چه بندگی آید پسند او
\\
سعدی چو صبر از اوت میسر نمی‌شود
&&
اولیتر آن که صبر کنی بر گزند او
\\
\end{longtable}
\end{center}
