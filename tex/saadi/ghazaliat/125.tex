\begin{center}
\section*{غزل ۱۲۵: کس ندانم که در این شهر گرفتار تو نیست}
\label{sec:125}
\addcontentsline{toc}{section}{\nameref{sec:125}}
\begin{longtable}{l p{0.5cm} r}
کس ندانم که در این شهر گرفتار تو نیست
&&
هیچ بازار چنین گرم که بازار تو نیست
\\
سرو زیبا و به زیبایی بالای تو نه
&&
شهد شیرین و به شیرینی گفتار تو نیست
\\
خود که باشد که تو را بیند و عاشق نشود
&&
مگرش هیچ نباشد که خریدار تو نیست
\\
کس ندیدست تو را یک نظر اندر همه عمر
&&
که همه عمر دعاگوی و هوادار تو نیست
\\
آدمی نیست مگر کالبدی بی‌جانست
&&
آن که گوید که مرا میل به دیدار تو نیست
\\
ای که شمشیر جفا بر سر ما آخته‌ای
&&
صلح کردیم که ما را سر پیکار تو نیست
\\
جور تلخست ولیکن چه کنم گر نبرم
&&
چون گریز از لب شیرین شکربار تو نیست
\\
من سری دارم و در پای تو خواهم بازید
&&
خجل از ننگ بضاعت که سزاوار تو نیست
\\
به جمال تو که دیدار ز من بازمگیر
&&
که مرا طاقت نادیدن دیدار تو نیست
\\
سعدیا گر نتوانی که کم خود گیری
&&
سر خود گیر که صاحب نظری کار تو نیست
\\
\end{longtable}
\end{center}
