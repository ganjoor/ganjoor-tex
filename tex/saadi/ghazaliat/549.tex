\begin{center}
\section*{غزل ۵۴۹: دانمت آستین چرا پیش جمال می‌بری}
\label{sec:549}
\addcontentsline{toc}{section}{\nameref{sec:549}}
\begin{longtable}{l p{0.5cm} r}
دانمت آستین چرا پیش جمال می‌بری
&&
رسم بود کز آدمی روی نهان کند پری
\\
معتقدان و دوستان از چپ و راست منتظر
&&
کبر رها نمی‌کند کز پس و پیش بنگری
\\
آمدمت که بنگرم باز نظر به خود کنم
&&
سیر نمی‌شود نظر بس که لطیف منظری
\\
غایت کام و دولت است آن که به خدمتت رسید
&&
بنده میان بندگان بسته میان به چاکری
\\
روی به خاک می‌نهم گر تو هلاک می‌کنی
&&
دست به بند می‌دهم گر تو اسیر می‌بری
\\
هر چه کنی تو برحقی حاکم و دست مطلقی
&&
پیش که داوری برند از تو که خصم و داوری
\\
بنده اگر به سر رود در طلبت کجا رسد
&&
گر نرسد عنایتی در حق بنده آن سری
\\
گفتم اگر نبینمت مهر فرامشم شود
&&
می‌روی و مقابلی غایب و در تصوری
\\
جان بدهند و در زمان زنده شوند عاشقان
&&
گر بکشی و بعد از آن بر سر کشته بگذری
\\
سعدی اگر هلاک شد عمر تو باد و دوستان
&&
ملک یمین خویش را گر بکشی چه غم خوری
\\
\end{longtable}
\end{center}
