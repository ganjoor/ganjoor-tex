\begin{center}
\section*{غزل ۳۵۱: نشسته بودم و خاطر به خویشتن مشغول}
\label{sec:351}
\addcontentsline{toc}{section}{\nameref{sec:351}}
\begin{longtable}{l p{0.5cm} r}
نشسته بودم و خاطر به خویشتن مشغول
&&
در سرای به هم کرده از خروج و دخول
\\
شب دراز دو چشمم بر آستان امید
&&
که بامداد در حجره می‌زند مأمول
\\
خمار در سر و دستش به خون هشیاران
&&
خضیب و نرگس مستش به جادویی مکحول
\\
بیار ساقی و همسایه گو دو چشم ببند
&&
که من دو گوش بیاکندم از حدیث عذول
\\
چنان تصور معشوق در خیال من است
&&
که دیگرم متصور نمی‌شود معقول
\\
حدیث عقل در ایام پادشاهی عشق
&&
چنان شده‌ست که فرمان عامل معزول
\\
شکایت از تو ندارم که شکر باید کرد
&&
گرفته خانه درویش پادشه به نزول
\\
بر آن سماط که منظور میزبان باشد
&&
شکم پرست کند التفات بر مأکول
\\
به دوستی که ز دست تو ضربت شمشیر
&&
چنان موافق طبع آیدم که ضرب اصول
\\
مرا به عاشقی و دوست را به معشوقی
&&
چه نسبت است بگویید قاتل و مقتول
\\
مرا به گوش تو باید حکایت از لب خویش
&&
دریغ باشد پیغام ما به دست رسول
\\
درون خاطر سعدی مجال غیر تو نیست
&&
چو خوش بود به تو از هر که در جهان مشغول
\\
\end{longtable}
\end{center}
