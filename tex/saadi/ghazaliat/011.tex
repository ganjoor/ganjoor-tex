\begin{center}
\section*{غزل ۱۱: وقت طرب خوش یافتم آن دلبر طناز را}
\label{sec:011}
\addcontentsline{toc}{section}{\nameref{sec:011}}
\begin{longtable}{l p{0.5cm} r}
وقت طرب خوش یافتم آن دلبر طناز را
&&
ساقی بیار آن جام می مطرب بزن آن ساز را
\\
امشب که بزم عارفان از شمع رویت روشن است
&&
آهسته تا نبود خبر رندان شاهدباز را
\\
دوش ای پسر می خورده‌ای چشمت گواهی می‌دهد
&&
باری حریفی جو که او مستور دارد راز را
\\
روی خوش و آواز خوش دارند هر یک لذتی
&&
بنگر که لذت چون بود محبوب خوش آواز را
\\
چشمان ترک و ابروان جان را به ناوک می‌زنند
&&
یا رب که دادست این کمان آن ترک تیرانداز را
\\
شور غم عشقش چنین حیف است پنهان داشتن
&&
در گوش نی رمزی بگو تا برکشد آواز را
\\
شیراز پرغوغا شدست از فتنه چشم خوشت
&&
ترسم که آشوب خوشت برهم زند شیراز را
\\
من مرغکی پربسته‌ام زان در قفس بنشسته‌ام
&&
گر زان که بشکستی قفس بنمودمی پرواز را
\\
سعدی تو مرغ زیرکی خوبت به دام آورده‌ام
&&
مشکل به دست آرد کسی مانند تو شهباز را
\\
\end{longtable}
\end{center}
