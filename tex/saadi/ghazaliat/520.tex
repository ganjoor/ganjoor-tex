\begin{center}
\section*{غزل ۵۲۰: که دست تشنه می‌گیرد به آبی}
\label{sec:520}
\addcontentsline{toc}{section}{\nameref{sec:520}}
\begin{longtable}{l p{0.5cm} r}
که دست تشنه می‌گیرد به آبی
&&
خداوندان فضل آخر ثوابی
\\
توقع دارم از شیرین زبانت
&&
اگر تلخ است و گر شیرین جوابی
\\
تو خود نایی و گر آیی بر من
&&
بدان ماند که گنجی در خرابی
\\
به چشمانت که گر زهرم فرستی
&&
چنان نوشم که شیرینتر شرابی
\\
اگر سروی به بالای تو باشد
&&
نباشد بر سر سرو آفتابی
\\
پری روی از نظر غایب نگردد
&&
اگر صد بار بربندد نقابی
\\
بدان تا یک نفس رویت ببینم
&&
شب و روز آرزومندم به خوابی
\\
امیدم هست اگر عطشان نمیرد
&&
که باز آید به جوی رفته آبی
\\
هلاک خویشتن می‌خواهد آن مور
&&
که خواهد پنجه کردن با عقابی
\\
شبی دانم که در زندان هجران
&&
سحرگاهم به گوش آید خطابی
\\
که سعدی چون فراق ما کشیدی
&&
نخواهی دید در دوزخ عذابی
\\
\end{longtable}
\end{center}
