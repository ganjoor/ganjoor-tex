\begin{center}
\section*{غزل ۲۰۶: در پای تو افتادن شایسته دمی باشد}
\label{sec:206}
\addcontentsline{toc}{section}{\nameref{sec:206}}
\begin{longtable}{l p{0.5cm} r}
در پای تو افتادن شایسته دمی باشد
&&
ترک سر خود گفتن زیبا قدمی باشد
\\
بسیار زبونی‌ها بر خویش روا دارد
&&
درویش که بازارش با محتشمی باشد
\\
زین سان که وجود توست ای صورت روحانی
&&
شاید که وجود ما پیشت عدمی باشد
\\
گر جمله صنم‌ها را صورت به تو مانستی
&&
شاید که مسلمان را قبله صنمی باشد
\\
با آن که اسیران را کشتی و خطا کردی
&&
بر کشته گذر کردن نوع کرمی باشد
\\
رقص از سر ما بیرون امروز نخواهد شد
&&
کاین مطرب ما یک دم خاموش نمی‌باشد
\\
هر کو به همه عمرش سودای گلی بودست
&&
داند که چرا بلبل دیوانه همی‌باشد
\\
کس بر الم ریشت واقف نشود سعدی
&&
الا به کسی گویی کاو را المی باشد
\\
\end{longtable}
\end{center}
