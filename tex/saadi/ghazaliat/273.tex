\begin{center}
\section*{غزل ۲۷۳: هفته‌ای می‌رود از عمر و به ده روز کشید}
\label{sec:273}
\addcontentsline{toc}{section}{\nameref{sec:273}}
\begin{longtable}{l p{0.5cm} r}
هفته‌ای می‌رود از عمر و به ده روز کشید
&&
کز گلستان صفا بوی وفایی ندمید
\\
آن که برگشت و جفا کرد به هیچم بفروخت
&&
به همه عالمش از من نتوانند خرید
\\
هر چه زان تلخ‌تر اندر همه عالم نبود
&&
گو بگو از لب شیرین که لطیف است و لذیذ
\\
گر من از خار بترسم نبرم دامن گل
&&
کام در کام نهنگ است بباید طلبید
\\
مرو ای دوست که ما بی تو نخواهیم نشست
&&
مبر ای یار که ما از تو نخواهیم برید
\\
از تو با مصلحت خویش نمی‌پردازم
&&
که محال است که در خود نگرد هر که تو دید
\\
آفرین کردن و دشنام شنیدن سهل است
&&
چه از آن به که بود با تو مرا گفت و شنید
\\
جهد بسیار بکردم که نگویم غم دل
&&
عاقبت جان به دهان آمد و طاقت برسید
\\
آخر ای مطرب از این پرده عشاق بگرد
&&
چند گویی که مرا پرده به چنگ تو درید
\\
تشنگانت به لب ای چشمه حیوان مردند
&&
چند چون ماهی بر خشک توانند طپید
\\
سخن سعدی بشنو که تو خود زیبایی
&&
خاصه آن وقت که در گوش کنی مروارید
\\
\end{longtable}
\end{center}
