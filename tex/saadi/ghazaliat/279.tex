\begin{center}
\section*{غزل ۲۷۹: فراق را دلی از سنگ سختتر باید}
\label{sec:279}
\addcontentsline{toc}{section}{\nameref{sec:279}}
\begin{longtable}{l p{0.5cm} r}
فراق را دلی از سنگ سختتر باید
&&
مرا دلیست که با شوق بر نمی‌آید
\\
هنوز با همه بدعهدیت دعاگویم
&&
بیا و گر همه دشنام می‌دهی شاید
\\
اگر چه هر چه جهانت به دل خریدارند
&&
منت به جان بخرم تا کسی نیفزاید
\\
بکش چنان که توانی که بنده را نرسد
&&
خلاف آن چه خداوندگار فرماید
\\
نه زنده را به تو میلست و مهربانی و بس
&&
که مرده را به نسیمت روان بیاساید
\\
مپرس کشته شمشیر عشق را چونی
&&
چنان که هر که ببیند بر او ببخشاید
\\
پدر که چون تو جگرگوشه از خدا می‌خواست
&&
خبر نداشت که دیگر چه فتنه می‌زاید
\\
توانگرا در رحمت به روی درویشان
&&
مبند و گر تو ببندی خدای بگشاید
\\
به خون سعدی اگر تشنه‌ای حلالت باد
&&
تو دیر زی که مرا عمر خود نمی‌پاید
\\
\end{longtable}
\end{center}
