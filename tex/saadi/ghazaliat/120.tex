\begin{center}
\section*{غزل ۱۲۰: خبرت هست که بی روی تو آرامم نیست}
\label{sec:120}
\addcontentsline{toc}{section}{\nameref{sec:120}}
\begin{longtable}{l p{0.5cm} r}
خبرت هست که بی روی تو آرامم نیست
&&
طاقت بار فراق این همه ایامم نیست
\\
خالی از ذکر تو عضوی چه حکایت باشد
&&
سر مویی به غلط در همه اندامم نیست
\\
میل آن دانه خالم نظری بیش نبود
&&
چون بدیدم ره بیرون شدن از دامم نیست
\\
شب بر آنم که مگر روز نخواهد بودن
&&
بامدادت که نبینم طمع شامم نیست
\\
چشم از آن روز که برکردم و رویت دیدم
&&
به همین دیده سر دیدن اقوامم نیست
\\
نازنینا مکن آن جور که کافر نکند
&&
ور جهودی بکنم بهره در اسلامم نیست
\\
گو همه شهر به جنگم به درآیند و خلاف
&&
من که در خلوت خاصم خبر از عامم نیست
\\
نه به زرق آمده‌ام تا به ملامت بروم
&&
بندگی لازم اگر عزت و اکرامم نیست
\\
به خدا و به سراپای تو کز دوستیت
&&
خبر از دشمن و اندیشه ز دشنامم نیست
\\
دوستت دارم اگر لطف کنی ور نکنی
&&
به دو چشم تو که چشم از تو به انعامم نیست
\\
سعدیا نامتناسب حیوانی باشد
&&
هر که گوید که دلم هست و دلارامم نیست
\\
\end{longtable}
\end{center}
