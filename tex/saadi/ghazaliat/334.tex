\begin{center}
\section*{غزل ۳۳۴: قیامت باشد آن قامت در آغوش}
\label{sec:334}
\addcontentsline{toc}{section}{\nameref{sec:334}}
\begin{longtable}{l p{0.5cm} r}
قیامت باشد آن قامت در آغوش
&&
شراب سلسبیل از چشمه نوش
\\
غلام کیست آن لعبت که ما را
&&
غلام خویش کرد و حلقه در گوش
\\
پری پیکر بتی کز سحر چشمش
&&
نیامد خواب در چشمان من دوش
\\
نه هر وقتم به یاد خاطر آید
&&
که خود هرگز نمی‌گردد فراموش
\\
حلالش باد اگر خونم بریزد
&&
که سر در پای او خوشتر که بر دوش
\\
نصیحتگوی ما عقلی ندارد
&&
برو گو در صلاح خویشتن کوش
\\
دهل زیر گلیم از خلق پنهان
&&
نشاید کرد و آتش زیر سرپوش
\\
بیا ای دوست ور دشمن ببیند
&&
چه خواهد کرد گو می‌بین و می‌جوش
\\
تو از ما فارغ و ما با تو همراه
&&
ز ما فریاد می‌آید تو خاموش
\\
حدیث حسن خویش از دیگری پرس
&&
که سعدی در تو حیران است و مدهوش
\\
\end{longtable}
\end{center}
