\begin{center}
\section*{غزل ۵۸۲: یار گرفته‌ام بسی چون تو ندیده‌ام کسی}
\label{sec:582}
\addcontentsline{toc}{section}{\nameref{sec:582}}
\begin{longtable}{l p{0.5cm} r}
یار گرفته‌ام بسی چون تو ندیده‌ام کسی
&&
شمع چنین نیامده‌ست از در هیچ مجلسی
\\
عادت بخت من نبود آن که تو یادم آوری
&&
نقد چنین کم اوفتد خاصه به دست مفلسی
\\
صحبت از این شریفتر صورت از این لطیفتر
&&
دامن از این نظیفتر وصف تو چون کند کسی
\\
خادمه سرای را گو در حجره بند کن
&&
تا به سر حضور ما ره نبرد موسوسی
\\
روز وصال دوستان دل نرود به بوستان
&&
یا به گلی نگه کند یا به جمال نرگسی
\\
گر بکشی کجا روم تن به قضا نهاده‌ام
&&
سنگ جفای دوستان درد نمی‌کند بسی
\\
قصه به هر که می‌برم فایده‌ای نمی‌دهد
&&
مشکل درد عشق را حل نکند مهندسی
\\
این همه خار می‌خورد سعدی و بار می‌برد
&&
جای دگر نمی‌رود هر که گرفت مونسی
\\
\end{longtable}
\end{center}
