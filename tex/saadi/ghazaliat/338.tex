\begin{center}
\section*{غزل ۳۳۸: دلی که دید که غایب شدست از این درویش}
\label{sec:338}
\addcontentsline{toc}{section}{\nameref{sec:338}}
\begin{longtable}{l p{0.5cm} r}
دلی که دید که غایب شده‌ست از این درویش
&&
گرفته از سر مستی و عاشقی سر خویش
\\
به دست آن که فتاده‌ست اگر مسلمان است
&&
مگر حلال ندارد مظالم درویش
\\
دل شکسته مروت بود که بازدهند
&&
که باز می‌دهد این دردمند را دل ریش
\\
مه دوهفته اسیرش گرفت و بند نهاد
&&
دو هفته رفت که از وی خبر نیامد بیش
\\
رمیده‌ای که نه از خویشتن خبر دارد
&&
نه از ملامت بیگانه و نصیحت خویش
\\
به شادکامی دشمن کسی سزاوار است
&&
که نشنود سخن دوستان نیک اندیش
\\
کنون به سختی و آسانیش بباید ساخت
&&
که در طبیعت زنبور نوش باشد و نیش
\\
دگر به یار جفاکار دل منه سعدی
&&
نمی‌دهیم و به شوخی همی‌برند از پیش
\\
\end{longtable}
\end{center}
