\begin{center}
\section*{غزل ۴۲: نشاید گفتن آن کس را دلی هست}
\label{sec:042}
\addcontentsline{toc}{section}{\nameref{sec:042}}
\begin{longtable}{l p{0.5cm} r}
نشاید گفتن آن کس را دلی هست
&&
که ندهد بر چنین صورت دل از دست
\\
نه منظوری که با او می‌توان گفت
&&
نه خصمی کز کمندش می‌توان رست
\\
به دل گفتم ز چشمانش بپرهیز
&&
که هشیاران نیاویزند با مست
\\
سرانگشتان مخضوبش نبینی
&&
که دست صبر برپیچید و بشکست
\\
نه آزاد از سرش بر می‌توان خاست
&&
نه با او می‌توان آسوده بنشست
\\
اگر دودی رود بی آتشی نیست
&&
و گر خونی بیاید کشته‌ای هست
\\
خیالش در نظر، چون آیدم خواب؟
&&
نشاید در به روی دوستان بست
\\
نشاید خرمن بیچارگان سوخت
&&
نمی‌باید دل درمندگان خست
\\
به آخر دوستی نتوان بریدن
&&
به اول خود نمی‌بایست پیوست
\\
دلی از دست بیرون رفته سعدی
&&
نیاید باز تیر رفته از شست
\\
\end{longtable}
\end{center}
