\begin{center}
\section*{غزل ۱۳۶: ای دیدنت آسایش و خندیدنت آفت}
\label{sec:136}
\addcontentsline{toc}{section}{\nameref{sec:136}}
\begin{longtable}{l p{0.5cm} r}
ای دیدنت آسایش و خندیدنت آفت
&&
گوی از همه خوبان بربودی به لطافت
\\
ای صورت دیبای خطایی به نکویی
&&
وی قطره باران بهاری به نظافت
\\
هر ملک وجودی که به شوخی بگرفتی
&&
سلطان خیالت بنشاندی به خلافت
\\
ای سرو خرامان گذری از در رحمت
&&
وی ماه درفشان نظری از سر رأفت
\\
گویند برو تا برود صحبتت از دل
&&
ترسم هوسم بیش کند بعد مسافت
\\
ای عقل نگفتم که تو در عشق نگنجی
&&
در دولت خاقان نتوان کرد خلافت
\\
با قد تو زیبا نبود سرو به نسبت
&&
با روی تو نیکو نبود مه به اضافت
\\
آن را که دلارام دهد وعده کشتن
&&
باید که ز مرگش نبود هیچ مخافت
\\
صد سفره دشمن بنهد طالب مقصود
&&
باشد که یکی دوست بیاید به ضیافت
\\
شمشیر ظرافت بود از دست عزیزان
&&
درویش نباید که برنجد به ظرافت
\\
سعدی چو گرفتار شدی تن به قضا ده
&&
دریا در و مرجان بود و هول و مخافت
\\
\end{longtable}
\end{center}
