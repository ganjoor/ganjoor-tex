\begin{center}
\section*{غزل ۱۵۳: سر تسلیم نهادیم به حکم و رایت}
\label{sec:153}
\addcontentsline{toc}{section}{\nameref{sec:153}}
\begin{longtable}{l p{0.5cm} r}
سر تسلیم نهادیم به حکم و رایت
&&
تا چه اندیشه کند رای جهان آرایت
\\
تو به هر جا که فرود آمدی و خیمه زدی
&&
کس دیگر نتواند که بگیرد جایت
\\
همچو مستسقی بر چشمه نوشین زلال
&&
سیر نتوان شدن از دیدن مهرافزایت
\\
روزگاریست که سودای تو در سر دارم
&&
مگرم سر برود تا برود سودایت
\\
قدر آن خاک ندارم که بر او می‌گذری
&&
که به هر وقت همی بوسه دهد بر پایت
\\
دوستان عیب کنندم که نبودی هشیار
&&
تا فرو رفت به گل پای جهان پیمایت
\\
چشم در سر به چه کار آید و جان در تن شخص
&&
گر تأمل نکند صورت جان آسایت
\\
دیگری نیست که مهر تو در او شاید بست
&&
هم در آیینه توان دید مگر همتایت
\\
روز آن است که مردم ره صحرا گیرند
&&
خیز تا سرو بماند خجل از بالایت
\\
دوش در واقعه دیدم که نگارین می‌گفت
&&
سعدیا گوش مکن بر سخن اعدایت
\\
عاشق صادق دیدار من آنگه باشی
&&
که به دنیا و به عقبی نبود پروایت
\\
طالب آن است که از شیر نگرداند روی
&&
یا نباید که به شمشیر بگردد رایت
\\
\end{longtable}
\end{center}
