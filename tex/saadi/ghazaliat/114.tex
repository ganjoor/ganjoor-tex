\begin{center}
\section*{غزل ۱۱۴: دردیست درد عشق که هیچش طبیب نیست}
\label{sec:114}
\addcontentsline{toc}{section}{\nameref{sec:114}}
\begin{longtable}{l p{0.5cm} r}
دردیست درد عشق که هیچش طبیب نیست
&&
گر دردمند عشق بنالد غریب نیست
\\
دانند عاقلان که مجانین عشق را
&&
پروای قول ناصح و پند ادیب نیست
\\
هر کو شراب عشق نخورده‌ست و درد درد
&&
آنست کز حیات جهانش نصیب نیست
\\
در مشک و عود و عنبر و امثال طیبات
&&
خوشتر ز بوی دوست دگر هیچ طیب نیست
\\
صید از کمند اگر بجهد بوالعجب بود
&&
ور نه چو در کمند بمیرد عجیب نیست
\\
گر دوست واقفست که بر من چه می‌رود
&&
باک از جفای دشمن و جور رقیب نیست
\\
بگریست چشم دشمن من بر حدیث من
&&
فضل از غریب هست و وفا در قریب نیست
\\
از خنده گل چنان به قفا اوفتاده باز
&&
کو را خبر ز مشغله عندلیب نیست
\\
سعدی ز دست دوست شکایت کجا بری
&&
هم صبر بر حبیب که صبر از حبیب نیست
\\
\end{longtable}
\end{center}
