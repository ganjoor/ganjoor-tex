\begin{center}
\section*{غزل ۴۵۶: ما نتوانیم و عشق پنجه درانداختن}
\label{sec:456}
\addcontentsline{toc}{section}{\nameref{sec:456}}
\begin{longtable}{l p{0.5cm} r}
ما نتوانیم و عشق پنجه درانداختن
&&
قوت او می‌کند بر سر ما تاختن
\\
گر دهیم ره به خویش یا نگذاری به پیش
&&
هر دو به دستت در است کشتن و بنواختن
\\
گر تو به شمشیر و تیر حمله بیاری رواست
&&
چاره ما هیچ نیست جز سپر انداختن
\\
کشتی در آب را از دو برون حال نیست
&&
یا همه سود ای حکیم یا همه درباختن
\\
مذهب اگر عاشقیست سنت عشاق چیست
&&
دل که نظرگاه اوست از همه پرداختن
\\
پایه خورشید نیست پیش تو افروختن
&&
یا قد و بالای سرو پیش تو افراختن
\\
هر که چنین روی دید جامه چو سعدی درید
&&
موجب دیوانگیست آفت بشناختن
\\
یا بگدازم چو شمع یا بکشندم به صبح
&&
چاره همین بیش نیست سوختن و ساختن
\\
ما سپر انداختیم با تو که در جنگ دوست
&&
زخم توان خورد و تیغ بر نتوان آختن
\\
\end{longtable}
\end{center}
