\begin{center}
\section*{غزل ۱: اول دفتر به نام ایزد دانا}
\label{sec:001}
\addcontentsline{toc}{section}{\nameref{sec:001}}
\begin{longtable}{l p{0.5cm} r}
اول دفتر به نام ایزد دانا
&&
صانع پروردگار حی توانا
\\
اکبر و اعظم خدای عالم و آدم
&&
صورت خوب آفرید و سیرت زیبا
\\
از در بخشندگی و بنده نوازی
&&
مرغ هوا را نصیب و ماهی دریا
\\
قسمت خود می‌خورند منعم و درویش
&&
روزی خود می‌برند پشه و عنقا
\\
حاجت موری به علم غیب بداند
&&
در بن چاهی به زیر صخره صما
\\
جانور از نطفه می‌کند شکر از نی
&&
برگ‌تر از چوب خشک و چشمه ز خارا
\\
شربت نوش آفرید از مگس نحل
&&
نخل تناور کند ز دانه خرما
\\
از همگان بی‌نیاز و بر همه مشفق
&&
از همه عالم نهان و بر همه پیدا
\\
پرتو نور سرادقات جلالش
&&
از عظمت ماورای فکرت دانا
\\
خود نه زبان در دهان عارف مدهوش
&&
حمد و ثنا می‌کند که موی بر اعضا
\\
هر که نداند سپاس نعمت امروز
&&
حیف خورد بر نصیب رحمت فردا
\\
بارخدایا مهیمنی و مدبر
&&
وز همه عیبی مقدسی و مبرا
\\
ما نتوانیم حق حمد تو گفتن
&&
با همه کروبیان عالم بالا
\\
سعدی از آن جا که فهم اوست سخن گفت
&&
ور نه کمال تو وهم کی رسد آن جا
\\
\end{longtable}
\end{center}
