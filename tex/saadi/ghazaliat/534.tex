\begin{center}
\section*{غزل ۵۳۴: دیدی که وفا به جا نیاوردی}
\label{sec:534}
\addcontentsline{toc}{section}{\nameref{sec:534}}
\begin{longtable}{l p{0.5cm} r}
دیدی که وفا به جا نیاوردی
&&
رفتی و خلاف دوستی کردی
\\
بیچارگیم به چیز نگرفتی
&&
درماندگیم به هیچ نشمردی
\\
من با همه جوری از تو خشنودم
&&
تو بی گنهی ز من بیازردی
\\
خود کردن و جرم دوستان دیدن
&&
رسمیست که در جهان تو آوردی
\\
نازت ببرم که نازک اندامی
&&
بارت بکشم که نازپروردی
\\
ما را که جراحت است خون آید
&&
درد تو چنم که فارغ از دردی
\\
گفتم که نریزم آب رخ زین بیش
&&
بر خاک درت که خون من خوردی
\\
وین عشق تو در من آفریدستند
&&
هرگز نرود ز زعفران زردی
\\
ای ذره تو در مقابل خورشید
&&
بیچاره چه می‌کنی بدین خردی
\\
در حلقه کارزار جان دادن
&&
بهتر که گریختن به نامردی
\\
سعدی سپر از جفا نیندازد
&&
گل با گیه است و صاف با دردی
\\
\end{longtable}
\end{center}
