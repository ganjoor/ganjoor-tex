\begin{center}
\section*{غزل ۱۴۶: ای جان خردمندان گوی خم چوگانت}
\label{sec:146}
\addcontentsline{toc}{section}{\nameref{sec:146}}
\begin{longtable}{l p{0.5cm} r}
ای جان خردمندان گوی خم چوگانت
&&
بیرون نرود گویی کافتاد به میدانت
\\
روز همه سر بر کرد از کوه و شب ما را
&&
سر بر نکند خورشید الا ز گریبانت
\\
جان در تن مشتاقان از ذوق به رقص آید
&&
چون باد بجنباند شاخی ز گلستانت
\\
دیوار سرایت را نقاش نمی‌باید
&&
تو زینت ایوانی نه صورت ایوانت
\\
هر چند نمی‌سوزد بر من دل سنگینت
&&
گویی دل من سنگیست در چاه زنخدانت
\\
جان باختن آسان است اندر نظرت لیکن
&&
این لاشه نمی‌بینم شایسته قربانت
\\
با داغ تو رنجوری به کز نظرت دوری
&&
پیش قدمت مردن خوشتر که به هجرانت
\\
ای بادیه هجران تا عشق حرم باشد
&&
عشاق نیندیشد از خار مغیلانت
\\
دیگر نتوانستم از فتنه حذر کردن
&&
زآنگه که در افتادم با قامت فتانت
\\
شاید که در این دنیا مرگش نبود هرگز
&&
سعدی که تو جان دارد بل دوست‌تر از جانت
\\
بسیار چو ذوالقرنین آفاق بگردیده‌ست
&&
این تشنه که می‌میرد بر چشمه حیوانت
\\
\end{longtable}
\end{center}
