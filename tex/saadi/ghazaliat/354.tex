\begin{center}
\section*{غزل ۳۵۴: وقت‌ها یک دم برآسودی تنم}
\label{sec:354}
\addcontentsline{toc}{section}{\nameref{sec:354}}
\begin{longtable}{l p{0.5cm} r}
وقت‌ها یک دم برآسودی تنم
&&
قال مولائی لطرفی لا تنم
\\
اسقیانی و دعانی افتضح
&&
عشق و مستوری نیامیزد به هم
\\
ما به مسکینی سلاح انداختیم
&&
لا تحلوا قتل من القی السلم
\\
یا غریب الحسن رفقا بالغریب
&&
خون درویشان مریز ای محتشم
\\
گر نکردستی به خونم پنجه تیز
&&
ما لذاک الکف مخضوبا بدم
\\
قد ملکت القلب ملکا دائما
&&
خواهی اکنون عدل کن خواهی ستم
\\
گر بخوانی ور برانی بنده‌ایم
&&
لا ابالی ان دعالی او شتم
\\
یا قضیب البان ما هذا الوقوف
&&
گر خلاف سرو می‌خواهی بچم
\\
عمرها پرهیز می‌کردم ز عشق
&&
ما حسبت الان الا قد هجم
\\
خلیانی نحو منظوری اقف
&&
تا چو شمع از سر بسوزم تا قدم
\\
در ازل رفته‌ست ما را دوستی
&&
لا تخونونی فعهدی ماانصرم
\\
بذل روحی فیک امر هین
&&
خود چه باشد در کف حاتم درم
\\
بنده‌ام تا زنده‌ام بی زینهار
&&
لم ازل عبدا و اوصالی رمم
\\
شنعة العذال عندی لم تفد
&&
کز ازل بر من کشیدند این رقم
\\
گر بنالم وقتی از زخمی قدیم
&&
لا تلومونی فجرحی ما التحم
\\
ان ترد محو البرایا فانکشف
&&
تا وجود خلق ریزی در عدم
\\
عقل و صبر از من چه می‌جویی که عشق
&&
کلما اسست بنیانا هدم
\\
انت فی قلبی الم تعلم به
&&
کز نصیحت کن نمی‌بیند الم
\\
سعدیا جان صرف کن در پای دوست
&&
ان غایات الامانی تغتنم
\\
\end{longtable}
\end{center}
