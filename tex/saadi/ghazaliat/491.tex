\begin{center}
\section*{غزل ۴۹۱: حناست آن که ناخن دلبند رشته‌ای}
\label{sec:491}
\addcontentsline{toc}{section}{\nameref{sec:491}}
\begin{longtable}{l p{0.5cm} r}
حناست آن که ناخن دلبند رشته‌ای
&&
یا خون بی دلیست که در بند کشته‌ای
\\
من آدمی به لطف تو دیگر ندیده‌ام
&&
این صورت و صفت که تو داری فرشته‌ای
\\
وین طرفه‌تر که تا دل من دردمند توست
&&
حاضر نبوده یک دم و غایب نگشته‌ای
\\
در هیچ حلقه نیست که یادت نمی‌رود
&&
در هیچ بقعه نیست که تخمی نکشته‌ای
\\
ما دفتر از حکایت عشقت نبسته‌ایم
&&
تو سنگدل حکایت ما درنوشته‌ای
\\
زیب و فریب آدمیان را نهایت است
&&
حوری مگر نه از گل آدم سرشته‌ای
\\
از عنبر و بنفشه تر بر سر آمده‌ست
&&
آن موی مشکبوی که در پای هشته‌ای
\\
من در بیان وصف تو حیران بمانده‌ام
&&
حدیست حسن را و تو از حد گذشته‌ای
\\
سر می‌نهند پیش خطت عارفان پارس
&&
بیتی مگر ز گفته سعدی نبشته‌ای
\\
\end{longtable}
\end{center}
