\begin{center}
\section*{غزل ۲۷۸: سروی چو تو می‌باید تا باغ بیاراید}
\label{sec:278}
\addcontentsline{toc}{section}{\nameref{sec:278}}
\begin{longtable}{l p{0.5cm} r}
سروی چو تو می‌باید تا باغ بیاراید
&&
ور در همه باغستان سروی نبود شاید
\\
در عقل نمی‌گنجد در وهم نمی‌آید
&&
کز تخم بنی آدم فرزند پری زاید
\\
چندان دل مشتاقان بربود لب لعلت
&&
کاندر همه شهر اکنون دل نیست که برباید
\\
هر کس سر سودایی دارند و تمنایی
&&
من بنده فرمانم تا دوست چه فرماید
\\
گر سر برود قطعا در پای نگارینش
&&
سهلست ولی ترسم کاو دست نیالاید
\\
حقا که مرا دنیا بی دوست نمی‌باید
&&
با تفرقه خاطر دنیا به چه کار آید
\\
سرهاست در این سودا چون حلقه زنان بر در
&&
تا بخت بلند این در بر روی که بگشاید
\\
ترسم نکند لیلی هرگز به وفا میلی
&&
تا خون دل مجنون از دیده نپالاید
\\
بر خسته نبخشاید آن سرکش سنگین دل
&&
باشد که چو بازآید بر کشته ببخشاید
\\
ساقی بده و بستان داد طرب از دنیا
&&
کاین عمر نمی‌ماند و این عهد نمی‌پاید
\\
گویند چرا سعدی از عشق نپرهیزد
&&
من مستم از این معنی هشیار سری باید
\\
\end{longtable}
\end{center}
