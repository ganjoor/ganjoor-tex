\begin{center}
\section*{غزل ۵۵۹: چونست حال بستان ای باد نوبهاری}
\label{sec:559}
\addcontentsline{toc}{section}{\nameref{sec:559}}
\begin{longtable}{l p{0.5cm} r}
چون است حال بستان ای باد نوبهاری
&&
کز بلبلان برآمد فریاد بی‌قراری
\\
ای گنج نوشدارو با خستگان نگه کن
&&
مرهم به دست و ما را مجروح می‌گذاری
\\
یا خلوتی برآور یا برقعی فرو هل
&&
ور نه به شکل شیرین شور از جهان برآری
\\
هر ساعت از لطیفی رویت عرق برآرد
&&
چون بر شکوفه آید باران نوبهاری
\\
عود است زیر دامن یا گل در آستینت
&&
یا مشک در گریبان بنمای تا چه داری
\\
گل نسبتی ندارد با روی دلفریبت
&&
تو در میان گل‌ها چون گل میان خاری
\\
وقتی کمند زلفت دیگر کمان ابرو
&&
این می‌کشد به زورم وآن می‌کشد به زاری
\\
ور قید می‌گشایی وحشی نمی‌گریزد
&&
در بند خوبرویان خوشتر که رستگاری
\\
زاول وفا نمودی چندان که دل ربودی
&&
چون مهر سخت کردم سست آمدی به یاری
\\
عمری دگر بباید بعد از فراق  ما را
&&
کاین عمر صرف کردیم  اندر امیدواری
\\
ترسم نماز صوفی با صحبت خیالت
&&
باطل بود که صورت بر قبله می‌نگاری
\\
هر درد را که بینی درمان و چاره‌ای هست
&&
درمان درد سعدی با دوست سازگاری
\\
\end{longtable}
\end{center}
