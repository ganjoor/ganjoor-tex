\begin{center}
\section*{غزل ۵۷: هر صبحدم نسیم گل از بوستان توست}
\label{sec:057}
\addcontentsline{toc}{section}{\nameref{sec:057}}
\begin{longtable}{l p{0.5cm} r}
هر صبحدم نسیم گل از بوستان توست
&&
الحان بلبل از نفس دوستان توست
\\
چون خضر دید آن لب جان بخش دلفریب
&&
گفتا که آب چشمهٔ حیوان دهان توست
\\
یوسف به بندگیت کمر بسته بر میان
&&
بودش یقین که ملک ملاحت از آن توست
\\
هر شاهدی که در نظر آمد به دلبری
&&
در دل نیافت راه که آنجا مکان توست
\\
هرگز نشان ز چشمهٔ کوثر شنیده‌ای
&&
کاو را نشانی از دهن بی‌نشان توست
\\
از رشک آفتاب جمالت بر آسمان
&&
هر ماه ماه دیدم چون ابروان توست
\\
این باد روح پرور از انفاس صبحدم
&&
گویی مگر ز طرهٔ عنبرفشان توست
\\
صد پیرهن قبا کنم از خرمی اگر
&&
بینم که دست من چو کمر در میان توست
\\
گفتند میهمانی عشاق می‌کنی
&&
سعدی به بوسه‌ای ز لبت میهمان توست
\\
\end{longtable}
\end{center}
