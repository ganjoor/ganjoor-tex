\begin{center}
\section*{غزل ۳۳۳: خطا کردی به قول دشمنان گوش}
\label{sec:333}
\addcontentsline{toc}{section}{\nameref{sec:333}}
\begin{longtable}{l p{0.5cm} r}
خطا کردی به قول دشمنان گوش
&&
که عهد دوستان کردی فراموش
\\
که گفت آن روی شهرآرای بنمای
&&
دگربارش که بنمودی فراپوش
\\
دل سنگینت آگاهی ندارد
&&
که من چون دیگ رویین می‌زنم جوش
\\
نمی‌بینم خلاص از دست فکرت
&&
مگر کافتاده باشم مست و مدهوش
\\
به ظاهر پند مردم می‌نیوشم
&&
نهانم عشق می‌گوید که منیوش
\\
مگر ساقی که بستانم ز دستش
&&
مگر مطرب که بر قولش کنم گوش
\\
مرا جامی بده وین جامه بستان
&&
مرا نقلی بنه وین خرقه بفروش
\\
نشستم تا برون آیی خرامان
&&
تو بیرون آمدی من رفتم از هوش
\\
تو در عالم نمی‌گنجی ز خوبی
&&
مرا هرگز کجا گنجی در آغوش
\\
خردمندان نصیحت می‌کنندم
&&
که سعدی چون دهل بیهوده مخروش
\\
ولیکن تا به چوگان می‌زنندش
&&
دهل هرگز نخواهد بود خاموش
\\
\end{longtable}
\end{center}
