\begin{center}
\section*{غزل ۸۷: گر کسی سرو شنیدست که رفتست اینست}
\label{sec:087}
\addcontentsline{toc}{section}{\nameref{sec:087}}
\begin{longtable}{l p{0.5cm} r}
گر کسی سرو شنیده‌ست که رفته‌ست این است
&&
یا صنوبر که بناگوش و برش سیمین است
\\
نه بلندیست به صورت که تو معلوم کنی
&&
که بلند از نظر مردم کوته‌بین است
\\
خواب در عهد تو در چشم من آید هیهات
&&
عاشقی کار سری نیست که بر بالین است
\\
همه آرام گرفتند و شب از نیمه گذشت
&&
وآنچه در خواب نشد چشم من و پروین است
\\
خود گرفتم که نظر بر رخ خوبان کفر است
&&
من از این بازنگردم که مرا این دین است
\\
وقت آن است که مردم ره صحرا گیرند
&&
خاصه اکنون که بهار آمد و فروردین است
\\
چمن امروز بهشت است و تو در می‌بایی
&&
تا خلایق همه گویند که حورالعین است
\\
هر چه گفتیم در اوصاف کمالیت او
&&
همچنان هیچ نگفتیم که صد چندین است
\\
آنچه سرپنجهٔ سیمین تو با سعدی کرد
&&
با کبوتر نکند پنجه که با شاهین است
\\
من دگر شعر نخواهم که نویسم که مگس
&&
زحمتم می‌دهد از بس که سخن شیرین است
\\
\end{longtable}
\end{center}
