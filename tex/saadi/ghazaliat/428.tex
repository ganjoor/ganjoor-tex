\begin{center}
\section*{غزل ۴۲۸: نه از چینم حکایت کن نه از روم}
\label{sec:428}
\addcontentsline{toc}{section}{\nameref{sec:428}}
\begin{longtable}{l p{0.5cm} r}
نه از چینم حکایت کن نه از روم
&&
که من دل با یکی دارم در این بوم
\\
هر آن ساعت که با یاد من آید
&&
فراموشم شود موجود و معدوم
\\
ز دنیا بخش ما غم خوردن آمد
&&
نشاید خوردن الا رزق مقسوم
\\
رطب شیرین و دست از نخل کوتاه
&&
زلال اندر میان و تشنه محروم
\\
از آن شاهد که در اندیشه ماست
&&
ندانم زاهدی در شهر معصوم
\\
به روی او نماند هیچ منظور
&&
به بوی او نماند هیچ مشموم
\\
نه بی او عشق می‌خواهم نه با او
&&
که او در سلک من حیف است منظوم
\\
رفیقان چشم ظاهربین بدوزید
&&
که ما را در میان سریست مکتوم
\\
همه عالم گر این صورت ببینند
&&
کس این معنی نخواهد کرد مفهوم
\\
چنان سوزم که خامانم نبینند
&&
نداند تندرست احوال محموم
\\
مرا گر دل دهی ور جان ستانی
&&
عبادت لازم است و بنده ملزوم
\\
نشاید برد سعدی جان از این کار
&&
مسافر تشنه و جلاب مسموم
\\
چو آهن تاب آتش می‌نیارد
&&
همی‌باید که پیشانی کند موم
\\
\end{longtable}
\end{center}
