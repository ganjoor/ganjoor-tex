\begin{center}
\section*{غزل ۵۸۰: گر درون سوخته‌ای با تو برآرد نفسی}
\label{sec:580}
\addcontentsline{toc}{section}{\nameref{sec:580}}
\begin{longtable}{l p{0.5cm} r}
گر درون سوخته‌ای با تو برآرد نفسی
&&
چه تفاوت کند اندر شکرستان مگسی
\\
ای که انصاف دل سوختگان می‌ندهی
&&
خود چنین روی نبایست نمودن به کسی
\\
روزی اندر قدمت افتم و گر سر برود
&&
به ز من در سر این واقعه رفتند بسی
\\
دامن دوست به دنیا نتوان داد از دست
&&
حیف باشد که دهی دامن گوهر به خسی
\\
تا به امروز مرا در سخن این سوز نبود
&&
که گرفتار نبودم به کمند هوسی
\\
چون سراییدن بلبل که خوش آید بر شاخ
&&
لیکن آن سوز ندارد که بود در قفسی
\\
سعدیا گر ز دل آتش به قلم در نزدی
&&
پس چرا دود به سر می‌رودش هر نفسی
\\
\end{longtable}
\end{center}
