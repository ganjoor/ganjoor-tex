\begin{center}
\section*{غزل ۴۴: بوی گل و بانگ مرغ برخاست}
\label{sec:044}
\addcontentsline{toc}{section}{\nameref{sec:044}}
\begin{longtable}{l p{0.5cm} r}
بوی گل و بانگ مرغ برخاست
&&
هنگام نشاط و روز صحراست
\\
فراش خزان ورق بیفشاند
&&
نقاش صبا چمن بیاراست
\\
ما را سر باغ و بوستان نیست
&&
هر جا که تویی تفرج آن جاست
\\
گویند نظر به روی خوبان
&&
نهیست نه این نظر که ما راست
\\
در روی تو سر صنع بی چون
&&
چون آب در آبگینه پیداست
\\
چشم چپ خویشتن برآرم
&&
تا چشم نبیندت به جز راست
\\
هر آدمیی که مهر مهرت
&&
در وی نگرفت سنگ خاراست
\\
روزی تر و خشک من بسوزد
&&
آتش که به زیر دیگ سوداست
\\
نالیدن بی‌حساب سعدی
&&
گویند خلاف رای داناست
\\
از ورطه ما خبر ندارد
&&
آسوده که بر کنار دریاست
\\
\end{longtable}
\end{center}
