\begin{center}
\section*{غزل ۶۳۳: ای باد صبحدم خبر دلستان بگوی}
\label{sec:633}
\addcontentsline{toc}{section}{\nameref{sec:633}}
\begin{longtable}{l p{0.5cm} r}
ای باد صبحدم خبر دلستان بگوی
&&
وصف جمال آن بت نامهربان بگوی
\\
بگذار مشک و بوی سر زلف او بیار
&&
یاد شکر مکن سخنی زآن دهان بگوی
\\
بستم به عشق موی میانش کمر چو مور
&&
گر وقت بینی این سخن اندر میان بگوی
\\
با بلبلان سوخته بال ضمیر من
&&
پیغام آن دو طوطی شکرفشان بگوی
\\
دانم که باز بر سر کویش گذر کنی
&&
گر بشنود حدیث منش در نهان بگوی
\\
کای دل ربوده از بر من حکم از آن توست
&&
گر نیز گوییم به مثل ترک جان بگوی
\\
هر لحظه راز دل جهدم بر سر زبان
&&
دل می‌تپد که عمر بشد وارهان بگوی
\\
سر دل از زبان نشود هرگز آشکار
&&
گر دل موافقت نکند کای زبان بگوی
\\
ای باد صبح دشمن سعدی مراد یافت
&&
نزدیک دوستان وی این داستان بگوی
\\
\end{longtable}
\end{center}
