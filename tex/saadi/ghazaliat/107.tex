\begin{center}
\section*{غزل ۱۰۷: صبحدم خاکی به صحرا برد باد از کوی دوست}
\label{sec:107}
\addcontentsline{toc}{section}{\nameref{sec:107}}
\begin{longtable}{l p{0.5cm} r}
صبحدم خاکی به صحرا برد باد از کوی دوست
&&
بوستان در عنبر سارا گرفت از بوی دوست
\\
دوست گر با ما بسازد دولتی باشد عظیم
&&
ور نسازد می‌بباید ساختن با خوی دوست
\\
گر قبولم می‌کند مملوک خود می‌پرورد
&&
ور براند پنجه نتوان کرد با بازوی دوست
\\
هر که را خاطر به روی دوست رغبت می‌کند
&&
بس پریشانی بباید بردنش چون موی دوست
\\
دیگران را عید اگر فرداست ما را این دمست
&&
روزه داران ماه نو بینند و ما ابروی دوست
\\
هر کسی بی خویشتن جولان عشقی می‌کند
&&
تا به چوگان که در خواهد فتادن گوی دوست
\\
دشمنم را بد نمی‌خواهم که آن بدبخت را
&&
این عقوبت بس که بیند دوست همزانوی دوست
\\
هر کسی را دل به صحرایی و باغی می‌رود
&&
هر کس از سویی به دررفتند و عاشق سوی دوست
\\
کاش باری باغ و بستان را که تحسین می‌کنند
&&
بلبلی بودی چو سعدی یا گلی چون روی دوست
\\
\end{longtable}
\end{center}
