\begin{center}
\section*{غزل ۲۲۱: مجلس ما دگر امروز به بستان ماند}
\label{sec:221}
\addcontentsline{toc}{section}{\nameref{sec:221}}
\begin{longtable}{l p{0.5cm} r}
مجلس ما دگر امروز به بستان ماند
&&
عیش خلوت به تماشای گلستان ماند
\\
می حلالست کسی را که بود خانه بهشت
&&
خاصه از دست حریفی که به رضوان ماند
\\
خط سبز و لب لعلت به چه ماننده کنی
&&
من بگویم به لب چشمه حیوان ماند
\\
تا سر زلف پریشان تو محبوب منست
&&
روزگارم به سر زلف پریشان ماند
\\
چه کند کشته عشقت که نگوید غم دل
&&
تو مپندار که خون ریزی و پنهان ماند
\\
هر که چون موم به خورشید رخت نرم نشد
&&
زینهار از دل سختش که به سندان ماند
\\
نادر افتد که یکی دل به وصالت ندهد
&&
یا کسی در بلد کفر مسلمان ماند
\\
تو که چون برق بخندی چه غمت دارد از آنک
&&
من چنان زار بگریم که به باران ماند
\\
طعنه بر حیرت سعدی نه به انصاف زدی
&&
کس چنین روی نبیند که نه حیران ماند
\\
هر که با صورت و بالای تواش انسی نیست
&&
حیوانیست که بالاش به انسان ماند
\\
\end{longtable}
\end{center}
