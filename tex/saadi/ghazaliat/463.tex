\begin{center}
\section*{غزل ۴۶۳: چه خوش بود دو دلارام دست در گردن}
\label{sec:463}
\addcontentsline{toc}{section}{\nameref{sec:463}}
\begin{longtable}{l p{0.5cm} r}
چه خوش بود دو دلارام دست در گردن
&&
به هم نشستن و حلوای آشتی خوردن
\\
به روزگار عزیزان که روزگار عزیز
&&
دریغ باشد بی دوستان به سر بردن
\\
اگر هزار جفا سروقامتی بکند
&&
چو خود بیاید عذرش بباید آوردن
\\
چه شکر گویمت ای باد مشک بوی وصال
&&
که بوستان امیدم بخواست پژمردن
\\
فراق روی تو هر روز نفس کشتن بود
&&
نظر به شخص تو امروز روح پروردن
\\
کسی که قیمت ایام وصل نشناسد
&&
ببایدش دو سه روزی مفارقت کردن
\\
اگر سری برود بی‌گناه در پایی
&&
به خرده‌ای ز بزرگان نشاید آزردن
\\
به تازیانه گرفتم که بی دلی بزنی
&&
کجا تواند رفتن کمند در گردن
\\
کمال شوق ندارند عاشقان صبور
&&
که احتمال ندارد بر آتش افسردن
\\
گر آدمی صفتی سعدیا به عشق بمیر
&&
که مذهب حیوان است همچنین مردن
\\
\end{longtable}
\end{center}
