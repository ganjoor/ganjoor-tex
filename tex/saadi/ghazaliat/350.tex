\begin{center}
\section*{غزل ۳۵۰: من ایستاده‌ام اینک به خدمتت مشغول}
\label{sec:350}
\addcontentsline{toc}{section}{\nameref{sec:350}}
\begin{longtable}{l p{0.5cm} r}
من ایستاده‌ام اینک به خدمتت مشغول
&&
مرا از آن چه که خدمت قبول یا نه قبول
\\
نه دست با تو درآویختن نه پای گریز
&&
نه احتمال فراق و نه اختیار وصول
\\
کمند عشق نه بس بود زلف مفتولت
&&
که روی نیز بکردی ز دوستان مفتول
\\
من آنم ار تو نه آنی که بودی اندر عهد
&&
به دوستی که نکردم ز دوستیت عدول
\\
ملامتت نکنم گر چه بی‌وفا یاری
&&
هزار جان عزیزت فدای طبع ملول
\\
مرا گناه خود است ار ملامت تو برم
&&
که عشق بار گران بود و من ظلوم جهول
\\
گر آن چه بر سر من می‌رود ز دست فراق
&&
علی التمام فروخوانم الحدیث یطول
\\
ز دست گریه کتابت نمی‌توانم کرد
&&
که می‌نویسم و در حال می‌شود مغسول
\\
من از کجا و نصیحت کنان بیهده گوی
&&
حکیم را نرسد کدخدایی بهلول
\\
طریق عشق به گفتن نمی‌توان آموخت
&&
مگر کسی که بود در طبیعتش مجبول
\\
اسیر بند غمت را به لطف خویش بخوان
&&
که گر به قهر برانی کجا شود مغلول
\\
نه زور بازوی سعدی که دست قوت شیر
&&
سپر بیفکند از تیغ غمزه مسلول
\\
\end{longtable}
\end{center}
