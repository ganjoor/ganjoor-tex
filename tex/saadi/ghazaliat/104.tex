\begin{center}
\section*{غزل ۱۰۴: مرا تو غایت مقصودی از جهان ای دوست}
\label{sec:104}
\addcontentsline{toc}{section}{\nameref{sec:104}}
\begin{longtable}{l p{0.5cm} r}
مرا تو غایت مقصودی از جهان ای دوست
&&
هزار جان عزیزت فدای جان ای دوست
\\
چنان به دام تو الفت گرفت مرغ دلم
&&
که یاد می‌نکند عهد آشیان ای دوست
\\
گرم تو در نگشایی کجا توانم رفت
&&
به راستان که بمیرم بر آستان ای دوست
\\
دلی شکسته و جانی نهاده بر کف دست
&&
بگو بیار که گویم بگیر هان ای دوست
\\
تنم بپوسد و خاکم به باد ریزه شود
&&
هنوز مهر تو باشد در استخوان ای دوست
\\
جفا مکن که بزرگان به خرده‌ای ز رهی
&&
چنین سبک ننشینند و سرگران ای دوست
\\
به لطف اگر بخوری خون من روا باشد
&&
به قهرم از نظر خویشتن مران ای دوست
\\
مناسب لب لعلت حدیث بایستی
&&
جواب تلخ بدیعست از آن دهان ای دوست
\\
مرا رضای تو باید نه زندگانی خویش
&&
اگر مراد تو قتلست وارهان ای دوست
\\
که گفت سعدی از آسیب عشق بگریزد
&&
به دوستی که غلط می‌برد گمان ای دوست
\\
که گر به جان رسد از دست دشمنانم کار
&&
ز دوستی نکنم توبه همچنان ای دوست
\\
\end{longtable}
\end{center}
