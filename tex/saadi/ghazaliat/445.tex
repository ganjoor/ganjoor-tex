\begin{center}
\section*{غزل ۴۴۵: در وصف نیاید که چه شیرین دهنست آن}
\label{sec:445}
\addcontentsline{toc}{section}{\nameref{sec:445}}
\begin{longtable}{l p{0.5cm} r}
در وصف نیاید که چه شیرین دهن است آن
&&
این است که دور از لب و دندان من است آن
\\
عارض نتوان گفت که دور قمر است این
&&
بالا نتوان خواند که سرو چمن است آن
\\
در سرو رسیده‌ست ولیکن به حقیقت
&&
از سرو گذشته‌ست که سیمین بدن است آن
\\
هرگز نبود جسم بدین حسن و لطافت
&&
گویی همه روح است که در پیرهن است آن
\\
خال است بر آن صفحه سیمین بناگوش
&&
یا نقطه‌ای از غالیه بر یاسمن است آن
\\
فی الجمله قیامت تویی امروز در آفاق
&&
در چشم تو پیداست که باب فتن است آن
\\
گفتم که دل از چنبر زلفت برهانم
&&
ترسم نرهانم که شکن بر شکن است آن
\\
هر کس که به جان آرزوی وصل تو دارد
&&
دشوار برآید که محقر ثمن است آن
\\
مردی که ز شمشیر جفا روی بتابد
&&
در کوی وفا مرد مخوانش که زن است آن
\\
گر خسته دلی نعره زند بر سر کویی
&&
عیبش نتوان گفت که بی خویشتن است آن
\\
نزدیک من آن است که هر جرم و خطایی
&&
کز صاحب وجه حسن آید حسن است آن
\\
سعدی سر سودای تو دارد نه سر خویش
&&
هر جامه که عیار بپوشد کفن است آن
\\
\end{longtable}
\end{center}
