\begin{center}
\section*{غزل ۸۶: بخت جوان دارد آن که با تو قرینست}
\label{sec:086}
\addcontentsline{toc}{section}{\nameref{sec:086}}
\begin{longtable}{l p{0.5cm} r}
بخت جوان دارد آن که با تو قرینست
&&
پیر نگردد که در بهشت برینست
\\
دیگر از آن جانبم نماز نباشد
&&
گر تو اشارت کنی که قبله چنینست
\\
آینه‌ای پیش آفتاب نهادست
&&
بر در آن خیمه یا شعاع جبینست
\\
گر همه عالم ز لوح فکر بشویند
&&
عشق نخواهد شدن که نقش نگینست
\\
گوشه گرفتم ز خلق و فایده‌ای نیست
&&
گوشه چشمت بلای گوشه نشینست
\\
تا نه تصور کنی که بی تو صبوریم
&&
گر نفسی می‌زنیم بازپسینست
\\
حسن تو هر جا که طبل عشق فروکوفت
&&
بانگ برآمد که غارت دل و دینست
\\
سیم و زرم گو مباش و دنیی و اسباب
&&
روی تو بینم که ملک روی زمینست
\\
عاشق صادق به زخم دوست نمیرد
&&
زهر مذابم بده که ماء معینست
\\
سعدی از این پس که راه پیش تو دانست
&&
گر ره دیگر رود ضلال مبینست
\\
\end{longtable}
\end{center}
