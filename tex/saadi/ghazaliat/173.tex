\begin{center}
\section*{غزل ۱۷۳: آن که بر نسترن از غالیه خالی دارد}
\label{sec:173}
\addcontentsline{toc}{section}{\nameref{sec:173}}
\begin{longtable}{l p{0.5cm} r}
آن که بر نسترن از غالیه خالی دارد
&&
الحق آراسته خلقی و جمالی دارد
\\
درد دل پیش که گویم که به جز باد صبا
&&
کس ندانم که در آن کوی مجالی دارد
\\
دل چنین سخت نباشد که یکی بر سر راه
&&
تشنه می‌میرد و شخص آب زلالی دارد
\\
زندگانی نتوان گفت و حیاتی که مراست
&&
زنده آنست که با دوست وصالی دارد
\\
من به دیدار تو مشتاقم و از غیر ملول
&&
گر تو را از من و از غیر ملالی دارد
\\
مرغ بر بام تو ره دارد و من بر سر کوی
&&
حبذا مرغ که آخر پر و بالی دارد
\\
غم دل با تو نگویم که نداری غم دل
&&
با کسی حال توان گفت که حالی دارد
\\
طالب وصل تو چون مفلس و اندیشه گنج
&&
حاصل آنست که سودای محالی دارد
\\
عاقبت سر به بیابان بنهد چون سعدی
&&
هر که در سر هوس چون تو غزالی دارد
\\
\end{longtable}
\end{center}
