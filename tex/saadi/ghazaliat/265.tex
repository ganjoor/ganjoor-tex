\begin{center}
\section*{غزل ۲۶۵: هر که را باغچه‌ای هست به بستان نرود}
\label{sec:265}
\addcontentsline{toc}{section}{\nameref{sec:265}}
\begin{longtable}{l p{0.5cm} r}
هر که را باغچه‌ای هست به بستان نرود
&&
هر که مجموع نشستست پریشان نرود
\\
آن که در دامنش آویخته باشد خاری
&&
هرگزش گوشه خاطر به گلستان نرود
\\
سفر قبله درازست و مجاور با دوست
&&
روی در قبله معنی به بیابان نرود
\\
گر بیارند کلید همه درهای بهشت
&&
جان عاشق به تماشاگه رضوان نرود
\\
گر سرت مست کند بوی حقیقت روزی
&&
اندرونت به گل و لاله و ریحان نرود
\\
هر که دانست که منزلگه معشوق کجاست
&&
مدعی باشد اگر بر سر پیکان نرود
\\
صفت عاشق صادق به درستی آنست
&&
که گرش سر برود از سر پیمان نرود
\\
به نصیحتگر دل شیفته می‌باید گفت
&&
برو ای خواجه که این درد به درمان نرود
\\
به ملامت نبرند از دل ما صورت عشق
&&
نقش بر سنگ نبشتست به طوفان نرود
\\
عشق را عقل نمی‌خواست که بیند لیکن
&&
هیچ عیار نباشد که به زندان نرود
\\
سعدیا گر همه شب شرح غمش خواهی گفت
&&
شب به پایان رود و شرح به پایان نرود
\\
\end{longtable}
\end{center}
