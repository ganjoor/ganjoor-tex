\begin{center}
\section*{غزل ۷۵: کارم چو زلف یار پریشان و درهمست}
\label{sec:075}
\addcontentsline{toc}{section}{\nameref{sec:075}}
\begin{longtable}{l p{0.5cm} r}
کارم چو زلف یار پریشان و درهمست
&&
پشتم به سان ابروی دلدار پرخمست
\\
غم شربتی ز خون دلم نوش کرد و گفت
&&
این شادی کسی که در این دور خرمست
\\
تنها دل منست گرفتار در غمان
&&
یا خود در این زمانه دل شادمان کمست
\\
زین سان که می‌دهد دل من داد هر غمی
&&
انصاف ملک عالم عشقش مسلمست
\\
دانی خیال روی تو در چشم من چه گفت
&&
آیا چه جاست این که همه روزه با نمست
\\
خواهی چو روز روشن دانی تو حال من
&&
از تیره شب بپرس که او نیز محرمست
\\
ای کاشکی میان منستی و دلبرم
&&
پیوندی این چنین که میان من و غمست
\\
\end{longtable}
\end{center}
