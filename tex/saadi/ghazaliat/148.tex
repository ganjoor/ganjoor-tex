\begin{center}
\section*{غزل ۱۴۸: چو نیست راه برون آمدن ز میدانت}
\label{sec:148}
\addcontentsline{toc}{section}{\nameref{sec:148}}
\begin{longtable}{l p{0.5cm} r}
چو نیست راه برون آمدن ز میدانت
&&
ضرورتست چو گوی احتمال چوگانت
\\
به راستی که نخواهم بریدن از تو امید
&&
به دوستی که نخواهم شکست پیمانت
\\
گرم هلاک پسندی ورم بقا بخشی
&&
به هر چه حکم کنی نافذست فرمانت
\\
اگر تو عید همایون به عهد بازآیی
&&
بخیلم ار نکنم خویشتن به قربانت
\\
مه دوهفته ندارد فروغ چندانی
&&
که آفتاب که می‌تابد از گریبانت
\\
اگر نه سرو که طوبی برآمدی در باغ
&&
خجل شدی چو بدیدی قد خرامانت
\\
نظر به روی تو صاحب دلی نیندازد
&&
که بی‌دلش نکند چشم‌های فتانت
\\
غلام همت شنگولیان و رندانم
&&
نه زاهدان که نظر می‌کنند پنهانت
\\
بیا و گر همه بد کرده‌ای که نیکت باد
&&
دعای نیکان از چشم بد نگهبانت
\\
به خاک پات که گر سر فدا کند سعدی
&&
مقصرست هنوز از ادای احسانت
\\
\end{longtable}
\end{center}
