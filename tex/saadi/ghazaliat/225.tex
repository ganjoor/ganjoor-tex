\begin{center}
\section*{غزل ۲۲۵: اینان مگر ز رحمت محض آفریده‌اند}
\label{sec:225}
\addcontentsline{toc}{section}{\nameref{sec:225}}
\begin{longtable}{l p{0.5cm} r}
اینان مگر ز رحمت محض آفریده‌اند
&&
کآرام جان و انس دل و نور دیده‌اند
\\
لطف آیتی‌ست در حق اینان و کبر و ناز
&&
پیراهنی که بر قد ایشان بریده‌اند
\\
آید هنوزشان ز لب لعل بوی شیر
&&
شیرین لبان نه شیر که شکر مزیده‌اند
\\
پندارم آهوان تتارند مشک ریز
&&
لیکن به زیر سایهٔ طوبی چریده‌اند
\\
رضوان مگر سراچهٔ فردوس برگشاد
&&
کاین حوریان به ساحت دنیا خزیده‌اند
\\
آب حیات در لب اینان به ظن من
&&
کز لوله‌های چشمهٔ کوثر مکیده‌اند
\\
دست گدا به سیب زنخدان این گروه
&&
نادر رسد که میوهٔ اول رسیده‌اند
\\
گل برچنند روز به روز از درخت گل
&&
زین گلبنان هنوز مگر گل نچیده‌اند
\\
عذر است هندوی بت سنگین پرست را
&&
بیچارگان مگر بت سیمین ندیده‌اند
\\
این لطف بین که با گل آدم سرشته‌اند
&&
وین روح بین که در تن آدم دمیده‌اند
\\
آن نقطه‌های خال چه شاهد نشانده‌اند
&&
وین خط‌های سبز چه موزون کشیده‌اند
\\
بر استوای قامتشان گویی ابروان
&&
بالای سرو راست هلالی خمیده‌اند
\\
با قامت بلند صنوبرخرامشان
&&
سرو بلند و کاج به شوخی چمیده‌اند
\\
سحر است چشم و زلف و بناگوششان دریغ
&&
کاین مؤمنان به سحر چنین بگرویده‌اند
\\
ز ایشان توان به خون جگر یافتن مراد
&&
کز کودکی به خون جگر پروریده‌اند
\\
دامن کشان حسن دلاویز را چه غم
&&
کآشفتگان عشق گریبان دریده‌اند
\\
در باغ حسن خوشتر از اینان درخت نیست
&&
مرغان دل بدین هوس از بر پریده‌اند
\\
با چابکان دلبر و شوخان دلفریب
&&
بسیار درفتاده و اندک رهیده‌اند
\\
هرگز جماعتی که شنیدند سر عشق
&&
نشنیده‌ام که باز نصیحت شنیده‌اند
\\
زنهار اگر به دانه خالی نظر کنی
&&
ساکن که دام زلف بر آن گستریده‌اند
\\
گر شاهدان نه دنیی و دین می‌برند و عقل
&&
پس زاهدان برای چه خلوت گزیده‌اند
\\
نادر گرفت دامن سودای وصلشان
&&
دستی که عاقبت نه به دندان گزیده‌اند
\\
بر خاک ره نشستن سعدی عجب مدار
&&
مردان چه جای خاک که بر خون طپیده‌اند
\\
\end{longtable}
\end{center}
