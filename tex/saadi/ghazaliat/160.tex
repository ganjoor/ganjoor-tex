\begin{center}
\section*{غزل ۱۶۰: حدیث عشق به طومار در نمی‌گنجد}
\label{sec:160}
\addcontentsline{toc}{section}{\nameref{sec:160}}
\begin{longtable}{l p{0.5cm} r}
حدیث عشق به طومار در نمی‌گنجد
&&
بیان دوست به گفتار در نمی‌گنجد
\\
سماع انس که دیوانگان از آن مستند
&&
به سمع مردم هشیار در نمی‌گنجد
\\
میسرت نشود عاشقی و مستوری
&&
ورع به خانه خمار در نمی‌گنجد
\\
چنان فراخ نشستست یار در دل تنگ
&&
که بیش زحمت اغیار در نمی‌گنجد
\\
تو را چنان که تویی من صفت ندانم کرد
&&
که عرض جامه به بازار در نمی‌گنجد
\\
دگر به صورت هیچ آفریده دل ندهم
&&
که با تو صورت دیوار در نمی‌گنجد
\\
خبر که می‌دهد امشب رقیب مسکین را
&&
که سگ به زاویه غار در نمی‌گنجد
\\
چو گل به بار بود همنشین خار بود
&&
چو در کنار بود خار در نمی‌گنجد
\\
چنان ارادت و شوقست در میان دو دوست
&&
که سعی دشمن خون خوار در نمی‌گنجد
\\
به چشم دل نظرت می‌کنم که دیده سر
&&
ز برق شعله دیدار در نمی‌گنجد
\\
ز دوستان که تو را هست جای سعدی نیست
&&
گدا میان خریدار در نمی‌گنجد
\\
\end{longtable}
\end{center}
