\begin{center}
\section*{غزل ۴۱۹: مرا تا نقره باشد می‌فشانم}
\label{sec:419}
\addcontentsline{toc}{section}{\nameref{sec:419}}
\begin{longtable}{l p{0.5cm} r}
مرا تا نقره باشد می‌فشانم
&&
تو را تا بوسه باشد می‌ستانم
\\
و گر فردا به زندان می‌برندم
&&
به نقد این ساعت اندر بوستانم
\\
جهان بگذار تا بر من سر آید
&&
که کام دل تو بودی از جهانم
\\
چه دامن‌های گل باشد در این باغ
&&
اگر چیزی نگوید باغبانم
\\
نمی‌دانستم از بخت همایون
&&
که سیمرغی فتد در آشیانم
\\
تو عشق آموختی در شهر ما را
&&
بیا تا شرح آن هم بر تو خوانم
\\
سخن‌ها دارم از دست تو در دل
&&
ولیکن در حضورت بی زبانم
\\
بگویم تا بداند دشمن و دوست
&&
که من مستی و مستوری ندانم
\\
مگو سعدی مراد خویش برداشت
&&
اگر تو سنگدل من مهربانم
\\
اگر تو سرو سیمین تن بر آنی
&&
که از پیشم برانی من بر آنم
\\
که تا باشم خیالت می‌پرستم
&&
و گر رفتم سلامت می‌رسانم
\\
\end{longtable}
\end{center}
