\begin{center}
\section*{غزل ۱۴۴: ای که رحمت می‌نیاید بر منت}
\label{sec:144}
\addcontentsline{toc}{section}{\nameref{sec:144}}
\begin{longtable}{l p{0.5cm} r}
ای که رحمت می‌نیاید بر منت
&&
آفرین بر جان و رحمت بر تنت
\\
قامتت گویم که دلبندست و خوب
&&
یا سخن یا آمدن یا رفتنت
\\
شرمش از روی تو باید آفتاب
&&
کاندرآید بامداد از روزنت
\\
حسن اندامت نمی‌گویم به شرح
&&
خود حکایت می‌کند پیراهنت
\\
ای که سر تا پایت از گل خرمنست
&&
رحمتی کن بر گدای خرمنت
\\
ماه رویا مهربانی پیشه کن
&&
سیرتی چون صورت مستحسنت
\\
ای جمال کعبه رویی باز کن
&&
تا طوافی می‌کنم پیرامنت
\\
دست گیر این پنج روزم در حیات
&&
تا نگیرم در قیامت دامنت
\\
عزم دارم کز دلت بیرون کنم
&&
و اندرون جان بسازم مسکنت
\\
درد دل با سنگدل گفتن چه سود
&&
باد سردی می‌دمم در آهنت
\\
گفتم از جورت بریزم خون خویش
&&
گفت خون خویشتن در گردنت
\\
گفتم آتش درزنم آفاق را
&&
گفت سعدی درنگیرد با منت
\\
\end{longtable}
\end{center}
