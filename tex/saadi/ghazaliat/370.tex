\begin{center}
\section*{غزل ۳۷۰: من همان روز که آن خال بدیدم گفتم}
\label{sec:370}
\addcontentsline{toc}{section}{\nameref{sec:370}}
\begin{longtable}{l p{0.5cm} r}
من همان روز که آن خال بدیدم گفتم
&&
بیم آن است بدین دانه که در دام افتم
\\
هرگز آشفته رویی نشدم یا مویی
&&
مگر اکنون که به روی تو چو موی آشفتم
\\
هیچ شک نیست که این واقعه با طاق افتد
&&
گو بدانید که من با غم رویش جفتم
\\
رنگ رویم غم دل پیش کسان می‌گوید
&&
فاش کرد آن که ز بیگانه همی‌بنهفتم
\\
پیش از آنم که به دیوانگی انجامد کار
&&
معرفت پند همی‌داد و نمی‌پذرفتم
\\
هر که این روی ببیند بدهد پشت گریز
&&
گر بداند که من از وی به چه پهلو خفتم
\\
آتشی بر سرم از داغ جدایی می‌رفت
&&
و آبی از دیده همی‌شد که زمین می‌سفتم
\\
عجب آن است که با زحمت چندینی خار
&&
بوی صبحی نشنیدم که چو گل نشکفتم
\\
پیش از این خاطر من خانه پرمشغله بود
&&
با تو پرداختمش وز همه عالم رفتم
\\
سعدی آن نیست که درخورد تو گوید سخنی
&&
آن چه در وسع خودم در دهن آمد گفتم
\\
\end{longtable}
\end{center}
