\begin{center}
\section*{غزل ۶۴: این بوی روح پرور از آن خوی دلبرست}
\label{sec:064}
\addcontentsline{toc}{section}{\nameref{sec:064}}
\begin{longtable}{l p{0.5cm} r}
این بوی روح پرور از آن خوی دلبر است
&&
وین آب زندگانی از آن حوض کوثر است
\\
ای باد بوستان مگرت نافه در میان
&&
وی مرغ آشنا مگرت نامه در پر است
\\
بوی بهشت می‌گذرد یا نسیم دوست
&&
یا کاروان صبح که گیتی منور است
\\
این قاصد از کدام زمین است مشک بوی
&&
وین نامه در چه داشت که عنوان معطرست
\\
بر راه باد عود در آتش نهاده‌اند
&&
یا خود در آن زمین که تویی خاک عنبر است
\\
بازآ و حلقه بر در رندان شوق زن
&&
کاصحاب را دو دیده چو مسمار بر در است
\\
بازآ که در فراق تو چشم امیدوار
&&
چون گوش روزه دار بر الله اکبر است
\\
دانی که چون همی‌گذرانیم روزگار
&&
روزی که بی تو می‌گذرد روز محشر است
\\
گفتیم عشق را به صبوری دوا کنیم
&&
هر روز عشق بیشتر و صبر کمتر است
\\
صورت ز چشم غایب و اخلاق در نظر
&&
دیدار در حجاب و معانی برابر است
\\
در نامه نیز چند بگنجد حدیث عشق
&&
کوته کنم که قصهٔ ما کار دفتر است
\\
همچون درخت بادیه سعدی به برق شوق
&&
سوزان و میوهٔ سخنش همچنان تر است
\\
آری خوش است وقت حریفان به بوی عود
&&
وز سوز غافلند که در جان مجمر است
\\
\end{longtable}
\end{center}
