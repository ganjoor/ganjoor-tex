\begin{center}
\section*{غزل ۵۵۵: گر برود به هر قدم در ره دیدنت سری}
\label{sec:555}
\addcontentsline{toc}{section}{\nameref{sec:555}}
\begin{longtable}{l p{0.5cm} r}
گر برود به هر قدم در ره دیدنت سری
&&
من نه حریف رفتنم از در تو به هر دری
\\
تا نکند وفای تو در دل من تغیری
&&
چشم نمی‌کنم به خود تا چه رسد به دیگری
\\
خود نبود و گر بود تا به قیامت آزری
&&
بت نکند به نیکویی چون تو بدیع پیکری
\\
سرو روان ندیده‌ام جز تو به هیچ کشوری
&&
هم نشنیده‌ام که زاد از پدری و مادری
\\
گر به کنار آسمان چون تو برآید اختری
&&
روی بپوشد آفتاب از نظرش به معجری
\\
حاجت گوش و گردنت نیست به زر و زیوری
&&
یا به خضاب و سرمه‌ای یا به عبیر و عنبری
\\
تاب وغا نیاورد قوت هیچ صفدری
&&
گر تو بدین مشاهدت حمله بری به لشکری
\\
بسته‌ام از جهانیان بر دل تنگ من دری
&&
تا نکنم به هیچ کس گوشه چشم خاطری
\\
گر چه تو بهتری و من از همه خلق کمتری
&&
شاید اگر نظر کند محتشمی به چاکری
\\
باک مدار سعدیا گر به فدا رود سری
&&
هر که به معظمی رسد ترک دهد محقری
\\
\end{longtable}
\end{center}
