\begin{center}
\section*{غزل ۴۷۷: چه روی و موی و بناگوش و خط و خالست این}
\label{sec:477}
\addcontentsline{toc}{section}{\nameref{sec:477}}
\begin{longtable}{l p{0.5cm} r}
چه روی و موی و بناگوش و خط و خال است این
&&
چه قد و قامت و رفتار و اعتدال است این
\\
کسی که در همه عمر این صفت مطالعه کرد
&&
به دیگری نگرد یا به خود محال است این
\\
کمال حسن وجودت ز هر که پرسیدم
&&
جواب داد که در غایت کمال است این
\\
نماز شام به بام ار کسی نگاه کند
&&
دو ابروان تو گوید مگر هلالست این
\\
لبت به خون عزیزان که می‌خوری لعل است
&&
تو خود بگوی که خون می‌خوری حلال است این
\\
چنان به یاد تو شادم که فرق می‌نکنم
&&
ز دوستی که فراق است یا وصال است این
\\
شبی خیال تو گفتم ببینم اندر خواب
&&
ولی ز فکر تو خواب آیدم خیال است این
\\
درازنای شب از چشم دردمندان پرس
&&
عزیز من که شبی یا هزار سال است این
\\
قلم به یاد تو در می‌چکاند از دستم
&&
مداد نیست کز او می‌رود زلال است این
\\
کسان به حال پریشان سعدی از غم عشق
&&
زنخ زنند و ندانند تا چه حال است این
\\
\end{longtable}
\end{center}
