\begin{center}
\section*{غزل ۱۲۹: خسرو آنست که در صحبت او شیرینیست}
\label{sec:129}
\addcontentsline{toc}{section}{\nameref{sec:129}}
\begin{longtable}{l p{0.5cm} r}
خسرو آنست که در صحبت او شیرینیست
&&
در بهشتست که همخوابه حورالعینیست
\\
دولت آنست که امکان فراغت باشد
&&
تکیه بر بالش بی دوست نه بس تمکینیست
\\
همه عالم صنم چین به حکایت گویند
&&
صنم ماست که در هر خم زلفش چینیست
\\
روی اگر باز کند حلقه سیمین در گوش
&&
همه گویند که این ماهی و آن پروینیست
\\
گر منش دوست ندارم همه کس دارد دوست
&&
تا چه ویسیست که در هر طرفش رامینیست
\\
سر مویی نظر آخر به کرم با ما کن
&&
ای که در هر بن موییت دل مسکینیست
\\
جز به دیدار توام دیده نمی‌باشد باز
&&
گویی از مهر تو با هر که جهانم کینیست
\\
هر که ماه ختن و سرو روانت گوید
&&
او هنوز از قد و بالای تو صورت بینیست
\\
بنده خویشتنم خوان که به شاهی برسم
&&
مگسی را که تو پرواز دهی شاهینیست
\\
نام سعدی همه جا رفت به شاهدبازی
&&
وین نه عیبست که در ملت ما تحسینیست
\\
کافر و کفر و مسلمان و نماز و من و عشق
&&
هر کسی را که تو بینی به سر خود دینیست
\\
\end{longtable}
\end{center}
