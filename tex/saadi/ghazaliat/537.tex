\begin{center}
\section*{غزل ۵۳۷: چه باز در دلت آمد که مهر برکندی}
\label{sec:537}
\addcontentsline{toc}{section}{\nameref{sec:537}}
\begin{longtable}{l p{0.5cm} r}
چه باز در دلت آمد که مهر برکندی
&&
چه شد که یار قدیم از نظر بیفکندی
\\
ز حد گذشت جدایی میان ما ای دوست
&&
هنوز وقت نیامد که بازپیوندی
\\
بود که پیش تو میرم اگر مجال بود
&&
و گر نه بر سر کویت به آرزومندی
\\
دری به روی من ای یار مهربان بگشای
&&
که هیچ کس نگشاید اگر تو در بندی
\\
مرا و گر همه آفاق خوبرویانند
&&
به هیچ روی نمی‌باشد از تو خرسندی
\\
هزار بار بگفتم که چشم نگشایم
&&
به روی خوب ولیکن تو چشم می‌بندی
\\
مگر در آینه بینی و گر نه در آفاق
&&
به هیچ خلق نپندارمت که مانندی
\\
حدیث سعدی اگر کائنات بپسندند
&&
به هیچ کار نیاید گرش تو نپسندی
\\
مرا چه بندگی از دست و پای برخیزد
&&
مگر امید به بخشایش خداوندی
\\
\end{longtable}
\end{center}
