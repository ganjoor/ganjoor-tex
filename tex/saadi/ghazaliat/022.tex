\begin{center}
\section*{غزل ۲۲: من بدین خوبی و زیبایی ندیدم روی را}
\label{sec:022}
\addcontentsline{toc}{section}{\nameref{sec:022}}
\begin{longtable}{l p{0.5cm} r}
من بدین خوبی و زیبایی ندیدم روی را
&&
وین دلاویزی و دلبندی نباشد موی را
\\
روی اگر پنهان کند سنگین دل سیمین بدن
&&
مشک غمازست نتواند نهفتن بوی را
\\
ای موافق صورت و معنی که تا چشم من است
&&
از تو زیباتر ندیدم روی و خوشتر خوی را
\\
گر به سر می‌گردم از بیچارگی عیبم مکن
&&
چون تو چوگان می‌زنی جرمی نباشد گوی را
\\
هر که را وقتی دمی بودست و دردی سوخته‌ست
&&
دوست دارد ناله مستان و هایاهوی را
\\
ما ملامت را به جان جوییم در بازار عشق
&&
کنج خلوت پارسایان سلامت جوی را
\\
بوستان را هیچ دیگر در نمی‌باید به حسن
&&
بلکه سروی چون تو می‌باید کنار جوی را
\\
ای گل خوش بوی اگر صد قرن بازآید بهار
&&
مثل من دیگر نبینی بلبل خوشگوی را
\\
سعدیا گر بوسه بر دستش نمی‌یاری نهاد
&&
چاره آن دانم که در پایش بمالی روی را
\\
\end{longtable}
\end{center}
