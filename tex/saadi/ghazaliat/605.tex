\begin{center}
\section*{غزل ۶۰۵: سروقدی میان انجمنی}
\label{sec:605}
\addcontentsline{toc}{section}{\nameref{sec:605}}
\begin{longtable}{l p{0.5cm} r}
سروقدی میان انجمنی
&&
به که هفتاد سرو در چمنی
\\
جهل باشد فراق صحبت دوست
&&
به تماشای لاله و سمنی
\\
ای که هرگز ندیده‌ای به جمال
&&
جز در آیینه مثل خویشتنی
\\
تو که همتای خویشتن بینی
&&
لاجرم ننگری به مثل منی
\\
در دهانت سخن نمی‌گویم
&&
که نگنجد در آن دهن سخنی
\\
بدنت در میان پیرهنت
&&
همچو روحیست رفته در بدنی
\\
وآن که بیند برهنه اندامت
&&
گوید این پرگل است پیرهنی
\\
با وجودت خطا بود که نظر
&&
به ختایی کنند یا ختنی
\\
باد اگر بر من اوفتد ببرد
&&
که نمانده‌ست زیر جامه تنی
\\
چاره بیچارگی بود سعدی
&&
چون ندانند چاره‌ای و فنی
\\
\end{longtable}
\end{center}
