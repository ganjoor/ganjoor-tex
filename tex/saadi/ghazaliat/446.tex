\begin{center}
\section*{غزل ۴۴۶: ای کودک خوبروی حیران}
\label{sec:446}
\addcontentsline{toc}{section}{\nameref{sec:446}}
\begin{longtable}{l p{0.5cm} r}
ای کودک خوبروی حیران
&&
در وصف شمایلت سخندان
\\
صبر از همه چیز و هر که عالم
&&
کردیم و صبوری از تو نتوان
\\
دیدی که وفا به سر نبردی
&&
ای سخت کمان سست پیمان
\\
پایان فراق ناپدیدار
&&
و امید نمی‌رسد به پایان
\\
هرگز نشنیده‌ام که کرده‌ست
&&
سرو آنچه تو می‌کنی به جولان
\\
باور که کند که آدمی را
&&
خورشید برآید از گریبان
\\
بیمار فراق به نباشد
&&
تا بو نکند به زنخدان
\\
وین گوی سعادت است و دولت
&&
تا با که در افکنی به میدان
\\
ترسم که به عاقبت بماند
&&
در چشم سکندر آب حیوان
\\
دل بود و به دست دلبر افتاد
&&
جان است و فدای روی جانان
\\
عاقل نکند شکایت از درد
&&
مادام که هست امید درمان
\\
بی مار به سر نمی‌رود گنج
&&
بی خار نمی‌دمد گلستان
\\
گر در نظرت بسوخت سعدی
&&
مه را چه غم از هلاک کتان
\\
پروانه بکشت خویشتن را
&&
بر شمع چه لازم است تاوان
\\
\end{longtable}
\end{center}
