\begin{center}
\section*{غزل ۵۱: آن نه زلفست و بناگوش که روزست و شب‌ست}
\label{sec:051}
\addcontentsline{toc}{section}{\nameref{sec:051}}
\begin{longtable}{l p{0.5cm} r}
آن نه زلف است و بناگوش که روز است و شب است
&&
وان نه بالای صنوبر که درخت رطب است
\\
نه دهانیست که در وهم سخندان آید
&&
مگر اندر سخن آیی و بداند که لب است
\\
آتش روی تو زین گونه که در خلق گرفت
&&
عجب از سوختگی نیست که خامی عجب است
\\
آدمی نیست که عاشق نشود وقت بهار
&&
هر گیاهی که به نوروز نجنبد حطب است
\\
جنبش سرو تو پنداری کز باد صباست
&&
نه که از نالهٔ مرغان چمن در طرب است
\\
هر کسی را به تو این میل نباشد که مرا
&&
کآفتابی تو و کوتاه نظر مرغ شب است
\\
خواهم اندر طلبت عمر به پایان آورد
&&
گر چه راهم نه به اندازهٔ پای طلب است
\\
هر قضایی سببی دارد و من در غم دوست
&&
اجلم می‌کشد و درد فراقش سبب است
\\
سخن خویش به بیگانه نمی‌یارم گفت
&&
گله از دوست به دشمن نه طریق ادب است
\\
لیکن این حال محال است که پنهان ماند
&&
تو زره می‌دری و پرده سعدی قصب است
\\
\end{longtable}
\end{center}
