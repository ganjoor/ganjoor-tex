\begin{center}
\section*{غزل ۴۹۷: خلاف سرو را روزی خرامان سوی بستان آی}
\label{sec:497}
\addcontentsline{toc}{section}{\nameref{sec:497}}
\begin{longtable}{l p{0.5cm} r}
خلاف سرو را روزی خرامان سوی بستان آی
&&
دهان چون غنچه بگشای و چو گلبن در گلستان آی
\\
دمادم حوریان از خلد رضوان می‌فرستندت
&&
که ای حوری انسانی دمی در باغ رضوان آی
\\
گرت اندیشه می‌باشد ز بدگویان بی معنی
&&
چو معنی معجری بربند و چون اندیشه پنهان آی
\\
دلم گرد لب لعلت سکندروار می‌گردد
&&
نگویی کآخر ای مسکین فراز آب حیوان آی
\\
چو عقرب دشمنان داری و من با تو چو میزانم
&&
برای مصلحت ماها ز عقرب سوی میزان آی
\\
جهانی عشقبازانند در عهد سر زلفت
&&
رها کن راه بدعهدی و اندر عهد ایشان آی
\\
خوش آمد نیست سعدی را در این زندان جسمانی
&&
اگر تو یک دلی با او چو او در عالم جان آی
\\
\end{longtable}
\end{center}
