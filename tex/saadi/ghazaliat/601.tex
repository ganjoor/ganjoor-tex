\begin{center}
\section*{غزل ۶۰۱: ای دریغا گر شبی در بر خرابت دیدمی}
\label{sec:601}
\addcontentsline{toc}{section}{\nameref{sec:601}}
\begin{longtable}{l p{0.5cm} r}
ای دریغا گر شبی در بر خرابت دیدمی
&&
سرگران از خواب و سرمست از شرابت دیدمی
\\
روز روشن دست دادی در شب تاریک هجر
&&
گر سحرگه روی همچون آفتابت دیدمی
\\
گر مرا عشقت به سختی کشت سهل است این قدر
&&
کاش کاندک مایه نرمی در خطابت دیدمی
\\
در چکانیدی قلم بر نامه دلسوز من
&&
گر امید صلح باری در جوابت دیدمی
\\
راستی خواهی سر از من تافتن بودی صواب
&&
گر چو کژبینان به چشم ناصوابت دیدمی
\\
آه اگر وقتی چو گل در بوستان یا چون سمن
&&
در گلستان یا چو نیلوفر در آبت دیدمی
\\
ور چو خورشیدت نبینم کاشکی همچون هلال
&&
اندکی پیدا و دیگر در نقابت دیدمی
\\
از منت دانم حجابی نیست جز بیم رقیب
&&
کاش پنهان از رقیبان در حجابت دیدمی
\\
سر نیارستی کشید از دست افغانم فلک
&&
گر به خدمت دست سعدی در رکابت دیدمی
\\
این تمنایم به بیداری میسر کی شود
&&
کاشکی خوابم گرفتی تا به خوابت دیدمی
\\
\end{longtable}
\end{center}
