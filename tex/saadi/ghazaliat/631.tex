\begin{center}
\section*{غزل ۶۳۱: وقت آن آمد که خوش باشد کنار سبزه جوی}
\label{sec:631}
\addcontentsline{toc}{section}{\nameref{sec:631}}
\begin{longtable}{l p{0.5cm} r}
وقت آن آمد که خوش باشد کنار سبزه جوی
&&
گر سر صحرات باشد سروبالایی بجوی
\\
ور به خلوت با دلارامت میسر می‌شود
&&
در سرایت خود گل افشان است سبزی گو مروی
\\
ای نسیم کوی معشوق این چه باد خرم است
&&
تا کجا بودی که جانم تازه می‌گردد به بوی
\\
مطربان گویی در آوازند و مستان در سماع
&&
شاهدان در حالت و شوریدگان در های و هوی
\\
ای رفیق آنچ از بلای عشق بر من می‌رود
&&
گر به ترک من نمی‌گویی به ترک من بگوی
\\
ای که پای رفتنت کند است و راه وصل تند
&&
بازگشتن هم نشاید تا قدم داری بپوی
\\
گر ببینی گریه زارم ندانی فرق کرد
&&
کآب چشم است این که پیشت می‌رود یا آب جوی
\\
گوی را گفتند کای بیچاره سرگردان مباش
&&
گوی مسکین را چه تاوان است چوگان را بگوی
\\
ای که گفتی دل بشوی از مهر یار مهربان
&&
من دل از مهرش نمی‌شویم تو دست از من بشوی
\\
سعدیا عاشق نشاید بودن اندر خانقاه
&&
شاهد بازی فراخ و زاهدان تنگ خوی
\\
\end{longtable}
\end{center}
