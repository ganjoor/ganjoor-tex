\begin{center}
\section*{غزل ۴۱۲: آن دوست که من دارم وان یار که من دانم}
\label{sec:412}
\addcontentsline{toc}{section}{\nameref{sec:412}}
\begin{longtable}{l p{0.5cm} r}
آن دوست که من دارم وان یار که من دانم
&&
شیرین دهنی دارد دور از لب و دندانم
\\
بخت این نکند با من کان شاخ صنوبر را
&&
بنشینم و بنشانم گل بر سرش افشانم
\\
ای روی دلارایت مجموعه زیبایی
&&
مجموع چه غم دارد از من که پریشانم
\\
دریاب که نقشی ماند از طرح وجود من
&&
چون یاد تو می‌آرم خود هیچ نمی‌مانم
\\
با وصل نمی‌پیچم وز هجر نمی‌نالم
&&
حکم آن چه تو فرمایی من بنده فرمانم
\\
ای خوبتر از لیلی بیم است که چون مجنون
&&
عشق تو بگرداند در کوه و بیابانم
\\
یک پشت زمین دشمن گر روی به من آرند
&&
از روی تو بیزارم گر روی بگردانم
\\
در دام تو محبوسم در دست تو مغلوبم
&&
وز ذوق تو مدهوشم در وصف تو حیرانم
\\
دستی ز غمت بر دل پایی ز پیت در گل
&&
با این همه صبرم هست وز روی تو نتوانم
\\
در خفیه همی‌نالم وین طرفه که در عالم
&&
عشاق نمی‌خسبند از ناله پنهانم
\\
بینی که چه گرم آتش در سوخته می‌گیرد
&&
تو گرمتری ز آتش من سوخته تر ز آنم
\\
گویند مکن سعدی جان در سر این سودا
&&
گر جان برود شاید من زنده به جانانم
\\
\end{longtable}
\end{center}
