\begin{center}
\section*{غزل ۵۶۶: نه تو گفتی که به جای آرم و گفتم که نیاری}
\label{sec:566}
\addcontentsline{toc}{section}{\nameref{sec:566}}
\begin{longtable}{l p{0.5cm} r}
نه تو گفتی که به جای آرم و گفتم که نیاری
&&
عهد و پیمان و وفاداری و دلبندی و یاری
\\
زخم شمشیر اجل به که سر نیش فراقت
&&
کشتن اولیتر از آن که‌م به جراحت بگذاری
\\
تن آسوده چه داند که دل خسته چه باشد
&&
من گرفتار کمندم تو چه دانی که سواری
\\
کس چنین روی ندارد تو مگر حور بهشتی
&&
وز کس این بوی نیاید مگر آهوی تتاری
\\
عرقت بر ورق روی نگارین به چه ماند
&&
همچو بر خرمن گل قطره باران بهاری
\\
طوطیان دیدم و خوش‌تر ز حدیثت نشنیدم
&&
شکر است آن نه دهان و لب و دندان که تو داری
\\
ای خردمند که گفتی نکنم چشم به خوبان
&&
به چه کار آیدت آن دل که به جانان نسپاری
\\
آرزو می‌کندم با تو شبی بودن و روزی
&&
یا شبی روز کنی چون من و روزی به شب آری
\\
هم اگر عمر بود دامن کامی به کف آید
&&
که گل از خار همی‌آید و صبح از شب تاری
\\
سعدی آن طبع ندارد که ز خوی تو برنجد
&&
خوش بود هر چه تو گویی و شکر هر چه تو باری
\\
\end{longtable}
\end{center}
