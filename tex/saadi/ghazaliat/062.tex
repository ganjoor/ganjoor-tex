\begin{center}
\section*{غزل ۶۲: ای لعبت خندان لب لعلت که مزیدست}
\label{sec:062}
\addcontentsline{toc}{section}{\nameref{sec:062}}
\begin{longtable}{l p{0.5cm} r}
ای لعبت خندان لب لعلت که مزیده‌ست؟
&&
وی باغ لطافت به رویت که گزیده‌ست؟
\\
زیباتر از این صید همه عمر نکرده‌ست
&&
شیرین‌تر از این خربزه هرگز نبریده‌ست
\\
ای خضر حلالت نکنم چشمهٔ حیوان
&&
دانی که سکندر به چه محنت طلبیده‌ست
\\
آن خون کسی ریخته‌ای یا می سرخ است
&&
یا توت سیاه است که بر جامه چکیده‌ست
\\
با جمله برآمیزی و از ما بگریزی
&&
جرم از تو نباشد گنه از بخت رمیده‌ست
\\
نیک است که دیوار به یک بار بیفتاد
&&
تا هیچکس این باغ نگویی که ندیده‌ست
\\
بسیار توقف نکند میوهٔ بر بار
&&
چون عام بدانست که شیرین و رسیده‌ست
\\
گل نیز در آن هفته دهن باز نمی‌کرد
&&
وامروز نسیم سحرش پرده دریده‌ست
\\
در دجله که مرغابی از اندیشه نرفتی
&&
کشتی رود اکنون که تتر جسر بریده‌ست
\\
رفت آن که فقاع از تو گشایند دگربار
&&
ما را بس از این کوزه که بیگانه مکیده‌ست
\\
سعدی در بستان هوای دگری زن
&&
وین کشته رها کن که در او گله چریده‌ست
\\
\end{longtable}
\end{center}
