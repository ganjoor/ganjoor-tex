\begin{center}
\section*{غزل ۴۲۵: منم یا رب در این دولت که روی یار می‌بینم}
\label{sec:425}
\addcontentsline{toc}{section}{\nameref{sec:425}}
\begin{longtable}{l p{0.5cm} r}
منم یا رب در این دولت که روی یار می‌بینم
&&
فراز سرو سیمینش گلی بر بار می‌بینم
\\
مگر طوبی برآمد در سرابستان جان من
&&
که بر هر شعبه‌ای مرغی شکرگفتار می‌بینم
\\
مگر دنیا سر آمد کاین چنین آزاد در جنت
&&
می بی درد می‌نوشم گل بی خار می‌بینم
\\
عجب دارم ز بخت خویش و هر دم در گمان افتم
&&
که مستم یا به خوابم یا جمال یار می‌بینم
\\
زمین بوسیده‌ام بسیار و خدمت کرده تا اکنون
&&
لب معشوق می‌بوسم رخ دلدار می‌بینم
\\
چه طاعت کرده‌ام گویی که این پاداش می‌یابم
&&
چه فرمان برده‌ام گویی که این مقدار می‌بینم
\\
تویی یارا که خواب آلود بر من تاختن کردی
&&
منم یا رب که بخت خود چنین بیدار می‌بینم
\\
چو خلوت با میان آمد نخواهم شمع کاشانه
&&
تمنای بهشتم نیست چون دیدار می‌بینم
\\
کدام آلاله می‌بویم که مغزم عنبرآگین شد
&&
چه ریحان دسته بندم چون جهان گلزار می‌بینم
\\
ز گردون نعره می‌آید که اینت بوالعجب کاری
&&
که سعدی را ز روی دوست برخوردار می‌بینم
\\
\end{longtable}
\end{center}
