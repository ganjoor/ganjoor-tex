\begin{center}
\section*{غزل ۳۶۰: شمع بخواهد نشست بازنشین ای غلام}
\label{sec:360}
\addcontentsline{toc}{section}{\nameref{sec:360}}
\begin{longtable}{l p{0.5cm} r}
شمع بخواهد نشست بازنشین ای غلام
&&
روی تو دیدن به صبح روز نماید تمام
\\
مطرب یاران برفت ساقی مستان بخفت
&&
شاهد ما برقرار مجلس ما بردوام
\\
بلبل باغ سرای صبح نشان می‌دهد
&&
وز در ایوان بخاست بانگ خروسان بام
\\
ما به تو پرداختیم خانه و هرچ اندر اوست
&&
هر چه پسند شماست بر همه عالم حرام
\\
خواهیم آزاد کن خواه قویتر ببند
&&
مثل تو صیاد را کس نگریزد ز دام
\\
هر که در آتش نرفت بی‌خبر از سوز ماست
&&
سوخته داند که چیست پختن سودای خام
\\
اولم اندیشه بود تا نشود نام زشت
&&
فارغم اکنون ز سنگ چون بشکستند جام
\\
سعدی اگر نام و ننگ در سر او شد چه شد
&&
مرد ره عشق نیست که‌ش غم ننگ است و نام
\\
\end{longtable}
\end{center}
