\begin{center}
\section*{غزل ۲۹۰: تو را سریست که با ما فرو نمی‌آید}
\label{sec:290}
\addcontentsline{toc}{section}{\nameref{sec:290}}
\begin{longtable}{l p{0.5cm} r}
تو را سریست که با ما فرو نمی‌آید
&&
مرا دلی که صبوری از او نمی‌آید
\\
کدام دیده به روی تو باز شد همه عمر
&&
که آب دیده به رویش فرو نمی‌آید
\\
جز این قدر نتوان گفت بر جمال تو عیب
&&
که مهربانی از آن طبع و خو نمی‌آید
\\
چه جور کز خم چوگان زلف مشکینت
&&
بر اوفتاده مسکین چو گو نمی‌آید
\\
اگر هزار گزند آید از تو بر دل ریش
&&
بد از منست که گویم نکو نمی‌آید
\\
گر از حدیث تو کوته کنم زبان امید
&&
که هیچ حاصل از این گفت و گو نمی‌آید
\\
گمان برند که در عودسوز سینه من
&&
بمرد آتش معنی که بو نمی‌آید
\\
چه عاشقست که فریاد دردناکش نیست
&&
چه مجلسست کز او های و هو نمی‌آید
\\
به شیر بود مگر شور عشق سعدی را
&&
که پیر گشت و تغیر در او نمی‌آید
\\
\end{longtable}
\end{center}
