\begin{center}
\section*{غزل ۳۳۰: زینهار از دهان خندانش}
\label{sec:330}
\addcontentsline{toc}{section}{\nameref{sec:330}}
\begin{longtable}{l p{0.5cm} r}
زینهار از دهان خندانش
&&
و آتش لعل و آب دندانش
\\
مگر آن دایه کاین صنم پرورد
&&
شهد بوده‌ست شیر پستانش
\\
باغبان گر ببیند این رفتار
&&
سرو بیرون کند ز بستانش
\\
ور چنین حور در بهشت آید
&&
همه خادم شوند غلمانش
\\
چاهی اندر ره مسلمانان
&&
نیست الا چه زنخدانش
\\
چند خواهی چو من بر این لب چاه
&&
متعطش بر آب حیوانش
\\
شاید این روی اگر سبیل کند
&&
بر تماشاکنان حیرانش
\\
ساربانا جمال کعبه کجاست
&&
که بمردیم در بیابانش
\\
بس که در خاک می‌طپند چو گوی
&&
از خم زلف همچو چوگانش
\\
لاجرم عقل منهزم شد و صبر
&&
که نبودند مرد میدانش
\\
ما دگر بی تو صبر نتوانیم
&&
که همین بود حد امکانش
\\
از ملامت چه غم خورد سعدی
&&
مرده از نیشتر مترسانش
\\
\end{longtable}
\end{center}
