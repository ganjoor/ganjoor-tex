\begin{center}
\section*{غزل ۲۳۶: توانگران که به جنب سرای درویشند}
\label{sec:236}
\addcontentsline{toc}{section}{\nameref{sec:236}}
\begin{longtable}{l p{0.5cm} r}
توانگران که به جنب سرای درویشند
&&
مروت است که هر وقت از او بیندیشند
\\
تو ای توانگر حسن از غنای درویشان
&&
خبر نداری اگر خسته‌اند و گر ریشند
\\
تو را چه غم که یکی در غمت به جان آید
&&
که دوستان تو چندان که می‌کشی بیشند
\\
مرا به علت بیگانگی ز خویش مران
&&
که دوستان وفادار بهتر از خویشند
\\
غلام همت رندان و پاکبازانم
&&
که از محبت با دوست دشمن خویشند
\\
هرآینه لب شیرین جواب تلخ دهد
&&
چنان که صاحب نوشند ضارب نیشند
\\
تو عاشقان مسلم ندیده‌ای سعدی
&&
که تیغ بر سر و سر بنده وار در پیشند
\\
نه چون منند و تو مسکین حریص کوته دست
&&
که ترک هر دو جهان گفته‌اند و درویشند
\\
\end{longtable}
\end{center}
