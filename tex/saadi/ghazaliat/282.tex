\begin{center}
\section*{غزل ۲۸۲: مرا چو آرزوی روی آن نگار آید}
\label{sec:282}
\addcontentsline{toc}{section}{\nameref{sec:282}}
\begin{longtable}{l p{0.5cm} r}
مرا چو آرزوی روی آن نگار آید
&&
چو بلبلم هوس ناله‌های زار آید
\\
میان انجمن از لعل او چو آرم یاد
&&
مرا سرشک چو یاقوت در کنار آید
\\
ز رنگ لاله مرا روی دلبر آید یاد
&&
ز شکل سبزه مرا یاد خط یار آید
\\
گلی به دست من آید چو روی تو هیهات
&&
هزار سال دگر گر چنین بهار آید
\\
خسان خورند بر از باغ وصل او و مرا
&&
ز گلستان جمالش نصیب خار آید
\\
طمع مدار وصالی که بی فراق بود
&&
هرآینه پس هر مستیی خمار آید
\\
مرا زمانه ز یاران به منزلی انداخت
&&
که راضیم به نسیمی کز آن دیار آید
\\
فراق یار به یک بار بیخ صبر بکند
&&
بهار وصل ندانم که کی به بار آید
\\
دلا اگر چه که تلخست بیخ صبر ولی
&&
چو بر امید وصالست خوشگوار آید
\\
پس از تحمل سختی امید وصل مراست
&&
که صبح از شب و تریاک هم ز مار آید
\\
ز چرخ عربده جو بس خدنگ تیر جفا
&&
بجست و در دل مردان هوشیار آید
\\
چو عمر خوش نفسی گر گذر کنی بر من
&&
مرا همان نفس از عمر در شمار آید
\\
بجز غلامی دلدار خویش سعدی را
&&
ز کار و بار جهان گر شهیست عار آید
\\
\end{longtable}
\end{center}
