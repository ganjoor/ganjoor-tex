\begin{center}
\section*{غزل ۲۱: تفاوتی نکند قدر پادشایی را}
\label{sec:021}
\addcontentsline{toc}{section}{\nameref{sec:021}}
\begin{longtable}{l p{0.5cm} r}
تفاوتی نکند قدر پادشایی را
&&
که التفات کند کمترین گدایی را
\\
به جان دوست که دشمن بدین رضا ندهد
&&
که در به روی ببندند آشنایی را
\\
مگر حلال نباشد که بندگان ملوک
&&
ز خیل خانه برانند بی‌نوایی را
\\
و گر تو جور کنی رای ما دگر نشود
&&
هزار شکر بگوییم هر جفایی را
\\
همه سلامت نفس آرزو کند مردم
&&
خلاف من که به جان می‌خرم بلایی را
\\
حدیث عشق نداند کسی که در همه عمر
&&
به سر نکوفته باشد در سرایی را
\\
خیال در همه عالم برفت و بازآمد
&&
که از حضور تو خوشتر ندید جایی را
\\
سری به صحبت بیچارگان فرود آور
&&
همین قدر که ببوسند خاک پایی را
\\
قبای خوشتر از این در بدن تواند بود
&&
بدن نیفتد از این خوبتر قبایی را
\\
اگر تو روی نپوشی بدین لطافت و حسن
&&
دگر نبینی در پارس پارسایی را
\\
منه به جان تو بار فراق بر دل ریش
&&
که پشه‌ای نبرد سنگ آسیایی را
\\
دگر به دست نیاید چو من وفاداری
&&
که ترک می‌ندهم عهد بی‌وفایی را
\\
دعای سعدی اگر بشنوی زیان نکنی
&&
که یحتمل که اجابت بود دعایی را
\\
\end{longtable}
\end{center}
