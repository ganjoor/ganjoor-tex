\begin{center}
\section*{غزل ۹۹: صبح می‌خندد و من گریه کنان از غم دوست}
\label{sec:099}
\addcontentsline{toc}{section}{\nameref{sec:099}}
\begin{longtable}{l p{0.5cm} r}
صبح می‌خندد و من گریه کنان از غم دوست
&&
ای دم صبح چه داری خبر از مقدم دوست
\\
بر خودم گریه همی‌آید و بر خنده تو
&&
تا تبسم چه کنی بی‌خبر از مبسم دوست
\\
ای نسیم سحر از من به دلارام بگوی
&&
که کسی جز تو ندانم که بود محرم دوست
\\
گو کم یار برای دل اغیار مگیر
&&
دشمن این نیک پسندد که تو گیری کم دوست
\\
تو که با جانب خصمت به ارادت نظرست
&&
به که ضایع نگذاری طرف معظم دوست
\\
من نه آنم که عدو گفت تو خود دانی نیک
&&
که ندارد دل دشمن خبر از عالم دوست
\\
نی نی ای باد مرو حال من خسته مگوی
&&
تا غباری ننشیند به دل خرم دوست
\\
هر کسی را غم خویشست و دل سعدی را
&&
همه وقتی غم آن تا چه کند با غم دوست
\\
\end{longtable}
\end{center}
