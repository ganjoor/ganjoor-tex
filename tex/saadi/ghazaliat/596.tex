\begin{center}
\section*{غزل ۵۹۶: مرا تو جان عزیزی و یار محترمی}
\label{sec:596}
\addcontentsline{toc}{section}{\nameref{sec:596}}
\begin{longtable}{l p{0.5cm} r}
مرا تو جان عزیزی و یار محترمی
&&
به هر چه حکم کنی بر وجود من حکمی
\\
غمت مباد و گزندت مباد و درد مباد
&&
که مونس دل و آرام جان و دفع غمی
\\
هزار تندی و سختی بکن که سهل بود
&&
جفای مثل تو بردن که سابق کرمی
\\
ندانم از سر و پایت کدام خوبتر است
&&
چه جای فرق که زیبا ز فرق تا قدمی
\\
اگر هزار الم دارم از تو بر دل ریش
&&
هنوز مرهم ریشی و داروی المی
\\
چنین که می‌گذری کافر و مسلمان را
&&
نگه به توست که هم قبله‌ای و هم صنمی
\\
چنین جمال نشاید که هر نظر بیند
&&
مگر که نام خدا گرد خویشتن بدمی
\\
نگویمت که گلی بر فراز سرو روان
&&
که آفتاب جهانتاب بر سر علمی
\\
تو مشکبوی سیه چشم را که دریابد
&&
که همچو آهوی مشکین از آدمی برمی
\\
کمند سعدی اگر شیر شرزه صید کند
&&
تو در کمند نیایی که آهوی حرمی
\\
\end{longtable}
\end{center}
