\begin{center}
\section*{بخش ۱۹ - تمثیل در بیان مقام نبوت و ولایت}
\label{sec:sh019}
\addcontentsline{toc}{section}{\nameref{sec:sh019}}
\begin{longtable}{l p{0.5cm} r}
نبی چون آفتاب آمد ولی ماه
&&
مقابل گردد اندر «لی مع‌الله»
\\
نبوت در کمال خویش صافی است
&&
ولایت اندر او پیدا نه مخفی است
\\
ولایت در ولی پوشیده باید
&&
ولی اندر نبی پیدا نماید
\\
ولی از پیروی چون همدم آمد
&&
نبی را در ولایت محرم آمد
\\
ز «ان کنتم تحبون» یابد او راه
&&
به خلوتخانهٔ «یحببکم الله»
\\
در آن خلوت‌سرا محبوب گردد
&&
به حق یکبارگی مجذوب گردد
\\
بود تابع ولی از روی معنی
&&
بود عابد ولی در کوی معنی
\\
ولی آنگه رسد کارش به اتمام
&&
که با آغاز گردد باز از انجام
\\
\end{longtable}
\end{center}
