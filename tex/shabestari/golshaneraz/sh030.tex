\begin{center}
\section*{بخش ۳۰ - جواب}
\label{sec:sh030}
\addcontentsline{toc}{section}{\nameref{sec:sh030}}
\begin{longtable}{l p{0.5cm} r}
انا الحق کشف اسرار است مطلق
&&
جز از حق کیست تا گوید انا الحق
\\
همه ذرات عالم همچو منصور
&&
تو خواهی مست گیر و خواه مخمور
\\
در این تسبیح و تهلیلند دائم
&&
بدین معنی همی‌باشند قائم
\\
اگر خواهی که گردد بر تو آسان
&&
«و ان من شیء» را یک ره فرو خوان
\\
چو کردی خویشتن را پنبه‌کاری
&&
تو هم حلاج‌وار این دم برآری
\\
برآور پنبهٔ پندارت از گوش
&&
ندای «واحد القهار» بنیوش
\\
ندا می‌آید از حق بر دوامت
&&
چرا گشتی تو موقوف قیامت
\\
درآ در وادی ایمن که ناگاه
&&
درختی گویدت «انی انا الله»
\\
روا باشد انا الحق از درختی
&&
چرا نبود روا از نیک‌بختی
\\
هر آن کس را که اندر دل شکی نیست
&&
یقین داند که هستی جز یکی نیست
\\
انانیت بود حق را سزاوار
&&
که هو غیب است و غایب وهم و پندار
\\
جناب حضرت حق را دویی نیست
&&
در آن حضرت من و ما و تویی نیست
\\
\end{longtable}
\end{center}
