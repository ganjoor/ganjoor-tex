\begin{center}
\section*{بخش ۳۲ - تمثیل در نمودهای بی‌بود}
\label{sec:sh032}
\addcontentsline{toc}{section}{\nameref{sec:sh032}}
\begin{longtable}{l p{0.5cm} r}
بنه آیینه‌ای اندر برابر
&&
در او بنگر ببین آن شخص دیگر
\\
یکی ره باز بین تا چیست آن عکس
&&
نه این است و نه آن پس کیست آن عکس
\\
چو من هستم به ذات خود معین
&&
ندانم تا چه باشد سایهٔ من
\\
عدم با هستی آخر چون شود ضم
&&
نباشد نور و ظلمت هر دو با هم
\\
چو ماضی نیست مستقبل مه و سال
&&
چه باشد غیر از آن یک نقطهٔ حال
\\
یکی نقطه است وهمی گشته ساری
&&
تو آن را نام کرده نهر جاری
\\
جز از من اندر این صحرا دگر کیست
&&
بگو با من که تا صوت و صدا چیست
\\
عرض فانی است جوهر زو مرکب
&&
بگو کی بود یا خود کو مرکب
\\
ز طول و عرض و از عمق است اجسام
&&
وجودی چون پدید آمد ز اعدام
\\
از این جنس است اصل جمله عالم
&&
چو دانستی بیار ایمان و فالزم
\\
جز از حق نیست دیگر هستی الحق
&&
هوالحق گو و گر خواهی انا الحق
\\
نمود وهمی از هستی جدا کن
&&
نه ای بیگانه خود را آشنا کن
\\
\end{longtable}
\end{center}
