\begin{center}
\section*{بخش ۳۷ - جواب}
\label{sec:sh037}
\addcontentsline{toc}{section}{\nameref{sec:sh037}}
\begin{longtable}{l p{0.5cm} r}
ز من بشنو حدیث بی کم و بیش
&&
ز نزدیکی تو دور افتادی از خویش
\\
چو هستی را ظهوری در عدم شد
&&
از آنجا قرب و بعد و بیش و کم شد
\\
قریب آن هست کو را رش نور است
&&
بعید آن نیستی کز هست دور است
\\
اگر نوری ز خود در تو رساند
&&
تو را از هستی خود وا رهاند
\\
چه حاصل مر تو را زین بود نابود
&&
کز او گاهیت خوف و گه رجا بود
\\
نترسد زو کسی کو را شناسد
&&
که طفل از سایهٔ خود می‌هراسد
\\
نماند خوف اگر گردی روانه
&&
نخواهد اسب تازی تازیانه
\\
تو را از آتش دوزخ چه باک است
&&
گر از هستی تن وجان تو پاک است
\\
از آتش زر خالص برفروزد
&&
چو غشی نبود اندر وی چه سوزد
\\
تو را غیر تو چیزی نیست در پیش
&&
ولیکن از وجود خود بیندیش
\\
اگر در خویشتن گردی گرفتار
&&
حجاب تو شود عالم به یک بار
\\
تویی در دور هستی جزو سافل
&&
تویی با نقطهٔ وحدت مقابل
\\
تعین‌های عالم بر تو طاری است
&&
از آن گویی چوشیطان همچو من کیست
\\
از آن گویی مرا خود اختیار است
&&
تن من مرکب و جانم سوار است
\\
زمام تن به دست جان نهادند
&&
همه تکلیف بر من زان نهادند
\\
ندانی کین ره آتش‌پرستی است
&&
همه این آفت و شومی ز هستی است
\\
کدامین اختیار ای مرد عاقل
&&
کسی را کو بود بالذات باطل
\\
چو بود توست یک سر همچو نابود
&&
نگویی که اختیارت از کجا بود
\\
کسی کو را وجود از خود نباشد
&&
به ذات خویش نیک و بد نباشد
\\
که را دیدی تو اندر جمله عالم
&&
که یک دم شادمانی یافت بی غم
\\
که را شد حاصل آخر جمله امید
&&
که ماند اندر کمالی تا به جاوید
\\
مراتب باقی و اهل مراتب
&&
به زیر امر حق والله غالب
\\
مؤثر حق شناس اندر همه جای
&&
ز حد خویشتن بیرون منه پای
\\
ز حال خویشتن پرس این قدر چیست
&&
وز آنجا باز دان کاهل قدر کیست
\\
هر آن کس را که مذهب غیر جبر است
&&
نبی فرمود کو مانند گبر است
\\
چنان کان گبر یزدان و اهرمن گفت
&&
مر آن نادان احمق او و من گفت
\\
به ما افعال را نسبت مجازی است
&&
نسب خود در حقیقت لهو و بازی است
\\
نبودی تو که فعلت آفریدند
&&
تو را از بهر کاری برگزیدند
\\
به قدرت بی‌سبب دانای بر حق
&&
به علم خویش حکمی کرده مطلق
\\
مقدر گشته پیش از جان و از تن
&&
برای هر یکی کاری معین
\\
یکی هفتصد هزاران ساله طاعت
&&
به جای آورد و کردش طوق لعنت
\\
دگر از معصیت نور و صفا دید
&&
چو توبه کرد نور «اصطفی» دید
\\
عجب‌تر آنکه این از ترک مامور
&&
شد از الطاف حق مرحوم و مغفور
\\
مر آن دیگر ز منهی گشته ملعون
&&
زهی فعل تو بی چند و چه و چون
\\
جناب کبریایی لاابالی است
&&
منزه از قیاسات خیالی است
\\
چه بود اندر ازل ای مرد نااهل
&&
که این یک شد محمد و آن ابوجهل
\\
کسی کو با خدا چون و چرا گفت
&&
چو مشرک حضرتش را ناسزا گفت
\\
ورا زیبد که پرسد از چه و چون
&&
نباشد اعتراض از بنده موزون
\\
خداوندی همه در کبریایی است
&&
نه علت لایق فعل خدایی است
\\
سزاوار خدایی لطف و قهر است
&&
ولیکن بندگی در جبر و فقر است
\\
کرامت آدمی را اضطرار است
&&
نه زان کو را نصیبی ز اختیار است
\\
نبوده هیچ چیزش هرگز از خود
&&
پس آنگه پرسدش از نیک و از بد
\\
ندارد اختیار و گشته مامور
&&
زهی مسکین که شد مختار مجبور
\\
نه ظلم است این که عین علم و عدل است
&&
نه جور است این که محض لطف و فضل است
\\
به شرعت زان سبب تکلیف کردند
&&
که از ذات خودت تعریف کردند
\\
چو از تکلیف حق عاجز شوی تو
&&
به یک بار از میان بیرون روی تو
\\
به کلیت رهایی یابی از خویش
&&
غنی گردی به حق ای مرد درویش
\\
برو جان پدر تن در قضا ده
&&
به تقدیرات یزدانی رضا ده
\\
\end{longtable}
\end{center}
