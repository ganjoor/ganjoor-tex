\begin{center}
\section*{بخش ۳۱ - قاعده در بطلان حلول و اتحاد}
\label{sec:sh031}
\addcontentsline{toc}{section}{\nameref{sec:sh031}}
\begin{longtable}{l p{0.5cm} r}
من و ما و تو او هست یک چیز
&&
که در وحدت نباشد هیچ تمییز
\\
هر آن کو خالی از خود چون خلا شد
&&
انا الحق اندر او صوت و صدا شد
\\
شود با وجه باقی غیر هالک
&&
یکی گردد سلوک و سیر و سالک
\\
حلول و اتحاد از غیر خیزد
&&
ولی وحدت همه از سیر خیزد
\\
تعین بود کز هستی جدا شد
&&
نه حق شد بنده نه بنده خدا شد
\\
حلول و اتحاد اینجا محال است
&&
که در وحدت دویی عین ضلال است
\\
وجود خلق و کثرت درنمود است
&&
نه هرچ آن می‌نماید عین بود است
\\
\end{longtable}
\end{center}
